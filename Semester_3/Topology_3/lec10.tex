% Коллоквиум закончился
\section{Предел последовательности в топологическом пространстве}
\begin{defin}
    Последовательность $\{x)n\}\subset X$ сходится к $a\in X$
$$\forall U(a)~\exists n_0\in \mathbb{N}~\forall n>n_0:x_n\in U(a)$$
\end{defin}
Также обозначается $a=\lim\limits_{n\to\infty}x_n$. 
%Звонит Полотовский. НИ ставит его на громкую звязь. 
%-Григорий Михайлович, у меня лекция!
%-Телефон Николь можете потом дать?
%-Да
\begin{theor}
Если предел последовательности существует в хаусдорфовом пространстве, 
то он единственный. 
\end{theor}
\textbf{Доказательство.}  Пусть существует два предела. По хаусдорфовости, 
они обладают непересекающимися окрестностями. Тогда и там и там лежат все 
номера последовательности, что невозможно. 
$\square$ 
\textbf{Пример.} Исследовать на сходимость в 
$(\mathbb{R},\tau_\text{ирр})$ последовательность $x_n=\frac{2}{n}$. 
Покажем, что каждая рациональная точка является пределом последовательности. 
Рассмтрим $U_a,~a\in \mathbb{Q}$. Тогда $U_a=\mathbb{R}$. Очевидно,
$\forall n\in \mathbb{N}:x_n\in U_a$. Докажем, что других нет. Очевидно,
наименьшая окрестность иррациональной точки - сама точка, в которой нет 
никаких членов последовательности. Формально, запишем отрицание:
$$\lim\limits_{n \to \infty} x_n\ne b\iff \exists U_b~\forall N~\exists n>N:
x_n\notin U_b$$ 
\section{Аксиомы счетности}
\begin{defin}
База окрестностей - семейство окрестностей $\{U_\alpha=U_\alpha(x)\}$, 
такое, что для каждой окрестноси $U_x$ точк $x$ имеет место
$x\in U_\alpha\subset U_x$
\end{defin}

\begin{defin}
\textbf{Аксиома счетности I}. В каждой точке $x$ существует счетная база
окрестностей в точке  $x$.
\end{defin}
\begin{defin}
\textbf{Аксиома счетности II}. База топологии счетна.
\end{defin}
\begin{theor}
    (о связи между аксиомами счетности)\\
    Вторая аксиома счетности влечет первую (обратное неверно).
\end{theor}
\textbf{Доказательство.} Пусть $\Sigma=\{W_i\}$ - счетная база топологии.
Положим для точки $x$  $\Sigma_x=\{W_i\in\Sigma\mid x\in W_i\}\subset\Sigma$.
Это счетное множество. Теперь покажем, что это счетная 
база окрестностей в точке
$x$. По определению базы, для любой окрестности: 
$x\in U_x=\bigcup\limits_{i\in \mathbb{N}_0\subset \mathbb{N}}W_i$

\textbf{Пример.} В дискретной топологии вторая аксиома счетности
не выполняется, коль скоро пространство несчетно, так как минимальная база 
из одноточечных подмножеств несчетна, при этом удовлетворяет первой аксиоме.

\textbf{Пример.} $(\mathbb{R}^2,\tau_{MN})$ удовлетворяет первой аксиоме, но 
не удовлетворяет второй. 
\begin{defin}
Подмножество А всюду плотно в Х, если $\overline{A}=X$.
\end{defin}
\begin{theor}
    (лемма о всюду плотном множестве)\\
    А всюду плотно в Х тогда и только тогда, когда для любого открытого U:
    $U\cap A\ne\varnothing$.
\end{theor}
\textbf{Доказательство.} Пусть $A$ всюду плотно. Для любой непустой

для любого непустого открытого множества 
$\exists x\in U$ 

Обратно, пусть

$\square$ \\

\begin{defin}
Топологическое пространство называется сепарабельным, если в нем существует
счетное всюду плотное подмножество.
\end{defin}
Так, $\mathbb{R}^n$ с обычной топологией 
сепарабельно, так как $\mathbb{Q}^n$ - всюду плотно и 
его замыкание равно $\mathbb{R}^n$. Упражнение:
$\overline{\mathbb{Q}\times...\times\mathbb{Q}}=\overline{\mathbb{Q}}\times...
\times\overline{\mathbb{Q}}$



