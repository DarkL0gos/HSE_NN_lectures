дописать то что было!!!!!!!!!!!!!!!!
\begin{theor}
Об эквивалентности определений непрерывного отображения
\end{theor}
\textbf{Доказательство.}  1 $\Rightarrow$ 2. Выполняется теоретико-множестве
нное соотношение 
$X\setminus f^{-1}(B)=f^{-1}(Y\setminus B)~\forall B\subset Y$.
Пусть $f$ - непрерывно по определению 1. Рассмотрим замкнутое множество в 
$F$ в  $Y,\Omega$  $\Rightarrow$ $X\setminus f^{-1}(F)=f^{-1}(Y\setminus 
F)\in\tau$ - следовтаельно, выполняется определение 2.\\
2 $\Rightarrow$ 1. Аналогично предыдущему.\\
1 $\Rightarrow$ 3. Рассмотрим любую точку $x\in X,~y=f(x)$, и рассмотрим
любую окрестность  $V_y$. Пусть  $U$ - прообраз окретсночти $V_y$. Оно открыто 
по определению 1. И значит, это окрестность икса. Значит,
$f(u)\subset V_y$.\\
3 $\Rightarrow$ 4. Пусть $\Sigma$ - база в  $\tau$,  $\sigma$ - база в 
$\Omega$. По определению 3 непрерывности, существует окрестность икса,
чьим образом является базовая окрестность игрека. Но эта окрестность
представляется в виде объединения элементов базы. В одном из них лежит икс
следовательно, выполняется условие.\\
4 $\Rightarrow$ 1. Воспользуемся леммой об открытом множестве. Покажем, что
$U:=f^{-1}(V)$ открыто в  $(X,\tau)$.  $\forall x\in U:y=f(x)\in V$ по 
критерию базы в топоплогическом пространстве. $\exists V_y\in\sigma:y\in V_y
\subset V$ значит, по определению 4 существует открытое множество из базы, 
$W_x\in\Sigma:f(W_x)\subset V_y\subset V\implies W_x\subset f^{-1}(V)=U$.
Таким образом, $U$ содержит вместе со всякой точкой окрестность  $\Rightarrow$ 
оно открыто. $\square$\\
\textbf{Пример.} (рисунок с пространствами).
$f\colon(X,\tau)\to(Y,\Omega)$ - непрерывное отображение по определению 3.

\subsection{Непрерывные отображения метрических пространств}
\begin{defin}
Пусть $(X,d,\tau_d),~(Y,\rho,\tau_\rho)$ - метрические топологические 
пространства. Отображение  $f\colon X\to Y$ непрерывно в точке  $x\in X$,
если  $\forall D_\varepsilon(y)\exists D_{\delta}(x):~f(D_\delta(x))\subset 
D_\varepsilon(y)$.
\end{defin}
\begin{theor}
Это определение эквивалентно предыдущим определениям непрерывности. 
\end{theor}
\textbf{Доказательство.}  Применим определение для баз топологй. Шары образуют
базу метрической топологии.\\
5 $\Rightarrow$4. \\
4 $\Rightarrow$5. Рассмотрим шар в точке у. По определению 4, найдется шар
(необязательно с центром в х), образ которого лежит в окрестности игрека. 
Но тогда найдется и шар с центром в х, лежащий в этом шаре. Покажем,
что образ этого шара также лежит в окрестности у. Используем неравенство
треугольника: $d(z,a)\leqslant d(z,x)+d(x,a)<r-d(x,a)+d(x,a)=r$, значит,
$f(D_\delta(x))\subset f(D_r(a))\subset D_\varepsilon(y)$.
$\square$ 

Какова свзяь с непрерывностью в смысле матанализа? Рассмотрим
$f\colon \mathbb{R}\to\mathbb{R}$ с обычной топологией. 
Тогда обычноя непрерывность $|x-x_0|<\delta\implies|f(x)-y_0|<\varepsilon$.
Можно нарисовать рисунок.\\
Можно налить воды, позвененть ключами\\
Одиночество есть человек в квадрате\\
Вывод: в мат. анализе используется определение 5, так как $\mathbb{R}$ - 
метрическое топологическое пространство. 
\begin{theor}
Композиция непрерывных функций непрерывна
\end{theor}
\textbf{Доказательство.}  Докжем эту классическую теорему анализа чисто
топологически. Пусть $f\colon X\to Y,~g\colon Y\to Z$ - непрерывны.
Докажем, что  $g\circ f$ - тоже непреывно. Рассмотрим любое открытое множество в 
$Z$. Для него рассмотрим  $U=(g\circ f)^{-1}(w)=f^{-1}(g^{-1}(w))$. 
Так как прообраз открытого открыт,  $g^{-1}(w)$ - открыт в Y, и значит
обратная композициия тоже открыта. $\square$\\
\textbf{Пример.} Любая сложная функция непрерывна, например, $f(x)=2^{\cos
{57(x-3)}+\ln{8-\sin x}}+x^170$. По индукции предложение распространяется на 
любое конечное число функций. 

Дадим несколько определений. Пусть $(X,tau)$ - топологическое пространство,
 $A\subset X$, $f\colon (X,\tau)\to(Y,\Omega)$ - отображение 
 топологических пространств.
\begin{defin}
Сужение отображения - отображение $f_A\colon(A,\tau_A)\to (Y,\Omega)$ (где
$\tau_A$ - индуцированная топология)
\end{defin}
\begin{defin}
    Приведение отображения - отображение $f_1\colon(X,\tau)\to(f(X),
    \Omega_{f(x)})$ (в частности, совпадает в случае сюръекции.)
\end{defin}
\begin{defin}
Приведение отображения -  
\end{defin}
\begin{theor}
Сужение непрерывного отображения непрерывно. 
\end{theor}
\textbf{Доказательство.} Для любого открытого его прообраз открыт.
$\square$ 
\begin{theor}
Приведение непрерывного отображения непрерывно.     
\end{theor}
\textbf{Доказательство.} Рассмотрим открытое множество $W$ в индуцированной 
топологии на образе $X$. Это - след открытого множества $V$ в $Y$, чей 
прообраз открыт. Но $f^{-1}(W)=f^{-1}(V)$, значит, $f_{1}^{-1}(V)=f^{-1}(W)$ 
так как образ $X$. 
$\square$ 
\begin{theor}

\end{theor}
\textbf{Доказательство.}  \
$\square$ 



