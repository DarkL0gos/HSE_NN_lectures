\textbf{Лемма.} Две топологии с общей базой совпадают.\\
\textbf{Доказательство.} Пусть $\tau,\omega$ - две топологии на множестве 
$X$,  имеющие общую базу $\Sigma=\{W_\beta\subset X\mid\beta\in B\} $. 
Для всех множеств из топологии $\tau$  они являются объединением множеств из
базы, но поскольку это объединение открытых множеств, то оно открыто, и 
является элементом топлогии $\omega$. Итак,  $\tau\subset \omega$, аналогично
и в другую сторону. \\
\textbf{Замечание.} Согласно этой лемме, база топологии однозначно определяет
топологию. Следовательно, критерий базы на множестве дает способ 
определения новых топологий. 
\section{Метрическая топология}
Напомним определение метрического пространства. \\
Пусть функция $\rho\colon M\times M\to\mathbb{R}$ удовлетворяет трем условиям:
\begin{enumerate}
    \item $\rho(x,y)\geqslant 0$
    \item $\rho(x,y)=\rho(y,x)$
    \item  $\rho(x,y)+\rho(x,z)\leqslant \rho(y,x)$
\end{enumerate}
Тогда множество $(M,\rho)$ называется метрическим пространством с метрикой  
$\rho$.
\begin{defin}
Пусть  $(M,\rho)$ - метричсекое пространство. Множество
$$D_r(a):=\{x\in M\mid\rho(x,a)<r\}$$ 
называется открытым шаром радиуса $r$ 
\end{defin}
Очевидно, центр шара принадлежит ему в любой метрике. 
\begin{defin}
Пусть $(M,\rho)$ - метрическое пространство. Множество весвозможных шаров
с разными уентрами и радиусами являются базой $\Sigma_\rho$ (единственной)
топологии,
которая называется метрической топологией. 
\end{defin}
Докажем, что множество шаров - база. Применим критерий базы на множестве.\\
1. Возьмем объединение всех шаров. Так как любой шар содержит свой центр,
то все точки множества лежат в объединении шаров. \\
2. Для пересекающихся шаров возьмем минимальную радиус до границы шара. \\
\textbf{Пример.} Евклидова топология - пример метрической топологии для
стандартной евклидовой метрики в $\mathbb{R}^n$.  Дискретная топология - 
топология, порожденная дисретной метрикой.\\
\textbf{Упражнение.} Докажите самстоятельно, что евклидова метрика индуцирует
евклидову топологию (используйте критерий базы) (вставить картинку.)\\
Решение. Докажем, что минимум из возможных расстояний до границы шара - 
искомый радиус окрестности, лежащей в пересечении шаров. Рассмторим
точку в этой окретсности. Она лежит в обоих шарах. (вставить выкладку)\\
\textbf{Замечание.} Мы будем использовать обычную топологию и рисовать 
картинки, котоыре помогут доказывать различные теоремы, но все доказательства
будут даны для произвольных метричсеких простарнств. \\
\textbf{Прмиер.} Рассмотрим множество непрерывных функций на отрезке. 
введем следующую метрику: $\rho(f,g)=max|f(x)-g(x)|$. Оперделение
корректно, посокльку на отрехзке супремум непрерыной функции достигается. 
Какие (картика) функции лежат в окретсности произвольной функции $y=f(x)$?
Это - непрерывные функции, заключенные в области $f(x)-r,f(x)+r$\\
\textbf{Замечание.} Если $\Sigma$ - база топологии  $\tau$, то  $\tau$
совпадает с семейством всевозможных объединений множеств из базы.
\subsection{Метризуемость топологических пространств.}
\begin{defin}
    Топологическое пространство называется метризуемым, если на множестве 
    существует метрика,идуцирующая эту топологию.
\end{defin}
Мы уже доказали, что обычная топология метризуема. Не все, однако, 
топологические пространства метризуемы. 
\begin{defin}
Пусть Х - топологическое пространство, $H\subset X$. Окрестностью 
подмножества в Х называется подмножество, содержащее его. 
Окрестностью точки называется любое открытое множество, содержащее точку
(обозначение: $U_x$)
\end{defin}
\begin{defin}
Топологическое пространство называется хаусдорфовым, если любые две точки 
обладают непересекающимися окрестностями. 
\end{defin}
\begin{theor}
Любое метризуемое топологическое пространство хаусдорфово.
\end{theor}
\textbf{Доказательство.}  \
$\square$ 












