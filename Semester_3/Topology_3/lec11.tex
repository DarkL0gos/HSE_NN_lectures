\begin{theor}
    (сепарабельность в пространстве со счетной базой)\\
    Пусть $(X,\tau)$ имеет счетную базу (AC II). Отсюда следует сеперабельность
    (обратное вообще говоря неверно).
\end{theor}
\textbf{Доказательство.} Пусть $\Sigma=\{W_i\mid i \in I\}$ - счетная база
топологии. Возьмем по точке из каждого множества базы и образуем из них
множество $S$ - счетное подмножество $X$. Покажем, что $S$ всюду плотно в $X$. 
Любое открытое множество можно представить в виде объединения множеств из
базы, значит, в любом открытом множестве лежит точка из множества $S$. 
По лемме о всюду плотном множестве, из этого следует плотность всюду $S$ в 
$X$.

Приведем контрпример к обратному утверждению. Рассмотрим  плоскость с 
топологией Зарисского. Множество $S=\{(0,n)|n\in \mathbb{N}\}$ - 
счетное подмножество. По лемме, $\overline{S}=\mathbb{R}^2$. Значит, это
пространство сепарабельно. Но у него нет счетной базы: действительно, каждое
множество базы открыто, то есть является дополнением до конечного числа 
точек. Рассмотрим множество $\mathbb{R}\setminus \{b\}$. Оно открыто, 
но точка $b$ (взятая так, чтобы она не лежала в любом множестве базы счетной 
базы) лежит в базе, значит, это множество не является объединением
базовых множеств. $\square$ 
\subsection{Сепарабельность в метрических пространствах}
\begin{theor}
Пусть $(X,d)$ - метрическое пространство с индуцированной топологией. 
Тогда $X$ сепарабельно тогда и только тогда, когда оно удовлетворяет
второй аксиоме счетности. 
\end{theor}
\textbf{Доказательство.} По предыдущей теореме из счетной базы следует
сеперабельность. Обратно, пусть $(X,\tau_d)$ сепарабельно. Тогда существует
такое счетное $S$, что  $\overline{S}=X$. Докажем, что у него имеется счетная 
база. По лемме о всюду плотном множестве, любое открытое $U_i$ пересекается
с $S$. Возьмем по точке $s_i$ из всех таких пересечений и рассмотрим счетное 
множество шаров  $\Sigma=\{D_{\frac{1}{n}}(s_i)\mid s_i\in S\cap U_i,
n\in \mathbb{N}\}$. Покажем, что это база, используя критерий базы топологии. 
Так как $S$ было всюду плотным, то оно пересекается с любой
$\varepsilon$-окрестностью любой точки $x$. Найдем такое $n\in \mathbb{N}:
\frac{1}{2n}<\varepsilon$, и рассмотрим $D_{\frac{1}{2n}}(x)$. Так как
$S$ всюду плотно в $X$, то найдется $s_i\in D_{\frac{1}{2n}}(x)\cap S$. Тогда
$x\in D_{\frac{1}{2n}}(s_i)\subset D_\varepsilon(x)$. 
Значит, семейство множеств
$\Sigma$ удовлетворяет критерию базы топологии. $\square$ 
\begin{theor}
(Линдлёфа)\\
Из любого открытого покрытия пространства со счетной базой можно выделить
счетное подпокрытие. 
\end{theor}
\textbf{Доказательство.} Пусть $\Sigma$ - счетная база на $(X,\tau)$,
$\xi=\{U_{\alpha}\mid \alpha\in \mathbb{N}\}$ - 
открытое покрытие $X$. Пусть  $x\in X=\bigcup\limits_{\alpha\in I}
U_\alpha$. Тогда $\exists \alpha:x\in U_\alpha$. Но по критерию базы
топологии, найдется такое множество из базы, что $x\in W_i\subset U_\alpha$.
Теперь для любого  $W_i$ выберем одно открытое множество  $U_{\alpha_i}$,
удовлетворяющее этому включению. Рассмотрим теперь 
$\tilde\xi=\{U_{\alpha_i}\mid i\in \mathbb{N}_0\subset \mathbb{N}\}$.
Очевидно, $\tilde\xi\subset \xi$ и согласно включению $\tilde\xi$ - покрытие.
По построению, оно счетное. Значит, это счетное подпокрытие. $\square$ \\
\subsection{Аксиомы отделимости}
\begin{defin}
Аксиомы отедлимости:
\end{defin}
\textbf{$T_0$}: если для любых двух различных точек найдется окрестность 
по крайней 
мере одной из них, не содержащей другую точку.\\
\textbf{$T_1$}: Пространство $(X,\tau)$ удовлетворяет  $T_1$, если для 
любых двух
точек существуют окрестности, не содержащие другую точку.\\
\textbf{$T_2$}: (аксиома Хаусдорфа): любые две различные точки имеют 
непересекающиеся окрестности.\\
\textbf{$T_3$}: для любого замкнутого множества и точки не из него 
существуют непересекающиеся окрестности.\\
\textbf{$T_4$}: для любых непересекающихся замкнутых множеств найдутся их
непересекающиеся окрестности. 

\textbf{Пример.} $\tau_{MN}$ - топология, не удовлетворяющаяя  $T_0$.
\begin{theor}
    (критерий $T_1$-пространства)\\
    Топологическое пространство удовлетворяет $T_1$ тогда и только тогда,
    когда каждое одноточечное множество замкнуто.
\end{theor}
\textbf{Доказательство.} Пусть $(X,\tau)$ удовлетворяет  $T_1$. Рассмотрим 
дополнение $X\setminus \{x\}$. Рассмотрим $y\in X\setminus \{x\}$. 
Из-за этого $x\ne y$, по аксиоме у точки $y$ есть окрестность, не
пересекающаяся с $x$, поэтому по лемме об открытом множестве  
$X\setminus \{x\}$ открыто, и значит $\{x\}$ замкнуто. 

Обратно, пусть $\{a\},\{b\}$ - замкнутые множества. Тогда $X\setminus \{b\}$,
$X\setminus \{a\}$ являются открытыми окрестностями для точек $a,b$ 
соответсвтенно, значит, выполнена  $T_1$. $\square$ \\

\textbf{Пример.} Связное двоеточие $T_0$, но не  $T_1$. 

\textbf{Пример.} Топология Зарисского $T_1$, но не  $T_2$



