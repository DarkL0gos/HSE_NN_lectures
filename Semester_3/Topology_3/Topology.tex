\documentclass[a4paper]{article}
%%%%%%%%%%%%%%%%%%%%%%%%%%%%%%%%%%%%%%%%%%
%  My documentation report
%  Objetive: Explain what I did and how, so someone can continue with the investigation
%
% Important note:
% Chapter heading images should have a 2:1 width:height ratio,
% e.g. 920px width and 460px height.
%
%%%%%%%%%%%%%%%%%%%%%%%%%%%%%%%%%%%%%%%%%


%----------------------------------------------------------------------------------------
%	PACKAGES AND OTHER DOCUMENT CONFIGURATIONS
%----------------------------------------------------------------------------------------

\documentclass[11pt,fleqn]{book} % Default font size and left-justified equations

\usepackage[top=3cm,bottom=3cm,left=3.2cm,right=3.2cm,headsep=10pt,letterpaper]{geometry} % Page margins

\usepackage{xcolor} % Required for specifying colors by name
\definecolor{ocre}{RGB}{52,177,201} % Define the orange color used for highlighting throughout the book
\usepackage{graphicx}
\usepackage{wrapfig}
% Font Settings
\usepackage{avant} % Use the Avantgarde font for headings
%\usepackage{times} % Use the Times font for headings
\usepackage{mathptmx} % Use the Adobe Times Roman as the default text font together with math symbols from the Sym­bol, Chancery and Com­puter Modern fonts
\usepackage{microtype} % Slightly tweak font spacing for aesthetics
\usepackage[utf8]{inputenc} % Required for including letters with accents
\usepackage[T2A]{fontenc} % Use 8-bit encoding that has 256 glyphs
\usepackage[english,russian]{babel}
\usepackage{amsthm}
\usepackage{xcolor}

% Bibliography
\usepackage[style=alphabetic,sorting=nyt,sortcites=true,autopunct=true,=hyphen,hyperref=true,abbreviate=false,backref=true,backend=biber]{biblatex}
\addbibresource{bibliography.bib}
\defbibheading{bibempty}{}

\input{structure} % Insert the commands.tex file which contains the majority of the structure behind the template

%----------------------------------------------------------------------------------------
%	Definitions of new commands
%----------------------------------------------------------------------------------------

\def\R{\mathbb{R}}

\begin{document}

%----------------------------------------------------------------------------------------
%	TITLE PAGE
%----------------------------------------------------------------------------------------

\begingroup
\thispagestyle{empty}
\AddToShipoutPicture*{\put(0,0){\includegraphics[scale=1.25]{esahubble}}} % Image background
\centering
\vspace*{6cm}
\par\normalfont\fontsize{32}{32}\sffamily\selectfont
\textcolor{white}{\textbf{Фундаментальная математика}}\\
\textcolor{white}{\LARGE Введение в топологию}\par % Book title
\vspace*{2cm}
\textcolor{white}{\huge Лобанов И.В., Зорин Ю.Ю.}\par % Author name
\endgroup

%----------------------------------------------------------------------------------------
%	COPYRIGHT PAGE
%----------------------------------------------------------------------------------------

\newpage
~\vfill
\thispagestyle{empty}

\noindent \textsc{Сентябрь 2022г - Январь 2023г.}\\

\noindent Это пособие было создано с целью помочь юным математиком в изучении такой интересной науки, как топология. Учебный материал основан на лекциях преподавателя НИУ ВШЭ в г. Нижний Новгород Жуковой Н.И. Курс по топологии читался с сентября 2022 г. по июнь 2023г. \\ 

\noindent \textit{Первое издание, сентябрь 2022г. }

%----------------------------------------------------------------------------------------
%	TABLE OF CONTENTS
%----------------------------------------------------------------------------------------

\chapterimage{head1.png} % Table of contents heading image

\pagestyle{empty} % No headers

\tableofcontents % Print the table of contents itself

\cleardoublepage % Forces the first chapter to start on an odd page so it's on the right

\pagestyle{fancy} % Print headers again

%----------------------------------------------------------------------------------------
%	CHAPTER 1
%----------------------------------------------------------------------------------------

%\input{structure.tex}
%\input{Styled.tex}
\usepackage[14pt]{extsizes}                                         %Размер шрифта
\usepackage[left=2.5cm,right=2.5cm,top=2.5cm,bottom=3cm]{geometry}  %Поля страницы

%Настройки ссылок и гиперссылок
\usepackage{graphicx}
\usepackage{hyperref}                 
\usepackage{xcolor}
\definecolor{linkcolor}{HTML}{799B03} % цвет ссылок
\definecolor{urlcolor}{HTML}{799B03}  % цвет гиперссылок
\hypersetup{pdfstartview=FitH,linkcolor=linkcolor,urlcolor=urlcolor,colorlinks=true}

%Пакеты символов
\usepackage{cmap}
\usepackage[T2A]{fontenc}
\usepackage[utf8]{inputenc}
\usepackage[russian]{babel}           
\usepackage{amsmath}
\usepackage{amssymb}
\usepackage{amsfonts}

%Новые команды 
\newtheorem{defin}{Определение}
\newtheorem{example}{Пример}
\newtheorem{zam}{Замечание}
\newtheorem{theor}{Теорема}

\author{}
\title{Топология}
\date{5.09.22}

\begin{document}
%\newgeometry{left=0cm,top=0cm}
%\begin{figure}
%    \includegraphics{pictures/folio.pdf}
%\end{figure}
%\restoregeometry
%\clearpage
%\maketitle
\tableofcontents
\newpage
%Семинар Шубина 05.09.2022
%\chapterimage{head2.png} % Chapter heading image		
%\chapter{Введение в топологию}
\begin{defin}
    Пусть Х - множество. Топологией на Х называется семейство подмножеств 
    $\tau\in\mathcal{P}(X)$, называемых открытыми множествами (данной 
    топологии), такое, что:\\
    1. $X,\varnothing\in\tau$\\
    2. $U_1,\ldots U_n\in\tau\Rightarrow\bigcap\limits^{n}_{i=1}U_i\in\tau$\\
    3. $\{U_i\mid i\in I\}\subset\tau\Rightarrow\bigcup\limits_{i\in I}U_i\in\tau$
\end{defin}
То есть, топологии принадлежит само множество и пустое множество, пересечение
конечного числа множеств и объединение любого числа множеств из топологии. 

Пример. Докажем, что открытые множества в смысле евклидовой метрики в 
$\mathbb{R}^n$ - топология. Очевидно, открыто само $\mathbb{R}^n$, также 
открытои пустое множество. Открытость пересечения доказывается тем, что
наименьшая эпсилон-окрестность принадлежит всем множествам,то есть лежит в их
пересечении, слеовательно, оно открыто. Для объединения: для каждой точки 
найдется множество, в которое она входит с окрестностью.

\begin{defin}
Тривиальная топология - $\tau_t=\{X,\varnothing\} $ \\
Дискретная топология - $\tau_0=\mathcal{P}(X)$ 
\end{defin}
Любая инетерсная топология содержит тривиальную и содержится в дискретной.

Пример. Множества, симметричные относительно выбранной прямой в евклидовом
пространстве,образуют топологию.

Пример. Множество эпсилон-окрестностей нуля $\tau=\{D_\varepsilon(0)\mid
\varepsilon>0\}\cup\{X,\varnothing\} $
- топология.

Пример. Топология Зарисского - топология множеств, дополнительных к конечным 
множествам (для конечных пространств совпадает с дискретной).

Пример. Пусть $f:X\to X$ - биекция. Докажем, что $\tau_f=\{U\subset X\mid$






\begin{theor}
    (критерий сходимости для неотрицательных рядов)\\
    Пустьдан ряд. Тогда ряд сходится $\Leftrightarrow$ последовательность
    частичных сумм ограничена сверху.
\end{theor}
\textbf{Доказательство.} $\Rightarrow$. По услови, существует предел 
$lim S_n=S\in \mathbb{R}$ $\Rightarrow$ $\{S_n\}_n\in\mathbb{N}$ - ограничена
В другую сторону. По условию, $\{S_n\}$ ограничена сверху, $\Rightarrow$ по тео
реме Вейрштрасса для ограниченной неубывающей последовательности имеется предел
$\square$ \\

\textbf{Признак сравнения.} С чем же сравнивать? С геометрической прогрессией, 
с обобщенным гармоническим рядом (с произвольной степенью числа). 
\begin{theor}
    (признак сравнения в оценочной форме)\\
    Дано $0\leqslant a_n\leqslant b_n~\forall n\in\mathbb{N}$ :
    Тогда из сходимости В следует сходимость А, из расходимости А следует
    расходимость В.
\end{theor}
\textbf{Доказательство.}  Докажем исходя из критерия сходимости. \\
1. Пусть $A_n,B_n$ - частичные суммы своих рядов. Так как ряд В сходится, 
то существует верхний предел для его частичных сумм. Так как ряд А меньше Б,
по транзитиавности неравенств верхняя граница В лежит выше чем А. ЧТо по тому 
же критерию дает сходимость. 
2. 
$\square$ \\

Пример. Рассмотрим $p<1$,  $n^p<1$,  $\frac{1}{n^p}>\frac{1}{n}$. Так как 
гармонический ряд расходится, то $sum \frac{1}{n^p}$ расходится. 

\textbf{Пример.} Найти сумму. $\sqrt{2}+\sqrt{2-\sqrt{2} }+\sqrt{2-
\sqrt{2+\sqrt{2} } } +...$, $a_{n+1}=\sqrt{2-b_n} $, $b_{n+1}=\sqrt{2+b_n} $.
Заметим, что $b_1=2\cos\frac{\pi}{4}$, $b_2\cos\frac{\pi}{8}$. Дальше
эта формула выводится по индукции. $b_n=2\cos\frac{\pi}{2^{n+1}}$. 
$a_n=\sqrt{2-b_{n-1}}=\sqrt{2-2\cos\frac{\pi}{2^n}}=2\sin\frac{\pi}{2^{n+1}}$ 
Ита, $a_n\leqslant 2\cdot \frac{\pi}{2^{n+1}}=\frac{\pi}{2^n}$ 

\begin{theor}
    (Признак сравнения в предельной форме)\\
    Пусть даны неотрицательные ряды $\sum\limits_{n=1}^{\infty} a_n$, 
    $\sum\limits_{n=1}^{\infty}b_n$. Если предел отношения общего члена\\
    1. Равен конечной (ненулевой) константе. Тогда ряды сходятся или расходятся 
    одновременно\\
    1.1. В частности, при mkk=1, ряды эквивалентны. 
2. Если $\lim\limits_{n \to \infty} \frac{a_n}{b_n}=0$, то имеет место
"В сходится $\Rightarrow$ А сходится"
3. Если этот предел равен $ooo$, то: "А сходится $\Rightarrow$ В сходится"


\end{theor}
\textbf{Доказательство.} По опреелению предела. 
$\lim\limits_{n \to \infty} a_\frac{n}{b_n}=k$ для $\varepsilon=k/2>0\exists 
n_0(\varepsilon)\forall n>n_0: k/2<\frac{a_n}{b_n}<3k/2$. тогда если В 
сходится, А сходится.

2. Пусть $\lim\limits_{n \to \infty} a_\frac{n}{b_n}=0$. Lkz $\varepsilon=1$, 
тогда для этого эпсилон  $\exists n_0$ утверждение следует из первого
признака сравнения. 

Пункт 3 напрямую следует из второго.
$\square$ 

\textbf{Пример. 3}  $\sum\limits_{n=1}^{\infty}(\frac{1}{n^\alpha}-
\frac{1}{(n+1)^\alpha})$. Имеем $S_n=1-\frac{1}{(n+1)^\alpha}$ Прии 
альфа>0 сходится к 1, при альфа<0 ряд расходится.
(ljнайддем область расходимости обобщенного гармонического
рядва с помощью уже известного)

\begin{theor}
    (тертий признак сравнения.)\\
Пусть даны ряды А и В ($\sum\limits_{n=1}^{\infty} a_n,~\sum
\limits_{n=1}^{\infty} b_n$),и выполняется $a_{n+1}/a_n\leqslant b_{n+1}/b_n$
Тогда В сходится $\Rightarrow$ А сходится
(если А расходится, В расходится)

\end{theor}
\textbf{Доказательство.}  так как все неравенства полоэительные, их всех можно
перемножить: тогда утверждение следует из первого признака сравнения. 
\
$\square$ 

\begin{theor}
    (Признак Даламбера в оценочной форме)\\
\end{theor}
\textbf{Доказательство.}  \
$\square$ 

\begin{theor}
Признак даламбера в предельной форме: $\lim\limits_{n \to \infty} $
\end{theor}
\textbf{Доказательство.}  \
$\square$ 



















\section{Элементарные методы интегрирования ДУ}
\subsection{Уравнения с разделяющимися переменными}
\begin{defin}
Уравнение с разделяющими переменными - уравнение вида
\begin{equation}
    \frac{dx}{dt}=f(x)g(t) \label{ODE_razdp}
\end{equation}
где $f,g$ непрерывны на  $x\in(a,b),~t\in(\alpha,\beta)$
\end{defin}
Как решать такие уравнения? Алгебраическая нтуиция подсказывает, что надо 
перенести 
дифференциалы к своим функциям и проинтегрировать. Но это ещё надо обосновать.
Сделаем следующее:\\
\begin{enumerate}
    \item Найти все $x_*:f(x_*)=0$. Тогда $x=x_*$ - решение-константа. 
    \item Пусть  $x^i_*,x^j_*$ - такие, что  $f(x^i_*)=f(x^j_*)=0$ и
    $\forall x\in(x^i_*,x^j_*):f(x)\ne0$. Тогда уравнение \ref{ODE_razdp}
эквивалентно уравнению 
$$\frac{dx}{f(x)}=g(t)dt$$
Эту штуку можно проинтегрировать с обеих сторон. Результат непрерывен и не
обращается в ноль. Значит, по теореме о неявной функции найдется решение. 
$\frac{dF}{dx}=\frac{1}{x}$(решение в области $(\alpha,\beta)\times
(x^i_*,x^j_*)$).
    \item Выписать решение на каждом интервале $(x^i_*,x^j_*)$
\end{enumerate}
Других решений не существует. Почему? Допустим, существует другое решение.
Оно не может быть константой, так как все константы были получены в п.1.
Если она \\
\textbf{Пример.} Решим уравнение $\frac{dx}{dt}=0$. Решение-константа: $x=0$.
Теперь рассмотрим два интервала: $x<0$ и  $x>0$. Если  $x<0$, имеем уравнение
 $$\frac{1}{x}\frac{dxdt}{dt}=dt$$
 Интегрируем:
 $$\int\frac{dx}{x}=\int dt$$
 Получаем, что $\ln|x|=t+C$. Выражаем искомую функцию (не забыв, на каком
 промежутке мы рассматриваем функцию, и раскрыв модуль соответственно):
 $$x=-Ce^t,~C>0$$
Для интервала $x>0$ точно такой же порядок действий, только получим другой 
знак. Итак, множество решений:
$$x=Ce^t,~C\in\mathbb{R}$$
\subsection{Уравнения, приводящиеся к уравнению с разделяющимися переменными}
\begin{defin}
Уравнение, приводящееся к уранвению с разделяющмися переменными - уравнение
вида 
\begin{equation}
    \frac{dx}{dt}=f(at+bx+c) \label{ODE_privrazd}
\end{equation}
\end{defin}
Давайте решим его. 
\begin{enumerate}
    \item Введем замену $z(t)=at+bx+c$. 
    Имеем
     $$\frac{dz}{dt}=a+b\frac{dx}{dt}$$ 
     Получаем уравнение с разделяющимися переменными. 
     $$\frac{dz}{a+f(z)}=dt$$
\end{enumerate}
\textbf{Пример.} Решим уравнение $\frac{dx}{dt}=\cos(x+t)$. Замена 
$z=x+t,~ \frac{dz}{dt}=1$. Уравнение имеет вид
$$\frac{dz}{dt}=\frac{dx}{dt}+1$$ 
Найдем $\cos{z_*}+1=0$: это, очевидно, $\pi+2\pi k,~k\in \mathbb{Z}$ 
Свели задачу кпрошлому пункту
\subsection{Однородные уравнения}
Сначала докажем, что два определения однородного уравнения эквивалентны.
\begin{defin}
Однородным называется уравнение вида
\begin{equation}\label{ODE_odn1}
    \frac{dx}{dt}=f\left(\frac{x}{t}\right) \label{ODE_odn1}
\end{equation} 
\end{defin}
Это уравнение инвариантно относительно замены $x\mapsto kx,~t\mapsto kt$.
Геометрически это означает, что совокупность интегральных кривых инвариантно
относительно преобразования $\theta(x,y)=(kx,ky)$.
Из этого следует, что если мы найдем одно решение, то мы найдем всю 
совокупность ему подобных. Вставить картинку.
\begin{defin}
    (вспомогательное)\\
Уравнение в форме дифференциалов:
    $M(x,y)dx+N(x,y)dy=0$.  
\end{defin}
Это таже форма, что и $\frac{dy}{dx}=f(x,y)$, поскольку 
$\frac{dy}{dx}=-\frac{M(x,y)}{N(x,y)}$. Обратно, $-f(x,y)dx+dy=0$.
Уравнение в форме дифференциалов имеет чуть большее множество решений. 
\begin{defin}\label{ODE_odn2}
Уравнение в форме дифференциалов называется однородным, если\\
$M(kx,ky)=k^nM(x,y)$\\ 
$N(kx,ky)=k^nN(x,y)$\\
n называется степенью однородности.
\end{defin}
\begin{theor}
    Определения \ref{ODE_odn1} и \ref{ODE_odn2} эквивалентны. 
\end{theor}
\textbf{Доказательство.} 1 $\Rightarrow$ 2. $\frac{dy}{dx}=f(\frac{y}{x})$\\
2 $\Rightarrow$ 1. Пусть дано уравнение в форме дифференциалов. Подставим $k$.
При $x\ne 0$ Имеем $$\frac{dx}{dy}=-\frac{k^nM(x,y)}{k^nN(x,y)}=
-\frac{M(kx,ky)}{N(kx,ky)}=-\frac{M(1,\frac{y}{x})}{N(1,\frac{y}{x})}=f(x)$$
$\square$ \\
\textbf{Пример.} $M=x^2+y^2$\\
\textbf{Пример (№31).} Найти уравнение, решение которых - параболы с осью, 
параллельной оси ординат и касающиеся прямых $y=0,~y=x$. 
Во-первых, поймем, как выглядит уравнение такой параболы. Исходя из геометрии,
получим, что уравнение параболы, удовлетворяющее первому условию, имеет вид 
$y=ax^2+bx+\frac{b^2}{4a}$, а первому и второму - $y=ax^2+\frac{1}{2}x+
\frac{1}{16a}$. Остался один параметр $\Rightarrow$ уравнение первого порядка. 
Подставляем и хаваем ответ бесплатно:
$$y=\left(\frac{y'-\frac{1}{2}}{2x}\right)x^2+\frac{1}{2}x+\frac{2x}{16y'-8}$$ 
\textbf{Пример (№72).} Найти линии, у которых треугольники, образованные 
касательными, осью ОХ и точкой касания, имеют одинаковую сумму катетов. 
Из геометрических соображений имеем уравнение 
$$\frac{|y|}{|y'|}+|y|=b=const$$ 
Раскрываем модули. В простейшем случае имеет уравнение с разделяющимися 
переменными. 
$$\frac{dy}{dx}=\frac{y}{b-y}$$ 
Остальные уравнения такие же в принципе. Так шо это идет в дз 
Его легчайшее (и, видимо, общее) решение: $x+C=\pm b\ln{|y|}\pm y$\\
\textbf{Пример (№76).} Геометрическая интуиция не должна подводить нас. 
Вставить картинку. Есть кароч такая формула: 
$\tg\gamma=\frac{r}{r'}$





\section{Связь признака Даламбера и Коши}
Если $\frac{a_n}{a_{n-1}}\leqslant q$ для 
всех n начиная с 1, то $a_n=a_1q^n$, откуда следует признак Коши. 
$$\sqrt[n]{a_n}\leqslant \sqrt[n]{a_1}\cdot q$$
Значит, Коши покрывает больше случаев. 
\section{Оценка погрешности приближения какой-то величины с помощью
положительного ряда}
$$\int^\infty_{n+1} f(x)dx<R_n\leqslant \int^\infty_nf(x)dx$$ 
Из доказательства интегрального признака
$$a_{k+1}<\int^{k+1}_kf(x)dx\leqslant a_k$$ 
$$\int^{k+1}_kf(x)dx\leqslant a_k\int^k_{k-1}f(x)dx$$ 
$$R_n=\sum\limits_{k=n+1}^{\infty} a_k$$ 
Итак, 
$$\int^\infty_{n+1}\leqslant R_n<\int^\infty_nf(x)dx$$
\textbf{Пример.} Вычислим с точностью до 0,001 ряд 
$\sum\limits_{n=1}^{\infty} \frac{1}{n^4}$. Ответ: $1,082\pm0,001$
(точный ответ $\frac{\pi^4}{90}$)
\section{Знакопеременные ряды}
Пусть теперь ряд знакопеременный.
\begin{defin}
Ряд сходится абсолютно, если сходится ряд из модулей. Ряд сходится условно,
если абсолютно расходится, но сам сходится. 
\end{defin}
\begin{theor}
Если ряд сходится абсолютно, то ряд сходится.
\end{theor}
\textbf{Доказательство.}  Следует напрямую из критерия Коши
и свойства модуля: $| |a_1|+...|a_n| |\geqslant|a_1+...+a_n|$.
$\square$ 
\begin{theor}
    (признак Лейбница для знакочередующихся рядов)\\
    Пусть ряд имеет вид $\sum\limits_{n=1}^{\infty} (-1)^nv_n$, 
    где $v_n>0$ и монотонно убывает. Тогда ряд сходится.  
    Более того, имеет место оценка погрешности $|R_n|\leqslant v_n$
\end{theor}
\textbf{Доказательство.}  1. Посчитаем частичную сумму для $2k:$
$$S_{2k}=v_1-v_2+...-v_{2k}$$ 
$$S_{2k+2}=S_{2k}+v_{2k+1}-v_{2k+2}$$ 
$$S_{2k+2}-S_{2k}=v_{2k+1}-v_{2k+2}$$ 
$$S_{2k}=v_1-(v_2-v_3)-(v_4-v_5)-...-(v_{2k-2}-v_{2k-1})-v_{2k}$$ 
Значит, эта последовательность возрастает и ограничена сверху, значит, у неё
есть конечный предел: $S_{2k}\leqslant u_1$
$$\lim\limits_{k \to \infty} S_{2k+1}=\lim\limits_{k \to \infty} (S_{2k}+
v_{2k+1})=S$$ 
Следовательно, 
$$\exists \lim\limits_{n \to \infty} S_n=S$$
Последовательность частичных сумм для нечетных чисел также убывает, 
доказательство аналогичное. \\
2. Докажем оценку погрешности. $|R_{2k}|=S-S_{2k}<S_{2k+1}-S_{2k}$. Итак,
$$|R_{2k}|\leqslant v_{2k+1}$$ 
$$R_{2k+1}=S_{2k+1}-S<S_{2k+1}-S_{2k+2}$$ 
$$|R_{2k+1}|\leqslant v_{2k+2}$$
$\square$ 
\subsection{Преобразование Абеля}
$$\sum\limits_{k=1}^{n} a_k b_k=\sum\limits_{k=1}^{n-1} (a_k-a_{k+1})B_k
+a_nB_n,~B_i=\sum\limits_{k=1}^{i} b_k$$ 
Доказательство. $b_k=B_{k}-B_{k-1},~k\in \{2,...,n\} $ ВСТАВКА

\begin{theor}
    (неравенство Абеля)\\
    Пусть последовательность монотонно возрастает или убывает. 
    И пусть $\exists M\forall k\in \{1...n\}|B_k|\leqslant M $.
    Тогда модуль конечной суммы $\leqslant M(|a_1|+2|a_n|)$ 
\end{theor}
\textbf{Доказательство.} Юра, допиши пж 
$\square$ 
\begin{theor}
    (признак Дирихле)


\end{theor}
\textbf{Доказательство.}  \
$\square$ 












\section{Действия над абсолютно сходящимися рядами}
\begin{theor}
Если ряд сходится абсолютно, то ряд, умноженный на константу, сходится абсо
лютно. 
\end{theor}
\textbf{Доказательство.} Зафиксируем $\varepsilon$. Найдем такой номер, что
ряд из модулей меньше чем $\frac{\varepsilon}{|c|}$. И в общем эта штука
сходится. 
$\square$ 
\begin{theor}
Сумма абсолютно сходящихся рядов абсолютно сходится.
\end{theor}
\textbf{Доказательство.}  Сумма модулей больше модуля суммы.
$\square$ 
\begin{theor}
    (О произведении абсолютносходящихся рядов)\\
    Сумма всевозможных произведений $a_ib_j$ сходится абсолютно, и сумма ряда
    равна произведению сумм.
\end{theor}
\textbf{Доказательство.} Введем две переменные с модулями. Введем новые
обозначения, как в прошлой теореме. Пользуясь этой же теоремой, мы можем
доказать абсолютную сходимость для хотя бы одного из упорядочиваний. 
Представим себе бесконечную матрицу $|a_ib_j|$. Будем рассматривать 
последовательность частичных сумм в угловых минорах. Для них имеем формулу
$S_{n^2}=S'_n\cdot S''_n$. По условию,в правой части есть оба предела, а 
значит и слева тоже есть. И ещё, $S_{n^2}\leqslant S_m\leqslant S_{(n+1)^2}$.
Ну кароч....че то мдэ, тут дофига текста.
$\square$ 
\begin{defin}
    (произведение рядов по Коши)\\
Пусть $S_a\cdot S_b=S_c$. имеемследующее произведение:\\
$c_1=a_1b_1$\\
 $c_2=a_1b_2+a_2b_1$\\
 $c_3=a_1b_3+a_2b_2+a_3b_3$\\
 То есть суммируем по диагональкам той бесконечной матрицы.
\end{defin}
\textbf{Пример 1.} $a_n=\frac{1}{n(n+1)}=1,~b_n=\frac{n}{2^n}$. Тогда
$\sum\limits_{n=1}^{\infty} c_n=\sum\limits_{n=1}^{\infty} \sum\limits_{k=1}
^{n} \frac{n+1-k}{k(k+1)-2^{n+1-k}}$.\\
\textbf{Пример 2.} Произведение расходящихся рядов $a_n=1,5^n,~b_n=1-1,5^n$
в смысле Коши - сходится, так как $c_n=0,75^n$. \\
Заметим, что условной сходимости недостаточно! Так, для $a_n=b_n=(-1)^{n-1}/
\sqrt{n}$ ничего не выйдет. Смиритесь. Ребят а че вы с пары то свалили. 
Чувствую себя лохом, и от этого неуютненько.
\section{Перестановки условно-сходящихся рядов}
\begin{theor}
Лемма о сходимости. Ряд $a_n$ сходится условно. Рассмотрим отдельно
подпоследовательности из положительных и отрицательных членов. Тогда их суммы
 $+\infty,-\infty$ соответственно. 
\end{theor}
\textbf{Доказательство.}  \
$\square$ 
\begin{theor}
    (Римана)\\
    Если рядсходится услвоно, то для любого действительного числа найдется
    такая перестановка ряда, при которой ряд сходится к этому числу.
\end{theor}
\textbf{Доказательство.} По предыдущей лемме, ряд из положительных членов расходится,
значит, найдется частичная сумма, большая чем искомое число. Дальше найдем 
такую частичну сумм из отрицатльных членов, чтобы, прибавв её к прошлому этапу,
получили снова меньше чем число. И так далее.  
$\square$ 


%Лекция 13.10.22
\section{Уравнения и ряды Тейлора}
Пусть $\frac{dx}{dt}=f(t,x)$. Рассмотрим $x(t_0)=x_0$. Разложим в ряд
Тейлора: $x(t)=x(t_0)+\frac{dx}{dt}(t_0)(t-t_0)+o(t-t_0)$.
Отбросив члены высшего порядка (прямо как топовые физики), получим 
приближенное решение. Приближенные решение можно итерировать, и это
будет широко известный \textbf{метод Эйлера} (первого
порядка). $t_{k+1}=t_k+h,~x_{k+1}=x_k+f(t_k,x_k)h$ 

\section{Практика}
\textbf{Пример (№111)}. $(y+\sqrt{xy})dx=xdy$. Уравнение однородно (
проверим умножением на $k$). Значит, делаем замену $u(x)=\frac{y}{x}$.
Имеем $dy=u\cdot dx+du\cdot x$. Переменные разделяются: 
$\frac{dx}{x}=\frac{du}{\sqrt{u}}$\\
\textbf{Пример (№113)}. $(2x-4y+6)dx+(x+y-3)dy$. Переносим начало координат
в точку пересечения.\\
\textbf{Пример (№126)}. $y'=y^2-\frac{2}{x^2}$. Это - обобщенно-однородное
уравнение, то есть приводится к однородному заменой $y=z^m(x)$.
$y'=mz^{m-1}z$ Далее
$mz^{m-1}z=z^{2m}-\frac{2}{x^2}$ 
Теперь уравнение однородно. Введем замену $\frac{z}{x}=u,~z=ux$.
Получим $u'x+u=-1+2u^2$\\
\textbf{Пример (№128)}. $\frac{2}{3}xyy'=\sqrt{x^6-y^4}+y^2$. 
Пусть $y=z^m$. Идея: сделать так, чтобы под корнем степень у $x$ и $y$ была
одинаковой.\\
\textbf{Пример (№)} $2xydx+(x^2-y^2)dy=0$. Подберем функцию, полным 
дифференицалом которого является это выражение; получим  $F(x,y)=
x^2y-\frac{1}{3}y^3$. Решние: $F=C=const$\\
\textbf{Пример (№192)}. $(1+y^2\sin{2x})dx-2y\cos^2{x}dy$. Мы должны 
показать, что вторые производные равны. Тогда это значит, что
$F_{xy}=F_{yx}$, и такая функция вообще существует на некотором диске
(где правая часть не обращается в ноль). Интегируем два раза, и найдем эту
функцию: $F(x,y)=x-y^2 \frac{1}{2}\cos{2x}-\frac{y^2}{2}+C_0$.
Итак, ответ: $\boxed{F=const}$ \\
\textbf{Пример (№202)}. $y^2dx+(xy+\tg{xy})dy=0$. Является ли однородным,
в полных дифференциалах? Давайте раскроем скобки и сгруппируем:
$y(ydx+xdy)+\tg{xy}dy$. Это то же, что и  $\frac{d(xy)}{\tg{xy}}+\frac{dy}{y}
=0$. Домножим на $\frac{1}{y\tg{xy}}$ и хаваем уравнение в полных 
дифференицалах бесплатно. То, на что домножили - интегрирующий множитель.















\begin{theor}
    (признак Абеля равномерной сходимости функционального ряда)\\
Дан ряд $\sum\limits_{n=1}^{\infty} a_n(x)b_n(x)$ и $\forall x\in X$:\\
1. $|a_n(x)|\leqslant M=const$ для всех $n$;\\
2.  $\{a_n(x)\} $ мнонотонна;\\
3. $\sum\limits_{n=1}^{\infty} b_n(x)$ равномерно сходится на $X$;\\
Тогда исходный ряд равномерно сходится на  $X$.
\end{theor}
\textbf{Доказательство.}  По определению Коши. Фиксируем $\varepsilon>0$.
Так как ряд с общим членом $b_n$ сходится равномерно, то по критерию Коши для
$$\frac{\varepsilon}{3M}>0~\exists n_0(\varepsilon)~\forall n>n_0~\forall p\in
\mathbb{N}~ \forall x\in X:\left|\sum\limits_{k=n+1}^{n+p} b_k(x)\right|<
\frac{\varepsilon}{3M}$$ Тогда по неравенству Абеля 
$$\right|\sum\limits_{k=n+1}^{n+p} b_k(x)a_k(x)\left|\leqslant 
\frac{\varepsilon}{3M}
(|a_{n+1}|+2|a_{n+p}(x)|)<\frac{\varepsilon}{3M}\cdot 3M=\varepsilon$$
Тогда по критерию Коши этот ряд сходится равномерно на $X$. $\square$ 

\textbf{Пример.} Исследуем на равномерную сходимоcть ряд 
$\sum\limits_{n=1}^{\infty} \frac{\cos{nx}\sin{x}arctg{nx}}{\sqrt{n^2+x^2}}$. 
Алгоритм:\\
1. Арктангенс монотонен и ограничен.\\
2. Все остальное сходится по Дирихле.
\subsection{Свойства равномерно сходящихся рядов}
\begin{theor}
    (о непрерывности суммы равномерно сходящегося ряда)\\
    Дан ряд $\sum\limits_{n=1}^{\infty} a_n(x)$, причем \\
    1. Все функции непрерывны на множестве $X$;\\
2. $\sum\limits_{n=1}^{\infty} a_n(x)$ сходится равномерно к $S(x)$ на $X$;\\
Тогда $S(x)$ непрерывна на $X$. 
\end{theor}
\textbf{Доказательство.}  По условию, сумма из  $a_n(x)$ сходится равномерно
на  $X$ к  $S(x)$, то есть  $S_n(x)\rightrightarrows S(x)$ на  $X$, 
$S_n(x)$ непрерывна как сумма. Тогда по теореме о непрерывности предела
равномерно сходящейся последовательности, составленной из непрерывных
функций,  $S(x)$  непрерывна. Другая формулировка:  
$$\lim\limits_{x\to x_0}\sum\limits_{n=1}^{\infty} a_n(x) =
\sum\limits_{n=1}^{\infty}\lim\limits_{x\to x_0} a_n(x)
$$
(то есть можно поменять местами сумму и предел). $\square$ 

\textbf{Пример.} $\sum\limits_{n=1}^{\infty} \frac{\sin{nx}}{n}=f(x)$ - 
непрерывна на $(0,2\pi)$
\begin{theor}
(об интегрировании равномерно сходящегося ряда)\\
Пусть дан ряд $\sum\limits_{n=1}^{\infty} a_n(x)$, причем \\
 1. все функции непрерывны на отрезке $[a,b];$\\
 2. $\sum\limits_{n=1}^{\infty} a_n(x)$ сходится равномерно на $[a,b]$ к $s
 (x)$;\\
 Тогда $$\forall x,x_0\in[a,b]:~\int\limits^x_{x_0}\left( \sum\limits_{n=1}^
 {\infty} a_n(t) \right)dt=\sum\limits_{n=1}^{\infty} \left( 
\int\limits_{x_0}^{x}a_n(t)dt \right) $$ 
 (можно менять интеграл и сумму).
\end{theor}
\textbf{Доказательство.} Докажем, что $\int\limits^x_{x_0}S(t)dt=\sum\limits_{n=1}^{\infty} \int\limits^x_{x_0}a_n(t)dt$. По предыдущей теореме $S(t)$ 
непрерывна на  $[a,b]$, значит,интегрируема на нем по Риману. 
Обозначим  $\sigma_n(x)=\sum\limits_{k=1}^{n}\int\limits^x_{x_0}a_k(t)dt$ и
докажем, что $\sigma_n(x)\rightrightarrows\int\limits^x_{x_0}S(t)dt$.\\
Зафиксируем $\varepsilon>0$. По условию, $S_n(t)$ равномерно сходится на 
$[a,b]$ для  
$$\frac{\varepsilon}{b-a}>0~\exists n_0(\varepsilon)~\forall 
n>n_0~\forall x\in[a,b]:|S_n(t)-S(t)|<\frac{\varepsilon}{b-a}$$
Тогда
$\left|\sigma_n(x)-\int\limits^x_{x_0}S(t)dt\right|=
\left| \sum\limits_{k=1}^{n} \int\limits_{x_0}^{x} a_k(t)dt-
\int\limits_{x_0}^{x}S(t)dt\right|=\left| \int\limits_{x_0}^{x}(S_n(t)-S(t))dt
\right|\leqslant \left| \int\limits_{x_0}^{x}|S_n(t)-S(t)|dt\right| 
<\frac{\varepsilon}{b-a}\cdot |x-x_0|<\varepsilon$.
Значит, $\sigma_n(x)\rightrightarrows \int\limits_{x_0}^{x} S_n(t)dt$.
$\square$ 
\begin{theor}
(о дифференцировании равномерно сходящегося ряда)\\
Пусть дан ряд $\sum\limits_{n=1}^{\infty} a_n(x)$, причем \\
 1. Производные всех функций непрерывны на отрезке $[a,b];$\\
 2. $\sum\limits_{n=1}^{\infty} a_n(x)$ сходится на $[a,b]$ поточечно;\\
 3. Ряд из производных сходится равномерно на $[a,b]$ к  $S(x)$;\\
 Тогда 
$$\sum\limits_{n=a}^{\infty} a'_n(x)=
\left( \sum\limits_{n=1}^{\infty} a_n \right)'$$
то есть в ряде  можно менять производную и сумму, причем 
$\sum\limits_{n=1}^{\infty} a_n$ сходится равномерно.
\end{theor}
\textbf{Доказательство.} 
1. Используем предыдущую теорему. Тогда
$$\int\limits_{x_0}^x\left( \sum\limits_{n=1}^{\infty} a'_n(t) \right)dt=
\sum\limits_{n=1}^{\infty} \int\limits_{x_0}^xa'_n(t)dt$$
Получаем, что в равенстве
$\int\limits_{x_0}^xS(t)dt=\sum\limits_{n=1}^{\infty} (a_n(x)-a_n
(x_0))$ справа стоит число (в силу непрерывности функции), ряд из $a_n(x_0)$
сходится по условию, следовательно, ряд из $a_n(x)$ сходится.
Поэтому, дифференцируя равенство 
$\int\limits_{x_0}^{x} \sum\limits_{n=1}^{\infty} a_n(t)\,dt=
\sum\limits_{n=1}^{\infty} a_n(x)-\sum\limits_{n=1}^{\infty} a_n(x_0)$,
получаем первое утверждение теоремы.

Теперь покажем равномерную сходимость исходного ряда. 
Для этого покажем, что остаток 
ряда из производных $r_n(x)=\sum\limits_{k=n+1}^{\infty} a'_n(x)$
равномерно стремится к нулю. 
Из этого следует применимость теоремы об инетгировании: 
$\int\limits_{x_0}^{x}\sum\limits_{k=n+1}^{\infty} a'_k(t)\,dt=
\sum\limits_{k=n+1}^{\infty} \int\limits_{x_0}^{x} a'_k(t)dt=
\sum\limits_{k=n+1}^{\infty} (a_k(x)-a_k(x_0))$. Если ряд удовлетворяет 
теореме об интегрировании, то и его остатки тоже, значит,
$\int\limits_{x_0}^{x} r_n(t)dt=R_n(x)-R_n(x_0)$, откуда
$$R_n(x)=\int\limits_{x_0}^{x} r_n(t)dt+R_n(x_0)\quad(1)$$.
Зафиксируем $\varepsilon>0$. По условию, остаток обычного ряда стремится
к нулю: $R_n(x)\to0$. Тогда для 
$$\frac{\varepsilon}{2}>0~\exists n_1
(\varepsilon)~\forall n>n_1:|R_n(x_0)|<\frac{\varepsilon}{2}$$
Остаток ряда из производных равномерно стремится к нулю, тогда
для 
$$\frac{\varepsilon}{2(b-a)}>0~\exists n_2(\varepsilon)~\forall n>n_2~
\forall x\in[a,b]:|r_n(x)|<\frac{\varepsilon}{2(b-a)}$$
По формуле (1) получаем: 
$|R_n(x)|\leqslant \left| \int\limits_{x_0}^{x} r_n(t)dt \right|+
|R_n(x_0)|\leqslant \left|\left| \int\limits_{x_0}^{x} r_n(t)dt \right|+
|R_n(x_0)| \right|<\frac{\varepsilon}{2(b-a)}\cdot |x-x_0|+
\frac{\varepsilon}{2}=\varepsilon$. $\square$ 



\section{Степенные ряды}
\subsection{Базовые определения}
\begin{defin}
Степенной ряд- ряд вида $\sum\limits_{n=0}^{\infty} C_n(x-x_0)^n$
\end{defin}
Числа $C_n$ - коэффициенты степенного ряда,  $x_0$ - число. Итак, степенной
ряд - обобщение понятия многочлена. Область сходимости степенного ряда 
непуста, так как так лежит как минимум  $x_0$ (в этом случае сумма ряда 
равна $C_0$). Сделав замену $t=x-x_0$, сведем любой степенной ряд к виду
 $\sum\limits_{n=0}^{\infty} C_nt^n$.
\begin{theor}
    (лемма Абеля)\\
    Если ряд $\sum\limits_{n=0}^{\infty} c_nx^n$ сходится в точке $x_0$ и 
     $|x|<|x_0|$, то ряд сходится сходится и в  $x$, причем абсолютно.
\end{theor}
\textbf{Доказательство.}  По условию ряд сходится, значит,
$c_nx^n\to0$. Тогда существует константа $M$, большая чем все члены ряда. 
Тогда $|c_nx^n|=\left| c_nx_0^n \left( \frac{x}{x_0} \right)^n  \right|
\leqslant M\cdot \left| \frac{x}{x_0} \right|^n $. Ряд $\sum\limits_{n=0}^{\infty} Mq^n$ сходится $\Rightarrow$ ряд из модулей сходится, т.е. ряд 
сходится абсолютно.
$\square$ 
\begin{theor}
Пусть $D$ - область сходимости ряда  $\sum\limits_{n=0}^{\infty} c_nx^n$,
$R=\sup\limits_{x\in D} |x|$. Тогда $(-R,R)\subset D\subset [-R,R]$.
\end{theor}
\textbf{Доказательство.} 
По лемме Абеля, второе включение очевидно: $\forall x\in D:|x|\leqslant R
\implies D\subset [-R,R]$.
Пусть $x\in(-R,R)$. Тогда  $|x|<R=R_1$. Тогда 
для него найдется  $x_0\in D:|x_0|>|x|$. Значит, ряд в точке  $x_0$ сходится,
и значит сходится в  $x$. Значит, интервал лежит в области сходимости.
$\square$ 
\subsection{Формулы для вычисления радиуса сходимости}
Пусть $\sum\limits_{n=0}^{\infty} c_nx^n=\sum\limits_{n=0}^{\infty} a_n$.
По признаку Даламбера 
$\lim\limits_{n \to \infty} \frac{|a_{n+1}(x)|}{|a_n(x)|}=|x|\cdot
\lim\limits_{n \to \infty} \frac{|c_{n+1}|}{|c_n|}<1$, то ряд сходится.
Итак, если предел существует, то 
$$\boxed{R=\lim\limits_{n \to \infty} \frac{|c_n|}{|c_{n+1}|}}$$
Аналогично, из признака Коши получим формулу Коши-Адамара:
$$\boxed{R=\frac{1}{\overline{\lim\limits_{n \to \infty}}\sqrt[n]{|c_n|}}}$$ 
В общем случае алгоритм такой:\\
1. Найти радиус сходимости.\\
2. Выписываем интервал сходимости $(x_0-R,x_0+R)$.\\
3. Исследуем на сходимость концы интервала.\\
\textbf{Пример.} Найдем область сходимости $\sum\limits_{n=0}^{\infty} 
\frac{(x-6)^n}{(n+2)3^n}$. Применим признак Даламбера:
$R=\lim\limits_{n \to \infty} \frac{(n+3)3^{n+1}}{(n+2)3^n}=3$.
Интервал сходимости: $(6-3,6+3)$. В точке $x=9$ ряд расходится (т.к.
гармонический), в точке  $x=3$ - условная сходимость (по признаку Лейбница).\\
\textbf{Пример.} Найдем область сходимости $\sum\limits_{n=0}^{\infty} 
\frac{n^2}{(n+1)^2}\cdot \frac{x^{2n}}{2^n}$. Заметим, что у этого ряда 
коэффициенты чередуются с нулем (лакунарный ряд). Используем два способа:\\
1. По формуле Коши-Адамара - возьмем четные номера, так как на них
доставляется супремум предела последовательности:
$R=\frac{1}{\lim\limits_{n \to \infty} \left( \frac{n}{n+1} \right)^
{\frac{1}{n}}\cdot \left( \frac{1}{2^{\frac{1}{2}}} \right) }=\sqrt{2}$.
Интервал сходимости $(-\sqrt{2},\sqrt{2})$, на концах расходится.\\
2. Исследуем как функциональный ряд по признаку Даламбера.
$\lim\limits_{n \to \infty} \frac{|a_{n+1}|}{|a_n|}=\frac{x^2}{2}
\lim\limits_{n \to \infty} \left( \frac{n^2+2n+1}{n^2+2n} \right)^2=
\frac{x^2}{2}$. Значит, ряд сходится, если $\frac{x^2}{2}<1$, откуда мы 
получаем тот же интервал сходимости.
\begin{theor}
    (о равномерной сходимости степенного ряда)\\
    Степенной ряд сходится равномерно на любом отрезке, лежащем внутри 
    интрвала сходимости.
\end{theor}
\textbf{Доказательство.} Для простоты рассмотрим ряд с центром в нуле. 
Пусть ряд сходится на $(-R,R)$. Возьмем  $[a,b]\subset (-R,R)$. Обозначим
$d=max(|a|,|b|)$. Тогда ряд  $\sum\limits_{n=0}^{\infty} c_nd^n$ сходится,
значит, его мы можем использовать для оценки сверху рядов на отрезке:
$|c_nx^n|\leqslant |c_nd^n|$, значит, по признаку Вейерштрасса ряд сходится
на $[a,b]$.
$\square$ 
\begin{theor}
    (о непреывной сумме степенного ряда)\\
    Сумма степенного ряда непрерывна в любой точке из интервала сходимости.
\end{theor}
\textbf{Доказательство.}  Пусть $\sum\limits_{n=0}^{\infty}c_nx^n$ 
сходится на $(-R,R)$ к  $f(x)$. Степенные функции непрерывны на интервале
(и вообще на всей прямой); по предыдущей теореме, на любом отрезке,
лежащем в интервале, ряд равномерно сходится. Значит, по теореме о 
непрерывности суммы равномерно сходящегося ряда, сумма непрерывна на 
отрезке. Так как этот отрезок произволен, то сумма непрерывна на интервале.
$\square$ 
\begin{theor}
    (об интегрировании и дифференцировании степенного ряда)\\
Пусть дан ряд $\sum\limits_{n=0}^{\infty} c_n(x-x_0)^n=f(x)$, $R$ - радиус 
сходимости. Тогда у функции  $f(x)$ существуют производные любого порядка
внутри интервала:
$$f'=\sum\limits_{n=0}^{\infty} nc_n(x-x_0)^{n-1}$$ 
Интегрирование тоже почленное. 
Причем при дифференцировании и интегрировании радиус сходимости не меняется.
\end{theor}
\textbf{Доказательство.}  Следует из соотвествующих теорем для функциональных
рядов. Последнее утверждение следует из формулы Коши-Адамара. 
$\square$\\ 
\textbf{Пример.} Вычислить сумму ряда $\sum\limits_{n=1}^{\infty} 
\frac{1}{n\cdot 2^n}$. Задания типа таких можно делать, используя 
свойства степенных рядов. Пусть $f(x)=\sum\limits_{n=1}^{\infty}\frac{x^n}{n}$.
Радиус сходимости $x\in[-1,1)$. Возьмем производную:
$f'(x)=\sum\limits_{n=1}^{\infty} x^{n-1}=\frac{1}{1-x}$. А вот теперь
проинтегрируем: $\int\limits^x_0\frac{dt}{1-t}=f(x)-f(0)$;
$f(x)=-\ln(1-x)+f(0)$. Значит, сумма искомого ряда равна $f(\frac{1}{2})=2$.
Цель этих телодвижений - привести к виду геометричсекой прогрессии, которую
легко посчитать. 









%Лекция 10.11.22
\section{Уравнение первого порядка}
\begin{defin}
Уравнение 
\begin{equation}\label{lin_de}
    \frac{dx}{dt}+a(t)x=b(t)
\end{equation}
где $a,b$ непрерывны на  $t\in (\alpha,\beta)$ (интервал непрерывности),
называется линейным ДУ первого порядка. Если при этом $b(t)\not\equiv 0$, то 
оно называется неоднородным.
\end{defin}
Как следствие из теоремы Коши-Пикара, для $\forall t_0\in (\alpha,\beta),~
\forall x_0\in \mathbb{R}$ существует и единственно решение задачи Коши.

\textbf{Замечание.} Решение задачи Коши для \ref{lin_de} можно продолжить
на весь интервал $(\alpha,\beta)$. Если этот интервал конечен, то функции
$a(t),b(t)$ ограниченны на нём, то есть  $|a(t)x+b(t)|\leqslant Ax+B$, 
и решениене выйдет за конус, образованный этой прямой. 

\begin{defin}
Линейныйй ператор - отображение $A\colon X\to Y$ %ЧЕЕЕЕ БЛЯЯЯЯЯЯЯЯЯТЬ ТАКОЕ
такое, что $A(x+y)=A(x)+A(y),~A(\lambda x)=\lambda A(x)$.
\end{defin}
Пусть $X=C^1(\alpha,\beta),~C^0(\alpha,\beta)$ - пространства дифференцируемых
и непрерывных функций. Положим $A(x)=\frac{dx}{dt}+a(t)x$. В силу линейности
производной, это - линейный оператор. Также и любая линейная комбинация 
производных (любого порядка) является линейным оператором. 

Итак, уравнение \ref{lin_de} в операторной записи эквивалентно 
$Ax=b(t)$. Обозначим за $x_{o.n.}$ множество решений неоднородного уравнения,
$x_{o.o.}$ - множество решений однородного уравнения,  $x_{o.n}+x_{o.o}$ -
множество вида $x+x$

\begin{theor}
    (о структуре решения линейного уравнения)\\
    Решение неоднородного уравнения - сумма общего решения однородного 
    уравнения и частного решения.
\end{theor}
\textbf{Доказательство.} Пусть $\varphi(t)$ - частное решение однородного
уравнения, $x_{p}$ - частное решение неоднородного уравнения. Применим
оператор  $A$ к их сумме:  $A(\varphi(t)+x_p)=A\varphi(t)+Ax_p=0+b(t)$. 
Значит, сумма этих функций обращает уравнение в тождество, значит,
$\varphi(t)+x_p\in x_{o.n.}$.

Докажем, что других решений нет. Допустим, $\psi(t)\in x_{o.n.}$ таков, что
его нельзя представить суммы решений однородного и неоднородного. Рассмотрим
$\psi-x_p$ - вычтем частное решение неоднородного. Подставляя в уравнение, 
получаем  $A(\psi-x_p)\equiv 0$, значит, их разность - решение однородного
уравнения. Но это противоречит предположению. $\square$

Как решать линейные уравнения? Сначале решаем однородное уравнение:
$\frac{dx}{dt}=-a(t)x$, $x=C(t)e^{-\int_{t_0}^{t} a(\tau)d\tau}$. 
Решать неоднородное 3мя способами:
1. Угадайка\\
2. Метод Лагранжа вариации постоянных\\
3. Формула Коши (см. справочник).
\subsection{Метод Лагранжа}
Мы знаем, что $x=Ce^{-\int a(t)dt}$ - решение однородного уравнения. 
Будем её варьировать, чтобы в уравнении было бы тождество:
$$\frac{d}{dt}\left( C(t)e^{-\int\limits_{t_0}^{t}a(\tau)d\tau} \right)+
a(t)C(t)e^{-\int\limits_{t_0}^{t}a(\tau)d\tau}=b(t)$$
Дифференцируя, получаем $C'=b(t)e^{-\int\limits_{t_0}^{t}a(\tau)d\tau}$, 
откуда  $$C=e^{-\int\limits_{t_0}^{t}a(\tau)d\tau}\int\limits_{t_0}^{t}\left( 
b(s)e^{-\int\limits_{s_0}^{s}a(\tau)d\tau} \right)ds+
C_0e^{-\int\limits_{t_0}^{t}a(\tau)d\tau}$$ 
Значит, мы нашли семейство всех решений неоднородного уравнения, произвольно
выбирая $C_0$. По предыдущей теореме, этим все решения исчерпываются. 

То, что мы получили - это и есть формула Коши. Она нужна в основном для
всяких теоретических свойств.

\textbf{Пример.} $\frac{dx}{dt}+\frac{x}{t}=t^2$.
Интервал непрерывности - $\mathbb{R}\setminus \{0\}$, поэтому вообще-то
надо рассматривать два интервала. Решение однородного уравнения:
$\frac{dx}{dt}=-\frac{x}{t}$, $x=\frac{C}{t}$. Подумаем, как можно подобрать
частное неоднородного уравнения. Поищем в виде $x=at^3$. Тогда при подстановке
$3at^2+t^2=t^2$, откуда $a=\frac{1}{4}$. Ответ: $x=\frac{t^3}{4}+\frac{C}{t}$.

\subsection{Уравнения, приводящееся к линейному}
Испортрим уравнение \ref{lin_de}, добавив нелинейности:
$$\frac{dx}{dt}+a(t)x=b(t)x^k,~k\in \mathbb{R}\setminus\{0,1\}$$ 
Это - уравнение Бернулли. Если разделим на $x^k$, получим
$$x^{-k} \frac{dx}{dt}+a(t)x^{1-k}=b(t)$$ 
Значит, оно сводится к линейному уравнению заменой $z=x^{1-k}$:
 $$\frac{1}{1-k} \frac{dz}{dx}+a(t)z=b(t)$$ 
Рассмотрим уравнение Риккати:
$$\frac{dx}{dt}+a(t)x=b(t)x^2+c(t),~c(t)\ne 0,c(t)\in C^0(\alpha,\beta)$$ 
В общем виде не решается, но можно частное решение угадать. 
Пусть $x=z+x_p$, где  $x_p$ - частное решение. Получим
$$\frac{dz}{dt}+a(t)z+\frac{dx_p}{dt}+a(t)x_p=b(t)x^2_t+2zx_pb(t)+c(t)$$
Свели к уравнению Бернулли
$$\frac{dz}{dt}+[a(t)-2x_pb(t)]z=b(t)z^2$$ 
Ну зато можно численно и приближенно решать. 

\textbf{Пример (№136).} $xy'-2y=2x^4,~x\ne0$. Разделим на  $x$, свели к
линейному (делить на $x$ можно, ибо  $x$ не является решением): 
$$\frac{dy}{dx}-\frac{y}{x}=2x^3$$
Общее решение неоднородного уравнения:
$$\int\limits_{}^{}\frac{dy}{2y}=\int\limits_{}^{}\frac{dx}{x}$$ 
откуда $y=Сx^2$. Подберем частное решение:  $y=ax^4$. Подставляя в уравнение,
получим  $a=1$, откуда общее решение  $y=x^4+Cx^2$. 

Теперь решим методом Лагранжа. Пусть $y=c(x)x^2$. Имеем
 $c'x^2+2xc-2cx=2x^3$, откуда $c(x)=x^2+C_0$. Значит, ответ  $y=x^4+C_0x^2$.

\textbf{Пример (№149).} $y'=\frac{y}{3x-y^2}$. Приведем к линейному 
(перевернем): $\frac{dx}{dy}=\frac{3x-y^2}{y}$. Общее решение 
однородного уравнения: $x=Cy^3$. Частное решение поищем в виде $x=ay^2$.
Отсюда $a=1$, общее решение  $x=Cy^3+y^2$.

\textbf{Пример (№158).} $2y'-\frac{x}{y}=\frac{xy}{x^2-1}$. Домножим на 
$y$:  $2y'y-x=\frac{xy^2}{x^2-1}$. Замена: $z=y^2$. Тогда уравнение
линеаризуется: 
$$\frac{dz}{dx}-\frac{xz}{x^2-1}=x$$
Общее решение однородного уравнения $z= C\sqrt{x^2+1}$. Метод 
внимательного взгляда: $z=x^2-1$ - частное решение. Итак, ответ:
$z=x^2-1+C\sqrt{x^2+1}$, $y=\sqrt{x^2-1+C\sqrt{x^2+1}}$.

\textbf{Пример (№164).} $(x^2-1)y'\sin y + 2x\cos y=2x-2x^3$. Наша 
нейросетка заметила, что здесь есть 
тригонометрическая замена. Именно, пусть $z=\cos x$. Тогда
$(x^2-1)(-z')+2xz=2x-2x^3$. Делим на 
$x^2-1$ получим однородное.

\textbf{Пример (№163).} $x(e^y-y')=2$. Введем замену $t=e^y$, получаем
 $1-\frac{dt}{dx}\cdot \frac{1}{t^2}=\frac{2}{xt}$. Далее $z=\frac{1}{t}$, 
 и наконец получаем линейное уравнение:
 $$1+\frac{dz}{dx}=\frac{2z}{x}$$

\textbf{Пример (№167).} Уравнение Риккати: $x^2y'+xy+x^2y^2=4$.
Частное решение $y=\frac{a}{x}$. Тогда
$-a+a+{a^2}=4$, $a=\pm2$. Пусть $y=\frac{2}{x}$. Общее решение тогда
$y=z+\frac{2}{x},~y'=z'-\frac{2}{x^2}$. Имеем уравнение Бернулли
$$-z^2=\frac{5z}{x}+z'$$
Сделаем замену $u=\frac{1}{z}$, получим $\int\limits_{}^{}\frac{du}{u} $


%дз 



\end{document}
