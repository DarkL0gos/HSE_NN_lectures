\begin{theor}
Непрерывное отображение компактного пространства в хаусдорфово замкнуто, 
то есть образ замкнутого замкнут. 
\end{theor}
\textbf{Доказательство.}  Пусть $f\colon(X,\tau)\to(Y,\omega)$ - непрерывное
отображение компактного топологического пространства в хаусдорфово. Пусть
 $F\subset X$ - замкнутое множество в $X$. Так как  $X$ компактно, то и 
  $F$ компактно, а значит, его непрерывный образ $f(F)\subset Y$ компактен
  (поскольку ограничение непрерывного отображения непрерывно).  
Так как $Y$ хаусдорфово, то компактное подмножество замкнуто, значит,  
отображение замкнуто. $\square$ 
\begin{theor}
Непрерывная биекция компактного пространства на хаусдорфово является
гомеоморфизмом.
\end{theor}
\textbf{Доказательство.} Пусть $f\colon(X,\tau)\to(Y,\omega)$ -
непрерывная биекция компактного пространства на хаусдорфово пространство.
Докажем, что обратное отображение непрерывно. По предыдущей теореме, 
образ любого замкнутого множества замкнут, но это значит, что прообраз 
\textit{обратного} отображения замкнут, значит, обратное отображение также 
непрерывно. $\square$ 
\subsection{Проекции и теорема Тихонова}
%Самая сложная теорема в общей топологии
Напомним определение произведения топологических пространств. 
\begin{defin}
Пусть $(X,\tau),(Y,\omega)$ - топологические пространства. Тогда 
$\Gamma=\{U\times V\mid U\in \tau,V\in \omega\}$ - база топологии на 
декартовом произведении $X\times Y$. Так как база задает топологию 
однозначно, то топология $\Omega$ с базой $\Gamma$ называется топологией 
произведения, пространство  $(X\times Y,\Omega)$ называется произведением
топологических пространств.
\end{defin}
Проверим, что это база, применив критерий базы на множестве.\\
1. Так как пространства открыты, то $X\times Y\subset X\times Y\in \Gamma$.\\
2. Пусть $U_1\times V_1\subset X\times Y,U_2\times V_2\subset X\times Y$ - 
два элемента базы, причем их пересечение непусто. Имеем
$(U_1\times V_1)\cap(U_2\times V_2)=(U_1\cap U_2)\times (V_1\cap V_2)$. 
Пересечение открытых открыто, поэтому любая точка лежит там с 
некоторой окрестностью. Произведение этих окрестностей доставляет 
искомый элемент базы. 

\textbf{Пример.} Покажем, что эта база -  не обязательно топология. База на
произведении двух прямых с обычной топологией не является топологией,
так как объединение прямоугольников не обязательно прямоугольник. С 
другой стороны, так как в каждом шаре лежит прямоугольник, а в каждом 
прямоугольнике - шар, то эта база задает обычную топологию на 
плоскости. Аналогично (или по индукции) доказывается, что топология
произведения $\mathbb{R}^k\times \mathbb{R}$ является обычной топологией на 
$\mathbb{R}^{k+1}$. 

\textbf{Упражнение.} Покажем, что произведение баз топологий также
является базой топологии произведения. 

\begin{defin}
Пусть $(X\times Y,\Omega)=(X,\tau)\times (Y,\omega)$ - произведение. 
Отображения
$$p_1\colon X\times Y\to X,(x,y)\mapsto x$$ 
$$p_1\colon X\times Y\to Y,(x,y)\mapsto y$$
называются каноническими проекциями.
\end{defin}
\begin{theor}
    (свойства канонических проекций)\\
    1. Канонические проекции непрерывны и открыты.\\
    2. Для любой точки $(x,y)$ слой $X\times \{y\}=p^{-1}_2(y)$ гомеоморфен
    $X$, слой  $Y\times \{x\}=p^{-1}_1(x)$ гомеоморфен $Y$.
\end{theor}
\textbf{Доказательство.} 1. Рассмотрим любое открытое множество $U\subset X$.
Его прообраз $p^{-1}_1(U)=U\times Y$ - произведение открытых множеств
(и вообще-то элемент базы), значит, открытое множество, поэтому отображение
непрерывно. Теперь рассмотрим любое открытое множество в $W\subset X\times Y$.
Пусть $W_1=p_1(W)$. По лемме об открытом множестве,  $\forall (x,y)\in W
\exists U\subset \Omega,(x,y)\in U\subset W$. Так как $U$ - множество из
базы, то оно есть произведение  $U=U_x \times U_y$. Значит, проекция точки
$(x,y)$ лежит в проекции  $W$ с некоторой окрестностью  $U_x$, поэтому
проекция  $W$ открыта,и отображение открыто.\\
2. Отображение $\{x\}\times Y\to Y$ - в обе стороны непрерывная биекция.
$\square$ 

\begin{theor}
    (Тихонова, частный случай)\\
    Произведение компактных пространств компактно.
\end{theor}
\textbf{Доказательство.} Пусть $(X\times Y,\Omega)=(X,\tau)\times (Y,\omega)$ 
- произведение компактных пространств. \\
Случай 1. Рассмотрим открытое покрытие $\xi=\{U_\alpha\times V_\alpha\mid
\alpha\in I\}$ подножествами из базы $\Gamma=\tau\times \omega$. Так как слой 
$p^{-1}_1(x)=\{x\}\times Y$ гомеоморфен $Y$, то он компактен, значит,
существует конечное подпокрытие $\xi^{(x)}=\{U_{\alpha_1}\times V_{\alpha_1},
...,U_{\alpha_m}\times V_{\alpha_m}\}$ слоя. Значит, 
$x\in \bigcap\limits_{i=1}^m U_(\alpha_i)$, и $\nu=\{U_x\mid x\in X\}$ - 
открытое покрытие, существует конченоеподпокрытие $\tilde\nu=
\{U_{x_1},...U_{x_k}\}$. В общем, у каждого слоя есть конечное покрытие. 
Взяв миниальную полоску, содержащуюся в этом покрытии, мы получим, что
такими полосками можно замостить все произведение. \\
Случай 2. Любое открытое покрытие 

$\zeta$

Итак, любое открытое покрытие имеет конечное подпокрытие имее
$\square$ \\

Методом индукции можно распространить на конечное произведение сомножителей.
\begin{defin}
Пусть $(X_\alpha,\tau_\alpha),\alpha\in J$ - любое семейство топологических
пространств. Тогда
$X=\prod\limits_{\alpha\in J}X_\alpha$ 
- произведение пространств, его элементы - последовательности
элементов из $X_\alpha$, индексированных элементами из $J$. 
Пусть $\tau$ - самая слабая топология на  $X$, для которой каждая
проекция непрерывна. существует,  так как это пересечение всех таких 
топологий. Тогда $(X,\tau)$ - тихоновское произведение с тихоновской
топологией. \end{defin}
\begin{theor}
    (Тихонова)\\
    Тихоновское произведение компактных пространств компактно. 
\end{theor}
\textbf{Доказательство.}  
$\square$ \\



