\begin{defin}
    Пусть Х - множество. Топологией на Х называется семейство подмножеств 
    $\tau\in\mathcal{P}(X)$, называемых открытыми множествами (данной 
    топологии), такое, что:\\
    1. $X,\varnothing\in\tau$\\
    2. $U_1,\ldots U_n\in\tau\Rightarrow\bigcap\limits^{n}_{i=1}U_i\in\tau$\\
    3. $\{U_i\mid i\in I\}\subset\tau\Rightarrow\bigcup\limits_{i\in I}U_i\in\tau$
\end{defin}
То есть, топологии принадлежит само множество и пустое множество, пересечение
конечного числа множеств и объединение любого числа множеств из топологии. 

Пример. Докажем, что открытые множества в смысле евклидовой метрики в 
$\mathbb{R}^n$ - топология. Очевидно, открыто само $\mathbb{R}^n$, также 
открытои пустое множество. Открытость пересечения доказывается тем, что
наименьшая эпсилон-окрестность принадлежит всем множествам,то есть лежит в их
пересечении, слеовательно, оно открыто. Для объединения: для каждой точки 
найдется множество, в которое она входит с окрестностью.

\begin{defin}
Тривиальная топология - $\tau_t=\{X,\varnothing\} $ \\
Дискретная топология - $\tau_0=\mathcal{P}(X)$ 
\end{defin}
Любая инетерсная топология содержит тривиальную и содержится в дискретной.

Пример. Множества, симметричные относительно выбранной прямой в евклидовом
пространстве,образуют топологию.

Пример. Множество эпсилон-окрестностей нуля $\tau=\{D_\varepsilon(0)\mid
\varepsilon>0\}\cup\{X,\varnothing\} $
- топология.

Пример. Топология Зарисского - топология множеств, дополнительных к конечным 
множествам (для конечных пространств совпадает с дискретной).

Пример. Пусть $f:X\to X$ - биекция. Докажем, что $\tau_f=\{U\subset X\mid





