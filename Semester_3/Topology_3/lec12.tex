\subsection{Регулярные и нормальные топологические пространства}
\begin{defin}
Топологическое пространство называется регулярным, если оно удовлетворяет
аксиомам Т1 и Т3.
\end{defin}
\begin{defin}
Топологическое пространство называется нормальным, если оно удовлетворяет
аксиомам Т1 и Т4.
\end{defin}
\begin{theor}
    (о связи между аксиомами отделимости)\\
    Нормальное пространство регулярно, регулярное пространство
    хаусдорфово, и так далее:
    $$(T_1\And T_4)\implies(T_1\And T_3)\implies T_2\implies T_1\implies T_0$$
\end{theor}
\textbf{Доказательство.} Более менее очевидно. $\square$\\
Напомним, что метрическое пространство хаусдорфово. Имеет место более сильное
утверждение:
\begin{theor}
Любое метрическое пространство нормально.
\end{theor}
По предыдущей теореме из этого следует, что любое метрическое пространство 
удовлетворяет всем аксиомам отделимости. \\
\textbf{Доказательство.}  Самим доказать. $\square$ 

\textbf{Пример.} $(\mathbb{R}^2,\tau_\text{об})$ - метрическая топология, а 
значит, нормальная.

\textbf{Пример.} $(\mathbb{R}^2,\tau_{MN})$ - не выполняется $T_0$, 
поэтому не выполняются $T_0,T_1,T_2$. Так как все открытые одномвременно
замкнуты и наоборот, поэтому окрестность любого замкнутого множества - оно
само, значит, множество обладает непересекающейся окрестность с точкой. 
Аналогично для другого замкнутого множества. Значит, множество
удовлетворяет $T_3,T_4$. 

\section{Компактность}
\begin{defin}
Топологическое пространство называется компактным, если из любого его 
открытого покрытия можно выделить конечное подпокрытие.
\end{defin}
Заметим, что это определение эквивалентно такому же, только с покрытием из
базы.

\textbf{Пример.} $([a,b],\tau_{об})$ - компактно по лемме Гейне-Бореля 
(из любого покрытия отрезка открытыми интервалами можно выделить конечное
подпокрытие). 

\textbf{Пример.} Докажем, что $(\mathbb{R},\tau_\text{об})$ не компактно.
Допустим, что существует конечное подпокрытие $\tilde \xi$. Тогда это
объединение конечного числа интервалов конечной длины. 

\begin{theor}
    Непрерывный образ компактного пространства компактен.
\end{theor}
\textbf{Доказательство.}  Пусть $f\colon(X,\tau)\to(Y,\omega)$ - непрерывное 
отображение. Рассмотрим любое открытое покрытие образа $\eta=\{V_\alpha\mid
\alpha\in I\}$, $Y=\bigcup\limits_{\alpha\in I} V_\alpha$. Прообразы
множеств этого покрытия открыты  $X$ и покрывают его. Выделим в нем конечное
подпокрытие. Его образы также покрывают  $Y$, ибо  $f(X)=Y$, значит, мы 
выделили конечное подпокрытие в  $Y$, и соответсвенно  $Y$ компактно.
$\square$ \\
\textbf{Следствие.} Компактность - топологический инвариант (в частности, 
сфера без точки некомпактна, так как гомеоморфна плоскости, что показывает
стереографическая проекция). 
\begin{theor}
Любое замкнутое подмножество компактного пространства компактно (то есть
компактно в индуцированной топологии).
\end{theor}
\textbf{Доказательство.} Пусть $H$ - замкнутое подмножество  $X$. 
Пусть  $H\subset \bigcup\limits_{\alpha\in I} U_\alpha$ - открытое покрытие
$H$. Так как  $X$ компактно, то существует конечное подпокрытие в 
покрытии  $\eta=\{X\setminus H,U_\alpha\mid\alpha\in I\}$. Так как в нем 
есть конечное подпокрытие, то значит есть и конечное подпокрытие, 
покрывающее $H$, значит, $H$ компактно. $\square$ 
\begin{theor}
Любое компактное подмножество хаусдорфова пространства замкнуто.
\end{theor}
\textbf{Доказательство.} Пусть $F$ - любое компактное подмножество в
хаусдорфовом пространстве $X$. Докажем, что его дополнение открыто. Пусть
 $x\in CF$ и $z\in F$. Тогда $z\ne x$, а в силу хаусдорфовости у них 
 существуют непересекающиеся окрестности  $U_z$ и  $V_x$. Так как точка
$z$ произвольна, то  $\xi=\{U_z\mid z\in F\}$ - открытое покрытие $F$, из
которого можно выделить конечное подпокрытие  $\tilde\xi$. Каждой окрестности
из $\tilde\xi$ соответсвует какая-то окрестность $V^i_x$ какой-то точки
$x$. Определим  $W_x=\bigcap\limits_{i \in  I} V^i_x$. Покажем, что 
$U_f\cap W_x=\varnothing$. От противного: если есть пересечение, то 
противоречие с тем, что 
Значит, $W_x\subset CF$. Отсюда по лемме об открытом множестве имеет
открытость $CF$, то есть  $F$ замкнуто.
$\square$ \\
\textbf{Пример.} Все бесконечные подножества в топологии Зарисского
компактны. 


