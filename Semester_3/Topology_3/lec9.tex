\subsection{Пути}
\begin{defin}
    Путь - непрерывное отображение 
    $h\colon([0,1],\tau_\text{об})\to (X,\tau)$,
    $f(0)$ - начало пути, $f(1)$ - конец пути.
\end{defin}
Пусть $f,g$ - пути в пространстве  $X$, причем
$h(1)=g(0)$. Определим произведение путей следующим образом:
$$(h\cdot g)(t)=\begin{cases}
    h(2t),~t\in [0,\frac{1}{2}]\\g(2t-1),~t\in [\frac{1}{2},1]
\end{cases}$$
Докажем, что мы снова получили путь. При $t=\frac{1}{2}$, имеем 
$h(t)=h(1),~g(t)=g(0)$. Но по условию  $h(1)=g(0)$, значит, отображение
опредлено корректно. Докажем непрерывность: сужения  
$h\cdot g|_{[0,\frac{1}{2}]}=h(2t)$ и $h\cdot g|_{[\frac{1}{2}],1}=g(2t-1)$
непрерывны как композиция непрерывных отображений 
(например, $h$ и умножения на 2). Теперь заметим, что 
$\xi=\{[0,\frac{1}{2}],[\frac{1}{2},1]\}$ - конечное замкнутое покрытие 
отрезка $[0,1]$. Значит, оно фундаментально, поэтому из непрерывности
сужений следует непрерывность отображения. 
\begin{defin}
    Петля $f\colon[0,1]\to X$ - путь, для которого $f(0)=f(1)=x_0$. 
\end{defin}
\begin{defin}
Обратный путь для $f$:  $f^{-1}(t):=f(t-1)$
\end{defin}
Другое определение петли: такой путь $h$, что  %$h^{-1}(t)=h(t-1)$
\begin{defin}
    Постоянный путь - $h:[0,1]\to x_0$ (обозначение: $e_{x_0}$)
\end{defin}
Теперь введем понятие линейной связности. 
\begin{defin}
Пусть $(X,\tau)$ - топологическое пространство. Говорят, что точки
 $x_0,x_1$ можно соединить путем, если существует путь h  в Х такой, что
 $h(0)=x_0,h(1)=x_1$
\end{defin}
\begin{defin}
Топологическое пространство называется линейно связным, если любые его две
точки можно соединить путем. 
\end{defin}
\subsubsection{Свойства линейно связных пространств}
\begin{enumerate}
    \item Непрерывный образ линейно связного пространства линейно связен
    \item Пусть $X_\alpha$  - линейно связное подмножество $X$. 
Тогда если $\bigcap\limits_{\alpha\in I}X_\alpha\ne\varnothing$, то их 
объединение линейно связно.
\item Любое линейно связное пространство связно (обратное неверно).

\end{enumerate}
\textbf{Доказательство.}\\
1. Пусть $f\colon X\to Y$ - непрерывное отображение топологических 
пространств, причем  $(X,\tau)$ линейно связное. Покажем, что  $(Y,\omega)$
линейно связно. Возьмем  $y_0,y_1\in Y$. Так как $Y=Im(f)$, то найдутся
такие  $x_0,x_1:f(x_0)=y_0,f(x_1)=y_1$. Из линейной связности $X$ имеем
путь  $h:h(0)=x_0,h(1)=x_1$. Значит, в  $Y$ имеется путь  $f\circ h\colon
[0,1]\to Y,~(f\circ h)(0)=y_0,~(f\circ h)(1)=y_1$. Непрерывность следует из 
непрерывности композиции. Как следствие, получаем, что линейная связность 
сохраянятеся при гомеоморфизме, то есть это - топологический инвариант.\\
2. Так как $\bigcap\limits_{\alpha\in I}X_\alpha\ne\varnothing$, то
$x\in\bigcap\limits_{\alpha\in I}X_\alpha\implies x\in X_\alpha~\forall\alpha$.
Рассмотрим произвольные $y,z$, лежащие в пересечении. Найдутся такие 
$\alpha',\alpha''$, индексирующие множества, содержащие $y,z$. 
Введем два пути: $h$ от  $y$ до  $x$ и  $g$ от  $x$ до  $z$.
Их произведение - искомый путь, доказывающий линейную связность: 
$h\cdot g\colon[0,1]\to\bigcap\limits_{\alpha\in I}X_\alpha,~
(h\cdot g)(0)=y,~(h\cdot g)(1)=z$.\\
3. Пусть $(X,\tau)$ линейно связно. Тогда для любых точек  $P_1,P_2$ 
найдется путь  $h:h(0)=P_1,~h(1)=P_2$. Так как отрезок связен, то его
непрерывный образ $h([0,1])=M(P_1,P_2)$ - связное подмножество. По теореме
19 (про две точки), получаем, что пространство связно.\\
Приведем контрпример к обратному утвеждению. Рассомотрим 
$y=\sin(\frac{1}{x}),~x>0$. 
Раасмотрим график синуса 
$A=\{(x,\sin(\frac{1}{x}))\mid x>0\}\subset \mathbb{R}^2$ 
с топологией, индуцированной из обычной топологии плоскости.
Множество $A$ связно, поскольку прообраз линейно связен, откуда 
$\overline{A}=A\cup B$, где $B=\{(0,y)\mid -1\leqslant y\leqslant 1\}$ - 
замыкание связно как замыкание связного множества. Однако $\overline{A}$ 
не является линейно связным, так как точку 0 нельзя соединить 
путем с произвольной точкой на графике. 
\begin{defin}
Компонента линейной связности пространства $(X,\tau)$ для точки  $x$ - 
максимальное линйено связное подпространство  $L_x$, содержащее 
 $x$. 
\end{defin}
\textbf{Утверждение.} Компонента линейной связности содержит все точки, 
которые можно соединить путем с $x$. 
\begin{theor}
    (свойства компонент связности)\\
    1. Компоненты линейной связности либо не пересекаются, либо совпадают;\\
    2. Компоненты линейной связности образуют разбиение $X$;\\
    3. $X$ линейное связно  $\Leftrightarrow$ $\forall x:X=L_x$
\end{theor}
\textbf{Доказательство.}  Аналогично компонентам связности. Отличие только 
в том, что компоненты связности всегда замкнуты. $\square$ \\
\textbf{Замечание.} Каждая компонента связности разбивается на компоненты 
линейной связности. Приведем пример: пусть $X=\overline{A}=A\cap B$ (из
контпримера прошлой теоремы) - связно как замыкание связного $A$,
т.е.  $X=K_x~\forall x\in X$. $X=L_0\cap L_a$, где  $L_a$ - график синуса,
$L_0=B$.
\begin{defin}
Топологическое пространство $(X,\tau)$ называется локально линейно связным,
если для любой точки $x\in X$ существует линейно связная окрестность.
\end{defin}
 \begin{theor}
     (о линейной связности в локально линейно связных пространствах)\\
     Если $(X,\tau)$ локально линейно связно, то  $(X,\tau)$ связно тогда и 
     только тогда, когда $X$ линейно связно.
 \end{theor}
 \textbf{Доказательство.} По теореме 9.3, из линейной связности следует
 связность. Докажем обратное. Пусть $L$ - компонента линейной связности.
 Покажем, что $L$ - открыто. Действительно, по условию существует
 линейно связная окрестность $U_x\subset L$ для точек $x\in L$, откуда по 
 лемме об открытом множестве $L$ открыто. Поскольку  
 $X=\bigcup\limits_{\alpha\in I}L_\alpha$, то рассмотрим $L_{\alpha_0}=
 X\setminus \bigcup\limits_{\alpha\in I\setminus\{\alpha_0\} }L_\alpha$. 
 Получаем, что $L_{\alpha_0}\ne\varnothing$
 замкнуто как дополнение до открытого. Так как это открыто-замкнутое 
 множество, а $X$ связно по условию, то $L_{\alpha_0}=X$. $\square$ \\

















