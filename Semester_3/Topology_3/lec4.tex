\section{Свойства замкнутых множеств}
\begin{theor}
    Пусть $(X,\tau)$ - топологическое прространство, и  $\mathcal{F}=\{
    CU\mid U\in\tau\}$ - совокупность всех замкнутых множеств. Тогда 
    выполняются условия:\\
    F1. $\varnothing,X\in\mathcal F$\\
    F2. Объединение любых двух замкнутых замкнуто.\\
    F3. Пересечение любого семейства замкнутых замкнуто.\\
\end{theor}
\textbf{Доказательство.} Применим законы де Моргана к аксиомам топологического
пространства.\\
1. $X=C\varnothing,~\varnothing=CX$\\
2. Дополнение к объединению открытых замкнуто, и есть пересечение дополнений.\\
3. Дополнение к пересечению двух открытых замкнуто, и есть объединение 
дополнений. 
$\square$ \\
\textbf{Замечание.} Как мы видим, замкнутые множества имеют свойства, очень
похожие на свойства топологии. На самом деле, топологию можно однозначно 
задать как семейство множеств, удовлетворяющих свойствам замкнутых множеств,
и объявить открытыми дополнения к ним.\\
\textbf{Замечание.} Из аксиомы $\tau_2$ по индукции
вытекает, что пересечение любого конечного числа открытых множеств открыто,
и объединение любого числа замкнутых множеств замкнуто. \\
\textbf{Пример.} Рассмотрим обычную топологию на прямой, и рассмотрим 
интервалы, верхняя граница котрых минорируется каким-то числом. Тогда
в бесконечном пересечении имеем отрезкоинтервал. Пример показывает, что 
пересечение любого числа открытых уже не обязательно открыто.
\begin{theor}
    (лемма об открытом множестве)\\
    Пусть $(X,\tau)$ - топологическое пространство.
    Множество открыто в топологии тогда и только тогда, когда любая точка
    содержится в нем с некоторой окрестностью. 
\end{theor}
\textbf{Доказательство.}  Возьмем любую точку $x\in U$. Положим окрестность
точки само множество  $U$; очевидно,  $U\subset U$.\\
Обратно, пусть каждая точка входит в $U$ вместе с какой-то окрестностью. 
Их объединение лежит в $U$, и ещё и  $U$ лежит в нем, так как окрестность
каждой точки содержит её. $\square$ 
\subsection{Классификация точек относительно подмножества}
Пусть $(X,\tau)$ - топологическое пространство, $A\subset X$ - непустое 
подмножество. Серия определений:
\begin{defin}
Точка $x\in X$ называется внутренней точкой множества  $A$, если существует
окрестность этой точки, лежащая в $A$.
\end{defin}
\begin{defin}
Точка внешняя, если она внутренняя для дополнения.
\end{defin}
\begin{defin}
Точка $x\in X$ называется точкой прикосновения, если для любой окрестности
 $U\cap A\ne\varnothing$
\end{defin}
\begin{defin}
Точка накопления, если в любой окрестности точки х найдется точка из A, не
совпадающая с х.
\end{defin}
\begin{defin}
Точка граничная, если в любой окрестности точки х лежат как точки из А,
так и из дополнения к А. 
\end{defin}
\begin{defin}
 Возьмем любое подмножество А топологического пространства Х. Объединение
 всех внутренних точек А называется внутренностью А (обозначается $A_0,~
 Int A$). Объединение всех точек прикосновения называтся замыканием А (
 обозначение: $\overline{A}$). Множество всех граничных точек - граница А. 
 (обозначение: $FrA,~\partial A$)
\end{defin}
Переходим к теоремам. 
\begin{theor}
 (свойства замыкания)\\
 Замыкание множества обладает следующими свойствами:\\
 1. $A\subset \overline{A}$, причем замыкание замкнуто.\\
 2. Если $A\subset B$, то $\overline{A}\subset \overline{B}$.\\
 3. Замыкание множество - минимальное по включению замкнутое множество, 
 содержащее А.\\
 4. $\overline{A}=\bigcap F_\sigma$ - замыкание есть пересечение всех 
 замкнутых множеств, содержащих А.\\
 5. Множество замкнуто тогда и только тогда, когда оно совпадает со своим
 замыканием.
\end{theor}
\textbf{Доказательство.} 1. Рассмотрим $x\in A$. Найдем окрестность
$x\in U_x$. 
Они пересекаются, и значит, $A\subset \overline{A}$. Докажем замкнутость
замыкания. По лемме об открытом множестве, точка из дополнения к замыканию
имеет окрестность, не пересекающуюся с А. Рассмотрев точку из этой окрестности,
заметим, что она тоже не лежит в замыкании. Итак, мы показали, что 
дополнение к замыканию открыто, так как каждая точка лежитв нем
с некоторой окрестностью.\\
2. Пусть $A\subset B$. Возьмем любую точку из замыкания. Она еть точка 
накопления для А и следоваельно для В (по включению), значит, она лежит в 
замыкании В.\\
3. Предположим, что существует замкнутое $F:A\subset F\subset \overline{A}$. 
Это эквивалентно тому, что существует точка из замыкания, но не лежащая в F. 
Она лежит в дополнении F, но лежит и в замыкании А, значит, является точкой 
накопления, но тогда $CF\cap F\ne\varnothing$ - противоречие.\\
4. Рассмотрим пересечение всех замкнутых, содержащих множество А. Оно 
замкнуто по свойтсву замыкания. Также, по свойству замыкания, $\overline{A}$ - 
одно их них, так как замкнуто. Но также и все пересечение лежит в замыкании. 
Обратно, А входит в пересечение. То есть в нем лежит и замыкание. Имеем 
в итоге равенство.\\
5. Пусть множество совпадает с замыканием. Тогда оно замкнуто по первому 
пункту теоремы. Обратно, пусть А замкнуто. По свойству 3, замыкание 
- минимальное замкнутое по включению. Но это и есть А.
$\square$ 
\begin{theor}
    (свойства внутренности)\\
    Пусть А - подножество топологического пространства.\\
    1. $A_0\subset A$, причем внутренность открыта.\\
    2. $A_0$ - максимально открытое по включению, лежащее в А.\\
    3. Внутренность есть объединение всех открытых множеств, лежащих в А.\\
    4.  $A=A_0$  $\Leftrightarrow$ A открыто.
\end{theor}
\textbf{Доказательство.}  
1. $A_0$ лежит\\
2. Рассмотрим открытое множество $U\subset A$\\
3. Рассмотрим объединение множеств $A_\alpha$. Это - открытое множество, 
которое лежит в $A$  $\Rightarrow$ лежит в $A^0$. Есть и обратное включение:
рассмтрим  $x\in A^0\subset A$. Значит, сущетсвует $A_{\alpha_0}=A^0$
$\Rightarrow$ $A^0\subset \bigcup A_\alpha$. Итак, доказано равенство.\\
4. Пусть $A^0=A$ - открыто по свойству 1. Обратно, если  $A$ открыто, то
 $A=A^0$ по свойству 2. 
$\square$ 
\begin{theor}
    (свойства границы)\\
Пусть $Fr\, A$ - граница подмножества $A$ топологического пространства 
$(X,\tau)$.\\
1. $Fr\,A=\overline{A}\cap\overline{CA}$ - замкнутое множество.\\
2. $Fr\,A=\overline{A}\setminus A^0$
\end{theor}
\textbf{Доказательство.} 
1. В любой окрестности любой точки границы содержатся как точки из $A$, так
и из  $CA$. Значит, граничные точки являются точками прикосновения, то 
есть принадлежат замыканию  $A$. С другой стороны, они приндлежат
замыканию дополнения множества  $A$ по тем же соображениям.\\
2. Рассмотрим $x\in Fr\,A$. Это точка, которая принадлежит как замыканию
множества, так и замыканию его дополнения. Значит, это не внутренняя точка. 
То есть $x\in Fr\,A\iff x\in \overline{A}\setminus A^0$
$\square$ \\
\subsubsection{Примеры weird и fancy топологий}
\textbf{Пример.} Нарисем бабочку на плоскости, у которой кусок границы открыт.
Значит, имеем $\mathbb{R}^2,\tau_{\text{об}}$. В ней оно ни открыто, ни
замкнуто. В топологии отражения относительно $OY$ (вспоним, что в неё все
открытые множества открыто-замкнутые!). Внутренность - брюшко бабочки,
замыкание - вся бабочка, граница - их разность. 
Топология Зарисского. Замыкание -  $$



