\documentclass[a4paper]{article}

%Общие настройки документа
\usepackage[14pt]{extsizes}                                         %Размер шрифта
\usepackage[left=2.5cm,right=2.5cm,top=2.5cm,bottom=3cm]{geometry}  %Поля страницы

%Настройки ссылок и гиперссылок
\usepackage{float}
%\usepackage{graphicx}
%\usepackage{hyperref}                 
%\usepackage{xcolor}
%\definecolor{linkcolor}{HTML}{799B03} % цвет ссылок
%\definecolor{urlcolor}{HTML}{799B03}  % цвет гиперссылок
%\hypersetup{pdfstartview=FitH,linkcolor=linkcolor,urlcolor=urlcolor,colorlinks=true}
%graphicspath{{\figures}}


%Пакеты символов
\usepackage{cmap}
\usepackage[T2A]{fontenc}
\usepackage[utf8]{inputenc}
\usepackage[russian]{babel}           
\usepackage{amsmath}
\usepackage{amssymb}
\usepackage{amsfonts}

%Новые команды 
\newtheorem{defin}{Определение}
\newtheorem{example}{Пример}
\newtheorem{zam}{Замечание}
\newtheorem{theor}{Теорема}

\author{Мудрое Загадочное Дерево}
\title{Пруф общей тоеремы Тихонова}
\date{канун экза}

\begin{document}
\maketitle
\begin{defin}\label{tikhon}
Пусть $(X_\alpha,\tau_\alpha),\alpha\in J$ - любое семейство топологических
пространств. Рассмотрим
$X=\prod\limits_{\alpha\in J}X_\alpha$ 
- декартово произведение пространств, его элементы - последовательности
элементов из $X_\alpha$, индексированных элементами из $J$ (то есть
$\{x_1,x_2,...,x_k,...\},~x_k \in  X_k,~k \in  J$). 
Пусть $\tau$ - самая слабая топология на  $X$, для которой каждая
проекция $p_k\colon X\to X_k,\{x_1,x_2,...,x_k,...\}\mapsto \{x_k\}$ 
непрерывна. Такая слабейшая топология существует, 
так как это пересечение всех топологий, для которых проекции непрерывны. 
Тогда $(X,\tau)$ - тихоновское произведение с тихоновской
топологией. 
\end{defin}
\begin{theor}
    (Тихонова)\\
    Тихоновское произведение компактных пространств компактно. 
\end{theor}
%\tableofcontents[План доказательства]
1. Начальные сведения: кольца и идеалы \\
2. Теорема Александера о предбазе \\
3. Определение тихоновской топологии через предбазу \\
3. Сведение теоремы Тихонова к теореме Александера о предбазе 

\textbf{Доказательство.} 
\begin{defin}
Пусть $R$ - коммутатичное кольцо с единицей. Идеалом $I$ в кольце $R$
называется аддитивная подгруппа, замкнутая относительно умножения на 
на элементы кольца.
\end{defin}
Очевидно, $\sum\limits_{\lambda in K} \lambda r$ - идеал, порожденный, 
множеством $S\subset R$. 

Идеал, порожденный единицей, совпадает с кольцом. 

Множество всех подмножеств данного множества является кольцом
относительно симметрической разности (сложения) и пересечения. 

\begin{theor}
    (лемма Цорна для идеалов)\\
    Пусть $I$ - идеал в кольце. Тогда существует максимальный идеал
     $I\subset I_1\subsetneq R$
\end{theor}
\textbf{Доказательство.}  
$\square$ \\



3. Рассмотрим следующую предбазу на $X=\prod\limits_{\alpha\in J}X_\alpha$:
для любого индекса $i \in  J$ и любого открытого множества $U \in X_i$
рассмотрим произведение
$$X_1\times X_2\times...X_{i-1}\times U \times X_{i+1}\times ... $$
(вместо $X_i$ стоит  $U\subset X_i$). Покажем, что множество всех
таких произведений является предбазой тихоновской топологии \ref{tikhon}.

Пусть 
$\Gamma = \{X_1\times X_2\times...X_{i-1}\times U \times X_{i+1}\times
...\mid i \in J, U\subset \tau_i\}$ - предполагаемая предбаза, 
$\Sigma$ - множество пересечений всех элементов из  $\Gamma$. 
Применим критерий базы топологического пространства:\\
1. Проекции, определенные на элементах предбазы, непрерывны: 
действительно, все проекции, кроме $i$-той, являются гомеоморфизмами слоев;
$i$-тая проекция также непрерывна, поскольку  $U$ - открытое подпространство 
в  $X_i$ и все открытые в нем множества - следы открытых. Ограничение 
непрерывной функции непрерывно, поэтому проекция, определенная на пересечениях
элементов из  $\Gamma$, непрерывна. Значит, элементы предбазы открыты в смысле
топологии Тихонова. 





$\square$ 
\end{document}
