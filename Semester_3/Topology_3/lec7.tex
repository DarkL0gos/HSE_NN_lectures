\begin{defin}
Пусть А - подмножество топологического пространства $(X,\tau)$. Тогда функция 
$i\colon A\to X,~i(a)=a$ называется включением А в Х.
\end{defin}
\begin{theor}
Включение топологического подпространства в объемлющее пространтсва
непрерывно.
\end{theor}
\textbf{Доказательство.}  Любой прообраз открыт $\Rightarrow$ непрерывно.
$\square$ 
\section{Гомеоморфизмы}
\begin{defin}
Отображение топологических пространств называется гомеоморфизмом, если\\
1. f - биекция\\
2. f - непрерывное отображение\\
3. $f^{-1}$ - непреывное отображение
\end{defin}
Отображение называется открытым, если образ открытого множества открыт. 
Условие 3 эквивалентно открытости обратного отображения. 
\begin{defin}
Топологические пространства гомеоморфны (топологически эквивалентны), если
между ними можно провести гомеоморфизм.
\end{defin}
\textbf{Упражнение.} Гомеоморфность - отношение эквивалентности на множестве
топологических пространств.
\subsection{Топологические классификации}
\begin{defin}
Свойства топологических пространств называются топологическим инвариантом,
если оно сохраняется при любых гомеоморфизмах.
\end{defin}
\textbf{Пример.} Хаусдорфовость - топологический инвариант. Докажем это.
Предположим, что $(X,\tau)$ и  $(Y,\omega)$ - гомеоморфные пространства и $X$ 
хаусдорфово.  Возьмем $y\ne z\in Y$. Рассмотрим их прообразы. Они различны,
поскольку у нас биекция, значит, они обладают непересекающимися 
окрестностями. Образы этих окрестностей открыты и являются окрестностями 
$y,z$. Но их пересечение пусто, так как биекция.\\
\textbf{Пример.} $(\mathbb{R},\tau_E)$ (обычная топология) не гомеоморфна
$\mathbb{R},\tau_{MN}$, так как у них разная хаусдорфовость.\\
\textbf{Задача.} Провести топологическую классификацию топологических 
пространств $\xi=\{(X_\alpha,\tau_\alpha \mid \alpha\in J\}$.  
Сделать эту классификацию значит найти систему топологических инвариантов,
характеризующую классы эквивалентности по гомеоморфизму. Это означает,
что если два пространства обладают инвариантами, то они гомеоморфны. Такая 
система инвариантов называется полной, то есть полностью характеризует класс.
\subsection{Связность топологических пространств}
\begin{defin}
Топологическое пространство называется связным, если в нем одновременно 
открыты и замкнуты только X и $\varnothing$.
\end{defin}
\begin{defin}
Топологическое пространство называется связным, если его нельзя представить
в виде двух непустых непересекающихся открытых (или замкнутых) подмножеств.
То есть $X\ne A_1\cup A_2,~A_1\cap A_2=\varnothing,$
\end{defin}
\begin{theor}
 Определения связности эквивалентны
\end{theor}
\textbf{Доказательство.} 1 $\Rightarrow$2. Предположим противное, именно, 
Х раскладывается в дизъюнктное объединение. Но тогда обе части 
открыто-замкнуты. \\
2 $\Rightarrow$1. Пусть выполнеятся 2, но существует такое замкнутое А, что 
$X=A_1\cap CA_1$. получили прямое противоречение с 2.
$\square$ 
\begin{theor}
    (о связности топологических пространств)\\
    Непрерывный образ связного пространства связно.
\end{theor}
\textbf{Доказательство.} Пусть $f\colon X\to Y$ -  непрерывная сюръекция. 
Допустим, что $Y$ не связен, а  $X$ связен. Тогда У можно
представить в виде объединения открытых множеств $Y=A_1\cup A_2$.  
Тогда Х можно представить в виде объединения двух полных прообразов этих
множеств,
но они непустые, непересекающиеся и открытые по свойству непрерывного 
определения. Противоречие со связностью Х.
$\square$ \\
\begin{theor}
    (основная лемма)\\
    Пусть А и В - открытые или замкнутые непересекающиеся подмножества в
    топологическом пространстве $X$. Если для связного М выполнено
    $M\subset A\cup B$, то оно лежит целиком либо в $A$, либо в $B$.
\end{theor}
\textbf{Доказательство.}  Предположим противное, и М лежит и в А и В.
Тогда $M=M\cap(A\cup B)=(M\cap A)\cup(M\cap B)=A_1\cup A_2$. ни одно из 
этих новых множеств, не равно М, иначе они пустые. Значит, предположение 
неверно. М не связно как подпространство, индуцированное на самаом себе
$\square$ 
\begin{theor}
Если любые две различные точки топологического пространства принадлежат
связному подмножеству. то само пространство связно.
\end{theor}
\textbf{Доказательство.} Предположимя, что пространство несвязно.  \
$\square$ 
\begin{theor}
Линейная связность эквивалентна связности.
\end{theor}
\textbf{Доказательство.}  \
$\square$ 
\begin{theor}
Если пересечие связных топологических подпространств непусто, тогда их 
бъединение связно.
\end{theor}
\textbf{Доказательство.} Предположим. что объединение несвязно. 
$\square$ 






