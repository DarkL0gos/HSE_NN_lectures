% Продолжаем топологию 15/09/2022
\section{База топологии}
\begin{defin}
    Пусть $(X,\tau)$ - топологическое пространство
    Семейство  $\Sigma=\{W_\beta\subset X\mid \beta\in B \} $ - база топологии,
    если удовлетворяет двум условиям:\\
    1. $\Sigma\in\tau~\forall W_\beta\in\Sigma$
    2. Любое открытое подмножество Х можно представить в виде
    объединения некоторых подмножеств из $\Sigma$: 
    $\forall U\in\tau\exists W_\alpha\in\Sigma,~\alpha\in A\subset B:
    U=\bigcup\limits_{\alpha\in A}W_\alpha $
\end{defin}
\textbf{Пример}. В обычной (евклидовой) топологии множество 
$\Sigma=\{D_r(a)\mid a\in\mathbb{R}^n,r>0\}$ является базой топологии.
Действительно, проверим аксиомы:\\
1. Открытая окрестность открыта.\\
2. По определению обычной топологии,каждая точка в открытом множестве
содержится в нем с некоторой окрестностью. Значит, объединение этих
окрестностей дает это множество. Более формально,
$\forall u\in\tau,\forall x\in U\Rightarrow \exists D_{\varepsilon_x}(x):
D_{\varepsilon_x}(x)\in U$. Очевидно доказывается. что 
$$\boxed{\bigcup_{x\in U}D_{\varepsilon_x}(x)=U}$$ 
\textbf{Замечание.} Если к базе добавить произвольное открытое множество, то
новое множество также будет базой.\\
\textbf{Упражнение.} Привести пример двух баз евклидовой топологии на 
плоскости, которые не пересекаются с обычной базой (открытых шаров). 
(Решение: например, база из открытых квадратных или звездчатых окрестностей).\\
\textbf{Пример.} В $(\mathbb{R}^2,\tau_{MN})$, 
$\Sigma_{MN}=\{(b,b^*)\mid b\in\} $ !!!!!!!!!!!!!!!!!!!!!\\
\textbf{Пример.} Топология ираациональных точек на прямой
$(\mathbb{R},\tau_{im}),~\tau_{im}=\{\varnothing,\mathbb{R}\}\cup
\{U\subset \mathbb{R}\setminus\mathbb{Q}\} $.
Множество иррациоанльных точек не является базой, поскольку их объединение не
содержит всю прямую. Решение: добавить саму прямую. !!!!!!!!!!!!\\
\begin{theor}
    (критерий базы в топологическом пространстве)\\
    Пусть $(X,\tau)$ - опологическое пространство, и семейство множеств 
    удовлетворяет условию $\sigma\subset \tau$. $\Sigma$ является базой 
    топологии тогда и только тогда, когда 
    $\forall u\in\tau,\forall x\in U\exists W_{\beta_0}\in\Sigma:
    x\in W_{\beta_0}\subset U$
\end{theor}
\textbf{Доказательство.} Пусть $\Sigma$ - база топологии. Тогда любое открытое 
множество можно представить в виде объединений множеств из базы. Значит, для
$x\in U$ найдется множество из базы, в котором лежит $x$.  \\
Обратно. Множесто $\Sigma$ удовлетворяет первой аксиоме базы по определению.
Докажем выполнение второй аксиомы. Для любой точки в открытом множестве
по условию теоремы найдется окрестность из $\Sigma$, лежащая в открытом
множестве. 
!!!!!!!!!!!!!!!!!!!!!!
$\square$ 

\begin{theor}
    (критерий базы на множестве)\\
    Пусть Х - произвольное множесто, $\Sigma=\{W_\beta\subset X\mid\beta
    \in B\}$ - семейство подмножеств из Х. ЧТобы на Х существовала
    топология с данной базой, необходимо и достаточно выполнения
    двух условий:\\
    1. $X=\bigcup\limits_{\beta\in B} W_\beta$\\
    2. Для любых множеств из базы найдется множество, лежащее в их
    пересечении и содержащее произвольную точку оттуда.

\end{theor}
\textbf{Доказательство.} Необходимость. Пусть $\Sigma$ - база некотрой 
топологии (Х,т). Из акиомы базы (2) следует,что что Х есть объединение
множеств из $\Sigma$. значит, выполняется первое условие теоремы. Докажем второе 
условие. Достаточно взять пересечение двух множеств из базы. Так как 
это открытые множества, его также можно представить в виде объединения
множеств из базы, и хотя бы в одном из которых лежит фиксированная точка
(по определению объединения).\\
Достаточность. Докажем, что всевозможные объедения множеств из $\Sigma$  
является топологией. пусть это есть $\tau$. Проверим аксиомы топологии:\\
1.  Пустое множество принадлежит всему, чему надо. Все простарнство 
лежит там по условию теоремы.
3. Пусть 


$\square$ 


