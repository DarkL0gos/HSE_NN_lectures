\begin{theor}
Замыкание связного множества связно
\end{theor}
\textbf{Доказательство.}  Рассмотрим замыкание связного множества $A\subset 
\overline{A}\subset X$. Допустим, что замыкание несвязно. По определению 2,
$\overline{A}=A_1\cup A_2$ - объединенеие непересекающихся непустых множеств.
По свойству замкнутых множеств, это замкнутые множества. 
Значит, $A_1=\overline{A}\cap F_1,~A_2=\overline{A}\cap F_2$. 
По основной лемме, $A\subset A_1$ (но тогда $A_2=\varnothing $),
или $A\subset A_2$ (но тогда $A_2=\varnothing $).  Это противоречит нашему
предположению, значит, предположение неверно. 
$\square$ \\
Рассмотрим $(\mathbb{R}^n,\tau_E)$ - обычная топология, $\rho$ - обычная 
метрика. Рассмотрим единичный шар $D^n_1=\{x=(x^1,...,x^n)\in \mathbb{R}^n\mid
\rho(0,x)<1 \}$. Тогда сфера
$\mathbb{S}^{n-1}:=Fr^nD_1=\{x=(x^1,...,x^n)\in \mathbb{R}^n\mid\rho(0,x)=1\}$.
Топология на $\mathbb{S}^0=\{-1,1\}$ - дисретная, хотя и индуцирована из 
обычной. 
\begin{theor} (критерий связности)\\
Пусть $(X,\tau)$ - топологическое пространство. Тогда  $A\subset X$ несвязно
тогда и только тогда, когда существует непрерывное отображение $A$ на 
нульмерную сферу.
\end{theor}
\textbf{Доказательство.}  Пусть А несвязно. Представим его в виде $A=A_1\cup
A_2$. Определеим отображение  $f(x)=\begin{cases}
    -1,~x\in A_1\\1,x\in A_2\end{cases}$. Построенное отображение 
непрерывно, так как прообразы открытых множеств открыты, это непрерывное
отображение.
Обратно, пусть существует непрерывное отображение. Рассмотрим прообразы 
1 и -1. Они не пересекаются, в противном случае отображение неоднозначно.
Но тогда и А разбивается в их объединение, значит оно несвязно. 
$\square$ \\
\textbf{Пример.} Отрезок $[a,b]$ связен в обычной топологии. По критерию,
рассмотрим непреывное отображение $[a,b]\to \{-1,1\}$. Здесь непрерывность
совпадает с непреывностью в смысле матанализа, поэтому по теореме Коши это
не непрерывное отображение (в противном случае оно принимало бы все значения 
в отрезке $[0,1]$).
\begin{defin}
Компонента связности точки x в пространстве $(X,\tau)$ - максимальное 
по включению связное множество, содержащее х.
\end{defin}
\begin{theor}
1. Компонента связности любой точки замкнута.\\
2. Компоненты связности любых двух точек либо не пересекаются, либо 
совпадают.\\
3. $\forall x: X=K_x\iff X$ связно.
\end{theor}
\textbf{Доказательство.} 1 $\Rightarrow$2.
$x\in K_x\subset \overline{K_x}$ - связно.
С другой стороны, $\overline{K_x}\subset K_x$ по свойству замыкания.\\
2 $\Rightarrow$ 3. \\
3. $X=K_x$  связно по определению  $K_x$. Обратно,  $X$ - связно, тогда оно
атоматически макисмальная компонента связности для всех точек.
$\square$\\
\textbf{Следствие.} Так как компонента связности точки является таковой для
каждой своей точки, то она называется \textbf{компонентой связности 
пространства}.\\
\textbf{Следствие}. Компоненты связности образуют дизъюнктное разбиение
на замкнутые пространства. 
\subsection{Вполне несвязные пространства}
\begin{defin}
Пространство называется вполне несвязным, если компонента связности любой
точки состоит из этой же точки. 
\end{defin}
\textbf{Пример.} Простраснтво с дискретной топологией вполне несвязно.\\
\textbf{Пример.} Топология, индуцированная на $\mathbb{Q}$ из обычной 
топологии, вполне несвязна. 
\section{Компакты}
\begin{defin}
Система множеств $\xi=\{A_\alpha \subset X \mid \alpha\in I\}$ - покрытие
множества $A$, если  $A\subset \xi$
\end{defin}

Подпокрытие лежит впокрытии и тоже покрытие

Покрытие открыто, если каждый элемент покрытия открыт.
Покрытие замкнуто, если каждый элемент замкнут. 
Покрытие конечно, когда $|I|<\infty$. 
\begin{defin}
Покрытие $\xi=\{A_\alpha \subset X \mid \alpha\in I\}$ 
называют фундаментальным, если для любого отображения 
$f\colon  (X,\tau)\to (Y,\Omega)$
топологических пространств из непрерывности сужений $f|_{A_\alpha}$ 
вытекает непреывность отображения $f$.
\end{defin}
\begin{theor} (о достаточных условиях фундаментальности покрытия)\\
Пусть $X,\tau$ - топологическое пространство,  $\xi$ - его покрытие. 
Если выполняется по крайней мере одно из условий:
1. $\xi$ - открытое покрытие\\
2. $\xi$ - конечное замкнутое покрытое\\
то  $xi$ фундаментально.
\end{theor}
\textbf{Доказательство.} Пусть $f\colon X\to Y$ - открытое отображение. 
Тогда выполняется равенство: 
$\forall B\subset Y:f^{-1}(B)=$
$\square$ 
