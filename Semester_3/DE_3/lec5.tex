\subsection{Автономные уравнения}
\begin{defin}
Автономное ДУ - дифференциальное уравнение, правая часть которого не зависит
от времени.
\end{defin}
Автономные уравнения не могут быть динамическими системами, так как они 
не зависят от времени, но можно искусственно этого достичь.
\subsection{Нелинейный маятник}
Также: нелинейный консервативный (то есть сохряняющий энергию)
осциллятор, физический маятник. 
\begin{figure}[H]
    \centering
    \input{figures/pendulum.pdf_tex} %для pdf_tex
    %\includegraphics{}  % для png, pdf
    \caption{Физический маятник}
    \label{fig:figures-pendulum-pdf_tex}
\end{figure}
Рассмотрим маятник с массой $m$ и длиной $l$, и пусть 
$\varphi$ - угол отклонения от положения равновесия. При повороте на малый 
угол движение можно представить как прямолинейное движение по касательной.
Запишем второй закон Ньютона в проекции на касательную: 
$$m\frac{d^2x}{dt^2}=-mg\sin\varphi$$ 
Пусть $\Delta x$ - длина дуги окружности, примерно равная малой части 
касательной. Тогда  $\Delta x=l\Delta\varphi+o(\Delta\varphi)$.
%Получим уравнение %$\frac{fx}{dt}=l \frac{d\varphi}{dt}$. Finally,
Пренебрегая бесконечно малыми членами, запишем уравнение колебаний. 
$$ml \frac{d^2\varphi}{dt^2}=-mg\sin\varphi$$ 
\textbf{Другой способ вывода уравнения (\cite{Landau},\S11)}. 
Кинетическая энергия маятника равна $-\frac{m(l\dot\varphi)^2}{2}$,
потенциальная энергия равна $mgl\cos\varphi$. Функция Лагранжа
(кинетическая энергия минус потенциальная энергия, выраженные в переменных
координата-скорость) для маятника тогда имеет вид
$L=\frac{ml^2\dot\varphi^2}{2}+mgl\cos\varphi$. Для консервативной системы
справедиво уравнение движения Лагранжа:
$$\frac{d}{dt}\frac{\partial L}{\partial \dot x}
=\frac{\partial L}{\partial x}$$
Подставляя сюда функцию Лагранжа для маятника, получаем точно такое же 
уравнение физического маятника. 
$$ml^2\ddot \varphi = -mgl\sin\varphi$$
В отличие от уравнения малых колебаний, оно нелинейное из-за синуса. 
Также, оно имеет второй порядок, значит, появятся две константы интегрирования,
и нам надо зафиксировать начальные условия: $\varphi(0),~\dot\varphi(0)$.
Теперь сведем уравнение к системе из двух уравнений первого порядка 
заменой $\dot\varphi=\psi$:
$$\begin{cases}
    \dot\varphi=\psi\\
    \dot\psi=-\omega^2\sin\varphi
\end{cases}$$
Как устроено фазовое пространство данной системы? Ясно, что оно двумерное
(поскольку имеется две зависимые переменные - $\varphi$ и $\psi$). Близкие
положения системы должны быть близки и в фазовом пространстве, поэтому
на оси $\varphi$ мы склеим точки $\pi,-\pi$, чтобы маятник мог делать
<<солнышко>> и его фазовая траектория не была разрывной. Более формально, 
введем факторпространство по отношению эквивалентности 
$(\varphi,\psi)\sim(\varphi+2\pi k,\psi)$). 
Получаем, что фазовое пространство - цилиндр. Любая фазовая траектория - 
определнная замкнутая кривая. На цилиндре есть два типа замкнутых кривых 
- стягиваемые в точку и нестягиваемые. Вторые отвечают за движение
через верхнюю точку равновесия. \\
Теперь решим систему. Поделив одно уравнение на другое, получаем 
$$\frac{d\psi}{d\varphi}=-\frac{\omega^2\sin\varphi}{\psi}$$
Полная энергия равна константе:
$E = \frac{m\psi^2}{2}+\frac{mg}{l}(1-\cos\varphi)$. Выражая 
отсюда угол, имеем. 
$$\varphi=\pm\sqrt{\frac{2}{m}\left( h-\frac{mg}{l}(1-\cos\varphi)\right)}$$ 
Функция $F(\varphi)=\frac{mg}{l}(1-\cos\varphi)$ сохряняется вдоль 
фазовых траекторий, так как $\frac{dF}{dt}=0$. Её изолинии (то есть
уровни постоянной энергии) - одномерные торы. Из анализа фазовых траекторий
можно выяснить, что период колебаний растет по
мере увеличения энергии. Также есть два состояния равновесия: верхнее 
(неустойчивое) и нижнее (устойчивое). 









