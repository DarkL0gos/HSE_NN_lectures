\begin{defin}
Автономное ДУ - дифференциальное уравнение, правая часть которого не зависит
от времени.
\end{defin}
Автономные уравнения не могут быть динамическими системами, так как они 
не зависят от времени, но можно искусственно этого достичь.\\
\textbf{Пример.} Нелинейный консервативный
осциллятор. Рассмотрим маятник с координатами
$\varphi$ - отклонение от положения равновесия. Рассмотрим плоские
колебания маятника массой $m$ и длиной $l$. При повороте на малый угол 
движение можно представить как прямолинейное движение по касательной.
Запишем второй закон Ньютона в проекции на касательную: 
$$\vec\tau:~m\frac{d^2x}{dt^2}=-mg\sin\varphi$$ 
Пусть $\Delta x$ - длина дуги окружности, примерно равная малой части 
касательной. Тогда  $\Delta x=l\Delta\varphi+o(\Delta\varphi)$.
Получим уравнение
$\frac{fx}{dt}=l \frac{d\varphi}{dt}$. Finally,
$$ml \frac{d^2\varphi}{dt^2}=-mg\sin\varphi$$ 
- уравнение колебания маятника. Оно нелинейное из-за синуса. 
Оно имеет порядок 2, значит, нам надо
зафиксировать начальные условия: $\varphi(0),~\dot\varphi(0)$.
Уравнение тогда превратится в систему
$$\begin{cases}
    \dot\varphi=\psi\\
    \dot\psi=-\omega^2\sin\varphi
\end{cases}$$
Кстати, если мы напишем функцию Лагранжа и напишем уравнение Лагранжа для
него, то получим это же уравнение.\\
Начнем решение. Сделаем замену $\dot\varphi=\psi$. Теперь введем фазовое 
пространство угол-скорость таким образом, чтобы близкие точки были близки.
В угловых координатах мы склеим точки $\pi,-\pi$ у координат углов (точнее,
создадим факторпространство по отношению $(\varphi,\psi)\sim(\varphi+2\pi k,
\psi)$). Получим, что фазовое пространство - цилиндр. Любая замкнутая кривая 
- это некоторая траектория (вообще говоря, определляемая уравнением).
На цилиндре есть два типа замкнутых кривых - стягиваемые в точку и 
нестягиваемые. Вторые отвечают за движение через верх. \\
Продолжаем решение. Из системы имеем 
$\frac{d\psi}{d\varphi}=-\frac{\omega^2\sin\varphi}{\psi}$.
Полная энергия равна константе:
$\frac{m\psi^2}{2}+\frac{mg}{l}(1-\cos\varphi)=h$.
Это мы вывели из формы уранвения в полных диференциалах. 
В общем, решаем. Получим
$$\varphi=\pm\sqrt{\frac{2}{m}\left( h-\frac{mg}{l}(1-\cos\varphi) \right) }$$ 
Нарисуем фазовые траектории, и ещё функцию 
$F(\varphi)=\frac{mg}{l}(1-\cos\varphi)$.
Уровни постоянной энергии - одномерные торы. Как и обычно с функцией
Гамильтона. 
Из анализа фазовых траекторий можно выяснить, что период колебаний растет по
мере увеличения энергии. Также есть два состояния равновесия: верхнее 
(неустойчивое) и нижнее (устойчивое). 









