\begin{defin}
Функция $\tilde x(t)$ определенная на интервале $(a,b)$, называется 
продолжением решения вправо, если она совпадает с $x(t)$ на некотором
подинтервале.
\end{defin}

\begin{theor}
    (о продолжении решения)\\
    Пусть дано уравнение $\frac{dx}{dt}=f(t,x),~x(t_0)=x_0$, 
    функции $f(t,x),~\frac{\partial f}{\partial x}$ непрерывны на компакте
    $D\subset \mathbb{R}^2$ (причем в $D$ лежит как минимум 1 шар),
    $x(t,t_0,x_0)$ - решение задачи Коши для  $(t_0,x_0)\in Int D$. 
    Тогда существует решение, определенное на отрезке $[a,b]$, причем
     $(a,\tilde x(a,t_0,x_0)),(b,\tilde x(b,t_0,x_0))\in \partial D$.
     Иначе говоря, решение продолжается на границу компакта.
\end{theor}
\textbf{Доказательство.} В силу теоремы о существовании и единственности 
решения, функция $(x,t_0,x_0)$ определена на отрезке $[t_0-\delta_0,
t_0+\delta_0]$, где $\delta_0=\frac{r_0}{\sqrt{1+m^2} }= \frac{\rho(P_1,
\partial D)}{\sqrt{1+m^2} }$. 

Положим $t_1=t_0+\delta_0,~x_1=x(t_1,t_0,x_0),p_1(t_1,x_1)$. 
определим 
$$\tilde x(t,t_0,x_0)=\begin{cases}
    x(t,t_0,x_0),~t\in [t_0-\delta_0,t_0+\delta_0];\\
    x(t,t_1,x_1),~t\in [t_1-\delta_1,t_1+\delta_1];
\end{cases}$$
Если $(x_1,t_1)$ лежит на границе, то все хорошо. Если нет, то будем 
увеличивать шар, пока не достигнем границы множества (в силу компактности 
это всегда можно сделать). 

Возможен вариант, когда последовательность $\delta_i$ стремится к нулю и 
сама не затрагивает границу компакта. Рассмотрим функцию, определенную 
на $t\in [t_0-\delta_0,t+k+\delta_k]$. Последовательность 
$t_k$невозрастающая и ограниченная, поэтому существует и предел 
$b$. Функция $\tilde x$  определена на объединении интервало
$\bigcup\limits_{k}[t_0-\delta_0,t_k+\delta_k]=[t_0-\delta_0,b)$.
Воспользуемся непрерывностью функций: пусть $0<h\ll1$.
Тогда $\forall \alpha,\beta\in (b-h,b):|\tilde| x(\alpha,t_0,x_0)-
\tilde x(\beta,t_0,x_0)|\leqslant m|\alpha-\beta|<mh$. 
Последовательность $\tilde x_k$ фундаментальна, значит по критерию Коши 
у неё есть конечный предел. Положим этот предел значением функции 
в точке $b$:  $x^*=\tilde x(b)$. Тогда функция непрерывна на $[t_0-\delta_0,
b]$. Вспомним про интегральное уравнение: заметим, что $\tilde(x)$ 
удовлетворяет интегральному уравнению на интервале. Функция, дополненная
на конце интервала, непрерывна и также удовлетворяет интегральному 
уравнению, поэтому в ней есть и производная (по эквивалентности
определений).

Покажем, что точка $b$ лежит на границе области  $D$. Предположим противное,
тогда она лежит во внутренности $D$. Тогда она лежит в нем вместе
с некоторой $2\varepsilon$-окрестностью с центром в $p^*=(t^*,x^*)$.
Так как точки $p\to p^*$, то  все $p_i,i>k$ лежат в  $\varepsilon$-шаре
точки $p^*$. Тогда расстояние до границы больше  $\varepsilon$, 
и мы получаем противоречие с тем, что ряд из $\delta_k$ сходится и также
удален от границы больше чем на $\varepsilon$. Значит, $p^*\in \partial D$.
$\square$ \\
\textbf{Следствие.} Пусть $D\subset \mathbb{R}^2$ - такое неограниченное 
замкнтуое 
подмножество плоскости, что для любых $(a,b):D_{a,b}=D\cap \{
(t,x):a\leqslant t\leqslant b\}$ компактно, функции 
$f(t,x),\frac{\partial f}{\partial x}$ непрерывны на $D$. Тогда 
решение задачи Коши продолжается либо неограниченно, либо до выхода на границу
 $D$. Доказать самостоятельно. 

\textbf{Пример.} $x'=t^3-x^3$. Показать, что любое решение этого
уравнения продолжается неограниченно вправо.  Нарисуем изоклину $x=t$. 
Заметим, что если $t_0>x_0$, то $x(t,t_0,x_0)\in D:$. Тогда в силу следствия
решение продолжается на границу, на граница не достигается, то есть

\textbf{Пример.} $x'=1+x^2$. 
Его решение - $x=\tg(x+C),~C=\arctg(x_0)-t_0$, 
поэтому его нельзя продолжить до бесконечности,
так как каждое решение определено на конечном интервале $(C_0-\frac{\pi}{2},
C_0+\frac{\pi}{2})$. 

\textbf{Практика.}

\textbf{Пример (№199).} $y^2dx-(xy-x^3)dy=0$. Раскроем скобки и перегруппируем
слагаемые: $y(ydx-xdy)-x^3dy=0$. Поделим все на  $x^2$, тогда получим:  
$-d\left( \frac{x}{y}-\frac{x}{y}dy=0 \right)$. Домножая на $- \frac{y}{x}$,
получаем $d\left( \frac{1}{2}\left( \frac{y}{x} \right)^2  \right) +dy=0$.
Итак, $d\left( \frac{1}{2}\left( \frac{y}{x} \right)^2 +y\right)=0$
В общем, мы нашли интегрирующий множитель методом внимательного взгляда. 
Ответ: $\frac{1}{2}\left( \frac{y}{x} \right)^2+y=const$.

\textbf{Пример (№202).} $d(\ln|\sin(xy)|)+\ln|y|=0$.

\begin{defin}
Интегрирующий множитель - такая функция $\mu(z(x,y))$, что при домножении на 
неё уравнение становится уравнением в полных дифференциалах.
\end{defin}
Тогда $\frac{\partial(\mu M)}{\partial y}=\frac{\partial(\mu N)}{\partial x}$.
То есть $\frac{\partial \mu}{\partial z}\frac{\partial z}{\partial y}M=
$
Получаем, что $\frac{d\mu}{\mu}= \frac{N_x-M_y}{z_yM-z_xN}=P(z)$. То есть, если
интегрирующий множитель существует, то он удовлетворяет этому условию. 
Значит, $\mu=e^{\int\limits_{}^{} P(z)dz}$.

\textbf{Пример (№212).} $(2x^2y^3-1)ydx+(4x^2y^3-1)xdx=0$.
Пусть $z=xy$. Найдем интегрирующий множитель: $\mu=\frac{1}{(xy)^2}$.


