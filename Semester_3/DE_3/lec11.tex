\textbf{Задача.} Что нам мешает провести через одну точку несколько
решений уравнения $y'=x-y^2$? Тот факт, что тангенс угла наклона задается
уравнением однозначно, поэтому трансверсальное пересечение невозможно.
А если касательные параллельны? Если такая ситуация имеет место, тогда по 
теореме о существовании и единственности в этой точке правая часть либо
её производная не непрерывны, но это не так. А что, если $y''=x-y^2$ - 
уравнение второго порядка? Тогда все-таки ничего неельзя сказать (там есть
свои теоремы). 
\section{Уравнение, не разрешенное относительно производной}
\subsection{Уравнение первого порядка}
Общий вид - 
\begin{equation}
    \label{ur_nerazresh}
F\left( t,x,\frac{dx}{dt} \right) = 0
\end{equation}

Уравнение нельзя разрешить относительно производной, если $\frac{dx}{dt}$ 
нельзя выразить единственным образом. 

\begin{figure}[H]
    \centering
    \input{figures/diffur_nerazresh.pdf_tex}
    \caption{Уравнение задает поверхность}
    \label{fig:}
\end{figure}

Допустим, функция $F( t,x,\frac{dx}{dt}) = 0$ задает какую-то поверхность в 
$\mathbb{R}^3$. Вектор нормали к этой поверхности:
 $\left( \frac{\partial F}{\partial t},\frac{\partial F}{\partial x},
 \frac{\partial F}{\partial p} \right) $ (где введено обзначение
 $p=\frac{dx}{dt}$). Уравнение нельзя разрешить, если этот вектор 
 параллелен плоскости $(x,t)$. Но зато мы можем выразить, например, 
 $x$ от  $p,t$.

 Пусть $F,\frac{\partial F}{\partial t},\frac{\partial F}{\partial x},
 \frac{\partial F}{\partial p}$ непрерывны в области $D\subset Otxp$, 
 и в каждой точке хотя бы одна из производных не равна нулю. Тогда по 
 теореме о неявной функции уравнение \ref{ur_nerazresh} можно
 разрешить относительно одной из переменных. 

\textbf{Пример.} $x'^2-x^2=0$. Два семейства решений:
$x=x_0e^{\pm t}$. Как видно, в каждой точке пересекаются 2 решения. 
Для таких уравнений  ситуация с пересечением решений 
типична, но их количество и взаимный наклон определены в зависимости от вида
уравнения.
\begin{defin}
Особая точка - точка,через которую проходит несколько решений.
\end{defin}
\begin{theor}
Пусть $F$ непрерывна по всем аргументам, имеющая непрерывные частные 
производные по  $x,t$ и  $\frac{\partial F}{\partial x}\ne 0$. Тогда
существует одна или несколько функций $f(t,x)$ такие, что 
$F(t,x,f(t,x))\equiv0$, и решение задачи Коши $x(t_0)=x_0,~x'(t_0)=x'_0$
существует и единственно. 
\end{theor}
\textbf{Доказательство.}  
Допустим, решение существует.
Рассмотрим полную производную по времени: 
$\frac{dF(t,x,x')}{dt}=\frac{\partial F}{\partial t}+
 \frac{\partial F}{\partial x}\frac{\partial x}{\partial t} +
 \frac{\partial F}{\partial x'}\frac{\partial x'}{\partial t}\equiv 0$
Тогда $\frac{dx}{dt}=\frac{-\frac{\partial F}{\partial t} -
\frac{\partial V}{\partial x} }{}$ списываем из фихтенгольца
Эти условия должны выполняться в окрестности какой-то точки
$F(t_0,x_0,x'_0)=0$ $\square$ \\
%\begin{defin}
%Обыкновенная точка уравнения $F(t,x,x')$ - точка, через которую ожидаемое 
%количество решений. 
%\end{defin}
\begin{theor}
Пусть $F$ непрерывна по всем аргументам в области $D$,
имеющая непрерывные частные 
производные по  $x,t$ и  $\frac{\partial F}{\partial x}\ne 0$. Тогда
для любой точки $t_0,x_0,x'_0$ существует и единственно решение задачи Коши. 
\end{theor}
\textbf{Доказательство.}  
$\square$ \\
\begin{defin}
Регулярная (обыкновенная) точка уравнения $F(t,x,x')$ - точка $(t,x)$, 
в которой задача Коши (для уравнения, разрешенного или не разрешенного 
относительно производной) имеет единственное решение. Если решений несколько
или ноль, то точка особая (сингулярная).
\end{defin}

\begin{theor}
Пусть $f(t,x)$ - такая функция, что  $F(t,x,f(t,x))\equiv 0$. Тогда
любое решение $x(t)$ уравнения \ref{ur_nerazresh}, удовлетворяющее 
условию $\frac{dx}{dt}=f(t,x)$, является решением 
уравнения  $ \frac{dx}{dt}=f(t,x)$
\end{theor}
\textbf{Доказательство.}  Упражнение. 
$\square$ \\


\subsection{Практика и методы решения}
Есть два метода решения уравнения \ref{ur_nerazresh}:\\
1. Если  $\frac{\partial F}{\partial p}\ne 0$, тогда разрешить относительно
$p$ и решать несколько уравнений типа  $\frac{dx}{dt}=f(t,x)$.\\
2. Метод введения параметра.
Если $\frac{\partial F}{\partial p}=0,\frac{\partial F}{\partial x}\ne 0$,
то \ref{ur_nerazresh} эквивалентно уравнению $x=\varphi(t,p)$, 
где $\varphi$ - такая функция, что $F(t,\varphi,p)\equiv 0$. 
Будем считать $p$ параметром и искать решение уравнения \ref{ur_nerazresh}
в виде $x=x(p),t=t(p)$. Найдя  $t(p)$, тогда  $x=\varphi(t(p),p)$.
Далее, имеем $dx=\varphi_t dt+\varphi+p dp$,
$p\,dt=\varphi_t+\varphi_p dp$. Finally,
$$\frac{dt}{dp}=\left( \frac{\varphi_p}{p-\varphi_t} \right)$$


\textbf{Пример на метод введения параметра.} 
$x=\dot xt-\dot x^2$. Введем параметр $p=\dot x$ и ищем решение в виде
$\begin{cases}x=x(p)\\t=t(p)\end{cases}$.
Имеем уравнение $x=pt-p^2$, которое можно записать в виде
$dx=dp\,t+p\,dt-2p\,dp$, то есть $dp(t-2p)=0$. Оно разбивается в два
уравнения:  $dp=0$ (то есть $p=const$,  $x=Ct-C^2$) и  $t=2p$ (то есть
$x=\left( \frac{t}{2} \right)^2$). 

Это - частный случай уравнения Клеро:
$$x=\frac{dx}{dt}t+\psi\Big(\frac{dx}{dt}\Big)$$
Сформулируем теорему:
\begin{theor}
Общее решение уравнения Клеро - семейство прямых
$$x=Ct+\psi(C)$$ 
и (возможно, вырожденная) кривая - огибающая для семейства кривых:
$$x=\eta(t)$$
\end{theor}
%Эта кривая является огибающей для семейства прямях.
\textbf{Доказательство.} Пусть $p=\dot x$,  $dx=p\,dt$. Тогда
$dp(t+\psi'_p(p))=0$. Значит, одно из решений - семейство прямых
 $x=Ct+\psi(C)$. Другое решение получаем из условия 
 $t=-\psi'(p)$. Подставляя её в исходное уравнение, получаем 
 $x=-p\psi'(p)+\psi(p)=\eta(p)$.

 Пусть $p=p_0$
 $\psi$ ????????????????? Юра скоро допишет

Убедимся в том, что касательные к кривой $x=\eta(t)$ и к прямой  $x=C_*t+
\psi(C_*)$ в точке  $(t_0,x_0)$ совпадают, то есть кривая - действительно 
огибающая. 
$\frac{d\eta}{dt}\big|_{t_0}=C_*$. 

$\psi'(p_0)(C_*-p_0)=\psi(C_*)-\psi(p_0)$. 
Это - уравнение Лагранжа. Значит, уравнение Клеро - частный случай уравнения 
Лагранжа:
$$x=\alpha(\dot x)t+\psi(\dot x)$$


$\square$ \\

\textbf{Пример (№292).} $y=x(y')^2-3(y')^3$. Вводим параметр 
 $y'=p$, $y=xp^2-2p^3$,  $dy=dx\,p^2+x\cdot 2p\,dp-6p^2\,dp$. С другой 
 стороны, $dy=p\,dx$, поэтому имеем
$$p\,dx=dx\,p^2+x\cdot 2p\,dp-6p^2\,dp$$ 
Группируя,  получаем уравнение $\frac{dx}{dp}=\frac{2xp-6p^2}{p-p^2}$.
Значит, введением параметра уравнение Лагранжа приводится к линейному
относительно $x$. Доказать самостоятельно. Итак,
$$\frac{dx}{dt}=x\cdot \frac{2}{1-p}-\frac{6p}{1-p}$$ 
Общее решение однородного уравнения $\frac{dx}{dt}=\frac{2x}{1-p}$ 
с разделяющимися переменными - $x_{oo}=\frac{C}{(1-p)^2}$. 
Частное решение неоднородного уравнения сложно угадать, используем 
метод Лагранжа: $\frac{C'(p)(1-p)^p+C(p)2(1-p)}{(1-p)^4}=\frac{2C-6p}{1-p}$.

Итак, ответ:
$$\begin{cases}
    y=xp^2-2p^3\\x=\frac{-3p^3\tilde C}{(1-p)^2}
\end{cases}$$
%дарбу эйлера пуассона

\textbf{Пример (№268).}
$x=(y')^3+y'$. Введем параметр  $p=y'$,  $\frac{dy}{dx}=p$, $dx=\frac{dy}{p}$.
Имеем $x=p^3+p$,  $dx=3p^2\,dp+dp$. Заметим, что это можно делать при 
условии  $p\ne 0$. 
Значит,  $y=\frac{3p^4}{4}+\frac{p^2}{2}+C$ - ответ.

Теперь - про особые решения. Найдем такие начальные условия, при которых
задача Коши имеет единственное решение. То есть
$$\begin{cases}
    y(x_0)=y_0\\y'(x_0)=y'_0\\x_0=y'^3_0+y'_0
\end{cases}$$
Чтобы получить единственность решения, необходимо проверить условия
теоремы Коши-Пикара.\\
1. Покажем, что уравнение $F=x-y'^3-y'=0$ разрешимо относительно производной, 
при этом функция  $f(x,y)$ будет гладкой. Фиксируя точку
$(x_0,y_0,y'_0)$ и в ней все частные производные и сама $F$ непрерывна и
$F'_y\ne 0$, то
существует гладкая функция $f(x,y)$:  $F(x,y,f(x,y))\equiv 0$ по теореме о 
неявной функции. \\
2. Покажем, что функция  $y'=f(x,y)$
Кароч в дз отлетает. Че за троллинг?
решени


\textbf{Пример (№249).} $(y')^3+y^2=y\cdot y'(y'+1)$. Группируем и выносим
общий множитель два раза, получаем $((y')^2-y)(y'-y)=0$. 
Получаем три уравнения: 
$\begin{cases}y'=y;\\y'=\sqrt{y};\\y'=-\sqrt{y}\end{cases}$. Решения:
$\begin{cases}y=y_0e^t;\\y=\left( -\frac{x}{2}+\frac{c}{2}\right)^2 \\
y=\begin{cases}\left( \frac{x+x_0}{2} \right)^2,~x<x_0\\0,~x\geqslant x_0
    
\end{cases}  
\end{cases}$
Найдем особые точки: $y=0$, ибо там бесконечно много решений.
Вообще говоря, 
 $y=\begin{cases}0,~x\leqslant x_0\\\left( \frac{x-x_0}{2} \right)^2,x>x_0
  
 \end{cases}$
Проверим условия теоремы для оставшихся точек (тем доказав, что других особых 
точек нет). Гладкость функции очевидна, нули производной:
$3(y')^2-y\cdot 2y'-y=0$

\begin{defin}
Особое решение - решение состоящее из особых точек
\end{defin}


