\textbf{Задача.} Что нам мешает провести через одну точку несколько
решений уравнения $y'=x-y^2$? Тот факт, что тангенс угла наклона задается
уравнением однозначно, поэтому трансверсальное пересечение невозможно.
А если касательные параллельны? Если такая ситуация имеет место, тогда по 
теореме о существовании и единственности в этой точке правая часть либо
её производная не непрерывны, но это не так. А что, если $y''=x-y^2$ - 
уравнение второго порядка? Тогда все-таки ничего неельзя сказать (там есть
свои теоремы). 
\subsection{УРавнение, не разрешенное относительно производной}
Общий вид - 
$$F\left( t,x,\frac{dx}{dt} \right) = 0$$ 
\textbf{Пример.} $x'^2-x^2=0$. Два семейства решений:
$x=x_0e^{\pm t}$. Как видно, в каждой точке пересекаются 2 решения. 
Для таких уравнений  ситуация с пересечением решений 
типична, но их количество и взаимный наклон определены в зависимости от вида
уравнения.
\begin{defin}
Особая точка - точка,через которую проходит несколько решений.
\end{defin}
\begin{theor}
Пусть $F$ непрерывна по всем аргументам, имеющая непрерывные частные 
производные по  $x,t$ и  $\frac{\partial F}{\partial x}\ne 0$. Тогда
существует одна или несколько функций $f(t,x)$ такие, что 
$F(t,x,f(t,x))\equiv0$, и решение задачи Коши $x(t_0)=x_0,~x'(t_0)=x'_0$
существует и единственно. 
\end{theor}
\textbf{Доказательство.}  
Допустим, решение существует.
Рассмотрим полную производную по времени: 
$\frac{dF(t,x,x')}{dt}=\frac{\partial F}{\partial t}+
 \frac{\partial F}{\partial x}\frac{\partial x}{\partial t} +
 \frac{\partial F}{\partial x'}\frac{\partial x'}{\partial t}\equiv 0$
Тогда $\frac{dx}{dt}=\frac{-\frac{\partial F}{\partial t} -
\frac{\partial V}{\partial x} }{}$ списываем из фихтенгольца
Эти условия должны выполняться в окрестности какой-то точки
$F(t_0,x_0,x'_0)=0$ $\square$ \\
%\begin{defin}
%Обыкновенная точка уравнения $F(t,x,x')$ - точка, через которую ожидаемое 
%количество решений. 
%\end{defin}
\begin{theor}
Пусть $F$ непрерывна по всем аргументам в области $D$,
имеющая непрерывные частные 
производные по  $x,t$ и  $\frac{\partial F}{\partial x}\ne 0$. Тогда
для любой точки $t_0,x_0,x'_0$ существует и единственно решение задачи Коши. 
\end{theor}
\textbf{Доказательство.}  
$\square$ \\
\begin{defin}
Регулярная (обыкновенная) точка уравнения $F(t,x,x')$ - точка $(t,x)$, 
в которой задача Коши (для уравнения, разрешенного или не разрешенного 
относительно производной) имеет единственное решение. Если решений несколько
или ноль, то точка особая (сингулярная).
\end{defin}

\textbf{Пример (№249).} $(y')^3+y^2=y\cdot y'(y'+1)$. Группируем и выносим
общий множитель два раза, получаем $((y')^2-y)(y'-y)=0$. 
Получаем три уравнения: 
$\begin{cases}y'=y;\\y'=\sqrt{y};\\y'=-\sqrt{y}\end{cases}$. Решения:
$\begin{cases}y=y_0e^t;\\y=\left( -\frac{x}{2}+\frac{c}{2}\right)^2 \\
y=\begin{cases}\left( \frac{x+x_0}{2} \right)^2,~x<x_0\\0,~x\geqslant x_0
    
\end{cases}  
\end{cases}$
Найдем особые точки: $y=0$, ибо там бесконечно много решений.
Вообще говоря, 
 $y=\begin{cases}0,~x\leqslant x_0\\\left( \frac{x-x_0}{2} \right)^2,x>x_0
  
 \end{cases}$
Проверим условия теоремы для оставшихся точек (тем доказав, что других особых 
точек нет). Гладкость функции очевидна, нули производной:
$3(y')^2-y\cdot 2y'-y=0$

\begin{defin}
Особое решение - решение состоящее из особых точек
\end{defin}
