\section{Элементарные методы интегрирования ДУ}
\subsection{Уравнения с разделяющимися переменными}
\begin{defin}
Уравнение с разделяющими переменными - уравнение вида
\begin{equation}
    \frac{dx}{dt}=f(x)g(t) \label{ODE_razdp}
\end{equation}
где $f,g$ непрерывны на  $x\in(a,b),~t\in(\alpha,\beta)$
\end{defin}
Как решать такие уравнения? Алгебраическая нтуиция подсказывает, что надо 
перенести 
дифференциалы к своим функциям и проинтегрировать. Но это ещё надо обосновать.
Сделаем следующее:\\
\begin{enumerate}
    \item Найти все $x_*:f(x_*)=0$. Тогда $x=x_*$ - решение-константа. 
    \item Пусть  $x^i_*,x^j_*$ - такие, что  $f(x^i_*)=f(x^j_*)=0$ и
    $\forall x\in(x^i_*,x^j_*):f(x)\ne0$. Тогда уравнение \ref{ODE_razdp}
эквивалентно уравнению 
$$\frac{dx}{f(x)}=g(t)dt$$
Эту штуку можно проинтегрировать с обеих сторон. Результат непрерывен и не
обращается в ноль. Значит, по теореме о неявной функции найдется решение. 
$\frac{dF}{dx}=\frac{1}{x}$(решение в области $(\alpha,\beta)\times
(x^i_*,x^j_*)$).
    \item Выписать решение на каждом интервале $(x^i_*,x^j_*)$
\end{enumerate}
Других решений не существует. Почему? Допустим, существует другое решение.
Оно не может быть константой, так как все константы были получены в п.1.
Если она \\
\textbf{Пример.} Решим уравнение $\frac{dx}{dt}=0$. Решение-константа: $x=0$.
Теперь рассмотрим два интервала: $x<0$ и  $x>0$. Если  $x<0$, имеем уравнение
 $$\frac{1}{x}\frac{dxdt}{dt}=dt$$
 Интегрируем:
 $$\int\frac{dx}{x}=\int dt$$
 Получаем, что $\ln|x|=t+C$. Выражаем искомую функцию (не забыв, на каком
 промежутке мы рассматриваем функцию, и раскрыв модуль соответственно):
 $$x=-Ce^t,~C>0$$
Для интервала $x>0$ точно такой же порядок действий, только получим другой 
знак. Итак, множество решений:
$$x=Ce^t,~C\in\mathbb{R}$$
\subsection{Уравнения, приводящиеся к уравнению с разделяющимися переменными}
\begin{defin}
Уравнение, приводящееся к уранвению с разделяющмися переменными - уравнение
вида 
\begin{equation}
    \frac{dx}{dt}=f(at+bx+c) \label{ODE_privrazd}
\end{equation}
\end{defin}
Давайте решим его. 
\begin{enumerate}
    \item Введем замену $z(t)=at+bx+c$. 
    Имеем
     $$\frac{dz}{dt}=a+b\frac{dx}{dt}$$ 
     Получаем уравнение с разделяющимися переменными. 
     $$\frac{dz}{a+f(z)}=dt$$
\end{enumerate}
\textbf{Пример.} Решим уравнение $\frac{dx}{dt}=\cos(x+t)$. Замена 
$z=x+t,~ \frac{dz}{dt}=1$. Уравнение имеет вид
$$\frac{dz}{dt}=\frac{dx}{dt}+1$$ 
Найдем $\cos{z_*}+1=0$: это, очевидно, $\pi+2\pi k,~k\in \mathbb{Z}$ 
Свели задачу кпрошлому пункту
\subsection{Однородные уравнения}
Сначала докажем, что два определения однородного уравнения эквивалентны.
\begin{defin}
Однородным называется уравнение вида
\begin{equation}\label{ODE_odn1}
    \frac{dx}{dt}=f\left(\frac{x}{t}\right) \label{ODE_odn}
\end{equation} 
\end{defin}
Это уравнение инвариантно относительно замены $x\mapsto kx,~t\mapsto kt$.
Геометрически это означает, что совокупность интегральных кривых инвариантно
относительно преобразования $\theta(x,y)=(kx,ky)$.
Из этого следует, что если мы найдем одно решение, то мы найдем всю 
совокупность ему подобных. Вставить картинку.
\begin{defin}
    (вспомогательное)\\
Уравнение в форме дифференциалов:
    $M(x,y)dx+N(x,y)dy=0$.  
\end{defin}
Это таже форма, что и $\frac{dy}{dx}=f(x,y)$, поскольку 
$\frac{dy}{dx}=-\frac{M(x,y)}{N(x,y)}$. Обратно, $-f(x,y)dx+dy=0$.
Уравнение в форме дифференциалов имеет чуть большее множество решений. 
\begin{defin}\label{ODE_odn2}
Уравнение в форме дифференциалов называется однородным, если\\
$M(kx,ky)=k^nM(x,y)$\\ 
$N(kx,ky)=k^nN(x,y)$\\
n называется степенью однородности.
\end{defin}
\begin{theor}
    Определения \ref{ODE_odn} и \ref{ODE_odn2} эквивалентны. 
\end{theor}
\textbf{Доказательство.} 1 $\Rightarrow$ 2. $\frac{dy}{dx}=f(\frac{y}{x})$\\
2 $\Rightarrow$ 1. Пусть дано уравнение в форме дифференциалов. Подставим $k$.
При $x\ne 0$ Имеем $$\frac{dx}{dy}=-\frac{k^nM(x,y)}{k^nN(x,y)}=
-\frac{M(kx,ky)}{N(kx,ky)}=-\frac{M(1,\frac{y}{x})}{N(1,\frac{y}{x})}=f(x)$$
$\square$ \\
\textbf{Пример.} $M=x^2+y^2$\\
\textbf{Пример (№31).} Найти уравнение, решение которых - параболы с осью, 
параллельной оси ординат и касающиеся прямых $y=0,~y=x$. 
Во-первых, поймем, как выглядит уравнение такой параболы. Исходя из геометрии,
получим, что уравнение параболы, удовлетворяющее первому условию, имеет вид 
$y=ax^2+bx+\frac{b^2}{4a}$, а первому и второму - $y=ax^2+\frac{1}{2}x+
\frac{1}{16a}$. Остался один параметр $\Rightarrow$ уравнение первого порядка. 
Подставляем и хаваем ответ бесплатно:
$$y=\left(\frac{y'-\frac{1}{2}}{2x}\right)x^2+\frac{1}{2}x+\frac{2x}{16y'-8}$$ 
\textbf{Пример (№72).} Найти линии, у которых треугольники, образованные 
касательными, осью ОХ и точкой касания, имеют одинаковую сумму катетов. 
Из геометрических соображений имеем уравнение 
$$\frac{|y|}{|y'|}+|y|=b=const$$ 
Раскрываем модули. В простейшем случае имеет уравнение с разделяющимися 
переменными. 
$$\frac{dy}{dx}=\frac{y}{b-y}$$ 
Остальные уравнения такие же в принципе. Так шо это идет в дз 
Его легчайшее (и, видимо, общее) решение: $x+C=\pm b\ln{|y|}\pm y$\\
\textbf{Пример (№76).} Геометрическая интуиция не должна подводить нас. 
Вставить картинку. Есть кароч такая формула: 
$\tg\gamma=\frac{r}{r'}$




