\section{Теоремы о непрерывной зависимости задачи Коши}
от начальных условий и правой части уравнения.

Дано: уравнение c задачей Коши
$$\begin{cases}
    \frac{dx}{dt}=f(t,x)\\
    x(t_0)=x_0
\end{cases}$$ 
$f,\frac{\partial f}{\partial x}$  непрерывны в области $D$. 
Утверждается, что решение задачи Коши $x=x(t,t_0,x_0)$, определенное на 
отрезке $I=[a,b]$. 
непрерывно по всем 
аргументам. Из этого следует, что решение при малом изменении начальных 
условий будет мало отличаться от исходного (разумеется, на конечном
интервале). 

Обозначим $V_\rho=\{(t,x)\in \mathbb{R}^2\mid t\in I,|x-x(t,t_0,x_0)|
\leqslant \rho\}$ - цилиндрическая окрестность решения. 

\begin{theor}
    (о непрерывной зависимости)\\
    Пусть $f,\frac{\partial f}{\partial x}$  непрерывны в области $V_\rho$.
    Тогда для любого $\varepsilon$ для любой функции $g(t,x)$, такой, что
$g,\frac{\partial g}{\partial x}$ непрерывны в $V_\rho$ найдется
 $\delta_(\varepsilon):|x_0-y_0|\leqslant \delta,|g(t,x)-f(t,x)|\leqslant S$.
 Решение $y(t)$ задачи Коши для уравнения $\frac{dy}{dt}=g(t,y)$ продолжается
 на $I$ и  $\forall t\in I: |y(t,t_0,y_0)-x(t,t_0,x_0)|<\varepsilon$.
    %$$\forall \varepsilon>0~\forall g(t,x)$$
\end{theor}
\begin{figure}[H]
    \centering
    \input{figures/nepr_zav_cauchi.pdf_tex}
    \caption{Теорема о непрерывной зависимости}
    \label{fig:}
\end{figure}
Иначе говоря, решение, проходящее ближе чем $\rho$ от решения задачи Коши,
не выходит из этой $\rho$-окрестности.\\
\textbf{Лемма Гронуолла.} Пусть $\varphi(t),\beta(t)$ - непрерывные функции
на отрезке $[t_1,t_2]$, причем на отрезке $\beta(t)>0$ и  $\varphi(t)\leqslant 
\alpha+\int\limits_{t_1}^{t_2} \beta(\tau)\varphi(\tau)d\tau$. 
Тогда 
$\varphi(t)\leqslant \alpha e^{\int\limits_{t_1}^{t_2}\beta(\tau)d\tau}$.\\
\textbf{Доказательство.} Положим $\Phi(t)=\alpha+
\int\limits_{t_1}^{t_2} \beta(\tau)\varphi(\tau)d\tau$, где 
$\varphi(t)\leqslant \Phi(t)$. Тогда
$$\frac{\partial \Phi}{\partial t}e^{-\int\limits_{t_1}^{t_2}\beta d\tau}
-e^{-\int\limits_{t_1}^{t_2}\beta d\tau}\beta(t)\Phi(t)\leqslant 0$$ 
Все это выражение на самом деле является производной:
$$\frac{d}{dt}\left( \Phi e^{-\int\limits_{t_1}^{t}\beta d\tau} \right)
\leqslant 0$$ 
 Так как производная этой функции отрицательна, то 
$$\varphi \leqslant \Phi\leqslant \alpha
e^{\int\limits_{t_1}^{t_2}\beta d\tau}$$
Мы установили равносильность неравенства из условия и неравенства
$\frac{d\varphi}{dt}\leqslant \beta(t)\varphi(t)$. $\square$ 

Теперь перейдем к доказательству основной теоремы. Поскольку $f$ непрерывна,
то она ограниченна на компакте, иными словами
$$\forall (t,x)\in V_\rho~\exists M\geqslant 0,L\geqslant 0:
|f(t,x)|\leqslant M,~\left| \frac{df}{dx} \right|\leqslant L $$
Значит, эта функция липшицева:
$$|f(t,x)-f(t,y)|\leqslant L\cdot |x-y|,~|f(t,x)-g(t,y)|\leqslant \delta$$
Тогда $|f(t,x)-f(t,y)+f(t,y)-g(t,y)|\leqslant |-(f(t,x)+f(t,y))|+
|f(t,y)-g(t,y)|\leqslant L\cdot |x-y|+\delta$.

Оценим разность $|y(t,t_0,y_0)-x(t,t_0,x_0)|$. По лемме об интегральной
форме уравнения, это то же самое, что и 
$$\left| y_0+\int\limits_{t_0}^{t} g(\tau,y(\tau))d\tau -x_0 -
 \int\limits_{t_0}^{t} f(\tau,x(\tau))d\tau \right| \leqslant 
$$
$$
\leqslant 
|y_0-x_0|+\left|\int\limits_{t_0}^{t} g(\tau,y(\tau))d\tau -
\int\limits_{t_0}^{t} f(\tau,x(\tau))d\tau \right|\leqslant 
\delta+\left| \int\limits_{t_0}^{t} (L(|y-x|+\delta))d\tau \right|\leqslant 
$$
Пусть решение $y(t,t_0,x_0)$ определено на отрезке $I_1=[a_1,b_1]$. 
Продолжим неравенства:
$$
\leqslant \delta(1+(b_1-a_1))+\left| \int\limits_{t_0}^{t} L|y-x|d\tau\right| 
$$
Обозначим $\alpha=\delta(1+(b_1-a_1))$, 
$\varphi(t)=|y(t,t_0,y_0)-x(t,t_0,x_0)|$. Применяя лемму Гронуолла, получаем
$\varphi(t)\leqslant \alpha e^{L|b_1-a_1|}<\varepsilon_1$ (при 
$\alpha=\frac{\varepsilon_1}{2e^{L|b_1-a_1|}}$).
Таким образом, разность между двумя решениями меньше чем $\varepsilon_1$ на 
отрезке $[a_1,b_1]$, иными словами $y(t,t_0,y_0)$ лежит в 
$V_{\varepsilon_1}$-трубочке решения $x(t,t_0,x_0)$. По теореме о продолжении
решения,  $y(t,t_0,y_0)$ продолжается до выхода на границу
$\partial V_{\varepsilon_1}$. Через верхнюю и нижнюю границу часть границы
мы не выходим, так как $|y-x|<\varepsilon_1$, значит, $y(t,t_0,y_0)$
продолжается на $I$. Аналогично, если  $t<t_0$. $\square$

\textbf{Следствие.} Пусть числовая последовательность 
$x^i_0\to x_0$ сходится при $i\to\infty$. 
Тогда $x(t,t_0,x^i_0)\to x(x,t_0,x_0)$. Доказать самостоятельно. Упражнение:
доказать теорему о непрерывной зависимости одновременно от $t_0$ и $x_0$.
Также доказать следствие о равномерной сходимости.

\subsection{Как решать Рикатти через Пикара (с оценкой погрешности)}
Рассмотрим уравнение $\frac{dy}{dx}=x-y^2(x),~y(0)=0$. Найдем формулу
для решения на отрезке $x\in [0,0.5]$.
Допустим, решение существует. Запишем последовательность Пикара, определенную
рекуррентной формулой $y_{k+1}(x)=\int\limits_{0}^{x}(s-y^2_k(s)ds)$.
Посчитаем первые члены: $y_1=0+\int\limits_{o}^{x}sds=\frac{x^2}{2}$,
$y_2=\int\limits_{0}^{x}\left( s-\frac{s^4}{4}\right)ds=\frac{x^2}{2}-
\frac{x^5}{20}$, $y_3=\frac{x^2}{2}-\frac{x^5}{20}+\frac{x^8}{160}-
\frac{x^{11}}{4400}$. Построим ряд из последовательности:
$$y=y_0+(y_1-y_0)+(y_2-y_1)+(y_3-y_2)...+a_k$$ 
Остаток $a_k$ оценивается по формуле  $|a_k|\leqslant \frac{M}{L}\cdot 
\frac{L^k|t-t_0|^k}{k!}$. 

Рассмотрим шар $B_r=\{|x|\leqslant \frac{1}{2},y\in [0,\frac{1}{2}]\}$. 
Тогда постоянные Липшица для функции и её производной равны
$M=\max\limits_{B_r}|f(t,x)|,~
L=\max\limits_{B_r}|\frac{\partial f}{\partial x} (t,x)|$. 

