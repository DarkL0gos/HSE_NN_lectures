\section{Уравнение первого порядка}
\begin{defin}
Уравнение 
\begin{equation}\label{lin_de}
    \frac{dx}{dt}+a(t)x=b(t)
\end{equation}
где $a,b$ непрерывны на  $t\in (\alpha,\beta)$ (интервал непрерывности),
называется линейным ДУ первого порядка. Если при этом $b(t)\not\equiv 0$, то 
оно называется неоднородным.
\end{defin}
Как следствие из теоремы Коши-Пикара, для $\forall t_0\in (\alpha,\beta),~
\forall x_0\in \mathbb{R}$ существует и единственно решение задачи Коши.

\textbf{Замечание.} Решение задачи Коши для \ref{lin_de} можно продолжить
на весь интервал $(\alpha,\beta)$. Если этот интервал конечен, то функции
$a(t),b(t)$ ограниченны на нём, то есть  $|a(t)x+b(t)|\leqslant Ax+B$, 
и решениене выйдет за конус, образованный этой прямой. 

\begin{defin}
Линейныйй ператор - отображение $A\colon X\to Y$ %ЧЕЕЕЕ БЛЯЯЯЯЯЯЯЯЯТЬ ТАКОЕ
такое, что $A(x+y)=A(x)+A(y),~A(\lambda x)=\lambda A(x)$.
\end{defin}
Пусть $X=C^1(\alpha,\beta),~C^0(\alpha,\beta)$ - пространства дифференцируемых
и непрерывных функций. Положим $A(x)=\frac{dx}{dt}+a(t)x$. В силу линейности
производной, это - линейный оператор. Также и любая линейная комбинация 
производных (любого порядка) является линейным оператором. 

Итак, уравнение \ref{lin_de} в операторной записи эквивалентно 
$Ax=b(t)$. Обозначим за $x_{o.n.}$ множество решений неоднородного уравнения,
$x_{o.o.}$ - множество решений однородного уравнения,  $x_{o.n}+x_{o.o}$ -
множество вида $x+x$

\begin{theor}
    (о структуре решения линейного уравнения)\\
    Решение неоднородного уравнения - сумма общего решения однородного 
    уравнения и частного решения.
\end{theor}
\textbf{Доказательство.} Пусть $\varphi(t)$ - частное решение однородного
уравнения, $x_{p}$ - частное решение неоднородного уравнения. Применим
оператор  $A$ к их сумме:  $A(\varphi(t)+x_p)=A\varphi(t)+Ax_p=0+b(t)$. 
Значит, сумма этих функций обращает уравнение в тождество, значит,
$\varphi(t)+x_p\in x_{o.n.}$.

Докажем, что других решений нет. Допустим, $\psi(t)\in x_{o.n.}$ таков, что
его нельзя представить суммы решений однородного и неоднородного. Рассмотрим
$\psi-x_p$ - вычтем частное решение неоднородного. Подставляя в уравнение, 
получаем  $A(\psi-x_p)\equiv 0$, значит, их разность - решение однородного
уравнения. Но это противоречит предположению. $\square$

Как решать линейные уравнения? Сначале решаем однородное уравнение:
$\frac{dx}{dt}=-a(t)x$, $x=C(t)e^{-\int_{t_0}^{t} a(\tau)d\tau}$. 
Решать неоднородное 3мя способами:
1. Угадайка\\
2. Метод Лагранжа вариации постоянных\\
3. Формула Коши (см. справочник).
\subsection{Метод Лагранжа}
Мы знаем, что $x=Ce^{-\int a(t)dt}$ - решение однородного уравнения. 
Будем её варьировать, чтобы в уравнении было бы тождество:
$$\frac{d}{dt}\left( C(t)e^{-\int\limits_{t_0}^{t}a(\tau)d\tau} \right)+
a(t)C(t)e^{-\int\limits_{t_0}^{t}a(\tau)d\tau}=b(t)$$
Дифференцируя, получаем $C'=b(t)e^{-\int\limits_{t_0}^{t}a(\tau)d\tau}$, 
откуда  $$C=e^{-\int\limits_{t_0}^{t}a(\tau)d\tau}\int\limits_{t_0}^{t}\left( 
b(s)e^{-\int\limits_{s_0}^{s}a(\tau)d\tau} \right)ds+
C_0e^{-\int\limits_{t_0}^{t}a(\tau)d\tau}$$ 
Значит, мы нашли семейство всех решений неоднородного уравнения, произвольно
выбирая $C_0$. По предыдущей теореме, этим все решения исчерпываются. 

То, что мы получили - это и есть формула Коши. Она нужна в основном для
всяких теоретических свойств.

\textbf{Пример.} $\frac{dx}{dt}+\frac{x}{t}=t^2$.
Интервал непрерывности - $\mathbb{R}\setminus \{0\}$, поэтому вообще-то
надо рассматривать два интервала. Решение однородного уравнения:
$\frac{dx}{dt}=-\frac{x}{t}$, $x=\frac{C}{t}$. Подумаем, как можно подобрать
частное неоднородного уравнения. Поищем в виде $x=at^3$. Тогда при подстановке
$3at^2+t^2=t^2$, откуда $a=\frac{1}{4}$. Ответ: $x=\frac{t^3}{4}+\frac{C}{t}$.

\subsection{Уравнения, приводящееся к линейному}
Испортрим уравнение \ref{lin_de}, добавив нелинейности:
$$\frac{dx}{dt}+a(t)x=b(t)x^k,~k\in \mathbb{R}\setminus\{0,1\}$$ 
Это - уравнение Бернулли. Если разделим на $x^k$, получим
$$x^{-k} \frac{dx}{dt}+a(t)x^{1-k}=b(t)$$ 
Значит, оно сводится к линейному уравнению заменой $z=x^{1-k}$:
 $$\frac{1}{1-k} \frac{dz}{dx}+a(t)z=b(t)$$ 
Рассмотрим уравнение Риккати:
$$\frac{dx}{dt}+a(t)x=b(t)x^2+c(t),~c(t)\ne 0,c(t)\in C^0(\alpha,\beta)$$ 
В общем виде не решается, но можно частное решение угадать. 
Пусть $x=z+x_p$, где  $x_p$ - частное решение. Получим
$$\frac{dz}{dt}+a(t)z+\frac{dx_p}{dt}+a(t)x_p=b(t)x^2_t+2zx_pb(t)+c(t)$$
Свели к уравнению Бернулли
$$\frac{dz}{dt}+[a(t)-2x_pb(t)]z=b(t)z^2$$ 
Ну зато можно численно и приближенно решать. 

\textbf{Пример (№136).} $xy'-2y=2x^4,~x\ne0$. Разделим на  $x$, свели к
линейному (делить на $x$ можно, ибо  $x$ не является решением): 
$$\frac{dy}{dx}-\frac{y}{x}=2x^3$$
Общее решение неоднородного уравнения:
$$\int\limits_{}^{}\frac{dy}{2y}=\int\limits_{}^{}\frac{dx}{x}$$ 
откуда $y=Сx^2$. Подберем частное решение:  $y=ax^4$. Подставляя в уравнение,
получим  $a=1$, откуда общее решение  $y=x^4+Cx^2$. 

Теперь решим методом Лагранжа. Пусть $y=c(x)x^2$. Имеем
 $c'x^2+2xc-2cx=2x^3$, откуда $c(x)=x^2+C_0$. Значит, ответ  $y=x^4+C_0x^2$.

\textbf{Пример (№149).} $y'=\frac{y}{3x-y^2}$. Приведем к линейному 
(перевернем): $\frac{dx}{dy}=\frac{3x-y^2}{y}$. Общее решение 
однородного уравнения: $x=Cy^3$. Частное решение поищем в виде $x=ay^2$.
Отсюда $a=1$, общее решение  $x=Cy^3+y^2$.

\textbf{Пример (№158).} $2y'-\frac{x}{y}=\frac{xy}{x^2-1}$. Домножим на 
$y$:  $2y'y-x=\frac{xy^2}{x^2-1}$. Замена: $z=y^2$. Тогда уравнение
линеаризуется: 
$$\frac{dz}{dx}-\frac{xz}{x^2-1}=x$$
Общее решение однородного уравнения $z= C\sqrt{x^2+1}$. Метод 
внимательного взгляда: $z=x^2-1$ - частное решение. Итак, ответ:
$z=x^2-1+C\sqrt{x^2+1}$, $y=\sqrt{x^2-1+C\sqrt{x^2+1}}$.

\textbf{Пример (№164).} $(x^2-1)y'\sin y + 2x\cos y=2x-2x^3$. Наша 
нейросетка заметила, что здесь есть 
тригонометрическая замена. Именно, пусть $z=\cos x$. Тогда
$(x^2-1)(-z')+2xz=2x-2x^3$. Делим на 
$x^2-1$ получим однородное.

\textbf{Пример (№163).} $x(e^y-y')=2$. Введем замену $t=e^y$, получаем
 $1-\frac{dt}{dx}\cdot \frac{1}{t^2}=\frac{2}{xt}$. Далее $z=\frac{1}{t}$, 
 и наконец получаем линейное уравнение:
 $$1+\frac{dz}{dx}=\frac{2z}{x}$$

\textbf{Пример (№167).} Уравнение Риккати: $x^2y'+xy+x^2y^2=4$.
Частное решение $y=\frac{a}{x}$. Тогда
$-a+a+{a^2}=4$, $a=\pm2$. Пусть $y=\frac{2}{x}$. Общее решение тогда
$y=z+\frac{2}{x},~y'=z'-\frac{2}{x^2}$. Имеем уравнение Бернулли
$$-z^2=\frac{5z}{x}+z'$$
Сделаем замену $u=\frac{1}{z}$, получим $\int\limits_{}^{}\frac{du}{u} $


%дз 


