\newpage
\section{Уравнения высших порядков}
\subsection{Уравнения, допускаюшие понижение степени}
\textbf{Пример (№451)} $xy^{(4)}=1$. 
Имеем $y^{(3)}=\ln|x|+C_1$. Далее тупа интегрируем:
$$y=\int\limits_{}^{}\int\limits_{}^{}\int\limits_{}^{}(\ln|x|)dx\,dx\,dx=
\frac{C_1x^3}{6}+\frac{C_2x^2}{2}+C_3x+C_4$$
Уравнение четвертого порядка $\Rightarrow$ 4 константы. 

\textbf{№435.} $xy^{(3)}=y^{(2)}-xy^{(2)}$. Идея: самую младшую производную
введем как новую переменную.
Пусть $t(x)=y''$. Имеем  $xt'=t(1-x)$
 $$\frac{t'}{t}=\frac{1-x}{x}$$
 Переменные разделились, значит, $y''=t=C_1xe^{-x}$. Свели к предыдущей 
 задаче - уравнению вида $y^{(n)}=f(x)$, где мы берем интеграл $n$ раз.

\textbf{№431.} $2yy''=y^2+(y')^2$. Замена  $p(y)=y'$. Идея:
в уравнении нет явно независимой переменной, поэтому её роль можно 
передать зависимой переменной, вводя такую замену. 
Тогда имеем $dx=\frac{dy}{p}$, и значит
$y''=\frac{dy'}{dx} = p\frac{dp}{dy}$. 
Уравнение имеет вид
$$2yp\frac{dp}{dy}=y^2+p^2$$ 
Уравнение однородное! Вводим замену $\frac{p}{y}=t(y)$. Получаем
$$\frac{dp}{dy}=\frac{dt}{dy}y+t(y)$$ 
Переменные разделяются:
$$\frac{2{t}dt}{1-t^2}=\frac{dy}{y}$$ 
Получаем ответ: $\frac{dy}{dx}=\frac{p}{y}=t(y)=\pm\sqrt{1-\frac{C_1}{y}}$. 
Теперь переменные снова разделились.

\textbf{№455.}  $yy^{(3)}+3y'y''=C$. Поделим на $yy''$:
$$\frac{y'''}{y''}=3 \frac{y'}{y}=0$$ 
Эти штуки похожи на производная логарифма, поэтому можем записать так:
$$\frac{d}{dx}\left( \ln|y''|+3\ln|y| \right)=0$$ 
Значит, $\ln|y''|+3\ln|y|=C_1$. Имеем $\ln|y''\cdot y^3|=C_1$,
$y''\cdot y^3 = C_1$. Сведем задачу к предыдущей заменой 
$p(y)=\frac{dy}{dx}$. Уравнение получаем такое:
$$pdp=\frac{dy}{y^3}C_1$$
Отсюда $p=\pm\sqrt{\frac{1}{y^2}C_1+C_2}=\frac{dy}{dx}$.
Переменные разделились.\\
Ответ: $y=0,y=C_1x+C_2$ - не забудем про потерянные решения, когда делили на 
$yy''$; $y=\pm\sqrt{\frac{1}{y^2}C_1+C_2}$. 

\textbf{№460.} $y''=xy'+y+1$. Полный букет всего. Единственная надежда -  
найти че-то типа первого интеграла: а именно, найти функцию от производных
меньшего порядка, чья производная зануляется. В данном случае имеем
$$\frac{d}{dx}\Big(y'-xy-x\Big)=0$$ 
откуда $y'-xy-x=C_1$. Линейное уравнение:
$y'=C_1+x(y+1)$. Не ну такое мы умеем решать))))))

\textbf{№463.} $xyy''-x(y')^2=yy'$. Это уравнение, однородное относительно
группы $y$ (перейдет в себя при замене $y^{(l)}\mapsto ky^{(l)}~\forall  l$).
Замена: $t(x)=\frac{y'}{y}$. Тогда имеем
$$xy(t'y+t^2y)-xt^2y^2=y^2t$$ 
Поделив на $y^2$, получим
$$xt'=t$$
Значит, $\frac{y'}{y}=t=C_1x$. Ну и дальше все понятно. 


\textbf{№468.} $y''=\frac{y'}{x}+\frac{y}{x^2}=\frac{y'^2}{y}$ - 
однородное. 

\textbf{№477.} $x^2(2yy''-y'^2)=1-2xyy'$.
Обобщенно-однородое уравнение - которое не меняется при замене
$x\mapsto kx,y^{(l)}\mapsto ky^{(l)}~\forall  l$. Тогда можно ввести замену
$\begin{cases}
    x=\pm e^t\\y=z(t)e^{mt}
\end{cases}$
Проверим это уравнение на обобщенную однородность: сделаем соответствующую
замену (считая $1=k^0$). Мы вынесем $k^m$, откуда  $2m=0$. Значит,
$x=\pm e^t,y=z(t)$. Имеем 
$$y'=\frac{dy}{dx}=\frac{dz}{dx}\frac{dt}{dt}=\frac{\dot z}{\frac{dx}{dt}}=
\frac{\dot z}{\pm e^t}$$
$$y''=\frac{dy' / dt}{dx / dt}=\frac{\ddot ze^{-t}-\dot ze^{-t}}{e^t}$$ 
Значит, уравнение перепишется в виде
$$???????e^{-t}\left( 2 z\cdot
\frac{\ddot ze^{-t}-\dot ze^{-t}}{e^t} - \frac{\dot z^2}{e^{2t}} \right) =
1\pm 2ze^t \frac{\dot z}{\pm e^t}
$$
Поделим на $2e^t$, получим
$$(2z(\ddot z-\dot z)-\dot z^2)=1-2z\dot z$$
которое сводится к $2z\ddot z - \dot z^2 =1$. Дорешать заменой  $p(z)=\dot z$.

Дз: 452, 454, 426, 459, 468, 480 

\subsection{Теория}
\begin{defin}
Уравнение высшего порядка - уравнение вида
\end{defin}
\begin{equation} \label{uvp}
    F\bigg(t,x,\frac{dx}{dt},...,\frac{d^{n}x}{dt^n}\bigg)=0
\end{equation}
Его решение: $x=x(t,C_1,...,C_n)$. 
Если $\frac{\partial F}{\partial x^{(n)}}\ne 0$, то уравнение \ref{uvp}
приводится к виду
\begin{equation}\label{uvp_1}
    \frac{d^nx}{dt^n}=f\bigg(t,x,...,\frac{d^{n-1}x}{dt}\bigg)
\end{equation}
Уравнение первого порядка задает слоение с особыми точками. Но для высших
порядков нет такой картинки, там куча параметров. Чтобы увидеть естественную
картинку, нам потребуется $n$-мерное пространство параметров. Чтобы 
отыскать единственное решение, проходящее через точку этого пространства, 
нам надо зафиксировать  $n$ начальных условий
\begin{equation}\label{uvp_usl}
 \begin{cases}
     x(t_0)=x_0\\ \dot x(t_0)=\dot x_0\\...\\x^{(n-1)}(t_0)=x_0^{(n-1)}
 \end{cases}
\end{equation}
\begin{defin}
    Задача Коши - уравнение \ref{uvp_1} + условия \ref{uvp_usl}
\end{defin}
Пусть решение задачи Коши единственно. Это означает, что каждая константа
$C_i$ зависит от набора начальных условий 
$t,x_0,\dot x_0,...,x_0^{(n-1)}$.
Запишем систему:
\begin{equation*}
\begin{cases}
  x=x\Big(t,C_1(t,x_0,\dot x_0,...,x_0^{(n-1)}),...,C_n(t,x_0,\dot x_0,...,x_0^{(n-1)})\Big) = 
    x(t,x_0,\dot x_0,...,x_0^{(n-1)})\\
    \ddot x = \ddot x(t,x_0,\dot x_0,...,x_0^{(n-1)})\\
    ...\\
    x^{(n-1)}=x^{(n-1)}(t,x_0,\dot x_0,...,x_0^{(n-1)}) \end{cases}  
\end{equation*}
Мы взяли $n+1$ переменную и связали их  $n$ уравнениями. Значит, эта система
определяет кривую в расширенном фазовом пространстве, проходящую через точку
начальных условий \ref{uvp_usl}. Эта кривая - интегральная кривая - то же, 
что и решение уравнения. Можно сказать, что она одномерна в силу того, что
функция $x$ на самом деле функция одной переменной  $x=x(t)$. 
\begin{defin}
Расширенное фазовое пространство - пространство независимой, зависимой 
переменной и её производных.
\end{defin}
\begin{theor}
(Коши-Пикара)\\
Если в уравнении \ref{uvp_1}  $f,\frac{\partial f}{\partial x},...,
\frac{\partial f}{\partial x^{(n-1)}}$ непрерывны в области $D\subset 
\mathbb{R}^n=(t,x,\dot x,...,x^{(n-1)})$, и точка 
$M(t_0,x_0,...,x^{(n-1)}_0)\in D$, тогда решение задачи Коши существует
и имеет единственное решение.
\end{theor}
\textbf{Доказательство.}  
$\square$ \\

\subsection{Уравнение в нормальной форме}
Введем вектор-функции 
\begin{equation*}
    \overline{x}(t)=\begin{pmatrix} x_1(t)\\...\\x_n(t)\end{pmatrix} 
\end{equation*}
Зададим дифференцирование по определению:
\begin{equation*}
\overline{\dot x}(t)=\begin{pmatrix}\dot x_1(t)\\...\\ \dot x_n(t)\end{pmatrix}
\end{equation*}
На основе \ref{uvp_1} введем векторы
$$X(t)=\begin{pmatrix} x(t)\\...\\ x^{(n-1)}(t) \end{pmatrix},a~
F(t,x)=\begin{pmatrix} \dot x(t)\\...\\ x^{(n)}(t) \end{pmatrix} = 
\begin{pmatrix} f_1(t,x_1,...,x_n)\\...\\f_n(t,x_1,...,x_n)\end{pmatrix}$$
\begin{defin}
    Уравнение в нормальной форме (в форме Коши) - 
    уравнение \ref{uvp_1} в векторном виде: 
\end{defin}
\begin{equation}
    \dot X = F(t,x),~X\in \mathbb{R}^n
\end{equation}
В частности, для обычного уравнения 
$$\begin{cases}
    \dot x_1 = x_2\\ \vdots \\ \dot x_n = f(t,x_1,...,x_n)
\end{cases}$$




