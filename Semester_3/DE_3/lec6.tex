\section{Уравнения и ряды Тейлора}
Пусть $\frac{dx}{dt}=f(t,x)$. Рассмотрим $x(t_0)=x_0$. Разложим в ряд
Тейлора: $x(t)=x(t_0)+\frac{dx}{dt}(t_0)(t-t_0)+o(t-t_0)$.
Отбросив члены высшего порядка (прямо как топовые физики), получим 
приближенное решение. Приближенные решение можно итерировать, и это
будет широко известный \textbf{метод Эйлера} (первого
порядка). $t_{k+1}=t_k+h,~x_{k+1}=x_k+f(t_k,x_k)h$ 

\subsection{Практика}
\textbf{Пример (№111)}. $(y+\sqrt{xy})dx=xdy$. Уравнение однородно (
проверим умножением на $k$). Значит, делаем замену $u(x)=\frac{y}{x}$.
Имеем $dy=u\cdot dx+du\cdot x$. Переменные разделяются: 
$\frac{dx}{x}=\frac{du}{\sqrt{u}}$\\
\textbf{Пример (№113)}. $(2x-4y+6)dx+(x+y-3)dy$. Переносим начало координат
в точку пересечения.\\
\textbf{Пример (№126)}. $y'=y^2-\frac{2}{x^2}$. Это - обобщенно-однородное
уравнение, то есть приводится к однородному заменой $y=z^m(x)$.
$y'=mz^{m-1}z$ Далее
$mz^{m-1}z=z^{2m}-\frac{2}{x^2}$ 
Теперь уравнение однородно. Введем замену $\frac{z}{x}=u,~z=ux$.
Получим $u'x+u=-1+2u^2$\\
\textbf{Пример (№128)}. $\frac{2}{3}xyy'=\sqrt{x^6-y^4}+y^2$. 
Пусть $y=z^m$. Идея: сделать так, чтобы под корнем степень у $x$ и $y$ была
одинаковой.\\
\textbf{Пример (№)} $2xydx+(x^2-y^2)dy=0$. Подберем функцию, полным 
дифференицалом которого является это выражение; получим  $F(x,y)=
x^2y-\frac{1}{3}y^3$. Решние: $F=C=const$\\
\textbf{Пример (№192)}. $(1+y^2\sin{2x})dx-2y\cos^2{x}dy$. Мы должны 
показать, что вторые производные равны. Тогда это значит, что
$F_{xy}=F_{yx}$, и такая функция вообще существует на некотором диске
(где правая часть не обращается в ноль). Интегируем два раза, и найдем эту
функцию: $F(x,y)=x-y^2 \frac{1}{2}\cos{2x}-\frac{y^2}{2}+C_0$.
Итак, ответ: $\boxed{F=const}$ \\
\textbf{Пример (№202)}. $y^2dx+(xy+\tg{xy})dy=0$. Является ли однородным,
в полных дифференциалах? Давайте раскроем скобки и сгруппируем:
$y(ydx+xdy)+\tg{xy}dy$. Это то же, что и  $\frac{d(xy)}{\tg{xy}}+\frac{dy}{y}
=0$. Домножим на $\frac{1}{y\tg{xy}}$ и хаваем уравнение в полных 
дифференицалах бесплатно. То, на что домножили - интегрирующий множитель.














