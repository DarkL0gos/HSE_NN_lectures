\section{Геометрические задачи, из которых возникают ДУ}
\textbf{Пример (№17).} Составим уравнение по решению: $y=e^{cx},~y'=ce^{cx}$.
Имеем $c=\frac{\ln{y}}{x}$, значит, $y'=\frac{\ln{y}}{x}e^{\ln{y}}$.\\
\textbf{Пример (№25).} Дано семейство функций $y=ax^2+be^x$, $y'=2ax+be^x$,
$y''=2a+be^x$. Найдем ДУ, решениями которого они являются. Так как у нас два 
параметра: $a$ и  $b$, то и уранвение будет второго порядка. Имеем
$$y-y''=2a(x-1)\implies a=\frac{y'-y''}{x-1}$$ 
$$y''=\frac{2(y'-y'')}{2(x-1)}+be^x\implies \frac{1}{e^x}(y''-
\frac{y'-y''}{x-1})=b$$ 
$$y=\frac{y'-y''}{2(x-1)}x^2+(y''-\frac{y'-y''}{x-1})$$
Возникает вопрос: а единственно это решение? Здесь мы пользуемся теоремой о 
неявной функции.\\
\textbf{Пример (№30).} Составим уравнение для окружностей, центры которых 
лежат на $y=2x$. Уравнение окружностей  $(x-x_0)^2+(y-2x_0)^2=1$. Ответом 
должно быть однопараметрическое семейство решений, которые соответствуют 
различным положениям центра на прямой. Дифференцируем:
$$2(x-x_0)+2(y-2x_0)y'=0\implies x_0=\frac{x+yy'}{1+2y'}$$
Подставим выражение для параметра обратно в уранвение:
$$(x-\frac{x+yy'}{1+2y'})^2+(y-2\frac{x+yy'}{1+2y'})^2=1$$
\textbf{Пример (№71).} Найдем кривые, касательные которых заметают 
одинаковые площади под своим графиком. Пусть $f(x)=y$ - искомая кривая. 
Её производная не может быть нулевой, иначе она не образует треугольник
с осью абсцисс. \\
Фикисруем точку  $x_0$.  Получаем условие: $\frac{y^2(x_0)}{2y'(x_0)}=a^2
\implies y'=\frac{y^2}{2a^2}$. Если производная отрицательная, то в этой 
формуле должен вылезти минус (и формально мы имеем два случая, поэтому 
$$y'=\pm\frac{y^2}{2a^2}$$
Проинтегрируем (переменные разделяются):
$\frac{1}{y}=\pm \frac{1}{2a^2}x+C$
Итак, $$y=\frac{2a^2}{2a^2C\pm x}$$
\textbf{Пример (№73).} Ещё одна геометрическая задачка. Беглый анализ: 
производная не равна нулю. Уравнение касательной: $y=y'(x_0)(x-x_0)+y_0$.
Точка пересечения с осью абсцисс: $x_k=\frac{-y_0}{y'(x_0)}+x_0$.
Уравнение нормали: $y=-\frac{1}{y'}(x-x_0)+y_0$. Точка пересечения нормали с
осью абсцисс: $x_n=y_0y'+x_0$. Диффур снова распадается на два случая...
$|KN|=|x_k-x_n|=|\frac{y}{y'}$. Рашаем дома кароч. 
\subsection{Мини-рассказ про число е}
Архимед в общем-то знал, что при умножении показатели степеней складываются. 
Это легко получить из анализа обычной геометрической пргрессии. В XV веке
начали торговать, используя сложные проценты. Возник вопрос, можно ли 
полутать бесконечное количество денег при уменьшении периода факторизации. 
Какой-то челик (Саймон вставить фамилию) решил написать таблицу сложных 
процентов, чтобы полутать денег с её использования, и оказалось, что ответ на
предыдущий вопрос отрицательный. Иоста Бюрге (помощник Кеплера) посмотрел
на таблицы и полутал с них инфу о том, что с их помощью можно перемножать 
огромные числа. Джон Непер составил более юзабельные таблицы, ввел понятие
логарифма, и кароч дальше вводим предел для натуральных чисел, переходим к 
непрерывной хрени... Теперь фокус: $e^k=\lim\limits_{n \to \infty}(1+
\frac{1}{n})^{nk}=(1+\frac{k}{m})^m=\sum\limits_{i=0}^{m}C^i_m (\frac{k}{m})
^i=\sum\limits_{i=0}^{m} \frac{k^i}{i!}$. Эту хрень придумал Бернулли, 
и она сходится к $e$ быстрее обычного предела. Можно это положить за 
определение $e^x$, и мгновенно распространить на любые действительные 
показатели степеней.


%ДЗ: №2, №18, №26, №32, №33, №72, №74, №76 - составить (и решить) диффур
% + последняя хрень с практики
