\section{Однородное уравнение}
$$\frac{dx}{dt}=f\left(\frac{x}{t}\right)$$ 
Как искать его решение? Заменой $u(t)=\frac{x}{t}$. 
Тогда уравнение перепишется в виде $\frac{dx}{dt}=\frac{du}{dt}t+u$.
В нем переемнные разделяются: $\frac{du}{f(u)-u}=\frac{dt}{t}$. 
Итак, типы уравнений:
\begin{enumerate}
    \item С разделяющимися переменными
    \item Приводящиеся к виду $\frac{dx}{dt}=f(ax+bx+c)$ 
    \item Првиодящиеся к виду $(a_1x+b_1t+c_1)dx+(a_2x+b_2x+c_2)dt=0$
\end{enumerate}
Подумаем, можно ли это последнее привести к однородному. Добавим условие 
$c_1^2+c_2^2\ne0$ (иначе система уже однородна). В общем, если эти 
две прямые пересекаются в точке
$(x_*,t_*)$, то можно ввести новые переменные, передвинув эту точку в начало
координат: $x\mapsto x-x_*,~t\mapsto t-t_*$. Тогда система перепишется 
без $c_1,~c_2$, и таким образом будет однородной. Если прямые не пересекаются, 
то прямые либо совпадают, либо параллельны. Тогда введем замену (для любой 
прямой) $z(t)=a_1x+b_1t+c_1$. Так как прямые параллельны, то
$\frac{a_1}{a_2}=\frac{b_1}{b_2}=k$, значит, мы можем выразить 
вторую прямую: $a_2x+b_2t+c_2=\frac{1}{k}(a_1x+b_1t+kc_2)=\frac{1}{k}(z-
c_1+kc_2)$. Уравнение приводится к виду $z(t)dx+\frac{1}{k}(z-c_1+kc_2)dt=0$.
Но у нас все равно многовато переменны. Выразим $dx$ через  $z$:
$$z(\frac{dz-b_1dt}{a_1})+\frac{1}{k}(z-c_1+kc_2)=0$$ 
Умножим на $a_1k$:
 $$kzdz=kb_1zdt-a_1zdt-a_1(kc_2-c_1)dt$$ 
Домножим на $\frac{1}{kzdt}$:
$$\frac{dz}{dt}=((b_1-\frac{a_1}{k})z-a_1(kc_2-c_1))\frac{1}{z}$$ 
Finally, уравнение с разделяющимися переменными! ПОБЕДА!
\subsection{Обобщенно-однородное уравнение}
\begin{defin}
Обобщенно-однородное уравнение - уравнение вида
$$M(x,t)dx+N(x,t)dt=0$$ 
причем $M,N$ - такие. что  $\exists n\in\mathbb{R}$: если
$x=z^n(t)$, то уравнение  $M(z^n,t)nz^{n-1}dz+N(z^n,t)dt=0$ однородно.
\end{defin}
\textbf{Пример.} Испортим однородное уравнения, чтобы сделать его 
обощенно-однородным. Роман придумал, чел харош.

Сведем и этого зверя к разделяющимся переменным. 
$$\begin{cases}
    n(kz)^{n-1}M((kz)^n,kt)=k^mM(z^n,t)nz^{n-1}\\
    N((kz)^n,kt)=k^mN(z^n,t)
\end{cases}$$ 

\subsection{Уравнение в полных дифференциалах}
Напомним, что полный дифференциал $dF(x,y)$  $C^1$-гладкой функции
равен  $\frac{\partial F}{\partial x} dx+\frac{\partial F}{\partial y}dy$.
\begin{defin}
Уравнение в полных дифференциалах - уравнение вида
$$dF(x,y)=0,~F\in C^2(\Omega),~\Omega\subset \mathbb{R}^2$$
\end{defin}
Если мы знаем саму функцию, то решение находится мгновенно: $dF(x,y)=const$.
Правда, оно неявное. Выразим  $y=y(x)$ по теореме о неявной функции.\\
\textbf{Пример.} $x^2\sin{t}dt+2x\cos{t}dx=0$\\
Уравнение является уравнение в полных дифференциалах, если существуют такие
функции, что $M=\frac{\partial F}{\partial x},~N=\frac{\partial F}{\partial y}$
\begin{theor}
    (необходимое условие представления в полных дифференциалах)\\
    $$\frac{\partial M}{\partial y}=\frac{\partial N}{\partial y}$$ 
    Достаточное условие - $M_y=N_x$ в односвязной области
\end{theor}
\textbf{Доказательство.} $\square$ \\
Как подбирать такие функции? Мы знаем, что $\frac{\partial F}{\partial x}=
M(x,y)$. Проинтегрируем это равенство по $x$. Имеем  $F=\int M(x,y)dx+
\varphi(y)$. Проделаем то же самое по переменной $y$:  $\frac{\partial F}
{\partial y}=\frac{\partial }{\partial y}(\int M(x,y)dx)+\varphi'=N(x,y)$,
откуда $\varphi=\int\left(N-\frac{\partial }{\partial y}(\int Mdx) \right)dy$.
Чтобы проверить себя при решении, помним, что $\varphi$ не зависит от $x$!
Итак,
$$F=\int M(x,y)dx+\int\left(N-\frac{\partial }{\partial y}\left(\int Mdx
\right)\right)dy$$
\subsubsection{Геометрический смысл решения уравнения в полных дифференциалах}
Так как $z=z(x,y)$ - какая-то поверхность, то запись  $z=const$ - это линии
уровня, которые можно спроецировать на плокость переменных и получить 
интегральные кривые.\\
\textbf{Пример (модель Лотки-Вольтерра).} Пусть $x(t)$ - плотность карасей,
 $y(t)$ - плотность щук в некотором пруду. Щуки сдерживают рост карасей,
но от количества карасей зависит также и количество щук. Запишем систему:
$$\begin{cases} \label{lotka_volterra}
    \dot{x}=x(a-by)\\
    \dot{y}=y(-c+ex)
\end{cases}$$
Лотка придумал эту систему для биоценозов, а Вольтерра - для химических
реакций.\\
Давайте решим эту систему. Её расширенное фазовое пространство, вообще говоря,
трехмерное, поэтому будем рассматривать фазовые кривые - проекции интегральных 
на плоскость независимых параметров. Они ориентированы в направлении роста
параметра $t$. Найдем эти кривые, найдя решение уравнения 
$\frac{dy}{dx}=\frac{dy /dt}{dx /dt}=\frac{-cy+exy}{ax-bxy}$. 
Переменные разделяются: $$\frac{(a-by)dy}{y}=\frac{(-c+ex)dx}{x}$$ 
Представим его в полных дифференциалах: 
$$d\left( a\ln y-by+c\ln x -ex \right)=0$$
Значит, решение имеет вид $a\ln y-by+c\ln x-ex=h=const$.
Выглядит очень сложно, но давайте попробуем построить изолинии. 
Введм функцию $F=\ln{(y^ax^c)}-by-ex$, и поищем её изолинии. Сначала найдем
критические точки: $(x_*,y_*)=(\frac{c}{e},\frac{a}{b})$ (получилась 
единственная точка). Определим тип критической точки (составим гессиан, 
посчитаем его знакоопределенность); получим, что это точка максимума.
Линии уровня - какие-то окружности/эллипсы.\\
\begin{figure}[H]
    \centering
    \input{figures/Lotka-Volterra.pdf_tex} %для pdf_tex
    %\includegraphics{}  % для png, pdf
    \caption{Первый интеграл системы Лотки-Вольтерры}
    \label{fig:}
\end{figure}
\textbf{Упражнение.} Доказать, что фазовые кривые замкнуты.

\textbf{Доказательство (\cite{Arnold},\S 2.7).} Так как 
$\frac{dF}{dt} = 0$, то функция $F$ сохряняется вдоль фазовых кривых. 
Иначе говоря, фазовые кривые являются изолиниями фунции $F$. Но 
график  $F$ является суммой двух <<холмов>>, образованных логарифмами, 
и поэтому сам является холмом. Посколько изолинии холма - замкнутые кривые,
то и фазовые кривые системы Лотки-Вольтерры замкнуты. $\square$\\
Теперь нам надо понять, куда закручиваются эти линии, как они ориентированы.
Они закручиваются против часовой стрелки вокруг критической точки, кстати,
область решения - первая координатная четверть. Чтобы избежать проблем с 
дискретностью, наши переменные - это плотность населения пруда. 

 



%Гомеоморфный (без самопересечений, так как биекция) образ окружности в
%плоскости можно непрерывно продеформировать в точку. Теорема Жордана: 
%гомеоморфный образ окружности делит плоскость на две компоненты связности. 
%Теорема Шёнфлиса: внутренняя часть этой плоскости гомеоморфна открытому 
%диску. Это кстати эквивалентно тому, что фундаментальная группа 
%тривиальна. 
