\section{Численные методы}
Сегодня поговорим о численных методах решения дифференциальных уравнений.
Именно, задача Коши
$$\begin{cases}
    \frac{dx}{dt}=f(t,x)\\x(t_0)=x_0
\end{cases}$$ 
имеет решение какое-то. 
Поскольку 
$$x_{k+1}=x(t_{k+1})=x(t_k+h)=x(t_k)+\frac{dx}{dt}\big|_{t_k}h+
\frac{d^2x}{dt^2}\big|_{t_k}\frac{h^2}{2}+o(h^3)$$
Если мы рассмотрим конечные приращения $h$, то 
получим  итеративную формулу 
$$x_{k+1}=x_k+f(t_k,x_k)h$$ 
- метод Эйлера первого порядка, основанный на разложении функции в ряд Тейлора
и отбрасывании членов высшего порядка. Таким образом, можно рассмотреть
более точные методы, основынные на использовании членов высшего пордяка.
Например, $\frac{d^2x}{dx}=\frac{d}{dt}\left( \frac{dx}{dt} \right)=
\frac{f(t_k+h,x_k+f(t_k,x_k)h)-f(t_k,x_k)}{h}$. Из этого мы получим 
метод Штермера. И так далее. Метод  Рунге-Кутты - 4го порядка.\\
\textbf{Пример.} Супер-простая функция:
$$\begin{cases}
    \frac{dx}{dt}=x\\x(0)=1
\end{cases}$$
Это - определение обычной экспоненты. Решим методом Эйлера. Возьмем
$t_1=h$. Далее,  $x_1=1+f(t_0,x_0)\cdot h=1+h$. Далее,
 $t_2=t_1+h=2h$, $x_2=x_1+f(t_1,x_1)\cdot h=1+h+(1+h)h=(1+h)^2$.
 В общем виде, в точке $x_{k+1}=x_k(1+h)=(1+h)^{k+1}$ 
 функция будет принимать значение $x_n=(1+h)^{n}=\left( 1+\frac{T}{n}\right)^n
\to e^T$.
Неслучайно тут вылез замечательный предел - определение экспоненты.
\section{Существование и единственность решения}
Эту задачу можно решить с помощью ф.п. Пикара. Доспустим, у нас есть 
решение задачи Коши в виде непреывной функции $f(t,x)$. Тогда мы можем
проинтегрировать:
 $$\int \frac{dx(t)}{dt} \equiv \int f(t,x)dt$$ 
 Справа стоит набор первообразных:
 $$x\equiv x_0+\int\limits_{t_0}^{t}f(\tau,x(\tau))d\tau$$ 
\begin{theor} (лемма)\\
    Задача Коши эквивалента решению интегрального уравнения
\end{theor}
\textbf{Доказательство.} Пусть $x(t)$ - решение задачи Коши. Тогда
при подставновке в уравнение имеем 
$$\frac{dx(t)}{dt}\equiv f(t,x(t))$$ 
Интегрируя, получим 
$x(t)=x_0+\int\limits_{t_0}^{t}f(\tau,x(\tau))d\tau$ - решение интегрального
уравнения.\\
Обратно, пусть $x(t)$ - непрерывное решение интегрального уравнения.
Тогда, взяв производную, получим
$$\frac{dx}{dt}=f(t,x(t))$$ 
Подставив $t=t_0$ в интегральное уравнение, получим  $x(t_0)=x_0$, т.е.
$x(t)$ - решение д.у. $\square$ \\
На самом деле, это обман, так как мы прсото записали в другом виде все так 
же не решаемую задачу. Запишем последовательность Пикара $\{x_k(t)\}$:
$x_0(t)=x_0$,  $x_{k+1}(t)=x_0+\int\limits_{t_0}^{t}f(\tau,x_k(\tau))d\tau$.
Теперь нам надо бы доказать, что эта последовательность сходится к 
решению. Хм, где же нас обманули... 
\begin{theor} (Коши-Пикара, или о существовании и единственности задачи 
Коши)\\
Пусть $f(t,x),~\frac{\partial f}{\partial x}$ непрерывны в области $D\subset 
\mathbb{R}^2$. Тогда для любой точки $(t_0,x_0)\in D$ существует решение
$x(t)$ задачи Коши, определенное на отрезке $I_\delta=[t_0-\delta,t_o+\delta]$,$\delta=\frac{r}{\sqrt{1+m^2}}$, где $r>0$ такое, что  $B_r\subset D$
(замкнутый шар), $m=\max|f(t,x)|,~(t,x)\in B_r$. Кроме того, если
$\tilde x(t)$ - другое решение задачи Коши, определенный на интервале
$[t_0-\tilde\delta,t_0+\tilde\delta]$, то существует такое
$\delta^*\in (0,\min(\delta,\tilde\delta))$, что $x(t)=\tilde x(t)$ 
для $t\in [t_0-\delta^*,\delta^*]$.
\end{theor}
\textbf{Доказательство.}  Докажем, что последовательность Пикара корректно
определена и её предел - непрерывная функция. Именно, каждый раз, когда
мы вычисляем $x_k$, она непрерывна и не выходит за пределы области  $D$,
и поэтому снова интегрируема. \\
Рассмотрим $$|x_1(t)-x_0(t)|=
\left| \int\limits_{t_0}^{t}f(\tau,x_0)d\tau\right|\leqslant 
\left|\int\limits_{t_0}^{t}m\,d\tau\right|\leqslant m|t-t_0|\leqslant
m\frac{r}{\sqrt{1+m^2}}\leqslant r$$ - значит, график функции лежит в 
$B_r$, пока  $t\in I_\delta$. По индукции доказывается, что 
$|x_k-x_0|\leqslant r$, значит, все эти приближения лежат в шаре и непрерывны.
Потому что у первообразной есть производная, значит она непрерывна. 
Итак, последовательность Пикара корректно определена и её члены - непрерывные 
функции.

Докажем, что последовательность сходится. Рассмотрим ряд $x_0(t)+x_1(t)-x_0(t)\
+x_2(t)-x_1(t)+...+x_k(t)-x_{k-1}(t)+...$. Частичные суммы $S_n$ 
этого ряда сумма этого ряда как раз равны $x_n$. Если мы докажем, что если
ряд сходится равномерно, то и последовтаельность Пиакара имеет непрерывный 
предел. 
Имеем $|f(t,x)-f(t,\tilde)|\leqslant L|\tilde x-x|$, где 
$L=\max\left|\frac{\partial f}{\partial x}(t,x)\right|,~(t,x)\in B_r$. 
Итак, $|x_1-x_0|\leqslant m|t-t_0|$. Далее
$|x_2-x_1|\leqslant \left| \int\limits_{t_0}^{t}(f(\tau,x_1(\tau))-
f(\tau,x_0))d\tau \right|\leqslant 
\left| \int\limits_{t_0}^{t}\left|f(\tau,x_1(\tau))-f(\tau,x_0))\right| 
d\tau \right| \leqslant 
\left| \int\limits_{t_0}^{t}L|x_2(\tau)-x_1|d\tau\right|\leqslant 
L\left| \int\limits_{t_0}^{t}\frac{Lm|x_2(\tau)-x_1|^2}{2}d\tau\right|
\leqslant \frac{L^3m|t-t_0|^3}{6L}$. То есть, 
$|x_n-x_{n-1}|\leqslant \frac{m}{L} \frac{L^n(t-t_0)^n}{n!}\leqslant 
\frac{m}{L}\frac{L^n\delta^n}{n!}$.
Где-то тут мы ссылаемся на теорему Лагранжа о среднем. 
Итак, мы оценили ряд из модулей сходящимся
числовым рядом, следовтаельно, по признаку Вейерштрасса сумма ряда сходится
равномерно: 
$$\frac{m}{L}\sum\limits_{n=1}^{\infty}\frac{L^n\delta^n}{n!}\rightrightarrows
\frac{m}{L}\left( e^{L\delta}-1 \right)$$
Обозначим $\lim\limits_{n \to \infty} x_n=x^*(t)$. 
Тогда $|x^*(t)-x_k(t)|\to 0$. Теперь рассмотрим разницу
$\left| \int\limits_{t_0}^{t} f(\tau,x^*(\tau))d\tau-\int\limits_{t_0}^{t}
f(\tau,x_k(\tau))d\tau\right|\leqslant \int\limits_{t_0}^{t}L|x^*-x_k|d\tau$.
Правая часть стремится к нулю, значит, и левая тоже. Поэтому, переходя к
пределу  по $t$ при  $k\to \infty$ в формуле 
$x_{k+1}(t)=x_0+\int\limits_{t_0}^{t}f(\tau,x_k(\tau))d\tau$, имеем
$x^*=x_0+\int\limits_{t_0}^{t}f(\tau,x^*(\tau))d\tau$, то есть $x^*$ - 
решение интегрального уравнения, а значит и задачи Коши. 

Теперь докажем единственность. Пусть на интервале  $I^*_\delta$ определено
два решения, $x$ и  $x^*$. Тогда
$|x^*(t)-x(t)|\leqslant \left| \int\limits_{t_0}^{t}f(\tau,x^*(\tau))-
f(\tau,x(\tau))) d\tau \right|\leqslant L \int\limits_{t_0}^{t}|x^*(\tau)-
x(\tau)|d\tau$. Пусть $t>t_0$. Положим  
$\Delta= \int\limits_{t_0}^{t}|x^*(\tau)-x(\tau)|d\tau$, тогда 
$\frac{d\Delta}{dt}\leqslant L\Delta$; $\Delta(t_2)\geqslant\Delta(t_2)$
для всех $t_2>t_2\geqslant t_{0}$. 
Кстати, $\frac{d\Delta}{dt}\leqslant L\Delta$
Значит, существует инфинум 
$T=\inf \{t\geqslant t_0\}$. Рассмотрим случай, когда $T=t_0$ (самая жесткая
оценка). Если это так, то пусть $а(t_0+\varepsilon)=\Delta_\varepsilon$. 
Тогда $\Delta_\varepsilon>0$. Поставим задачу Коши:
$$\begin{cases}
    \frac{d\Delta}{dt}=L\Delta\\ \Delta(t_0+\varepsilon)=\Delta_\varepsilon
\end{cases}$$
Отсюда $\Delta=\Delta_\varepsilon e^{L(t-t_0-\varepsilon)}$. Для всех
$t>t_0$,  $\Delta(t)\leqslant\Delta_\varepsilon e^{L(t-t_0-\varepsilon)}$.
Устремим $\varepsilon\to0$. Тогда и $\Delta(t)=0$ в пределе. 
Рассуждение при $t<t_0$ аналогично. $\square$\\
\textbf{Пример.}  Что можно сказать о решении задачи Коши для
$$\begin{cases}
    \frac{dx}{dt}=|x|\\x(0)=x_0
\end{cases}$$
Теорема Коши-Пикара не работает в нуле, так как там функция не дифференцируема.
Но не обманывают ли нас?
$| |x_1|-|x_2| |\leqslant 1\cdot |x_1-x_2|$. Модуль - липшицева функция, 
поэтому условия теоремы работают. 
А если 
$$\begin{cases}
    \frac{dx}{dt}=\sqrt{x} \\x(0)=x_0
\end{cases}$$
Производная растет неограниченно, функция не липшецева: 
$|\sqrt{x_1}-\sqrt{x_2}|=\frac{|x_1-x_2|}{\sqrt{x_1}+\sqrt{x_2}}\leqslant L
|x_1-x_2$ - при приближении к нулю $L\geqslant\frac{1}{\sqrt{x_1}+\sqrt{x_2}}$.
Но это только в нуле. А не в нуле можно $\Rightarrow$ решение сущетсвует и 
единственно. В общем, давайте зарешаем. Получаем $x=\frac{(t+C)^2}{4},~t+C>0$.
По условию $x(0)=0$, откуда $x=\frac{t^2}{2}$. Но ведь ещё есть куча решений
типа $x(t_0)=0$,  $x=\frac{(t-t_0)^2}{4}$, и даже $x=0$. 




