\subsection{Численные методы}
%Пусть $\frac{dx}{dt}=f(t,x)$. Рассмотрим $x(t_0)=x_0$. Разложим в ряд
%Тейлора: $x(t)=x(t_0)+\frac{dx}{dt}(t_0)(t-t_0)+o(t-t_0)$.
%Отбросив члены высшего порядка (прямо как топовые физики), получим 
%приближенное решение. Приближенные решение можно итерировать, и это
%будет широко известный \textbf{метод Эйлера} (первого
%порядка). $t_{k+1}=t_k+h,~x_{k+1}=x_k+f(t_k,x_k)h$ 
%Сегодня поговорим о численных методах решения дифференциальных уравнений.
Рассмотрим задачу Коши для уравнения, разрешенного относительно производной:
$$\begin{cases}
    \frac{dx}{dt}=f(t,x)\\x(t_0)=x_0
\end{cases}$$ 
Будем искать его приближенное решение, перейдя к дискретному времени:
именно, возьмем малое приращение времени $h$ и рассмотрим последовательность
$t_0,t_0+h,t_0+2h,...,t_0+kh,...$. Тогда уравнение можно переписать в виде 
$$\frac{x_{k+1}-x_k}{h}=f(x_k,t_k),~(\textit{где } x_i=x(t_i))$$ 
откуда имеем итеративную формулу
$$x_{k+1}=x_k+f(t_k,x_k)h$$ 
Мы получили знаменитый \textbf{метод Эйлера} численного интегрирования
уравнения. Его геометрический смысл прост: начиная из точки 
$(t_0,x_0)$, мы в течение времени $h$ двигаемся вдоль касательной к 
интегральной кривой. Попав в новую точку $(t_1,x_1)$, мы снова движемся вдоль 
касательной к кривой, проходящей через новую точку, и так далее. Поэтому 
метод Эйлера ещё называют методом ломаных.

Другая точка зрения на метод Эйлера может привести к его обобщению. 
На самом деле, формула, использованная при выводе метода Эйлера, является
лишь частным случаем разложения решения в ряд Тейлора:
$$x_{k+1}=x(t_{k+1})=x(t_k+h)=x(t_k)+\frac{dx}{dt}\bigg|_{t_k}h+
\frac{d^2x}{dt^2}\bigg|_{t_k}\frac{h^2}{2}+...$$
Таким образом, можно рассмотреть более точные методы, основанные на 
использовании членов ряда Тейлора высшего пордяка.
Например, метод Штермера - 3го порядка:
$$\frac{d^2x}{dx}=\frac{d}{dt}\left( \frac{dx}{dt} \right)=
\frac{f(t_k+h,x_k+f(t_k,x_k)h)-f(t_k,x_k)}{h}$$
Наиболее распространенный численный метод - метод  Рунге-Кутты - использует
разложение 4го порядка.

\textbf{Пример.} Супер-простая функция:
$$\begin{cases}
    \frac{dx}{dt}=x\\x(0)=1
\end{cases}$$
Это - определение обычной экспоненты. Решим методом Эйлера. Возьмем
$t_1=h$. Далее,  $x_1=1+f(t_0,x_0)\cdot h=1+h$. Далее,
 $t_2=t_1+h=2h$, $x_2=x_1+f(t_1,x_1)\cdot h=1+h+(1+h)h=(1+h)^2$.
 В общем виде, в точке $x_{k+1}=x_k(1+h)=(1+h)^{k+1}$ 
 функция будет принимать значение $x_n=(1+h)^{n}=\left( 1+\frac{T}{n}\right)^n
\to e^T$.
Неслучайно тут вылез замечательный предел - определение экспоненты.


\section{Теоремы о существовании и единственности решения}
\begin{defin}
Задача Коши для уравнения 
$$\begin{cases}
    \frac{dx(t)}{dt}=f(t,x)\\x(t_0)=x_0
\end{cases}$$
 - задача отыскания его решения, проходящего через точку $(t_0,x_0)$
\end{defin}
Доспустим, решение задачи Коши существует, а правая часть является 
непрерывной функцией $f(t,x)$.
Тогда мы можем проинтегрировать уравнение слева и справа:
 $$\int \frac{dx(t)}{dt} \equiv \int f(t,x)dt$$ 
Слева стоит решение, а справа определенный интеграл:
\begin{equation}\label{in}
 x\equiv x_0+\int\limits_{t_0}^{t}f(\tau,x(\tau))d\tau   
\end{equation}
\begin{defin}
    Назовем уравнение \ref{in} является интегральной формой задачи Коши.
\end{defin}
Следующая теорема оправдывает это название:
\begin{theor} (лемма)\\
    Задача Коши эквивалента решению интегрального уравнения \ref{in}
\end{theor}
\textbf{Доказательство.} Пусть $x(t)$ - решение задачи Коши. Тогда
при подставновке в уравнение имеем 
$$\frac{dx(t)}{dt}\equiv f(t,x(t))$$ 
Интегрируя, получим 
$x(t)=x_0+\int\limits_{t_0}^{t}f(\tau,x(\tau))d\tau$ - решение интегрального
уравнения.\\
Обратно, пусть $x(t)$ - непрерывное решение интегрального уравнения.
Тогда, взяв производную, получим
$$\frac{dx}{dt}=f(t,x(t))$$ 
Подставив $t=t_0$ в интегральное уравнение, получим  $x(t_0)=x_0$, т.е.
$x(t)$ - решение д.у. $\square$ 

Введением интегрального уравнения мы не решили задачу Коши, лишь
переписали её в другом виде, но мы сделали первый шаг к тому, чтобы 
доказать существование решения задачи Коши.
\begin{defin}
Функциональная последовательность Пикара $\{x_k(t)\}$\\
определяется следующим образом:  
$$x_0(t)=x_0,~x_{k+1}(t)=x_0+\int\limits_{t_0}^{t}f(\tau,x_k(\tau))d\tau$$
\end{defin}
\begin{defin}
Функция называется липшицевой с константой Липшица $L$, если
 $$\forall x,y \in D:|f(x)-f(y)|\leqslant L|x-y|$$
\end{defin}
Введем обозначения:\\
$D\subset \mathbb{R}^2$ - область на плоскости (открытое множество);\\
$B_r(x)=\{y\in \mathbb{R}\mid \rho(x,y)\leqslant r\}$ - замкнутый шар, причем
$B_r(x)\subset D$;\\
$m=\max\limits_{(t,x)\in B_r}|f(t,x)|$;\\
$\delta = \frac{r}{\sqrt{1+m^2}}$.\\
Теперь сформулируем основную теорему:
\begin{theor} (Коши-Пикара, или о существовании и единственности задачи 
Коши)\\
Пусть $f(t,x),~\frac{\partial f}{\partial x}$ непрерывны в области $D$. 
Тогда для любой точки $(t_0,x_0)\in D$ существует решение
$x(t)$ задачи Коши, определенное на отрезке 
$I_\delta=[t_0-\delta,t_o+\delta]$.\\
Кроме того, если $\tilde x(t)$ - другое решение задачи Коши, определенное
на интервале $[t_0-\tilde\delta,t_0+\tilde\delta]$, то существует такое
$\delta^*\in (0,\min(\delta,\tilde\delta))$, что $x(t)=\tilde x(t)$ 
для $t\in [t_0-\delta^*,t_0+\delta^*]$.
\end{theor}

\textbf{План доказательства.}\\
1. Последовательность Пикара корректно определена на D,
и её члены - непрерывные функции;\\
2. Существует непрерывный предел последовательности Пикара $x^*(t)$;\\
%и её предел - непрерывная функция;\\
3. $x^*(t)$ явлется решением задачи Коши;\\
4. В силу единственности предела последовательности, решение единственно.  

\textbf{Доказательство.}\\
1. Оценим разность между нулевым и первым членом последовательности:
$$|x_1(t)-x_0|=
\left| \int\limits_{t_0}^{t}f(\tau,x_0)d\tau\right|\leqslant 
\left|\int\limits_{t_0}^{t}m\,d\tau\right|\leqslant m|t-t_0|\leqslant
m\cdot \frac{r}{\sqrt{1+m^2}}\leqslant r$$
- значит, график функции лежит в $B_r$, пока  $t\in I_\delta$. 
Функция $x_1(t)$ непрерывна, поскольку, согласно интегральному уравнению,
она является первообразной для $f(t,x_0)$. 

Предположим, что для $k$ верно, что $x_k(t)$ определена в шаре, непрерывна и
$|x_k(t)-x_0|\leqslant r$. Тогда при $t\in I_\delta$ функция
$f(t,x_k(t))$ определена, непрерывна и ограниченна константной  $m$. Значит,
имеется оценка
$$x_{k+1}=x_0+
 \int\limits_{t_0}^{t}f(\tau,x_k(t))d\tau\leqslant  r$$
Интеграл непрерывен и ограничен, поэтому на $I_\delta$ имеет место
$|x_{k+1}(t)-x_0|\leqslant r$. Мы доказали по индукции, что все члены 
последовательности Пикара определены в шаре, непрерывны и ограниченны.

2. Докажем, что последовательность сходится. Рассмотрим ряд 
$x_0(t)+x_1(t)-x_0(t)+x_2(t)-x_1(t)+...+x_k(t)-x_{k-1}(t)+...$
Частичные суммы $S_n$ этого ряда равны $x_n$. Если мы докажем, что этот
ряд сходится равномерно, то по матанской теореме последовательность Пикара 
имеет непрерывный предел. Снова применим индукцию для оценки членов ряда
на $I_\delta$. База индукции: по неравенству для производн, 
$$|f(t,x)-f(t,\tilde x)|\leqslant L|\tilde x-x|,$$
где $L=\max\left|\frac{\partial f}{\partial x}(t,x)\right|,~(t,x),
(t,\tilde x)\in B_r$. Значит, $|x_1-x_0|\leqslant m|t-t_0|$. 

Пусть для $k$ выполнено  $|x_{k}-x_{k-1}|\leqslant \frac{m}{L}\cdot 
\frac{L^k(t-t_0)^k}{k!}\leqslant \frac{m}{L}\frac{L^k\delta^k}{k!}$.
Имеем 
$$|x_{k+1}-x_k|= \left|\int\limits_{t_0}^{t}f(\tau,x_{k+1}(\tau))d\tau-
\int\limits_{t_0}^{t}f(\tau,x_{k}(\tau))d\tau\right| = $$
$$=
\left|\int\limits_{t_0}^{t}\big(f(\tau,x_{k+1}(\tau))-f(\tau,x_{k}(\tau))\big)
d\tau\right|\leqslant
 \left|\int\limits_{t_0}^{t}\big|f(\tau,x_{k+1}(\tau))-
f(\tau,x_{k}(\tau))\big|d\tau\right|\leqslant
$$
$$\leqslant 
\left| \int\limits_{t_0}^{t}L|x_{k+1}(\tau)-x_k(\tau)|d\tau\right|
\leqslant 
\left| \int\limits_{t_0}^{t}\frac{mL^{k}|\tau-t_0|^{k}}{k!}d\tau\right|
= \frac{m}{L}\frac{L^{k+1}|t-t_0|^{k+1}}{(k+1)!}
$$
Здесь нам потребовалась следующая\\
\textbf{Лемма.} Непрерывно дифференцируемая на отрезке $[a,b]$ функция $f(x)$  
удовлетворяет условию Липшица с константой $L$, причем  
$L=\sup\limits_{x\in [a,b]}|f'(x)|$.\\
\textbf{Доказательство (\cite{Arnold},\S31.4).} Пусть $x,y\in [a,b]$. 
Введем параметр $z(t)=x+t(y-x),t\in [0,1]$. Тогда по формуле 
Ньютона-Лейбница
$$f(y)-f(x)=\int\limits_{0}^{1}\frac{d}{dt}(f(z(\tau)))d\tau=
\int\limits_{0}^{1}f'(z(\tau))\dot z(\tau)d\tau$$ 
Так как $\dot z = y - x$, то получаем
$$\left| \int\limits_{0}^{1}f'(z(\tau))\dot z(\tau)d\tau\right|\leqslant 
\int\limits_{0}^{1}L|y-x|d\tau = L|y-x| \quad\square $$

%То есть, 
%$|x_{k+1}-x_{k}|\leqslant \frac{m}{L} \frac{L^{k+1}(t-t_0)^{k+1}}{(k+1)!}
%\leqslant \frac{m}{L}\frac{L^{k+1}\delta^{k+1}}{(k+1)!}$ - мы доказали шаг 
%индукции. 
%Где-то тут мы ссылаемся на теорему Лагранжа о среднем. 
Итак, мы доказали шаг индукции, и заодно оценили ряд сходящимся рядом, не 
зависящим от $t:$
$$\frac{m}{L}\sum\limits_{n=1}^{\infty}\frac{L^n\delta^n}{n!}\to
\frac{m}{L}\left( e^{L\delta}-1 \right)$$
Значит, по признаку Вейерштрасса сумма ряда сходится равномерно, и её 
предел $\lim\limits_{n \to \infty}x_n(t)=x^*(t)$ непрерывен. 

%Обозначим $\lim\limits_{n \to \infty} x_n=x^*(t)$. 
3. По предыдущему пункту, $|x^*(t)-x_k(t)|\to 0$. Оценим эту разницу:
$$0\leqslant 
\left| \int\limits_{t_0}^{t} f(\tau,x^*(\tau))d\tau-\int\limits_{t_0}^{t}
f(\tau,x_k(\tau))d\tau\right|\leqslant 
\int\limits_{t_0}^{t}L|x^*(\tau)-x_k(\tau)|d\tau$$
Правая часть стремится к нулю, значит, и левая тоже. Поэтому, переходя к
пределу  по $t$ при  $k\to \infty$ в интегральной формуле, имеем
%$x_{k+1}(t)=x_0+\int\limits_{t_0}^{t}f(\tau,x_k(\tau))d\tau$, имеем
$$x^*=x_0+\int\limits_{t_0}^{t}f(\tau,x^*(\tau))d\tau$$
Итак, $x^*(\tau)$ решение интегрального уравнения, а значит и задачи Коши. 

4. Докажем единственность. Пусть на интервале  $I^*_\delta=
[t_0-\delta^*,t_0-\delta^*]$ определено
два решения, $x$ и  $x^*$. Тогда
$$|x^*(t)-x(t)|\leqslant \left| \int\limits_{t_0}^{t}f(\tau,x^*(\tau))-
f(\tau,x(\tau))) d\tau \right|\leqslant L \int\limits_{t_0}^{t}|x^*(\tau)-
x(\tau)|d\tau$$
Пусть $t>t_0$. Положим  
$\Delta= \int\limits_{t_0}^{t}|x^*(\tau)-x(\tau)|d\tau$, тогда 
$\frac{d\Delta}{dt}\leqslant L\Delta$; $\Delta(t_2)\geqslant\Delta(t_2)$
для всех $t_2>t_2\geqslant t_{0}$. 
%Кстати, $\frac{d\Delta}{dt}\leqslant L\Delta$
Значит, существует инфинум 
$T=\inf \{t\geqslant t_0\}$. Рассмотрим случай, когда $T=t_0$ (самая жесткая
оценка). Если это так, то пусть $а(t_0+\varepsilon)=\Delta_\varepsilon$. 
Тогда $\Delta_\varepsilon>0$. Поставим задачу Коши:
$$\begin{cases}
    \frac{d\Delta}{dt}=L\Delta\\ \Delta(t_0+\varepsilon)=\Delta_\varepsilon
\end{cases}$$
Отсюда $\Delta=\Delta_\varepsilon e^{L(t-t_0-\varepsilon)}$. Для всех
$t>t_0$,  $\Delta(t)\leqslant\Delta_\varepsilon e^{L(t-t_0-\varepsilon)}$.
Устремим $\varepsilon\to0$. Тогда и $\Delta(t)=0$ в пределе. 
Рассуждение при $t<t_0$ аналогично. $\square$


\textbf{Пример.}  Что можно сказать о решении задачи Коши для
$$\begin{cases}
    \frac{dx}{dt}=|x|\\x(0)=x_0
\end{cases}$$
Теорема Коши-Пикара не работает в нуле, так как там функция не дифференцируема.
Но не обманывают ли нас?
$| |x_1|-|x_2| |\leqslant 1\cdot |x_1-x_2|$. Модуль - липшицева функция, 
поэтому условия теоремы работают. 
А если 
$$\begin{cases}
    \frac{dx}{dt}=\sqrt{x} \\x(0)=x_0
\end{cases}$$
Производная растет неограниченно, функция не липшецева: 
$|\sqrt{x_1}-\sqrt{x_2}|=\frac{|x_1-x_2|}{\sqrt{x_1}+\sqrt{x_2}}\leqslant L
|x_1-x_2$ - при приближении к нулю $L\geqslant\frac{1}{\sqrt{x_1}+\sqrt{x_2}}$.
Но это только в нуле. А не в нуле можно $\Rightarrow$ решение сущетсвует и 
единственно. В общем, давайте зарешаем. Получаем $x=\frac{(t+C)^2}{4},~t+C>0$.
По условию $x(0)=0$, откуда $x=\frac{t^2}{2}$. Но ведь ещё есть куча решений
типа $x(t_0)=0$,  $x=\frac{(t-t_0)^2}{4}$, и даже $x=0$. 




