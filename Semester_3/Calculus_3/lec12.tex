\section{Критерии сходимости несобственного интеграла}
\begin{theor} (критерий Коши)\\
    Пусть $\forall b\geqslant a$ функция интегрируема на $[a,b]$. 
    Тогда $\int_a^\infty f(x)dx$ сходится $\Leftrightarrow$ $\forall 
    \varepsilon>0~\exists b_0(\varepsilon)>a~\forall b_1,b_2>b_0:
    \left| \int_{b_1}^{b_2}f(x)dx \right|<\varepsilon$
\end{theor}
\textbf{Доказательство.} По условию, существует предел 
$\lim\limits_{b \to +\infty} F(b)=A\in \mathbb{R}$, где
$F(b)=\int^b_af(x)dx$. 
Зафиксируем $\varepsilon>0$. Тогда из существования предела следует
для $\frac{\varepsilon}{2}$: $\exists b_o(\varepsilon)>a:\left| 
F(b)-A\right|<\frac{\varepsilon}{2}$. Пусть $b_1>b_0,~b_2>b_0$. Тогда
$|F(b_2)-F(b_1)|=|F(b_2)-A|+|F(b_1)-A|<\frac{\varepsilon}{2}+
\frac{\varepsilon}{2}=\varepsilon$.\\
Достаточность. Докажем существование предела  $\lim\limits_{b\to\infty}F(b)$
из определения предела по Гейне. Пусть $b_n\to \infty$, тогда $\forall b_0>a
~\exists n_0(\varepsilon)\in \mathbb{N}~\forall n>n_0:b_n>b_0$. 
Зафиксируем $n,m>n_0$. Тогда $b_n>b_0$ и $b_m>b_0$. По условию отсюда 
следует, что $|F(b_n)-F(b_m)|<\varepsilon$. По критерию Коши для числовой 
последовательности $F(b_n)$ существует предел 
$\lim\limits_{n \to \infty} F(b_n)=B\in \mathbb{R}$.\\
Покажем, что предел не зависит от выбора последовательности $b_n$. 
Выберем другую последовательность  $b^*_n$. Обозначим предел 
$\lim\limits_{n \to \infty} F(b^*_n)=B$. Составим последовательность
$b_1,b^*_1,b_2,b^*_2,...\to \infty$. Тогда предел $F$ от этой
последовательности обозначим как  $C$. Так как пределы подпоследовательностей 
сходятся к пределу последовательности, то  $A=B=C$. Значит, выполняется
условие определения предела по Гейне, значит, интеграл сходится. $\square$

\textbf{Пример.} $\int_1^\infty \frac{\sin x}{x^\alpha}dx$ сходится при 
$\alpha>0$, расходится при $\alpha\leqslant 0$. Докажем это.\\
1. $\alpha>0$. Поехали: $\forall \varepsilon>o~\exists b_0(\varepsilon)>1~
\forall b_1>b_0,b_2>b_0: \left| \int^{b_2}_{b_1} \frac{\sin x}{x^\alpha}dx
\right|<\varepsilon$. Доказываем: 
$\left| \int^{b_2}_{b_1} \frac{\sin x}{x^\alpha}dx
\right|=\left| \int^{b_2}_{b_1} \frac{1}{x^\alpha}d\cos x\right|=
\left| \frac{\cos x}{x^\alpha} \right|^{b_2}_{b_1}-\int^{b_2}_{b_1} 
\cos x d(\frac{1}{x^\alpha})\leqslant ... \leqslant\frac{4}{b^\alpha_0}$.
Значит, $b_0>(\frac{4}{\varepsilon})^\frac{1}{\alpha}$.\\
2. $\alpha\leqslant 0$. Синус теперь принимает разные знаки. Пусть 
$b_k=2\pi k$. Тогда по критерию Коши интеграл расходится.

\begin{theor} (критерий сходимости через остаток)\\ \label{skhod_ost}
Пусть $\int^\infty_af(x)dx=\int^b_af(x)dx+\int^\infty_bf(x)dx,~(b>a)$. 
Тогда:\\
1. Если несобственный интеграл сходится, то и любой из его остатков сходится.\\
2. Если хотя бы один из остатков сходится, то несобственный интеграл сходится.
\end{theor}
\textbf{Доказательство.} $\int^b_af(x)dx$ - число, поэтому сходимость 
равносильна сходимости остатка. $\square$ 

\begin{theor} (критерий сходимости несобственного интеграла от неотрицательной
функции)\\
Пусть $\forall b>a$ функция интегрируема на $[a,b]$ и неотрицательна.Тогда
$\int^\infty_af(x)dx$ сходится $\Leftrightarrow$ первообразная $F(b)<M$ 
ограниченна.
\end{theor}
\textbf{Доказательство.} Пусть $a<b_1<b_2$. Имеем
$$F(b_2)=\int\limits_{a}^{b_2}f(x)dx=\int\limits_{a}^{b_1}f(x)dx+
\int\limits_{b_1}^{b_2}f(x)dx=F(b_1)+\int\limits_{b_1}^{b_2}f(x)dx$$
откуда $F(b_2)-F(b_1)=\int\limits_{b_1}^{b_2}f(x)dx\geqslant 0$. Значит,
$F(b)$ неубывает и ограниченна сверху. Значит, существует предел
$\lim\limits_{b \to \infty}F(b)$, и интеграл сходится.\\
Обратно, пусть существует конечный предел $\lim\limits_{b \to \infty} F(b)$,
то $F(b)$ ограниченна в некоторой окрестности бесконечности $U$.
Тогда существует такое $b_0>a~\forall b\geqslant b_0:F(b)\leqslant M$.
Значит, $B(b^*)\leqslant M$ - ограниченна. Если $b^*<b_0$, то
$F(b^*)<F(b_0)$. Итак, $F$ ограниченна. $\square$

\section{Признаки сравнения}
В данном разделе признаки работают для неотрицательных функций.
\begin{theor} (первый признак сравнения/в оценочной форме)\\
Пусть $f(x)>g(x)>0$ начиная с некоторого $x>a$, и для любого  $b>a$
функции интегрируемы на $[a,b]$. Тогда\\
1. Если  $\int^\infty_a f(x)$ сходится, то и  $\int^\infty_a g(x)$ сходится.\\
2. Если  $\int^\infty_a g(x)$ расходится, то и $\int^\infty_a f(x)$ расходится.
\end{theor}
\textbf{Доказательство.} 1. Обозначим $F(b)=\int\limits_{a}^{b}f(x)dx$ и
$G(x)=\int\limits_{a}^{b}g(x)dx$. По условию, интеграл сходится, поэтому
$F(b)\leqslant M$. По свойству определенного интеграла, $F(b)\geqslant G(x)$,
поэтому $G(b)$ также ограниченна и сходится по предыдущему критерию.\\
2. Допустим, что $\int\limits_{a}^{\infty}g(x)dx$ сходится. Тогда из 
первого пункта следует, что и $\int\limits_{a}^{\infty}f(x)dx$ сходится, 
что противоречит условию. $\square$ 

\begin{theor} (второй признак сравнения/в предельной форме)\\
Если $\lim\limits_{x \to \infty}\frac{f(x)}{g(x)}=k,~\infty\ne k\ne0$, 
то их несобственные интегралы сходятся или расходятся одновременно. 
\end{theor}
\textbf{Доказательство.} Пусть $k>0$. Тогда для
 $$\varepsilon>\frac{k}{2}~\exists b_0>a~\forall x>b_0:\frac{k}{2}<
 \frac{f(x)}{g(x)}<\frac{3k}{2}$$ 
Значит, $f(x)< \frac{3k}{2}g(x)$, и из сходимости интеграла от $g(x)$ следует
сходимость интеграла от $f(x)$. C другой стороны, $\frac{k}{2}g(x)<f(x)$, 
поэтому из сходимости интеграла от $f(x)$ следует сходимость интеграла от 
$g(x)$. $\square$ 

\textbf{Следствие.} Если $f(x)\sim g(x)$ при $x\to \infty$, то интегралы 
сходятся или расходятся одновременно. 




















