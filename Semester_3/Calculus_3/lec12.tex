\subsection{Критерии сходимости несобственного интеграла}
\begin{theor} (критерий Коши)
    Пусть $\forall b\geqslant a$ функция интегрируема на $[a,b]$. 
    Тогда $\int_a^\infty f(x)dx$ сходится $\Leftrightarrow$ $\forall 
    \varepsilon>0~\exists b_0(\varepsilon)>0~\forall b_1,b_2>b_0:
    \left| \int_{b_1}^{b_2}f(x)dx \right|<\varepsilon$
\end{theor}
\textbf{Доказательство.} По условию, существует предел 
$\lim\limits_{b \to +\infty} F(b)=A\in \mathbb{R}$, где
$F(b)=\int^b_af(x)dx$. 
Зафиксируем $\varepsilon>0$. Тогда из существования предела следует
для $\frac{\varepsilon}{2}$: $\exists b_o(\varepsilon)>a:\left| 
F(b)-A\right|<\frac{\varepsilon}{2}$. Пусть $b_1>b_0,~b_2>b_0$. Тогда
$|F(b_2)-F(b_1)|=|F(b_2)-A|+|F(b_1)-A|<\frac{\varepsilon}{2}+
\frac{\varepsilon}{2}=\varepsilon$.\\
Достаточность. Докажем существование предела  $\lim\limits_{b\to\infty}F(b)$
из определения предела по Гейне. Пусть $b_n\to \infty$, тогда $\forall b_0>a
~\exists n_0(\varepsilon)\in \mathbb{N}~\forall n>n_0$ 
Покажем, что предел не зависит от выбора последовательности $b_n$. 
Выберем другую последовательность  $b^*_n$. Обозначим предел 
$\lim\limits_{n \to \infty} F(b^*_n)=B$. Составим последовательность
$b_1,b^*_1,b_2,b^*_2,...\to \infty$. Тогда предел $F$ от этой
последовательности обозначим как  $C$. Так как пределы подпоследовательностей 
сходятся к пределу последовательности, то  $A=B=C$. Значит, выполняется
условие определения предела по Гейне, значит, интеграл сходится. $\square$\\
\textbf{Пример.} $\int_1^\infty \frac{\sin x}{x^\alpha}dx$ сходится при 
$\alpha>0$, расходится при $\alpha\leqslant 0$. Докажем это.\\
1. $\alpha>0$. Поехали: $\forall \varepsilon>o~\exists b_0(\varepsilon)>1~
\forall b_1>b_0,b_2>b_0: \left| \int^{b_2}_{b_1} \frac{\sin x}{x^\alpha}dx
\right|<\varepsilon$. Доказываем: 
$\left| \int^{b_2}_{b_1} \frac{\sin x}{x^\alpha}dx
\right|=\left| \int^{b_2}_{b_1} \frac{1}{x^\alpha}d\cos x\right|=
\left| \frac{\cos x}{x^\alpha} \right|^{b_2}_{b_1}-\int^{b_2}_{b_1} 
\cos x d(\frac{1}{x^\alpha})\leqslant ... \leqslant\frac{4}{b^\alpha_0}$.
Значит, $b_0>(\frac{4}{\varepsilon})^\frac{1}{\alpha}$.\\
2. $\alpha\leqslant 0$. Синус теперь принимает разные знаки. Пусть 
$b_k=2\pi k$. Тогда по критерию Коши интеграл расходится.
\begin{theor} (критерий сходимости через остаток)\\
Пусть $\int^\infty_a=\int^b_a+\int^\infty_b,~(b>0)$.\\
1. Если интеграл сходится, то и любой из его остатков сходится.\\
2. Если хотя бы один из остатков сходится, то интеграл сходится.
\end{theor}
\textbf{Доказательство.}  
$\square$ 
\begin{theor} (критерий сходимости несобственного интеграла от несобственной
функции)\\
Пусть $\forall b>a$ функция интегрируема на $[a,b]$ и неотрицательная .Тогда
$\int^\infty_af(x)dx$ сходится $\Leftrightarrow$ первообразная $F(b)<M$ 
ограниченна.
\end{theor}
\textbf{Доказательство.} $F(b)$ неубывает и имеет конечный предел. Значит,
интеграл сходится. Обратно, пусть существует конечный предел 
$\lim\limits_{b \to \infty} F(b)$, то $F(b)$ ограниченна в некоторой 
окрестности. $\square$\\
\subsection{Признаки сравнения в предельной форме}
\begin{theor} (признак сравнения)\\
Пусть $f(x)>g(x)>0$ начиная с некоторого $x>a$, и для любого  $b>a$
функции интегрируемы на $[a,b]$. Тогда\\
1. Если  $\int f(x)$ сходится, то и  $\int g(x)$  сходится.\\
2. Если  $\int g(x)$ расходится, то и $\int f(x)$ расходится.
\end{theor}
\textbf{Доказательство.}  По свойству определенного интеграла (транзитивность 
числовых неравенств), $F(b)\leqslant M$. Тогда по критерию 3 интеграл 
сходится.
2. Погодите, это реально?
$\square$ 
\begin{theor} (второй признак сравнения)\\
Если $\frac{f(x)}{g(x)}=k,~\infty\ne k\ne0$, то их интегралы сходятся или
расходятся одновременно. 
\end{theor}
\textbf{Доказательство.}  \
$\square$ 




















