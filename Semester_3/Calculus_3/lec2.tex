\subsection{Знакопостоянные ряды}
Исследуем подробнее знакопостоянные ряды. Ряд называется знакопостоянным,
если, начиная с некоторого номера, все его члены имеют одинаковый знак
(конечное число членов в начале не влияет на сходимость). Следующие теоремы 
устанавливаются для положительных рядов, для отрицательных рядов применимы
эти же рассуждения, стот лишь поменять знак.
\begin{theor}
    (критерий сходимости для неотрицательных рядов)\\
    Ряд сходится тогда и только тогда, когда последовательность его
    частичных сумм ограничена сверху.
\end{theor}
\textbf{Доказательство.}
$\Rightarrow$. По условию, существует предел 
$\lim\limits_{n \to \infty} S_n=S\in \mathbb{R}$. Значит, последовательность 
частичных сумм ограничена сверху.\\
$\Leftarrow$. По условию, ограниченная неубывающая последовательность 
$\{S_n\}$ ограничена сверху, значит, по теореме
Вейерштрасса у неё есть предел  $S$. $\square$\\
Следующее важное утверждение о положительных рядах - признак сравнения. Он
позволяет делать выводы о сходимости ряда, сравнивая его с известными рядами:
геометрической прогрессией, обобщенным гармоническим рядом (то есть с 
произвольным показателем степени). 
\begin{theor}
    (признак сравнения в оценочной форме)\\
    Пусть даны последовательности
    $0\leqslant a_n\leqslant b_n~\forall n\in\mathbb{N}$. 
    Тогда из сходимости ряда с общим членом $b_n$ следует сходимость ряда с 
    общим членом $a_n$ (из расходимости ряда с общим членом $a_n$ следует
    расходимость ряда с общим членом $b_n$).
\end{theor}
\textbf{Доказательство.}  Докажем исходя из критерия сходимости.
Пусть $A_n,B_n$ - частичные суммы рядов с членами $a_n,~b_n$.
Так как ряд $B$ сходится, то существует верхний предел $M$ для его частичных 
сумм.
Так как члены ряда $A$ меньше членов ряда $B$, то 
$A_n\leqslant B_n\leqslant M$, откуда по транзитивности неравенств 
$A_n\leqslant M$, значит, у $A_n$ есть предел. $\square$

\textbf{Пример.} Обобщенный гармонический ряд
$\sum\limits_{n=1}^{\infty} \frac{1}{n^p}$.
Рассмотрим $p\leqslant 1$,  $n^p\leqslant 1$,
$\frac{1}{n^p}>\frac{1}{n}$. Так как 
гармонический ряд расходится, то $\sum\limits_{n=1}^{\infty}\frac{1}{n^p}$
расходится. C другой стороны, при $p>1$ ряд сходится по интегральному 
признаку. 

\textbf{Пример.} Найти сумму. $\sqrt{2}+\sqrt{2-\sqrt{2} }+\sqrt{2-
\sqrt{2+\sqrt{2} } } +...$, $a_{n+1}=\sqrt{2-b_n} $, $b_{n+1}=\sqrt{2+b_n} $.
Заметим, что $b_1=2\cos\frac{\pi}{4}$, $b_2=\cos\frac{\pi}{8}$. Дальше
эта формула выводится по индукции. $b_n=2\cos\frac{\pi}{2^{n+1}}$. 
$a_n=\sqrt{2-b_{n-1}}=\sqrt{2-2\cos\frac{\pi}{2^n}}=2\sin\frac{\pi}{2^{n+1}}$ 
Ита, $a_n\leqslant 2\cdot \frac{\pi}{2^{n+1}}=\frac{\pi}{2^n}$.

\begin{theor}
    (Признак сравнения в предельной форме)\\
    Пусть даны неотрицательные ряды $\sum\limits_{n=1}^{\infty} a_n$, 
    $\sum\limits_{n=1}^{\infty}b_n$. Пусть $k=\lim\limits_{n \to \infty}
    \frac{a_n}{b_n}$. Тогда, если\\
1. $k=const~(k\ne 0)$: ряды сходятся или расходятся одновременно.

1.1. $k=1$: ряды эквивалентны.\\
2. $k=0$: если В сходится, то и А сходится.\\
3. $k=\infty$: если А сходится, то и В сходится.
\end{theor}
\textbf{Доказательство.}\\
1. Запишем определение предела $\lim\limits_{n \to \infty} \frac{a_n}{b_n}=k$
для $\varepsilon=\frac{k}{2}>0$:
$$\exists N(\varepsilon)~\forall n>N:\frac{k}{2}<\frac{a_n}{b_n}<\frac{3k}{2}$$
откуда $a_n<\frac{3k}{2}b_n$.
Значит, если ряд $B$ сходится, то и ряд $A$ сходится.\\
2. Пусть $\lim\limits_{n \to \infty}\frac{a_n}{b_n}=0$. Для $\varepsilon=1~
\exists N~\forall n>N: \frac{a_n}{b_n}<1$, 
значит $a_n<b_n$ и сходимость рядов следует из признака сравнения в оценочной 
форме.\\
3. Переворачивая предел в п.2, получаем все аналогично. $\square$ 

\textbf{Пример.} Исследуем на сходимость ряд
$\sum\limits_{n=1}^{\infty}(\frac{1}{n^\alpha}-
\frac{1}{(n+1)^\alpha})$. Имеем $S_n=1-\frac{1}{(n+1)^\alpha}$.
При $\alpha>0$ $S_n$ сходится к 1, при $\alpha<0$ ряд расходится.

\begin{theor} (третий признак сравнения)\\
Пусть даны ряды $A=\sum\limits_{n=1}^{\infty} a_n$ и 
$B=\sum\limits_{n=1}^{\infty} b_n$, причем 
$\frac{a_{n+1}}{a_n}\leqslant \frac{b_{n+1}}{b_n}$. Тогда если
В сходится, то и А сходится.
\end{theor}
\textbf{Доказательство.} 
Перемножив положительные неравенства $\frac{a_2}{a_1}\leqslant \frac{b_2}{b_1}
...\frac{a_{k+1}}{a_k}\leqslant 
\frac{b_{k+1}}{b_k}$, получим $\frac{a_n}{a_1}\leqslant \frac{b_n}{b_1}$, 
откуда $a_n\leqslant b_n\cdot const$. Из признака сравнения в оценочной форме
получаем, что ряд $A$ сходится, если сходится ряд  $B$. $\square$ 
%\begin{theor}
%    (Признак Даламбера в оценочной форме)\\
%\end{theor}
%\textbf{Доказательство.}  \
%$\square$ 
%\begin{theor}
%Признак даламбера в предельной форме: $\lim\limits_{n \to \infty} $
%\end{theor}
%\textbf{Доказательство.}  \
%$\square$ 

