\begin{theor}
    (критерий сходимости для неотрицательных рядов)\\
    Пустьдан ряд. Тогда ряд сходится $\Leftrightarrow$ последовательность
    частичных сумм ограничена сверху.
\end{theor}
\textbf{Доказательство.} $\Rightarrow$. По услови, существует предел 
$lim S_n=S\in \mathbb{R}$ $\Rightarrow$ $\{S_n\}_n\in\mathbb{N}$ - ограничена
В другую сторону. По условию, $\{S_n\}$ ограничена сверху, $\Rightarrow$ по тео
реме Вейрштрасса для ограниченной неубывающей последовательности имеется предел
$\square$ \\

\textbf{Признак сравнения.} С чем же сравнивать? С геометрической прогрессией, 
с обобщенным гармоническим рядом (с произвольной степенью числа). 
\begin{theor}
    (признак сравнения в оценочной форме)\\
    Дано $0\leqslant a_n\leqslant b_n~\forall n\in\mathbb{N}$ :
    Тогда из сходимости В следует сходимость А, из расходимости А следует
    расходимость В.
\end{theor}
\textbf{Доказательство.}  Докажем исходя из критерия сходимости. \\
1. Пусть $A_n,B_n$ - частичные суммы своих рядов. Так как ряд В сходится, 
то существует верхний предел для его частичных сумм. Так как ряд А меньше Б,
по транзитиавности неравенств верхняя граница В лежит выше чем А. ЧТо по тому 
же критерию дает сходимость. 
2. 
$\square$ \\

Пример. Рассмотрим $p<1$,  $n^p<1$,  $\frac{1}{n^p}>\frac{1}{n}$. Так как 
гармонический ряд расходится, то $sum \frac{1}{n^p}$ расходится. 

\textbf{Пример.} Найти сумму. $\sqrt{2}+\sqrt{2-\sqrt{2} }+\sqrt{2-
\sqrt{2+\sqrt{2} } } +...$, $a_{n+1}=\sqrt{2-b_n} $, $b_{n+1}=\sqrt{2+b_n} $.
Заметим, что $b_1=2\cos\frac{\pi}{4}$, $b_2\cos\frac{\pi}{8}$. Дальше
эта формула выводится по индукции. $b_n=2\cos\frac{\pi}{2^{n+1}}$. 
$a_n=\sqrt{2-b_{n-1}}=\sqrt{2-2\cos\frac{\pi}{2^n}}=2\sin\frac{\pi}{2^{n+1}}$ 
Ита, $a_n\leqslant 2\cdot \frac{\pi}{2^{n+1}}=\frac{\pi}{2^n}$ 

\begin{theor}
    (Признак сравнения в предельной форме)\\
    Пусть даны неотрицательные ряды $\sum\limits_{n=1}^{\infty} a_n$, 
    $\sum\limits_{n=1}^{\infty}b_n$. Если предел отношения общего члена\\
    1. Равен конечной (ненулевой) константе. Тогда ряды сходятся или расходятся 
    одновременно\\
    1.1. В частности, при mkk=1, ряды эквивалентны. 
2. Если $\lim\limits_{n \to \infty} \frac{a_n}{b_n}=0$, то имеет место
"В сходится $\Rightarrow$ А сходится"
3. Если этот предел равен $ooo$, то: "А сходится $\Rightarrow$ В сходится"


\end{theor}
\textbf{Доказательство.} По опреелению предела. 
$\lim\limits_{n \to \infty} a_\frac{n}{b_n}=k$ для $\varepsilon=k/2>0\exists 
n_0(\varepsilon)\forall n>n_0: k/2<\frac{a_n}{b_n}<3k/2$. тогда если В 
сходится, А сходится.

2. Пусть $\lim\limits_{n \to \infty} a_\frac{n}{b_n}=0$. Lkz $\varepsilon=1$, 
тогда для этого эпсилон  $\exists n_0$ утверждение следует из первого
признака сравнения. 

Пункт 3 напрямую следует из второго.
$\square$ 

\textbf{Пример. 3}  $\sum\limits_{n=1}^{\infty}(\frac{1}{n^\alpha}-
\frac{1}{(n+1)^\alpha})$. Имеем $S_n=1-\frac{1}{(n+1)^\alpha}$ Прии 
альфа>0 сходится к 1, при альфа<0 ряд расходится.
(ljнайддем область расходимости обобщенного гармонического
рядва с помощью уже известного)

\begin{theor}
    (тертий признак сравнения.)\\
Пусть даны ряды А и В ($\sum\limits_{n=1}^{\infty} a_n,~\sum
\limits_{n=1}^{\infty} b_n$),и выполняется $a_{n+1}/a_n\leqslant b_{n+1}/b_n$
Тогда В сходится $\Rightarrow$ А сходится
(если А расходится, В расходится)

\end{theor}
\textbf{Доказательство.}  так как все неравенства полоэительные, их всех можно
перемножить: тогда утверждение следует из первого признака сравнения. 
\
$\square$ 

\begin{theor}
    (Признак Даламбера в оценочной форме)\\
\end{theor}
\textbf{Доказательство.}  \
$\square$ 

\begin{theor}
Признак даламбера в предельной форме: $\lim\limits_{n \to \infty} $
\end{theor}
\textbf{Доказательство.}  \
$\square$ 


















