\chapter{Несобственный интеграл}
\section{Основные определения}
\begin{defin}
Пусть функция f интегрируема на писе длины $[a,b]$ для всех  $b>a$. 
Тогда \textbf{несобственный интеграл первого рода} (c одной особой точкой)
- предел 
$$\int\limits_{a}^{\infty}f(x)dx:=\lim\limits_{b\to\infty}\int\limits_{a}^{b}
f(x)dx$$
\end{defin}
Если таковой предел существует, то интеграл сходится; если предел равен 
бесконечности или не существует, то интеграл расходится. Анал
определяется и интеграл с нижним пределом $-\infty$.
\begin{defin}
    Пусть $\forall \varepsilon>0$ функция $f$ интегрируема на $[a+\varepsilon,
    b]$, и $\lim\limits_{x \to a+0} f(x)=\infty$. Тогда \textbf{несобственный
    интеграл второго рода} (с особой точкой $a$) - предел 
    $$\int\limits_{a}^{b} f(x)dx:=\lim\limits_{\varepsilon \to 0} 
    \int\limits_{a+\varepsilon}^{b} f(x)dx$$
\end{defin}

\textbf{Пример.} $\int\limits_{0}^{1} \ln xdx=\lim\limits_{\varepsilon \to 
+0} \left( \int\limits_{\varepsilon}^{1}\ln xdx\right)=
\lim\limits_{\varepsilon \to +0}\left( x\ln x\big|_\varepsilon^1-
\int\limits_{\varepsilon}^1dx  \right)=\lim\limits_{\varepsilon \to +0}
\frac{-\varepsilon^2}{1 /\varepsilon}-1=-1$ - интеграл сходится.

Если на некотором промежутке интеграл имеет конечное число особых точек, то 
всегда можно разбить промежуток на такие области, в которых каждый интеграл 
имеет лишь одну особую точку. Говорят, что интеграл \textbf{сходится, 
если он сходится на в каждой особой точке!} Так,
$\int\limits_{0}^{\infty} \frac{dx}{x^p}$ расходится при любом $p$, так 
как он расходится хотя бы на одном из промежутков  $(0,1)$ или
$[1,\infty)$ /

