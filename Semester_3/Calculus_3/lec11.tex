\chapter{Несобственный интеграл}
\section{Основные определения}
\begin{defin}
Пусть функция f интегрируема на отрезке $[a,b]$ для всех  $b>a$. 
Тогда \textbf{несобственный интеграл первого рода} (c одной особой точкой)
- предел 
$$\int\limits_{a}^{\infty}f(x)dx:=\lim\limits_{b\to\infty}\int\limits_{a}^{b}
f(x)dx$$
\end{defin}
Если таковой предел существует, то интеграл сходится; если предел равен 
бесконечности или не существует, то интеграл расходится. Аналолгично
определяется и интеграл с нижним пределом $-\infty$.
\begin{defin}
    Пусть $\forall \varepsilon>0$ функция $f$ интегрируема на $[a+\varepsilon,
    b]$, и $\lim\limits_{x \to a+0} f(x)=\infty$. Тогда \textbf{несобственный
    интеграл второго рода} (с особой точкой $a$) - предел 
    $$\int\limits_{a}^{b} f(x)dx:=\lim\limits_{\varepsilon \to 0} 
    \int\limits_{a+\varepsilon}^{b} f(x)dx$$
\end{defin}

\textbf{Пример.} $\int\limits_{0}^{1} \ln xdx=\lim\limits_{\varepsilon \to 
+0} \left( \int\limits_{\varepsilon}^{1}\ln xdx\right)=
\lim\limits_{\varepsilon \to +0}\left( x\ln x\big|_\varepsilon^1-
\int\limits_{\varepsilon}^1dx  \right)=\lim\limits_{\varepsilon \to +0}
\frac{-\varepsilon^2}{1 /\varepsilon}-1=-1$ - интеграл сходится.

Если на некотором промежутке интеграл имеет конечное число особых точек, то 
всегда можно разбить промежуток на такие области, в которых каждый интеграл 
имеет лишь одну особую точку. Говорят, что интеграл \textbf{сходится, 
если он сходится в каждой особой точке!} Так,
$\int\limits_{0}^{\infty} \frac{dx}{x^p}$ расходится при любом $p$, так 
как он расходится хотя бы на одном из промежутков  $(0,1)$ или
$[1,\infty)$.

Будем обозначать интегрируемость в смысле несобственнного интеграла как
$f\in \tilde R$.
\begin{theor}
    (формула Ньютона-Лейбница)\\
    Пусть $f\in \tilde R[a,b)$, b - особая точка первого или второго рода, 
    и на этом интервале существует функция $F(x)$ - первообразная для  $f$. 
    Тогда, если интеграл сходится, то
     $$\int\limits_{a}^{b}f(x)dx=F(b)-F(a)$$ 
     где $F(b)=\lim\limits_{x \to b}f(b)$.
\end{theor}
%Будем называть промежутком отрезок, интервал или полуинтервал.
\begin{defin}
Пусть $\omega \in (a,b)$ - особая точка функции $f$, причем
$\forall \varepsilon>0:f\in R[a,\omega-\varepsilon],f\in R[\omega+\varepsilon,
b]$. Тогда интеграл в смысле главного значения по Коши 
(<<valeur principale>>) - 
$$v.p. \int\limits_{a}^{b}=\lim\limits_{\varepsilon \to 0}\left(
\int\limits_{a}^{\omega-\varepsilon}f(x)dx+\int\limits_{\omega+\varepsilon}^{b}
f(x)dx\right)$$
%где $\omega\in (a,b)$ - особая точка.
\end{defin}
\textbf{Пример.} $v.p. \int\limits_{0}^{2}\frac{dx}{1-x^2}=\frac{1}{2}\ln3$.

\section{Cвойства несобственного интеграла}
Будем называть промежутком отрезок, интервал или полуинтервал и обозначать
его как $\langle a, b\rangle$

\begin{enumerate}

\item Пусть $f\in \tilde R[a,\infty)$, $g\in \tilde R[a,\infty)$.
Тогда  
$\forall \lambda,\mu\in \mathbb{R}:\lambda f+\mu g\in \tilde R[a,\infty)$.
(линейность).

\item Пусть $f\in \tilde R[a,\infty)$,  $g\in \tilde R[a,\infty)$ и
    $f(x)\leqslant g(x)~\forall x\geqslant a$. Тогда
    $\int\limits_{a}^{\infty}f(x)dx\leqslant \int\limits_{a}^{\infty}g(x)dx$.

\item Формула замены переменной.\\
    Пусть $f\colon\langle a,b \rangle\to \mathbb{R},f\in \tilde R$, 
    $\varphi\colon\langle \alpha,\beta\rangle\to\langle a,b\rangle$,
    причем $\varphi(a)=a,\varphi(\beta)=b$, $\varphi$ возрастает и у неё 
    существует и непрерывна производная на $\langle\alpha,\beta\rangle$.
    Тогда $\int\limits_{\alpha}^{\beta}f(\varphi(x))\cdot \varphi'(x)dx=
    \int\limits_{a}^{b} f(t)dt$.


\item Формула интегрирования по частям.\\
    Пусть функции $u(x),v(x)$ непрерывно дифференцируемы на $[a,\infty)$ и 
    существует $\lim\limits_{x \to \infty}u(x)v(x)$, тогда оба интеграла
    $\int\limits_{a}^{\infty}u(x)v'(x)dx,\int\limits_{a}^{\infty}
    u'(x)v(x)dx$ сходятся или расходятся одновременно, и в случае 
    сходимости\\
    $ \int\limits_{a}^{\infty}u'(x)v(x)dx+\int\limits_{a}^{\infty}v'(x)u(x)dx=
    uv\Big|^\infty_a$.


\end{enumerate}
\textbf{Доказательство.}\\
1. Линейность следует из линейности определенного интеграла и линейности
предела. Действительно,
$$\int\limits_{a}^{\infty}(\lambda f+\mu g)dx = 
\lim\limits_{b \to \infty} \int\limits_{a}^{b} (\lambda f+\mu g)dx = 
\lim\limits_{b \to \infty} \left(\lambda \int\limits_{a}^{b}f(x)dx + \mu
\int\limits_{a}^{b} g(x)dx\right) =
$$ 
$$ = \lambda \lim\limits_{b \to \infty} \int\limits_{a}^{b} f(x)dx + 
\mu \lim\limits_{b \to \infty} \int\limits_{a}^{b} g(x)dx = 
\lambda \int\limits_{a}^{\infty}f(x) dx + \mu \int\limits_{a}^{\infty}g(x)dx
$$
2. Рассмотрим $x_0\in [a,b\rangle$. По аналогичному свойству для
собcтвенных инетгралов имеем
$$f(x)\leqslant g(x)\implies \int\limits_{a}^{x_0}f(x)\leqslant    
\int\limits_{a}^{x_0}g(x)dx$$ 
откуда, переходя к пределу при $x_0\to\infty$ имеем искомое свойство.\\
3. Рассмотрим $x_0\in [a,b\rangle$. По аналогичному свойству для
собcтвенных инетгралов имеем
$$\int\limits_{\alpha}^{\beta_0}f(\varphi(x))\cdot \varphi'(x)dx=
    \int\limits_{a}^{x_0} f(t)dt$$ 
откуда, переходя к пределу при $x_0\to\infty$ имеем искомое свойство.\\
4 (Фихтенгольц). Рассмотрим $x_0\in [a,b\rangle$. По обычной формуле 
интегрирования по частям имеем
$$\int\limits_{a}^{x_0} u\,dv = (u(x_0)v(x_0)-u(a)v(a))-\int\limits_{a}^{x_0}
v\,du$$ 
По условию, что правая часть имеет конечные пределы при $x_0\to \infty$, 
поэтому и левая часть тоже сходится. 









