\chapter{Несобственный интеграл}
\section{Основные определения}
\begin{defin}
Пусть функция f интегрируема на отрезке $[a,b]$ для всех  $b>a$. 
Тогда несобственный интеграл первого рода (c одной особой точкой) - предел 
$$\int\limits_{a}^{\infty}f(x)dx:=\lim\limits_{b\to\infty}\int\limits_{a}^{b}
f(x)dx$$
\end{defin}
Если таковой предел существует, то интеграл сходится; если предел равен 
бесконечности или не существует, то интеграл расходится. Аналогично
определяется и интеграл с нижним пределом $-\infty$.\\
\textbf{Пример.} $\int\limits_{0}^{1} \ln xdx=\lim\limits_{\varepsilon \to 
+0} \left( \int\limits_{\varepsilon}^{1}\ln xdx\right)=
\lim\limits_{\varepsilon \to +0}\left( x\ln x\big|_\varepsilon^1-
\int\limits_{\varepsilon}^1dx  \right)=\lim\limits_{\varepsilon \to +0}
\frac{-\varepsilon^2}{1 /\varepsilon}-1=-1$ - интеграл сходится.

Рассмотрим случай конечного числа особых точек.
