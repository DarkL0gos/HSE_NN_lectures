\section{Абсолютная и условная сходимость}
\begin{defin}
Интеграл сходится абсолютно, если сходится интеграл от модуля функции.
\end{defin}

\begin{theor}
    Если интеграл сходится абсолютно, то он сходится.
\end{theor}
\textbf{Доказательство.}  Используем критерий Коши: 
$\int\limits_{a}^{\infty}|f(x)|dx$ сходится $\Leftrightarrow$
$$\forall \varepsilon>0~\exists b_0>a~\forall b_1,b_2>b_0:
\left| \int\limits_{b_1}^{b_2}|f(x)|dx \right|<\varepsilon$$
Так как $\left| \int\limits_{b_1}^{b_2}f(x)dx \right|\leqslant 
\left| \int\limits_{b_1}^{b_2}|f(x)|dx \right|$, то и для
$\int\limits_{b_1}^{b_2}f(x)dx$ выполянется условие Коши. $\square$ 

\textbf{Пример.} Интеграл $\int\limits_{1}^{\infty}\frac{\sin x}{x}dx$ 
сходится условно, так как 
$\int\limits_{1}^{\infty}\frac{\sin x}{x}dx=-\int\limits_{0}^{\infty}
\frac{1}{x}\cos x=-\frac{\cos x}{x}\bigg|^\infty_1+
\int\limits_{1}^{\infty}\cos x\,d(\frac{1}{x})=\cos 1-\int\limits_{1}^{b}
\frac{\cos x}{x^2}$. Второй интеграл сходится по признаку сравнения. 
Но абсолютно он не сходится, так как 
$\left| \frac{\sin x}{x} \right|\leqslant \left| \frac{1-\cos^2}{2x} \right|$,
а инетграл от $\frac{1}{2x}$ расходится. 


\begin{theor}
    (признак Дирихле)\\
    Пусть:\\
    1. $f\in C[a,\infty)$ 
    (и существует интеграл $F(b)=\int\limits_{a}^{b}f(x)dx$);\\
    2. $\exists M=const~\forall b\geqslant a:|f(b)|\leqslant M$;\\
    3. $g'(x)\in C[a,\infty)$;\\
    4. $g'(x)$ знакопостоянна на $(a,\infty)$;\\
    5. $\lim\limits_{x \to \infty}g(x)=0$;\\
    Тогда $\int\limits_{a}^{\infty}f(x)g(x)dx$ сходится. 
\end{theor}
\textbf{Доказательство.} Зафиксируем $\varepsilon>0$. Тогда из условия 5 для
$$\frac{\varepsilon}{4M}~\exists b_0>a~\forall x>b_0:|g(x)|<
\frac{\varepsilon}{4M}$$
Возьмем произвольные $b_1,b_2>b_0$, тогда 
$$\left|\int\limits_{b_1}^{b_2}f(x)g(x)dx\right| =
\left|\int\limits_{b_1}^{b_2}g(x)dF(x)\right| = 
\left| g(x)F(x)\bigg|^{b_2}_{b_1} - \int\limits_{b_1}^{b_2}F(x)g'(x)dx\right|= 
$$
$$
= \left|g(b_2)F(b_2) - g(b_1)F(b_1) - F(\xi)\int\limits_{b_1}^{b_2}g'(x)dx
\right|$$ 
Здесь мы воспользовались теоремой о среденем, взяв $\xi\in(b_1,b_2)$. 
Итак, 
$$ \left| \int\limits_{b_1}^{b_2} f(x)g(x)dx \right|\leqslant 
|g(b_2)F(b_2)|+|g(b_1)F(b_1)|+|F(\xi)g(b_2)|+
|f(\xi)g(b_1)|\leqslant\frac{4M\cdot \varepsilon}{4M}$$ 
Значит, $\int\limits_{a}^{\infty}f(x)dx$ сходится по критерию Коши. $\square$ 

\begin{theor}
    (признак Абеля)\\
    Пусть:\\
    1. $f\in C[a,\infty)$ 
    2. $\int\limits_{a}^{\infty}f(x)dx$ сходится;\\
    %2. $\exists M=const~\forall b\geqslant a:|f(b)|\leqslant M$;\\
    2. $g'(x)\in C[a,\infty)$;\\
    3. $g'(x)$ знакопостоянна;\\
    %5. $\lim\limits_{x \to \infty}g(x)=0$;\\
    4. $\exists M=const~\forall x\geqslant a:|g(x)|\leqslant M$
    Тогда $\int\limits_{a}^{\infty}f(x)g(x)dx$ сходится. 
\end{theor}
\textbf{Доказательство.} Зафиксируем $\varepsilon>0$. Тогда из условия 2 по 
критерию Коши для
$$\frac{\varepsilon}{2M}~\exists b_0>a~\forall b_1,b_2>b_0:\left| 
\int\limits_{b_1}^{b_2}f(x)dx\right| \leqslant \frac{\varepsilon}{2M}$$
Произведем оценку:
$$\left| \int\limits_{b_1}^{b_2}f(x)g(x)dx \right| = 
\left| g(b_2)F(b_2) - g(b_1)F(b_1) - F(\xi)g(b_2) +F(\xi)g(b_1) \right| = 
$$
$$ = \left| g(b_2)\int\limits_{\xi}^{b_2}f(x)dx+ g(b_1)\int\limits_{b_1}^{\xi}
f(x)dx \right| \leqslant 
|g(b_2)|\cdot \left|\int\limits_{\xi}^{b_2}f(x)dx\right| + 
|g(b_1)|\cdot \left|\int\limits_{b_1}^{\xi}f(x)dx\right| <
$$
$$ < M\cdot \frac{\varepsilon}{2M} + M\cdot \frac{\varepsilon}{2M} = 
\varepsilon $$
Значит, интеграл сходится по критерию Коши.



