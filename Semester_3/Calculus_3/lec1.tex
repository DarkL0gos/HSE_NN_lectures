\chapter{Ряды}
В данном разделе мы будем изучать следующие объекты:
\begin{itemize}
    \item Числовые ряды
    \item Функциональные ряды (в т.ч. степенные, ряды Фурье)
\end{itemize}

\section{Числовые ряды}
\subsection{Базовые определения и теоремы}
\begin{defin}
Ряд - сумма счетного числа слагаемых: $$\sum_{n=1}^{\infty} a_n =a_1+a_2+
\ldots$$
\end{defin}
\begin{defin}
Частичная сумма $S_n$ - сумма первых n слагаемых
\end{defin}
\begin{defin}
Сумма ряда - предел последовательности частичных сумм 
$$S=\lim\limits_{n \to \infty}S_n$$
\end{defin}
Если предел существует и конечен, то ряд сходится. Если предел бесконечен, ряд 
расходится. Заметим, что, согласно теоремам о пределе суммы последовательностей
и пределе последовательности, умноженной на число, сходящиеся ряды образуют
линейное пространство относительно сложения и умножения на константу.
\begin{defin}
Остаток ряда - разность между частичной суммой ряда и самим рядом: 
$$R_k=S-S_k=\sum_{n=k}^{\infty} a_k$$
\end{defin}
\textbf{Пример.} Геометрический ряд $a+aq+aq^2+\ldots$. По школьной 
формуле $S_n=\frac{1-q^n}{1-q}$. Имеем случаи:
\begin{enumerate}
    \item $|q|<1:~S=\frac{a}{1-q}$ 
    \item $|q|>1:~S=\infty$
    \item $q=1:~S=\infty$
\end{enumerate}
Итак, ряд сходится, только если $|q|<1$.\\
Следующие теоремы устанавливаются для любых рядов:
\begin{theor}(необходимое условие сходимости ряда)\\
Если ряд сходится, то предел общего члена равен 0.
Равносильная формулировка: если
$\lim\limits_{n\to\infty} a_n\ne0$, то ряд $\sum\limits_{n=1}^{\infty} a_n $
расходится.
\end{theor}
\textbf{Доказательство.} По условию, существует число $S$ - предел частичных 
сумм ряда.
Тогда $\lim\limits_{n \to \infty}a_n=\lim\limits_{n \to \infty}(S_{n}-S_{n-1})
=S-S=0$. 
$\square$

\textbf{Пример.} $\sum\limits_{n=1}^{\infty} \sin{nx},~x\ne\pi k,~k 
\in\mathbb{Z}$. 
Зафиксируем $x$. Допустим, что $\lim\limits_{n \to \infty}\sin nx=0 $. 
Но это противоречит тому, что $\sin^2(x)+\cos^2(x)=1$. Значит, ряд расходится.

\textbf{Пример.} Гармонический ряд расходится, т.к. расходится 
последовательность частичных сумм: 
$S_{2^n}>1+\frac{1}{2}+2\cdot \frac{1}{4}\ldots=1+\frac{n}{2}$
\begin{theor} (критерий Коши сходимости ряда)\\
Ряд $\sum\limits_{n=1}^{\infty} a_n$ сходится тогда и только тогда, когда
$$\forall\varepsilon>0~\exists N(\varepsilon)~\forall n>N~\forall p\in
\mathbb{N}:|a_{n+1}+\ldots+a_{n+p}|<\varepsilon$$
\end{theor}
\textbf{Доказательство.} По определению, ряд сходится, когда существует предел
частичных сумм. Применим к ним критерий Коши, получим условие: 
$|S_{n+p}-S_n|<\varepsilon$. Но $S_{n+p}-S_n\equiv a_{n+1}+...+a_{n+p}$.
$\square$ 
\begin{theor} (критерий сходимости через остаток)\\
    1. Если ряд сходится, то сходится любой из его остатков.\\
    2. Если хотя бы один остаток сходится, то ряд тоже сходится.
\end{theor}
\textbf{Доказательство.}
1. По условию, существует сумма ряда $S$. Зафиксируем номер $N\in\mathbb{N}$ и 
рассмотрим остаток $R_N=\sum\limits_{k=N+1}^{\infty} a_k$, а также 
последовательность $\sigma$ частичных сумм ряда-остатка $R_N$: 
$\sigma_n=a_{N+1}+...+a_{N+n}=\sum\limits_{k=N+1}^{N+n} a_k$.
Рассмотрим её предел:
$\lim\limits_{n \to \infty} \sigma_n=\lim\limits_{n \to \infty}(S_{n+N}-S_N)
=S-S_N=R_N$. Значит, остаток сходится.\\
2. По условию, существует такой номер $n_0$, что остаток $R_{n_0}$ сходится.
Тогда существует предел частичных сумм $\sigma_n$ этого остатка:
$\lim\limits_{n \to \infty}\sigma_n=\sigma$, 
$\sigma_n=a_{n_0}+\ldots+a_{n_0+n}$. Пусть $n_0+n=m$, тогда
$\lim\limits_{n \to \infty}S_m=\lim\limits_{n \to \infty}
(S_{n_0}+\sigma_{m-n_0})=S_{n_0}+\sigma$, то есть основной ряд сходится.
$\square$ 
%\textbf{Пример.} $\sum_{n=1}^{\infty} (\sqrt{n+2} -2\sqrt{n+1} +\sqrt{n} )$.
%Введем $a_n=b_{n+1}-b_n,~b_n=\sqrt{n+1}-\sqrt{n}$ 
%Итак, $S=\lim_{n \to \infty}$.\\
%\textbf{Пример.} $\sum_{n=1}^{\infty} \frac{n}{2^n}=2$
%\subsection{Знакопостоянные ряды}
%Докажем несколько теорем о свойствах знакопостоянных рядов.








