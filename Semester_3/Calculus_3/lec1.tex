\section{Ряды}
\begin{itemize}
    \item Числовые ряды
    \item Функциональные ряды (в т.ч. степенные, ряды Фурье)
\end{itemize}

\subsection{Числовые ряды}
\begin{defin}
Ряд - сумма счетного числа слагаемых: $$\sum_{n=1}^{\infty} a_n =a_1+a_2+
\ldots$$
\end{defin}
\begin{defin}
Частичная сумма $S_n$ - сумма первых n слагаемых
\end{defin}
\begin{defin}
Сумма ряда - предел последовательности частичных сумм $S=\lim_{n \to \infty} 
S_n$
\end{defin}
Если предел существует и конечен, то ряд сходится. Если предел бесконечен, ряд 
расходится.
\begin{defin}
Остаток ряда - разность между частичной суммой и суммой $R_k=S-S_k=
\sum_{n=k}^{\infty} a_k $
\end{defin}
\textbf{Пример.} Геометрический ряд $a+aq+aq^2+\ldots$. Имеем $S_n=\frac{1-q^n
}{1-q}$. Имеем случаи:
\begin{enumerate}
    \item $|q|<1:~S=\frac{a}{1-q}$ 
    \item $|q|>0:~S=\infty$
    \item $q=1:~S=\infty$
\end{enumerate}
Итак, ряд сходится только если $|q|>1$.

Пример. $\sum_{n=1}^{\infty} (\sqrt{n+2} -2\sqrt{n+1} +\sqrt{n} )$.
Введем $a_n=b_{n+1}-b_n,~b_n=\sqrt{n+1}-\sqrt{n}$ 
Итак, $S=\lim_{n \to \infty} $

Пример. $\sum_{n=1}^{\infty} \frac{n}{2^n}=2$
\begin{theor}
    (необходимое условие сходимости ряда). \\ Если ряд сходится, то предел
    общего члена равен 0.\\
    Равносильная формулировка: $\lim_{n \to \infty}a_n\ne_0\implies \sum_{n=1}^{\infty} a_n $ расходится.
\end{theor}
\textbf{Доказательство.} По условию, существвует число - предел ряда.
Тогда $\lim_{n \to \infty}a_n=\lim_{n \to \infty}(S-S_n)=S-S=0$.
$\square$ 

Пример. $\sum_{n=1}^{\infty} \sin{nx},~x\ne\pi k,~k \in\mathbb{Z}$. 
Зафиксируем х. Допустим, что $\lim_{n \to \infty}\sin nx=0 $. Но это противоре
чит тому, что $\sin^2+\cos^2=1$. Значит, ряд расходится.

Пример. Гармонический ряд расходится, т.к. расходится последовательность 
частичных сумм: $S_{2^n}>1+\frac{1}{2}+2\cdot \frac{1}{4}\ldots=1+\frac{n}{2}$

Сходящиеся ряды образуют линйеное пространство!
\begin{theor}
    (критерий Коши сходимости ряда)\\ Ряд $\sum_{n=1}^{\infty} a_n$ сходится 
    $\iff\forall \varepsilon>0\exists N(\varepsilon) \forall n>N \forall p\in\mathbb{N}:|a_{n+1}+\ldots+a_{n+p}|<\varepsilon$
\end{theor}
\textbf{Доказательство.} Ряд сходится ←→ существует предел частичных
сумм. Применим к ним критерий Коши: $|S_{n+p}-S_n|<\varepsilon$
$\square$ 
\begin{theor}
    (критерий сходимости через остаток)\\1. Если ряд сходится, что сходится
    любой из его остатков.\\2. Если хотя бы один остаток сходится, то ряд
    тоже сходится.
\end{theor}
\textbf{Доказательство.} 1. По условию, существует сумма ряда. Рассмотрим
частичный остаток с фиксированным номером $N\in\mathbb{N}$, рассмотрим 
$\sigma=\sum_{k=N+1}^{N+n} a_k$ - последовательность частичных сумм ряда $R_N$.
Предел сигм равен пределу  $(S_{n+N}-S_N)=S-S_N$.

2. По условию, существует такое $n_0$, что  $R_{n_0}$ сходится. Тогда
$\exists \lim_{n \to \infty}\sigma_n=\sigma$, 
$\sigma_n=a_{n_0}+\ldots+a_{n_0+n}$. Пусть $n_0+n=m$, тогда
$\lim_{n \to \infty}S_m=\lim_{n \to \infty} (S_{n_0}+\sigma_{m-n_0})=
S_{n_0}+\sigma$, то есть основный рядсходится.
$\square$ 





