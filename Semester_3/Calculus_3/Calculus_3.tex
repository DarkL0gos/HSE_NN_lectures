\documentclass[a4paper]{book}

%Общие настройки документа
\usepackage[14pt]{extsizes}                                         %Размер шрифта
\usepackage[left=2.5cm,right=2.5cm,top=2.5cm,bottom=3cm]{geometry}  %Поля страницы

%Настройки ссылок и гиперссылок
\usepackage{hyperref}                 
\usepackage{xcolor}
\definecolor{linkcolor}{HTML}{799B03} % цвет ссылок
\definecolor{urlcolor}{HTML}{799B03}  % цвет гиперссылок
\hypersetup{pdfstartview=FitH,linkcolor=linkcolor,urlcolor=urlcolor,colorlinks=true}

%Пакеты символов
\usepackage{cmap}
\usepackage[T2A]{fontenc}
\usepackage[utf8]{inputenc}
\usepackage[russian]{babel}           
\usepackage{amsmath}
\usepackage{amssymb}
\usepackage{amsfonts}

%Новые команды 
\newtheorem{defin}{Определение}
\newtheorem{example}{Пример}
\newtheorem{zam}{Замечание}
\newtheorem{theor}{Теорема}

\author{Галкина}
\title{Анализ}
\date{05.09.2022}

\begin{document}
\maketitle
\tableofcontents
\newpage
%5.09.22
%Коэффициенты: Контр*0,4 Коллок*0,3 Экз*0,3
\chapter{Ряды}
В данном разделе мы будем изучать следующие объекты:
\begin{itemize}
    \item Числовые ряды
    \item Функциональные ряды (в т.ч. степенные, ряды Фурье)
\end{itemize}

\section{Числовые ряды}
\subsection{Базовые определения и теоремы}
\begin{defin}
Ряд - сумма счетного числа слагаемых: $$\sum_{n=1}^{\infty} a_n =a_1+a_2+
\ldots$$
\end{defin}
\begin{defin}
Частичная сумма $S_n$ - сумма первых n слагаемых
\end{defin}
\begin{defin}
Сумма ряда - предел последовательности частичных сумм 
$$S=\lim\limits_{n \to \infty}S_n$$
\end{defin}
Если предел существует и конечен, то ряд сходится. Если предел бесконечен, ряд 
расходитс. Заметим, что, согласно теоремам о пределе суммы последовательностей
и пределе последовательности, умноженной на число, сходящиеся ряды образуют
линейное пространство относительно сложения и умножения на константу.
\begin{defin}
Остаток ряда - разность между частичной суммой ряда и самим рядом: 
$$R_k=S-S_k=\sum_{n=k}^{\infty} a_k$$
\end{defin}
\textbf{Пример.} Геометрический ряд $a+aq+aq^2+\ldots$. По школьной 
формуле $S_n=\frac{1-q^n}{1-q}$. Имеем случаи:
\begin{enumerate}
    \item $|q|<1:~S=\frac{a}{1-q}$ 
    \item $|q|>1:~S=\infty$
    \item $q=1:~S=\infty$
\end{enumerate}
Итак, ряд сходится, только если $|q|<1$.\\
Следующие теоремы устанавливаются для любых рядов:
\begin{theor}(необходимое условие сходимости ряда)\\
Если ряд сходится, то предел общего члена равен 0.
Равносильная формулировка: если
$\lim\limits_{n\to\infty} a_n\ne0$, то ряд $\sum\limits_{n=1}^{\infty} a_n $
расходится.
\end{theor}
\textbf{Доказательство.} По условию, существует число $S$ - предел частичных 
сумм ряда.
Тогда $\lim\limits_{n \to \infty}a_n=\lim\limits_{n \to \infty}(S_{n}-S_{n-1})
=S-S=0$. 
$\square$

\textbf{Пример.} $\sum\limits_{n=1}^{\infty} \sin{nx},~x\ne\pi k,~k 
\in\mathbb{Z}$. 
Зафиксируем $x$. Допустим, что $\lim\limits_{n \to \infty}\sin nx=0 $. 
Но это противоречит тому, что $\sin^2(x)+\cos^2(x)=1$. Значит, ряд расходится.

\textbf{Пример.} Гармонический ряд расходится, т.к. расходится 
последовательность частичных сумм: 
$S_{2^n}>1+\frac{1}{2}+2\cdot \frac{1}{4}\ldots=1+\frac{n}{2}$
\begin{theor} (критерий Коши сходимости ряда)\\
Ряд $\sum\limits_{n=1}^{\infty} a_n$ сходится тогда и только тогда, когда
$$\forall\varepsilon>0~\exists N(\varepsilon)~\forall n>N~\forall p\in
\mathbb{N}:|a_{n+1}+\ldots+a_{n+p}|<\varepsilon$$
\end{theor}
\textbf{Доказательство.} По определению, ряд сходится, когда существует предел
частичных сумм. Применим к ним критерий Коши, получим условие: 
$|S_{n+p}-S_n|<\varepsilon$. Но $S_{n+p}-S_n\equiv a_{n+1}+...+a_{n+p}$.
$\square$ 
\begin{theor} (критерий сходимости через остаток)\\
    1. Если ряд сходится, то сходится любой из его остатков.\\
    2. Если хотя бы один остаток сходится, то ряд тоже сходится.
\end{theor}
\textbf{Доказательство.}
1. По условию, существует сумма ряда $S$. Зафиксируем номер $N\in\mathbb{N}$ и 
рассмотрим остаток $R_N=\sum\limits_{k=N+1}^{\infty} a_k$, а также 
последовательность $\sigma$ частичных сумм ряда-остатка $R_N$: 
$\sigma_n=a_{N+1}+...+a_{N+n}=\sum\limits_{k=N+1}^{N+n} a_k$.
Рассмотрим её предел:
$\lim\limits_{n \to \infty} \sigma_n=\lim\limits_{n \to \infty}(S_{n+N}-S_N)
=S-S_N=R_N$. Значит, остаток сходится.\\
2. По условию, существует такой номер $n_0$, что остаток $R_{n_0}$ сходится.
Тогда существует предел частичных сумм $\sigma_n$ этого остатка:
$\lim\limits_{n \to \infty}\sigma_n=\sigma$, 
$\sigma_n=a_{n_0}+\ldots+a_{n_0+n}$. Пусть $n_0+n=m$, тогда
$\lim\limits_{n \to \infty}S_m=\lim\limits_{n \to \infty}
(S_{n_0}+\sigma_{m-n_0})=S_{n_0}+\sigma$, то есть основной ряд сходится.
$\square$ 





%\textbf{Пример.} $\sum_{n=1}^{\infty} (\sqrt{n+2} -2\sqrt{n+1} +\sqrt{n} )$.
%Введем $a_n=b_{n+1}-b_n,~b_n=\sqrt{n+1}-\sqrt{n}$ 
%Итак, $S=\lim_{n \to \infty}$.\\
%\textbf{Пример.} $\sum_{n=1}^{\infty} \frac{n}{2^n}=2$
%\subsection{Знакопостоянные ряды}
%Докажем несколько теорем о свойствах знакопостоянных рядов.







%Лекция+семинар 08.09.22
% Продолжаем топологию 15/09/2022
\section{База топологии}
\begin{defin}
    Пусть $(X,\tau)$ - топологическое пространство
    Семейство  $\Sigma=\{W_\beta\subset X\mid \beta\in B \} $ - база топологии,
    если удовлетворяет двум условиям:\\
    1. $\Sigma\in\tau~\forall W_\beta\in\Sigma$
    2. Любое открытое подмножество Х можно представить в виде
    объединения некоторых подмножеств из $\Sigma$: 
    $\forall U\in\tau\exists W_\alpha\in\Sigma,~\alpha\in A\subset B:
    U=\Cup\limits_{\alpha\in A}W_\alpha $
\end{defin}
\textbf{Пример}. В обычной (евклидовой) топологии множество 
$\Sigma=\{D_r(a)\mid a\in\mathbb{R}^n,r>0\}$ является базой топологии.
Действительно, проверим аксиомы:\\
1. Открытая окрестность открыта.\\
2. По определению обычной топологии,каждая точка в открытом множестве
содержится в нем с некоторой окрестностью. Значит, объединение этих
окрестностей дает это множество. Более формально,
$\forall u\in\tau,\forall x\in U\Rightarrow \exists D_{\varepsilon_x}(x):
D_{\varepsilon_x}(x)\in U$. Очевидно доказывается. что 
$$\boxed{\bigcup_{x\in U}D_{\varepsilon_x}(x)=U}$$ 
\textbf{Замечание.} Если к базе добавить произвольное открытое множество, то
новое множество также будет базой.\\
\textbf{Упражнение.} Привести пример двух баз евклидовой топологии на 
плоскости, которые не пересекаются с обычной базой (открытых шаров). 
(Решение: например, база из открытых квадратных или звездчатых окрестностей).\\
\textbf{Пример.} В $(\mathbb{R}^2,\tau_{MN})$, 
$\Sigma_{MN}=\{(b,b^*)\mid b\in\} $ !!!!!!!!!!!!!!!!!!!!!\\
\textbf{Пример.} Топология ираациональных точек на прямой
$(\mathbb{R},\tau_{im}),~\tau_{im}=\{\varnothing,\mathbb{R}\}\cup
\{U\subset \mathbb{R}\setminus\mathbb{Q}\} $.
Множество иррациоанльных точек не является базой, поскольку их объединение не
содержит всю прямую. Решение: добавить саму прямую. !!!!!!!!!!!!\\

\begin{theor}
    (критерий базы в топологическом пространстве)\\
    Пусть $(X,\tau)$ - опологическое пространство, и семейство множеств 
    удовлетворяет условию $\sigma\subset \tau$. $\Sigma$ является базой 
    топологии тогда и только тогда, когда 
    $\forall u\in\tau,\forall x\in U\exists W_{\beta_0}\in\Sigma:
    x\in W_{\beta_0}\subset U$
\end{theor}
\textbf{Доказательство.} Пусть $\Sigma$ - база топологии. Тогда любое открытое 
множество можно представить в виде объединений множеств из базы. Значит, для
$x\in U$ найдется множество из базы, в котором лежит $x$.  \\
Обратно. Множесто $\Sigma$ удовлетворяет первой аксиоме базы по определению.
Докажем выполнение второй аксиомы. Для любой точки в открытом множестве
по условию теоремы найдется окрестность из $\Sigma$, лежащая в открытом
множестве. 
!!!!!!!!!!!!!!!!!!!!!!
$\square$ 

\begin{theor}
    (критерий базы на множестве)\\
    Пусть Х - произвольное множесто, $\Sigma=\{W_\beta\subset X\mid\beta
    \in B\}$ - семейство подмножеств из Х. ЧТобы на Х существовала
    топология с данной базой, необходимо и достаточно выполнения
    двух условий:\\
    1. $X=\bigcup\limits_{\beta\in B} W_\beta$\\
    2. Для любых множеств из базы найдется множество, лежащее в их
    пересечении и содержащее произвольную точку оттуда.

\end{theor}
\textbf{Доказательство.} Необходимость. Пусть $\Sigma$ - база некотрой 
топологии (Х,т). Из акиомы базы (2) следует,что что Х есть объединение
множеств из $\Sigma$. значит, выполняется первое условие теоремы. Докажем второе 
условие. Достаточно взять пересечение двух множеств из базы. Так как 
это открытые множества, его также можно представить в виде объединения
множеств из базы, и хотя бы в одном из которых лежит фиксированная точка
(по определению объединения).\\
Достаточность. Докажем, что всевозможные объедения множеств из $\Sigma$  
является топологией. пусть это есть $\tau$. Проверим аксиомы топологии:\\
1.  Пустое множество принадлежит всему, чему надо. Все простарнство 
лежит там по условию теоремы.
3. Пусть 


$\square$ 



%Лекция 19.09
\section{Элементарные методы интегрирования ДУ}
\subsection{Уравнения с разделяющимися переменными}
\begin{defin}\label{ODE_razdp}
Уравнение с разделяющими переменными - уравнение вида
\begin{equation}
    \frac{dx}{dt}=f(x)g(t) 
\end{equation}
где $f,g$ непрерывны на  $x\in(a,b),~t\in(\alpha,\beta)$
\end{defin}
Как решать такие уравнения? Алгебраическая интуиция подсказывает, что надо 
перенести 
дифференциалы к своим функциям и проинтегрировать. Но это ещё надо обосновать.
Сделаем следующее:\\
\begin{enumerate}
    \item Найти все $x_*:f(x_*)=0$. Тогда $x=x_*$ - решение-константа. 
    \item Пусть  $x^i_*,x^j_*$ - такие, что  $f(x^i_*)=f(x^j_*)=0$ и
    $\forall x\in(x^i_*,x^j_*):f(x)\ne0$. Тогда уравнение \ref{ODE_razdp}
эквивалентно уравнению 
$$\frac{dx}{f(x)}=g(t)dt$$
Эту штуку можно проинтегрировать с обеих сторон. Результат непрерывен и не
обращается в ноль. Значит, по теореме о неявной функции найдется решение. 
$\frac{dF}{dx}=\frac{1}{x}$(решение в области $(\alpha,\beta)\times
(x^i_*,x^j_*)$).
    \item Выписать решение на каждом интервале $(x^i_*,x^j_*)$
\end{enumerate}
Других решений не существует. Почему? Допустим, существует другое решение.
Оно не может быть константой, так как все константы были получены в п.1.
Если она \\
\textbf{Пример.} Решим уравнение $\frac{dx}{dt}=0$. Решение-константа: $x=0$.
Теперь рассмотрим два интервала: $x<0$ и  $x>0$. Если  $x<0$, имеем уравнение
 $$\frac{1}{x}\frac{dxdt}{dt}=dt$$
 Интегрируем:
 $$\int\frac{dx}{x}=\int dt$$
 Получаем, что $\ln|x|=t+C$. Выражаем искомую функцию (не забыв, на каком
 промежутке мы рассматриваем функцию, и раскрыв модуль соответственно):
 $$x=-Ce^t,~C>0$$
Для интервала $x>0$ точно такой же порядок действий, только получим другой 
знак. Итак, множество решений:
$$x=Ce^t,~C\in\mathbb{R}$$
\subsection{Уравнения, приводящиеся к уравнению с разделяющимися переменными}
\begin{defin}
Уравнение, приводящееся к уранвению с разделяющмися переменными - уравнение
вида 
\begin{equation}
    \frac{dx}{dt}=f(at+bx+c) \label{ODE_privrazd}
\end{equation}
\end{defin}
Давайте решим его. 
\begin{enumerate}
    \item Введем замену $z(t)=at+bx+c$. 
    Имеем
     $$\frac{dz}{dt}=a+b\frac{dx}{dt}$$ 
     Получаем уравнение с разделяющимися переменными. 
     $$\frac{dz}{a+bf(z)}=dt$$
\end{enumerate}
\textbf{Пример.} Решим уравнение $\frac{dx}{dt}=\cos(x+t)$. Замена 
$z=x+t,~ \frac{dz}{dt}=1$. Уравнение имеет вид
$$\frac{dz}{dt}=\frac{dx}{dt}+1$$ 
Найдем $\cos{z_*}+1=0$: это, очевидно, $\pi+2\pi k,~k\in \mathbb{Z}$ 
Свели задачу кпрошлому пункту
\subsection{Однородные уравнения}
Сначала докажем, что два определения однородного уравнения эквивалентны.
\begin{defin}\label{ODE_odn}
Однородным называется уравнение вида
\begin{equation}%\label{ODE_odn1}
    \frac{dx}{dt}=f\left(\frac{x}{t}\right) 
\end{equation} 
\end{defin}
Это уравнение инвариантно относительно замены $x\mapsto kx,~t\mapsto kt$.
Геометрически это означает, что совокупность интегральных кривых инвариантно
относительно преобразования $\theta(x,y)=(kx,ky)$.
Из этого следует, что если мы найдем одно решение, то мы найдем всю 
совокупность ему подобных. %Вставить картинку.
\begin{defin}
    (вспомогательное)\\
Уравнение в форме дифференциалов:
    $M(x,y)dx+N(x,y)dy=0$.  
\end{defin}
Это таже форма, что и $\frac{dy}{dx}=f(x,y)$, поскольку 
$\frac{dy}{dx}=-\frac{M(x,y)}{N(x,y)}$. Обратно, $-f(x,y)dx+dy=0$.
Уравнение в форме дифференциалов имеет чуть большее множество решений. 
\begin{defin}\label{ODE_odn2}
Уравнение в форме дифференциалов называется однородным, если\\
$M(kx,ky)=k^nM(x,y)$\\ 
$N(kx,ky)=k^nN(x,y)$\\
n называется степенью однородности.
\end{defin}
\begin{theor}
    Определения \ref{ODE_odn} и \ref{ODE_odn2} эквивалентны. 
\end{theor}
\textbf{Доказательство.} 1 $\Rightarrow$ 2. $\frac{dy}{dx}=f(\frac{y}{x})$\\
2 $\Rightarrow$ 1. Пусть дано уравнение в форме дифференциалов. Подставим $k$.
При $x\ne 0$ имеем 
$$\frac{dx}{dy}=-\frac{k^nM(x,y)}{k^nN(x,y)}=
-\frac{M(kx,ky)}{N(kx,ky)}=-\frac{M(1,\frac{y}{x})}{N(1,\frac{y}{x})}=
f\left( \frac{y}{x} \right) $$
$\square$ \\
\textbf{Пример.} $M=x^2+y^2$\\
\textbf{Пример (№31).} Найти уравнение, решение которых - параболы с осью, 
параллельной оси ординат и касающиеся прямых $y=0,~y=x$. 
Во-первых, поймем, как выглядит уравнение такой параболы. Исходя из геометрии,
получим, что уравнение параболы, удовлетворяющее первому условию, имеет вид 
$y=ax^2+bx+\frac{b^2}{4a}$, а первому и второму - $y=ax^2+\frac{1}{2}x+
\frac{1}{16a}$. Остался один параметр $\Rightarrow$ уравнение первого порядка. 
Подставляем и хаваем ответ бесплатно:
$$y=\left(\frac{y'-\frac{1}{2}}{2x}\right)x^2+\frac{1}{2}x+\frac{2x}{16y'-8}$$ 
\textbf{Пример (№72).} Найти линии, у которых треугольники, образованные 
касательными, осью ОХ и точкой касания, имеют одинаковую сумму катетов. 
Из геометрических соображений имеем уравнение 
$$\frac{|y|}{|y'|}+|y|=b=const$$ 
Раскрываем модули. В простейшем случае имеет уравнение с разделяющимися 
переменными. 
$$\frac{dy}{dx}=\frac{y}{b-y}$$ 
Остальные уравнения такие же в принципе. Так шо это идет в дз 
Его легчайшее (и, видимо, общее) решение: $x+C=\pm b\ln{|y|}\pm y$\\
\textbf{Пример (№76).} Геометрическая интуиция не должна подводить нас. 
Вставить картинку. Есть кароч такая формула: 
$\tg\gamma=\frac{r}{r'}$





%Лекция 22.09
\section{Однородное уравнение}
$$\frac{dx}{dt}=f\left(\frac{x}{t}\right)$$ 
Как искать его решение? Заменой $u(t)=\frac{x}{t}$. 
Тогда уравнение перепишется в виде $\frac{dx}{dt}=\frac{du}{dt}t+u$.
В нем переемнные разделяются: $\frac{du}{f(u)-u}=\frac{dt}{t}$. 
Итак, типы уравнений:
\begin{enumerate}
    \item С разделяющимися переменными
    \item Приводящиеся к виду $\frac{dx}{dt}=f(ax+bx+c)$ 
    \item Првиодящиеся к виду $(a_1x+b_1t+c_1)dx+(a_2x+b_2x+c_2)dt=0$
\end{enumerate}
Подумаем, можно ли это последнее привести к однородному. Добавим условие 
$c_1^2+c_2^2\ne0$ (иначе система уже однородна). В общем, если эти 
две прямые пересекаются в точке
$(x_*,t_*)$, то можно ввести новые переменные, передвинув эту точку в начало
координат: $x\mapsto x-x_*,~t\mapsto t-t_*$. Тогда система перепишется 
без $c_1,~c_2$, и таким образом будет однородной. Если прямые не пересекаются, 
то прямые либо совпадают, либо параллельны. Тогда введем замену (для любой 
прямой) $z(t)=a_1x+b_1t+c_1$. Так как прямые параллельны, то
$\frac{a_1}{a_2}=\frac{b_1}{b_2}=k$, значит, мы можем выразить 
вторую прямую: $a_2x+b_2t+c_2=\frac{1}{k}(a_1x+b_1t+kc_2)=\frac{1}{k}(z-
c_1+kc_2)$. Уравнение приводится к виду $z(t)dx+\frac{1}{k}(z-c_1+kc_2)dt=0$.
Но у нас все равно многовато переменны. Выразим $dx$ через  $z$:
$$z(\frac{dz-b_1dt}{a_1})+\frac{1}{k}(z-c_1+kc_2)=0$$ 
Умножим на $a_1k$:
 $$kzdz=kb_1zdt-a_1zdt-a_1(kc_2-c_1)dt$$ 
Домножим на $\frac{1}{kzdt}$:
$$\frac{dz}{dt}=((b_1-\frac{a_1}{k})z-a_1(kc_2-c_1))\frac{1}{z}$$ 
Finally, уравнение с разделяющимися переменными! ПОБЕДА!
\subsection{Обобщенно-однородное уравнение}
\begin{defin}
Обобщенно-однородное уравнение - уравнение вида
$$M(x,t)dx+N(x,t)dt=0$$ 
причем $M,N$ - такие. что  $\exists n\in\mathbb{R}$: если
$x=z^n(t)$, то уравнение  $M(z^n,t)nz^{n-1}dz+N(z^n,t)dt=0$ однородно.
\end{defin}
\textbf{Пример.} Испортим однородное уравнения, чтобы сделать его 
обощенно-однородным. Роман придумал, чел харош.

Сведем и этого зверя к разделяющимся переменным. 
$$\begin{cases}
    n(kz)^{n-1}M((kz)^n,kt)=k^mM(z^n,t)nz^{n-1}\\
    N((kz)^n,kt)=k^mN(z^n,t)
\end{cases}$$ 

\subsection{Уравнение в полных дифференциалах}
Напомним, что полный дифференциал $dF(x,y)$  $C^1$-гладкой функции
равен  $\frac{\partial F}{\partial x} dx+\frac{\partial F}{\partial y}dy$.
\begin{defin}
Уравнение в полных дифференциалах - уравнение вида
$$dF(x,y)=0,~F\in C^2(\Omega),~\Omega\subset \mathbb{R}^2$$
\end{defin}
Если мы знаем саму функцию, то решение находится мгновенно: $dF(x,y)=const$.
Правда, оно неявное. Выразим  $y=y(x)$ по теореме о неявной функции.\\
\textbf{Пример.} $x^2\sin{t}dt+2x\cos{t}dx=0$\\
Уравнение является уравнение в полных дифференциалах, если существуют такие
функции, что $M=\frac{\partial F}{\partial x},~N=\frac{\partial F}{\partial y}$
\begin{theor}
    (необходимое условие представления в полных дифференциалах)\\
    $$\frac{\partial M}{\partial y}=\frac{\partial N}{\partial y}$$ 
    Достаточное условие - $M_y=N_x$ в односвязной области
\end{theor}
\textbf{Доказательство.} $\square$ \\
Как подбирать такие функции? Мы знаем, что $\frac{\partial F}{\partial x}=
M(x,y)$. Проинтегрируем это равенство по $x$. Имеем  $F=\int M(x,y)dx+
\varphi(y)$. Проделаем то же самое по переменной $y$:  $\frac{\partial F}
{\partial y}=\frac{\partial }{\partial y}(\int M(x,y)dx)+\varphi'=N(x,y)$,
откуда $\varphi=\int\left(N-\frac{\partial }{\partial y}(\int Mdx) \right)dy$.
Чтобы проверить себя при решении, помним, что $\varphi$ не зависит от $x$!
Итак,
$$F=\int M(x,y)dx+\int\left(N-\frac{\partial }{\partial y}\left(\int Mdx
\right)\right)dy$$
\subsubsection{Геометрический смысл решения уравнения в полных дифференциалах}
Так как $z=z(x,y)$ - какая-то поверхность, то запись  $z=const$ - это линии
уровня, которые можно спроецировать на плокость переменных и получить 
интегральные кривые.\\
\textbf{Пример (модель Лотки-Вольтерра).} Пусть $x(t)$ - плотность карасей,
 $y(t)$ - плотность щук в некотором пруду. Щуки сдерживают рост карасей,
но от количества карасей зависит также и количество щук. Запишем систему:
$$\begin{cases} \label{lotka_volterra}
    \dot{x}=x(a-by)\\
    \dot{y}=y(-c+ex)
\end{cases}$$
Лотка придумал эту систему для биоценозов, а Вольтерра - для химических
реакций.\\
Давайте решим эту систему. Её расширенное фазовое пространство, вообще говоря,
трехмерное, поэтому будем рассматривать фазовые кривые - проекции интегральных 
на плоскость независимых параметров. Они ориентированы в направлении роста
параметра $t$. Найдем эти кривые, найдя решение уравнения 
$\frac{dy}{dx}=\frac{dy /dt}{dx /dt}=\frac{-cy+exy}{ax-bxy}$. 
Переменные разделяются: $$\frac{(a-by)dy}{y}=\frac{(-c+ex)dx}{x}$$ 
Представим его в полных дифференциалах: 
$$d\left( a\ln y-by+c\ln x -ex \right)=0$$
Значит, решение имеет вид $a\ln y-by+c\ln x-ex=h=const$.
Выглядит очень сложно, но давайте попробуем построить изолинии. 
Введм функцию $F=\ln{(y^ax^c)}-by-ex$, и поищем её изолинии. Сначала найдем
критические точки: $(x_*,y_*)=(\frac{c}{e},\frac{a}{b})$ (получилась 
единственная точка). Определим тип критической точки (составим гессиан, 
посчитаем его знакоопределенность); получим, что это точка максимума.
Линии уровня - какие-то окружности/эллипсы.\\
\begin{figure}[H]
    \centering
    \input{figures/Lotka-Volterra.pdf_tex} %для pdf_tex
    %\includegraphics{}  % для png, pdf
    \caption{Первый интеграл системы Лотки-Вольтерры}
    \label{fig:}
\end{figure}
\textbf{Упражнение.} Доказать, что фазовые кривые замкнуты.

\textbf{Доказательство (\cite{Arnold},\S 2.7).} Так как 
$\frac{dF}{dt} = 0$, то функция $F$ сохряняется вдоль фазовых кривых. 
Иначе говоря, фазовые кривые являются изолиниями фунции $F$. Но 
график  $F$ является суммой двух <<холмов>>, образованных логарифмами, 
и поэтому сам является холмом. Посколько изолинии холма - замкнутые кривые,
то и фазовые кривые системы Лотки-Вольтерры замкнуты. $\square$\\
Теперь нам надо понять, куда закручиваются эти линии, как они ориентированы.
Они закручиваются против часовой стрелки вокруг критической точки, кстати,
область решения - первая координатная четверть. Чтобы избежать проблем с 
дискретностью, наши переменные - это плотность населения пруда. 

 



%Гомеоморфный (без самопересечений, так как биекция) образ окружности в
%плоскости можно непрерывно продеформировать в точку. Теорема Жордана: 
%гомеоморфный образ окружности делит плоскость на две компоненты связности. 
%Теорема Шёнфлиса: внутренняя часть этой плоскости гомеоморфна открытому 
%диску. Это кстати эквивалентно тому, что фундаментальная группа 
%тривиальна. 

%лекция 
\begin{defin}
Автономное ДУ - дифференциальное уравнение, правая часть которого не зависит
от времени.
\end{defin}
Автономные уравнения не могут быть динамическими системами, так как они 
не зависят от времени, но можно искусственно этого достичь.\\
\textbf{Пример.} Нелинейный консервативный
осциллятор. Рассмотрим маятник с координатами
$\varphi$ - отклонение от положения равновесия. Рассмотрим плоские
колебания маятника массой $m$ и длиной $l$. При повороте на малый угол 
движение можно представить как прямолинейное движение по касательной.
Запишем второй закон Ньютона в проекции на касательную: 
$$\vec\tau:~m\frac{d^2x}{dt^2}=-mg\sin\varphi$$ 
Пусть $\Delta x$ - длина дуги окружности, примерно равная малой части 
касательной. Тогда  $\Delta x=l\Delta\varphi+o(\Delta\varphi)$.
Получим уравнение
$\frac{fx}{dt}=l \frac{d\varphi}{dt}$. Finally,
$$ml \frac{d^2\varphi}{dt^2}=-mg\sin\varphi$$ 
- уравнение колебания маятника. Оно нелинейное из-за синуса. 
Оно имеет порядок 2, значит, нам надо
зафиксировать начальные условия: $\varphi(0),~\dot\varphi(0)$.
Уравнение тогда превратится в систему
$$\begin{cases}
    \dot\varphi=\psi\\
    \dot\psi=-\omega^2\sin\varphi
\end{cases}$$
Кстати, если мы напишем функцию Лагранжа и напишем уравнение Лагранжа для
него, то получим это же уравнение.\\
Начнем решение. Сделаем замену $\dot\varphi=\psi$. Теперь введем фазовое 
пространство угол-скорость таким образом, чтобы близкие точки были близки.
В угловых координатах мы склеим точки $\pi,-\pi$ у координат углов (точнее,
создадим факторпространство по отношению $(\varphi,\psi)\sim(\varphi+2\pi k,
\psi)$). Получим, что фазовое пространство - цилиндр. Любая замкнутая кривая 
- это некоторая траектория (вообще говоря, определляемая уравнением).
На цилиндре есть два типа замкнутых кривых - стягиваемые в точку и 
нестягиваемые. Вторые отвечают за движение через верх. \\
Продолжаем решение. Из системы имеем 
$\frac{d\psi}{d\varphi}=-\frac{\omega^2\sin\varphi}{\psi}$.
Полная энергия равна константе:
$\frac{m\psi^2}{2}+\frac{mg}{l}(1-\cos\varphi)=h$.
Это мы вывели из формы уранвения в полных диференциалах. 
В общем, решаем. Получим
$$\varphi=\pm\sqrt{\frac{2}{m}\left( h-\frac{mg}{l}(1-\cos\varphi) \right) }$$ 
Нарисуем фазовые траектории, и ещё функцию 
$F(\varphi)=\frac{mg}{l}(1-\cos\varphi)$.
Уровни постоянной энергии - одномерные торы. Как и обычно с функцией
Гамильтона. 
Из анализа фазовых траекторий можно выяснить, что период колебаний растет по
мере увеличения энергии. Также есть два состояния равновесия: верхнее 
(неустойчивое) и нижнее (устойчивое). 










%лекция 03.10.22
\section{Уравнения и ряды Тейлора}
Пусть $\frac{dx}{dt}=f(t,x)$. Рассмотрим $x(t_0)=x_0$. Разложим в ряд
Тейлора: $x(t)=x(t_0)+\frac{dx}{dt}(t_0)(t-t_0)+o(t-t_0)$.
Отбросив члены высшего порядка (прямо как топовые физики), получим 
приближенное решение. Приближенные решение можно итерировать, и это
будет широко известный \textbf{метод Эйлера} (первого
порядка). $t_{k+1}=t_k+h,~x_{k+1}=x_k+f(t_k,x_k)h$ 

\section{Практика}
\textbf{Пример (№111)}. $(y+\sqrt{xy})dx=xdy$. Уравнение однородно (
проверим умножением на $k$). Значит, делаем замену $u(x)=\frac{y}{x}$.
Имеем $dy=u\cdot dx+du\cdot x$. Переменные разделяются: 
$\frac{dx}{x}=\frac{du}{\sqrt{u}}$\\
\textbf{Пример (№113)}. $(2x-4y+6)dx+(x+y-3)dy$. Переносим начало координат
в точку пересечения.\\
\textbf{Пример (№126)}. $y'=y^2-\frac{2}{x^2}$. Это - обобщенно-однородное
уравнение, то есть приводится к однородному заменой $y=z^m(x)$.
$y'=mz^{m-1}z$ Далее
$mz^{m-1}z=z^{2m}-\frac{2}{x^2}$ 
Теперь уравнение однородно. Введем замену $\frac{z}{x}=u,~z=ux$.
Получим $u'x+u=-1+2u^2$\\
\textbf{Пример (№128)}. $\frac{2}{3}xyy'=\sqrt{x^6-y^4}+y^2$. 
Пусть $y=z^m$. Идея: сделать так, чтобы под корнем степень у $x$ и $y$ была
одинаковой.\\
\textbf{Пример (№)} $2xydx+(x^2-y^2)dy=0$. Подберем функцию, полным 
дифференицалом которого является это выражение; получим  $F(x,y)=
x^2y-\frac{1}{3}y^3$. Решние: $F=C=const$\\
\textbf{Пример (№192)}. $(1+y^2\sin{2x})dx-2y\cos^2{x}dy$. Мы должны 
показать, что вторые производные равны. Тогда это значит, что
$F_{xy}=F_{yx}$, и такая функция вообще существует на некотором диске
(где правая часть не обращается в ноль). Интегируем два раза, и найдем эту
функцию: $F(x,y)=x-y^2 \frac{1}{2}\cos{2x}-\frac{y^2}{2}+C_0$.
Итак, ответ: $\boxed{F=const}$ \\
\textbf{Пример (№202)}. $y^2dx+(xy+\tg{xy})dy=0$. Является ли однородным,
в полных дифференциалах? Давайте раскроем скобки и сгруппируем:
$y(ydx+xdy)+\tg{xy}dy$. Это то же, что и  $\frac{d(xy)}{\tg{xy}}+\frac{dy}{y}
=0$. Домножим на $\frac{1}{y\tg{xy}}$ и хаваем уравнение в полных 
дифференицалах бесплатно. То, на что домножили - интегрирующий множитель.















%лекция 10.10.22
\begin{theor}
    (признак Абеля равномерной сходимости функционального ряда)\\
Дан ряд $\sum\limits_{n=1}^{\infty} a_n(x)b_n(x)$ и $\forall x\in X$:\\
1. $|a_n(x)|\leqslant M=const$ для всех $n$;\\
2.  $\{a_n(x)\} $ мнонотонна;\\
3. $\sum\limits_{n=1}^{\infty} b_n(x)$ равномерно сходится на $X$;\\
Тогда исходный ряд равномерно сходится на  $X$.
\end{theor}
\textbf{Доказательство.}  По определению Коши. Фиксируем $\varepsilon>0$.
Так как ряд с общим членом $b_n$ сходится равномерно, то по критерию Коши для
$$\frac{\varepsilon}{3M}>0~\exists n_0(\varepsilon)~\forall n>n_0~\forall p\in
\mathbb{N}~ \forall x\in X:\left|\sum\limits_{k=n+1}^{n+p} b_k(x)\right|<
\frac{\varepsilon}{3M}$$ Тогда по неравенству Абеля 
$$\left|\sum\limits_{k=n+1}^{n+p} b_k(x)a_k(x)\right|\leqslant 
\frac{\varepsilon}{3M}(|a_{n+1}|+2|a_{n+p}(x)|)<\frac{\varepsilon}{3M}\cdot 
3M=\varepsilon$$
Тогда по критерию Коши этот ряд сходится равномерно на $X$. $\square$ 

\textbf{Пример.} Исследуем на равномерную сходимоcть ряд 
$\sum\limits_{n=1}^{\infty} \frac{\cos{nx}\sin{x}arctg{nx}}{\sqrt{n^2+x^2}}$. 
Алгоритм:\\
1. Арктангенс монотонен и ограничен.\\
2. Все остальное сходится по Дирихле.
\subsection{Свойства равномерно сходящихся рядов}
\begin{theor}
    (о непрерывности суммы равномерно сходящегося ряда)\\
    Дан ряд $\sum\limits_{n=1}^{\infty} a_n(x)$, причем \\
    1. Все функции непрерывны на множестве $X$;\\
2. $\sum\limits_{n=1}^{\infty} a_n(x)$ сходится равномерно к $S(x)$ на $X$;\\
Тогда $S(x)$ непрерывна на $X$. 
\end{theor}
\textbf{Доказательство.}  По условию, сумма из  $a_n(x)$ сходится равномерно
на  $X$ к  $S(x)$, то есть  $S_n(x)\rightrightarrows S(x)$ на  $X$, 
$S_n(x)$ непрерывна как сумма. Тогда по теореме о непрерывности предела
равномерно сходящейся последовательности, составленной из непрерывных
функций,  $S(x)$  непрерывна. Другая формулировка:  
$$\lim\limits_{x\to x_0}\sum\limits_{n=1}^{\infty} a_n(x) =
\sum\limits_{n=1}^{\infty}\lim\limits_{x\to x_0} a_n(x)
$$
(то есть можно поменять местами сумму и предел). $\square$ 

\textbf{Пример.} $\sum\limits_{n=1}^{\infty} \frac{\sin{nx}}{n}=f(x)$ - 
непрерывна на $(0,2\pi)$
\begin{theor}
(об интегрировании равномерно сходящегося ряда)\\
Пусть дан ряд $\sum\limits_{n=1}^{\infty} a_n(x)$, причем \\
 1. все функции непрерывны на отрезке $[a,b];$\\
 2. $\sum\limits_{n=1}^{\infty} a_n(x)$ сходится равномерно на $[a,b]$ к $s
 (x)$;\\
 Тогда $$\forall x,x_0\in[a,b]:~\int\limits^x_{x_0}\left( \sum\limits_{n=1}^
 {\infty} a_n(t) \right)dt=\sum\limits_{n=1}^{\infty} \left( 
\int\limits_{x_0}^{x}a_n(t)dt \right) $$ 
 (можно менять интеграл и сумму).
\end{theor}
\textbf{Доказательство.} Докажем, что $\int\limits^x_{x_0}S(t)dt=\sum\limits_{n=1}^{\infty} \int\limits^x_{x_0}a_n(t)dt$. По предыдущей теореме $S(t)$ 
непрерывна на  $[a,b]$, значит,интегрируема на нем по Риману. 
Обозначим  $\sigma_n(x)=\sum\limits_{k=1}^{n}\int\limits^x_{x_0}a_k(t)dt$ и
докажем, что $\sigma_n(x)\rightrightarrows\int\limits^x_{x_0}S(t)dt$.\\
Зафиксируем $\varepsilon>0$. По условию, $S_n(t)$ равномерно сходится на 
$[a,b]$ для  
$$\frac{\varepsilon}{b-a}>0~\exists n_0(\varepsilon)~\forall 
n>n_0~\forall x\in[a,b]:|S_n(t)-S(t)|<\frac{\varepsilon}{b-a}$$
Тогда
$\left|\sigma_n(x)-\int\limits^x_{x_0}S(t)dt\right|=
\left| \sum\limits_{k=1}^{n} \int\limits_{x_0}^{x} a_k(t)dt-
\int\limits_{x_0}^{x}S(t)dt\right|=\left| \int\limits_{x_0}^{x}(S_n(t)-S(t))dt
\right|\leqslant \left| \int\limits_{x_0}^{x}|S_n(t)-S(t)|dt\right| 
<\frac{\varepsilon}{b-a}\cdot |x-x_0|<\varepsilon$.
Значит, $\sigma_n(x)\rightrightarrows \int\limits_{x_0}^{x} S_n(t)dt$.
$\square$ 
\begin{theor}
(о дифференцировании равномерно сходящегося ряда)\\
Пусть дан ряд $\sum\limits_{n=1}^{\infty} a_n(x)$, причем \\
 1. Производные всех функций непрерывны на отрезке $[a,b];$\\
 2. $\sum\limits_{n=1}^{\infty} a_n(x)$ сходится на $[a,b]$ поточечно;\\
 3. Ряд из производных сходится равномерно на $[a,b]$ к  $S(x)$;\\
 Тогда 
$$\sum\limits_{n=a}^{\infty} a'_n(x)=
\left( \sum\limits_{n=1}^{\infty} a_n \right)'$$
то есть в ряде  можно менять производную и сумму, причем 
$\sum\limits_{n=1}^{\infty} a_n$ сходится равномерно.
\end{theor}
\textbf{Доказательство.} 
1. Используем предыдущую теорему. Тогда
$$\int\limits_{x_0}^x\left( \sum\limits_{n=1}^{\infty} a'_n(t) \right)dt=
\sum\limits_{n=1}^{\infty} \int\limits_{x_0}^xa'_n(t)dt$$
Получаем, что в равенстве
$\int\limits_{x_0}^xS(t)dt=\sum\limits_{n=1}^{\infty} (a_n(x)-a_n
(x_0))$ справа стоит число (в силу непрерывности функции), ряд из $a_n(x_0)$
сходится по условию, следовательно, ряд из $a_n(x)$ сходится.
Поэтому, дифференцируя равенство 
$\int\limits_{x_0}^{x} \sum\limits_{n=1}^{\infty} a_n(t)\,dt=
\sum\limits_{n=1}^{\infty} a_n(x)-\sum\limits_{n=1}^{\infty} a_n(x_0)$,
получаем первое утверждение теоремы.

Теперь покажем равномерную сходимость исходного ряда. 
Для этого покажем, что остаток 
ряда из производных $r_n(x)=\sum\limits_{k=n+1}^{\infty} a'_n(x)$
равномерно стремится к нулю. 
Из этого следует применимость теоремы об инетгировании: 
$\int\limits_{x_0}^{x}\sum\limits_{k=n+1}^{\infty} a'_k(t)\,dt=
\sum\limits_{k=n+1}^{\infty} \int\limits_{x_0}^{x} a'_k(t)dt=
\sum\limits_{k=n+1}^{\infty} (a_k(x)-a_k(x_0))$. Если ряд удовлетворяет 
теореме об интегрировании, то и его остатки тоже, значит,
$\int\limits_{x_0}^{x} r_n(t)dt=R_n(x)-R_n(x_0)$, откуда
$$R_n(x)=\int\limits_{x_0}^{x} r_n(t)dt+R_n(x_0)\quad(1)$$.
Зафиксируем $\varepsilon>0$. По условию, остаток обычного ряда стремится
к нулю: $R_n(x)\to0$. Тогда для 
$$\frac{\varepsilon}{2}>0~\exists n_1
(\varepsilon)~\forall n>n_1:|R_n(x_0)|<\frac{\varepsilon}{2}$$
Остаток ряда из производных равномерно стремится к нулю, тогда
для 
$$\frac{\varepsilon}{2(b-a)}>0~\exists n_2(\varepsilon)~\forall n>n_2~
\forall x\in[a,b]:|r_n(x)|<\frac{\varepsilon}{2(b-a)}$$
По формуле (1) получаем: 
$|R_n(x)|\leqslant \left| \int\limits_{x_0}^{x} r_n(t)dt \right|+
|R_n(x_0)|\leqslant \left|\left| \int\limits_{x_0}^{x} r_n(t)dt \right|+
|R_n(x_0)| \right|<\frac{\varepsilon}{2(b-a)}\cdot |x-x_0|+
\frac{\varepsilon}{2}=\varepsilon$. $\square$ 



%лекция 13.10.22
\begin{defin}
Функция $\tilde x(t)$ определенная на интервале $(a,b)$, называется 
продолжением решения вправо, если она совпадает с $x(t)$ на некотором
подинтервале.
\end{defin}

\begin{theor}
    (о продолжении решения)\\
    Пусть дано уравнение $\frac{dx}{dt}=f(t,x),~x(t_0)=x_0$, 
    функции $f(t,x),~\frac{\partial f}{\partial x}$ непрерывны на компакте
    $D\subset \mathbb{R}^2$ (причем в $D$ лежит как минимум 1 шар),
    $x(t,t_0,x_0)$ - решение задачи Коши для  $(t_0,x_0)\in Int D$. 
    Тогда существует решение, определенное на отрезке $[a,b]$, причем
     $(a,\tilde x(a,t_0,x_0)),(b,\tilde x(b,t_0,x_0))\in \partial D$.
     Иначе говоря, решение продолжается на границу компакта.
\end{theor}
\textbf{Доказательство.} В силу теоремы о существовании и единственности 
решения, функция $(x,t_0,x_0)$ определена на отрезке $[t_0-\delta_0,
t_0+\delta_0]$, где $\delta_0=\frac{r_0}{\sqrt{1+m^2} }= \frac{\rho(P_1,
\partial D)}{\sqrt{1+m^2} }$. 

Положим $t_1=t_0+\delta_0,~x_1=x(t_1,t_0,x_0),p_1(t_1,x_1)$. 
определим 
$$\tilde x(t,t_0,x_0)=\begin{cases}
    x(t,t_0,x_0),~t\in [t_0-\delta_0,t_0+\delta_0];\\
    x(t,t_1,x_1),~t\in [t_1-\delta_1,t_1+\delta_1];
\end{cases}$$
Если $(x_1,t_1)$ лежит на границе, то все хорошо. Если нет, то будем 
увеличивать шар, пока не достигнем границы множества (в силу компактности 
это всегда можно сделать). 

Возможен вариант, когда последовательность $\delta_i$ стремится к нулю и 
сама не затрагивает границу компакта. Рассмотрим функцию, определенную 
на $t\in [t_0-\delta_0,t+k+\delta_k]$. Последовательность 
$t_k$невозрастающая и ограниченная, поэтому существует и предел 
$b$. Функция $\tilde x$  определена на объединении интервало
$\bigcup\limits_{k}[t_0-\delta_0,t_k+\delta_k]=[t_0-\delta_0,b)$.
Воспользуемся непрерывностью функций: пусть $0<h\ll1$.
Тогда $\forall \alpha,\beta\in (b-h,b):|\tilde| x(\alpha,t_0,x_0)-
\tilde x(\beta,t_0,x_0)|\leqslant m|\alpha-\beta|<mh$. 
Последовательность $\tilde x_k$ фундаментальна, значит по критерию Коши 
у неё есть конечный предел. Положим этот предел значением функции 
в точке $b$:  $x^*=\tilde x(b)$. Тогда функция непрерывна на $[t_0-\delta_0,
b]$. Вспомним про интегральное уравнение: заметим, что $\tilde(x)$ 
удовлетворяет интегральному уравнению на интервале. Функция, дополненная
на конце интервала, непрерывна и также удовлетворяет интегральному 
уравнению, поэтому в ней есть и производная (по эквивалентности
определений).

Покажем, что точка $b$ лежит на границе области  $D$. Предположим противное,
тогда она лежит во внутренности $D$. Тогда она лежит в нем вместе
с некоторой $2\varepsilon$-окрестностью с центром в $p^*=(t^*,x^*)$.
Так как точки $p\to p^*$, то  все $p_i,i>k$ лежат в  $\varepsilon$-шаре
точки $p^*$. Тогда расстояние до границы больше  $\varepsilon$, 
и мы получаем противоречие с тем, что ряд из $\delta_k$ сходится и также
удален от границы больше чем на $\varepsilon$. Значит, $p^*\in \partial D$.
$\square$ \\
\textbf{Следствие.} Пусть $D\subset \mathbb{R}^2$ - такое неограниченное 
замкнтуое 
подмножество плоскости, что для любых $(a,b):D_{a,b}=D\cap \{
(t,x):a\leqslant t\leqslant b\}$ компактно, функции 
$f(t,x),\frac{\partial f}{\partial x}$ непрерывны на $D$. Тогда 
решение задачи Коши продолжается либо неограниченно, либо до выхода на границу
 $D$. Доказать самостоятельно. 

\textbf{Пример.} $x'=t^3-x^3$. Показать, что любое решение этого
уравнения продолжается неограниченно вправо.  Нарисуем изоклину $x=t$. 
Заметим, что если $t_0>x_0$, то $x(t,t_0,x_0)\in D:$. Тогда в силу следствия
решение продолжается на границу, на граница не достигается, то есть

\textbf{Пример.} $x'=1+x^2$. 
Его решение - $x=\tg(x+C),~C=\arctg(x_0)-t_0$, 
поэтому его нельзя продолжить до бесконечности,
так как каждое решение определено на конечном интервале $(C_0-\frac{\pi}{2},
C_0+\frac{\pi}{2})$. 

%\textbf{Практика.}
\subsection{Практика}

\textbf{Пример (№199).} $y^2dx-(xy-x^3)dy=0$. Раскроем скобки и перегруппируем
слагаемые: $y(ydx-xdy)-x^3dy=0$. Поделим все на  $x^2$, тогда получим:  
$-d\left( \frac{x}{y}-\frac{x}{y}dy=0 \right)$. Домножая на $- \frac{y}{x}$,
получаем $d\left( \frac{1}{2}\left( \frac{y}{x} \right)^2  \right) +dy=0$.
Итак, $d\left( \frac{1}{2}\left( \frac{y}{x} \right)^2 +y\right)=0$
В общем, мы нашли интегрирующий множитель методом внимательного взгляда. 
Ответ: $\frac{1}{2}\left( \frac{y}{x} \right)^2+y=const$.

\textbf{Пример (№202).} $d(\ln|\sin(xy)|)+\ln|y|=0$.

\begin{defin}
Интегрирующий множитель - такая функция $\mu(z(x,y))$, что при домножении на 
неё уравнение становится уравнением в полных дифференциалах.
\end{defin}
Тогда $\frac{\partial(\mu M)}{\partial y}=\frac{\partial(\mu N)}{\partial x}$.
То есть $\frac{\partial \mu}{\partial z}\frac{\partial z}{\partial y}M=
$
Получаем, что $\frac{d\mu}{\mu}= \frac{N_x-M_y}{z_yM-z_xN}=P(z)$. То есть, если
интегрирующий множитель существует, то он удовлетворяет этому условию. 
Значит, $\mu=e^{\int\limits_{}^{} P(z)dz}$.

\textbf{Пример (№212).} $(2x^2y^3-1)ydx+(4x^2y^3-1)xdx=0$.
Пусть $z=xy$. Найдем интегрирующий множитель: $\mu=\frac{1}{(xy)^2}$.



%лекция 17.10.22
\subsection{Ряды Тейлора}
\begin{defin}
Пусть в некоторой окрестности $U(x_0)$ существуют производные всех порядков
у функции. Тогда для функции $y=f(x)$ в точке $x_0$ существует ряд Тейлора:
$$f(x_0)=\sum\limits_{n=1}^{\infty} \frac{f^{(n)}(x_0)}{n!}(x-x_0)^n$$
\end{defin}
Если $x_0=0$, то ряд называется рядом Маклорена.
\begin{theor}
Если функция представляется в виде степенного ряда, то он совпадает с 
её рядом Тейлора. $f(x)=\sum\limits_{n=1}^{\infty} c_n(x-x_0)^n$.
\end{theor}
\textbf{Доказательство.}  Пусть $(x_0-R,x_0+R)$ - интервал сходимости ряда. 
Из разложения функции в ряд имеем  $f(x_0)=c_0$. Беря производную, получаем,
что  $f'(x_0)=c_1$. Дифференцируя дальше, получаем, что  $c_n=
\frac{f^{(n)}(x_0)}{n!}$. $\square$ \\
Если по произвольной функции составить ряд Тейлора, то совсем не обязательно,
что он сойдется к этой функции. Сейчас поясним:\\
\textbf{Пример.} Рассмотрим
$$f(x)=\begin{cases}
    e^{-\frac{1}{x^2}},~x\ne0\\0,~x=0    
\end{cases}$$
Очевидно (по индукции), что производная порядка $f^{(n)}(x)=e^{-\frac{1}{x^2}}
\cdot p\left( \frac{1}{x} \right)$, где $p(t)$ - многочлен. Посчитаем
производнуюв нуле; первая производная в нуле - ноль. По индукции получаем,
что все остальные производные тоже равны нулю. Значит, ряд Маклорена 
тождественно равен нулю, и сходится \textit{не к исходной функции, а
к тождественно нулевой}.
 \begin{theor}
     (достаточное условие сходимости ряда Тейлора)\\
Пусть $\exists h>0,~\exists M=const$ такие, что $\forall x\in \mathbb{N}~
\forall x\in(x_0-h,x_0+h): |f^{(n)}(x)|\leqslant M$. Тогда на всей
$h$-окрестности точки $x_0$ функция равна своему ряду Тейлора, 
причем он сходится равномерно на данном интервале. 
\end{theor}
\textbf{Доказательство.}  Разложим функцию $f(x)$ в ряд Тейлора и запишем  
остаток в форме Лагранжа: $r_n(x)=\frac{f^{(n+1)}(\xi)}{(n+1)!}(x-x_0)^{n+1}$,
$\xi\in(x_0,x)$ (лежит между ними).  Остаток по модуля меньше, чем
$M\cdot \frac{h^{n+1}}{(n+1)!}$ - значит, он равномерно сходится к нулю. 
Поэтому и сам ряд сходится равномерно на $(x_0-h,x_0+h)$. $\square$ 
\subsubsection{Ряды Маклорена для основных функций}
\begin{enumerate}
    \item $e^x=1+x+\frac{x^2}{2!}+...+\frac{x^n}{n!}...,~x\in\mathbb{R}$
\item $\sh(x)=x+\frac{x^3}{3!}+...+\frac{x^{2n+1}}{(2n+1)!}+...,~
x\in\mathbb{R}$
\item $\ch(x)=1+\frac{x^2}{2!}+...+\frac{x^{2n}}{(2n)!}+...,~x\in\mathbb{R}$
\item $\sin(x)=x-\frac{x^3}{3!}-...+(-1)^n\frac{x^{2n+1}}{(2n+1)!}+...,
    ~x\in \mathbb{R}$
\item $\cos(x)=1-\frac{x^2}{2!}+...+(-1)^n \frac{x^{2n}}{(2n)!}+...,~
    x\in\mathbb{R}$ 
\item $\ln(1+x)=1-\frac{x^2}{2}+\frac{x^3}{3}-...+(-1)^n
    \frac{x^n}{n}+...,~x\in(-1,1]$
\item $\ln(1-x)=,~x\in[-1,1)$
\item $\ln \frac{1+x}{1-x}=2 \sum\limits_{n=0}^{\infty} \frac{x^{2n+1}}
    {2n+1},~ x\in(-1,1)$ - в этой формуле функция примнимает все положительные
значения, поэтому она круче. 
\item $(1+x)^\alpha=1+\alpha x+...+\frac{\alpha(\alpha-1)...(\alpha-n+1)}
    {n!}x^n+...$
\item $arctg(x)=x-\frac{x^3}{3}+\frac{x^5}{5}+...+(-1)^n \frac{x^{2n+1}}
    {2n+1}+...~,x\in[-1,1]$
\item $arcsin(x)=x+\sum\limits_{n=1}^{\infty} \frac{(2n-1)!!\cdot x^{2n+1}}
    {n!*82^n(2n+1)},~x\in(-1,1)$


\end{enumerate}
(Для логарифма) покажем, что остаток ряда стремится к нулю.\\
1. $x\in[0,1]:~r_n(x)=\frac{f^{(n+1)}(\xi)}{(n+1)!}x^{n+1}$. 
Подставим $\xi=x_0+\theta(x-x_0),~\theta=\theta(x,n)$.
При этом имем оценку $0\leqslant x\leqslant 1\leqslant 1+\theta x$.
Получим
$|r_n(x)|=\frac{1}{n+1}\cdot \left( \frac{x}{1+\theta x} \right)^{n+1}
\leqslant \frac{1}{n+1}$. Значит, остаток равномерно сходится к 0 на $[0,1]$.
Чтобы доказать равномерную сходимость на  $(-1,0)$, запишем остаток в 
форме Коши. Получим  $|r_n(x)|=\left( \frac{1-\theta}{1+\theta x} \right)^n
\cdot \frac{|x|^{n+1}}{1+\theta x}$. Первая дробь меньше 1, вторую 
оценим как $\frac{|x|^{n+1}}{1-{|x|}}$, что при фиксированном $x$ стремится 
к нулю.
Значит, мы можем писать разложение для логарифма!\\
\textbf{Пример.} $\sum\limits_{n=0}^{\infty} \frac{2^n}{n!}=e^2$\\
\textbf{Ряд $(1+x)^\alpha$}. Найдем радиус сходимости:
$R=\lim\limits_{n \to \infty} |\frac{a_n}{a_{n+1}}|=|\frac{n+1}{\alpha-n}|=1$.
Запишем остаток в форме Коши: $(1+x)^\alpha=1+\alpha x+...+\frac{\alpha
(\alpha-1)...(\alpha-n+1)}{n!}x^n+r_n,~r_n(x)=\frac{f^{(n+1)}(\theta x)}{n!}
(1-\theta)^nx^{n+1}$. Если остаток стремится к нулю, то и ряд сходится к 
данной функции. Пусть $r_n=A_n\cdot B_n\cdot C_n$, где $B_n(x)=(1+\theta x)^{
\alpha-1},~C_n(x)=\left( \frac{1-\theta}{1+\theta x} \right)^n,~A_n=
\frac{\alpha(\alpha-1)...(\alpha-n)}{n!}x^{n+1}$. $A_n\to0$ по признаку 
Даламбера, $|B_n(x)|\leqslant max \{(1-|x|)^{\alpha-1},(1+|x|)^{\alpha-1}\}$,
$C_n(x)<1$, значит, остаток стремится к нулю, и ряд сходится к функции.\\
\textbf{Задача.} Доказать, что в $x=1$ ряд сходится при $\alpha>-1$,
расходится при $\alpha\leqslant -1$. В точке $x=-1$ сходится абсолютно при
 $\alpha\geqslant0$, расходится при $\alpha<0$\\
Выражения для арксинуса и арктангенса получаются интегрированием разложния
их производных. \\
Рассмотрим сходимость арксинуса на концах!!!!!!!!!!!!!!!!!!!!!!






%20.10.22
\subsection{Использование степенных рядов}
Разложение функции в ряд - мощнейшая тема. Иногда вфизике и других прикладных
областях делают так:\\
\textbf{Пример.} Возьмем интеграл
$\int\limits$ 


\textbf{Пример}. Решм диффур $y''=2xy'+4y$

%31.10.22
\chapter{Несобственный интеграл}
\section{Основные определения}
\begin{defin}
Пусть функция f интегрируема на писе длины $[a,b]$ для всех  $b>a$. 
Тогда \textbf{несобственный интеграл первого рода} (c одной особой точкой)
- предел 
$$\int\limits_{a}^{\infty}f(x)dx:=\lim\limits_{b\to\infty}\int\limits_{a}^{b}
f(x)dx$$
\end{defin}
Если таковой предел существует, то интеграл сходится; если предел равен 
бесконечности или не существует, то интеграл расходится. Анал
определяется и интеграл с нижним пределом $-\infty$.
\begin{defin}
    Пусть $\forall \varepsilon>0$ функция $f$ интегрируема на $[a+\varepsilon,
    b]$, и $\lim\limits_{x \to a+0} f(x)=\infty$. Тогда \textbf{несобственный
    интеграл второго рода} (с особой точкой $a$) - предел 
    $$\int\limits_{a}^{b} f(x)dx:=\lim\limits_{\varepsilon \to 0} 
    \int\limits_{a+\varepsilon}^{b} f(x)dx$$
\end{defin}

\textbf{Пример.} $\int\limits_{0}^{1} \ln xdx=\lim\limits_{\varepsilon \to 
+0} \left( \int\limits_{\varepsilon}^{1}\ln xdx\right)=
\lim\limits_{\varepsilon \to +0}\left( x\ln x\big|_\varepsilon^1-
\int\limits_{\varepsilon}^1dx  \right)=\lim\limits_{\varepsilon \to +0}
\frac{-\varepsilon^2}{1 /\varepsilon}-1=-1$ - интеграл сходится.

Если на некотором промежутке интеграл имеет конечное число особых точек, то 
всегда можно разбить промежуток на такие области, в которых каждый интеграл 
имеет лишь одну особую точку. Говорят, что интеграл \textbf{сходится, 
если он сходится на в каждой особой точке!} Так,
$\int\limits_{0}^{\infty} \frac{dx}{x^p}$ расходится при любом $p$, так 
как он расходится хотя бы на одном из промежутков  $(0,1)$ или
$[1,\infty)$ /


%03.11.22
\section{Критерии сходимости несобственного интеграла}
\begin{theor} (критерий Коши)\\
    Пусть $\forall b\geqslant a$ функция интегрируема на $[a,b]$. 
    Тогда $\int_a^\infty f(x)dx$ сходится $\Leftrightarrow$ $\forall 
    \varepsilon>0~\exists b_0(\varepsilon)>a~\forall b_1,b_2>b_0:
    \left| \int_{b_1}^{b_2}f(x)dx \right|<\varepsilon$
\end{theor}
\textbf{Доказательство.} По условию, существует предел 
$\lim\limits_{b \to +\infty} F(b)=A\in \mathbb{R}$, где
$F(b)=\int^b_af(x)dx$. 
Зафиксируем $\varepsilon>0$. Тогда из существования предела следует
для $\frac{\varepsilon}{2}$: $\exists b_o(\varepsilon)>a:\left| 
F(b)-A\right|<\frac{\varepsilon}{2}$. Пусть $b_1>b_0,~b_2>b_0$. Тогда
$|F(b_2)-F(b_1)|=|F(b_2)-A|+|F(b_1)-A|<\frac{\varepsilon}{2}+
\frac{\varepsilon}{2}=\varepsilon$.\\
Достаточность. Докажем существование предела  $\lim\limits_{b\to\infty}F(b)$
из определения предела по Гейне. Пусть $b_n\to \infty$, тогда $\forall b_0>a
~\exists n_0(\varepsilon)\in \mathbb{N}~\forall n>n_0:b_n>b_0$. 
Зафиксируем $n,m>n_0$. Тогда $b_n>b_0$ и $b_m>b_0$. По условию отсюда 
следует, что $|F(b_n)-F(b_m)|<\varepsilon$. По критерию Коши для числовой 
последовательности $F(b_n)$ существует предел 
$\lim\limits_{n \to \infty} F(b_n)=B\in \mathbb{R}$.\\
Покажем, что предел не зависит от выбора последовательности $b_n$. 
Выберем другую последовательность  $b^*_n$. Обозначим предел 
$\lim\limits_{n \to \infty} F(b^*_n)=B$. Составим последовательность
$b_1,b^*_1,b_2,b^*_2,...\to \infty$. Тогда предел $F$ от этой
последовательности обозначим как  $C$. Так как пределы подпоследовательностей 
сходятся к пределу последовательности, то  $A=B=C$. Значит, выполняется
условие определения предела по Гейне, значит, интеграл сходится. $\square$

\textbf{Пример.} $\int_1^\infty \frac{\sin x}{x^\alpha}dx$ сходится при 
$\alpha>0$, расходится при $\alpha\leqslant 0$. Докажем это.\\
1. $\alpha>0$. Поехали: $\forall \varepsilon>o~\exists b_0(\varepsilon)>1~
\forall b_1>b_0,b_2>b_0: \left| \int^{b_2}_{b_1} \frac{\sin x}{x^\alpha}dx
\right|<\varepsilon$. Доказываем: 
$\left| \int^{b_2}_{b_1} \frac{\sin x}{x^\alpha}dx
\right|=\left| \int^{b_2}_{b_1} \frac{1}{x^\alpha}d\cos x\right|=
\left| \frac{\cos x}{x^\alpha} \right|^{b_2}_{b_1}-\int^{b_2}_{b_1} 
\cos x d(\frac{1}{x^\alpha})\leqslant ... \leqslant\frac{4}{b^\alpha_0}$.
Значит, $b_0>(\frac{4}{\varepsilon})^\frac{1}{\alpha}$.\\
2. $\alpha\leqslant 0$. Синус теперь принимает разные знаки. Пусть 
$b_k=2\pi k$. Тогда по критерию Коши интеграл расходится.

\begin{theor} (критерий сходимости через остаток)\\ \label{skhod_ost}
Пусть $\int^\infty_af(x)dx=\int^b_af(x)dx+\int^\infty_bf(x)dx,~(b>a)$. 
Тогда:\\
1. Если несобственный интеграл сходится, то и любой из его остатков сходится.\\
2. Если хотя бы один из остатков сходится, то несобственный интеграл сходится.
\end{theor}
\textbf{Доказательство.} $\int^b_af(x)dx$ - число, поэтому сходимость 
равносильна сходимости остатка. $\square$ 

\begin{theor} (критерий сходимости несобственного интеграла от неотрицательной
функции)\\
Пусть $\forall b>a$ функция интегрируема на $[a,b]$ и неотрицательна.Тогда
$\int^\infty_af(x)dx$ сходится $\Leftrightarrow$ первообразная $F(b)<M$ 
ограниченна.
\end{theor}
\textbf{Доказательство.} Пусть $a<b_1<b_2$. Имеем
$$F(b_2)=\int\limits_{a}^{b_2}f(x)dx=\int\limits_{a}^{b_1}f(x)dx+
\int\limits_{b_1}^{b_2}f(x)dx=F(b_1)+\int\limits_{b_1}^{b_2}f(x)dx$$
откуда $F(b_2)-F(b_1)=\int\limits_{b_1}^{b_2}f(x)dx\geqslant 0$. Значит,
$F(b)$ неубывает и ограниченна сверху. Значит, существует предел
$\lim\limits_{b \to \infty}F(b)$, и интеграл сходится.\\
Обратно, пусть существует конечный предел $\lim\limits_{b \to \infty} F(b)$,
то $F(b)$ ограниченна в некоторой окрестности бесконечности $U$.
Тогда существует такое $b_0>a~\forall b\geqslant b_0:F(b)\leqslant M$.
Значит, $B(b^*)\leqslant M$ - ограниченна. Если $b^*<b_0$, то
$F(b^*)<F(b_0)$. Итак, $F$ ограниченна. $\square$

\section{Признаки сравнения}
В данном разделе признаки работают для неотрицательных функций.
\begin{theor} (первый признак сравнения/в оценочной форме)\\
Пусть $f(x)>g(x)>0$ начиная с некоторого $x>a$, и для любого  $b>a$
функции интегрируемы на $[a,b]$. Тогда\\
1. Если  $\int^\infty_a f(x)$ сходится, то и  $\int^\infty_a g(x)$ сходится.\\
2. Если  $\int^\infty_a g(x)$ расходится, то и $\int^\infty_a f(x)$ расходится.
\end{theor}
\textbf{Доказательство.} 1. Обозначим $F(b)=\int\limits_{a}^{b}f(x)dx$ и
$G(x)=\int\limits_{a}^{b}g(x)dx$. По условию, интеграл сходится, поэтому
$F(b)\leqslant M$. По свойству определенного интеграла, $F(b)\geqslant G(x)$,
поэтому $G(b)$ также ограниченна и сходится по предыдущему критерию.\\
2. Допустим, что $\int\limits_{a}^{\infty}g(x)dx$ сходится. Тогда из 
первого пункта следует, что и $\int\limits_{a}^{\infty}f(x)dx$ сходится, 
что противоречит условию. $\square$ 

\begin{theor} (второй признак сравнения/в предельной форме)\\
Если $\lim\limits_{x \to \infty}\frac{f(x)}{g(x)}=k,~\infty\ne k\ne0$, 
то их несобственные интегралы сходятся или расходятся одновременно. 
\end{theor}
\textbf{Доказательство.} Пусть $k>0$. Тогда для
 $$\varepsilon>\frac{k}{2}~\exists b_0>a~\forall x>b_0:\frac{k}{2}<
 \frac{f(x)}{g(x)}<\frac{3k}{2}$$ 
Значит, $f(x)< \frac{3k}{2}g(x)$, и из сходимости интеграла от $g(x)$ следует
сходимость интеграла от $f(x)$. C другой стороны, $\frac{k}{2}g(x)<f(x)$, 
поэтому из сходимости интеграла от $f(x)$ следует сходимость интеграла от 
$g(x)$. $\square$ 

\textbf{Следствие.} Если $f(x)\sim g(x)$ при $x\to \infty$, то интегралы 
сходятся или расходятся одновременно. 


















\end{document}
