\documentclass[a4paper]{article}

%Общие настройки документа
\usepackage[14pt]{extsizes}                                         %Размер шрифта
\usepackage[left=2.5cm,right=2.5cm,top=2.5cm,bottom=3cm]{geometry}  %Поля страницы

%Настройки ссылок и гиперссылок
%\usepackage{hyperref}                 
%\usepackage{xcolor}
%\definecolor{linkcolor}{HTML}{799B03} % цвет ссылок
%\definecolor{urlcolor}{HTML}{799B03}  % цвет гиперссылок
%\hypersetup{pdfstartview=FitH,linkcolor=linkcolor,urlcolor=urlcolor,colorlinks=true}

%Пакеты символов
\usepackage{cmap}
\usepackage[T2A]{fontenc}
\usepackage[utf8]{inputenc}
\usepackage[russian]{babel}           
\usepackage{amsmath}
\usepackage{amssymb}
\usepackage{amsfonts}

%Новые команды 
\newtheorem{defin}{Определение}
\newtheorem{example}{Пример}
\newtheorem{zam}{Замечание}
\newtheorem{theor}{Теорема}

\author{Галкина}
\title{Анализ}
\date{05.09.2022}

\begin{document}
\maketitle
\tableofcontents
\newpage
%5.09.22
Коэффициенты: Контр*0,4 Коллок*0,3 Экз*0,3
\chapter{Ряды}
В данном разделе мы будем изучать следующие объекты:
\begin{itemize}
    \item Числовые ряды
    \item Функциональные ряды (в т.ч. степенные, ряды Фурье)
\end{itemize}

\section{Числовые ряды}
\subsection{Базовые определения и теоремы}
\begin{defin}
Ряд - сумма счетного числа слагаемых: $$\sum_{n=1}^{\infty} a_n =a_1+a_2+
\ldots$$
\end{defin}
\begin{defin}
Частичная сумма $S_n$ - сумма первых n слагаемых
\end{defin}
\begin{defin}
Сумма ряда - предел последовательности частичных сумм 
$$S=\lim\limits_{n \to \infty}S_n$$
\end{defin}
Если предел существует и конечен, то ряд сходится. Если предел бесконечен, ряд 
расходитс. Заметим, что, согласно теоремам о пределе суммы последовательностей
и пределе последовательности, умноженной на число, сходящиеся ряды образуют
линейное пространство относительно сложения и умножения на константу.
\begin{defin}
Остаток ряда - разность между частичной суммой ряда и самим рядом: 
$$R_k=S-S_k=\sum_{n=k}^{\infty} a_k$$
\end{defin}
\textbf{Пример.} Геометрический ряд $a+aq+aq^2+\ldots$. По школьной 
формуле $S_n=\frac{1-q^n}{1-q}$. Имеем случаи:
\begin{enumerate}
    \item $|q|<1:~S=\frac{a}{1-q}$ 
    \item $|q|>1:~S=\infty$
    \item $q=1:~S=\infty$
\end{enumerate}
Итак, ряд сходится, только если $|q|<1$.\\
Следующие теоремы устанавливаются для любых рядов:
\begin{theor}(необходимое условие сходимости ряда)\\
Если ряд сходится, то предел общего члена равен 0.
Равносильная формулировка: если
$\lim\limits_{n\to\infty} a_n\ne0$, то ряд $\sum\limits_{n=1}^{\infty} a_n $
расходится.
\end{theor}
\textbf{Доказательство.} По условию, существует число $S$ - предел частичных 
сумм ряда.
Тогда $\lim\limits_{n \to \infty}a_n=\lim\limits_{n \to \infty}(S_{n}-S_{n-1})
=S-S=0$. 
$\square$

\textbf{Пример.} $\sum\limits_{n=1}^{\infty} \sin{nx},~x\ne\pi k,~k 
\in\mathbb{Z}$. 
Зафиксируем $x$. Допустим, что $\lim\limits_{n \to \infty}\sin nx=0 $. 
Но это противоречит тому, что $\sin^2(x)+\cos^2(x)=1$. Значит, ряд расходится.

\textbf{Пример.} Гармонический ряд расходится, т.к. расходится 
последовательность частичных сумм: 
$S_{2^n}>1+\frac{1}{2}+2\cdot \frac{1}{4}\ldots=1+\frac{n}{2}$
\begin{theor} (критерий Коши сходимости ряда)\\
Ряд $\sum\limits_{n=1}^{\infty} a_n$ сходится тогда и только тогда, когда
$$\forall\varepsilon>0~\exists N(\varepsilon)~\forall n>N~\forall p\in
\mathbb{N}:|a_{n+1}+\ldots+a_{n+p}|<\varepsilon$$
\end{theor}
\textbf{Доказательство.} По определению, ряд сходится, когда существует предел
частичных сумм. Применим к ним критерий Коши, получим условие: 
$|S_{n+p}-S_n|<\varepsilon$. Но $S_{n+p}-S_n\equiv a_{n+1}+...+a_{n+p}$.
$\square$ 
\begin{theor} (критерий сходимости через остаток)\\
    1. Если ряд сходится, то сходится любой из его остатков.\\
    2. Если хотя бы один остаток сходится, то ряд тоже сходится.
\end{theor}
\textbf{Доказательство.}
1. По условию, существует сумма ряда $S$. Зафиксируем номер $N\in\mathbb{N}$ и 
рассмотрим остаток $R_N=\sum\limits_{k=N+1}^{\infty} a_k$, а также 
последовательность $\sigma$ частичных сумм ряда-остатка $R_N$: 
$\sigma_n=a_{N+1}+...+a_{N+n}=\sum\limits_{k=N+1}^{N+n} a_k$.
Рассмотрим её предел:
$\lim\limits_{n \to \infty} \sigma_n=\lim\limits_{n \to \infty}(S_{n+N}-S_N)
=S-S_N=R_N$. Значит, остаток сходится.\\
2. По условию, существует такой номер $n_0$, что остаток $R_{n_0}$ сходится.
Тогда существует предел частичных сумм $\sigma_n$ этого остатка:
$\lim\limits_{n \to \infty}\sigma_n=\sigma$, 
$\sigma_n=a_{n_0}+\ldots+a_{n_0+n}$. Пусть $n_0+n=m$, тогда
$\lim\limits_{n \to \infty}S_m=\lim\limits_{n \to \infty}
(S_{n_0}+\sigma_{m-n_0})=S_{n_0}+\sigma$, то есть основной ряд сходится.
$\square$ 





%\textbf{Пример.} $\sum_{n=1}^{\infty} (\sqrt{n+2} -2\sqrt{n+1} +\sqrt{n} )$.
%Введем $a_n=b_{n+1}-b_n,~b_n=\sqrt{n+1}-\sqrt{n}$ 
%Итак, $S=\lim_{n \to \infty}$.\\
%\textbf{Пример.} $\sum_{n=1}^{\infty} \frac{n}{2^n}=2$
%\subsection{Знакопостоянные ряды}
%Докажем несколько теорем о свойствах знакопостоянных рядов.







%Лекция+семинар 08.09.22
% Продолжаем топологию 15/09/2022
\section{База топологии}
\begin{defin}
    Пусть $(X,\tau)$ - топологическое пространство
    Семейство  $\Sigma=\{W_\beta\subset X\mid \beta\in B \} $ - база топологии,
    если удовлетворяет двум условиям:\\
    1. $\Sigma\in\tau~\forall W_\beta\in\Sigma$
    2. Любое открытое подмножество Х можно представить в виде
    объединения некоторых подмножеств из $\Sigma$: 
    $\forall U\in\tau\exists W_\alpha\in\Sigma,~\alpha\in A\subset B:
    U=\Cup\limits_{\alpha\in A}W_\alpha $
\end{defin}
\textbf{Пример}. В обычной (евклидовой) топологии множество 
$\Sigma=\{D_r(a)\mid a\in\mathbb{R}^n,r>0\}$ является базой топологии.
Действительно, проверим аксиомы:\\
1. Открытая окрестность открыта.\\
2. По определению обычной топологии,каждая точка в открытом множестве
содержится в нем с некоторой окрестностью. Значит, объединение этих
окрестностей дает это множество. Более формально,
$\forall u\in\tau,\forall x\in U\Rightarrow \exists D_{\varepsilon_x}(x):
D_{\varepsilon_x}(x)\in U$. Очевидно доказывается. что 
$$\boxed{\bigcup_{x\in U}D_{\varepsilon_x}(x)=U}$$ 
\textbf{Замечание.} Если к базе добавить произвольное открытое множество, то
новое множество также будет базой.\\
\textbf{Упражнение.} Привести пример двух баз евклидовой топологии на 
плоскости, которые не пересекаются с обычной базой (открытых шаров). 
(Решение: например, база из открытых квадратных или звездчатых окрестностей).\\
\textbf{Пример.} В $(\mathbb{R}^2,\tau_{MN})$, 
$\Sigma_{MN}=\{(b,b^*)\mid b\in\} $ !!!!!!!!!!!!!!!!!!!!!\\
\textbf{Пример.} Топология ираациональных точек на прямой
$(\mathbb{R},\tau_{im}),~\tau_{im}=\{\varnothing,\mathbb{R}\}\cup
\{U\subset \mathbb{R}\setminus\mathbb{Q}\} $.
Множество иррациоанльных точек не является базой, поскольку их объединение не
содержит всю прямую. Решение: добавить саму прямую. !!!!!!!!!!!!\\

\begin{theor}
    (критерий базы в топологическом пространстве)\\
    Пусть $(X,\tau)$ - опологическое пространство, и семейство множеств 
    удовлетворяет условию $\sigma\subset \tau$. $\Sigma$ является базой 
    топологии тогда и только тогда, когда 
    $\forall u\in\tau,\forall x\in U\exists W_{\beta_0}\in\Sigma:
    x\in W_{\beta_0}\subset U$
\end{theor}
\textbf{Доказательство.} Пусть $\Sigma$ - база топологии. Тогда любое открытое 
множество можно представить в виде объединений множеств из базы. Значит, для
$x\in U$ найдется множество из базы, в котором лежит $x$.  \\
Обратно. Множесто $\Sigma$ удовлетворяет первой аксиоме базы по определению.
Докажем выполнение второй аксиомы. Для любой точки в открытом множестве
по условию теоремы найдется окрестность из $\Sigma$, лежащая в открытом
множестве. 
!!!!!!!!!!!!!!!!!!!!!!
$\square$ 

\begin{theor}
    (критерий базы на множестве)\\
    Пусть Х - произвольное множесто, $\Sigma=\{W_\beta\subset X\mid\beta
    \in B\}$ - семейство подмножеств из Х. ЧТобы на Х существовала
    топология с данной базой, необходимо и достаточно выполнения
    двух условий:\\
    1. $X=\bigcup\limits_{\beta\in B} W_\beta$\\
    2. Для любых множеств из базы найдется множество, лежащее в их
    пересечении и содержащее произвольную точку оттуда.

\end{theor}
\textbf{Доказательство.} Необходимость. Пусть $\Sigma$ - база некотрой 
топологии (Х,т). Из акиомы базы (2) следует,что что Х есть объединение
множеств из $\Sigma$. значит, выполняется первое условие теоремы. Докажем второе 
условие. Достаточно взять пересечение двух множеств из базы. Так как 
это открытые множества, его также можно представить в виде объединения
множеств из базы, и хотя бы в одном из которых лежит фиксированная точка
(по определению объединения).\\
Достаточность. Докажем, что всевозможные объедения множеств из $\Sigma$  
является топологией. пусть это есть $\tau$. Проверим аксиомы топологии:\\
1.  Пустое множество принадлежит всему, чему надо. Все простарнство 
лежит там по условию теоремы.
3. Пусть 


$\square$ 



%Лекция 19.09
\section{Элементарные методы интегрирования ДУ}
\subsection{Уравнения с разделяющимися переменными}
\begin{defin}\label{ODE_razdp}
Уравнение с разделяющими переменными - уравнение вида
\begin{equation}
    \frac{dx}{dt}=f(x)g(t) 
\end{equation}
где $f,g$ непрерывны на  $x\in(a,b),~t\in(\alpha,\beta)$
\end{defin}
Как решать такие уравнения? Алгебраическая интуиция подсказывает, что надо 
перенести 
дифференциалы к своим функциям и проинтегрировать. Но это ещё надо обосновать.
Сделаем следующее:\\
\begin{enumerate}
    \item Найти все $x_*:f(x_*)=0$. Тогда $x=x_*$ - решение-константа. 
    \item Пусть  $x^i_*,x^j_*$ - такие, что  $f(x^i_*)=f(x^j_*)=0$ и
    $\forall x\in(x^i_*,x^j_*):f(x)\ne0$. Тогда уравнение \ref{ODE_razdp}
эквивалентно уравнению 
$$\frac{dx}{f(x)}=g(t)dt$$
Эту штуку можно проинтегрировать с обеих сторон. Результат непрерывен и не
обращается в ноль. Значит, по теореме о неявной функции найдется решение. 
$\frac{dF}{dx}=\frac{1}{x}$(решение в области $(\alpha,\beta)\times
(x^i_*,x^j_*)$).
    \item Выписать решение на каждом интервале $(x^i_*,x^j_*)$
\end{enumerate}
Других решений не существует. Почему? Допустим, существует другое решение.
Оно не может быть константой, так как все константы были получены в п.1.
Если она \\
\textbf{Пример.} Решим уравнение $\frac{dx}{dt}=0$. Решение-константа: $x=0$.
Теперь рассмотрим два интервала: $x<0$ и  $x>0$. Если  $x<0$, имеем уравнение
 $$\frac{1}{x}\frac{dxdt}{dt}=dt$$
 Интегрируем:
 $$\int\frac{dx}{x}=\int dt$$
 Получаем, что $\ln|x|=t+C$. Выражаем искомую функцию (не забыв, на каком
 промежутке мы рассматриваем функцию, и раскрыв модуль соответственно):
 $$x=-Ce^t,~C>0$$
Для интервала $x>0$ точно такой же порядок действий, только получим другой 
знак. Итак, множество решений:
$$x=Ce^t,~C\in\mathbb{R}$$
\subsection{Уравнения, приводящиеся к уравнению с разделяющимися переменными}
\begin{defin}
Уравнение, приводящееся к уранвению с разделяющмися переменными - уравнение
вида 
\begin{equation}
    \frac{dx}{dt}=f(at+bx+c) \label{ODE_privrazd}
\end{equation}
\end{defin}
Давайте решим его. 
\begin{enumerate}
    \item Введем замену $z(t)=at+bx+c$. 
    Имеем
     $$\frac{dz}{dt}=a+b\frac{dx}{dt}$$ 
     Получаем уравнение с разделяющимися переменными. 
     $$\frac{dz}{a+bf(z)}=dt$$
\end{enumerate}
\textbf{Пример.} Решим уравнение $\frac{dx}{dt}=\cos(x+t)$. Замена 
$z=x+t,~ \frac{dz}{dt}=1$. Уравнение имеет вид
$$\frac{dz}{dt}=\frac{dx}{dt}+1$$ 
Найдем $\cos{z_*}+1=0$: это, очевидно, $\pi+2\pi k,~k\in \mathbb{Z}$ 
Свели задачу кпрошлому пункту
\subsection{Однородные уравнения}
Сначала докажем, что два определения однородного уравнения эквивалентны.
\begin{defin}\label{ODE_odn}
Однородным называется уравнение вида
\begin{equation}%\label{ODE_odn1}
    \frac{dx}{dt}=f\left(\frac{x}{t}\right) 
\end{equation} 
\end{defin}
Это уравнение инвариантно относительно замены $x\mapsto kx,~t\mapsto kt$.
Геометрически это означает, что совокупность интегральных кривых инвариантно
относительно преобразования $\theta(x,y)=(kx,ky)$.
Из этого следует, что если мы найдем одно решение, то мы найдем всю 
совокупность ему подобных. %Вставить картинку.
\begin{defin}
    (вспомогательное)\\
Уравнение в форме дифференциалов:
    $M(x,y)dx+N(x,y)dy=0$.  
\end{defin}
Это таже форма, что и $\frac{dy}{dx}=f(x,y)$, поскольку 
$\frac{dy}{dx}=-\frac{M(x,y)}{N(x,y)}$. Обратно, $-f(x,y)dx+dy=0$.
Уравнение в форме дифференциалов имеет чуть большее множество решений. 
\begin{defin}\label{ODE_odn2}
Уравнение в форме дифференциалов называется однородным, если\\
$M(kx,ky)=k^nM(x,y)$\\ 
$N(kx,ky)=k^nN(x,y)$\\
n называется степенью однородности.
\end{defin}
\begin{theor}
    Определения \ref{ODE_odn} и \ref{ODE_odn2} эквивалентны. 
\end{theor}
\textbf{Доказательство.} 1 $\Rightarrow$ 2. $\frac{dy}{dx}=f(\frac{y}{x})$\\
2 $\Rightarrow$ 1. Пусть дано уравнение в форме дифференциалов. Подставим $k$.
При $x\ne 0$ имеем 
$$\frac{dx}{dy}=-\frac{k^nM(x,y)}{k^nN(x,y)}=
-\frac{M(kx,ky)}{N(kx,ky)}=-\frac{M(1,\frac{y}{x})}{N(1,\frac{y}{x})}=
f\left( \frac{y}{x} \right) $$
$\square$ \\
\textbf{Пример.} $M=x^2+y^2$\\
\textbf{Пример (№31).} Найти уравнение, решение которых - параболы с осью, 
параллельной оси ординат и касающиеся прямых $y=0,~y=x$. 
Во-первых, поймем, как выглядит уравнение такой параболы. Исходя из геометрии,
получим, что уравнение параболы, удовлетворяющее первому условию, имеет вид 
$y=ax^2+bx+\frac{b^2}{4a}$, а первому и второму - $y=ax^2+\frac{1}{2}x+
\frac{1}{16a}$. Остался один параметр $\Rightarrow$ уравнение первого порядка. 
Подставляем и хаваем ответ бесплатно:
$$y=\left(\frac{y'-\frac{1}{2}}{2x}\right)x^2+\frac{1}{2}x+\frac{2x}{16y'-8}$$ 
\textbf{Пример (№72).} Найти линии, у которых треугольники, образованные 
касательными, осью ОХ и точкой касания, имеют одинаковую сумму катетов. 
Из геометрических соображений имеем уравнение 
$$\frac{|y|}{|y'|}+|y|=b=const$$ 
Раскрываем модули. В простейшем случае имеет уравнение с разделяющимися 
переменными. 
$$\frac{dy}{dx}=\frac{y}{b-y}$$ 
Остальные уравнения такие же в принципе. Так шо это идет в дз 
Его легчайшее (и, видимо, общее) решение: $x+C=\pm b\ln{|y|}\pm y$\\
\textbf{Пример (№76).} Геометрическая интуиция не должна подводить нас. 
Вставить картинку. Есть кароч такая формула: 
$\tg\gamma=\frac{r}{r'}$





%Лекция 22.09
\section{Однородное уравнение}
$$\frac{dx}{dt}=f\left(\frac{x}{t}\right)$$ 
Как искать его решение? Заменой $u(t)=\frac{x}{t}$. 
Тогда уравнение перепишется в виде $\frac{dx}{dt}=\frac{du}{dt}t+u$.
В нем переемнные разделяются: $\frac{du}{f(u)-u}=\frac{dt}{t}$. 
Итак, типы уравнений:
\begin{enumerate}
    \item С разделяющимися переменными
    \item Приводящиеся к виду $\frac{dx}{dt}=f(ax+bx+c)$ 
    \item Првиодящиеся к виду $(a_1x+b_1t+c_1)dx+(a_2x+b_2x+c_2)dt=0$
\end{enumerate}
Подумаем, можно ли это последнее привести к однородному. Добавим условие 
$c_1^2+c_2^2\ne0$ (иначе система уже однородна). В общем, если эти 
две прямые пересекаются в точке
$(x_*,t_*)$, то можно ввести новые переменные, передвинув эту точку в начало
координат: $x\mapsto x-x_*,~t\mapsto t-t_*$. Тогда система перепишется 
без $c_1,~c_2$, и таким образом будет однородной. Если прямые не пересекаются, 
то прямые либо совпадают, либо параллельны. Тогда введем замену (для любой 
прямой) $z(t)=a_1x+b_1t+c_1$. Так как прямые параллельны, то
$\frac{a_1}{a_2}=\frac{b_1}{b_2}=k$, значит, мы можем выразить 
вторую прямую: $a_2x+b_2t+c_2=\frac{1}{k}(a_1x+b_1t+kc_2)=\frac{1}{k}(z-
c_1+kc_2)$. Уравнение приводится к виду $z(t)dx+\frac{1}{k}(z-c_1+kc_2)dt=0$.
Но у нас все равно многовато переменны. Выразим $dx$ через  $z$:
$$z(\frac{dz-b_1dt}{a_1})+\frac{1}{k}(z-c_1+kc_2)=0$$ 
Умножим на $a_1k$:
 $$kzdz=kb_1zdt-a_1zdt-a_1(kc_2-c_1)dt$$ 
Домножим на $\frac{1}{kzdt}$:
$$\frac{dz}{dt}=((b_1-\frac{a_1}{k})z-a_1(kc_2-c_1))\frac{1}{z}$$ 
Finally, уравнение с разделяющимися переменными! ПОБЕДА!
\subsection{Обобщенно-однородное уравнение}
\begin{defin}
Обобщенно-однородное уравнение - уравнение вида
$$M(x,t)dx+N(x,t)dt=0$$ 
причем $M,N$ - такие. что  $\exists n\in\mathbb{R}$: если
$x=z^n(t)$, то уравнение  $M(z^n,t)nz^{n-1}dz+N(z^n,t)dt=0$ однородно.
\end{defin}
\textbf{Пример.} Испортим однородное уравнения, чтобы сделать его 
обощенно-однородным. Роман придумал, чел харош.

Сведем и этого зверя к разделяющимся переменным. 
$$\begin{cases}
    n(kz)^{n-1}M((kz)^n,kt)=k^mM(z^n,t)nz^{n-1}\\
    N((kz)^n,kt)=k^mN(z^n,t)
\end{cases}$$ 

\subsection{Уравнение в полных дифференциалах}
Напомним, что полный дифференциал $dF(x,y)$  $C^1$-гладкой функции
равен  $\frac{\partial F}{\partial x} dx+\frac{\partial F}{\partial y}dy$.
\begin{defin}
Уравнение в полных дифференциалах - уравнение вида
$$dF(x,y)=0,~F\in C^2(\Omega),~\Omega\subset \mathbb{R}^2$$
\end{defin}
Если мы знаем саму функцию, то решение находится мгновенно: $dF(x,y)=const$.
Правда, оно неявное. Выразим  $y=y(x)$ по теореме о неявной функции.\\
\textbf{Пример.} $x^2\sin{t}dt+2x\cos{t}dx=0$\\
Уравнение является уравнение в полных дифференциалах, если существуют такие
функции, что $M=\frac{\partial F}{\partial x},~N=\frac{\partial F}{\partial y}$
\begin{theor}
    (необходимое условие представления в полных дифференциалах)\\
    $$\frac{\partial M}{\partial y}=\frac{\partial N}{\partial y}$$ 
    Достаточное условие - $M_y=N_x$ в односвязной области
\end{theor}
\textbf{Доказательство.} $\square$ \\
Как подбирать такие функции? Мы знаем, что $\frac{\partial F}{\partial x}=
M(x,y)$. Проинтегрируем это равенство по $x$. Имеем  $F=\int M(x,y)dx+
\varphi(y)$. Проделаем то же самое по переменной $y$:  $\frac{\partial F}
{\partial y}=\frac{\partial }{\partial y}(\int M(x,y)dx)+\varphi'=N(x,y)$,
откуда $\varphi=\int\left(N-\frac{\partial }{\partial y}(\int Mdx) \right)dy$.
Чтобы проверить себя при решении, помним, что $\varphi$ не зависит от $x$!
Итак,
$$F=\int M(x,y)dx+\int\left(N-\frac{\partial }{\partial y}\left(\int Mdx
\right)\right)dy$$
\subsubsection{Геометрический смысл решения уравнения в полных дифференциалах}
Так как $z=z(x,y)$ - какая-то поверхность, то запись  $z=const$ - это линии
уровня, которые можно спроецировать на плокость переменных и получить 
интегральные кривые.\\
\textbf{Пример (модель Лотки-Вольтерра).} Пусть $x(t)$ - плотность карасей,
 $y(t)$ - плотность щук в некотором пруду. Щуки сдерживают рост карасей,
но от количества карасей зависит также и количество щук. Запишем систему:
$$\begin{cases} \label{lotka_volterra}
    \dot{x}=x(a-by)\\
    \dot{y}=y(-c+ex)
\end{cases}$$
Лотка придумал эту систему для биоценозов, а Вольтерра - для химических
реакций.\\
Давайте решим эту систему. Её расширенное фазовое пространство, вообще говоря,
трехмерное, поэтому будем рассматривать фазовые кривые - проекции интегральных 
на плоскость независимых параметров. Они ориентированы в направлении роста
параметра $t$. Найдем эти кривые, найдя решение уравнения 
$\frac{dy}{dx}=\frac{dy /dt}{dx /dt}=\frac{-cy+exy}{ax-bxy}$. 
Переменные разделяются: $$\frac{(a-by)dy}{y}=\frac{(-c+ex)dx}{x}$$ 
Представим его в полных дифференциалах: 
$$d\left( a\ln y-by+c\ln x -ex \right)=0$$
Значит, решение имеет вид $a\ln y-by+c\ln x-ex=h=const$.
Выглядит очень сложно, но давайте попробуем построить изолинии. 
Введм функцию $F=\ln{(y^ax^c)}-by-ex$, и поищем её изолинии. Сначала найдем
критические точки: $(x_*,y_*)=(\frac{c}{e},\frac{a}{b})$ (получилась 
единственная точка). Определим тип критической точки (составим гессиан, 
посчитаем его знакоопределенность); получим, что это точка максимума.
Линии уровня - какие-то окружности/эллипсы.\\
\begin{figure}[H]
    \centering
    \input{figures/Lotka-Volterra.pdf_tex} %для pdf_tex
    %\includegraphics{}  % для png, pdf
    \caption{Первый интеграл системы Лотки-Вольтерры}
    \label{fig:}
\end{figure}
\textbf{Упражнение.} Доказать, что фазовые кривые замкнуты.

\textbf{Доказательство (\cite{Arnold},\S 2.7).} Так как 
$\frac{dF}{dt} = 0$, то функция $F$ сохряняется вдоль фазовых кривых. 
Иначе говоря, фазовые кривые являются изолиниями фунции $F$. Но 
график  $F$ является суммой двух <<холмов>>, образованных логарифмами, 
и поэтому сам является холмом. Посколько изолинии холма - замкнутые кривые,
то и фазовые кривые системы Лотки-Вольтерры замкнуты. $\square$\\
Теперь нам надо понять, куда закручиваются эти линии, как они ориентированы.
Они закручиваются против часовой стрелки вокруг критической точки, кстати,
область решения - первая координатная четверть. Чтобы избежать проблем с 
дискретностью, наши переменные - это плотность населения пруда. 

 



%Гомеоморфный (без самопересечений, так как биекция) образ окружности в
%плоскости можно непрерывно продеформировать в точку. Теорема Жордана: 
%гомеоморфный образ окружности делит плоскость на две компоненты связности. 
%Теорема Шёнфлиса: внутренняя часть этой плоскости гомеоморфна открытому 
%диску. Это кстати эквивалентно тому, что фундаментальная группа 
%тривиальна. 

%лекция 
\begin{defin}
Автономное ДУ - дифференциальное уравнение, правая часть которого не зависит
от времени.
\end{defin}
Автономные уравнения не могут быть динамическими системами, так как они 
не зависят от времени, но можно искусственно этого достичь.\\
\textbf{Пример.} Нелинейный консервативный
осциллятор. Рассмотрим маятник с координатами
$\varphi$ - отклонение от положения равновесия. Рассмотрим плоские
колебания маятника массой $m$ и длиной $l$. При повороте на малый угол 
движение можно представить как прямолинейное движение по касательной.
Запишем второй закон Ньютона в проекции на касательную: 
$$\vec\tau:~m\frac{d^2x}{dt^2}=-mg\sin\varphi$$ 
Пусть $\Delta x$ - длина дуги окружности, примерно равная малой части 
касательной. Тогда  $\Delta x=l\Delta\varphi+o(\Delta\varphi)$.
Получим уравнение
$\frac{fx}{dt}=l \frac{d\varphi}{dt}$. Finally,
$$ml \frac{d^2\varphi}{dt^2}=-mg\sin\varphi$$ 
- уравнение колебания маятника. Оно нелинейное из-за синуса. 
Оно имеет порядок 2, значит, нам надо
зафиксировать начальные условия: $\varphi(0),~\dot\varphi(0)$.
Уравнение тогда превратится в систему
$$\begin{cases}
    \dot\varphi=\psi\\
    \dot\psi=-\omega^2\sin\varphi
\end{cases}$$
Кстати, если мы напишем функцию Лагранжа и напишем уравнение Лагранжа для
него, то получим это же уравнение.\\
Начнем решение. Сделаем замену $\dot\varphi=\psi$. Теперь введем фазовое 
пространство угол-скорость таким образом, чтобы близкие точки были близки.
В угловых координатах мы склеим точки $\pi,-\pi$ у координат углов (точнее,
создадим факторпространство по отношению $(\varphi,\psi)\sim(\varphi+2\pi k,
\psi)$). Получим, что фазовое пространство - цилиндр. Любая замкнутая кривая 
- это некоторая траектория (вообще говоря, определляемая уравнением).
На цилиндре есть два типа замкнутых кривых - стягиваемые в точку и 
нестягиваемые. Вторые отвечают за движение через верх. \\
Продолжаем решение. Из системы имеем 
$\frac{d\psi}{d\varphi}=-\frac{\omega^2\sin\varphi}{\psi}$.
Полная энергия равна константе:
$\frac{m\psi^2}{2}+\frac{mg}{l}(1-\cos\varphi)=h$.
Это мы вывели из формы уранвения в полных диференциалах. 
В общем, решаем. Получим
$$\varphi=\pm\sqrt{\frac{2}{m}\left( h-\frac{mg}{l}(1-\cos\varphi) \right) }$$ 
Нарисуем фазовые траектории, и ещё функцию 
$F(\varphi)=\frac{mg}{l}(1-\cos\varphi)$.
Уровни постоянной энергии - одномерные торы. Как и обычно с функцией
Гамильтона. 
Из анализа фазовых траекторий можно выяснить, что период колебаний растет по
мере увеличения энергии. Также есть два состояния равновесия: верхнее 
(неустойчивое) и нижнее (устойчивое). 










%лекция 03.10.22
\section{Уравнения и ряды Тейлора}
Пусть $\frac{dx}{dt}=f(t,x)$. Рассмотрим $x(t_0)=x_0$. Разложим в ряд
Тейлора: $x(t)=x(t_0)+\frac{dx}{dt}(t_0)(t-t_0)+o(t-t_0)$.
Отбросив члены высшего порядка (прямо как топовые физики), получим 
приближенное решение. Приближенные решение можно итерировать, и это
будет широко известный \textbf{метод Эйлера} (первого
порядка). $t_{k+1}=t_k+h,~x_{k+1}=x_k+f(t_k,x_k)h$ 

\section{Практика}
\textbf{Пример (№111)}. $(y+\sqrt{xy})dx=xdy$. Уравнение однородно (
проверим умножением на $k$). Значит, делаем замену $u(x)=\frac{y}{x}$.
Имеем $dy=u\cdot dx+du\cdot x$. Переменные разделяются: 
$\frac{dx}{x}=\frac{du}{\sqrt{u}}$\\
\textbf{Пример (№113)}. $(2x-4y+6)dx+(x+y-3)dy$. Переносим начало координат
в точку пересечения.\\
\textbf{Пример (№126)}. $y'=y^2-\frac{2}{x^2}$. Это - обобщенно-однородное
уравнение, то есть приводится к однородному заменой $y=z^m(x)$.
$y'=mz^{m-1}z$ Далее
$mz^{m-1}z=z^{2m}-\frac{2}{x^2}$ 
Теперь уравнение однородно. Введем замену $\frac{z}{x}=u,~z=ux$.
Получим $u'x+u=-1+2u^2$\\
\textbf{Пример (№128)}. $\frac{2}{3}xyy'=\sqrt{x^6-y^4}+y^2$. 
Пусть $y=z^m$. Идея: сделать так, чтобы под корнем степень у $x$ и $y$ была
одинаковой.\\
\textbf{Пример (№)} $2xydx+(x^2-y^2)dy=0$. Подберем функцию, полным 
дифференицалом которого является это выражение; получим  $F(x,y)=
x^2y-\frac{1}{3}y^3$. Решние: $F=C=const$\\
\textbf{Пример (№192)}. $(1+y^2\sin{2x})dx-2y\cos^2{x}dy$. Мы должны 
показать, что вторые производные равны. Тогда это значит, что
$F_{xy}=F_{yx}$, и такая функция вообще существует на некотором диске
(где правая часть не обращается в ноль). Интегируем два раза, и найдем эту
функцию: $F(x,y)=x-y^2 \frac{1}{2}\cos{2x}-\frac{y^2}{2}+C_0$.
Итак, ответ: $\boxed{F=const}$ \\
\textbf{Пример (№202)}. $y^2dx+(xy+\tg{xy})dy=0$. Является ли однородным,
в полных дифференциалах? Давайте раскроем скобки и сгруппируем:
$y(ydx+xdy)+\tg{xy}dy$. Это то же, что и  $\frac{d(xy)}{\tg{xy}}+\frac{dy}{y}
=0$. Домножим на $\frac{1}{y\tg{xy}}$ и хаваем уравнение в полных 
дифференицалах бесплатно. То, на что домножили - интегрирующий множитель.
















\end{document}
