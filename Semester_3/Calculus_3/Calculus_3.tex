\documentclass[a4paper]{book}

%Общие настройки документа
\usepackage[14pt]{extsizes}                                         %Размер шрифта
\usepackage[left=2.5cm,right=2.5cm,top=2.5cm,bottom=3cm]{geometry}  %Поля страницы

%Настройки ссылок и гиперссылок
\usepackage{hyperref}                 
\usepackage{xcolor}
\definecolor{linkcolor}{HTML}{799B03} % цвет ссылок
\definecolor{urlcolor}{HTML}{799B03}  % цвет гиперссылок
\hypersetup{pdfstartview=FitH,linkcolor=linkcolor,urlcolor=urlcolor,colorlinks=true}
\usepackage{manfnt}
%Пакеты символов
\usepackage{cmap}
\usepackage[T2A]{fontenc}
\usepackage[utf8]{inputenc}
\usepackage[russian]{babel}           
\usepackage{amsmath}
\usepackage{amssymb}
\usepackage{amsfonts}
\usepackage{array}

%Новые команды 
\newtheorem{defin}{Определение}
\newtheorem{example}{Пример}
\newtheorem{zam}{Замечание}
\newtheorem{theor}{Теорема}

\newcommand{\dnote}[1]{%
    \noindent % I guess this is intended...
    \begin{tabular}{@{}m{0.13\textwidth}@{}m{0.87\textwidth}@{}}%
        \huge\textdbend &#1%
    \end{tabular}%
    \par % ... and this too.
}
\DeclareMathOperator{\sign}{sign}
\author{Галкина}
\title{Анализ 3}
\date{05.09.2022}

\begin{document}
\maketitle
\tableofcontents
\newpage
%5.09.22
%Коэффициенты: Контр*0,4 Коллок*0,3 Экз*0,3
\chapter{Идеи и номера с практики}
Идеи Тимура, достойные того, чтобы быть запечатленными.
Те места, которые на слух отмечаются словами типа <<финт ушами>>, будут 
отмечаться знаком <<опасный поворот>> \dbend в стиле Бурбаки (а не то, что
вы подумали). 

\section{Знакопостоянные несобственные интегралы}
Для знакопеременных интегралов можно использовать признак сравнения.
Обычно сравнение происходит с обобщенной степенной функцией.
При этом имеется два различных типа особых точек: на бесконечности и 
с уходом на бесконечность в точке. Разберем подробнее.

\textbf{Пример 1.} Интеграл 
$$\int\limits_{1}^{\infty}\frac{1}{x^\alpha}dx$$ 
сходится при $\alpha>1$ и расходится при $\alpha\leqslant 1$.

\textbf{Пример 2.} Интеграл
$$\int\limits_{a}^{b}\frac{1}{(x-a)^\alpha}dx$$
сходится при $\alpha<1$ и расходится при $\alpha\geqslant 1$. 

\textbf{Пример.} Интеграл $\int\limits_{1}^{\infty} \frac{x^2dx}{x^4-x^2+1}$ 
сходится, поскольку подынтегральная функция эквивалентна $\frac{1}{x^2}$ - 
сходящейся штуке.


\textbf{Пример (№2374)}. Исследуем на сходимость в зависимости от параметров
интеграл
$$\int\limits_{1}^{\infty} \frac{1}{x^p\ln^q x}dx$$
Имеем 2 особые точки: 1 и $\infty$, поэтому разобъем область исследования
на две части и будет исследовать интеграл $\int\limits_{10}^{\infty}$.\\
Нам поторебуется следующий признак сравнения: для $\varepsilon>0$
$$\frac{1}{x^\varepsilon}<\ln^\alpha(x)<x^\varepsilon,~
x>\delta(\alpha,\varepsilon)$$ 
(доказательство через правило Лопиталя: действительно, 
$\lim\limits_{n \to \infty} \frac{\ln^\alpha(x)}{x^\varepsilon}=0$).\\
Значит, имеем
$$\frac{1}{x^{p+\varepsilon}}\leqslant \frac{1}{x^p\ln^qx}\leqslant 
\frac{1}{x^{p-\varepsilon}}$$ 
\dnote{Итак, интеграл сходится при $p>1+\varepsilon$ и расходится при
$p<1-\varepsilon$. Так как $\varepsilon$ вообще-то произвольный,
то и условие сходимости не должно зависеть от него; иначе говоря, 
интеграл сходится при $p>1$ и расходится $p<1$.}
Рассмотрим случай, когда $p=1$. Имеем 
$$\int\limits_{10}^{\infty} \frac{1}{x\ln^qx}dx = 
\begin{cases}\ln(x)=t\\dt=\frac{dx}{x}\end{cases} = 
\int\limits_{\ln 10}^{\infty}\frac{dt}{t^q}$$ 
Значит, этот интеграл сходится при $q>1$.
Соберем ответ:
$$\begin{cases}
    1.~ p>1 - \text{сходится;}\\
    2.~ p<1 - \text{расходится;}\\
    3.~ p=1,q>1 - \text{сходится;}\\
    4.~ p=1,q\leqslant 1 - \text{расходится.}
\end{cases}$$



\section{Знакопеременные несобственные интегралы}
Напомним, что для применения признаков Абеля и Дирихле в интеграле 
$\int\limits_{a}^{\infty} f(x)g(x)dx$, необходимо, чтобы $f(x)$ и $g'(x)$ были 
непрерывными функциями.

\textbf{Пример.} $$\int\limits_{1}^{\infty}\frac{\sin(x)}{x^\alpha}$$ 
Интеграл имеет одну особую точку: $+\infty$.\\
Сначала расмотрим абсолютную сходимость: 
$\frac{|\sin(x)|}{x^\alpha}\leqslant \frac{1}{x^\alpha}$, откуда
по признаку сравнения получаем, что интеграл сходится абсолютно при 
$\alpha>1$.\\
Рассмотрим обычную сходимость: интеграл удовлетворяет признаку Дирихле,
поскольку 
$\forall y>a: \int\limits_{a}^{y}\sin(x)dx=-\cos(y)+\cos(a)\leqslant 20$ и
$\frac{1}{x^\alpha}\to 0$ монотонно. Значит, интеграл сходится при 
$\alpha>0$.\\
Теперь рассмотрим расходимость интерала. Докажем условную сходимость на
$(0,1]$.
Оценим снизу увадратом синуса:
$$\frac{|\sin(x)|}{x^\alpha}\geqslant \frac{\sin^2(x)}{x^\alpha}=
\frac{1-\cos(2x)}{2x^\alpha}=\frac{1}{2x^\alpha}-\frac{\cos(2x)}{2x^\alpha}$$ 
Вторая дробь сходится по Дирихле, откуда весь интеграл расходится абсолютно
при $\alpha\leqslant 1$.\\
Осталось установить сходимость при $\alpha\leqslant 0$. Вспомним определение
\textbf{предела по Гейне}:
$$\forall \{y_n\}\to 0: \lim\limits_{n \to \infty} \int\limits_{a}^{y_n}
f(x)dx\to const$$ 
Тогда интеграл можно предстваить в виде $\sum\limits_{n=1}^{\infty} 
\int\limits_{y_n}^{y_{n+1}}f(x)dx$. Найдем какую-нибудь последовательность,
на которой будет расходимость. Итак, пусть $y_n=\pi n$.\\
Теперь нам потребуется следующая 
\begin{theor}(о среднем)\\
Еcли $f(x)$ непрерывна и  $g(x)$ знакопостоянна, тогда
$$\int\limits_{a}^{b}=f(\xi)\cdot\int\limits_{a}^{b} g(x)dx,~\xi\in(a,b)$$
\end{theor}
Из теоремы получаем, что
$$\int\limits_{\pi n}^{\pi n+\pi} \frac{\sin(x)}{x^\alpha}dx = 
\frac{1}{\xi^\alpha_n}\int\limits_{\pi n}^{\pi n+\pi}\sin(x)dx = 
\frac{2\cdot (-1)^n}{\xi^\alpha_n}$$ 
Тогда интеграл равен
$$\sum\limits_{n=1}^{\infty}\frac{2\cdot (-1)^n}{\xi^\alpha_n}$$
Ряд расходится по необходимому признаку, поэтому интеграл расходится по 
опредлению Гейне.\\
Можно доказать то же самое по критерию Коши. Именно, при $\alpha\leqslant 0$:
$$\exists \varepsilon>0~\forall \delta~\exists y_1,y_2>\delta:
\left| \int\limits_{y_1}^{y_2} \frac{\sin(x)}{x^\alpha}\right|>
\varepsilon$$ 
Чтобы убить модули, выберем такие пределы интегрирования, на которых
синус знакопостоянен. Имеем 
$$\int\limits_{2\pi n}^{2\pi n+n} \frac{\sin(x)}{x^\alpha}dx=
\frac{1}{\xi^\alpha_n}\cdot 2,~2\pi n\leqslant \xi_n\leqslant 2\pi n+\pi$$
Подставив худший вариант, получаем $\frac{2}{(2\pi n)^\alpha}\geqslant 2$,
то есть расходимость.\\
Соберем ответ: 
$$\begin{cases}
    1.~ \alpha>1 - \text{сходится абсолютно;}\\
    2.~ 0<\alpha\leqslant 1 - \text{сходится условно;}\\
    3.~ \alpha\leqslant 0 - \text{расходится.}
\end{cases}$$
%17.11.22
Поговорим про суммы. Более-менее очевидно, что если 
$\int\limits_{a}^{b}f(x)dx$ и $\int\limits_{a}^{b}g(x)dx$ сходятся, 
то и $\int\limits_{a}^{b}(f(x)+g(x))dx$ сходится. Так же и для абсолютной
сходимости. Так, $f(x)=\frac{\sin(x)}{x},~g(x)=-\frac{\sin(x)}{x}$ - сходятся 
условно, но их сумма сходится абсолютно. 

\textbf{Пример (№2380в).} $\int\limits_{0}^{\infty}x^2\cos(e^x)dx$. Одна 
особая точка - $\infty$. Невероятно, но он сходится, так как пики косинуса 
становятся всё тоньше и тоньше. \\
\dnote{Идея: чтобы проинтегрировать, надо добавить что-то такое, что можно 
внести под дифференциал. Домножим и разделим подынтегральную функцию на 
экспоненту, получим $\frac{x^2e^x\cos(e^x)}{e^x}$, и проинтегрируем
$\frac{\cos(e^x)}{e^x}$, а остальное выкинем из интеграла по теореме о 
среднем}.\\
Оценим монотонность: $(x^2e^{-x})'=2xe^{-x}-x^2e^{-x}$. При  $x>2$ 
производная отрицательна, значит, стремеление к нулю монотонно. 
Значит, по Дирихле он сходится, так как первообразная косинуса ограниченна.\\
Рассмотрим абсолютную сходимость: 
$$|x^2\cos(e^x)|\geqslant x^2\cos^2(e^x)=\frac{x^2}{2}+
\frac{x^22\cos(2e^x)}{2}$$
Здесь первая дробь расходится, вторая сходится аналогично самому интегралу,
то есть интеграл не сходится абсолютно. 

\textbf{Пример.}
Построим пример положительной функции, которая неограниченна, но интеграл
от неё сходится. Будем строить штуки с площадью $\frac{1}{2^n}$ 
интервалах $(n,n+1)$. Суммировав площади, получим, что интеграл сходится.\\
\dnote{Но по определению Гейне мы должны показать сходимость при любом выборе
последовательности! (а в данном случае мы взяли $x_n=n$). На самом деле, 
для знакоположительных функций при выборе любой последовательности 
пределы частичных сумм
$\sum\limits_{n=1}^{k} \int\limits_{n}^{n+1}f(x)dx $ одинаковы!}
Действительно, любую частичную сумму последовательности можно 
зажать между членами $x_{n}$ и $x_{n+1}$ последовательности $x_n=n$, а 
её предел одинаков.\\
Если мы возьмем знакопеременную функцию, то если она сходится при самой 
<<плохой>> последовательности, то она сходится при любой другой 
последовательности. Причем самая плохая последовательность состоит из тех 
точек, где функция меняет знак. Имеет место следующая
\begin{theor}
    Если $f(x)$ - знакопеременная функция и  $\{x_n\}$ - последовательность,
    состоящая из точек, где функция меняет знак, то из сходимости
    $\sum\limits_{n=1}^{\infty} \int\limits_{x_n}^{x_{n+1}}f(x)dx$ 
    следует сходимость $\int\limits_{1}^{\infty}f(x)dx$
\end{theor}
Геометрический смысл: при данном выборе последовательности отрицательные и 
положительные члены имеют наибольший возможный размер. 

\textbf{Пример (№2380а)} $\int\limits_{0}^{\infty}x^p\sin(x^q)dx,~q\ne 0$. 
Две особые точки: $0$ и  $\infty$. Рассмотрим сначала на бесконечности.\\
1. Если $q<0$, то интеграл знакопостоянный:
$$x^p\sin(x^q)\sim x^{p+q}$$ 
Значит, интеграл сходится при $p+q<-1$.\\
2. Теперь займемся ситуацией, когда $q>1$, и интеграл знакопеременный. 
Применим идею идею из предыдущего номера:
домножим сверху и снизу на какую-нибудь штуку, в данном случае $qx^{q-1}$. 
Получим
$$\frac{x^p\sin(x^q)\cdot qx^{q-1}}{qx^{q-1}}=- \frac{x^{p-q+1}}{q}\cdot 
(\cos(x^q))'$$
Эта штука сходится по Дирихле при $p-q+1<0$, так как $-\frac{1}{q}x^{p-q+1}$ 
монотонно стремится к нулю.\\
3. Рассмотрим абсолютную сходимость:
$$|x^p\sin(x^q)|\leqslant x^p $$
Сходится абсолютно при $p<-1$.\\
4. Рассмотрим ситуацию, когда  $-1< p\leqslant -1+q$. Докажем, что 
здесь сходимость условная.
$$|x^p\sin(x^q)|\geqslant x^p\sin^2(x^q)=\frac{x^p}{2}-
\frac{x^p\cos(2x^q)}{2}$$
Первая дробь расходится, вторая дробь сходится при $p-q+1<0$ по аналогии
с самим интегралом. Значит, интеграл не сходится абсолютно.\\
5. Докажем, что интеграл расходится при $p\geqslant -1 + q$.
Рассмоторим последовательность, из точек, где синус меняет знак, и, согласно
теореме, оценим интеграл рядом 
$\sum\limits_{n=1}^{\infty}\int\limits_{(\pi n)^
{\frac{1}{q}}}^{(\pi n+\pi)^{\frac{1}{q}}}x^p\sin(x)dx$. 
Заметим, что $(\pi n)^{\frac{1}{q}}\to \infty$. Снова домножим на $qx^{q-1}$.
Тогда
$$\frac{1}{q}\xi_n^{p-q+1}\int\limits_{(\pi n)^{\frac{1}{q}}}^{(\pi n+\pi)^
{\frac{1}{q}}}\sin(x^q)qx^{q-1}dx=
\frac{1}{q}\xi_n^{p-q+1}\left( -\cos(x^q) \right)
\big|_{(\pi n)^{\frac{1}{q}}}^{(\pi n+\pi)^{\frac{1}{q}}}=
\frac{2(-1)^{n+1}}{q}\xi_n^{p-q+1}$$
Значит, ряд $\sum\limits_{n=1}^{\infty}\frac{2(-1)^{n+1}}{q}\xi_n^{p-q+1}$
расходится по необходимому признаку, так как $\xi_n^{p-q+1}\geqslant
(\pi n)^{\frac{p-q+1}{q}}\to \infty$.\\
Теперь рассмотрим интеграл в нуле. 
$$\int\limits_{0}^{1}x^p\sin(x^q)dx=\begin{cases}t=\frac{1}{x}\\
dt=-\frac{dx}{x^2}\\0\mapsto \infty\\1\mapsto 1 \end{cases}=
\int\limits_{1}^{\infty}\frac{1}{t^{p+2}}\sin\left(\frac{1}{t^q}\right)dt$$
Получили ситуацию один в один, только вместо $p$ и $q$ будет
$p+2$ и $-q$. 

\textbf{Пример (№2373).}
$\int\limits_{0}^{\infty}\frac{\sin(\ln(x))}{\sqrt{x}}dx$. Сначала исследуем
в нуле:
$$\left| \frac{\sin\ln(x)}{\sqrt{x}} \right|\leqslant \frac{1}{\sqrt{x}}$$ 
значит, сходится абсолютно.\\
Теперь исследуем на сходимость на бесконечности. Так как 
$(\cos\ln(x))'=- \frac{\sin\ln(x)}{x}$, представим функцию в виде 
$\frac{\sin\ln(x)}{x}\cdot \sqrt{x}$ и возьмем худшую последовательность
$x_n=e^{\pi n}$:
$$\int\limits_{e^{\pi n}}^{e^{\pi n+\pi}}\frac{\sin\ln(x)}{x}\cdot\sqrt{x}dx=
\sqrt{\xi_n}\cdot\int\limits_{e^{\pi n}}^{e^{\pi n+\pi}}\frac{\sin\ln(x)}{x}dx=
\sqrt{\xi_n}\cdot(\cos\ln(x))\big|_{e^{\pi n}}^{e^{\pi n+\pi}}=
$$ $$=\sqrt{\xi_n}(-1)^{-1}\cdot 2\to \infty$$
- интеграл расходится.

















%\begin{defin}
    Пусть Х - множество. Топологией на Х называется семейство подмножеств 
    $\tau\in\mathcal{P}(X)$, называемых открытыми множествами (данной 
    топологии), такое, что:\\
    1. $X,\varnothing\in\tau$\\
    2. $U_1,\ldots U_n\in\tau\Rightarrow\bigcap\limits^{n}_{i=1}U_i\in\tau$\\
    3. $\{U_i\mid i\in I\}\subset\tau\Rightarrow\bigcup\limits_{i\in I}U_i\in\tau$
\end{defin}
То есть, топологии принадлежит само множество и пустое множество, пересечение
конечного числа множеств и объединение любого числа множеств из топологии. 

Пример. Докажем, что открытые множества в смысле евклидовой метрики в 
$\mathbb{R}^n$ - топология. Очевидно, открыто само $\mathbb{R}^n$, также 
открытои пустое множество. Открытость пересечения доказывается тем, что
наименьшая эпсилон-окрестность принадлежит всем множествам,то есть лежит в их
пересечении, слеовательно, оно открыто. Для объединения: для каждой точки 
найдется множество, в которое она входит с окрестностью.

\begin{defin}
Тривиальная топология - $\tau_t=\{X,\varnothing\} $ \\
Дискретная топология - $\tau_0=\mathcal{P}(X)$ 
\end{defin}
Любая инетерсная топология содержит тривиальную и содержится в дискретной.

Пример. Множества, симметричные относительно выбранной прямой в евклидовом
пространстве,образуют топологию.

Пример. Множество эпсилон-окрестностей нуля $\tau=\{D_\varepsilon(0)\mid
\varepsilon>0\}\cup\{X,\varnothing\} $
- топология.

Пример. Топология Зарисского - топология множеств, дополнительных к конечным 
множествам (для конечных пространств совпадает с дискретной).

Пример. Пусть $f:X\to X$ - биекция. Докажем, что $\tau_f=\{U\subset X\mid$






%Лекция+семинар 08.09.22
%\begin{theor}
    (критерий сходимости для неотрицательных рядов)\\
    Пустьдан ряд. Тогда ряд сходится $\Leftrightarrow$ последовательность
    частичных сумм ограничена сверху.
\end{theor}
\textbf{Доказательство.} $\Rightarrow$. По услови, существует предел 
$lim S_n=S\in \mathbb{R}$ $\Rightarrow$ $\{S_n\}_n\in\mathbb{N}$ - ограничена
В другую сторону. По условию, $\{S_n\}$ ограничена сверху, $\Rightarrow$ по тео
реме Вейрштрасса для ограниченной неубывающей последовательности имеется предел
$\square$ \\

\textbf{Признак сравнения.} С чем же сравнивать? С геометрической прогрессией, 
с обобщенным гармоническим рядом (с произвольной степенью числа). 
\begin{theor}
    (признак сравнения в оценочной форме)\\
    Дано $0\leqslant a_n\leqslant b_n~\forall n\in\mathbb{N}$ :
    Тогда из сходимости В следует сходимость А, из расходимости А следует
    расходимость В.
\end{theor}
\textbf{Доказательство.}  Докажем исходя из критерия сходимости. \\
1. Пусть $A_n,B_n$ - частичные суммы своих рядов. Так как ряд В сходится, 
то существует верхний предел для его частичных сумм. Так как ряд А меньше Б,
по транзитиавности неравенств верхняя граница В лежит выше чем А. ЧТо по тому 
же критерию дает сходимость. 
2. 
$\square$ \\

Пример. Рассмотрим $p<1$,  $n^p<1$,  $\frac{1}{n^p}>\frac{1}{n}$. Так как 
гармонический ряд расходится, то $sum \frac{1}{n^p}$ расходится. 

\textbf{Пример.} Найти сумму. $\sqrt{2}+\sqrt{2-\sqrt{2} }+\sqrt{2-
\sqrt{2+\sqrt{2} } } +...$, $a_{n+1}=\sqrt{2-b_n} $, $b_{n+1}=\sqrt{2+b_n} $.
Заметим, что $b_1=2\cos\frac{\pi}{4}$, $b_2\cos\frac{\pi}{8}$. Дальше
эта формула выводится по индукции. $b_n=2\cos\frac{\pi}{2^{n+1}}$. 
$a_n=\sqrt{2-b_{n-1}}=\sqrt{2-2\cos\frac{\pi}{2^n}}=2\sin\frac{\pi}{2^{n+1}}$ 
Ита, $a_n\leqslant 2\cdot \frac{\pi}{2^{n+1}}=\frac{\pi}{2^n}$ 

\begin{theor}
    (Признак сравнения в предельной форме)\\
    Пусть даны неотрицательные ряды $\sum\limits_{n=1}^{\infty} a_n$, 
    $\sum\limits_{n=1}^{\infty}b_n$. Если предел отношения общего члена\\
    1. Равен конечной (ненулевой) константе. Тогда ряды сходятся или расходятся 
    одновременно\\
    1.1. В частности, при mkk=1, ряды эквивалентны. 
2. Если $\lim\limits_{n \to \infty} \frac{a_n}{b_n}=0$, то имеет место
"В сходится $\Rightarrow$ А сходится"
3. Если этот предел равен $ooo$, то: "А сходится $\Rightarrow$ В сходится"


\end{theor}
\textbf{Доказательство.} По опреелению предела. 
$\lim\limits_{n \to \infty} a_\frac{n}{b_n}=k$ для $\varepsilon=k/2>0\exists 
n_0(\varepsilon)\forall n>n_0: k/2<\frac{a_n}{b_n}<3k/2$. тогда если В 
сходится, А сходится.

2. Пусть $\lim\limits_{n \to \infty} a_\frac{n}{b_n}=0$. Lkz $\varepsilon=1$, 
тогда для этого эпсилон  $\exists n_0$ утверждение следует из первого
признака сравнения. 

Пункт 3 напрямую следует из второго.
$\square$ 

\textbf{Пример. 3}  $\sum\limits_{n=1}^{\infty}(\frac{1}{n^\alpha}-
\frac{1}{(n+1)^\alpha})$. Имеем $S_n=1-\frac{1}{(n+1)^\alpha}$ Прии 
альфа>0 сходится к 1, при альфа<0 ряд расходится.
(ljнайддем область расходимости обобщенного гармонического
рядва с помощью уже известного)

\begin{theor}
    (тертий признак сравнения.)\\
Пусть даны ряды А и В ($\sum\limits_{n=1}^{\infty} a_n,~\sum
\limits_{n=1}^{\infty} b_n$),и выполняется $a_{n+1}/a_n\leqslant b_{n+1}/b_n$
Тогда В сходится $\Rightarrow$ А сходится
(если А расходится, В расходится)

\end{theor}
\textbf{Доказательство.}  так как все неравенства полоэительные, их всех можно
перемножить: тогда утверждение следует из первого признака сравнения. 
\
$\square$ 

\begin{theor}
    (Признак Даламбера в оценочной форме)\\
\end{theor}
\textbf{Доказательство.}  \
$\square$ 

\begin{theor}
Признак даламбера в предельной форме: $\lim\limits_{n \to \infty} $
\end{theor}
\textbf{Доказательство.}  \
$\square$ 



















%Лекция 19.09
%\section{Элементарные методы интегрирования ДУ}
\subsection{Уравнения с разделяющимися переменными}
\begin{defin}
Уравнение с разделяющими переменными - уравнение вида
\begin{equation}
    \frac{dx}{dt}=f(x)g(t) \label{ODE_razdp}
\end{equation}
где $f,g$ непрерывны на  $x\in(a,b),~t\in(\alpha,\beta)$
\end{defin}
Как решать такие уравнения? Алгебраическая нтуиция подсказывает, что надо 
перенести 
дифференциалы к своим функциям и проинтегрировать. Но это ещё надо обосновать.
Сделаем следующее:\\
\begin{enumerate}
    \item Найти все $x_*:f(x_*)=0$. Тогда $x=x_*$ - решение-константа. 
    \item Пусть  $x^i_*,x^j_*$ - такие, что  $f(x^i_*)=f(x^j_*)=0$ и
    $\forall x\in(x^i_*,x^j_*):f(x)\ne0$. Тогда уравнение \ref{ODE_razdp}
эквивалентно уравнению 
$$\frac{dx}{f(x)}=g(t)dt$$
Эту штуку можно проинтегрировать с обеих сторон. Результат непрерывен и не
обращается в ноль. Значит, по теореме о неявной функции найдется решение. 
$\frac{dF}{dx}=\frac{1}{x}$(решение в области $(\alpha,\beta)\times
(x^i_*,x^j_*)$).
    \item Выписать решение на каждом интервале $(x^i_*,x^j_*)$
\end{enumerate}
Других решений не существует. Почему? Допустим, существует другое решение.
Оно не может быть константой, так как все константы были получены в п.1.
Если она \\
\textbf{Пример.} Решим уравнение $\frac{dx}{dt}=0$. Решение-константа: $x=0$.
Теперь рассмотрим два интервала: $x<0$ и  $x>0$. Если  $x<0$, имеем уравнение
 $$\frac{1}{x}\frac{dxdt}{dt}=dt$$
 Интегрируем:
 $$\int\frac{dx}{x}=\int dt$$
 Получаем, что $\ln|x|=t+C$. Выражаем искомую функцию (не забыв, на каком
 промежутке мы рассматриваем функцию, и раскрыв модуль соответственно):
 $$x=-Ce^t,~C>0$$
Для интервала $x>0$ точно такой же порядок действий, только получим другой 
знак. Итак, множество решений:
$$x=Ce^t,~C\in\mathbb{R}$$
\subsection{Уравнения, приводящиеся к уравнению с разделяющимися переменными}
\begin{defin}
Уравнение, приводящееся к уранвению с разделяющмися переменными - уравнение
вида 
\begin{equation}
    \frac{dx}{dt}=f(at+bx+c) \label{ODE_privrazd}
\end{equation}
\end{defin}
Давайте решим его. 
\begin{enumerate}
    \item Введем замену $z(t)=at+bx+c$. 
    Имеем
     $$\frac{dz}{dt}=a+b\frac{dx}{dt}$$ 
     Получаем уравнение с разделяющимися переменными. 
     $$\frac{dz}{a+f(z)}=dt$$
\end{enumerate}
\textbf{Пример.} Решим уравнение $\frac{dx}{dt}=\cos(x+t)$. Замена 
$z=x+t,~ \frac{dz}{dt}=1$. Уравнение имеет вид
$$\frac{dz}{dt}=\frac{dx}{dt}+1$$ 
Найдем $\cos{z_*}+1=0$: это, очевидно, $\pi+2\pi k,~k\in \mathbb{Z}$ 
Свели задачу кпрошлому пункту
\subsection{Однородные уравнения}
Сначала докажем, что два определения однородного уравнения эквивалентны.
\begin{defin}
Однородным называется уравнение вида
\begin{equation}\label{ODE_odn1}
    \frac{dx}{dt}=f\left(\frac{x}{t}\right) \label{ODE_odn1}
\end{equation} 
\end{defin}
Это уравнение инвариантно относительно замены $x\mapsto kx,~t\mapsto kt$.
Геометрически это означает, что совокупность интегральных кривых инвариантно
относительно преобразования $\theta(x,y)=(kx,ky)$.
Из этого следует, что если мы найдем одно решение, то мы найдем всю 
совокупность ему подобных. Вставить картинку.
\begin{defin}
    (вспомогательное)\\
Уравнение в форме дифференциалов:
    $M(x,y)dx+N(x,y)dy=0$.  
\end{defin}
Это таже форма, что и $\frac{dy}{dx}=f(x,y)$, поскольку 
$\frac{dy}{dx}=-\frac{M(x,y)}{N(x,y)}$. Обратно, $-f(x,y)dx+dy=0$.
Уравнение в форме дифференциалов имеет чуть большее множество решений. 
\begin{defin}\label{ODE_odn2}
Уравнение в форме дифференциалов называется однородным, если\\
$M(kx,ky)=k^nM(x,y)$\\ 
$N(kx,ky)=k^nN(x,y)$\\
n называется степенью однородности.
\end{defin}
\begin{theor}
    Определения \ref{ODE_odn1} и \ref{ODE_odn2} эквивалентны. 
\end{theor}
\textbf{Доказательство.} 1 $\Rightarrow$ 2. $\frac{dy}{dx}=f(\frac{y}{x})$\\
2 $\Rightarrow$ 1. Пусть дано уравнение в форме дифференциалов. Подставим $k$.
При $x\ne 0$ Имеем $$\frac{dx}{dy}=-\frac{k^nM(x,y)}{k^nN(x,y)}=
-\frac{M(kx,ky)}{N(kx,ky)}=-\frac{M(1,\frac{y}{x})}{N(1,\frac{y}{x})}=f(x)$$
$\square$ \\
\textbf{Пример.} $M=x^2+y^2$\\
\textbf{Пример (№31).} Найти уравнение, решение которых - параболы с осью, 
параллельной оси ординат и касающиеся прямых $y=0,~y=x$. 
Во-первых, поймем, как выглядит уравнение такой параболы. Исходя из геометрии,
получим, что уравнение параболы, удовлетворяющее первому условию, имеет вид 
$y=ax^2+bx+\frac{b^2}{4a}$, а первому и второму - $y=ax^2+\frac{1}{2}x+
\frac{1}{16a}$. Остался один параметр $\Rightarrow$ уравнение первого порядка. 
Подставляем и хаваем ответ бесплатно:
$$y=\left(\frac{y'-\frac{1}{2}}{2x}\right)x^2+\frac{1}{2}x+\frac{2x}{16y'-8}$$ 
\textbf{Пример (№72).} Найти линии, у которых треугольники, образованные 
касательными, осью ОХ и точкой касания, имеют одинаковую сумму катетов. 
Из геометрических соображений имеем уравнение 
$$\frac{|y|}{|y'|}+|y|=b=const$$ 
Раскрываем модули. В простейшем случае имеет уравнение с разделяющимися 
переменными. 
$$\frac{dy}{dx}=\frac{y}{b-y}$$ 
Остальные уравнения такие же в принципе. Так шо это идет в дз 
Его легчайшее (и, видимо, общее) решение: $x+C=\pm b\ln{|y|}\pm y$\\
\textbf{Пример (№76).} Геометрическая интуиция не должна подводить нас. 
Вставить картинку. Есть кароч такая формула: 
$\tg\gamma=\frac{r}{r'}$





%Лекция 22.09
%\section{Связь признака Даламбера и Коши}
Если $\frac{a_n}{a_{n-1}}\leqslant q$ для 
всех n начиная с 1, то $a_n=a_1q^n$, откуда следует признак Коши. 
$$\sqrt[n]{a_n}\leqslant \sqrt[n]{a_1}\cdot q$$
Значит, Коши покрывает больше случаев. 
\section{Оценка погрешности приближения какой-то величины с помощью
положительного ряда}
$$\int^\infty_{n+1} f(x)dx<R_n\leqslant \int^\infty_nf(x)dx$$ 
Из доказательства интегрального признака
$$a_{k+1}<\int^{k+1}_kf(x)dx\leqslant a_k$$ 
$$\int^{k+1}_kf(x)dx\leqslant a_k\int^k_{k-1}f(x)dx$$ 
$$R_n=\sum\limits_{k=n+1}^{\infty} a_k$$ 
Итак, 
$$\int^\infty_{n+1}\leqslant R_n<\int^\infty_nf(x)dx$$
\textbf{Пример.} Вычислим с точностью до 0,001 ряд 
$\sum\limits_{n=1}^{\infty} \frac{1}{n^4}$. Ответ: $1,082\pm0,001$
(точный ответ $\frac{\pi^4}{90}$)
\section{Знакопеременные ряды}
Пусть теперь ряд знакопеременный.
\begin{defin}
Ряд сходится абсолютно, если сходится ряд из модулей. Ряд сходится условно,
если абсолютно расходится, но сам сходится. 
\end{defin}
\begin{theor}
Если ряд сходится абсолютно, то ряд сходится.
\end{theor}
\textbf{Доказательство.}  Следует напрямую из критерия Коши
и свойства модуля: $| |a_1|+...|a_n| |\geqslant|a_1+...+a_n|$.
$\square$ 
\begin{theor}
    (признак Лейбница для знакочередующихся рядов)\\
    Пусть ряд имеет вид $\sum\limits_{n=1}^{\infty} (-1)^nv_n$, 
    где $v_n>0$ и монотонно убывает. Тогда ряд сходится.  
    Более того, имеет место оценка погрешности $|R_n|\leqslant v_n$
\end{theor}
\textbf{Доказательство.}  1. Посчитаем частичную сумму для $2k:$
$$S_{2k}=v_1-v_2+...-v_{2k}$$ 
$$S_{2k+2}=S_{2k}+v_{2k+1}-v_{2k+2}$$ 
$$S_{2k+2}-S_{2k}=v_{2k+1}-v_{2k+2}$$ 
$$S_{2k}=v_1-(v_2-v_3)-(v_4-v_5)-...-(v_{2k-2}-v_{2k-1})-v_{2k}$$ 
Значит, эта последовательность возрастает и ограничена сверху, значит, у неё
есть конечный предел: $S_{2k}\leqslant u_1$
$$\lim\limits_{k \to \infty} S_{2k+1}=\lim\limits_{k \to \infty} (S_{2k}+
v_{2k+1})=S$$ 
Следовательно, 
$$\exists \lim\limits_{n \to \infty} S_n=S$$
Последовательность частичных сумм для нечетных чисел также убывает, 
доказательство аналогичное. \\
2. Докажем оценку погрешности. $|R_{2k}|=S-S_{2k}<S_{2k+1}-S_{2k}$. Итак,
$$|R_{2k}|\leqslant v_{2k+1}$$ 
$$R_{2k+1}=S_{2k+1}-S<S_{2k+1}-S_{2k+2}$$ 
$$|R_{2k+1}|\leqslant v_{2k+2}$$
$\square$ 
\subsection{Преобразование Абеля}
$$\sum\limits_{k=1}^{n} a_k b_k=\sum\limits_{k=1}^{n-1} (a_k-a_{k+1})B_k
+a_nB_n,~B_i=\sum\limits_{k=1}^{i} b_k$$ 
Доказательство. $b_k=B_{k}-B_{k-1},~k\in \{2,...,n\} $ ВСТАВКА

\begin{theor}
    (неравенство Абеля)\\
    Пусть последовательность монотонно возрастает или убывает. 
    И пусть $\exists M\forall k\in \{1...n\}|B_k|\leqslant M $.
    Тогда модуль конечной суммы $\leqslant M(|a_1|+2|a_n|)$ 
\end{theor}
\textbf{Доказательство.} Юра, допиши пж 
$\square$ 
\begin{theor}
    (признак Дирихле)


\end{theor}
\textbf{Доказательство.}  \
$\square$ 












%лекция 
%\section{Действия над абсолютно сходящимися рядами}
\begin{theor}
Если ряд сходится абсолютно, то ряд, умноженный на константу, сходится абсо
лютно. 
\end{theor}
\textbf{Доказательство.} Зафиксируем $\varepsilon$. Найдем такой номер, что
ряд из модулей меньше чем $\frac{\varepsilon}{|c|}$. И в общем эта штука
сходится. 
$\square$ 
\begin{theor}
Сумма абсолютно сходящихся рядов абсолютно сходится.
\end{theor}
\textbf{Доказательство.}  Сумма модулей больше модуля суммы.
$\square$ 
\begin{theor}
    (О произведении абсолютносходящихся рядов)\\
    Сумма всевозможных произведений $a_ib_j$ сходится абсолютно, и сумма ряда
    равна произведению сумм.
\end{theor}
\textbf{Доказательство.} Введем две переменные с модулями. Введем новые
обозначения, как в прошлой теореме. Пользуясь этой же теоремой, мы можем
доказать абсолютную сходимость для хотя бы одного из упорядочиваний. 
Представим себе бесконечную матрицу $|a_ib_j|$. Будем рассматривать 
последовательность частичных сумм в угловых минорах. Для них имеем формулу
$S_{n^2}=S'_n\cdot S''_n$. По условию,в правой части есть оба предела, а 
значит и слева тоже есть. И ещё, $S_{n^2}\leqslant S_m\leqslant S_{(n+1)^2}$.
Ну кароч....че то мдэ, тут дофига текста.
$\square$ 
\begin{defin}
    (произведение рядов по Коши)\\
Пусть $S_a\cdot S_b=S_c$. имеемследующее произведение:\\
$c_1=a_1b_1$\\
 $c_2=a_1b_2+a_2b_1$\\
 $c_3=a_1b_3+a_2b_2+a_3b_3$\\
 То есть суммируем по диагональкам той бесконечной матрицы.
\end{defin}
\textbf{Пример 1.} $a_n=\frac{1}{n(n+1)}=1,~b_n=\frac{n}{2^n}$. Тогда
$\sum\limits_{n=1}^{\infty} c_n=\sum\limits_{n=1}^{\infty} \sum\limits_{k=1}
^{n} \frac{n+1-k}{k(k+1)-2^{n+1-k}}$.\\
\textbf{Пример 2.} Произведение расходящихся рядов $a_n=1,5^n,~b_n=1-1,5^n$
в смысле Коши - сходится, так как $c_n=0,75^n$. \\
Заметим, что условной сходимости недостаточно! Так, для $a_n=b_n=(-1)^{n-1}/
\sqrt{n}$ ничего не выйдет. Смиритесь. Ребят а че вы с пары то свалили. 
Чувствую себя лохом, и от этого неуютненько.
\section{Перестановки условно-сходящихся рядов}
\begin{theor}
Лемма о сходимости. Ряд $a_n$ сходится условно. Рассмотрим отдельно
подпоследовательности из положительных и отрицательных членов. Тогда их суммы
 $+\infty,-\infty$ соответственно. 
\end{theor}
\textbf{Доказательство.}  \
$\square$ 
\begin{theor}
    (Римана)\\
    Если рядсходится услвоно, то для любого действительного числа найдется
    такая перестановка ряда, при которой ряд сходится к этому числу.
\end{theor}
\textbf{Доказательство.} По предыдущей лемме, ряд из положительных членов расходится,
значит, найдется частичная сумма, большая чем искомое число. Дальше найдем 
такую частичну сумм из отрицатльных членов, чтобы, прибавв её к прошлому этапу,
получили снова меньше чем число. И так далее.  
$\square$ 


%лекция 03.10.22
%\section{Уравнения и ряды Тейлора}
Пусть $\frac{dx}{dt}=f(t,x)$. Рассмотрим $x(t_0)=x_0$. Разложим в ряд
Тейлора: $x(t)=x(t_0)+\frac{dx}{dt}(t_0)(t-t_0)+o(t-t_0)$.
Отбросив члены высшего порядка (прямо как топовые физики), получим 
приближенное решение. Приближенные решение можно итерировать, и это
будет широко известный \textbf{метод Эйлера} (первого
порядка). $t_{k+1}=t_k+h,~x_{k+1}=x_k+f(t_k,x_k)h$ 

\section{Практика}
\textbf{Пример (№111)}. $(y+\sqrt{xy})dx=xdy$. Уравнение однородно (
проверим умножением на $k$). Значит, делаем замену $u(x)=\frac{y}{x}$.
Имеем $dy=u\cdot dx+du\cdot x$. Переменные разделяются: 
$\frac{dx}{x}=\frac{du}{\sqrt{u}}$\\
\textbf{Пример (№113)}. $(2x-4y+6)dx+(x+y-3)dy$. Переносим начало координат
в точку пересечения.\\
\textbf{Пример (№126)}. $y'=y^2-\frac{2}{x^2}$. Это - обобщенно-однородное
уравнение, то есть приводится к однородному заменой $y=z^m(x)$.
$y'=mz^{m-1}z$ Далее
$mz^{m-1}z=z^{2m}-\frac{2}{x^2}$ 
Теперь уравнение однородно. Введем замену $\frac{z}{x}=u,~z=ux$.
Получим $u'x+u=-1+2u^2$\\
\textbf{Пример (№128)}. $\frac{2}{3}xyy'=\sqrt{x^6-y^4}+y^2$. 
Пусть $y=z^m$. Идея: сделать так, чтобы под корнем степень у $x$ и $y$ была
одинаковой.\\
\textbf{Пример (№)} $2xydx+(x^2-y^2)dy=0$. Подберем функцию, полным 
дифференицалом которого является это выражение; получим  $F(x,y)=
x^2y-\frac{1}{3}y^3$. Решние: $F=C=const$\\
\textbf{Пример (№192)}. $(1+y^2\sin{2x})dx-2y\cos^2{x}dy$. Мы должны 
показать, что вторые производные равны. Тогда это значит, что
$F_{xy}=F_{yx}$, и такая функция вообще существует на некотором диске
(где правая часть не обращается в ноль). Интегируем два раза, и найдем эту
функцию: $F(x,y)=x-y^2 \frac{1}{2}\cos{2x}-\frac{y^2}{2}+C_0$.
Итак, ответ: $\boxed{F=const}$ \\
\textbf{Пример (№202)}. $y^2dx+(xy+\tg{xy})dy=0$. Является ли однородным,
в полных дифференциалах? Давайте раскроем скобки и сгруппируем:
$y(ydx+xdy)+\tg{xy}dy$. Это то же, что и  $\frac{d(xy)}{\tg{xy}}+\frac{dy}{y}
=0$. Домножим на $\frac{1}{y\tg{xy}}$ и хаваем уравнение в полных 
дифференицалах бесплатно. То, на что домножили - интегрирующий множитель.















%лекция 10.10.22
%\begin{theor}
    (признак Абеля равномерной сходимости функционального ряда)\\
Дан ряд $\sum\limits_{n=1}^{\infty} a_n(x)b_n(x)$ и $\forall x\in X$:\\
1. $|a_n(x)|\leqslant M=const$ для всех $n$;\\
2.  $\{a_n(x)\} $ мнонотонна;\\
3. $\sum\limits_{n=1}^{\infty} b_n(x)$ равномерно сходится на $X$;\\
Тогда исходный ряд равномерно сходится на  $X$.
\end{theor}
\textbf{Доказательство.}  По определению Коши. Фиксируем $\varepsilon>0$.
Так как ряд с общим членом $b_n$ сходится равномерно, то по критерию Коши для
$$\frac{\varepsilon}{3M}>0~\exists n_0(\varepsilon)~\forall n>n_0~\forall p\in
\mathbb{N}~ \forall x\in X:\left|\sum\limits_{k=n+1}^{n+p} b_k(x)\right|<
\frac{\varepsilon}{3M}$$ Тогда по неравенству Абеля 
$$\right|\sum\limits_{k=n+1}^{n+p} b_k(x)a_k(x)\left|\leqslant 
\frac{\varepsilon}{3M}
(|a_{n+1}|+2|a_{n+p}(x)|)<\frac{\varepsilon}{3M}\cdot 3M=\varepsilon$$
Тогда по критерию Коши этот ряд сходится равномерно на $X$. $\square$ 

\textbf{Пример.} Исследуем на равномерную сходимоcть ряд 
$\sum\limits_{n=1}^{\infty} \frac{\cos{nx}\sin{x}arctg{nx}}{\sqrt{n^2+x^2}}$. 
Алгоритм:\\
1. Арктангенс монотонен и ограничен.\\
2. Все остальное сходится по Дирихле.
\subsection{Свойства равномерно сходящихся рядов}
\begin{theor}
    (о непрерывности суммы равномерно сходящегося ряда)\\
    Дан ряд $\sum\limits_{n=1}^{\infty} a_n(x)$, причем \\
    1. Все функции непрерывны на множестве $X$;\\
2. $\sum\limits_{n=1}^{\infty} a_n(x)$ сходится равномерно к $S(x)$ на $X$;\\
Тогда $S(x)$ непрерывна на $X$. 
\end{theor}
\textbf{Доказательство.}  По условию, сумма из  $a_n(x)$ сходится равномерно
на  $X$ к  $S(x)$, то есть  $S_n(x)\rightrightarrows S(x)$ на  $X$, 
$S_n(x)$ непрерывна как сумма. Тогда по теореме о непрерывности предела
равномерно сходящейся последовательности, составленной из непрерывных
функций,  $S(x)$  непрерывна. Другая формулировка:  
$$\lim\limits_{x\to x_0}\sum\limits_{n=1}^{\infty} a_n(x) =
\sum\limits_{n=1}^{\infty}\lim\limits_{x\to x_0} a_n(x)
$$
(то есть можно поменять местами сумму и предел). $\square$ 

\textbf{Пример.} $\sum\limits_{n=1}^{\infty} \frac{\sin{nx}}{n}=f(x)$ - 
непрерывна на $(0,2\pi)$
\begin{theor}
(об интегрировании равномерно сходящегося ряда)\\
Пусть дан ряд $\sum\limits_{n=1}^{\infty} a_n(x)$, причем \\
 1. все функции непрерывны на отрезке $[a,b];$\\
 2. $\sum\limits_{n=1}^{\infty} a_n(x)$ сходится равномерно на $[a,b]$ к $s
 (x)$;\\
 Тогда $$\forall x,x_0\in[a,b]:~\int\limits^x_{x_0}\left( \sum\limits_{n=1}^
 {\infty} a_n(t) \right)dt=\sum\limits_{n=1}^{\infty} \left( 
\int\limits_{x_0}^{x}a_n(t)dt \right) $$ 
 (можно менять интеграл и сумму).
\end{theor}
\textbf{Доказательство.} Докажем, что $\int\limits^x_{x_0}S(t)dt=\sum\limits_{n=1}^{\infty} \int\limits^x_{x_0}a_n(t)dt$. По предыдущей теореме $S(t)$ 
непрерывна на  $[a,b]$, значит,интегрируема на нем по Риману. 
Обозначим  $\sigma_n(x)=\sum\limits_{k=1}^{n}\int\limits^x_{x_0}a_k(t)dt$ и
докажем, что $\sigma_n(x)\rightrightarrows\int\limits^x_{x_0}S(t)dt$.\\
Зафиксируем $\varepsilon>0$. По условию, $S_n(t)$ равномерно сходится на 
$[a,b]$ для  
$$\frac{\varepsilon}{b-a}>0~\exists n_0(\varepsilon)~\forall 
n>n_0~\forall x\in[a,b]:|S_n(t)-S(t)|<\frac{\varepsilon}{b-a}$$
Тогда
$\left|\sigma_n(x)-\int\limits^x_{x_0}S(t)dt\right|=
\left| \sum\limits_{k=1}^{n} \int\limits_{x_0}^{x} a_k(t)dt-
\int\limits_{x_0}^{x}S(t)dt\right|=\left| \int\limits_{x_0}^{x}(S_n(t)-S(t))dt
\right|\leqslant \left| \int\limits_{x_0}^{x}|S_n(t)-S(t)|dt\right| 
<\frac{\varepsilon}{b-a}\cdot |x-x_0|<\varepsilon$.
Значит, $\sigma_n(x)\rightrightarrows \int\limits_{x_0}^{x} S_n(t)dt$.
$\square$ 
\begin{theor}
(о дифференцировании равномерно сходящегося ряда)\\
Пусть дан ряд $\sum\limits_{n=1}^{\infty} a_n(x)$, причем \\
 1. Производные всех функций непрерывны на отрезке $[a,b];$\\
 2. $\sum\limits_{n=1}^{\infty} a_n(x)$ сходится на $[a,b]$ поточечно;\\
 3. Ряд из производных сходится равномерно на $[a,b]$ к  $S(x)$;\\
 Тогда 
$$\sum\limits_{n=a}^{\infty} a'_n(x)=
\left( \sum\limits_{n=1}^{\infty} a_n \right)'$$
то есть в ряде  можно менять производную и сумму, причем 
$\sum\limits_{n=1}^{\infty} a_n$ сходится равномерно.
\end{theor}
\textbf{Доказательство.} 
1. Используем предыдущую теорему. Тогда
$$\int\limits_{x_0}^x\left( \sum\limits_{n=1}^{\infty} a'_n(t) \right)dt=
\sum\limits_{n=1}^{\infty} \int\limits_{x_0}^xa'_n(t)dt$$
Получаем, что в равенстве
$\int\limits_{x_0}^xS(t)dt=\sum\limits_{n=1}^{\infty} (a_n(x)-a_n
(x_0))$ справа стоит число (в силу непрерывности функции), ряд из $a_n(x_0)$
сходится по условию, следовательно, ряд из $a_n(x)$ сходится.
Поэтому, дифференцируя равенство 
$\int\limits_{x_0}^{x} \sum\limits_{n=1}^{\infty} a_n(t)\,dt=
\sum\limits_{n=1}^{\infty} a_n(x)-\sum\limits_{n=1}^{\infty} a_n(x_0)$,
получаем первое утверждение теоремы.

Теперь покажем равномерную сходимость исходного ряда. 
Для этого покажем, что остаток 
ряда из производных $r_n(x)=\sum\limits_{k=n+1}^{\infty} a'_n(x)$
равномерно стремится к нулю. 
Из этого следует применимость теоремы об инетгировании: 
$\int\limits_{x_0}^{x}\sum\limits_{k=n+1}^{\infty} a'_k(t)\,dt=
\sum\limits_{k=n+1}^{\infty} \int\limits_{x_0}^{x} a'_k(t)dt=
\sum\limits_{k=n+1}^{\infty} (a_k(x)-a_k(x_0))$. Если ряд удовлетворяет 
теореме об интегрировании, то и его остатки тоже, значит,
$\int\limits_{x_0}^{x} r_n(t)dt=R_n(x)-R_n(x_0)$, откуда
$$R_n(x)=\int\limits_{x_0}^{x} r_n(t)dt+R_n(x_0)\quad(1)$$.
Зафиксируем $\varepsilon>0$. По условию, остаток обычного ряда стремится
к нулю: $R_n(x)\to0$. Тогда для 
$$\frac{\varepsilon}{2}>0~\exists n_1
(\varepsilon)~\forall n>n_1:|R_n(x_0)|<\frac{\varepsilon}{2}$$
Остаток ряда из производных равномерно стремится к нулю, тогда
для 
$$\frac{\varepsilon}{2(b-a)}>0~\exists n_2(\varepsilon)~\forall n>n_2~
\forall x\in[a,b]:|r_n(x)|<\frac{\varepsilon}{2(b-a)}$$
По формуле (1) получаем: 
$|R_n(x)|\leqslant \left| \int\limits_{x_0}^{x} r_n(t)dt \right|+
|R_n(x_0)|\leqslant \left|\left| \int\limits_{x_0}^{x} r_n(t)dt \right|+
|R_n(x_0)| \right|<\frac{\varepsilon}{2(b-a)}\cdot |x-x_0|+
\frac{\varepsilon}{2}=\varepsilon$. $\square$ 



%лекция 13.10.22
%\section{Степенные ряды}
\subsection{Базовые определения}
\begin{defin}
Степенной ряд- ряд вида $\sum\limits_{n=0}^{\infty} C_n(x-x_0)^n$
\end{defin}
Числа $C_n$ - коэффициенты степенного ряда,  $x_0$ - число. Итак, степенной
ряд - обобщение понятия многочлена. Область сходимости степенного ряда 
непуста, так как так лежит как минимум  $x_0$ (в этом случае сумма ряда 
равна $C_0$). Сделав замену $t=x-x_0$, сведем любой степенной ряд к виду
 $\sum\limits_{n=0}^{\infty} C_nt^n$.
\begin{theor}
    (лемма Абеля)\\
    Если ряд $\sum\limits_{n=0}^{\infty} c_nx^n$ сходится в точке $x_0$ и 
     $|x|<|x_0|$, то ряд сходится сходится и в  $x$, причем абсолютно.
\end{theor}
\textbf{Доказательство.}  По условию ряд сходится, значит,
$c_nx^n\to0$. Тогда существует константа $M$, большая чем все члены ряда. 
Тогда $|c_nx^n|=\left| c_nx_0^n \left( \frac{x}{x_0} \right)^n  \right|
\leqslant M\cdot \left| \frac{x}{x_0} \right|^n $. Ряд $\sum\limits_{n=0}^{\infty} Mq^n$ сходится $\Rightarrow$ ряд из модулей сходится, т.е. ряд 
сходится абсолютно.
$\square$ 
\begin{theor}
Пусть $D$ - область сходимости ряда  $\sum\limits_{n=0}^{\infty} c_nx^n$,
$R=\sup\limits_{x\in D} |x|$. Тогда $(-R,R)\subset D\subset [-R,R]$.
\end{theor}
\textbf{Доказательство.} 
По лемме Абеля, второе включение очевидно: $\forall x\in D:|x|\leqslant R
\implies D\subset [-R,R]$.
Пусть $x\in(-R,R)$. Тогда  $|x|<R=R_1$. Тогда 
для него найдется  $x_0\in D:|x_0|>|x|$. Значит, ряд в точке  $x_0$ сходится,
и значит сходится в  $x$. Значит, интервал лежит в области сходимости.
$\square$ 
\subsection{Формулы для вычисления радиуса сходимости}
Пусть $\sum\limits_{n=0}^{\infty} c_nx^n=\sum\limits_{n=0}^{\infty} a_n$.
По признаку Даламбера 
$\lim\limits_{n \to \infty} \frac{|a_{n+1}(x)|}{|a_n(x)|}=|x|\cdot
\lim\limits_{n \to \infty} \frac{|c_{n+1}|}{|c_n|}<1$, то ряд сходится.
Итак, если предел существует, то 
$$\boxed{R=\lim\limits_{n \to \infty} \frac{|c_n|}{|c_{n+1}|}}$$
Аналогично, из признака Коши получим формулу Коши-Адамара:
$$\boxed{R=\frac{1}{\overline{\lim\limits_{n \to \infty}}\sqrt[n]{|c_n|}}}$$ 
В общем случае алгоритм такой:\\
1. Найти радиус сходимости.\\
2. Выписываем интервал сходимости $(x_0-R,x_0+R)$.\\
3. Исследуем на сходимость концы интервала.\\
\textbf{Пример.} Найдем область сходимости $\sum\limits_{n=0}^{\infty} 
\frac{(x-6)^n}{(n+2)3^n}$. Применим признак Даламбера:
$R=\lim\limits_{n \to \infty} \frac{(n+3)3^{n+1}}{(n+2)3^n}=3$.
Интервал сходимости: $(6-3,6+3)$. В точке $x=9$ ряд расходится (т.к.
гармонический), в точке  $x=3$ - условная сходимость (по признаку Лейбница).\\
\textbf{Пример.} Найдем область сходимости $\sum\limits_{n=0}^{\infty} 
\frac{n^2}{(n+1)^2}\cdot \frac{x^{2n}}{2^n}$. Заметим, что у этого ряда 
коэффициенты чередуются с нулем (лакунарный ряд). Используем два способа:\\
1. По формуле Коши-Адамара - возьмем четные номера, так как на них
доставляется супремум предела последовательности:
$R=\frac{1}{\lim\limits_{n \to \infty} \left( \frac{n}{n+1} \right)^
{\frac{1}{n}}\cdot \left( \frac{1}{2^{\frac{1}{2}}} \right) }=\sqrt{2}$.
Интервал сходимости $(-\sqrt{2},\sqrt{2})$, на концах расходится.\\
2. Исследуем как функциональный ряд по признаку Даламбера.
$\lim\limits_{n \to \infty} \frac{|a_{n+1}|}{|a_n|}=\frac{x^2}{2}
\lim\limits_{n \to \infty} \left( \frac{n^2+2n+1}{n^2+2n} \right)^2=
\frac{x^2}{2}$. Значит, ряд сходится, если $\frac{x^2}{2}<1$, откуда мы 
получаем тот же интервал сходимости.
\begin{theor}
    (о равномерной сходимости степенного ряда)\\
    Степенной ряд сходится равномерно на любом отрезке, лежащем внутри 
    интрвала сходимости.
\end{theor}
\textbf{Доказательство.} Для простоты рассмотрим ряд с центром в нуле. 
Пусть ряд сходится на $(-R,R)$. Возьмем  $[a,b]\subset (-R,R)$. Обозначим
$d=max(|a|,|b|)$. Тогда ряд  $\sum\limits_{n=0}^{\infty} c_nd^n$ сходится,
значит, его мы можем использовать для оценки сверху рядов на отрезке:
$|c_nx^n|\leqslant |c_nd^n|$, значит, по признаку Вейерштрасса ряд сходится
на $[a,b]$.
$\square$ 
\begin{theor}
    (о непреывной сумме степенного ряда)\\
    Сумма степенного ряда непрерывна в любой точке из интервала сходимости.
\end{theor}
\textbf{Доказательство.}  Пусть $\sum\limits_{n=0}^{\infty}c_nx^n$ 
сходится на $(-R,R)$ к  $f(x)$. Степенные функции непрерывны на интервале
(и вообще на всей прямой); по предыдущей теореме, на любом отрезке,
лежащем в интервале, ряд равномерно сходится. Значит, по теореме о 
непрерывности суммы равномерно сходящегося ряда, сумма непрерывна на 
отрезке. Так как этот отрезок произволен, то сумма непрерывна на интервале.
$\square$ 
\begin{theor}
    (об интегрировании и дифференцировании степенного ряда)\\
Пусть дан ряд $\sum\limits_{n=0}^{\infty} c_n(x-x_0)^n=f(x)$, $R$ - радиус 
сходимости. Тогда у функции  $f(x)$ существуют производные любого порядка
внутри интервала:
$$f'=\sum\limits_{n=0}^{\infty} nc_n(x-x_0)^{n-1}$$ 
Интегрирование тоже почленное. 
Причем при дифференцировании и интегрировании радиус сходимости не меняется.
\end{theor}
\textbf{Доказательство.}  Следует из соотвествующих теорем для функциональных
рядов. Последнее утверждение следует из формулы Коши-Адамара. 
$\square$\\ 
\textbf{Пример.} Вычислить сумму ряда $\sum\limits_{n=1}^{\infty} 
\frac{1}{n\cdot 2^n}$. Задания типа таких можно делать, используя 
свойства степенных рядов. Пусть $f(x)=\sum\limits_{n=1}^{\infty}\frac{x^n}{n}$.
Радиус сходимости $x\in[-1,1)$. Возьмем производную:
$f'(x)=\sum\limits_{n=1}^{\infty} x^{n-1}=\frac{1}{1-x}$. А вот теперь
проинтегрируем: $\int\limits^x_0\frac{dt}{1-t}=f(x)-f(0)$;
$f(x)=-\ln(1-x)+f(0)$. Значит, сумма искомого ряда равна $f(\frac{1}{2})=2$.
Цель этих телодвижений - привести к виду геометричсекой прогрессии, которую
легко посчитать. 









%лекция 17.10.22
%\section{Уравнение первого порядка}
\begin{defin}
Уравнение 
\begin{equation}\label{lin_de}
    \frac{dx}{dt}+a(t)x=b(t)
\end{equation}
где $a,b$ непрерывны на  $t\in (\alpha,\beta)$ (интервал непрерывности),
называется линейным ДУ первого порядка. Если при этом $b(t)\not\equiv 0$, то 
оно называется неоднородным.
\end{defin}
Как следствие из теоремы Коши-Пикара, для $\forall t_0\in (\alpha,\beta),~
\forall x_0\in \mathbb{R}$ существует и единственно решение задачи Коши.

\textbf{Замечание.} Решение задачи Коши для \ref{lin_de} можно продолжить
на весь интервал $(\alpha,\beta)$. Если этот интервал конечен, то функции
$a(t),b(t)$ ограниченны на нём, то есть  $|a(t)x+b(t)|\leqslant Ax+B$, 
и решениене выйдет за конус, образованный этой прямой. 

\begin{defin}
Линейныйй ператор - отображение $A\colon X\to Y$ %ЧЕЕЕЕ БЛЯЯЯЯЯЯЯЯЯТЬ ТАКОЕ
такое, что $A(x+y)=A(x)+A(y),~A(\lambda x)=\lambda A(x)$.
\end{defin}
Пусть $X=C^1(\alpha,\beta),~C^0(\alpha,\beta)$ - пространства дифференцируемых
и непрерывных функций. Положим $A(x)=\frac{dx}{dt}+a(t)x$. В силу линейности
производной, это - линейный оператор. Также и любая линейная комбинация 
производных (любого порядка) является линейным оператором. 

Итак, уравнение \ref{lin_de} в операторной записи эквивалентно 
$Ax=b(t)$. Обозначим за $x_{o.n.}$ множество решений неоднородного уравнения,
$x_{o.o.}$ - множество решений однородного уравнения,  $x_{o.n}+x_{o.o}$ -
множество вида $x+x$

\begin{theor}
    (о структуре решения линейного уравнения)\\
    Решение неоднородного уравнения - сумма общего решения однородного 
    уравнения и частного решения.
\end{theor}
\textbf{Доказательство.} Пусть $\varphi(t)$ - частное решение однородного
уравнения, $x_{p}$ - частное решение неоднородного уравнения. Применим
оператор  $A$ к их сумме:  $A(\varphi(t)+x_p)=A\varphi(t)+Ax_p=0+b(t)$. 
Значит, сумма этих функций обращает уравнение в тождество, значит,
$\varphi(t)+x_p\in x_{o.n.}$.

Докажем, что других решений нет. Допустим, $\psi(t)\in x_{o.n.}$ таков, что
его нельзя представить суммы решений однородного и неоднородного. Рассмотрим
$\psi-x_p$ - вычтем частное решение неоднородного. Подставляя в уравнение, 
получаем  $A(\psi-x_p)\equiv 0$, значит, их разность - решение однородного
уравнения. Но это противоречит предположению. $\square$

Как решать линейные уравнения? Сначале решаем однородное уравнение:
$\frac{dx}{dt}=-a(t)x$, $x=C(t)e^{-\int_{t_0}^{t} a(\tau)d\tau}$. 
Решать неоднородное 3мя способами:
1. Угадайка\\
2. Метод Лагранжа вариации постоянных\\
3. Формула Коши (см. справочник).
\subsection{Метод Лагранжа}
Мы знаем, что $x=Ce^{-\int a(t)dt}$ - решение однородного уравнения. 
Будем её варьировать, чтобы в уравнении было бы тождество:
$$\frac{d}{dt}\left( C(t)e^{-\int\limits_{t_0}^{t}a(\tau)d\tau} \right)+
a(t)C(t)e^{-\int\limits_{t_0}^{t}a(\tau)d\tau}=b(t)$$
Дифференцируя, получаем $C'=b(t)e^{-\int\limits_{t_0}^{t}a(\tau)d\tau}$, 
откуда  $$C=e^{-\int\limits_{t_0}^{t}a(\tau)d\tau}\int\limits_{t_0}^{t}\left( 
b(s)e^{-\int\limits_{s_0}^{s}a(\tau)d\tau} \right)ds+
C_0e^{-\int\limits_{t_0}^{t}a(\tau)d\tau}$$ 
Значит, мы нашли семейство всех решений неоднородного уравнения, произвольно
выбирая $C_0$. По предыдущей теореме, этим все решения исчерпываются. 

То, что мы получили - это и есть формула Коши. Она нужна в основном для
всяких теоретических свойств.

\textbf{Пример.} $\frac{dx}{dt}+\frac{x}{t}=t^2$.
Интервал непрерывности - $\mathbb{R}\setminus \{0\}$, поэтому вообще-то
надо рассматривать два интервала. Решение однородного уравнения:
$\frac{dx}{dt}=-\frac{x}{t}$, $x=\frac{C}{t}$. Подумаем, как можно подобрать
частное неоднородного уравнения. Поищем в виде $x=at^3$. Тогда при подстановке
$3at^2+t^2=t^2$, откуда $a=\frac{1}{4}$. Ответ: $x=\frac{t^3}{4}+\frac{C}{t}$.

\subsection{Уравнения, приводящееся к линейному}
Испортрим уравнение \ref{lin_de}, добавив нелинейности:
$$\frac{dx}{dt}+a(t)x=b(t)x^k,~k\in \mathbb{R}\setminus\{0,1\}$$ 
Это - уравнение Бернулли. Если разделим на $x^k$, получим
$$x^{-k} \frac{dx}{dt}+a(t)x^{1-k}=b(t)$$ 
Значит, оно сводится к линейному уравнению заменой $z=x^{1-k}$:
 $$\frac{1}{1-k} \frac{dz}{dx}+a(t)z=b(t)$$ 
Рассмотрим уравнение Риккати:
$$\frac{dx}{dt}+a(t)x=b(t)x^2+c(t),~c(t)\ne 0,c(t)\in C^0(\alpha,\beta)$$ 
В общем виде не решается, но можно частное решение угадать. 
Пусть $x=z+x_p$, где  $x_p$ - частное решение. Получим
$$\frac{dz}{dt}+a(t)z+\frac{dx_p}{dt}+a(t)x_p=b(t)x^2_t+2zx_pb(t)+c(t)$$
Свели к уравнению Бернулли
$$\frac{dz}{dt}+[a(t)-2x_pb(t)]z=b(t)z^2$$ 
Ну зато можно численно и приближенно решать. 

\textbf{Пример (№136).} $xy'-2y=2x^4,~x\ne0$. Разделим на  $x$, свели к
линейному (делить на $x$ можно, ибо  $x$ не является решением): 
$$\frac{dy}{dx}-\frac{y}{x}=2x^3$$
Общее решение неоднородного уравнения:
$$\int\limits_{}^{}\frac{dy}{2y}=\int\limits_{}^{}\frac{dx}{x}$$ 
откуда $y=Сx^2$. Подберем частное решение:  $y=ax^4$. Подставляя в уравнение,
получим  $a=1$, откуда общее решение  $y=x^4+Cx^2$. 

Теперь решим методом Лагранжа. Пусть $y=c(x)x^2$. Имеем
 $c'x^2+2xc-2cx=2x^3$, откуда $c(x)=x^2+C_0$. Значит, ответ  $y=x^4+C_0x^2$.

\textbf{Пример (№149).} $y'=\frac{y}{3x-y^2}$. Приведем к линейному 
(перевернем): $\frac{dx}{dy}=\frac{3x-y^2}{y}$. Общее решение 
однородного уравнения: $x=Cy^3$. Частное решение поищем в виде $x=ay^2$.
Отсюда $a=1$, общее решение  $x=Cy^3+y^2$.

\textbf{Пример (№158).} $2y'-\frac{x}{y}=\frac{xy}{x^2-1}$. Домножим на 
$y$:  $2y'y-x=\frac{xy^2}{x^2-1}$. Замена: $z=y^2$. Тогда уравнение
линеаризуется: 
$$\frac{dz}{dx}-\frac{xz}{x^2-1}=x$$
Общее решение однородного уравнения $z= C\sqrt{x^2+1}$. Метод 
внимательного взгляда: $z=x^2-1$ - частное решение. Итак, ответ:
$z=x^2-1+C\sqrt{x^2+1}$, $y=\sqrt{x^2-1+C\sqrt{x^2+1}}$.

\textbf{Пример (№164).} $(x^2-1)y'\sin y + 2x\cos y=2x-2x^3$. Наша 
нейросетка заметила, что здесь есть 
тригонометрическая замена. Именно, пусть $z=\cos x$. Тогда
$(x^2-1)(-z')+2xz=2x-2x^3$. Делим на 
$x^2-1$ получим однородное.

\textbf{Пример (№163).} $x(e^y-y')=2$. Введем замену $t=e^y$, получаем
 $1-\frac{dt}{dx}\cdot \frac{1}{t^2}=\frac{2}{xt}$. Далее $z=\frac{1}{t}$, 
 и наконец получаем линейное уравнение:
 $$1+\frac{dz}{dx}=\frac{2z}{x}$$

\textbf{Пример (№167).} Уравнение Риккати: $x^2y'+xy+x^2y^2=4$.
Частное решение $y=\frac{a}{x}$. Тогда
$-a+a+{a^2}=4$, $a=\pm2$. Пусть $y=\frac{2}{x}$. Общее решение тогда
$y=z+\frac{2}{x},~y'=z'-\frac{2}{x^2}$. Имеем уравнение Бернулли
$$-z^2=\frac{5z}{x}+z'$$
Сделаем замену $u=\frac{1}{z}$, получим $\int\limits_{}^{}\frac{du}{u} $


%дз 



%20.10.22
%% Коллоквиум закончился
\section{Предел последовательности в топологическом пространстве}
\begin{defin}
    Последовательность $\{x)n\}\subset X$ сходится к $a\in X$
$$\forall U(a)~\exists n_0\in \mathbb{N}~\forall n>n_0:x_n\in U(a)$$
\end{defin}
Также обозначается $a=\lim\limits_{n\to\infty}x_n$. 
%Звонит Полотовский. НИ ставит его на громкую звязь. 
%-Григорий Михайлович, у меня лекция!
%-Телефон Николь можете потом дать?
%-Да
\begin{theor}
Если предел последовательности существует в хаусдорфовом пространстве, 
то он единственный. 
\end{theor}
\textbf{Доказательство.}  Пусть существует два предела. По хаусдорфовости, 
они обладают непересекающимися окрестностями. Тогда и там и там лежат все 
номера последовательности, что невозможно. 
$\square$ 
\textbf{Пример.} Исследовать на сходимость в 
$(\mathbb{R},\tau_\text{ирр})$ последовательность $x_n=\frac{2}{n}$. 
Покажем, что каждая рациональная точка является пределом последовательности. 
Рассмтрим $U_a,~a\in \mathbb{Q}$. Тогда $U_a=\mathbb{R}$. Очевидно,
$\forall n\in \mathbb{N}:x_n\in U_a$. Докажем, что других нет. Очевидно,
наименьшая окрестность иррациональной точки - сама точка, в которой нет 
никаких членов последовательности. Формально, запишем отрицание:
$$\lim\limits_{n \to \infty} x_n\ne b\iff \exists U_b~\forall N~\exists n>N:
x_n\notin U_b$$ 
\section{Аксиомы счетности}
\begin{defin}
База окрестностей - семейство окрестностей $\{U_\alpha=U_\alpha(x)\}$, 
такое, что для каждой открытой окрестности $U_x$ точк $x$ имеет место
$x\in U_\alpha\subset U_x$
\end{defin}
\dnote{
Окрестности, из которых состоит база окрестностей, понимаются в обычном
смысле окрестностей, то есть как множество, содержащее точку, не 
обязательно открытое.
}
\begin{defin}
\textbf{Аксиома счетности I}. В каждой точке $x$ существует счетная база
окрестностей в точке  $x$.
\end{defin}
\begin{defin}
\textbf{Аксиома счетности II}. База топологии счетна.
\end{defin}
\begin{theor}
    (о связи между аксиомами счетности)\\
    Вторая аксиома счетности влечет первую (обратное неверно).
\end{theor}
\textbf{Доказательство.} Пусть $\Sigma=\{W_i\}$ - счетная база топологии.
Положим для точки $x$  $\Sigma_x=\{W_i\in\Sigma\mid x\in W_i\}\subset\Sigma$.
Это счетное множество. Теперь покажем, что это счетная 
база окрестностей в точке
$x$. По определению базы, для любой окрестности: 
$x\in U_x=\bigcup\limits_{i\in \mathbb{N}_0\subset \mathbb{N}}W_i$

\textbf{Пример.} В дискретной топологии вторая аксиома счетности
не выполняется, коль скоро пространство несчетно, так как минимальная база 
из одноточечных подмножеств несчетна, при этом удовлетворяет первой аксиоме.

\textbf{Пример.} $(\mathbb{R}^2,\tau_{MN})$ удовлетворяет первой аксиоме, но 
не удовлетворяет второй. 
\begin{defin}
Подмножество А всюду плотно в Х, если $\overline{A}=X$.
\end{defin}
\begin{theor}
    (лемма о всюду плотном множестве)\\
    А всюду плотно в Х тогда и только тогда, когда для любого открытого U:
    $U\cap A\ne\varnothing$.
\end{theor}
\textbf{Доказательство.} Пусть $A$ всюду плотно. Для любой непустой

для любого непустого открытого множества 
$\exists x\in U$ 

Обратно, пусть

$\square$ \\

\begin{defin}
Топологическое пространство называется сепарабельным, если в нем существует
счетное всюду плотное подмножество.
\end{defin}
Так, $\mathbb{R}^n$ с обычной топологией 
сепарабельно, так как $\mathbb{Q}^n$ - всюду плотно и 
его замыкание равно $\mathbb{R}^n$. Упражнение:
$\overline{\mathbb{Q}\times...\times\mathbb{Q}}=\overline{\mathbb{Q}}\times...
\times\overline{\mathbb{Q}}$




%31.10.22
\textbf{Задача.} Что нам мешает провести через одну точку несколько
решений уравнения $y'=x-y^2$? Тот факт, что тангенс угла наклона задается
уравнением однозначно, поэтому трансверсальное пересечение невозможно.
А если касательные параллельны? Если такая ситуация имеет место, тогда по 
теореме о существовании и единственности в этой точке правая часть либо
её производная не непрерывны, но это не так. А что, если $y''=x-y^2$ - 
уравнение второго порядка? Тогда все-таки ничего неельзя сказать (там есть
свои теоремы). 
\subsection{УРавнение, не разрешенное относительно производной}
Общий вид - 
$$F\left( t,x,\frac{dx}{dt} \right) = 0$$ 
\textbf{Пример.} $x'^2-x^2=0$. Два семейства решений:
$x=x_0e^{\pm t}$. Как видно, в каждой точке пересекаются 2 решения. 
Для таких уравнений  ситуация с пересечением решений 
типична, но их количество и взаимный наклон определены в зависимости от вида
уравнения.
\begin{defin}
Особая точка - точка,через которую проходит несколько решений.
\end{defin}
\begin{theor}
Пусть $F$ непрерывна по всем аргументам, имеющая непрерывные частные 
производные по  $x,t$ и  $\frac{\partial F}{\partial x}\ne 0$. Тогда
существует одна или несколько функций $f(t,x)$ такие, что 
$F(t,x,f(t,x))\equiv0$, и решение задачи Коши $x(t_0)=x_0,~x'(t_0)=x'_0$
существует и единственно. 
\end{theor}
\textbf{Доказательство.}  
Допустим, решение существует.
Рассмотрим полную производную по времени: 
$\frac{dF(t,x,x')}{dt}=\frac{\partial F}{\partial t}+
 \frac{\partial F}{\partial x}\frac{\partial x}{\partial t} +
 \frac{\partial F}{\partial x'}\frac{\partial x'}{\partial t}\equiv 0$
Тогда $\frac{dx}{dt}=\frac{-\frac{\partial F}{\partial t} -
\frac{\partial V}{\partial x} }{}$ списываем из фихтенгольца
Эти условия должны выполняться в окрестности какой-то точки
$F(t_0,x_0,x'_0)=0$ $\square$ \\
%\begin{defin}
%Обыкновенная точка уравнения $F(t,x,x')$ - точка, через которую ожидаемое 
%количество решений. 
%\end{defin}
\begin{theor}
Пусть $F$ непрерывна по всем аргументам в области $D$,
имеющая непрерывные частные 
производные по  $x,t$ и  $\frac{\partial F}{\partial x}\ne 0$. Тогда
для любой точки $t_0,x_0,x'_0$ существует и единственно решение задачи Коши. 
\end{theor}
\textbf{Доказательство.}  
$\square$ \\
\begin{defin}
Регулярная (обыкновенная) точка уравнения $F(t,x,x')$ - точка $(t,x)$, 
в которой задача Коши (для уравнения, разрешенного или не разрешенного 
относительно производной) имеет единственное решение. Если решений несколько
или ноль, то точка особая (сингулярная).
\end{defin}

\textbf{Пример (№249).} $(y')^3+y^2=y\cdot y'(y'+1)$. Группируем и выносим
общий множитель два раза, получаем $((y')^2-y)(y'-y)=0$. 
Получаем три уравнения: 
$\begin{cases}y'=y;\\y'=\sqrt{y};\\y'=-\sqrt{y}\end{cases}$. Решения:
$\begin{cases}y=y_0e^t;\\y=\left( -\frac{x}{2}+\frac{c}{2}\right)^2 \\
y=\begin{cases}\left( \frac{x+x_0}{2} \right)^2,~x<x_0\\0,~x\geqslant x_0
    
\end{cases}  
\end{cases}$
Найдем особые точки: $y=0$, ибо там бесконечно много решений.
Вообще говоря, 
 $y=\begin{cases}0,~x\leqslant x_0\\\left( \frac{x-x_0}{2} \right)^2,x>x_0
  
 \end{cases}$
Проверим условия теоремы для оставшихся точек (тем доказав, что других особых 
точек нет). Гладкость функции очевидна, нули производной:
$3(y')^2-y\cdot 2y'-y=0$

\begin{defin}
Особое решение - решение состоящее из особых точек
\end{defin}
%Списать с фоток Миши
%03.11.22
\subsection{Критерии сходимости несобственного интеграла}
\begin{theor} (критерий Коши)
    Пусть $\forall b\geqslant a$ функция интегрируема на $[a,b]$. 
    Тогда $\int_a^\infty f(x)dx$ сходится $\Leftrightarrow$ $\forall 
    \varepsilon>0~\exists b_0(\varepsilon)>0~\forall b_1,b_2>b_0:
    \left| \int_{b_1}^{b_2}f(x)dx \right|<\varepsilon$
\end{theor}
\textbf{Доказательство.} По условию, существует предел 
$\lim\limits_{b \to +\infty} F(b)=A\in \mathbb{R}$, где
$F(b)=\int^b_af(x)dx$. 
Зафиксируем $\varepsilon>0$. Тогда из существования предела следует
для $\frac{\varepsilon}{2}$: $\exists b_o(\varepsilon)>a:\left| 
F(b)-A\right|<\frac{\varepsilon}{2}$. Пусть $b_1>b_0,~b_2>b_0$. Тогда
$|F(b_2)-F(b_1)|=|F(b_2)-A|+|F(b_1)-A|<\frac{\varepsilon}{2}+
\frac{\varepsilon}{2}=\varepsilon$.\\
Достаточность. Докажем существование предела  $\lim\limits_{b\to\infty}F(b)$
из определения предела по Гейне. Пусть $b_n\to \infty$, тогда $\forall b_0>a
~\exists n_0(\varepsilon)\in \mathbb{N}~\forall n>n_0$ 
Покажем, что предел не зависит от выбора последовательности $b_n$. 
Выберем другую последовательность  $b^*_n$. Обозначим предел 
$\lim\limits_{n \to \infty} F(b^*_n)=B$. Составим последовательность
$b_1,b^*_1,b_2,b^*_2,...\to \infty$. Тогда предел $F$ от этой
последовательности обозначим как  $C$. Так как пределы подпоследовательностей 
сходятся к пределу последовательности, то  $A=B=C$. Значит, выполняется
условие определения предела по Гейне, значит, интеграл сходится. $\square$\\
\textbf{Пример.} $\int_1^\infty \frac{\sin x}{x^\alpha}dx$ сходится при 
$\alpha>0$, расходится при $\alpha\leqslant 0$. Докажем это.\\
1. $\alpha>0$. Поехали: $\forall \varepsilon>o~\exists b_0(\varepsilon)>1~
\forall b_1>b_0,b_2>b_0: \left| \int^{b_2}_{b_1} \frac{\sin x}{x^\alpha}dx
\right|<\varepsilon$. Доказываем: 
$\left| \int^{b_2}_{b_1} \frac{\sin x}{x^\alpha}dx
\right|=\left| \int^{b_2}_{b_1} \frac{1}{x^\alpha}d\cos x\right|=
\left| \frac{\cos x}{x^\alpha} \right|^{b_2}_{b_1}-\int^{b_2}_{b_1} 
\cos x d(\frac{1}{x^\alpha})\leqslant ... \leqslant\frac{4}{b^\alpha_0}$.
Значит, $b_0>(\frac{4}{\varepsilon})^\frac{1}{\alpha}$.\\
2. $\alpha\leqslant 0$. Синус теперь принимает разные знаки. Пусть 
$b_k=2\pi k$. Тогда по критерию Коши интеграл расходится.
\begin{theor} (критерий сходимости через остаток)\\
Пусть $\int^\infty_a=\int^b_a+\int^\infty_b,~(b>0)$.\\
1. Если интеграл сходится, то и любой из его остатков сходится.\\
2. Если хотя бы один из остатков сходится, то интеграл сходится.
\end{theor}
\textbf{Доказательство.}  
$\square$ 
\begin{theor} (критерий сходимости несобственного интеграла от несобственной
функции)\\
Пусть $\forall b>a$ функция интегрируема на $[a,b]$ и неотрицательная .Тогда
$\int^\infty_af(x)dx$ сходится $\Leftrightarrow$ первообразная $F(b)<M$ 
ограниченна.
\end{theor}
\textbf{Доказательство.} $F(b)$ неубывает и имеет конечный предел. Значит,
интеграл сходится. Обратно, пусть существует конечный предел 
$\lim\limits_{b \to \infty} F(b)$, то $F(b)$ ограниченна в некоторой 
окрестности. $\square$\\
\subsection{Признаки сравнения в предельной форме}
\begin{theor} (признак сравнения)\\
Пусть $f(x)>g(x)>0$ начиная с некоторого $x>a$, и для любого  $b>a$
функции интегрируемы на $[a,b]$. Тогда\\
1. Если  $\int f(x)$ сходится, то и  $\int g(x)$  сходится.\\
2. Если  $\int g(x)$ расходится, то и $\int f(x)$ расходится.
\end{theor}
\textbf{Доказательство.}  По свойству определенного интеграла (транзитивность 
числовых неравенств), $F(b)\leqslant M$. Тогда по критерию 3 интеграл 
сходится.
2. Погодите, это реально?
$\square$ 
\begin{theor} (второй признак сравнения)\\
Если $\frac{f(x)}{g(x)}=k,~\infty\ne k\ne0$, то их интегралы сходятся или
расходятся одновременно. 
\end{theor}
\textbf{Доказательство.}  \
$\square$ 





















%07.11.22
\subsection{Абсолютная и условная сходимость}
\begin{defin}
Интеграл сходится абсолютно, если сходится интеграл от модуля.
\end{defin}
Очевидно, из абсолютной сходимости следует обычная. 
\begin{theor}
    (признак Дирихле)\\
    Пусть:\\
    1. $f\in C[a,\infty)$ 
    (и существует интеграл $F(b)=\int\limits_{a}^{b}f(x)dx$);\\
    2. $\exists M=const~\forall b\geqslant a:|f(b)|\leqslant M$;\\
    3. $g'(x)\in C[a,\infty)$;\\
    4. $g'(x)$ знакопостоянна;\\
    5. $\lim\limits_{x \to \infty}g(x)=0$;\\
    Тогда $\int\limits_{a}^{\infty}f(x)g(x)dx$ сходится условно. 
\end{theor}
\textbf{Доказательство.} Зафиксируем $\varepsilon>0$. Тогда из условия 5 для
$$\frac{\varepsilon}{4M}~\exists b_0>a~\forall x>b_0:|g(x)|<
\frac{\varepsilon}{4M}$$
Возьмем произвольные $b_1,b_2>b_0$, тогда 
$$\int\limits_{b_1}^{b_2}f(x)g(x)dx>\int\limits_{b_1}^{b_2}g(x)df(x)=
g(x)f(x)\Big|^{b_2}_{b_1}-\int\limits_{b_1}^{b_2}F(x)g(x)dx=???$$
Итак, 
$$ \left| \int\limits_{b_1}^{b_2} f(x)g(x)dx \right|\leqslant 
|g(b_2)|\cdot |F(b_2)|+|g(b_1)|\cdot |F(b_1)|+|F(\xi)|\cdot |g(b_2)|+
|f(\xi)|\cdot |g(b_1)|\leqslant\frac{4M\cdot \varepsilon}{4M}$$ 
$\square$ 

\begin{theor}
    (признак Абеля)\\
    Пусть:\\
    1. $f\in C[a,\infty)$ 
    (и существует интеграл $F(b)=\int\limits_{a}^{b}f(x)dx$);\\
    %2. $\exists M=const~\forall b\geqslant a:|f(b)|\leqslant M$;\\
    2. $g'(x)\in C[a,\infty)$;\\
    3. $g'(x)$ знакопостоянна;\\
    %5. $\lim\limits_{x \to \infty}g(x)=0$;\\
    4. $\exists M=const~\forall x\geqslant a:|g(x)|\leqslant M$
    Тогда $\int\limits_{a}^{\infty}f(x)g(x)dx$ сходится условно. 
\end{theor}
\textbf{Доказательство.} Зафиксируем $\varepsilon>0$. Тогда из условия 2 по 
критерию Коши для
$$\frac{\varepsilon}{2M}~\exists b_0>a~\forall b_1,b_2>b_0:\left| 
\int\limits_{b_1}^{b_2}f(x)dx\right| \leqslant \frac{\varepsilon}{2M}$$
????????????????

По критеорию Коши,
$$\left|\int\limits_{b_1}^{b_2}f(x)g(x)dx \right|\leqslant g(b_2)\cdot \
\left| \int\limits_{b_1}^{b_2}f(x)dx\right|+|g(b_1)|\cdot 
\left| \int\limits_{b_1}^{b_2} f(x)dx\right|\leqslant 
$$
$$\leqslant M\cdot \frac{\varepsilon}{2M}\cdot 2=\varepsilon\quad\square $$




%10.11.22
\begin{theor}
    (о непрерывности интеграла)\\
    Если функция определена и непрерывна, 
\end{theor}
\textbf{Доказательство.}  
$\square$ \\

\begin{theor}
    (о дифференцируемости собственного интеграла, зависящего от параметра/
    правило Лейбница)\\
    Пусть $f(x,y)$ \\
    1. непрерывна на  $P=[a,b]\times[c,d]$;\\
    2. $\frac{\partial f}{\partial y}(x,y)$ непрерывна на $P$;\\
    Тогда:\\
    1. $F(y)=\int\limits_{a}^{b} f(x,y)dx$ дифференцируема на $[c,d]$;
    2. $F'(y)=\int\limits_{a}^{b}\frac{\partial f}{\partial y}(x,y)dx$
\end{theor}
\textbf{Доказательство.} Пусть $y\in[c,d],~y+h\in[c,d]$. 
Рассмотрим  $F(y+h)-F(y)=\int\limits_{a}^{b} (f(x,y+h)-f(x,y))dx$, 
значит, по теореме Лагранжа это равно $\int\limits_{a}^{b}
\frac{\partial f}{\partial y}(x,y+\theta h)h\,dx$, где $\theta\in(0,1)$. 
Дифференцируем:
$F'(y)=\lim\limits_{h \to 0}\frac{F(y+h)-F(y)}{h}=\lim\limits_{h \to 0}
\int\limits_{a}^{b} \frac{\partial f}{\partial y}(x,y+\theta h)dx$.
При $h\to 0$ делаем замену  $u=y+\theta h,~u\to y$. Тогда предел
 $\lim\limits_{h \to 0}
\int\limits_{a}^{b} \frac{\partial f}{\partial y}(x,u)dx=$
по теореме о предельном переходе!!!!!!!!!!
$\square$ \\
Следующая теорема обощает правило Лейбница:
\begin{theor} (обобщенное правило Лейбница)\\
    Пусть $f(x,y)$ непрерывна на  $D=\{(x,y)\mid a(y)\leqslant x\leqslant 
    b(y),c\leqslant y\leqslant d\}$, $\frac{\partial f}{\partial x}(x,y)$ 
    непрерывна на $D$ и $a'(y),b'(y)$ непрерывны на  $[c,d]$. Тогда
     $F(y)=\int\limits_{a(y)}^{b(y)}f(x,y)dx$ дифференцируема на 
     $y\in[c,d]$, причем  $F'(y)=\int\limits_{a(y)}^{b(y)} 
     \frac{\partial f}{\partial y} (x,y)dx+f(b(y),y)\cdot b'(y)-
     f(a(y),y)\cdot a'(y)$.
\end{theor}
\textbf{Доказательство.}  $F(y)=F(y,a(y),b(y))$. По правилу производной
сложной функции  $\frac{dF}{dy}=\frac{\partial F}{\partial y}+
\frac{\partial F}{\partial a}\cdot \frac{\partial a}{\partial y} 
\frac{\partial F}{\partial b}\cdot \frac{\partial b}{\partial y}=
\int\limits_{a(y)}^{b(y)}\frac{\partial F}{\partial y}(x,y)dx
+f(b(y),y)\cdot b'(y)-f(a(y),y)\cdot a'(y)$ $\square$

Заметим, что по правилу Лейбница, $\frac{\partial F}{\partial y}=
\int\limits_{a(y)}^{b(y)} \frac{\partial f}{\partial y}(x,y)dx$,
$\frac{\partial F}{\partial b} f(b(y),y)$,
$\frac{\partial F}{\partial a}=
\left(-\int\limits_{b(y)}^{a(y)}f(x,y)dx\right)'_a=-f(a(y),y)$
%Дальше здеь была куча поясняющего текста (см фото 10.11.22 в 13620)

\textbf{Пример.} Посчитаем $F(a)=\int\limits_{0}^{\frac{\pi}{2}}
\frac{\ln(1+a^2\sin^2(x))}{\sin(x)}$. тут я отрубился

\begin{theor}
    (об интегрировании интеграла, зависящего от параметра)\\
    Пусть $f(x,y)$ непрерывна на  $P=[a,b]\times[c,d]$. 
    Тогда
     $$\int\limits_{c}^{d}dy\int\limits_{a}^{b} f(x,y)dx=
     \int\limits_{a}^{b}dx \int\limits_{c}^{d} f(x,y)dy$$
\end{theor}
\textbf{Доказательство.}  Введем функции $G(t)=\int\limits_{c}^{d}dy
\int\limits_{a}^{t}f(x,y)dy,~H(t)=\int\limits_{a}^{t}dx
\int\limits_{c}^{d}f(x,y)dy$. Докажем, что $G(b)=H(b)$ (что доказывает 
требуемое утверждение). Введем функцию  $g(t,y)=\int\limits_{a}^{t}f(x,y)dx$,
тогда $G(t)=\int\limits_{c}^{d}g(t,y)$ - применима теорема о дифференцировании
сложной функции: $\frac{\partial g}{\partial t}=\left( 
\int\limits_{a}^{t} f(x,y)dx\right)'_t=f(t,y)$ - непрерывна на $P$ по условию. 
Теперь докажем, что $g(t,y)$ непрерывна на $P$, для этого покажем, что 
 $\lim\limits_{\Delta t \to 0,\Delta y\to 0}\Delta g=0$. 
Имеем $\Delta g=g(t+\Delta t,y+\Delta y)-g(t,y)=\int\limits_{a}^{t+\Delta t}
f(x,y+\Delta y)dx-\int\limits_{a}^{t}f(x,y)dx=\int\limits_{a}^{t}(
f(x,y+\Delta y)-f(x,y))dx+\int\limits_{t}^{t+\Delta t}f(x,y+\Delta y)dx$. 
Так как $f(x,y)$ непрерывна на компакте $P$,  то она равномерно непрерывна
на $P$ и ограниченна константой  $M$. Зафиксируем  $\varepsilon>0$. 
Из равномерной непрерывности для 
$$\frac{\varepsilon}{2(b-a)}>0~\exists \delta_1>0~
\forall (x_1,y_1)\in P~\forall (x_2,y_2)\in P:\sqrt{(x_1-x_2)^2-(y_1-y_2)^2}<
\delta_1$$ 
$$\implies|f(x_1,y_1)-f(x_2,y_2)|< \frac{\varepsilon}{2(b-a)}$$
Если $|\Delta y|<\delta$, то $|f(x,y+\Delta y)-f(x,y)|<
\frac{\varepsilon}{2(b-a)}$; тогда можно оценить интеграл:
$$\left| \int\limits_{a}^{t}(f(x,y+\Delta y)-f(x,y))dx\right|<
\frac{\varepsilon}{2}\cdot \frac{t-a}{b-a}\leqslant \frac{\varepsilon}{2}$$ 
Еcли $|\Delta t|< \frac{\varepsilon}{2M}$, то 
$$\left| \int\limits_{t}^{t+\Delta t}f(x,y+\Delta y)dx\right|\leqslant 
\left| \int\limits_{t}^{t+\Delta t}|f(x,y+\Delta y)|dx\right|\leqslant 
M\cdot \frac{\varepsilon}{2M}=\frac{\varepsilon}{2}$$
Пусть $\delta=\min \{\delta,\frac{\varepsilon}{2M}\}$, тогда
$$\forall \varepsilon>0~\exists \delta(\varepsilon)>0~\forall |\Delta t|
<\delta~\forall |\Delta y|<\delta:|\Delta g|<\varepsilon$$
Это означает, что $\lim\limits_{\substack{\Delta t\to 0\\
\Delta y\to 0}}\Delta g=0$. Заметим, что $g(t,y)$ непрерывна на  $P$.
Применим теорему о дифференцировании к функции $G(t)$: 
$$G'(t)=\int\limits_{c}^{d} \frac{\partial g}{\partial t}(t,y)dy=
\int\limits_{c}^{d}f(t,y)dy$$
С другой стороны,
$$H'(t)=\frac{\partial}{\partial t}\left( \int\limits_{a}^{t}dx
\int\limits_{c}^{d}f(x,y)dy \right)=\int\limits_{c}^{d}f(t,y)dy$$
Итак, мы получили, что $G'(t)=H'(t),~G(a)=H(a)=0$, откуда
$G(t)=H(t)\implies G(b)=H(b)$. $\square$ \\













%17.11.22
\textbf{Пример.} $\int_{0}^{1} \frac{x^b-x^a}{\ln(x)}dx$, $0<a<b$.  
Имеем $\int\limits_{a}^{b} x^ydy=\frac{x^b-x^a}{\ln(x)}$, 
подынтегральная функция непрерывна на бруске $[0,1]\times[a,b]$, 
тогда интеграл равен $\int\limits_{0}^{1}dx\int\limits_{a}^{b}x^ydy=
\int\limits_{a}^{b}\frac{dy}{y+1}=\ln(b+1)-\ln(a+1)$.

\textbf{Пример 2.} Вычислить $\int\limits_{0}^{1}\sin(\ln(\frac{1}{x}))
\frac{x^b-x^a}{\ln(x)}=\int\limits_{0}^{1}dx \int\limits_{a}^{b}
\sin(\ln(\frac{1}{x}))x^ydy$. Функция $f(x,y)=\sin(\ln(\frac{1}{x}))x^y$ 
непрерывна на $[0,1]\times[a,b],~f(0,y)=0$. Тогда
$\int=\int\limits_{a}^{b}dy \int\limits_{0}^{1}\sin(\ln(\frac{1}{x}))x^ydx=
\begin{cases} t=\ln(\frac{1}{x})=-\ln(x)\\dx=-e^{-t}dt\end{cases}= 
\int\limits_{a}^{b} dy \int\limits_{0}^{\infty} \sin(t)e^{-ty}e^{-t}dt$. 
Внутренний интеграл возьмем по частям: $I=\int\limits_{0}^{\infty}
\sin(t)e^{-t(y+1)}dt=-\cos(t)e^{-t(y+1)}\big|_0^\infty
-(y+1)\int\limits_{0}^{\infty}cos(t)e^{-t(y+1)}dt,~I=1-(y+1)^2I$.
Значит, искомый интеграл равен 
$\int\limits_{a}^{b}\frac{dy}{(y+1)+1}=arctg\left( \frac{a-b}{1+ab}\right) $
Домашка: тоже самое для косинуса.
\subsection{Несобственные интегралы, зависящие от параметра}
Рассмотрим семейство функций $f(x,y),~x\in X,y\in Y$. Пусть
$M\subset Y$ - множество сходимости.
\begin{defin}
$f(x,y)$ сходится поточечно к  $\varphi(x)$ на М при $x\to x_0$, если
 $$\forall y\in M~\forall \varepsilon>0~\exists \delta(\varepsilon,y)~
 \forall x\in X\cap U^\circ_\delta(x_0):
 |f(x,y)-\varphi(y)|<\varepsilon$$
\end{defin}
Определение предела: 
$$\lim\limits_{x\to x_0}f(x)=A\iff \forall \varepsilon>0~\exists 
\delta(\varepsilon)~\forall x\in X:0<|x-x_0|<\delta\implies|f(x)-A|
<\varepsilon$$
\begin{defin}
$f(x,y)$ сходится равномерно к $\varphi(x)$ на Е при $x\to x_0$, если
$$\forall \varepsilon>0~\exists \delta(\varepsilon)>0~\forall x\in X\cap
U^\circ_\delta(x_0)~\forall y\in E:|f(x,y)-\varphi(x)|<\varepsilon$$
\end{defin}

\textbf{Пример.} $f(x,y)=\sin(y^x)$ - непрерывен (??)

\begin{defin}
$\int\limits_{a}^{\infty}f(x,y)dx$ сходится поточечно к $F(y)$ на М, если
$$\forall y\in M~\forall \varepsilon>0~\exists b_0(\varepsilon,y)>a~
\forall b>b_0:\left| \int\limits_{b}^{\infty}f(x,y)dx\right|<\varepsilon$$
\end{defin}
Обозначим $F(b,y)=\int\limits_{a}^{b}f(x,y)dx$. Тогда $|F(b,y)-F(y)|=
\left| \int\limits_{b}^{\infty} f(x,y)dx \right|$. $F(b,y)$ сходится к $F(y)$
поточечно на  $M$ при  $b\to \infty$. Обозначим остаток интеграла
$R(b,y)=\int\limits_{b}^{\infty}f(x,y)dx$. Остаток сходится поточечно 
$R(b,y)\to 0~\forall y\in N$ при $b\to \infty$. 
\begin{defin}
Интеграл сходится равномерно к $F(y)$ на  $E$, если
$$\forall \varepsilon>0~\exists b_0(\varepsilon)~\forall b>b_0~\forall y\in E:
\left| \int\limits_{b}^{\infty}f(x,y)dx\right|<\varepsilon$$
\end{defin}
Как и с рядями, есть супремум-критерий.
\begin{theor}
    (супремум-критерий)\\
    Несобственный интеграл $\int\limits_{a}^{\infty} f(x,y)dx$, зависящий от
    параметра, сходится равномерно на $E$ тогда и только тогда, когда
    $$\lim\limits_{b\to \infty}\sup\limits_{y\in  E}\left| 
    \int\limits_{b}^{\infty}f(x,y)dx \right|=0$$
\end{theor}
\textbf{Доказательство.}  Самостоятельно
$\square$ \\
\begin{theor}.
    (метод оценки остатка)\\
Пусть интеграл $\int\limits_{a}^{\infty} f(x,y)dx$ сходится на $E$, и 
$r(b)$ - какая-то оценка остатка. Тогда если
 $|R(b,y)|\leqslant r(b)~\forall y\in E$ и $r(b)\to 0$ при  $b\to \infty$,
 тогда $\int\limits_{a}^{\infty}f(x,y)dx$ сходится равномерно на 
 $E$. Если же существует такая функция  $y(b)$, что $R(b,y(b))\to s\ne 0$,
 то интеграл не сходится равномерно на $X$.
\end{theor}

\textbf{Пример.} $F(y)=\int\limits_{0}^{\infty}ye^{-xy}dx$. Доказать, что 
сходимость равномерная при $[y_0,\infty),~y_0\geqslant0$, но на 
$(0,\infty)$ нет равномерной сходимости. Решение: пусть остаток
$R(b,y)=\int\limits_{b}^{\infty}ye^{-xy}dx=e^{-by}$. По методу оценки
остатка при оценке $r(b)=e^{-by_0}$ имеем равномерную сходимость.
Если мы возьмем $y=\frac{1}{b}$, то и $R(b,\frac{1}{b})=e^{-1}\ne0$, поэтому
нет равномерной сходимости. 




%21.11.22
\textbf{Пример.} Пусть $f(x)>0$ при  $x\geqslant 0$,
$\int\limits_{0}^{\infty} f(x)dx$ сходится.
Тогда $\forall \alpha>0$ интеграл $\int\limits_{0}^{\infty}f(y^\alpha x)dx$
сходится равномерно на $[y_0,\infty),~y_0>0$, и сходится неравномерно
на $(0,\infty)$.\\
\textbf{Решение.} 1. Методом оценки остатка: 
$R(b,y)=\int\limits_{b}^{\infty}f(y^\alpha x)dx=\frac{1}{y^\alpha}
\int\limits_{b}^{\infty} f(y^\alpha x)d(y^\alpha x)=\frac{1}{y^\alpha}
\int\limits_{by^\alpha}^{\infty}f(t)dt\leqslant \frac{1}{y^\alpha_0}
\int\limits_{by^\alpha}^{\infty} f(t)dt=r(b)\to 0$ при $b\to \infty$,
так как $\int\limits_{0}^{\infty} f(t)dt$ сходится.\\
2. Докажем неравномерную сходимость по супремум-критерию:
$\sup\limits_{y>0}\int\limits_{b}^{\infty} f(y^\alpha x)dx=
\sup\limits_{y>0}\frac{1}{y^\alpha}\int\limits_{by^\alpha}^{\infty} f(t)dt
\geqslant
\sup\limits_{0<y\leqslant 1}\frac{1}{y^\alpha}\int\limits_{by^\alpha}^{\infty}
f(t)dt\geqslant
\sup\limits_{0<y\leqslant 1}\frac{1}{y^\alpha}\int\limits_{b}^{\infty}f(t)dt=
\infty$ при $y\to \infty$. Тогда по супремум-критерию нет равномерной 
сходимости.

\textbf{Пример.} Интеграл Пуассона аналогично сходится равномерно на 
бесконечности, если интервал начинается не с нуля. 
\subsubsection{Свойства несобственных интегралов, зависящих от параметров}
\begin{enumerate}
    \item Равномерно сходящиеся интегралы образуют линейное пространство.
    \item Если равномерно сходится на множестве, то сходится и на его 
        подмножестве.
    \item Сходится равномерно на конечном объединении областей, 
        где сходится равномерно.
\end{enumerate}
\textbf{Пример.} Покажем, что свойство 3 нельзя обощить на объединения
бесконечного числа множеств. Так, $\int\limits_{0}^{\infty}e^{-x^2y}dx$
сходится равномерно на $E_n=[\frac{1}{n},\infty)$, но расходится на 
бесконечном объединении таких областей. 
\begin{theor}(критерий Коши)\\
    $\int\limits_{a}^{\infty}f(x,y)dx$ сходится равномерно на $Y$ тогда и 
    только тогда, когда 
     $$\forall \varepsilon>0~\exists b_0(\varepsilon)>a~\forall b_1,b_2>b_0~
 \forall y\in Y:\left|\int\limits_{b_1}^{b_2}f(x,y)dx\right|<\varepsilon$$
\end{theor}
\textbf{Доказательство.}  Фиксируем $\varepsilon>0$. Для 
$$\frac{\varepsilon}{2}~\exists b_0>a~\forall b>b_0~\forall y\in Y:
\left|\int\limits_{b_1}^{\infty}f(x,y)dx\right|<
\left|\frac{\varepsilon}{2}\right|$$
Пусть $b_1,b_2>b_0$, тогда $\left| \int\limits_{b_1}^{b_2}f(x,y)dx\right|=
\left|\int\limits_{b_1}^{\infty}f(x,y)dx-\int\limits_{b_2}^{\infty} f(x,y)dx
\right|\leqslant 
\left|\int\limits_{b_1}^{\infty}f(x,y)dx\right|
-\left|\int\limits_{b_2}^{\infty} f(x,y)dx\right|<
\frac{\varepsilon}{2}+\frac{\varepsilon}{2}=\varepsilon$.\\
Обратно, для
$$\frac{\varepsilon}{2}~\exists b_0>a~\forall b_1,b_2>b_0~\forall y\in Y:
\left|\int\limits_{b_1}^{b_2}f(x,y)dx\right|<
\left|\frac{\varepsilon}{2}\right|$$
Тогда 
$\left|\int\limits_{b_1}^{\infty}f(x,y)dx-\int\limits_{b_2}^{\infty} f(x,y)dx
\right|<\frac{\varepsilon}{2}$. Если $b_2\to \infty$, то
$$\forall y\in Y: \int\limits_{b_2}^{\infty}f(x,y)dx\to 0$$ 
(так как есть поточечная сходимость);
$\left|\int\limits_{b_1}^{\infty}f(x,y)dx\right|\leqslant\frac{\varepsilon}{2}
<\varepsilon$
Итак, выполняется определение. $\square$ \\

\begin{theor}
    (метод граничной точки)\\
    Пусть 1. $f(x,y)$ непрерывна на $[a,\infty)\times[c,d)$\\
    2. $\forall y\in (c,d)$ интеграл $\int\limits_{a}^{\infty} f(x,y)dx$ 
    сходится \\
    3. $\int\limits_{a}^{\infty}f(x,c)dx$ расходится.\\
    Тогда $\int\limits_{a}^{\infty}f(x,y)dx$ не сходится равномерно на 
    $(c,d)$
\end{theor}
\textbf{Доказательство.} Допустим, на $(c,d)$ есть равномерная сходимость.
Тогда по супремум-критерию
Самостоятельно от противного. 
$\square$ 

\begin{theor}
    (признак Вейерштрасса равномерной сходимости)\\
    Пусть\\
    1. $\forall x\geqslant a~\forall y\in Y:|f(x,y)|\leqslant g(x)$;\\
    2. $\int\limits_{a}^{\infty}g(x)dx$ сходится.\\
Тогда $\int\limits_{a}^{\infty}f(x,y)dx$ сходится равномерно на  $Y$.
\end{theor}
\textbf{Доказательство.}  Используем критерий Коши. Зафиксируем 
$\varepsilon>0$. Тогда интеграл сходится тогда и только тогда, когда
$$\forall \varepsilon>0~\exists b_0>a~\forall b_1,b_2>b_0:
\left| \int\limits_{b_1}^{b_2} g(x)dx \right|<\varepsilon$$
По условию $\forall y\in Y~\forall x\geqslant a:|f(x,y)|\leqslant g(x)$, 
откуда $g(x)\geqslant 0$. Значит,
$\left|\int\limits_{b_1}^{b_2}g(x)dx\right|=\int\limits_{b_1}^{b_2}g(x)dx$,
поэтому $\left| \int\limits_{b_1}^{b_2}f(x)dx \right| \leqslant 
\left| \int\limits_{b_1}^{b_2} |f(x)|dx \right| \leqslant 
\left| \int\limits_{b_1}^{b_2}g(x)dx\right| <\varepsilon$. По критерию Коши,
$\int\limits_{a}^{\infty}f(x,y)dx$ сходится равномерно на $Y$. $\square$ 



%24.11.22
\begin{theor}
    (признак Дирихле равномерной сходимости несобственного интеграла, 
    зависящего от параметра)\\
    1. $\forall y\in Y$ $f(x,y)$ непрерывна на  $[a,\infty)$\\
    2. $\forall y\in Y$ $\frac{\partial g}{\partial x}(x,y)$ 
    непрерывна на  $[a,\infty)$\\
    3. $\forall y\in Y$ $g(x,y)$ монотонна по $x\in [a,\infty)$\\
    4. $g(x,y)\rightrightarrows 0$ при $x\to \infty$\\
    5. $\exists M=const~\forall y\in Y~\forall x\geqslant a:
    \left| \int\limits_{a}^{x}f(t,y)dt \right|\leqslant M$.\\
    Тогда $\int\limits_{a}^{\infty}f(x,y)g(x,y)dx$ сходится равномерно на
    $Y$.
\end{theor}
\textbf{Доказательство.}  По критерию Коши. 
Для
$$\frac{\varepsilon}{4M}>0~\exists b_0(\varepsilon)>a~\forall x>b_0~
\forall y\in Y:|g(x,y)|<\frac{\varepsilon}{4m}$$ 
Возьмем $b_1,b_2>b_0$. Тогда

\begin{equation*}
\left| \int\limits_{b_1}^{b_2}f(x,y)g(x,y)dx \right| = 
\left| \int\limits_{b_1}^{b_2}g(x,y)d\left( \int\limits_{a}^{x}f(t,y)dt
\right)   \right| = 
\end{equation*}
$$=\left|\left( g(x,y)\cdot\int\limits_{a}^{x}f(t,y)dt\right)\bigg|^{b_2}_{b_1}
- \int\limits_{b_1}^{b_2} \left( \int\limits_{a}^{x} f(t,y)dt \right) \cdot 
\frac{\partial g}{\partial x} (x,y)dx \right|\leqslant$$
$$\leqslant 
\left| g(b_2,y) \right|\cdot \left| \int\limits_{a}^{b_2}f(t,y)dt \right| +
\left| g(b_1,y) \right|\cdot \left| \int\limits_{a}^{b_1}f(t,y)dt \right| +
\left| \int\limits_{b_1}^{b_2}\biggl| \int\limits_{a}^{x}f(t,y)dt\biggr|\cdot 
\frac{\partial g}{\partial x} (x,y)dx \right| \leqslant $$ 
$$\leqslant \frac{\varepsilon}{4M}\cdot M+\frac{\varepsilon}{4M}\cdot M+
M\cdot |g(b_2,y)-g(b,y)|<\frac{\varepsilon}{2}+\frac{\varepsilon}{2}=
\varepsilon$$ 
Итак, 
$$\forall \varepsilon>0~\exists b_0(\varepsilon)>a~\forall b_1,b_2>b_0~
\forall y\in Y:\left| \int\limits_{b_1}^{b_2}f(x,y)g(x,y)dx\right| 
<\varepsilon$$ 
Тогда по 
критерию Коши $\int\limits_{a}^{\infty}f(x,y)g(x,y)dx$ сходится равномерно на
$Y$. $\square$ 

\begin{theor}
    (признак Абеля равномерной сходимости несобственного интеграла, 
    зависящего от параметра)\\
    1. $\forall y\in Y$ $f(x,y)$ непрерывна на  $[a,\infty)$\\
    2. $\forall y\in Y$ $\frac{\partial g}{\partial x}(x,y)$ 
    непрерывна на  $[a,\infty)$\\
    3. $\forall y\in Y$ $g(x,y)$ монотонна по $x\in [a,\infty)$\\
    %4. $g(x,y)\rightrightarrows 0$ при $x\to \infty$\\
    4. $\exists M=const~\forall y\in Y~\forall x\geqslant a:
    \left| g(x,y)\right|\leqslant M$.\\
    5. $\int\limits_{a}^{\infty}f(x,y)dx$ сходится равномерно на $Y$.\\
    Тогда $\int\limits_{a}^{\infty}f(x,y)g(x,y)dx$ сходится равномерно на
    $Y$.
\end{theor}
\textbf{Доказательство.}  По критерию Коши. 
Для
$$\frac{\varepsilon}{3M}>0~\exists b_0(\varepsilon)>a~\forall b_1,b_2>b_0~
\forall y\in Y: \int\limits_{b_1}^{b_2}f(x,y)dx<\frac{\varepsilon}{3M}$$ 
Возьмем $b_1,b_2>b_0$. Тогда
\begin{equation*}
\left| \int\limits_{b_1}^{b_2}f(x,y)g(x,y)dx \right| = 
\left| \int\limits_{b_1}^{b_2}g(x,y)d\left( \int\limits_{b_1}^{x}f(t,y)dt
\right)   \right| = 
\end{equation*}
$$=\left|\left( g(x,y)\cdot\int\limits_{b_1}^{x}f(t,y)dt\right)
\bigg|^{b_2}_{b_1}
- \int\limits_{b_1}^{b_2} \left( \int\limits_{b_1}^{x} f(t,y)dt \right) \cdot 
\frac{\partial g}{\partial x} (x,y)dx \right|\leqslant$$
$$\leqslant 
\left| g(b_2,y) \right|\cdot \left| \int\limits_{b_1}^{b_2}f(t,y)dt \right| +
\left| \int\limits_{b_1}^{b_2} \bigg| \int\limits_{b_1}^{x}f(t,y)dt\bigg|\cdot 
\frac{\partial g}{\partial x} (x,y)dx \right| \leqslant $$ 

$$\leqslant \frac{\varepsilon}{3M}\cdot M+
M\cdot |g(b_2,y)-g(b,y)|<\frac{\varepsilon}{3}+\frac{2\varepsilon}{3}=
\varepsilon$$ 
Итак, 
$$\forall \varepsilon>0~\exists b_0(\varepsilon)>a~\forall b_1,b_2>b_0~
\forall y\in Y:\left| \int\limits_{b_1}^{b_2}f(x,y)g(x,y)dx\right| 
<\varepsilon$$ 
Тогда по 
критерию Коши $\int\limits_{a}^{\infty}f(x,y)g(x,y)dx$ сходится равномерно на
$Y$. $\square$ \\

\textbf{Пример.} $\int\limits_{1}^{\infty} \frac{y^2\cos xy}{x+y^2}$,
$y\in [0,\infty)$. Исследуем на равномерную сходимость. Пусть
$f(x,y)=y\cos xy,~g(x,y)=\frac{y}{x+y^2}$. Условия проверяются очевидным
образом, интеграл сходится равномерно по Дирихле. 

\begin{theor}
    (о непрерывности несобственного интеграла, зависящего от параметра)\\
    1. $f(x,y)$ непрерывна на  $[a,\infty)\times Y$\\
    2. $\int\limits_{a}^{\infty} f(x,y)dx$ сходится равномерно на $Y$.\\
    Тогда  $\Phi(y)=\int\limits_{a}^{\infty} f(x,y)dx$ непрерывна на $Y$
\end{theor}
\textbf{Доказательство.} 
Функция непрерывна, если она непрерывна в каждой точке. 
$\Phi(y)$ непрерывна в  $y_0$ тогда и только тогда
$$\forall \varepsilon>0~\exists \delta>0~\forall y\in Y:
|y-y_0|<\delta \implies |\Phi(y)-\Phi(y_0)|<\varepsilon$$
По второму условию, так как интеграл сходится равномерно, то
для любого 
 $$\frac{\varepsilon}{3}>0~\exists b_0>a~\forall b>b_0~\forall y\in Y:
\left|\int\limits_{b}^{\infty} f(x,y)dx\right|<\frac{\varepsilon}{3}
 $$
Тогда 
$$
\Phi(y)-\Phi(y_0)=\int\limits_{a}^{\infty} f(x,y)dx-
\int\limits_{a}^{\infty} f(x,y_0)dx=
$$
$$
=\int\limits_{a}^{b} f(x,y)dx+ \int\limits_{b}^{\infty}f(x,y)dx -
\int\limits_{a}^{b} f(x,y_0)dx -\int\limits_{b}^{\infty}f(x,y_0)dx=
$$
$$
=\left( \int\limits_{a}^{b} f(x,y)dx-\int\limits_{a}^{b} f(x,y_0)dx\right) +
\int\limits_{b}^{\infty}f(x,y)dx-\int\limits_{b}^{\infty}f(x,y_0)dx
$$
По теореме о непрерывности собственного интеграла, зависящего от параметра, 
для
$$
\forall\,\frac{\varepsilon}{3}>0~\exists \delta>0~\forall y\in Y:
|y-y_0|<\delta \implies |F(y)-F(y_0)|<\frac{\varepsilon}{3}
$$
Тогда 
$$|\Phi(y)-\Phi(y_0)|\leqslant |F(y)-F(y_0)|+ 
\left|\int\limits_{b}^{\infty}f(x,y)dx  \right| + 
\left|\int\limits_{b}^{\infty}f(x,y_0)dx \right|\leqslant 
\frac{\varepsilon}{3}+\frac{\varepsilon}{3}+\frac{\varepsilon}{3}=
\varepsilon$$
Значит, $\Phi(y)$ непрерывна в любой точке на $Y$, то есть она непрерывна
на $Y$. $\square$ 

\begin{theor}
    (о предельном переходе под знаком несобственного интеграла)\\
    1. $f(x,y)$ непрерывна на  $[a,\infty)\times Y$\\
    2. $\int\limits_{a}^{\infty} f(x,y)dx$ сходится равномерно на $Y$.\\
    Тогда  
$$\lim\limits_{y \to y_0}\int\limits_{a}^{\infty} f(x,y)dx=
\int\limits_{a}^{\infty} \lim\limits_{y \to y_0}f(x,y)dx = 
\int\limits_{a}^{\infty}f(x,y_0)dx$$
\end{theor}
\textbf{Доказательство.} В предыдущей теореме было доказано, что 
функция $\Phi(y)=\int\limits_{a}^{\infty} f(x,y)dx$ непрерывна на $Y$. 
Значит, 
$$\lim\limits_{y \to y_0}\int\limits_{a}^{\infty} f(x,y)dx=
\lim\limits_{y \to y_0} \Phi(y) = \Phi(y_0) = 
\int\limits_{a}^{\infty}f(x,y_0)dx$$
Так как функция $f(x,y)$ непрерывна, то получаем второе равенство:
$$\int\limits_{a}^{\infty}f(x,y_0)dx = 
\int\limits_{a}^{\infty}\lim\limits_{y \to y_0} f(x,y)dx\quad\square$$



\begin{theor}
    (об интегрировании несобственного интеграла, зависящего от параметра)\\
    1. $f(x,y)$ непрерывна на  $[a,\infty)\times [c,d]$\\
    2. $\int\limits_{a}^{\infty} f(x,y)dx$ сходится равномерно на $[c,d]$.\\
    Тогда 
    $$\int\limits_{c}^{d}dy \int\limits_{a}^{\infty}f(x,y)dx =
    \int\limits_{a}^{\infty}dx \int\limits_{c}^{d}f(x,y)dy$$
\end{theor}
\textbf{Доказательство.} Обозначим $\Phi(y)=\int\limits_{a}^{\infty}
f(x,y)dx$. Эта функция непрерывна на $[c,d]$ по теореме о 
непрерывности несобственного интеграла, зависящего от параметра. Значит,
$\Phi(y)$ интегрируема на  $[c,d]$, то есть существует и конечен 
интеграл  $\int\limits_{c}^{d}dy\int\limits_{a}^{\infty}f(x,y)dx=const$.
Покажем, что несобственный интеграл справа сходится к этой константе, то
есть при $b\to \infty$ имеет место
$\int\limits_{a}^{b} dx\int\limits_{c}^{d}f(x,y)dy\to
\int\limits_{c}^{d} dy \int\limits_{a}^{\infty} f(x,y)dx$. \\
Зафиксируем $\varepsilon>0$. По условию, $\int\limits_{a}^{\infty}f(x,y)dx$
сходится равномерно на $[c,d]$ Тогда для 
$$\frac{\varepsilon}{d-c}>0~\exists b_0(\varepsilon)~\forall y\in [c,d]~
\forall b>b_0:\left| \int\limits_{b}^{\infty}f(x,y)dx \right|<
\frac{\varepsilon}{d-c}$$
 Отсюда
$$\left| \int\limits_{a}^{b} dx\int\limits_{c}^{d}f(x,y)dy-
\int\limits_{c}^{d}dy \int\limits_{a}^{\infty}f(x,y)dx \right|\leqslant 
 \int\limits_{c}^{d}dy\left| \int\limits_{a}^{b} f(x,y)dx-
 \int\limits_{a}^{\infty} f(x,y)dx\right| =$$
$$=\int\limits_{c}^{d} dy\left| \int\limits_{b}^{\infty} f(x,y)dx \right|
<\frac{\varepsilon}{d-c}\cdot (d-c)=\varepsilon$$ 
Итак, 
$$\forall \varepsilon>0~\exists b_0(\varepsilon)>a~\forall b>b_0~\forall y\in 
Y:\left|  \int\limits_{a}^{b} dx\int\limits_{c}^{d}f(x,y)dy-
\int\limits_{c}^{d}dy \int\limits_{a}^{\infty}f(x,y)dx \right| <
\varepsilon \quad\square$$









%28.11.22
\begin{theor}
    (о дифференцировании несобственного интеграла, зависящего от параметра)\\
    Пусть\\
    1. $f(x,y)$ непрерывна на  $[a,\infty)\times Y$\\
    2. $\int\limits_{a}^{\infty} f(x,y)dx$ сходится  $\forall y\in Y$.\\
3. $\frac{\partial f}{\partial y}(x,y)$ непрерывна на $[a,\infty)\times Y$.\\
4. $\int\limits_{a}^{\infty} \frac{\partial f}{\partial y}(x,y)dx$ сходится
равномерно на $Y$.\\
    Тогда $\forall y\in Y$ 
$$\left( \int\limits_{a}^{\infty}f(x,y) dx \right)'_y=\int\limits_{a}^{\infty}
 \frac{\partial f}{\partial y}(x,y)dx$$

   % $$\int\limits_{c}^{d}dy \int\limits_{a}^{\infty}f(x,y)dx =
   % \int\limits_{a}^{\infty}dx \int\limits_{c}^{d}f(x,y)dy$$
\end{theor}
\textbf{Доказательство.} Так как выполняются условия теоремы об интегрировании
н.и.з.от п. для $\int\limits_{a}^{\infty}\frac{\partial f}{\partial y}(x,y)dx$,
то зафиксируем $y_0\in Y,~y\in \tilde Y$ без крайних точек, и тогда
$$\int\limits_{y_0}^{y}dy \int\limits_{a}^{\infty} 
\frac{\partial f}{\partial y}(x,y)dx=\int\limits_{a}^{\infty}dx
\int\limits_{y_0}^{y} \frac{\partial f}{\partial y}(x,y)dy=
\int\limits_{a}^{\infty}f(x,y)dx-\int\limits_{a}^{\infty}f(x,y_0)dx$$
Второй интеграл равен числу, поэтому
$$\left( \int\limits_{y_0}^{y} dy \int\limits_{a}^{\infty}
\frac{\partial f}{\partial y}(x,y) dx \right)'_y=
\int\limits_{a}^{\infty}\frac{\partial f}{\partial y}(x,y)dx 
=\left( \int\limits_{a}^{\infty} f(x,y)dx \right)'_y$$ $\square$ 
%готово

\textbf{Пример.} $\int\limits_{0}^{\infty} \frac{\sin\alpha x}{x}
e^{-\beta x}dx$, где $\alpha\in \mathbb{R}$, $\beta\in 0$. 
Легко проверяются условия теоремы об интегрировании, и мы можем 
привести интеграл к виду $\int\limits_{0}^{\infty}dx
\int\limits_{0}^{\alpha} \cos{xy}\cdot e^{-\beta x}dy$, который берется по
частям, ответ $arctg \frac{\alpha}{\beta}$. Так как арктангенс нечетный, 
то эта формула справедлива как для положительных, так и отрицательных
$\alpha$ (при условии $\beta>0$). Другой способ - по теореме о
дифференцировании. Снова обозначим $\Phi(\alpha)=
\int\limits_{0}^{\infty} \frac{\sin{\alpha x}}{x}e^{-\beta x}dx$. 
$\Phi(x,\alpha)$ непрерывна на $[0,\infty)\times \mathbb{R}$. Доопределим
функцию: $\Phi(0,\alpha)=\lim\limits_{x \to +0} \Phi=\alpha$. Снова легко
проверяются условия теоремы. Имеем $\Phi(\alpha)=
\int\limits_{0}^{\infty} \cos{\alpha x}e^{-\beta x}dx=\frac{\beta}{
    \beta^2+\alpha^2}$. Интегрируя и подставляя начально условие, снова 
    получаем арктангенс
\section{Вычисление некоторых классических интегралов}

\textbf{Интеграл Дирихле}
$$\boxed{
\int\limits_{0}^{\infty}\frac{\sin\alpha x}{x}dx=\frac{\pi}{2}\sign\alpha}$$
Пусть $\Phi(\beta)=\int\limits_{a}^{\infty}\frac{\sin\alpha x}{x}e^{
-\beta x}dx$.\\
1) $\varphi(x,\beta)=\frac{\sin\alpha x}{x}e^{-\beta x}$ непрерывна на 
$[0,\infty)\times [0,\infty)$, $\varphi(0,\beta)=\alpha$. \\
2) Докажем, что $\Phi(\beta)$ сходится равномерно на $[0,\infty)$.
Так как $\forall \alpha\in \mathbb{R}: \int\limits_{0}^{\infty}
\frac{\sin\alpha x}{x}dx$ сходится по признаку Дирихле, то сходимость
исходного интеграла равномерная (так как не зависит от $\beta$). 
Далее, $0\leqslant e^{-\beta x}\leqslant 1$ при $\beta\geqslant 0$,
$x\geqslant 0$. Значит, $e^{-\beta x}$ убывает при данных условиях.
По признаку Абеля $\Phi(\beta)=\int\limits_{0}^{\infty}\frac{\sin\alpha x}{
x}e^{-\beta x}dx$ сходится равномерно. По теореме о предельном переходе под
знаком несобственного интеграла на $[0,\infty)$, имеем
$\Phi(0)=\lim\limits_{\beta \to 0}\Phi(\beta)$. Теперь найдем этот интеграл:
$\int\limits_{0}^{\infty}\frac{\sin\alpha x}{x}e^{-\beta x}=
\int\limits_{0}^{\infty}dx \int\limits_{0}^{\alpha} \cos(xy)e^{-\beta x}dy=
\int\limits_{0}^{\alpha}dy \int\limits_{0}^{\infty}\cos(xy)e^{-\beta x}dx=
\int\limits_{0}^{\infty}\left(\frac{1}{\beta^2+y^2}\right)dy=
\arctg \frac{y}{\beta}\Big|^\alpha_0=\arctg \frac{\alpha}{\beta}$.
Значит, $\Phi(0)=\lim\limits_{\beta \to 0}\arctg\frac{\alpha}{\beta}=
\frac{\pi}{2}\sign\alpha$. 

\textbf{Интеграл Лапласа}
$$\boxed{
    \int\limits_{0}^{\infty} \frac{\cos\alpha x}{1+x^2}dx=\frac{\pi}{2}
    e^{-|\alpha|}
}$$
Дифференцировать много раз не получится, так как мы придем к расходящемуся 
ряду. Нам нужем финт ушами, а именно прибавить $\frac{\pi}{2}$:
$$\Phi'(\alpha)+\frac{\pi}{2}=-\int\limits_{0}^{\infty} \frac{x\sin\alpha x}{
1+x^2}dx+\int\limits_{0}^{\infty} \frac{\sin\alpha x}{x}dx=
\int\limits_{0}^{\infty}\frac{1+x^2-x^2}{x(1+x^2)}\sin\alpha x\,dx$$
$$\Phi''(\alpha)=\left(\Phi'(\alpha)+\frac{\pi}{2}\right)'
=\int\limits_{0}^{\infty}
\left( \frac{\sin\alpha x}{x(1+x^2)}\right)'_\alpha dx$$
Внезапно, мы получили диффур $\Phi''(\alpha)=\Phi(\alpha)$. Общее решение
$\Phi(\alpha)=C_1e^{-\alpha}+C_2e^{\alpha}$. Поскольку $\Phi$ ограничена,
то  $C_2=0$, а поскольку $\Phi(0)=\int\limits_{0}^{\infty}\frac{dx}{1+x^2}=
\frac{\pi}{2}$, то $C_1=\frac{\pi}{2}$. 






%
\chapter{Функции Эйлера}
\section{Гамма-функция}
$$\Gamma(s)=\int\limits_{0}^{\infty}x^{s-1}\cdot e^{-x}dx$$
На бесконечности сходится всегда, в нуле сходится при $s>0$.
Если проинтегрировать по частям, беря первообразуную от экспоненты, 
получим  $0+\Gamma(x-1)$. Причем $\Gamma(1)=e^{-t}\Big|_0^\infty=1$.
В точке $1/2$ сам Бог велел делать замену  $u=\frac{x^1}{2}$, и мы 
сведем к интегралу Пуассона. 

\textbf{Свойства гамма-функции}\\
1. Область определения $\equiv$ множество таких $s$, на котором 
$\Gamma(s)$ сходится:  $s>0$;\\
2. Равномерная сходимость на $[s_1,s_2]$, где $0<s_1<s_2<\infty$;\\
3. $\Gamma(s)$ непререрывна при  $s>0$;\\
4.  $\Gamma(s)>0$ при  $s>0$;\\
5.  $\boxed{\Gamma(s+1)=s\cdot \Gamma(s)}$;\\
6. $\Gamma(1)=1,~\Gamma(n+1)=n!$;\\
7.  $\Gamma\left( \frac{1}{2} \right)=\sqrt{\pi}$, $\Gamma\left( 
n+\frac{1}{2}\right)=\frac{(2n-1)!!}{2^n}\cdot \sqrt{\pi}$ ;\\
8. $\forall s>1$: $\Gamma(s)=(s-1)(s-2)...(s-n)\Gamma(s-n)$, где  $n=[s]$; 
любое значение гамма-функции можно выразить через её значения на $(0,1]$\\
9. $\Gamma^{(n)}(s)=\int\limits_{0}^{\infty}x^{s-1}e^{-x}\ln^nx$, 
причем сходится при $s>0$, равномерно сходится там же, где и гамма-функция.\\
10. $\Gamma(s)\Gamma(1-s)=\frac{\pi}{\sin(\pi n)}$ - формула дополнения\\
11. $\Gamma(x+1)=x^xe^{-x}\sqrt{2\pi n}(1+\alpha(x))$ - асимптотическая 
формула.\\
12. График: $\lim\limits_{s \to +0} \Gamma(s)=
\lim\limits_{s \to +0} \frac{\Gamma(s+1)}{s}=\frac{1}{+0}=\infty$.
$\Gamma(1)=\Gamma(2)=1$. Сначала убывает, затем возрастает.\\
13. $\Gamma(s)=\frac{1}{se^{\gamma s}}\prod\limits_{n=1}^{\infty}
\left( 1+\frac{s}{n} \right)^{-1}e^{\frac{s}{n}}$ - продолжение определения
функции на отрицательные числа (кроме отрицательных целых). 

\textbf{Доказательство.}

1. Докажем, что гамма-функция определена при $s>0$. 
Рассмотрим сумму интегралов
$\Gamma(s)=\int\limits_{0}^{1}x^{s-1}\cdot e^{-x}dx+
\int\limits_{1}^{\infty}x^{s-1}\cdot e^{-x}dx$. 
Первый интеграл сходится при $s>0$ и расходится при $s\leqslant 0$
по предельному признаку сравнения с интегралом
$\int\limits_{0}^{1}x^{s-1}dx$. Второй интеграл: имеем
$$\forall s\in \mathbb{R}~\exists x(s)~\forall x>x(s):
x^{s-1}\leqslant e^{\frac{x}{2}}$$ 
Значит, $\int\limits_{x(s)}^{\infty}e^{-\frac{x}{2}}dx$ сходится, и исходный
интеграл сходится по признаку сравнения при любом $s$. Значит, область 
определения гамма-функции -  $s>0$.

2. Докажем равномерную сходимость по признаку Вейерштрасса. Получаем, что
$x^{s-1}\cdot e^{-x}\leqslant x^{s-1}\leqslant x^{s_1-1}$ при фиксированном
$x\in [0,1]$. При этом интеграл $\int\limits_{0}^{1}x^{s_1-1}dx$ сходится, 
поэтому интеграл сходится на $[s_1,s_2]$ равномерно. Если $x\geqslant 1$,
то $x^{s-1}\cdot e^{-x}\leqslant x^{s_2-1}\cdot e^{-x}$, правая часть 
сходится и не зависит от $s$, значит, сходимость равномерная. Значит,
на объединении $1>s>0$ и  $s>0$ сходимость непрерывная.

3. Непрерывность следует из равномерной сходимости интеграла и непрерывности
подынтегральной функции по теореме о непрерывности несобственного интеграла,
зависящего от параметра. 

4. $\forall x>0~\forall s>0:x^{s-1}\cdot e^{-x}>0$ значит, 
$\Gamma(s)>0~\forall s$.

5. Имеем $\Gamma(s+1)=\int\limits_{0}^{\infty}x^{s}\cdot e^{-x}dx=
-\int\limits_{0}^{\infty}x^sd(e^{-x})=-x^{s}\cdot e^{-x}\Big|^\infty_0+
s \int\limits_{0}^{\infty}x^{s-1}\cdot e^{-x}$, откуда $\Gamma(s+1)=
s\cdot \Gamma(s)$.

6. $\Gamma(1)=\int\limits_{0}^{\infty}e^{-x}dx=1$. Факториальность следует 
по индукции из основного свойства. 

7. $\int\limits_{0}^{\infty}e^{-x^2}dx=\frac{\sqrt{\pi} }{2}$ - интеграл 
Пуассона. Значит, $\Gamma(\frac{1}{2})=\int\limits_{0}^{\infty}x^{-\frac{1}{2}}
e^{-x}dx=\int\limits_{0}^{\infty}e^{-x}\frac{dx}{\sqrt{x}}$. Заменим 
$x=t^2$, откуда  имеем $2 \int\limits_{0}^{\infty}e^{-t^2}dt=\sqrt{\pi}$.
Общая формула для полуцелых чисел следует по индукции из основного свойства.

8. По индукции.

9. Докажем, что 
$\Gamma'(s)=\int\limits_{0}^{\infty}x^{s-1}\cdot e^{-x}\ln x\,dx$. Для 
применения теоремы о дифференцировании надо доказать,
что этот интеграл равномерно сходится на $[s_1,s_2],0<s_1<s<s_2<\infty$. 
Рассмотрим
$\int\limits_{0}^{\infty} = \int\limits_{0}^{1} + \int\limits_{1}^{\infty}$.
В особой точке 0 при $s_1\geqslant1$ имеем
$|x^{s-1}\cdot e^{-x}\ln x|\leqslant 1\cdot 1\cdot |\ln x|=-\ln x$. 
Интеграл $-\int\limits_{0}^{1}\ln x\,dx=-x\ln x\big|^1_0+
\int\limits_{0}^{1}dx=1$ - сходится. Значит, гамма-функция сходится равномерно
на $[s_1,s_2]<1$ по признаку Вейерштрасса. Если же $s_1<1$, то
$|x^{s-1}\cdot e^{-x}\ln x|\leqslant x^{s_1-1}\cdot\ln x$. 
Правая часть сходится, и интеграл сходится по признаку Вейерштрасса.\\
Если $x>1$, то  $0<x^{s-1}\cdot e^{-x}\ln x<x^{s_2-1}e^{-x}\ln x
<e^{-\frac{x}{3}}$. Также сходится по Вейерштрассу. Поэтому в итоге он сходится
на объединении областей. Поэтому можно дифференцировать.

10. Доказательство слишком длинное и использует комплексные числа. И прочие 
тоже. 

\section{Бета-функция}
$$B(p,q)=\int\limits_{0}^{1} x^{p-1}(1-x)^{q-1}dx$$ 
1. Область определения - $p>0\And q>0$;\\
2. Равномерная сходимость -  $p\geqslant p_0>0\And q\geqslant q_0>0$.\\
3. $B(p,q)$ непрерывна на области определения\\
4. $B(p,q)>0$  на области определения\\
5. $B(p,q)=B(q,p)$\\ 
6. $B(p,q)=\frac{p-1}{p+q-1}B(p-1,q)=\frac{q-1}{p+q-1}B(p,q-1)$ \\
7. Несобственный интеграл как первого, так и второго рода:
$\int\limits_{0}^{\infty} \frac{t^{p-1}dt}{(1+t)^{p+q}}$\\
8. $\boxed{B(p,q)=\frac{\Gamma(p)\Gamma(q)}{\Gamma(p+q)}}$ \\
9. $B(p,1-p)=\frac{\pi}{\sin\pi n}$ \\
10. Формула Лежандра: $\boxed{B(p,p)=
\frac{1}{2^{2p-1}}B\left(\frac{1}{2},p\right)}$.
Или же: $\Gamma(p)\Gamma(p+\frac{1}{2})=\frac{\sqrt{\pi}}{2^{2p-1}}\Gamma(2p)$.

\textbf{Доказательство.}

1. Имеем  $B(p,q)=\int\limits_{0}^{\frac{1}{2}}x^{p-1}(1-x)^{q-1}dx+
\int\limits_{\frac{1}{2}}^{1}x^{p-1}(1-x)^{q-1}dx$. При $x\to 0$,
$x^{p-1}(1-x)^{q-1}\sim x^{p-1}$, и интеграл  $\int\limits_{0}^{\frac{1}{2}}
x^{p-1}dx$ сходится при $p>0$. При  $x\to 1$, 
$x^{p-1}(1-x)^{q-1}\sim(1-x)^{q-1}$, аналогично сходится при  $q>0$. 


2. Аналогично предыдущему пункту,
$x^{p-1}(1-x)^{q-1}\leqslant x^{p_0-1}(1-x)^(q_0-1)$ - 
сходится равномерно по признаку Вейерштрасса.

3. Не на что сослаться, так как не было предела от двух переменных.

4. Очевидно.

5. Введем замену $t=1-x$, и получим точно такой же интеграл. 

6.  $B(p,q)=\int\limits_{0}^{1}x^{p-1}(1-x)^{q-1}dx=
-\frac{1}{q}\int\limits_{0}^{1}x^{p-1}d((1-x)^{q-1})=
-\frac{1}{q}x^{p-1}(1-x)^{q-1}\big|^1_0+\frac{1}{q}\int\limits_{0}^{1}
(1-x)^{q}dx^{p-1}=\frac{p-1}{q}\int\limits_{0}^{1}x^{p-2}(1-x)^{q-1}(1-x)dx$.
Отсюда $q\cdot B(p,q)=(p-1)\left(
 \int\limits_{0}^{1}x^{p-2}(1-x)^{q-1}dx-
\int\limits_{0}^{1}x^{p-1}(1-x)^{q-1}dx\right)$. Значит,
$q\cdot B(p,q)=(p-1)\cdot B(p-1,q)-(p-1)\cdot B(p,q)$, и в итоге получаем
$B(p,q)=\frac{p-1}{p+q-1}B(p-1,q)$

7. Сделаем замену $x=\frac{t}{t+1}$. Изменим пределы: 
$0\to 0,1\to \infty$. И тогда $1-x=\frac{1}{t+1}$, $dx=\frac{dt}{(t+1)^2}$.
В итоге имеем $B(p,q)=\int\limits_{0}^{\infty} \frac{t^{p-1}}{(1+t)^{p-1}}
\cdot \frac{1}{(t+1)^{q-1}}\cdot \frac{dt}{(1+t^2)}=
\int\limits_{0}^{\infty} \frac{t^{p-1}dt}{(1+t)^{p+q}}$. 

Чтобы доказать следующее свойство бета-функции, нам потребуется следующая
\begin{theor}
    (о перестановке двух несобственных интегралов)\\
    Пусть\\
    1. $f(x,y)$ определена и непрерывна на $[a,\infty)\times [c,\infty)$;\\
    2. $\int\limits_{a}^{\infty}f(x,y)dx$ сходится равномерно на 
    $[c,d]~\forall d>c$;\\
    3. $\int\limits_{c}^{\infty}f(x,y)dy$ сходится равномерно на 
    $[a,b]~\forall b>a$;\\
    4. Существует $\int\limits_{a}^{\infty}dx \int\limits_{c}^{\infty}
    |f(x,y)|dy$ или  $\int\limits_{c}^{\infty}dy \int\limits_{a}^{\infty}
    |f(x,y)|dx$ ;\\
    Тогда существуют  оба интеграла 
    $\int\limits_{a}^{\infty}dx \int\limits_{c}^{\infty}f(x,y)dy$
    и  $\int\limits_{c}^{\infty}dy \int\limits_{a}^{\infty}f(x,y)dx$,
    и они равны между собой. 
\end{theor}
\textbf{Доказательство.} Допустим, существует интеграл  
$\int\limits_{a}^{\infty}dx \int\limits_{c}^{\infty}|f(x,y)|dx$.
Тогда 
$$\int\limits_{c}^{\infty}dy \int\limits_{a}^{\infty}f(x,y)dx=
\lim\limits_{d \to \infty} \int\limits_{c}^{d} dy \int\limits_{a}^{\infty} 
f(x,y)dx$$
По теореме об интегрировании интеграла, зависящего от параметра, это все равно
$$\lim\limits_{d \to \infty}\int\limits_{a}^{\infty}dx
\int\limits_{c}^{d}f(x,y)dy$$
Обозначим $\Phi(x,d)=\int\limits_{c}^{d}f(x,y)dy$. Применяя теорему о 
предельном переходе, 
$$|\Phi(x,d)|=\left| \int\limits_{c}^{d} f(x,y)dy \right|\leqslant 
\int\limits_{c}^{d} |f(x,y)|dy\leqslant \int\limits_{c}^{\infty}|f(x,y)|dy
$$
По условию, $\int\limits_{a}^{\infty}dx \int\limits_{c}^{\infty}|f(x,y)|dy$ 
сходится, поэтому $\int\limits_{a}^{\infty}\Phi(x,d)dx$ сходится 
равномерно по $d\in (c,\infty)$. Итак,
$$\lim\limits_{d \to \infty}\int\limits_{a}^{\infty}\Phi(x,d)dx=
\int\limits_{a}^{\infty} \left( \lim\limits_{d \to \infty}\Phi(x,d)\right)dx=
\int\limits_{a}^{\infty}\left( \int\limits_{c}^{\infty}f(x,y)dy \right)dx $$
$\square$ \\

8. Теперь докажем, что ${B(p,q)=\frac{\Gamma(p)\Gamma(q)}{\Gamma(p+q)}}$.\\
Случай 1: $p>1,q>1$.  %$\Gamma(s)=\int\limits_{0}^{\infty}x^$
Сделаем замену $x=(t+1)y,t>0,y>0$. Тогда
$\Gamma(s)=\int\limits_{0}^{\infty}(t-1)^{s-1}y^{s-1}e^{-(t+1)y}(t+1)dy$.
Тогда $\frac{\Gamma(s)}{(t+1)^s}=\int\limits_{0}^{\infty}y^{s-1}
e^{-(t+1)y}dy$. Пусть $S=p+q$. Имеем
$\frac{\Gamma(p+q)}{(t+1)^{p+q}}=\int\limits_{0}^{\infty}y^{p+q-1}e^{-(t+1)y}
dy$. Домножим на $t^{p-1}$:
$$\Gamma(p+q)\cdot \frac{t^{p-1}}{(t+1)^{p+q}}=t^{p-1}
\int\limits_{0}^{\infty}y^{p+q-1}e^{-(t+1)y}dy$$
Интегрируя, получаем
\begin{equation} \label{gamma_beta}
\Gamma(p+q)\cdot \int\limits_{0}^{\infty} \frac{t^{p-1}dt}{(t+1)^{p+q}}=
\int\limits_{0}^{\infty}t^{p-1}dt \int\limits_{0}^{\infty}y^{p+q-1}
e^{-(t+1)y}dy
\end{equation}

Внезапно, 
$\Gamma(p+q)\cdot \int\limits_{0}^{\infty} \frac{t^{p-1}dt}{(t+1)^{p+q}}=
\Gamma(p+q)\cdot B(p,q)$. По простому (нестрого): если поменять 
порядок интегрирования, то и получим $\Gamma(p+q)B(p,q)=
\Gamma(p)\cdot \Gamma(q)$. Более формально, мы должны проверять условия 
теоремы, доказанной выше. Давайте сделаем это (на отл):\\
1. $f(t,y)=t^{p-1}y^{p+q-1}e^{-(t+1)y}$ определена и непрерывна на 
$[0,\infty)\times [0,\infty)$.\\
2. $f(t,y)>0$ при  $t\geqslant 0,y\geqslant 0$.\\
По \ref{gamma_beta}, существует интеграл $\int\limits_{0}^{\infty}
dt \int\limits_{0}^{\infty}|f(t,y)|dy=\Gamma(p+q)B(p,q)$.
3. Покажем равномерную сходимость. $|f(t,y)|=t^{p-1}y^{p+q-1}
e^{-ty-y}\leqslant a^{p-1}y^{p+q-1}e^{-y}$. Значит, интеграл 
$\int\limits_{0}^{\infty}f(t,y)dy$ сходится равномерно на $t\in [0,\infty)$
по признаку Вейерштрасса. \\
4. То же самое для $\int\limits_{0}^{\infty}f(t,y)dt$. Здесь
нам нужна равномерная сходимость на $u\in [\xi,b]$. Если 
$0<\xi<y<b$, то  $|f(t,y)|=t^{p-1}y^{p+q-1}e^{-ty}e^{-y}\leqslant 
b^{p+q-1}t^{p-1}e^{-\xi}$. Интеграл от этой штуки сходится, тогда
$\int\limits_{0}^{\infty}f(t,y)dt$ сходится равномерно на $[\xi,b]$ по 
признаку Вейерштрасса. 

Итак, из \ref{gamma_beta} имеем  $\Phi(t,\xi)=\int\limits_{\xi}^{\infty} 
y^{p+q-1}e^{-(t+1)y}dy$, поэтому \ref{gamma_beta} перепишется в виде
сходящегося интеграла:
$$\int\limits_{0}^{\infty}t^{p-1}\Phi(t,0)dt=\Gamma(p+q)B(p,q)$$ 
Имеем $0\leqslant \Phi(t,0)-\Phi(t,\xi)=
\int\limits_{0}^{\xi}y^{p+q-1}e^{-(t+1)y}dy\to 0$ при $\xi\to 0$,
так как это интеграл с переменным верхним пределом, дифференцируема, 
значит, непрерывна. Оценим
$$\int\limits_{0}^{\infty}t^{p-1}\Phi(t,\xi)dt\leqslant 
\int\limits_{0}^{\infty} t^{p-1}\Phi(t,0)dt$$
- интеграл сходится, значит, по признаку Вейерштрасса интеграл сходится
равномерно по $x \in (0,\infty)$. Теперь делаем предельный переход:
$$\lim\limits_{\xi \to +0}\int\limits_{0}^{\infty}t^{p+1}\Phi(t,\xi)dt
=\int\limits_{0}^{\infty}t^{p+1}\Phi(t,0)dt$$

9. Докажем формулу Лежандра, начиная с левой части. Идея - 
выделить полный квадрат: $B(p,p)=$
$$=\int\limits_{0}^{1}x^{p-1}(1-x)^{p-1}dx=\int\limits_{0}^{1}
(x-x^2)^{p-1}dx=\int\limits_{0}^{1}\left( \frac{1}{4}-
\left( \frac{1}{4}-x+x^2 \right) \right)^{p-1}dx=$$
$$=\frac{1}{4^{p-1}}\int\limits_{0}^{1} (1-1(1-2x)^2)^{p-1}dx=
\begin{cases}t=1-2x\\dt=-2dx\end{cases}=\frac{1}{2^{2p-1}}=
\int\limits_{-1}^{1} (1-t^2)^{p-1}dt=
$$
$$=\begin{cases}
    u=t^2\\t=u^{\frac{1}{2}}\\dt=\frac{1}{2}u^{-\frac{1}{2}}du\end{cases}=
    \frac{1}{2^{2p-1}}\int\limits_{0}^{1}(1-u)^{p-1}u^{-\frac{1}{2}}du=
=\frac{1}{2^{2p-1}}B(\frac{1}{2},p)$$ 



\section{Примеры}
\textbf{Пример.} $\int\limits_{0}^{\infty} x^{2n+1}e^{-x^2}dx$. Замена:
$t=x^2$,  $I=\frac{1}{2}\int\limits_{0}^{\infty} t^{n+\frac{1}{2}}
t^{-\frac{1}{2}}e^{-t}dt=\Gamma(n+1)=\frac{1}{2}n!$ 

\textbf{Пример.} $\int\limits_{0}^{1} (\ln x)^ndx$. Берем $t=-\ln x$, откуда
$I=\int\limits_{0}^{\infty}(-t)^ne^{-t}dt=(-1)^nn!$

\textbf{Пример.} $\int\limits_{0}^{1}\sqrt{(1-x^2)^3}dx$. Положим
$B(p,q)=\int\limits_{0}^{1}x^{p-1}(1-x)^{q-1}dx$. Замена $t=x^2$, значит,
$I=-\frac{1}{2}\int\limits_{0}^{1}(1-t)^{\frac{3}{2}}t^{-\frac{1}{2}}dt=
\frac{1}{2}B(\frac{1}{2},\frac{5}{2})=\frac{1}{2}\cdot \frac{\frac{5}{2}-1}
{\frac{1}{2}+\frac{5}{2}-1}B(\frac{1}{2},\frac{3}{2})=...=
\frac{3\pi}{16}$ 

\textbf{Пример.} $\int\limits_{0}^{1} \frac{x^4dx}{\sqrt[3]t{(1-x^3)^2}}$,
Свести к бета-функции.

\textbf{Пример.} $\int\limits_{0}^{\infty}\frac{\sqrt[5]{x}dx}{(1+x)^2}$. 
Свести к бета-функции.

\textbf{Пример.} $\int\limits_{0}^{\frac{\pi}{2}}(\sin x)^p(\cos x)^qdx$.
Сведем к бета-функции заменой $t=\sin^2x,~dx=\frac{1}{2}\frac{dt}{
t^\frac{1}{2}(1-t)^\frac{1}{2}}$, откуда
$I=\frac{1}{2}B(\frac{p+1}{2},\frac{q+1}{2})$ 

\textbf{Пример.} $\int\limits_{0}^{\frac{\pi}{2}}(\tg x)^{\frac{1}{4}}dx$ - 
то же самое.

\textbf{Пример.} Вычислить площадь фигуры
$(x^2+y^2)^6=x^4y^2$. Перейдя в полярку  $x=r\cos\varphi,~y=r\sin\varphi$,
имеем $r^6=\cos^4\varphi\sin^2\varphi$.  Значит, интегрируя, 
получаем $S=\frac{1}{4}\int\limits_{0}^{\pi/2}r(\varphi)d\varphi$. 

\section{Заключительные вопросы}
\textbf{№22.} Теорема о перестановке двух несобственных интегралов\\
    Пусть\\
    1. $f(x,y)$ определена и непрерывна на $[a,\infty)\times [c,\infty)$;\\
    2. $\int\limits_{a}^{\infty}f(x,y)dx$ сходится равномерно на 
    $[c,d]~\forall d>c$;\\
    3. $\int\limits_{c}^{\infty}f(x,y)dy$ сходится равномерно на 
    $[a,b]~\forall b>a$;\\
    4. Существует $\int\limits_{a}^{\infty}dx \int\limits_{c}^{\infty}
    |f(x,y)|dy$ или  $\int\limits_{c}^{\infty}dy \int\limits_{a}^{\infty}
    |f(x,y)|dx$ ;\\
    Тогда существуют  оба интеграла 
    $\int\limits_{a}^{\infty}dx \int\limits_{c}^{\infty}f(x,y)dy$
    и  $\int\limits_{c}^{\infty}dy \int\limits_{a}^{\infty}f(x,y)dx$,
    и они равны между собой. 

\textbf{Доказательство.} Допустим, существует интеграл  
$\int\limits_{a}^{\infty}dx \int\limits_{c}^{\infty}|f(x,y)|dx$.
Тогда 
$$\int\limits_{c}^{\infty}dy \int\limits_{a}^{\infty}f(x,y)dx=
\lim\limits_{d \to \infty} \int\limits_{c}^{d} dy \int\limits_{a}^{\infty} 
f(x,y)dx$$
По теореме об интегрировании интеграла, зависящего от параметра, этот предел
равен
$$\lim\limits_{d \to \infty}\int\limits_{a}^{\infty}dx
\int\limits_{c}^{d}f(x,y)dy$$
Обозначим $\Phi(x,d)=\int\limits_{c}^{d}f(x,y)dy$. Применяя теорему о 
предельном переходе, 
$$|\Phi(x,d)|=\left| \int\limits_{c}^{d} f(x,y)dy \right|\leqslant 
\int\limits_{c}^{d} |f(x,y)|dy\leqslant \int\limits_{c}^{\infty}|f(x,y)|dy
$$
По условию, $\int\limits_{a}^{\infty}dx \int\limits_{c}^{\infty}|f(x,y)|dy$ 
сходится, поэтому $\int\limits_{a}^{\infty}\Phi(x,d)dx$ сходится 
равномерно по $d\in (c,\infty)$. Итак,
$$\lim\limits_{d \to \infty}\int\limits_{a}^{\infty}\Phi(x,d)dx=
\int\limits_{a}^{\infty} \left( \lim\limits_{d \to \infty}\Phi(x,d)\right)dx=
\int\limits_{a}^{\infty}\left( \int\limits_{c}^{\infty}f(x,y)dy \right)dx $$
$\square$ 

\textbf{№23.} Гамма-функция 
$$\Gamma(s)=\int\limits_{0}^{\infty}x^{s-1}\cdot e^{-x}dx$$
определена при $s>0$.\\
\textbf{Доказательство.} Рассмотрим сумму интегралов
$\Gamma(s)=\int\limits_{0}^{1}x^{s-1}\cdot e^{-x}dx+
\int\limits_{1}^{\infty}x^{s-1}\cdot e^{-x}dx$. 
Первый интеграл сходится при $s>0$ и расходится при $s\leqslant 0$
по предельному признаку сравнения с интегралом
$\int\limits_{0}^{1}x^{s-1}dx$. Второй интеграл: имеем
$$\forall s\in \mathbb{R}~\exists x(s)~\forall x>x(s):
x^{s-1}\leqslant e^{\frac{x}{2}}$$ 
Значит, $\int\limits_{x(s)}^{\infty}e^{-\frac{x}{2}}dx$ сходится, и исходный
интеграл сходится по признаку сравнения при любом $s$. Значит, область 
определения гамма-функции -  $s>0$. $\square$ 

\textbf{№24.} Гамма-функция Эйлера сходится равномерно на любом
отрезке положительной полуоси действительных чисел.\\
\textbf{Доказательство}. Докажем равномерную сходимость на $[s_1,s_2]$
по признаку Вейерштрасса. Получаем 
$x^{s-1}\cdot e^{-x}\leqslant x^{s-1}\leqslant x^{s_1-1}$ при фиксированном
$x\in [0,1]$. При этом интеграл $\int\limits_{0}^{1}x^{s_1-1}dx$ сходится, 
поэтому интеграл сходится на $[s_1,s_2]$ равномерно. Если $x\geqslant 1$,
то $x^{s-1}\cdot e^{-x}\leqslant x^{s_2-1}\cdot e^{-x}$, правая часть 
сходится и не зависит от $s$, значит, сходимость равномерная. Значит,
на объединении $1>s>0$ и  $s>0$ сходимость непрерывная.\\
Гамма-функция непрерывна при $s>0$. Непрерывность следует из равномерной 
сходимости интеграла и непрерывности
подынтегральной функции по теореме о непрерывности несобственного интеграла,
зависящего от параметра.

\textbf{№25.} Основное свойство - $\boxed{\Gamma(s+1)=s\Gamma(s)}$.\\
\textbf{Доказательство.} 
Имеем $$\Gamma(s+1)=\int\limits_{0}^{\infty}x^{s}\cdot e^{-x}dx=
-\int\limits_{0}^{\infty}x^sd(e^{-x})=-x^{s}\cdot e^{-x}\Big|^\infty_0+
s \int\limits_{0}^{\infty}x^{s-1}\cdot e^{-x}dx$$ откуда $\Gamma(s+1)=
s\cdot \Gamma(s)$.\\
Значение в единице и факториал: 
$\Gamma(1)=\int\limits_{0}^{\infty}e^{-x}dx=1$. Факториальность следует 
по индукции из основного свойства:  $\Gamma(1)=1,~\Gamma(n+1)=n!$ \\
Полуцелые числа:
$\int\limits_{0}^{\infty}e^{-x^2}dx=\frac{\sqrt{\pi} }{2}$ - интеграл 
Пуассона. Значит, $\Gamma(\frac{1}{2})=\int\limits_{0}^{\infty}x^{-\frac{1}{2}}
e^{-x}dx=\int\limits_{0}^{\infty}e^{-x}\frac{dx}{\sqrt{x}}$. Заменим 
$x=t^2$, откуда  имеем $2 \int\limits_{0}^{\infty}e^{-t^2}dt=\sqrt{\pi}$.
Общая формула для полуцелых чисел следует по индукции из основного свойства:
$\Gamma\left( \frac{1}{2} \right)=\sqrt{\pi}$, $\Gamma\left( 
n+\frac{1}{2}\right)=\frac{(2n-1)!!}{2^n}\cdot \sqrt{\pi}$.

\textbf{№26.} Формула для производной:
$\Gamma^{(n)}(s)=\int\limits_{0}^{\infty}x^{s-1}\cdot e^{-x}\ln^n x\,dx$.\\
\textbf{Доказательство.} Сначала докажем для первой производной:
$\Gamma'(s)=\int\limits_{0}^{\infty}x^{s-1}\cdot e^{-x}\ln x\,dx$. 
Заметим, что дифференцируя подынтегральное выражение, мы получим определение
гамма-функции. Значит, нам надо доказать применимость теоремы о 
дифференцировании несобственного интеграла. Для этого докажем 
равномерную сходимость на $[s_1,s_2],0<s_1<s<s_2<\infty$. 
Рассмотрим
$\int\limits_{0}^{\infty} = \int\limits_{0}^{1} + \int\limits_{1}^{\infty}$.
В особой точке 0 при $s_1\geqslant1$ имеем
$|x^{s-1}\cdot e^{-x}\ln x|\leqslant 1\cdot 1\cdot |\ln x|=-\ln x$. 
Интеграл $-\int\limits_{0}^{1}\ln x\,dx=-x\ln x\big|^1_0+
\int\limits_{0}^{1}dx=1$ - сходится. Значит, гамма-функция сходится равномерно
на $[s_1,s_2]<1$ по признаку Вейерштрасса. Если же $s_1<1$, то
$|x^{s-1}\cdot e^{-x}\ln x|\leqslant x^{s_1-1}\cdot\ln x$. 
Правая часть сходится, и интеграл сходится по признаку Вейерштрасса.\\
Если $x>1$, то  $0<x^{s-1}\cdot e^{-x}\ln x<x^{s_2-1}e^{-x}\ln x
<e^{-\frac{x}{3}}$. Также сходится по Вейерштрассу. Поэтому в итоге он 
сходится на объединении областей. Поэтому можно дифференцировать. 

\textbf{№27.} Основное определение - 
$B(p,q)=\int\limits_{0}^{1} x^{p-1}(1-x)^{q-1}dx$. 
Подстановка  $x=\cos\varphi$ приводит к определению через 
инетграл второго рода: $B(a,b)=2 \int\limits_{0}^{\pi/2}\cos^{2a-1}\varphi
\sin^{2b-1}\varphi\,d\varphi$. \\
\textbf{Область определения}: $p>0$ и  $q>0$.
Имеем  $B(p,q)=\int\limits_{0}^{\frac{1}{2}}x^{p-1}(1-x)^{q-1}dx+
\int\limits_{\frac{1}{2}}^{1}x^{p-1}(1-x)^{q-1}dx$. При $x\to 0$,
$x^{p-1}(1-x)^{q-1}\sim x^{p-1}$, и интеграл  $\int\limits_{0}^{\frac{1}{2}}
x^{p-1}dx$ сходится при $p>0$. При  $x\to 1$, 
$x^{p-1}(1-x)^{q-1}\sim(1-x)^{q-1}$, аналогично сходится при  $q>0$.\\
\textbf{Симметричность} $B(p,q)=B(q,p)$. Доказательство через замену
 $t=1-x$. \\
\textbf{Основное свойство:} 
$B(p,q)=\frac{p-1}{p+q-1}B(p-1,q)=\frac{q-1}{p+q-1}B(p,q-1)$.
Доказательство:
$B(p,q)=\int\limits_{0}^{1}x^{p-1}(1-x)^{q-1}dx=
-\frac{1}{q}\int\limits_{0}^{1}x^{p-1}d((1-x)^{q-1})=
-\frac{1}{q}x^{p-1}(1-x)^{q-1}\big|^1_0+\frac{1}{q}\int\limits_{0}^{1}
(1-x)^{q}dx^{p-1}=\frac{p-1}{q}\int\limits_{0}^{1}x^{p-2}(1-x)^{q-1}(1-x)dx$.
Отсюда $q\cdot B(p,q)=(p-1)\left(
 \int\limits_{0}^{1}x^{p-2}(1-x)^{q-1}dx-
\int\limits_{0}^{1}x^{p-1}(1-x)^{q-1}dx\right)$. Значит,
$q\cdot B(p,q)=(p-1)\cdot B(p-1,q)-(p-1)\cdot B(p,q)$, и в итоге получаем
$B(p,q)=\frac{p-1}{p+q-1}B(p-1,q)$. Аналогично, если внесем под
дифференциал $x^{p-1}$, получим  $B(p,q)=\frac{q-1}{p+q-1}B(p,q-1)$.\\
\textbf{Через несобсвтенный инетграл 1 рода:} замена
$x=\frac{y}{y+1}$: $B(a,b)=\int\limits_{0}^{\infty} \frac{y^{a-1}}
{(1+y)^{a+b}}dy$. 


\textbf{№28.}  
Докажем, что ${B(p,q)=\frac{\Gamma(p)\Gamma(q)}{\Gamma(p+q)}}$.\\
Случай 1: $p>1,q>1$.  %$\Gamma(s)=\int\limits_{0}^{\infty}x^$
Сделаем замену $x=(t+1)y,t>0,y>0$. Тогда
$\Gamma(s)=\int\limits_{0}^{\infty}(t+1)^{s-1}y^{s-1}e^{-(t+1)y}(t+1)dy$.
Тогда $\frac{\Gamma(s)}{(t+1)^s}=\int\limits_{0}^{\infty}y^{s-1}
e^{-(t+1)y}dy$. Пусть $S=p+q$. Имеем
$\frac{\Gamma(p+q)}{(t+1)^{p+q}}=\int\limits_{0}^{\infty}y^{p+q-1}e^{-(t+1)y}
dy$. Домножим на $t^{p-1}$:
$$\Gamma(p+q)\cdot \frac{t^{p-1}}{(t+1)^{p+q}}=t^{p-1}
\int\limits_{0}^{\infty}y^{p+q-1}e^{-(t+1)y}dy$$
Интегрируя, получаем
\begin{equation} \label{gamma_beta}
\Gamma(p+q)\cdot \int\limits_{0}^{\infty} \frac{t^{p-1}dt}{(t+1)^{p+q}}=
\int\limits_{0}^{\infty}t^{p-1}dt \int\limits_{0}^{\infty}y^{p+q-1}
e^{-(t+1)y}dy
\end{equation}

Внезапно, 
$\Gamma(p+q)\cdot \int\limits_{0}^{\infty} \frac{t^{p-1}dt}{(t+1)^{p+q}}=
\Gamma(p+q)\cdot B(p,q)$. По простому (нестрого): если поменять 
порядок интегрирования, то и получим $\Gamma(p+q)B(p,q)=
\Gamma(p)\cdot \Gamma(q)$. Более формально, мы должны проверять условия 
теоремы, доказанной выше. Давайте сделаем это (на отл):\\
1. $f(t,y)=t^{p-1}y^{p+q-1}e^{-(t+1)y}$ определена и непрерывна на 
$[0,\infty)\times [0,\infty)$.\\
2. $f(t,y)>0$ при  $t\geqslant 0,y\geqslant 0$.\\
По \ref{gamma_beta}, существует интеграл $\int\limits_{0}^{\infty}
dt \int\limits_{0}^{\infty}|f(t,y)|dy=\Gamma(p+q)B(p,q)$.
3. Покажем равномерную сходимость. $|f(t,y)|=t^{p-1}y^{p+q-1}
e^{-ty-y}\leqslant a^{p-1}y^{p+q-1}e^{-y}$. Значит, интеграл 
$\int\limits_{0}^{\infty}f(t,y)dy$ сходится равномерно на $t\in [0,\infty)$
по признаку Вейерштрасса. \\
4. То же самое для $\int\limits_{0}^{\infty}f(t,y)dt$. Здесь
нам нужна равномерная сходимость на $u\in [\xi,b]$. Если 
$0<\xi<y<b$, то  $|f(t,y)|=t^{p-1}y^{p+q-1}e^{-ty}e^{-y}\leqslant 
b^{p+q-1}t^{p-1}e^{-\xi}$. Интеграл от этой штуки сходится, тогда
$\int\limits_{0}^{\infty}f(t,y)dt$ сходится равномерно на $[\xi,b]$ по 
признаку Вейерштрасса. 

Итак, из \ref{gamma_beta} имеем  $\Phi(t,\xi)=\int\limits_{\xi}^{\infty} 
y^{p+q-1}e^{-(t+1)y}dy$, поэтому \ref{gamma_beta} перепишется в виде
сходящегося интеграла:
$$\int\limits_{0}^{\infty}t^{p-1}\Phi(t,0)dt=\Gamma(p+q)B(p,q)$$ 
Имеем $0\leqslant \Phi(t,0)-\Phi(t,\xi)=
\int\limits_{0}^{\xi}y^{p+q-1}e^{-(t+1)y}dy\to 0$ при $\xi\to 0$,
так как это интеграл с переменным верхним пределом, дифференцируема, 
значит, непрерывна. Оценим
$$\int\limits_{0}^{\infty}t^{p-1}\Phi(t,\xi)dt\leqslant 
\int\limits_{0}^{\infty} t^{p-1}\Phi(t,0)dt$$
- интеграл сходится, значит, по признаку Вейерштрасса интеграл сходится
равномерно по $x \in (0,\infty)$. Теперь делаем предельный переход:
$$\lim\limits_{\xi \to +0}\int\limits_{0}^{\infty}t^{p+1}\Phi(t,\xi)dt
=\int\limits_{0}^{\infty}t^{p+1}\Phi(t,0)dt$$














\end{document}
