\section{Функциональные последовательности}
\subsection{Базовые определения}
\begin{defin}
Пусть функции $f_1,f_2...$ заданы на некотором множестве 
$X\subset \mathbb{R}$. Тогда задана функциональная последовательность
$$f_1(x),f_2(x),f_3(x)...~,x\in X$$
\end{defin}
\begin{defin}
    Функциональная последовательность $f_n$ сходится \textbf{поточечно}
    к $f(x)$ в точке  $x_0$, если
$$\forall \varepsilon>0~\exists N(\varepsilon)~\forall n>N:
|f_n(x_0)-f(x_0)|<\varepsilon$$
\end{defin}
Множество всех $x_0\in X$, для которых
предел существует, называется областью сходимости последовательности. 
\begin{defin}
Функциональная последовательность $f_n$ \textbf{равномерно сходится} к 
функции $f$ на множестве $X$, если 
$$\forall \varepsilon>0~\exists N(\varepsilon)~\forall n>N~\forall x\in X:
|f_n(x)-f(x)|<\varepsilon$$
Обозначается как  $f_n\rightrightarrows f$
\end{defin}
Это определение эквивалентно \textbf{супремум-критерию сходимости}:
$$\lim\limits_{n \to \infty} \sup\limits_{x\in X}|f_n(x)-f(x)|=0$$
\begin{theor}
    Супремум-критерий эквивалентен определению равномерной сходимости.
\end{theor}
\textbf{Доказательство.} Пусть $f_n\rightrightarrows f$ на $X$.
Фиксируем $\varepsilon>0$. Тогда по определению для
$$\frac{\varepsilon}{2}>0~\exists n_0~\forall n>n_0~\forall x\in X:
|f_n(x)-f(x)|<\frac{\varepsilon}{2}$$
Пусть $a_n=\sup|f_n(x)-f(x)|\leqslant \frac{\varepsilon}{2}<\varepsilon$.
Тогда
$$\forall \varepsilon>0~\exists n_0~\forall n>n_0:|a_n|<\varepsilon
\implies \lim\limits_{n \to \infty} a_n=0$$
Обратно, пусть 
$\lim\limits_{n \to \infty} \sup\limits_{x\in X} |f_n(x)-f(x)|=0$. 
Тогда
$$\forall \varepsilon>0~\exists n_0~\forall n>n_0:\sup|f_n(x)-f(x)|
<\varepsilon$$
Значит, выполняется неравенство 
$$\forall x\in X:|f_n(x)-f(x)|\leqslant \sup|f_n(x)-f(x)|$$ 
откуда
$$\forall \varepsilon>0~\exists n_0~\forall n>n_0~\forall x\in X:
|f_n(x)-f(x)|<\varepsilon\iff f_n\rightrightarrows f$$
$\square$ 
\begin{theor}
    (критерий Коши равномерной сходимости)\\
    $f_n(x)\rightrightarrows f(x)$ на множестве  $X$ тогда и только тогда,
    когда
$$\forall \varepsilon>0~\exists n_0(\varepsilon)\in\mathbb{N}~\forall n>n_0~
    \forall p\in\mathbb{N}~\forall x\in X: |f_{n+p}(x)-f_n(x)|<\varepsilon$$

\end{theor}
\textbf{Доказательство.} Зафиксируем $\varepsilon>0$. По условию,
$f_n(x)\rightrightarrows f$ на $X$. Тогда для  
$$\frac{\varepsilon}{2}>0~\exists n_0\in\mathbb{N}~\forall n>n_0~\forall p\in
\mathbb{N}~\forall x\in X:|f_{n}(x)-f(x)|<\frac{\varepsilon}{2}$$
Тогда $\forall p\in \mathbb{N}~\forall n+p>n>n_0:|f_{n+p}(x)-f_{n}(x)|
\leqslant |f_{n+p}(x)-f(x)|+|f(x)-f_{n}(x)|<\frac{\varepsilon}{2}
+\frac{\varepsilon}{2}=\varepsilon$.

Обратно, по условию для
$$\frac{\varepsilon}{2}>0~\exists n_0\in\mathbb{N}~\forall n>n_0~\forall p\in
\mathbb{N}~\forall x\in X:|f_{n+p}(x)-f_{n}(x)|<\frac{\varepsilon}{2}
\quad(1)$$
Сходимость имеет место для каждой отдельной точки $x\in X$ (поточечная 
сходимость), тогда по критерию Коши для числовой последовательности 
сущетсвует конечный предел  $\lim\limits_{n \to \infty} f_n(x)=f(x)\in
\mathbb{R}$. Тогда в неравенстве $(1)$ можно перейти к пределу:
 $\lim\limits_{n \to \infty} |f(x)-f_n(x)|<\frac{\varepsilon}{2}<
 \varepsilon$. $\square$\\
\textbf{Следствие (метод граничной точки).} Если $f_n(x)\in C[a,b)$ и 
$f_n(x)\to f(x)~\forall x\in(a,b)$, и $f_n(a)$ расходится. Тогда  $f_n(x)$
не сходится равномерно к  $f(x)$ на  $(a,b)$.\\
\textbf{Доказательство.} Допустим, что сходимость равномерная. Тогда 
$$\forall \varepsilon>0~\exists n_0\in\mathbb{N}~\forall n>n_0~\forall p\in
\mathbb{N}~\forall x\in X:|f_{n+p}(x)-f_{n}(x)|<\frac{\varepsilon}{2}$$
По условию, функции $f_n$ непрерывны на $[a,b)$, тогда 
$\lim\limits_{x \to a+0}|f_{n+p}(x)-f_n(x)|\leqslant \frac{\varepsilon}{2}$,
откуда $|f_{n+p}(a)-f_n(a)|<\varepsilon$. По критерию Коши для числовой
последовательности $f_n(a)$ сходится. Но это противоречит условию. 
$\square$
%че топроисходит
%\textbf{Пример.} $f_n(x)=n^{x+1}e^{-nx},~x>0$.
\begin{theor}
    (метод оценки остатка функциональной последовательности)\\
    Пусть $f_n(x)\to f(x)$ на множестве  $E$ и $r_n(x)=f_n(x)-f(x)$ - 
    остаток ряда. Тогда:\\
    1. Если  $\forall x\in E:|r_n(x)|\leqslant b_n\to 0$, то
    ф.п. сходится равномерно.\\
    2. Если $\exists X_n\in E:r_n(x_n)\to a\ne 0$, то ф.п. не сходится
    равомерно. 
\end{theor}
\textbf{Доказательство.} 
1. По условию, $\forall x\in E:|r_n(x)|\leqslant b_n$, тогда
$\sup\limits_{x\in E}|r_n(x)|\leqslant b_n$. Отсюда получаем, что
$$0\leqslant \lim\limits_{n \to \infty} \sup\limits_{x\in E}\leqslant 
\lim\limits_{n \to \infty} b_n=0$$
то есть функция сходится равномерно на множестве $E$.\\
2. Пусть $\exists X_n\in E:r_n(x_n)\to a\ne 0$. Но это эквивалентно 
отрицанию супремум-критерия, поэтому последовательность не сходится 
равномерно. $\square$ 
\begin{theor}
    (о пределе равномерно сходящейся функциональной последовательности)\\
    Пусть функции $f_n$ непрерывны на  $X$ и  $f_n\rightrightarrows f$.
    Тогда $f$ непрерывна на  $X$. 
\end{theor}
\textbf{Доказательство.} По определению, непрерывность на $X$  
$\Leftrightarrow$ непрерывность в каждой точке $x\in X$. Рассмотрим
 $x_0$ и зафиксируем  $\varepsilon>0$. По условию имеется равномерная 
 сходимость, тогда для 
$$\frac{\varepsilon}{3}>0~\exists n_0\in \mathbb{N}~\forall n>n_0~
\forall x\in X:|f_n(x)-f(x)|<\frac{\varepsilon}{3}$$ 
Пусть $n>n_0$, функция  $f_n(x)$ непрерывна в точке  $x_0$, тогда для
$$\frac{\varepsilon}{3}>0~\exists \delta>0~\forall x\in X:0\leqslant 
|x-x_0|<\delta\implies|f_n(x)-f_n(x_0)|<\frac{\varepsilon}{3}$$
Оценим эту разницу: $|f(x)-f(x_0)|=|f(x)-f_n(x)+f_n(x)-f_n(x_0)+f_n(x_0)-
f(x_0)|\leqslant|f(x)-f_n(x)|+|f_n(x)-f_n(x_0)|+|f_n(x_0)-f(x_0)|<
\frac{\varepsilon}{3}+\frac{\varepsilon}{3}+\frac{\varepsilon}{3}$.
Значит, $$\forall \varepsilon>0~\exists \delta>0:|x-x_0|<\delta\implies
|f(x)-f(x_0)|<\varepsilon$$
то есть функция непрерывна в точке $x_0$. $\square$

\subsection{Свойства равномерно сходящихся функциональных последовательностей}
\begin{enumerate}
    \item Линейные комбинации сходятся с соответствующим линейным комбинациям
        пределов. 
    \item Умножение на ограниченную (на $X$) функцию:  $(gf_n)\rightrightarrows
        (gf)$
    \item На любом подмножестве $X$ функция равномерно сходится. 
    \item Если  $\forall x\in X: f_n(x)\to f(x)$ и $E\subset X$ - конечное 
        множество, то на $E$ $f_n(x)\rightrightarrows f(x)$ - 
        функция сходится равномерно.
\item Последовательность, равномерно сходящаяся на двух множествах, равномерно
        сходится на их объединении.
\end{enumerate}
\textbf{Доказательство.}\\
1. Докажем для линейной комбинации $\alpha f+\beta g$. 
Зафиксируем $\varepsilon>0$. По условию $f_n\rightrightarrows f$ на $X$,
тогда для  $\frac{\varepsilon}{|\alpha|+|\beta|}>0~\exists n_1\in\mathbb{N}~
\forall x\in X:|f_n(x)-f(x)|<\frac{\varepsilon}{|\alpha|+|\beta|}$. 
Тоже самое для функции $g$. Для неё существует константа $n_2$. Выбрав 
максимум из них, получаем
$$|(\alpha f_n(x)+\beta g_n(x))-(\alpha f+\beta g)|\leqslant |\alpha|\cdot 
|f_n(x)-f(x)|+|\beta|\cdot |g_n(x)-g(x)|<\varepsilon$$.
Значит, $\alpha f_n+\beta g_n\rightrightarrows\alpha f+\beta g$.\\
2. По условию, $g(x)$ ограничена на  $X$. Значит, существует такое $M$, что
$\forall x\in X:|g(x)|<M$. Тогда для
$$\frac{\varepsilon}{M}>0~\exists n_0\in \mathbb{N}~\forall x\in X:
|f_n(x)-f(x)|<\frac{\varepsilon}{M}$$ 
Отсюда $|g(x)f_n(x)-g(x)f(x)|=|g(x)|\cdot |f_n(x)-f(x)|\leqslant M\cdot 
\frac{\varepsilon}{M}=\varepsilon$.\\
3. Допустим, функция не сходится равномерно
на $E\subset X$. Тогда существует такая 
последовательность $\{x_n\}\subset E$, что $\lim\limits_{n \to \infty} 
|f_n(x_n)-f(x_n)|>0$. Но поскольку $\{x_n\}\subset E\subset X$, функция 
не сходится равномерно и на $X$, что противоречит условию.\\
4. Так как у любого конечного множества 
есть супремум, то найдется такой $x_0\in E$, что на нем достигается
супремум предела $|f_n(x_0)-f(x_0)|$. Так как по условию $f_n(x_0)\to f(x_0)$,
то по супремум-критерию последовательность сходится равномерно.\\
5. Используем супремум-критерий. 
Допустим, последовательность равномерно не сходится на объединении.
Тогда 
$$\exists \{x_n\}\subset A\cup B:
\lim\limits_{n \to \infty}|f_n(x_n)-f(x_n)|=c\ne0$$
Если последовательность $\{x_n\} $ целиком лежит в одном из двух множеств
(или если лишь конечное число членов лежит в другом множестве),
тогда $f_n$ не сходится равномерно на этом множестве, что противоречит 
условию. Допустим, последовательность разбивается на две 
подпоследовательности $\{x^a_n\}\subset A$ и $\{x^b_n\}\subset B$. 
По определению, на них достигается супремум величины $|f_n-f|$. 
Рассмотрим сумму пределов
$\lim\limits_{n \to \infty}|f_n(x^a_n)-f(x^a_n)|+
\lim\limits_{n \to \infty}|f_n(x^b_n)-f(x^b_n)|=c_a+c_b$. По свойству сумм
пределов $c_a+c_b=c$, но тогда  $c_a\ne 0$, либо  $c_b\ne 0$, что 
противоречит пред положению согласно супремум-критерию.
\section{Функциональные ряды}
\subsection{Базовые определения}
\begin{defin}
Область $X\subset D$ сходимости  ряда $\sum\limits_{n=1}^{\infty} a_n(x)$ -
область, лежащая в области определения всех функций ряда и для каждого
$x$ на ней последовательность частичных сумм ряда сходится поточечно.
\end{defin}
\textbf{Пример.} $\sum\limits_{n=1}^{\infty} \frac{8^n}{n}(\sin x)^{3n}$.
Область сходимости - $|\sin x|<\frac{1}{2}$.
\begin{defin}
Ряд $\sum\limits_{n=1}^{\infty} a_n(x)$ сходится равномерно к $S(x)$ на  $X$,
если  $S_n\rightrightarrows S$ на  $X$ ($S_n$ - частичная сумма ряда).
\end{defin}
\textbf{Пример.} Исследуем на равномерную сходимость ряд $\sum\limits_{n=1}
^{\infty} \frac{x(2n-1)}{((n-1)^2+x^2)(n^2+x^2)},~x\in[1,\infty)$. Здесь
предел частичных сумм можно найти по определению: $S_n(x)=\sum\limits_{n=1}
^{\infty} a_k=x(\frac{1}{x^2}-\frac{1}{n^2+x^2})$. При фиксированном
$x\in D: \lim\limits_{n \to \infty} S_n(x)=\frac{1}{x},~S(x)=\frac{1}{x}$.
Проверим, что остаток равномерно стремится к нулю (тогда это верно 
и для суммы): $R_n(x)=S(x)-S_n(x)=\frac{x}{n^2+x^2}\leqslant \frac{x}{2nx}=
\frac{1}{2n}\to0,~n\to\infty$ (по методу оценки остатка). Итак, ряд сходится
равномерно к своей сумме.\\
\textbf{Пример.} Исследуем на равномерную сходимость $\sum\limits_{n=1}
^{\infty} \frac{x^2(2n-1)}{((n-1)^2+x^2)(n^2+x^2)},~x\in[1,\infty)$. Имеем 
$S_n(x)=1-\frac{x^2}{n^2+x^2},~S(x)=1$ 
\begin{theor}
    (необходимое условие равномерной сходимости функционального ряда)\\
    Ряд $\sum\limits_{n=1}^{\infty} a_n(x)$ сходится равномерно к $S(x)$ на
     $X$. Тогда  $a_n\rightrightarrows0$ на $X$. 
\end{theor}
\textbf{Доказательство.}  По условию, 
$$\forall \varepsilon>0~\exists n_0\in\mathbb{N}~\forall n>n_0~\forall x\in X: |S_n(x)-S(x)|<\frac{\varepsilon}{2}\implies|S_{n+1}(x)-S(x)|<
\frac{\varepsilon}{2}$$
Отсюда имеем 
$$|a_n(x)|=|S_{n+1}(x)-S_n(x)|\leqslant |S_{n+1}(x)-S(x)|+|S(x)-S_n(x)|<
\frac{\varepsilon}{2}+\frac{\varepsilon}{2}=\varepsilon$$
Значит, $a_n\rightrightarrows0$ на $X$. $\square$ 
\begin{theor}
    (критерий Коши равномерной сходимости функционального ряда)\\
    $\sum\limits_{n=1}^{\infty} a_n(x)$ равномерно сходится на Х к $S(x)$
    тогда и только тогда, когда 
    $$\forall \varepsilon>0~\exists n_0(\varepsilon)\in\mathbb{N}
    ~\forall n>n_0~\forall p\in\mathbb{N}~\forall x\in X:|\sum\limits_{k=n+1}
    ^{n+p}a_k(x)|<\varepsilon$$
\end{theor}
\textbf{Доказательство.} Применим определение Коши равномерной сходимости для
последовательности. $\square$ \\
\textbf{Пример.} Докажем, что у ряда $\sum\limits_{n=1}^{\infty} \frac{\sqrt{
x} }{n^2x^2+\sqrt{n} }$, $x\in(0,1)$ нет равномерной сходимости. Возьмем 
$x=\frac{1}{2n}$; $a_k(x)\geqslant \frac{1}{4n}$. Поэтому для $\varepsilon
\geqslant \frac{1}{4}$ по критерию Коши ряд расходится.
\begin{theor}
    (метод граничной точки)\\
    Пусть дан ряд $\sum\limits_{n=1}^{\infty} a_n(x)$, его члены непрерывны
    на отрезке $[a,b]$ и ряд сходится на интервале  $(a,b)$, но расходится
    на каком-либо конце интервала. Тогда равномерной сходимости нет. 
\end{theor}
\textbf{Доказательство.}  Повторяет доказательство для последовательностей.
$\square$ \\ 
\textbf{Пример.} $\sum\limits_{n=1}^{\infty} \frac{1}{n^x},~x\in(1,2)$. Ряд
сходится на интрвале как обобщенный гармонический ряд. При $x=1$ ряд 
расходится, значит, равномерной сходимости нет.
 \begin{theor}
     (признак Вейерштрасса равномерной сходимости функционального ряда
     /мажорантный признак)\\
     Пусть дан ряд с общим членом $a_n(x)$ и мы можем оценить 
    $|a_n(x)|\leqslant  a_n$ (то есть мажорирующим рядом, не зависящим от $x$),
причем $\sum\limits_{n=1}^{\infty}a_n$ сходится.
Тогда ряд $\sum\limits_{n=1}^{\infty} a_n(x)$ сходится равномерно на том
множестве, на котором верна оценка. 
\end{theor}
\textbf{Доказательство.}  Испоьзуем критерий Коши: фиксируем 
$\varepsilon>0$. Ряд $a_n$ сходится, значит, по критерию Коши
$$\forall \varepsilon>0~\exists n_0(\varepsilon)\in\mathbb{N}~\forall n>n_0~
 \forall p\in\mathbb{N}: \sum\limits_{k=n+1}^{n+p} a_k<\varepsilon$$
По условию, $\forall x\in X~\forall n\in\mathbb{N}: |a_n(x)|\leqslant a_k$,
значит 
$$\left|\sum\limits_{k=n+1}^{n+p} a_k(x)\right|\leqslant 
\sum\limits_{k=n+1}^{n+p}|a_k(x)|\leqslant \sum\limits_{k=n+1}^{n+p}
a_k<\varepsilon$$. Тогда 
по критерию Коши для функционального ряда следует равномерная сходимость.
$\square$ 

\textbf{Пример.} Исследуем на равномерную сходимость $\sum\limits_{n=1}^
{\infty} \frac{arcctg(nx)}{n},~x\in(\varepsilon,\infty),~\varepsilon>0$.
Подставив ноль, по методу граничной точки нет равномерной сходимости.

\textbf{Пример.} Исследуем сходимость $\sum\limits_{n=1}^{\infty} e^{-n^5x^2}
\sin{nx}$ на прямой. Спойлер: сходится равномерно. Сделаем оценку:  
$|a_n(x)|\leqslant e^{-n^5x^2}n|x|$. Функция симметрична при замене $x\mapsto
-x$, значит, будем оценивать на положительном луче, откинув модуль. Оценим
максимумом, вычислив производную и решив уравнение. Имеем  $x=\frac{1}
{\sqrt{2n^5}}$. Подставляем: $f_n(x)\leqslant f(\frac{1}{\sqrt{2n^5} })=
\frac{1}{\sqrt{2e}n^{\frac{3}{2}}}=a_n$. Значит, $|a_n(x)|\leqslant |f_n(x)|
\leqslant a_n~\forall x\in\mathbb{R}$. Итак, сходимость равномерная. 
\begin{theor}
(признак Дирихле равномерной сходимости функционального ряда)\\
Пусть дан ряд $\sum\limits_{n=1}^{\infty} a_n(x)b_n(x)$ и\\
1. $\forall x\in X: \{a_n(x)\}$ монотонна по $n$;\\
2.  $\exists M=const~ \forall x\in X~\forall n\in N:|B_n(x)|\leqslant M$,
где $B_n(x)$ - частичные суммы ряда  $b_n$.\\
Тогда ряд сходится равномерно на $X$.
\end{theor}
\textbf{Доказательство.} Фиксируем $\varepsilon>0$. Так как 
$a_n\rightrightarrows0$ на $X$, то для  
$$\frac{\varepsilon}{6B}>0~\exists n_0\in \mathbb{N}~\forall n>n_0~
\forall x\in X:|a_n(x)|<\frac{\varepsilon}{6B}$$
Из пункта 3 условия имеем
$$\left| \sum\limits_{k=n+1}^{n+i}b_k(x) \right|=|B_{n+i}(x)-B_n(x)|
\leqslant |B_{n+1}(x)|+|B_n(x)|\leqslant 2B$$
По неравенству Абеля получаем
$$\forall x\in X:\left| \sum\limits_{k=n+1}^{n+p} a_k(x)b_k(x) \right| 
\leqslant 2B(|a_{n+1}(x)|+2|a_{n+p}(x)|)<\varepsilon$$
Значит, исходный ряд сходится равномерно по критерию Коши. $\square$ 

\textbf{Пример.} $\sum\limits_{n=1}^{\infty} \sin{nx}/n$. Исследовать на 
равномерную сходимость на интервалах $(\varepsilon,2\pi-\varepsilon)$,
$(0,2\pi)$. Ну, раз говорят что уже было, то не пишем.
На втором интервале нет равномерной сходимости по краевому критерию. 



 


