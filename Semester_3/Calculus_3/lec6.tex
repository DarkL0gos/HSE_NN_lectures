\section{Равномерная сходимсоть функциональных рядов}
\begin{theor}
    (критерий Коши равномерной сходимости)\\
    $f_n(x)\rightrightarrows f(x)$ на множестве  $X$  $\Leftrightarrow$ 
    $\forall \varepsilon>0\exists n_0(\varepsilon)\in\mathbb{N}\forall n>n_0
    \forall p\in\mathbb{N} \forall x\in X: |f_{n+p}(x)-f_n(x)|<\varepsilon$

\end{theor}
\textbf{Доказательство.} 1. Зафиксируем $\varepsilon>0$. По условию,
$f_n(x)\rightrightarrows f$ на $X$. Тогда для  $\frac{\varepsilon}{2}>0$ 



$\square$ \\
\textbf{Следствие (метод граничной точки).} Если $f_n(x)\in[a,b)$ и 
$f_n(x)\to f(x)~\forall x\in_9a,b$, $f_n(a)$ расходится. Тогда  $f_n(x)$
не сходится равномерно к  $f(x)$ на  $(a,b)$.\\
\textbf{Доказательство.} Допустим, что сходимсоть равномерная. Тогда 
че топроисходит\\
\textbf{Пример.} $f_n(x)=n^{x+1}e^{-nx},~x>0$.\\
\subsection{Свойства равномерно сходящихся ф. п.}
\begin{enumerate}
    \item Линейные комбинации сходятся с соответствующим линейным комбинациям
        пределов. 
    \item Умножение на ограниченную на $X$ функцию:  $(gf_n)\rightrightarrows
        (gf)$
    \item На любом подмножестве $X$ функция равномерно сходится. 
    \item Если  $\forall x\in x: f_n(x)\to f(x)$ и $E\subset X$ - конечное 
        множество, то на $E$ функция сходится равномерно.
    \item Функция, равномерно сходящаяся на двух множествах, равномерно
        сходится на их объединении.
\end{enumerate}
\textbf{Доказательство.} 


\section{Функциональные ряды.}
\begin{defin}
Область $X\subset D$ сходимости  ряда $\sum\limits_{n=1}^{\infty} a_n(x)$ -
область, лежащая в области определения всех функций ряда и для каждого
$x$ на ней ряд сходится.
\end{defin}
\textbf{Пример.} $\sum\limits_{n=1}^{\infty} \frac{8^n}{n}(\sin x)^{3n}$.
Область сходимости - $|\sin x|<\frac{1}{2}$.
\begin{defin}
Ряд $\sum\limits_{n=1}^{\infty} a_n(x)$ сходится равномерно к $S(x)$ на  $X$,
если  $S_n\rightrightarrows S$ на  $X$ ($S_n$ - частичная сумма ряда).
\end{defin}
\textbf{Пример.} Исследуем на равномерную сходимость ряд $\sum\limits_{n=1}
^{\infty} \frac{x(2n-1)}{((n-1)^2+x^2)(n^2+x^2)},~x\in[1,\infty)$. Здесь
предел частичных сумм можно найти по определению: $S_n(x)=\sum\limits_{n=1}
^{\infty} a_k=x(\frac{1}{x^2}-\frac{1}{n^2+x^2})$. При фиксированном
$x\in D: \lim\limits_{n \to \infty} S_n(x)=\frac{1}{x},~S(x)=\frac{1}{x}$.
Проверим, что остаток равномерно стремится к нулю (тогда это верно 
и для суммы): $R_n(x)=S(x)-S_n(x)=\frac{x}{n^2+x^2}\leqslant \frac{x}{2nx}=
\frac{1}{2n}\to0,~n\to\infty$ (по методу оценки остатка). Итак, ряд сходится
равномерно к своей сумме.\\
\textbf{Пример.} Исследуем на равномерную сходимость $\sum\limits_{n=1}
^{\infty} \frac{x^2(2n-1)}{((n-1)^2+x^2)(n^2+x^2)},~x\in[1,\infty)$. Имеем 
$S_n(x)=1-\frac{x^2}{n^2+x^2},~S(x)=1$ 
\begin{theor}
    (необходимое условие равномерной сходимости ф.р)\\
    Ряд $\sum\limits_{n=1}^{\infty} a_n(x)$ сходится равномерно к $S(x)$ на
     $X$. Тогда  $a_n\rightrightarrows0$ на $X$. 
\end{theor}
\textbf{Доказательство.}  По условию, $\forall \varepsilon>0\exists n_0\in
\mathbb{N}\forall n>n_0\forall x\in X: |S_n(x)-S(x)|<\frac{\varepsilon}{2}$
$\square$ 
\begin{theor}
    (критерий Коши равномерной сходимости функционального ряда)\\
    $\sum\limits_{n=1}^{\infty} a_n(x)$ равномерно сходится на Х к $S(x)$
    тогда и только тогда, когда последовательность частичных сумм равномерно
    сходится:  $\forall \varepsilon>0~\exists n_0(\varepsilon)\in\mathbb{N}
    ~\forall n>n_0~\forall p\in\mathbb{N}~\forall x\in X:|\sum\limits_{k=n+1}
    ^{n+p}a_k(x)|<\varepsilon$
\end{theor}
\textbf{Доказательство.} Прсото применим определение Коши сходимости.  
$\square$ \\
\textbf{Пример.} Докажем, что у ряда $\sum\limits_{n=1}^{\infty} \frac{\sqrt{
x} }{n^2x^2+\sqrt{n} }$, $x\in(0,1)$ нет равномерной сходимости. Возьмем 
$x=\frac{1}{2n}$; $a_k(x)\geqslant \frac{1}{4n}$. Поэтому для $\varepsilon
\geqslant \frac{1}{4}$ по критерию Коши ряд расходится.\\
\begin{theor}
    (метод граничной точки)\\
    пусть дан ряд $\sum\limits_{n=1}^{\infty} a_n(x)$, его члены непрерывны
    на отрезке $[a,b]$ и ряд сходится на интервале  $(a,b)$, но расходится
    на конце интервала. Тогда равномерной сходимости нет. 
\end{theor}
\textbf{Доказательство.}  Повторяет доказательство для последовательностей.
$\square$ \\ 
\textbf{Пример.} $\sum\limits_{n=1}^{\infty} \frac{1}{n^x},~x\in(1,2)$. Ряд
сходится на интрвале как обобщенный гармонический ряд. При $x=1$ ряд 
расходится, значит, равномерной сходимости нет.
 \begin{theor}
     (признак Вейерштрасса равномерной сходимости ф.р./мажарантный)\\
     Пусть дан ряд с общим членом $a_n(x)$ и мы можем оценить 
    $|a_n(x)|\leqslant  a_n$ (то есть мажорирующим рядом, не зависящим от $x$),
причем $\sum\limits_{n=1}^{\infty}a_n$ сходится.
Тогда ряд $\sum\limits_{n=1}^{\infty} a_n(x)$ сходится равномерно на том
множестве, на котором верна оценка. 
\end{theor}
\textbf{Доказательство.}  Испоьзуем критерий Коши: фиксируем некоторое 
$\varepsilon>0$. Ряд $a_n$ сходится, значит, по критерию коши
 $\forall \varepsilon>0~\exists n_0(\varepsilon)\in\mathbb{N}~\forall n>n_0
 \forall p\in\mathbb{N}: \sum\limits_{k=n+1}^{n+p} a_k<\varepsilon$. Из 
 пункта 1 имеем $\forall x\in X\forall n\in\mathbb{N}: |a_n(x)|\leqslant a_k$.
Тогда $|\sum\limits_{k=n+1}^{n+p} a_k(x)|\leqslant \sum\limits_{k=n+1}
^{n+p}|a_k(x)|\leqslant \sum\limits_{k=n+1}^{n+p} a_k<\varepsilon$. Тогда
по критерию Коши для функционального ряда следует равномерная сходимость.
$\square$ \\
\textbf{Пример.} Исследуем на равномерную сходимость $\sum\limits_{n=1}^
{\infty} \frac{arcctg(nx)}{n},~x\in(\varepsilon,\infty),~\varepsilon>0$.
Подставив ноль, по методу граничной точки нет равномерной сходимости.\\
\textbf{Пример.} Исследуем сходимость $\sum\limits_{n=1}^{\infty} e^{-n^5x^2}
\sin{nx}$ на прямой. Спойлер: сходится равномерно. Сделаем оценку:  
$|a_n(x)|\leqslant e^{-n^5x^2}n|x|$. Функция симметрична при замене $x\mapsto
-x$, значит, будем оценивать на положительном луче, откинув модуль. Оценим
максимумом, вычислив производную и решив уравнение. Имеем  $x=\frac{1}
{\sqrt{2n^5}}$. Подставляем: $f_n(x)\leqslant f(\frac{1}{\sqrt{2n^5} })=
\frac{1}{\sqrt{2e}n^{\frac{3}{2}}}=a_n$. Значит, $|a_n(x)|\leqslant |f_n(x)|
\leqslant a_n~\forall x\in\mathbb{R}$. Итак, сходимость равномерная. \\
\begin{theor}
(признак Дирихле равномерной сходимости функционального ряда)\\
Дан ряд $\sum\limits_{n=1}^{\infty} a_n(x)b_n(x)$ и\\
1. $\forall x\in X: \{a_n(x)\}$ монотонна по $n$;\\
2.  $\exists M=const~ \forall x\in X \forall n\in N:|B_n(x)|\leqslant M$,
где $B_n(x)$ - частичные суммы ряда  $b_n$.\\
Тогда ряд сходится равномерно на $X$
\end{theor}
\textbf{Доказательство.} Фиксируем $\varepsilon>0$. Так как 
$a_n\rightrightarrows0$ на $X$, то для  $\frac{\varepsilon}{6B}>0$
$\square$ \\
\textbf{Пример.} $\sum\limits_{n=1}^{\infty} \sin{nx}/n$. Исследовать на 
равномерную сходимость на интервалах $(\varepsilon,2\pi-\varepsilon)$,
$(0,2\pi)$. Ну, раз говорят что уже было. То не пишем.
На втором интервале нет равномерной сходимости по краевому критерию. 



 


