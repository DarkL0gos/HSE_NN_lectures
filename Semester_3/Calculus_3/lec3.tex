Переходим к более тонким признакам сходимости ряда. Алгоритм вырисовывается 
следующий: сначала даламберим, потом кошируем. Если не помогает, пробуем 
признак Раабе, но все вопросы снимает гауссирование. 
\begin{theor} (признак Даламебра в оценочной форме)\\
Пусть дан ряд с общим членом $a_n$. Тогда\\
1. Если $\frac{a_{n+1}}{a_n}\leqslant q<1$, то ряд сходится;\\
2. Если $\frac{a_{n+1}}{a_n}\geqslant q<1$, то ряд расходится.
\end{theor}
\textbf{Доказательство.} 
1. Ряд с общим членом $b_n=q^n,~q\in(0,1)$, сходится. По условию, 
$\frac{a_{n+1}}{a_n}\leqslant \frac{b_{n+1}}{b_n}$, значит, ряд
сходится по 3-му признаку сравнения.\\
2. Ряд с общим членом $b_n=1$ расходится. По условию, 
$\frac{a_{n+1}}{a_n}\geqslant\frac{b_{n+1}}{b_n}$, значит, ряд
расходится по 3-му признаку сравнения. $\square$
\begin{theor}
    (признак Даламбера в предельной форме)\\
    Пусть дан ряд с общим членом $a_n$. Тогда\\
    1. $\varlimsup\limits_{n \to \infty}\frac{a_{n+1}}{a_n}=q<1$, 
    то ряд сходится;\\
    2. $\varliminf\limits_{n \to \infty}\frac{a_{n+1}}{a_n}=r>1$,
    то ряд расходится.
\end{theor}
\textbf{Доказательство.}  
1. Пусть верхний предел равен $q<1$. Возьмем  $\varepsilon=\frac{1-q}{2}$.
Тогда $\exists n_0\in \mathbb{N}~\forall n>n_0: \frac{a_{n+1}}{a_n}\leqslant 
q+\varepsilon=q_1<1$. Тогда по признаку Даламебра в оценочной форме ряд 
сходится.\\
2. Так как для некоторой подпоследовательности $\frac{a_{n+1}}{a_n}>1$, то
не выполняется необходимый признак, следовательно, ряд расходится. 
$\square$\\ 
\textbf{Замечание.} Если предел равен 1, то $r=q=1$.\\
\textbf{Замечание.} В отличие от признака Коши, в п.2 нельзя заменить
нижний предел на верхний.\\
\textbf{Замечание.} Если все-таки получилась единица, то ряд может как 
сходиться, так и расходиться. Но если предел подходит к единице сверху, то
ряд расходится (в силу невыполнения необходимого признака). 
\begin{theor}
    (признак Коши в оценочной форме)\\
    Пусть дан ряд с общим членом $a_n$.\\
    Если $\sqrt[n]{a_n}\leqslant q<1$, то ряд сходится.\\
    Если $\sqrt[n]{a_n}\geqslant1$, то ряд расходится.
\end{theor}
\textbf{Доказательство.} Сравним с геометрической прогрессией.\\
1. $a_n\leqslant q^n,~q<1$, значит ряд сходится по признаку сравнения.\\
2. $a_n>1$, значит ряд расходится по необходимому признаку. $\square$ 
\begin{theor}
    (признак Коши в предельной форме)\\
    Пусть дан ряд с общим членом $a_n$ и 
    $\varlimsup\limits_{n \to \infty}\sqrt[n]{a_n}=q$. Тогда:
    \begin{enumerate}
        \item Если $q<1$, то ряд сходится.
        \item Если $q>1$, то ряд расходится.
    \end{enumerate}
\end{theor}
\textbf{Доказательство.} Аналогично признаку Даламбера.\\
1. Рассмотрим предел $\varlimsup\limits_{n \to \infty}\sqrt[n]{a_n}=q<1$. 
Возьмем $\varepsilon=\frac{1-q}{2}$. Тогда
$$\exists n_0~\forall n>n_0:\sqrt[n]{a_n}=q+\varepsilon=\frac{q+1}{2}<1$$
Тогда ряд сходится по признаку Коши в оценочной форме.\\
2. Рассмотрим предел $\varlimsup\limits_{n \to \infty}\sqrt[n]{a_n}=q<1$.
Выделим подпоследовательность $a_{n_k}$, на которой достигается этот верхний 
предел. Возьмем $\varepsilon=q-1$. Тогда
$$\exists k_0~\forall k>k_0:\sqrt[n_k]{a_{n_k}}>1$$
Значит, $a_{n_k}>1$, и ряд расходится по необходимому условию. 
$\square$

\textbf{Пример.} $\sum\limits_{n=1}^{\infty} \left(\frac{2+(-1)^{n}}
{5+(-1)^{n+1}}\right)^n$. Кошируя это ряд, взяв наибольшую
подпоследовательность, получим предел $\frac{3}{4}$, значит, ряд сходится. 
Можно ещё просто посчитать две подпоследовательности. 

\textbf{Пример.} $\sum\limits_{n=1}^{\infty}\left(\frac{1+\cos{n}}{2+\cos{n}}
\right)^{2n-\ln{n}}$. Оценим это рядом $b_n=\left(\frac{1+n}{2+n}\right)^{2n
-\ln{n}}$. В итоге получится, что ряд сходится.
\begin{theor}
    (признак Раабе в оценочной форме)\\
    Пусть дан знакопостоянный ряд с общим членом $a_n>0$. Тогда:\\
    1. Если $\frac{a_{n+1}}{a_n}\geqslant1-\frac{1}{n}$, то ряд расходится.\\
2. Если $\exists \alpha>1: \frac{a_{n+1}}{a_n}\leqslant 1-\frac{\alpha}{n}$
тогда ряд сходится. 
\end{theor}
\textbf{Доказательство.}  
1. Пусть $\frac{a_{n+1}}{a_n}\geqslant \frac{n-1}{n}$.
Введем ряд с общим членом $b_n=\frac{1}{n-1}$. 
$\frac{a_{n+1}}{a_n}\geqslant\frac{b_{n+1}}{b_n}$, 
и так как ряд $b_n$ расходится, то ряд $a_n$ расходится по третьему 
признаку сравнения.\\
2. Пусть  $\beta\in(1,\alpha)$, тогда $\sum\limits_{n=1}^{\infty} b_n=
\sum\limits_{n=1}^{\infty} \frac{1}{n^\beta}$ сходится. Далее,
$\frac{b_{n+1}}{b_n}=(\frac{n}{n+1})^\beta=(1+\frac{1}{n})^{-\beta}=
1+\frac{\beta}{n}+O(\frac{1}{n^2})$. Затем, $- \frac{\beta}{n}>- 
\frac{\alpha}{n}\implies 1-\frac{\beta}{n}>1-\frac{\alpha}{n}$.
Так как $O(\frac{1}{n^2})$ - бесконечно малая более высокого порядка, чем 
$\frac{\alpha}{n}$ и $\frac{\beta}{n}$, то $\exists n_0\in\mathbb{N}~\forall
n>n_0:1-\frac{\alpha}{n}<1-\frac{\beta}{n}+O(\frac{1}{n^2})$. Правая часть
равна $\frac{b_{n+1}}{b_n}$. По условию, $\frac{a_{n+1}}{a_n}\leqslant 
1-\frac{\alpha}{n}$. Из этих двух условий по свойству транзитивности 
неравенств получаем оценку $\frac{a_{n+1}}{a_n}<\frac{b_{n+1}}{b_n}$, откуда 
следует сходимость ряда. $\square$ 
\begin{theor}
    (Признак Раабе в предельной форме)\\
    Пусть дан ряд с общим членом $a_n$ и
    $\lim\limits_{n \to \infty}n(1-\frac{a_{n+1}}{a_n})=R$. 
    Тогда:\\
    1. $R<1$ - ряд расходится\\
    2. $R>1$ - ряд сходится. 
\end{theor}
\textbf{Доказательство.} 
1. Пусть $\varepsilon=1-R$. Тогда
$$\exists n_0~\forall n>n_0:
n\left(1-\frac{a_{n+1}}{a_n}\right)<R+\varepsilon=1$$
Значит, $1-\frac{a_{n+1}}{a_n}<\frac{1}{n_n}$,
$\frac{a_{n+1}}{a_n}>1+\frac{1}{n}$,
тогда ряд расходится по необходимому признаку.\\
2. Пусть $\alpha\in(1,R)$, $\varepsilon=R-\alpha$. Тогда 
$$\exists n_0~\forall n>n_0:n\left(1-\frac{a_{n+1}}{a_n}\right)>\alpha$$ 
откуда $\frac{a_{n+1}}{a}<1-\frac{\alpha}{n}$. Значит, ряд сходится по
признаку Раабе в оценочной форме. $\square$\\
\textbf{Замечание.} 
$\lim\limits_{n \to \infty}n(\frac{a_n}{a_{n+1}}-1)=
\lim\limits_{n \to \infty}n(1-\frac{a_{n+1}}{a_n})$.

Теперь докажем очень прикольный признак, из которого следуют почти все 
остальные признаки. 
\begin{theor}
    (признак Куммера)\\
    Пусть даны две последовательности $\{a_n\}$ и  $\{c_n\}$. Тогда:\\
    1. Если $\exists \alpha>0~\exists n_0\in\mathbb{N}~\forall n>n_0:
    C_n-C_{n+1}\cdot \frac{a_{n+1}}{a_n}\geqslant\alpha$ - ряд с общим членом
    $a_n$ сходится.\\
    2. Если ряд $\sum\limits_{n=1}^{\infty} \frac{1}{C_n}$ расходится 
    и $C_n-C_{n+1}\cdot\frac{a_{n+1}}{a_n}\leqslant 0$, то ряд c общим
    членом $a_n$ расходится.
\end{theor}
\textbf{Доказательство.} %Пж убейте меня бля я больше не могу
1. $\alpha\cdot a_k\leqslant C_ka_k-C_{k+1}a_{k+1}$. Далее 
$\alpha\cdot \sum\limits_{k=1}^{\infty} a_k\leqslant C_1a_1-C_{n+1}a_{n+1}
\leqslant C_1a_1$. Значит, $S_n\leqslant \frac{C_1a_1}{\alpha}$, и ряд
сходится по третьему признаку сравнения.\\
2. $C_n\leqslant C_{n+1}\cdot \frac{a_{n+1}}{a_n}$, значит,
$\frac{C_n}{C_{n+1}}\leqslant \frac{a_{n+1}}{a_n}$. Пусть $b_n=\frac{1}{C_n}$.
Тогда ряд с общим членом $b_n$ расходится и $\frac{b_{n+1}}{b_n}\leqslant 
\frac{a_{n+1}}{a_n}$, поэтому ряд с общим членом $a_n$ расходится по 3-му
признаку сравнения. $\square$ \\
\textbf{Следствие 1.} Признак Даламбера при $C_n\equiv1$\\
\textbf{Следствие 2.} Признак Раабе. Возьмем $C_n=n-1$. Имеем\\ 
1. $n-1-n\cdot \frac{a_{n+1}}{a_n}\geqslant\alpha\implies1-\frac{1}{n}
-\frac{a_{n+1}}{a_n}\geqslant\frac{\alpha}{n}\implies\frac{a_{n+1}}{a_n}
\leqslant 1-\frac{1+\alpha}{n}$.\\
\textbf{Следствие 3.} Признак Бертрана. Возьмем $C_n=(n-2)\ln(n-1)$. 
Тогда $\frac{a_{n+1}}{a_n}\geqslant1-\frac{1}{n}-\frac{1}{n\ln{n}}$
ряд сходится.
\begin{theor}
    (признак Гаусса)\\
    Пусть дан положительный ряд. Представим его в виде
    $$\frac{a_{n+1}}{a_n}=D-\frac{R}{n}+ \frac{\theta_n}{n^{1+\varepsilon}}$$ 
    Тогда:\\
    1. Если $D>1$ - ряд расходится;\\
    2. Если $D<1$ - ряд сходится;\\
    3. Если $D=1$,  $R\leqslant 1$ - ряд расходится;\\
    4. Если $D=1$,  $R>1$ - ряд сходится.\\
    Здесь $\theta_n$ - ограниченная монотонная последовательность,
     $\alpha>0$.
\end{theor}
\textbf{Доказательство.} Пункты 1 и 2 следуют из признака Даламбера,
так как $D=\lim\limits_{n \to \infty}\frac{a_{n+1}}{a_n}$.\\
3. По условию, имеем
$$\frac{a_{n+1}}{a_n}=1-\frac{R}{n}+\frac{\theta_n}{n^{1+\varepsilon}}$$
Тогда $n(1-\frac{a_{n+1}}{a_n})=R-\frac{\theta_n}{n^\varepsilon}$, откуда
$\lim\limits_{n \to \infty}\left(n\left(1-\frac{a_{n+1}}{a_n}\right)\right)=R$,
значит, ряд сходится по признаку Раабе.\\
4. По условию, имеем
$$\frac{a_{n+1}}{a_n}=1-\frac{1}{n}+\frac{\theta_n}{n^{1+\varepsilon}}$$
откуда $n\left( 1-\frac{a_{n+1}}{a_n}\right)-1=
-\frac{\theta_b}{n^\varepsilon}$. Домножим на логарифм и рассмотрим предел
получившегося выражения:
$$\lim\limits_{n \to \infty}
\ln n\cdot n\left(n\left(1-\frac{a_{n+1}}{a_n}\right)-1\right)=
\lim\limits_{n \to \infty}\left(\ln n\cdot 
-\frac{\theta_b}{n^\varepsilon}\right)=0$$
Значит, по признаку Бертрана ряд расходится. $\square$ 
\begin{theor}
    (интегральный признак)\\
    Пусть дана непрерывная неотрицательная невозрастающая функция $f(x)$,
    определенная на $[1,\infty)$. Тогда 
    ряд $\sum\limits_{n=1}^{\infty} a_n$ и интеграл $\int\limits^\infty_1
    f(x)dx$ сходятся и расходятся одновременно, где $a_n=f(n)$ - значения
    функции в натураьных числах.
\end{theor}
\textbf{Доказательство.} Очевидно,что
$\forall x\geqslant1~\exists k\in\mathbb{N}:k\leqslant x\leqslant k+1$.
По условию невозрастания имеем $f(k)\geqslant f(x)>f(k+1)$.
Значит, $a_{k+1}<f(x)\leqslant a_k$. Определенный интеграл от функции 
на единичном отрезке не больше её максимального значения, поэтому
$$a_{k+1}<\int\limits_{k}^{k+1}f(x)dx\leqslant a_k$$ 
Просуммируем: 
$$\sum\limits_{k=1}^{\infty}a_{k+1}<\sum\limits_{k=1}^{\infty}
\int\limits_{k}^{k+1}f(x)dx \leqslant \sum\limits_{k=1}^{\infty} a_k$$
Отсюда получаем, что 
$$S_{n+1}-a_1<\int\limits_{1}^{n+1}f(x)dx\leqslant S_n$$ 
Если ряд сходится, то он ограничен. Значит, ограничен и интеграл, а поскольку
это интеграл от положительной функции, он тоже сходится. Обратно, если 
интеграл сходится, то ряд ограничен, значит, он сходится по теореме 
Вейерштрасса. $\square$

%\textbf{Пример.} Исследуем $\sum\limits_{n=1}^{\infty} \frac{1}{n^p}$. Взятием
%интеграла получаем условия сходимости:
%$$
%\begin{cases}
%    \text{Сходится при } p>1\\   
%    \text{Расходится при } p\leqslant 1
%\end{cases}$$

Рассмотрим ещё несколько интересных свойств знакопостоянных рядов. 
\begin{theor}
    (связь признаков Даламбера и Коши)\\
    Если для ряда с общим членом $a_n$ выполняются условия признака Даламбера,
    то для него выполняются условия признака Коши. 
\end{theor}
\textbf{Доказательство.} Условие для признака Даламбера: 
$\forall n\in\mathbb{N}:\frac{a_{n+1}}{a_n}\leqslant q<1$. Перемножая 
неравенства $\frac{a_2}{a_1}\leqslant q,~\frac{a_3}{a_2}\leqslant q,...
,\frac{a_n}{a_{n-1}}\leqslant q$, получим $\frac{a_n}{a_1}\leqslant q^n$, 
откуда $\sqrt[n]{a_n}\leqslant\sqrt[n]{a_1}q$. Зафиксируем 
$\varepsilon=\frac{1-q}{2}$. Тогда $\exists n_0~\forall n>n_0:\sqrt[n]{a_1}q<
\frac{q+1}{2}=q_1$. Отсюда получаем условие применимости признака Коши:
$\sqrt[n]{a_n}<q_1<1$. $\square$ 

Ещё одна область применения рядов - \textbf{оценка погрешности приближенной 
величины с помощью положительного ряда}. Действительно, пусть 
$R_n$ -  $n$-ный остаток ряда  $\sum\limits_{n=1}^{\infty} a_n$. Тогда
Из доказательства интегрального признака $a_{k+1}<\int\limits_{k}^{k+1}f(x)dx
\leqslant a_k$, но поскольку по определению $R_n=\sum\limits_{k=n+1}^{\infty}
a_k$, получаем оценку:
$$\int\limits_{n+1}^{\infty}f(x)dx\leqslant R_n<
\int\limits_{n}^{\infty}f(x)dx$$ 

%\section{Связь признака Даламбера и Коши}
%Если $\frac{a_n}{a_{n-1}}\leqslant q$ для 
%всех n начиная с 1, то $a_n=a_1q^n$, откуда следует признак Коши. 
%$$\sqrt[n]{a_n}\leqslant \sqrt[n]{a_1}\cdot q$$
%Значит, Коши покрывает больше случаев. 
%\section{Оценка погрешности приближения какой-то величины с помощью
%положительного ряда}
%$$\int^\infty_{n+1} f(x)dx<R_n\leqslant \int^\infty_nf(x)dx$$ 
%Из доказательства интегрального признака
%$$a_{k+1}<\int^{k+1}_kf(x)dx\leqslant a_k$$ 
%$$\int^{k+1}_kf(x)dx\leqslant a_k\int^k_{k-1}f(x)dx$$ 
%$$R_n=\sum\limits_{k=n+1}^{\infty} a_k$$ 
%Итак, 
%$$\int^\infty_{n+1}\leqslant R_n<\int^\infty_nf(x)dx$$
\textbf{Пример.} Вычислим с точностью до 0,001 ряд 
$\sum\limits_{n=1}^{\infty} \frac{1}{n^4}$. Ответ: $1,082\pm0,001$
(точный ответ $\frac{\pi^4}{90}$)











