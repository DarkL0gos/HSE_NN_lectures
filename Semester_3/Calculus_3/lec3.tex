\begin{theor}
    (признак Даламбера в предельной форме)\\
    Пусть дан знакоположительный ряд. Тогда\\
    1. $\overline{\lim\limits_{n \to \infty} }\frac{a_{n+1}}{a_n}=q<1$, 
    то ряд сходится.\\
    2. $\underline{\lim\limits_{n \to \infty} }\frac{a_{n+1}}{a_n}=r>1$,
    то ряд расходится.
\end{theor}
\textbf{Доказательство.}  
Пусть верхний предел равен $q<1$. Возьмем  $\varepsilon=\frac{1-q}{2}$.
Тогда $\exists n_0\in \mathbb{N}\, \forall n>n_0: \frac{a_{n+1}}{a_n}\leqslant 
q+\varepsilon$. По теореме Больцано-Вейерштрасса. Тогда по признаку
Даламебра в оценочной форме ряд сходится. \\
Далее, пусть существует нижний предел. Тогда ряд сходится
по признаку Даламберав оценочной форме, или от противного: через отрицание
необходимого признака.
$\square$\\ 
\textbf{Замечание.} Если предел равен 1, то $r=q=1$ \\
\textbf{Замечание.} В отличие от признака Коши, в п.2 нельзя заменить
нижний предел на верхний. \\
\textbf{Замечание.} Если все-таки получилась единица, то ряд может как 
сходиться, так и расходиться. 
\begin{theor}
    (признак Коши в оценочной форме)\\
    Пусть дан знакоположительный ряд. Пусть 
    $\sqrt[n]{a_n}\leqslant q<1$. Тогда ряд
    сходится. \\
    Пусть $\sqrt[n]{a_n}\geqslant1$. Тогда ряд
    расходится.
\end{theor}
\textbf{Доказательство.} Сравним с геометрической прогрессией:
$a_n\leqslant q^n\implies$ из сходимости прогрессии следует сходимость 
ряда. 
$\square$ 
\begin{theor}
    (признак Коши в предельной форме)\\
    Пусть $\overline{\lim\limits_{n \to \infty} }\sqrt[n]{a_n}=q$.\\
    \begin{enumerate}
        \item $q<1 \implies$ ряд сходится.
        \item $q>1\implies$ ряд расходится.
    \end{enumerate}
\end{theor}
\textbf{Доказательство.} Аналогично признаку Даламбера. Избавимся от верхнего 
предела, взяв предел подпоследовательности. Значит, тогда все числа попадают в 
эпсилон-окрестность числа $q$. Но тогда не выполнено необходимое условие. 
$\square$ \\
\textbf{Пример.} $\sum\limits_{n=1}^{\infty} \left(\frac{2+(-1)^{n}}
{5+(-1)^{n+1}}\right)^n$. Кошируя это ряд, взяв наибольшую
подпоследовательность, получим предел $\frac{3}{4}$, значит, ряд сходится. 
Можно ещё просто втупую посчитать две подпоследовательности. \\
\textbf{Пример.} $\sum\limits_{n=1}^{\infty}\left(\frac{1+\cos{n}}{2+\cos{n}}
\right)^{2n-\ln{n}}$. Оценим это рядом $b_n=\left(\frac{1+n}{2+n}\right)^{2n
-\ln{n}}$. В итоге получится, что ряд сходится.
\begin{theor}
    (признак Раабе в оценочной форме)\\
    Пусть дан знакоположительный ряд с общим членом $a_n>0$. Тогда:\\
    1. Если $\frac{a_{n+1}}{a_n}\geqslant_1-\frac{1}{n}$, ряд расходится.\\
    2. Если $\exists \alpha>0: \frac{a_{n+1}}{a_n}\leqslant 1-\frac{\alpha}{n}$ тогда ряд сходится. 
\end{theor}
\textbf{Доказательство.}  1. $\frac{a_{n+1}}{a_n}\geqslant \frac{n-1}{n}$,
$b_n=\frac{1}{n-1}$. $\frac{a_{n+1}}{a_n}\geqslant\frac{b_{n+1}}{b_n}$, 
если ряд $b_n$ расходится, то ряд расходится по третьему признаку сравнения.\\
2. Пусть  $\beta\in(1,\alpha)$, тогда $\sum\limits_{n=1}^{\infty} b_n=
\sum\limits_{n=1}^{\infty} \frac{1}{n^\beta}$ сходится. Далее,
$\frac{b_{n+1}}{b_n}=(\frac{n}{n+1})^\beta=(1-\frac{1}{n})^{-\beta}=
1-\frac{\beta}{n}+O*(\frac{1}{n^2})$. Затем, $- \frac{\beta}{n}>- 
\frac{\alpha}{n}\implies 1-\frac{\beta}{n}>1-\frac{\alpha}{n}$.
Так как $O(\frac{1}{n^2})$ - бесконечно малая более высокого порядка, чем 
$\frac{\alpha}{n}$ и $\frac{\beta}{n}$, то $\exists n_0\in\mathbb{N}~\forall
n>n_0:1-\frac{\alpha}{n}<1-\frac{\beta}{n}+O(\frac{1}{n^2})$. Правая часть
равна $\frac{b_{n+1}}{b_n}$. По условию, $\frac{a_{n+1}}{a_n}\leqslant 
1-\frac{\alpha}{n}$. Из этих двух условий по свойству транзитивности 
неравенств получаем оценку $\frac{a_{n+1}}{a_n}<\frac{b_{n+1}}{b_n}$, откуда 
следует сходимость ряда. $\square$ 
\begin{theor}
    (Признак Раабе в предельной форме)\\
    Пусть $\lim\limits_{n \to \infty}n(1-\frac{a_{n+1}}{a_n})=R$. 
    Тогда:\\
    1. $R<1$ - ряд расходится\\
    2. $R>1$ - ряд сходится. 
\end{theor}
\textbf{Доказательство.}  \
$\square$ 
\begin{theor}
    (признак Куммера)\\
    Даны две последовательности $a_n$ и  $c_n$. Тогда:\\
    1. Если $\exists \alpha>0\exists n_0\in\mathbb{N}~\forall n>n_0:
    C_n-C_{n+1}\cdot \frac{a_{n+1}}{a_n}\geqslant\alpha$ - ряд сходится.\\
    2. Если ряд $\sum\limits_{n=1}^{\infty} \frac{1}{C_n}$ расходится 
    и $C_n-C_{n+1} \frac{a_{n+1}}{a_n}\leqslant 0$, то ряд расходится.
\end{theor}
\textbf{Доказательство.} Пж убейте меня бля я больше не могу
$\square$ \\
\textbf{Следствие 1.} Признак Даламбера при $C_n\equiv1$\\
\textbf{Следствие 2.} Признак Раабе. Возьмем $C_n=n-1$. Имеем 
1. $n-1-n\cdot \frac{a_{n+1}}{a_n}\geqslant\alpha\implies1-\frac{1}{n}
-\frac{a_{n+1}}{a_n}\geqslant\frac{\alpha}{n}\implies\frac{a_{n+1}}{a_n}
\leqslant 1-\frac{1+\alpha}{n}$. Подставляя в пункт  
\begin{theor}
    (признак Бертрана/следствие из признака Куммера)\\
    1. $C_n=(n-2)\ln(n-1)$. $\frac{a_{n+1}}{a_n}\geqslant_1-\frac{1}{n}
    -\frac{1}{n\ln{n}}$ - ряд сходится\\
    2. 

\end{theor}
\textbf{Доказательство.}  \
$\square$ 
\begin{theor}
    (признак Гаусса)\\
    Пусть дан положительный ряд. Пусть его можно представить в виде
    $$\frac{a_{n+1}}{a_n}=D-\frac{r}{n}+ \frac{\theta_n}{n^{1+\alpha}}$$ 
    Тогда:\\
    1. Если $D>1$ - ряд расходится\\
    2. Если $D<1$ - ряд сходится\\
    3. Если $D=1$,  $R\leqslant 1$ - ряд расходится\\
    4. Если $D=1$,  $R>1$ - ряд сходится. 
\end{theor}
\textbf{Доказательство.}
$\square$ 
\begin{theor}
    (интегральный признак)\\
    Пусть ряд знакопостоянен. 
    Ряд $\sum\limits_{n=1}^{\infty} a_n$ и интеграл $\int\limits^\infty_1
    f(x)dx$ сходятся и расходятся одновременно, причем $f(n)=a_n$, функция
    определена, непрерывна, неотрицательна и невозрастающая на  $[1,\infty)$.
    Оценка погрешности: 

\end{theor}
\textbf{Доказательство.} $\forall x\geqslant1~\exists k\in\mathbb{N}:
k\leqslant x\leqslant k+1$.  По условию невозрастания имеем $f(k)\geqslant 
f(x)>f(k+1)$. $a_{k+1}<f(x)\leqslant a_k$,  $a_{k+1}$
$\square$ 
\textbf{Пример.} Исследуем $\sum\limits_{n=1}^{\infty} \frac{1}{n^p}$. Взятием
интеграла получаем условия сходимости:
$$
\begin{cases}
    \text{сходится при } p>1   
    \text{расходится при } p\leqslant 1
\end{cases}$$

