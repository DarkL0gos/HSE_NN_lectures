\chapter{Функции Эйлера}
\section{Гамма-функция}
$$\Gamma(s)=\int\limits_{0}^{\infty}x^{s-1}\cdot e^{-x}dx$$
На бесконечности сходится всегда, в нуле сходится при $s>0$.
Если проинтегрировать по частям, беря первообразуную от экспоненты, 
получим  $0+\Gamma(x-1)$. Причем $\Gamma(1)=e^{-t}\Big|_0^\infty=1$.
В точке $1/2$ сам Бог велел делать замену  $u=\frac{x^1}{2}$, и мы 
сведем к интегралу Пуассона. 

\textbf{Свойства гамма-функции}\\
1. Область определения $\equiv$ множество таких $s$, на котором 
$\Gamma(s)$ сходится:  $s>0$;\\
2. Равномерная сходимость на $[s_1,s_2]$, где $0<s_1<s_2<\infty$;\\
3. $\Gamma(s)$ непререрывна при  $s>0$;\\
4.  $\Gamma(s)>0$ при  $s>0$;\\
5.  $\boxed{\Gamma(s+1)=s\cdot \Gamma(s)}$;\\
6. $\Gamma(1)=1,~\Gamma(n+1)=n!$;\\
7.  $\Gamma\left( \frac{1}{2} \right)=\sqrt{\pi}$, $\Gamma\left( 
n+\frac{1}{2}\right)=\frac{(2n-1)!!}{2^n}\cdot \sqrt{\pi}$ ;\\
8. $\forall s>1$: $\Gamma(s)=(s-1)(s-2)...(s-n)\Gamma(s-n)$, где  $n=[s]$; 
любое значение гамма-функции можно выразить через её значения на $(0,1]$\\
9. $\Gamma^{(n)}(s)=\Gamma(s)(\ln)^n$, причем сходится при $s>0$, равномерно
сходится там же, где и гамма-функция.\\
10. $\Gamma(s)\Gamma(1-s)=\frac{\pi}{\sin(\pi n)}$ - формула дополнения\\
11. $\Gamma(x+1)=x^xe^{-x}\sqrt{2\pi n}(1+\alpha(x))$ - асимптотическая 
формула.\\
12. График: $\lim\limits_{s \to +0} \Gamma(s)=
\lim\limits_{s \to +0} \frac{\Gamma(s+1)}{s}=\frac{1}{+0}=\infty$.
$\Gamma(1)=\Gamma(2)=1$. Сначала убывает, затем возрастает.\\
13. $\Gamma(s)=\frac{1}{se^{\gamma s}}\prod\limits_{n=1}^{\infty}
\left( 1+\frac{s}{n} \right)^{-1}e^{\frac{s}{n}}$ - продолжение определения
функции на отрицательные числа (кроме отрицательных целых). 

\textbf{Доказательство.}

1. Докажем про область определения. Рассмотрим сумму интегралов
$\Gamma(s)=\int\limits_{0}^{1}+\int\limits_{1}^{\infty}$. 
Первый интеграл сходится при $s>0$ и расходится при $s\leqslant 0$
по предельному признаку сравнения с интегралом
$\int\limits_{0}^{1}x^{s-1}e^{-x}dx$. Второй интеграл: имеем
$$\forall s>\mathbb{R}~\exists x(s)~\forall x>x(s):
x^{s-1}\leqslant e^{\frac{x}{2}}$$ 
Значит, $\int\limits_{x(s)}^{\infty}e^{-\frac{x}{2}}dx$ сходится, и исходный
инетграл сходится по признаку сравнения при любом $s$. Значит, область 
определения гамма-функции -  $s>0$.

2. Докажем равномерную сходимость по признаку Вейерштрасса. Получаем, что
$x^{s-1}\cdot e^{-x}\leqslant x^{s-1}\leqslant x^{s_1-1}$ при фиксированном
$x\in [0,1]$. При этом интеграл $\int\limits_{0}^{1}x^{s_1-1}dx$ сходится, 
поэтому интеграл сходится на $[s_1,s_2]$ равномерно. Если $x\geqslant 1$,
то $x^{s-1}\cdot e^{-x}\leqslant x^{s_2-1}\cdot e^{-x}$, правая часть 
сходится и не зависит от $s$, значит, сходимость равномерная. Значит,
на объединении $1>s>0$ и  $s>0$ сходимость непрерывная.

3. Непрерывность следует из равномерной сходимости интеграла и непрерывности
подынтегральной функции по теореме о непрерывности несобственного интеграла,
зависящего от параметра. 

4. $\forall x>0~\forall s>0:x^{s-1}\cdot e^{-x}>0$ значит, 
$\Gamma(s)>0~\forall s$.

5. Имеем $\Gamma(s+1)=\int\limits_{0}^{\infty}x^{s}\cdot e^{-x}dx=
-\int\limits_{0}^{\infty}x^sd(e^{-x})=-x^{s}\cdot e^{-x}\Big|^\infty_0+
s \int\limits_{0}^{\infty}x^{s-1}\cdot e^{-x}$, откуда $\Gamma(s+1)=
s\cdot \Gamma(s)$.

6. $\Gamma(1)=\int\limits_{0}^{\infty}e^{-x}dx=1$. Факториальность следует 
по индукции из основного свойства. 

7. $\int\limits_{0}^{\infty}e^{-x^2}dx=\frac{\sqrt{\pi} }{2}$ - интеграл 
Пуассона. Значит, $\Gamma(\frac{1}{2})=\int\limits_{0}^{\infty}x^{-\frac{1}{2}}
e^{-x}dx=\int\limits_{0}^{\infty}e^{-x}\frac{dx}{\sqrt{x}}$. Заменим 
$x=t^2$, откуда  имеем $2 \int\limits_{0}^{\infty}e^{-t^2}dt=\sqrt{\pi}$.
Общая формула для полуцелых чисел следует по индукции из основного свойства.

8. По индукции.

9. Докажем, что 
$\Gamma'(s)=\int\limits_{0}^{\infty}x^{s-1}\cdot e^{-x}\ln x\,dx$. Для 
применения теоремы о дифференцировании надо доказать,
что этот интеграл равномерно сходится на $[s_1,s_2],0<s_1<s<s_2<\infty$. 
Рассмотрим
$\int\limits_{0}^{\infty} = \int\limits_{0}^{1} + \int\limits_{1}^{\infty}$.
В особой точке 0 при $s_1\geqslant_1$ имеем
$|x^{s-1}\cdot e^{-x}\ln x|\leqslant 1\cdot 1\cdot |\ln x|=-\ln x$. 
Интеграл $-\int\limits_{0}^{1}\ln x\,dx=-x\ln x\big|^1_0+
\int\limits_{0}^{1}dx=1$ - сходится. Значит, гамма-функция сходится равномерно
на $[s_1,s_2]<1$ по признаку Вейерштрасса. Если же $s_1<1$, то
$|x^{s-1}\cdot e^{-x}\ln x|\leqslant x^{s_1-1}\cdot\ln x$. 
Правая часть сходится, и интеграл сходится по признаку Вейерштрасса.\\
Если $x>1$, то  $0<x^{s-1}\cdot e^{-x}\ln x<x^{s_2-1}e^{-x}\ln x
<e^{-\frac{x}{3}}$. Также сходится по Вейерштрассу. Поэтому в итоге он сходится
на объединении областей. Поэтому можно дифференцировать.

10. Доказательство слишком длинное и использует комплексные числа. И прочие 
тоже. 

\section{Бета-функция}
$$B(p,q)=\int\limits_{0}^{1} x^{p-1}(1-x)^{q-1}dx$$ 
1. Область определения - $p>0\And q>0$;\\
2. Равномерная сходимость -  $p\geqslant p_0>0\And q\geqslant q_0>0$.\\
3. $B(p,q)$ непрерывна на области определения\\
4. $B(p,q)>0$  на области определения\\
5. $B(p,q)=B(q,p)$\\ 
6. $B(p,q)=\frac{p-1}{p+q-1}B(p-1,q)=\frac{q-1}{p+q-1}B(p,q-1)$ \\
7. Несобственный интеграл как первого, так и второго рода:
$\int\limits_{0}^{\infty} \frac{t^{p-1}dt}{(1+t)^{p+q}}$\\
8. $\boxed{B(p,q)=\frac{\Gamma(p)\Gamma(q)}{\Gamma(p+q)}}$ \\
9. $B(p,1-p)=\frac{\pi}{\sin\pi n}$ \\
10. Формула Лежандра: $\boxed{B(p,p)=
\frac{1}{2^{2p-1}}B\left(\frac{1}{2},p\right)}$.
Или же: $\Gamma(p)\Gamma(p+\frac{1}{2})=\frac{\sqrt{\pi}}{2^{2p-1}}\Gamma(2p)$.

\textbf{Доказательство.}

1. Имеем  $B(p,q)=\int\limits_{0}^{\frac{1}{2}}x^{p-1}(1-x)^{q-1}dx+
\int\limits_{\frac{1}{2}}^{1}x^{p-1}(1-x)^{q-1}dx$. При $x\to 0$,
$x^{p-1}(1-x)^{q-1}\sim x^{p-1}$, и интеграл  $\int\limits_{0}^{\frac{1}{2}}
x^{p-1}dx$ сходится при $p>0$. При  $x\to 1$, 
$x^{p-1}(1-x)^{q-1}\sim(1-x)^{q-1}$, аналогично сходится при  $q>0$. 


2. Аналогично предыдущему пункту,
$x^{p-1}(1-x)^{q-1}\leqslant x^{p_0-1}(1-x)^(q_0-1)$ - 
сходится равномерно по признаку Вейерштрасса.

3. Не на что сослаться, так как не было предела от двух переменных.

4. Очевидно.

5. Введем замену $t=1-x$, и получим точно такой же интеграл. 

6.  $B(p,q)=\int\limits_{0}^{1}x^{p-1}(1-x)^{q-1}dx=
-\frac{1}{q}\int\limits_{0}^{1}x^{p-1}d((1-x)^{q-1})=
-\frac{1}{q}x^{p-1}(1-x)^{q-1}\big|^1_0+\frac{1}{q}\int\limits_{0}^{1}
(1-x)^{q}dx^{p-1}=\frac{p-1}{q}\int\limits_{0}^{1}x^{p-2}(1-x)^{q-1}(1-x)dx$.
Отсюда $q\cdot B(p,q)=(p-1)\left(
 \int\limits_{0}^{1}x^{p-2}(1-x)^{q-1}dx-
\int\limits_{0}^{1}x^{p-1}(1-x)^{q-1}dx\right)$. Значит,
$q\cdot B(p,q)=(p-1)\cdot B(p-1,q)-(p-1)\cdot B(p,q)$, и в итоге получаем
$B(p,q)=\frac{p-1}{p+q-1}B(p-1,q)$

7. Сделаем замену $x=\frac{t}{t+1}$. Изменим пределы: 
$0\to 0,1\to \infty$. И тогда $1-x=\frac{1}{t+1}$, $dx=\frac{dt}{(t+1)^2}$.
В итоге имеем $B(p,q)=\int\limits_{0}^{\infty} \frac{t^{p-1}}{(1+t)^{p-1}}
\cdot \frac{1}{(t+1)^{q-1}}\cdot \frac{dt}{(1+t^2)}=
\int\limits_{0}^{\infty} \frac{t^{p-1}dt}{(1+t)^{p+q}}$. 

Чтобы доказать следующее свойство бета-функции, нам потребуется следующая
\begin{theor}
    (о перестановке двух несобственных интегралов)\\
    Пусть\\
    1. $f(x,y)$ определена на $[a,\infty)\times [c,\infty)$ ;\\
    2. $\int\limits_{a}^{\infty}f(x,y)dx$ сходится равномерно на 
    $[c,d]~\forall d>c$;\\
    3. $\int\limits_{c}^{\infty}f(x,y)dy$ сходится равномерно на 
    $[a,b]~\forall b>a$;\\
    4. Существует $\int\limits_{a}^{\infty}dx \int\limits_{c}^{\infty}
    |f(x,y)|dy$ или  $\int\limits_{c}^{\infty}dy \int\limits_{a}^{\infty}
    |f(x,y)|dx$ ;\\
    Тогда существуют  оба интеграла 
    $\int\limits_{a}^{\infty}dx \int\limits_{c}^{\infty}f(x,y)dy$
    и  $\int\limits_{c}^{\infty}dy \int\limits_{a}^{\infty}f(x,y)dx$,
    и они равны между собой. 
\end{theor}
\textbf{Доказательство.} Допустим, существует интеграл  
$\int\limits_{a}^{\infty}dx \int\limits_{c}^{\infty}|f(x,y)|dx$.
Тогда 
$$\int\limits_{c}^{\infty}dy \int\limits_{a}^{\infty}f(x,y)dx=
\lim\limits_{d \to \infty} \int\limits_{c}^{d} dy \int\limits_{a}^{\infty} 
f(x,y)dx$$
По теореме об интегрировании интеграла, зависящего от параметра, это все равно
$$\lim\limits_{d \to \infty}\int\limits_{a}^{\infty}dx
\int\limits_{c}^{d}f(x,y)dy$$
Обозначим $\Phi(x,d)=\int\limits_{c}^{d}f(x,y)dy$. Применяя теорему о 
предельном переходе, 
$$|\Phi(x,d)|=\left| \int\limits_{c}^{d} f(x,y)dy \right|\leqslant 
\int\limits_{c}^{d} |f(x,y)|dy\leqslant \int\limits_{c}^{\infty}|f(x,y)|dy
$$
По условию, $\int\limits_{a}^{\infty}dx \int\limits_{c}^{\infty}|f(x,y)|dy$ 
сходится, поэтому $\int\limits_{a}^{\infty}\Phi(x,d)dx$ сходится 
равномерно по $d\in (c,\infty)$. Итак,
$$\lim\limits_{d \to \infty}\int\limits_{a}^{\infty}\Phi(x,d)dx=
\int\limits_{a}^{\infty} \left( \lim\limits_{d \to \infty}\Phi(x,d)\right)dx=
\int\limits_{a}^{\infty}\left( \int\limits_{c}^{\infty}f(x,y)dy \right)dx $$
$\square$ \\

8. Теперь докажем, что ${B(p,q)=\frac{\Gamma(p)\Gamma(q)}{\Gamma(p+q)}}$.\\
Случай 1: $p>1,q>1$.  %$\Gamma(s)=\int\limits_{0}^{\infty}x^$
Сделаем замену $x=(t+1)y,t>0,y>0$. Тогда
$\Gamma(s)=\int\limits_{0}^{\infty}(t-1)^{s-1}y^{s-1}e^{-(t+1)y}(t+1)dy$.
Тогда $\frac{\Gamma(s)}{(t+1)^s}=\int\limits_{0}^{\infty}y^{s-1}
e^{-(t+1)y}dy$. Пусть $S=p+q$. Имеем
$\frac{\Gamma(p+q)}{(t+1)^{p+q}}=\int\limits_{0}^{\infty}y^{p+q-1}e^{-(t+1)y}
dy$. Домножим на $t^{p-1}$:
$$\Gamma(p+q)\cdot \frac{t^{p-1}}{(t+1)^{p+q}}=t^{p-1}
\int\limits_{0}^{\infty}y^{p+q-1}e^{-(t+1)y}dy$$
Интегрируя, получаем
\begin{equation} \label{gamma_beta}
\Gamma(p+q)\cdot \int\limits_{0}^{\infty} \frac{t^{p-1}dt}{(t+1)^{p+q}}=
\int\limits_{0}^{\infty}t^{p-1}dt \int\limits_{0}^{\infty}y^{p+q-1}
e^{-(t+1)y}dy
\end{equation}

Внезапно, 
$\Gamma(p+q)\cdot \int\limits_{0}^{\infty} \frac{t^{p-1}dt}{(t+1)^{p+q}}=
\Gamma(p+q)\cdot B(p,q)$. По простому (нестрого): если поменять 
порядок интегрирования, то и получим $\Gamma(p+q)B(p,q)=
\Gamma(p)\cdot \Gamma(q)$. Более формально, мы должны проверять условия 
теоремы, доказанной выше. Давайте сделаем это (на отл):\\
1. $f(t,y)=t^{p-1}y^{p+q-1}e^{-(t+1)y}$ определена и непрерывна на 
$[0,\infty)\times [0,\infty)$.\\
2. $f(t,y)>0$ при  $t\geqslant 0,y\geqslant 0$.\\
По \ref{gamma_beta}, существует интеграл $\int\limits_{0}^{\infty}
dt \int\limits_{0}^{\infty}|f(t,y)|dy=\Gamma(p+q)B(p,q)$.
3. Покажем равномерную сходимость. $|f(t,y)|=t^{p-1}y^{p+q-1}
e^{-ty-y}\leqslant a^{p-1}y^{p+q-1}e^{-y}$. Значит, интеграл 
$\int\limits_{0}^{\infty}f(t,y)dy$ сходится равномерно на $t\in [0,\infty)$
по признаку Вейерштрасса. \\
4. То же самое для $\int\limits_{0}^{\infty}f(t,y)dt$. Здесь
нам нужна равномерная сходимость на $u\in [\xi,b]$. Если 
$0<\xi<y<b$, то  $|f(t,y)|=t^{p-1}y^{p+q-1}e^{-ty}e^{-y}\leqslant 
b^{p+q-1}t^{p-1}e^{-\xi}$. Интеграл от этой штуки сходится, тогда
$\int\limits_{0}^{\infty}f(t,y)dt$ сходится равномерно на $[\xi,b]$ по 
признаку Вейерштрасса. 

Итак, из \ref{gamma_beta} имеем  $\Phi(t,\xi)=\int\limits_{\xi}^{\infty} 
y^{p+q-1}e^{-(t+1)y}dy$, поэтому \ref{gamma_beta} перепишется в виде
сходящегося интеграла:
$$\int\limits_{0}^{\infty}t^{p-1}\Phi(t,0)dt=\Gamma(p+q)B(p,q)$$ 
Имеем $0\leqslant \Phi(t,0)-\Phi(t,\xi)=
\int\limits_{0}^{\xi}y^{p+q-1}e^{-(t+1)y}dy\to 0$ при $\xi\to 0$,
так как это интеграл с переменным верхним пределом, дифференцируема, 
значит, непрерывна. Оценим
$$\int\limits_{0}^{\infty}t^{p-1}\Phi(t,\xi)dt\leqslant 
\int\limits_{0}^{\infty} t^{p-1}\Phi(t,0)dt$$
- интеграл сходится, значит, по признаку Вейерштрасса интеграл сходится
равномерно по $x \in (0,\infty)$. Теперь делаем предельный переход:
$$\lim\limits_{\xi \to +0}\int\limits_{0}^{\infty}t^{p+1}\Phi(t,\xi)dt
=\int\limits_{0}^{\infty}t^{p+1}\Phi(t,0)dt$$

9. Докажем формулу Лежандра, начиная с левой части. Идея - 
выделить полный квадрат: $B(p,p)=$
$$=\int\limits_{0}^{1}x^{p-1}(1-x)^{p-1}dx=\int\limits_{0}^{1}
(x-x^2)^{p-1}dx=\int\limits_{0}^{1}\left( \frac{1}{4}-
\left( \frac{1}{4}-x+x^2 \right) \right)^{p-1}dx=$$
$$=\frac{1}{4^{p-1}}\int\limits_{0}^{1} (1-1(1-2x)^2)^{p-1}dx=
\begin{cases}t=1-2x\\dt=-2dx\end{cases}=\frac{1}{2^{2p-1}}=
\int\limits_{-1}^{1} (1-t^2)^{p-1}dt=
$$
$$=\begin{cases}
    u=t^2\\t=u^{\frac{1}{2}}\\dt=\frac{1}{2}u^{-\frac{1}{2}}du\end{cases}=
    \frac{1}{2^{2p-1}}\int\limits_{0}^{1}(1-u)^{p-1}u^{-\frac{1}{2}}du=
=\frac{1}{2^{2p-1}}B(\frac{1}{2},p)$$ 



\section{Примеры}
\textbf{Пример.} $\int\limits_{0}^{\infty} x^{2n+1}e^{-x^2}dx$. Замена:
$t=x^2$,  $I=\frac{1}{2}\int\limits_{0}^{\infty} t^{n+\frac{1}{2}}
t^{-\frac{1}{2}}e^{-t}dt=\Gamma(n+1)=\frac{1}{2}n!$ 

\textbf{Пример.} $\int\limits_{0}^{1} (\ln x)^ndx$. Берем $t=-\ln x$, откуда
$I=\int\limits_{0}^{\infty}(-t)^ne^{-t}dt=(-1)^nn!$

\textbf{Пример.} $\int\limits_{0}^{1}\sqrt{(1-x^2)^3}dx$. Положим
$B(p,q)=\int\limits_{0}^{1}x^{p-1}(1-x)^{q-1}dx$. Замена $t=x^2$, значит,
$I=-\frac{1}{2}\int\limits_{0}^{1}(1-t)^{\frac{3}{2}}t^{-\frac{1}{2}}dt=
\frac{1}{2}B(\frac{1}{2},\frac{5}{2})=\frac{1}{2}\cdot \frac{\frac{5}{2}-1}
{\frac{1}{2}+\frac{5}{2}-1}B(\frac{1}{2},\frac{3}{2})=...=
\frac{3\pi}{16}$ 

\textbf{Пример.} $\int\limits_{0}^{1} \frac{x^4dx}{\sqrt[3]t{(1-x^3)^2}}$,
Свести к бета-функции.

\textbf{Пример.} $\int\limits_{0}^{\infty}\frac{\sqrt[5]{x}dx}{(1+x)^2}$. 
Свести к бета-функции.

\textbf{Пример.} $\int\limits_{0}^{\frac{\pi}{2}}(\sin x)^p(\cos x)^qdx$.
Сведем к бета-функции заменой $t=\sin^2x,~dx=\frac{1}{2}\frac{dt}{
t^\frac{1}{2}(1-t)^\frac{1}{2}}$, откуда
$I=\frac{1}{2}B(\frac{p+1}{2},\frac{q+1}{2})$ 

\textbf{Пример.} $\int\limits_{0}^{\frac{\pi}{2}}(\tg x)^{\frac{1}{4}}dx$ - 
то же самое.

\textbf{Пример.} Вычислить площадь фигуры
$(x^2+y^2)^6=x^4y^2$. Перейдя в полярку  $x=r\cos\varphi,~y=r\sin\varphi$,
имеем $r^6=\cos^4\varphi\sin^2\varphi$.  Значит, интегрируя, 
получаем $S=\frac{1}{4}\int\limits_{0}^{\pi/2}r(\varphi)d\varphi$. 

















