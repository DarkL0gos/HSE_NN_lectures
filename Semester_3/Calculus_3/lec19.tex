\section{Функции Эйлера}\
\subsection{Гамма-функция Эйлера}
$$\Gamma(s)=\int\limits_{0}^{\infty}x^{s-1}\cdot e^{-x}dx$$
На бесконечности сходится всегда, в нуле сходится при $s>0$.
Если проинтегрировать по частям, беря первообразуную от экспоненты, 
получим  $0+\Gamma(x-1)$. Причем $\Gamma(1)=e^{-t}\Big|_0^\infty=1$.
В точке $1/2$ сам Бог велел делать замену  $u=\frac{x^1}{2}$, и мы 
сведем к интегралу Пуассона. 

\textbf{Свойства гамма-функции}\\
1. Область определения $\equiv$ множество таких $s$, на котором 
$\Gamma(s)$ сходится:  $s>0$;\\
2. Равномерная сходимость на $[s_1,s_2]$, где $0<s_1<s_2<\infty$;\\
3. $\Gamma(s)$ непререрывна при  $s>0$;\\
4.  $\Gamma(s)>0$ при  $s>0$;\\
5.  $\boxed{\Gamma(s+1)=s\cdot \Gamma(s)}$;\\
6. $\Gamma(1)=1,~\Gamma(n+1)=n!$;\\
7.  $\Gamma\left( \frac{1}{2} \right)=\sqrt{\pi}$, $\Gamma\left( 
n+\frac{1}{2}\right)=\frac{(2n-1)!!}{2^n}\cdot \sqrt{\pi}$ ;\\
8. $\forall s>1$: $\Gamma(s)=(s-1)(s-2)...(s-n)\Gamma(s-n)$, где  $n=[s]$; 
любое значение гамма-функции можно выразить через её значения на $(0,1]$\\
9. $\Gamma^{(n)}(s)=\Gamma(s)(\ln)^n$, причем сходится при $s>0$, равномерно
сходится там же, где и гамма-функция.\\
10. $\Gamma(s)\Gamma(1-s)=\frac{\pi}{\sin(\pi n)}$ - формула дополнения\\
11. $\Gamma(x+1)=x^xe^{-x}\sqrt{2\pi n}(1+\alpha(x))$ - асимптотическая 
формула.\\
12. График: $\lim\limits_{s \to +0} \Gamma(s)=
\lim\limits_{s \to +0} \frac{\Gamma(s+1)}{s}=\frac{1}{+0}=\infty$.
$\Gamma(1)=\Gamma(2)=1$. Сначала убывает, затем возрастает.\\
13. $\Gamma(s)=\frac{1}{se^{\gamma s}}\prod\limits_{n=1}^{\infty}
\left( 1+\frac{s}{n} \right)^{-1}e^{\frac{s}{n}}$ - продолжение определения
функции на отрицательные числа (кроме отрицательных целых). 

\textbf{Доказательство.}

1. Докажем про область определения. Рассмотрим сумму интегралов
$\Gamma(s)=\int\limits_{0}^{1}+\int\limits_{1}^{\infty}$. 
Первый интеграл сходится при $s>0$ и расходится при $s\leqslant 0$
по предельному признаку сравнения с интегралом
$\int\limits_{0}^{1}x^{s-1}e^{-x}dx$. Второй интеграл: имеем
$$\forall s>\mathbb{R}~\exists x(s)~\forall x>x(s):
x^{s-1}\leqslant e^{\frac{x}{2}}$$ 
Значит, $\int\limits_{x(s)}^{\infty}e^{-\frac{x}{2}}dx$ сходится, и исходный
инетграл сходится по признаку сравнения при любом $s$. Значит, область 
определения гамма-функции -  $s>0$.

2. Докажем равномерную сходимость по признаку Вейерштрасса. Получаем, что
$x^{s-1}\cdot e^{-x}\leqslant x^{s-1}\leqslant x^{s_1-1}$ при фиксированном
$x\in [0,1]$. При этом интеграл $\int\limits_{0}^{1}x^{s_1-1}dx$ сходится, 
поэтому интеграл сходится на $[s_1,s_2]$ равномерно. Если $x\geqslant 1$,
то $x^{s-1}\cdot e^{-x}\leqslant x^{s_2-1}\cdot e^{-x}$, правая часть 
сходится и не зависит от $s$, значит, сходимость равномерная. Значит,
на объединении $1>s>0$ и  $s>0$ сходимость непрерывная.

3. Непрерывность следует из равномерной сходимости интеграла и непрерывности
подынтегральной функции по теореме о непрерывности несобственного интеграла,
зависящего от параметра. 

4. $\forall x>0~\forall s>0:x^{s-1}\cdot e^{-x}>0$ значит, 
$\Gamma(s)>0~\forall s$.

5. Имеем $\Gamma(s+1)=\int\limits_{0}^{\infty}x^{s}\cdot e^{-x}dx=
-\int\limits_{0}^{\infty}x^sd(e^{-x})=-x^{s}\cdot e^{-x}\Big|^\infty_0+
s \int\limits_{0}^{\infty}x^{s-1}\cdot e^{-x}$, откуда $\Gamma(s+1)=
s\cdot \Gamma(s)$.

6. $\Gamma(1)=\int\limits_{0}^{\infty}e^{-x}dx=1$. Факториальность следует 
по индукции из основного свойства. 

7. $\int\limits_{0}^{\infty}e^{-x^2}dx=\frac{\sqrt{\pi} }{2}$ - интеграл 
Пуассона. Значит, $\Gamma(\frac{1}{2})=\int\limits_{0}^{\infty}x^{-\frac{1}{2}}
e^{-x}dx=\int\limits_{0}^{\infty}e^{-x}\frac{dx}{\sqrt{x}}$. Заменим 
$x=t^2$, откуда  имеем $2 \int\limits_{0}^{\infty}e^{-t^2}dt=\sqrt{\pi}$.
Общая формула для полуцелых чисел следует по индукции из основного свойства.

8. По индукции.

9. Докажем, что 
$\Gamma'(s)=\int\limits_{0}^{\infty}x^{s-1}\cdot e^{-x}\ln x\,dx$. Для 
применения теоремы о дифференцировании надо доказать,
что этот интеграл равномерно сходится на $[s_1,s_2],0<s_1<s<s_2<\infty$. 
Рассмотрим
$\int\limits_{0}^{\infty} = \int\limits_{0}^{1} + \int\limits_{1}^{\infty}$.
В особой точке 0 при $s_1\geqslant_1$ имеем
$|x^{s-1}\cdot e^{-x}\ln x|\leqslant 1\cdot 1\cdot |\ln x|=-\ln x$. 
Интеграл $-\int\limits_{0}^{1}\ln x\,dx=-x\ln x\big|^1_0+
\int\limits_{0}^{1}dx=1$ - сходится. Значит, гамма-функция сходится равномерно
на $[s_1,s_2]<1$ по признаку Вейерштрасса. Если же $s_1<1$, то
$|x^{s-1}\cdot e^{-x}\ln x|\leqslant x^{s_1-1}\cdot\ln x$. 
Правая часть сходится, и интеграл сходится по признаку Вейерштрасса.\\
Если $x>1$, то  $0<x^{s-1}\cdot e^{-x}\ln x<x^{s_2-1}e^{-x}\ln x
<e^{-\frac{x}{3}}$. Также сходится по Вейерштрассу. Поэтому в итоге он сходится
на объединении областей. Поэтому можно дифференцировать.

10. Доказательство слишком длинное и использует комплексные числа. И прочие 
тоже. 

\subsection{Бета-функция}
$$B(p,q)=\int\limits_{0}^{1} x^{p-1}(1-x)^{q-1}dx$$ 
1. Область определения - $p>0\And q>0$;\\
2. Равномерная сходимость -  $p\geqslant p_0>0\And q\geqslant q_0>0$.\\
3. $B(p,q)$ непрерывна на области определения\\
4. $B(p,q)>0$  на области определения\\
5. $B(p,q)=B(q,p)$\\ 
6. $B(p,q)=\frac{p-1}{p+q-1}B(p-1,q)=\frac{q-1}{p+q-1}B(p,q-1)$ \\
7. Несобственный интеграл как первого, так и второго рода:
$\int\limits_{0}^{\infty} \frac{t^{p-1}dt}{(1+t)^{p+q}}$\\
8. $\boxed{B(p,q)=\frac{\Gamma(p)\Gamma(q)}{\Gamma(p+q)}}$ \\
9. $B(p,1-p)=\frac{\pi}{\sin\pi n}$ \\
10. Формула Лежандра: $\boxed{B(p,p)=
\frac{1}{2^{2p-1}}B\left(\frac{1}{2},p\right)}$.
Или же: $\Gamma(p)\Gamma(p+\frac{1}{2})=\frac{\sqrt{\pi}}{2^{2p-1}}\Gamma(2p)$.

\textbf{Доказательство.}

1. Имеем  $B(p,q)=\int\limits_{0}^{\frac{1}{2}}x^{p-1}(1-x)^{q-1}dx+
\int\limits_{\frac{1}{2}}^{1}x^{p-1}(1-x)^{q-1}dx$. При $x\to 0$,
$x^{p-1}(1-x)^{q-1}\sim x^{p-1}$, и интеграл  $\int\limits_{0}^{\frac{1}{2}}
x^{p-1}dx$ сходится при $p>0$. При  $x\to 1$, 
$x^{p-1}(1-x)^{q-1}\sim(1-x)^{q-1}$, аналогично сходится при  $q>0$. 


2. Аналогично предыдущему пункту,
$x^{p-1}(1-x)^{q-1}\leqslant x^{p_0-1}(1-x)^(q_0-1)$ - 
сходится равномерно по признаку Вейерштрасса.

3. Не на что сослаться, так как не было предела от двух переменных.

4. Очевидно.

5. Самостоятельно.

6.  $B(p,q)=\int\limits_{0}^{1}x^{p-1}(1-x)^{q-1}dx=
-\frac{1}{q}\int\limits_{0}^{1}x^{p-1}d((1-x)^{q-1})=
-\frac{1}{q}x^{p-1}(1-x)^{q-1}\big|^1_0+\frac{1}{q}\int\limits_{0}^{1}
(1-x)^{q}dx^{p-1}=\frac{p-1}{q}\int\limits_{0}^{1}x^{p-2}(1-x)^{q-1}(1-x)dx$.
Отсюда $q\cdot B(p,q)=(p-1)\left(
 \int\limits_{0}^{1}x^{p-2}(1-x)^{q-1}dx-
\int\limits_{0}^{1}x^{p-1}(1-x)^{q-1}dx\right)$. Значит,
$q\cdot B(p,q)=(p-1)\cdot B(p-1,q)-(p-1)\cdot B(p,q)$, и в итоге получаем
$B(p,q)=\frac{p-1}{p+q-1}B(p-1,q)$


7. Сделаем замену $x=\frac{t}{t+1}$. Изменим пределы: 
$0\to 0,1\to \infty$. И тогда $1-x=\frac{1}{t+1}$, $dx=\frac{dt}{(t+1)^2}$.
В итоге имеем $B(p,q)=\int\limits_{0}^{\infty} \frac{t^{p-1}}{(1+t)^{p-1}}
\cdot \frac{1}{(t+1)^{q-1}}\cdot \frac{dt}{(1+t^2)}=
\int\limits_{0}^{\infty} \frac{t^{p-1}dt}{(1+t)^{p+q}}$. 





\textbf{Пример.} $\int\limits_{0}^{\infty} x^{2n+1}e^{-x^2}dx$. Замена:
$t=x^2$,  $I=\frac{1}{2}\int\limits_{0}^{\infty} t^{n+\frac{1}{2}}
t^{-\frac{1}{2}}e^{-t}dt=\Gamma(n+1)=\frac{1}{2}n!$ 

\textbf{Пример.} $\int\limits_{0}^{1} (\ln x)^ndx$. Берем $t=-\ln x$, откуда
$I=\int\limits_{0}^{\infty}(-t)^ne^{-t}dt=(-1)^nn!$

\textbf{Пример.} $\int\limits_{0}^{1}\sqrt{(1-x^2)^3}dx$. Положим
$B(p,q)=\int\limits_{0}^{1}x^{p-1}(1-x)^{q-1}dx$. Замена $t=x^2$, значит,
$I=-\frac{1}{2}\int\limits_{0}^{1}(1-t)^{\frac{3}{2}}t^{-\frac{1}{2}}dt=
\frac{1}{2}B(\frac{1}{2},\frac{5}{2})=\frac{1}{2}\cdot \frac{\frac{5}{2}-1}
{\frac{1}{2}+\frac{5}{2}-1}B(\frac{1}{2},\frac{3}{2})=...=
\frac{3\pi}{16}$ 

\textbf{Пример.} $\int\limits_{0}^{1} \frac{x^4dx}{\sqrt[3]t{(1-x^3)^2}}$,
Свести к бета-функции.

\textbf{Пример.} $\int\limits_{0}^{\infty}\frac{\sqrt[5]{x}dx}{(1+x)^2}$. 
Свести к бета-функции.

\textbf{Пример.} $\int\limits_{0}^{\frac{\pi}{2}}(\sin x)^p(\cos x)^qdx$.
Сведем к бета-функции заменой $t=\sin^2x,~dx=\frac{1}{2}\frac{dt}{
t^\frac{1}{2}(1-t)^\frac{1}{2}}$, откуда
$I=\frac{1}{2}B(\frac{p+1}{2},\frac{q+1}{2})$ 

\textbf{Пример.} $\int\limits_{0}^{\frac{\pi}{2}}(\tg x)^{\frac{1}{4}}dx$ - 
то же самое.

\textbf{Пример.} Вычислить площадь фигуры
$(x^2+y^2)^6=x^4y^2$. Перейдя в полярку  $x=r\cos\varphi,~y=r\sin\varphi$,
имеем $r^6=\cos^4\varphi\sin^2\varphi$.  Значит, интегрируя, 
получаем $S=\frac{1}{4}\int\limits_{0}^{\pi/2}r(\varphi)d\varphi$. 

















