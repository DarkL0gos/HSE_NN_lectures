\begin{theor}
Если ряд сходится абсолютно, то ряд, умноженный на константу, сходится 
абсолютно. 
\end{theor}
\textbf{Доказательство.} 
Так как ряд сходится, то по определению $\forall \varepsilon>0~\exists n_0\in
\mathbb{N}~\forall n>n_0~\forall p\in\mathbb{N}:| |a_{n+1}|+...+|a_{n+p}| |
<\varepsilon$. Теперь возьмем  $\frac{\varepsilon}{|c|}$. Тогда для ряда, 
умноженного на константу, получаем, что $| |ca_{n+1}|+...+|ca_{n+p}| |=
|c|\cdot | |a_{n+1}|+...+|a_{n+p}| |<\frac{\varepsilon\cdot|c|}{|c|}$.  
$\square$ 
\begin{theor}
Сумма абсолютно сходящихся рядов сходится абсолютно.
\end{theor}
\textbf{Доказательство.}  Так как сумма модулей больше модуля суммы, то
$\sum\limits_{n=1}^{k} |a_n|+\sum\limits_{n=1}^{k} |b_n|\geqslant
\sum\limits_{n=1}^{k} |a_n+b_n|$. Но это значит, что частичные суммы 
ряда $|a_n+b_n|$ ограничены числом $S_a+S_b$ (суммами рядов), поэтому ряд
сходится. $\square$ 
\begin{theor}
    (О произведении абсолютно сходящихся рядов)\\
    Сумма всевозможных произведений $a_ib_j$ сходится абсолютно, и сумма ряда
    равна произведению сумм.
\end{theor}
\textbf{Доказательство.} Введем обозначения
$u_n=|a_n|,~v_n=|b_n|,S^i=\sum\limits_{n=1}^{\infty} i_n,\overline{S^i}=
\sum\limits_{n=1}^{\infty} |i_n|$. Частичные суммы $S_n$ произведения рядов
будут суммами элементов угловых миноров бесконечной матрицы
$$
\begin{pmatrix} 
    u_1v_1 & u_1v_2 & u_1v_3 &\ldots\\
    u_2v_1 & u_2v_2 & u_2v_3 &\ldots\\
    u_3v_1 & u_3v_2 & u_3v_3 &\ldots\\
    \vdots & \vdots & \vdots &\ddots
\end{pmatrix} 
$$
Получим
$S_1=u_1v_1,~S_2=(u_1+u_2)(v_1+v_2),~S_3=(u_1+u_2+u_3)(v_1+v_2+v_3)...$.
Получаем $S_n=\overline{S^a_n}\cdot\overline{S^b_n}$. 
Так как ряды сходятся абсолютно,
то $\lim\limits_{n \to \infty} S_n=\lim\limits_{n \to \infty}\overline{S^a_n}
\cdot \lim\limits_{n \to \infty}\overline{S^b_n}=\overline{S^a}\cdot 
\overline{S^b}$. 
$\square$ 
\begin{defin}
    (произведение рядов по Коши)\\
Пусть $S_a\cdot S_b=S_c$. Определим ряд-произведение следующим образом:\\
$c_1=a_1b_1$\\
 $c_2=a_1b_2+a_2b_1$\\
 $c_3=a_1b_3+a_2b_2+a_3b_3$\\
\end{defin}
То есть суммируем по диагоналям бесконечной матрицы
$$
\begin{pmatrix} 
    a_1b_1 & a_1b_2 & a_1b_3 & \ldots\\
    a_2b_1 & a_2b_2 & \ldots & \\
    a_3b_1 & \ldots\\
    \ldots
\end{pmatrix} 
$$

\textbf{Пример 1.} $a_n=\frac{1}{n(n+1)}=1,~b_n=\frac{n}{2^n}$. Тогда
$\sum\limits_{n=1}^{\infty} c_n=\sum\limits_{n=1}^{\infty} \sum\limits_{k=1}
^{n} \frac{n+1-k}{k(k+1)-2^{n+1-k}}$.

\textbf{Пример 2.} Произведение расходящихся рядов $a_n=1,5^n,~b_n=1-1,5^n$
в смысле Коши - сходится, так как $c_n=0,75^n$.

Заметим, что условной сходимости недостаточно! Так, для $a_n=b_n=(-1)^{n-1}/
\sqrt{n}$ ничего не выйдет. Смиритесь. Ребят а че вы с пары то свалили?
Неуютненько. %Чувствую себя лохом, и от этого неуютненько.

\begin{theor}
    (о перестановках в абсолютно сходящемся ряде)\\
    Если ряд сходится
\end{theor}
\textbf{Доказательство.}  
$\square$ \\



\subsection{Свойства условно-сходящихся рядов}
\begin{theor} (лемма о сходимости)\\
    Пусть ряд с общим членом $a_n$ сходится условно. 
    Рассмотрим отдельно подпоследовательности из положительных и 
    отрицательных членов ряда. Тогда их суммы 
    $+\infty,-\infty$ соответственно. 
\end{theor}
\textbf{Доказательство.} Пусть $S^+,~S^-$ - суммы положительных и 
отрицательных членов. Тогда $\overline{S}=S^+-S^-$, $S=S^++S^-$.
Поскльку $\overline{S}=\infty$, а $S=const$. Тогда $S^+=+\infty$, 
$S^-=-\infty$. $\square$ 
\begin{theor}
    (Римана)\\
    Если ряд сходится условно, то для любого действительного числа найдется
    такая перестановка ряда, при которой ряд сходится к этому числу.
\end{theor}
\textbf{Доказательство.} 
Пусть $\alpha\in\mathbb{R}$ - искомое число. По предыдущей лемме, 
ряд $a^+_n$ из положительных членов расходится, значит, найдется такая его
частичная сумма $S^+_1$, что $S^+_1>\alpha$. Дальше найдем 
такую частичную сумму $S^-_1$ из отрицательных членов, что 
$S^+_1+S^-_1<\alpha$. Будем повторять эту операцию, беря частичные суммы
из остатков рядов с положительными или отрицательными членами. 
Поскольку исходный ряд сходится, то его общий член
стремится к нулю, поэтому и частичные суммы отрицательных и положительных 
членов стремятся к нулю. Поэтому ряд $S^+_1+S^-_1+S^+_2+S^-_2+...$ 
является рядом Лейбница и сходится к числу $\alpha$. Переставив члены 
исходного ряда в соответствии с этими частичными суммами, получим 
искомую перестановку. $\square$ 

