\section{Действия над абсолютно сходящимися рядами}
\begin{theor}
Если ряд сходится абсолютно, то ряд, умноженный на константу, сходится абсо
лютно. 
\end{theor}
\textbf{Доказательство.} Зафиксируем $\varepsilon$. Найдем такой номер, что
ряд из модулей меньше чем $\frac{\varepsilon}{|c|}$. И в общем эта штука
сходится. 
$\square$ 
\begin{theor}
Сумма абсолютно сходящихся рядов абсолютно сходится.
\end{theor}
\textbf{Доказательство.}  Сумма модулей больше модуля суммы.
$\square$ 
\begin{theor}
    (О произведении абсолютносходящихся рядов)\\
    Сумма всевозможных произведений $a_ib_j$ сходится абсолютно, и сумма ряда
    равна произведению сумм.
\end{theor}
\textbf{Доказательство.} Введем две переменные с модулями. Введем новые
обозначения, как в прошлой теореме. Пользуясь этой же теоремой, мы можем
доказать абсолютную сходимость для хотя бы одного из упорядочиваний. 
Представим себе бесконечную матрицу $|a_ib_j|$. Будем рассматривать 
последовательность частичных сумм в угловых минорах. Для них имеем формулу
$S_{n^2}=S'_n\cdot S''_n$. По условию,в правой части есть оба предела, а 
значит и слева тоже есть. И ещё, $S_{n^2}\leqslant S_m\leqslant S_{(n+1)^2}$.
Ну кароч....че то мдэ, тут дофига текста.
$\square$ 
\begin{defin}
    (произведение рядов по Коши)\\
Пусть $S_a\cdot S_b=S_c$. имеемследующее произведение:\\
$c_1=a_1b_1$\\
 $c_2=a_1b_2+a_2b_1$\\
 $c_3=a_1b_3+a_2b_2+a_3b_3$\\
 То есть суммируем по диагональкам той бесконечной матрицы.
\end{defin}
\textbf{Пример 1.} $a_n=\frac{1}{n(n+1)}=1,~b_n=\frac{n}{2^n}$. Тогда
$\sum\limits_{n=1}^{\infty} c_n=\sum\limits_{n=1}^{\infty} \sum\limits_{k=1}
^{n} \frac{n+1-k}{k(k+1)-2^{n+1-k}}$.\\
\textbf{Пример 2.} Произведение расходящихся рядов $a_n=1,5^n,~b_n=1-1,5^n$
в смысле Коши - сходится, так как $c_n=0,75^n$. \\
Заметим, что условной сходимости недостаточно! Так, для $a_n=b_n=(-1)^{n-1}/
\sqrt{n}$ ничего не выйдет. Смиритесь. Ребят а че вы с пары то свалили. 
Чувствую себя лохом, и от этого неуютненько.
\section{Перестановки условно-сходящихся рядов}
\begin{theor}
Лемма о сходимости. Ряд $a_n$ сходится условно. Рассмотрим отдельно
подпоследовательности из положительных и отрицательных членов. Тогда их суммы
 $+\infty,-\infty$ соответственно. 
\end{theor}
\textbf{Доказательство.}  \
$\square$ 
\begin{theor}
    (Римана)\\
    Если рядсходится услвоно, то для любого действительного числа найдется
    такая перестановка ряда, при которой ряд сходится к этому числу.
\end{theor}
\textbf{Доказательство.} По предыдущей лемме, ряд из положительных членов расходится,
значит, найдется частичная сумма, большая чем искомое число. Дальше найдем 
такую частичну сумм из отрицатльных членов, чтобы, прибавв её к прошлому этапу,
получили снова меньше чем число. И так далее.  
$\square$ 

