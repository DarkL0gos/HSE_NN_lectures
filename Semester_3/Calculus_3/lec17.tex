\begin{theor}
    (признак Дирихле равномерной сходимости несобственного интеграла, 
    зависящего от параметра)\\
    1. $\forall y\in Y$ $f(x,y)$ непрерывна на  $[a,\infty)$\\
    2. $\forall y\in Y$ $\frac{\partial g}{\partial x}(x,y)$ 
    непрерывна на  $[a,\infty)$\\
    3. $\forall y\in Y$ $g(x,y)$ монотонна по $x\in [a,\infty)$\\
    4. $g(x,y)\rightrightarrows 0$ при $x\to \infty$\\
    5. $\exists M=const~\forall y\in Y~\forall x\geqslant a:
    \left| \int\limits_{a}^{x}f(t,y)dt \right|\leqslant M$.\\
    Тогда $\int\limits_{a}^{\infty}f(x,y)g(x,y)dx$ сходится равномерно на
    $Y$.
\end{theor}
\textbf{Доказательство.}  По критерию Коши. 
Для
$$\frac{\varepsilon}{4M}>0~\exists b_0(\varepsilon)>a~\forall x>b_0~
\forall y\in Y:|g(x,y)|<\frac{\varepsilon}{4m}$$ 
Возьмем $b_1,b_2>b_0$. Тогда

\begin{equation*}
\left| \int\limits_{b_1}^{b_2}f(x,y)g(x,y)dx \right| = 
\left| \int\limits_{b_1}^{b_2}g(x,y)d\left( \int\limits_{a}^{x}f(t,y)dt
\right)   \right| = 
\end{equation*}
$$=\left|\left( g(x,y)\cdot\int\limits_{a}^{x}f(t,y)dt\right)\bigg|^{b_2}_{b_1}
- \int\limits_{b_1}^{b_2} \left( \int\limits_{a}^{x} f(t,y)dt \right) \cdot 
\frac{\partial g}{\partial x} (x,y)dx \right|\leqslant$$
$$\leqslant 
\left| g(b_2,y) \right|\cdot \left| \int\limits_{a}^{b_2}f(t,y)dt \right| +
\left| g(b_1,y) \right|\cdot \left| \int\limits_{a}^{b_1}f(t,y)dt \right| +
\left| \int\limits_{b_1}^{b_2}\biggl| \int\limits_{a}^{x}f(t,y)dt\biggr|\cdot 
\frac{\partial g}{\partial x} (x,y)dx \right| \leqslant $$ 
$$\leqslant \frac{\varepsilon}{4M}\cdot M+\frac{\varepsilon}{4M}\cdot M+
M\cdot |g(b_2,y)-g(b,y)|<\frac{\varepsilon}{2}+\frac{\varepsilon}{2}=
\varepsilon$$ 
Итак, 
$$\forall \varepsilon>0~\exists b_0(\varepsilon)>a~\forall b_1,b_2>b_0~
\forall y\in Y:\left| \int\limits_{b_1}^{b_2}f(x,y)g(x,y)dx\right| 
<\varepsilon$$ 
Тогда по 
критерию Коши $\int\limits_{a}^{\infty}f(x,y)g(x,y)dx$ сходится равномерно на
$Y$. $\square$ 

\begin{theor}
    (признак Абеля равномерной сходимости несобственного интеграла, 
    зависящего от параметра)\\
    1. $\forall y\in Y$ $f(x,y)$ непрерывна на  $[a,\infty)$\\
    2. $\forall y\in Y$ $\frac{\partial g}{\partial x}(x,y)$ 
    непрерывна на  $[a,\infty)$\\
    3. $\forall y\in Y$ $g(x,y)$ монотонна по $x\in [a,\infty)$\\
    %4. $g(x,y)\rightrightarrows 0$ при $x\to \infty$\\
    4. $\exists M=const~\forall y\in Y~\forall x\geqslant a:
    \left| g(x,y)\right|\leqslant M$.\\
    5. $\int\limits_{a}^{\infty}f(x,y)dx$ сходится равномерно на $Y$.\\
    Тогда $\int\limits_{a}^{\infty}f(x,y)g(x,y)dx$ сходится равномерно на
    $Y$.
\end{theor}
\textbf{Доказательство.}  По критерию Коши. 
Для
$$\frac{\varepsilon}{3M}>0~\exists b_0(\varepsilon)>a~\forall b_1,b_2>b_0~
\forall y\in Y: \int\limits_{b_1}^{b_2}f(x,y)dx<\frac{\varepsilon}{3M}$$ 
Возьмем $b_1,b_2>b_0$. Тогда
\begin{equation*}
\left| \int\limits_{b_1}^{b_2}f(x,y)g(x,y)dx \right| = 
\left| \int\limits_{b_1}^{b_2}g(x,y)d\left( \int\limits_{b_1}^{x}f(t,y)dt
\right)   \right| = 
\end{equation*}
$$=\left|\left( g(x,y)\cdot\int\limits_{b_1}^{x}f(t,y)dt\right)
\bigg|^{b_2}_{b_1}
- \int\limits_{b_1}^{b_2} \left( \int\limits_{b_1}^{x} f(t,y)dt \right) \cdot 
\frac{\partial g}{\partial x} (x,y)dx \right|\leqslant$$
$$\leqslant 
\left| g(b_2,y) \right|\cdot \left| \int\limits_{b_1}^{b_2}f(t,y)dt \right| +
\left| \int\limits_{b_1}^{b_2} \bigg| \int\limits_{b_1}^{x}f(t,y)dt\bigg|\cdot 
\frac{\partial g}{\partial x} (x,y)dx \right| \leqslant $$ 

$$\leqslant \frac{\varepsilon}{3M}\cdot M+
M\cdot |g(b_2,y)-g(b,y)|<\frac{\varepsilon}{3}+\frac{2\varepsilon}{3}=
\varepsilon$$ 
Итак, 
$$\forall \varepsilon>0~\exists b_0(\varepsilon)>a~\forall b_1,b_2>b_0~
\forall y\in Y:\left| \int\limits_{b_1}^{b_2}f(x,y)g(x,y)dx\right| 
<\varepsilon$$ 
Тогда по 
критерию Коши $\int\limits_{a}^{\infty}f(x,y)g(x,y)dx$ сходится равномерно на
$Y$. $\square$ \\

\textbf{Пример.} $\int\limits_{1}^{\infty} \frac{y^2\cos xy}{x+y^2}$,
$y\in [0,\infty)$. Исследуем на равномерную сходимость. Пусть
$f(x,y)=y\cos xy,~g(x,y)=\frac{y}{x+y^2}$. Условия проверяются очевидным
образом, интеграл сходится равномерно по Дирихле. 

\begin{theor}
    (о непрерывности несобственного интеграла, зависящего от параметра)\\
    1. $f(x,y)$ непрерывна на  $[a,\infty)\times Y$\\
    2. $\int\limits_{a}^{\infty} f(x,y)dx$ сходится равномерно на $Y$.\\
    Тогда  $\Phi(y)=\int\limits_{a}^{\infty} f(x,y)dx$ непрерывна на $Y$
\end{theor}
\textbf{Доказательство.} 
Функция непрерывна, если она непрерывна в каждой точке. 
$\Phi(y)$ непрерывна в  $y_0$ тогда и только тогда
$$\forall \varepsilon>0~\exists \delta>0~\forall y\in Y:
|y-y_0|<\delta \implies |\Phi(y)-\Phi(y_0)|<\varepsilon$$
По второму условию, так как интеграл сходится равномерно, то
для любого 
 $$\frac{\varepsilon}{3}>0~\exists b_0>a~\forall b>b_0~\forall y\in Y:
\left|\int\limits_{b}^{\infty} f(x,y)dx\right|<\frac{\varepsilon}{3}
 $$
Тогда 
$$
\Phi(y)-\Phi(y_0)=\int\limits_{a}^{\infty} f(x,y)dx-
\int\limits_{a}^{\infty} f(x,y_0)dx=
$$
$$
=\int\limits_{a}^{b} f(x,y)dx+ \int\limits_{b}^{\infty}f(x,y)dx -
\int\limits_{a}^{b} f(x,y_0)dx -\int\limits_{b}^{\infty}f(x,y_0)dx=
$$
$$
=\left( \int\limits_{a}^{b} f(x,y)dx-\int\limits_{a}^{b} f(x,y_0)dx\right) +
\int\limits_{b}^{\infty}f(x,y)dx-\int\limits_{b}^{\infty}f(x,y_0)dx
$$
По теореме о непрерывности собственного интеграла, зависящего от параметра, 
для
$$
\forall\,\frac{\varepsilon}{3}>0~\exists \delta>0~\forall y\in Y:
|y-y_0|<\delta \implies |F(y)-F(y_0)|<\frac{\varepsilon}{3}
$$
Тогда 
$$|\Phi(y)-\Phi(y_0)|\leqslant |F(y)-F(y_0)|+ 
\left|\int\limits_{b}^{\infty}f(x,y)dx  \right| + 
\left|\int\limits_{b}^{\infty}f(x,y_0)dx \right|\leqslant 
\frac{\varepsilon}{3}+\frac{\varepsilon}{3}+\frac{\varepsilon}{3}=
\varepsilon$$
Значит, $\Phi(y)$ непрерывна в любой точке на $Y$, то есть она непрерывна
на $Y$. $\square$ 

\begin{theor}
    (о предельном переходе под знаком несобственного интеграла)\\
    1. $f(x,y)$ непрерывна на  $[a,\infty)\times Y$\\
    2. $\int\limits_{a}^{\infty} f(x,y)dx$ сходится равномерно на $Y$.\\
    Тогда  
$$\lim\limits_{y \to y_0}\int\limits_{a}^{\infty} f(x,y)dx=
\int\limits_{a}^{\infty} \lim\limits_{y \to y_0}f(x,y)dx = 
\int\limits_{a}^{\infty}f(x,y_0)dx$$
\end{theor}
\textbf{Доказательство.} В предыдущей теореме было доказано, что 
функция $\Phi(y)=\int\limits_{a}^{\infty} f(x,y)dx$ непрерывна на $Y$. 
Значит, 
$$\lim\limits_{y \to y_0}\int\limits_{a}^{\infty} f(x,y)dx=
\lim\limits_{y \to y_0} \Phi(y) = \Phi(y_0) = 
\int\limits_{a}^{\infty}f(x,y_0)dx$$
Так как функция $f(x,y)$ непрерывна, то получаем второе равенство:
$$\int\limits_{a}^{\infty}f(x,y_0)dx = 
\int\limits_{a}^{\infty}\lim\limits_{y \to y_0} f(x,y)dx\quad\square$$



\begin{theor}
    (об интегрировании несобственного интеграла, зависящего от параметра)\\
    1. $f(x,y)$ непрерывна на  $[a,\infty)\times [c,d]$\\
    2. $\int\limits_{a}^{\infty} f(x,y)dx$ сходится равномерно на $[c,d]$.\\
    Тогда 
    $$\int\limits_{c}^{d}dy \int\limits_{a}^{\infty}f(x,y)dx =
    \int\limits_{a}^{\infty}dx \int\limits_{c}^{d}f(x,y)dy$$
\end{theor}
\textbf{Доказательство.} Обозначим $\Phi(y)=\int\limits_{a}^{\infty}
f(x,y)dx$. Эта функция непрерывна на $[c,d]$ по теореме о 
непрерывности несобственного интеграла, зависящего от параметра. Значит,
$\Phi(y)$ интегрируема на  $[c,d]$, то есть существует и конечен 
интеграл  $\int\limits_{c}^{d}dy\int\limits_{a}^{\infty}f(x,y)dx=const$.
Покажем, что несобственный интеграл справа сходится к этой константе, то
есть при $b\to \infty$ имеет место
$\int\limits_{a}^{b} dx\int\limits_{c}^{d}f(x,y)dy\to
\int\limits_{c}^{d} dy \int\limits_{a}^{\infty} f(x,y)dx$. \\
Зафиксируем $\varepsilon>0$. По условию, $\int\limits_{a}^{\infty}f(x,y)dx$
сходится равномерно на $[c,d]$ Тогда для 
$$\frac{\varepsilon}{d-c}>0~\exists b_0(\varepsilon)~\forall y\in [c,d]~
\forall b>b_0:\left| \int\limits_{b}^{\infty}f(x,y)dx \right|<
\frac{\varepsilon}{d-c}$$
 Отсюда
$$\left| \int\limits_{a}^{b} dx\int\limits_{c}^{d}f(x,y)dy-
\int\limits_{c}^{d}dy \int\limits_{a}^{\infty}f(x,y)dx \right|\leqslant 
 \int\limits_{c}^{d}dy\left| \int\limits_{a}^{b} f(x,y)dx-
 \int\limits_{a}^{\infty} f(x,y)dx\right| =$$
$$=\int\limits_{c}^{d} dy\left| \int\limits_{b}^{\infty} f(x,y)dx \right|
<\frac{\varepsilon}{d-c}\cdot (d-c)=\varepsilon$$ 
Итак, 
$$\forall \varepsilon>0~\exists b_0(\varepsilon)>a~\forall b>b_0~\forall y\in 
Y:\left|  \int\limits_{a}^{b} dx\int\limits_{c}^{d}f(x,y)dy-
\int\limits_{c}^{d}dy \int\limits_{a}^{\infty}f(x,y)dx \right| <
\varepsilon \quad\square$$








