\section{Степенные ряды}
\subsection{Базовые определения}
\begin{defin}
Степенной ряд- ряд вида $\sum\limits_{n=0}^{\infty} C_n(x-x_0)^n$
\end{defin}
Числа $C_n$ - коэффициенты степенного ряда,  $x_0$ - число. Итак, степенной
ряд - обобщение понятия многочлена. Область сходимости степенного ряда 
непуста, так как так лежит как минимум  $x_0$ (в этом случае сумма ряда 
равна $C_0$). Сделав замену $t=x-x_0$, сведем любой степенной ряд к виду
 $\sum\limits_{n=0}^{\infty} C_nt^n$.
\begin{theor}
    (лемма Абеля)\\
    Если ряд $\sum\limits_{n=0}^{\infty} c_nx^n$ сходится в точке $x_0$ и 
     $|x|<|x_0|$, то ряд сходится сходится и в  $x$, причем абсолютно.
\end{theor}
\textbf{Доказательство.}  По условию ряд сходится, значит,
$c_nx^n\to0$. Тогда существует константа $M$, большая чем все члены ряда. 
Тогда $|c_nx^n|=\left| c_nx_0^n \left( \frac{x}{x_0} \right)^n  \right|
\leqslant M\cdot \left| \frac{x}{x_0} \right|^n $. Ряд $\sum\limits_{n=0}^{\infty} Mq^n$ сходится $\Rightarrow$ ряд из модулей сходится, т.е. ряд 
сходится абсолютно.
$\square$ 
\begin{theor}
Пусть $D$ - область сходимости ряда  $\sum\limits_{n=0}^{\infty} c_nx^n$,
$R=\sup\limits_{x\in D} |x|$. Тогда $(-R,R)\subset D\subset [-R,R]$.
\end{theor}
\textbf{Доказательство.} 
По лемме Абеля, второе включение очевидно: $\forall x\in D:|x|\leqslant R
\implies D\subset [-R,R]$.
Пусть $x\in(-R,R)$. Тогда  $|x|<R=R_1$. Тогда 
для него найдется  $x_0\in D:|x_0|>|x|$. Значит, ряд в точке  $x_0$ сходится,
и значит сходится в  $x$. Значит, интервал лежит в области сходимости.
$\square$ 
\subsection{Формулы для вычисления радиуса сходимости}
Пусть $\sum\limits_{n=0}^{\infty} c_nx^n=\sum\limits_{n=0}^{\infty} a_n$.
По признаку Даламбера 
$\lim\limits_{n \to \infty} \frac{|a_{n+1}(x)|}{|a_n(x)|}=|x|\cdot
\lim\limits_{n \to \infty} \frac{|c_{n+1}|}{|c_n|}<1$, то ряд сходится.
Итак, если предел существует, то 
$$\boxed{R=\lim\limits_{n \to \infty} \frac{|c_n|}{|c_{n+1}|}}$$
Аналогично, из признака Коши получим формулу Коши-Адамара:
$$\boxed{R=\frac{1}{\overline{\lim\limits_{n \to \infty}}\sqrt[n]{|c_n|}}}$$ 
В общем случае алгоритм такой:\\
1. Найти радиус сходимости.\\
2. Выписываем интервал сходимости $(x_0-R,x_0+R)$.\\
3. Исследуем на сходимость концы интервала.\\
\textbf{Пример.} Найдем область сходимости $\sum\limits_{n=0}^{\infty} 
\frac{(x-6)^n}{(n+2)3^n}$. Применим признак Даламбера:
$R=\lim\limits_{n \to \infty} \frac{(n+3)3^{n+1}}{(n+2)3^n}=3$.
Интервал сходимости: $(6-3,6+3)$. В точке $x=9$ ряд расходится (т.к.
гармонический), в точке  $x=3$ - условная сходимость (по признаку Лейбница).\\
\textbf{Пример.} Найдем область сходимости $\sum\limits_{n=0}^{\infty} 
\frac{n^2}{(n+1)^2}\cdot \frac{x^{2n}}{2^n}$. Заметим, что у этого ряда 
коэффициенты чередуются с нулем (лакунарный ряд). Используем два способа:\\
1. По формуле Коши-Адамара - возьмем четные номера, так как на них
доставляется супремум предела последовательности:
$R=\frac{1}{\lim\limits_{n \to \infty} \left( \frac{n}{n+1} \right)^
{\frac{1}{n}}\cdot \left( \frac{1}{2^{\frac{1}{2}}} \right) }=\sqrt{2}$.
Интервал сходимости $(-\sqrt{2},\sqrt{2})$, на концах расходится.\\
2. Исследуем как функциональный ряд по признаку Даламбера.
$\lim\limits_{n \to \infty} \frac{|a_{n+1}|}{|a_n|}=\frac{x^2}{2}
\lim\limits_{n \to \infty} \left( \frac{n^2+2n+1}{n^2+2n} \right)^2=
\frac{x^2}{2}$. Значит, ряд сходится, если $\frac{x^2}{2}<1$, откуда мы 
получаем тот же интервал сходимости.
\begin{theor}
    (о равномерной сходимости степенного ряда)\\
    Степенной ряд сходится равномерно на любом отрезке, лежащем внутри 
    интрвала сходимости.
\end{theor}
\textbf{Доказательство.} Для простоты рассмотрим ряд с центром в нуле. 
Пусть ряд сходится на $(-R,R)$. Возьмем  $[a,b]\subset (-R,R)$. Обозначим
$d=max(|a|,|b|)$. Тогда ряд  $\sum\limits_{n=0}^{\infty} c_nd^n$ сходится,
значит, его мы можем использовать для оценки сверху рядов на отрезке:
$|c_nx^n|\leqslant |c_nd^n|$, значит, по признаку Вейерштрасса ряд сходится
на $[a,b]$.
$\square$ 
\begin{theor}
    (о непреывной сумме степенного ряда)\\
    Сумма степенного ряда непрерывна в любой точке из интервала сходимости.
\end{theor}
\textbf{Доказательство.}  Пусть $\sum\limits_{n=0}^{\infty}c_nx^n$ 
сходится на $(-R,R)$ к  $f(x)$. Степенные функции непрерывны на интервале
(и вообще на всей прямой); по предыдущей теореме, на любом отрезке,
лежащем в интервале, ряд равномерно сходится. Значит, по теореме о 
непрерывности суммы равномерно сходящегося ряда, сумма непрерывна на 
отрезке. Так как этот отрезок произволен, то сумма непрерывна на интервале.
$\square$ 
\begin{theor}
    (об интегрировании и дифференцировании степенного ряда)\\
Пусть дан ряд $\sum\limits_{n=0}^{\infty} c_n(x-x_0)^n=f(x)$, $R$ - радиус 
сходимости. Тогда у функции  $f(x)$ существуют производные любого порядка
внутри интервала:
$$f'=\sum\limits_{n=0}^{\infty} nc_n(x-x_0)^{n-1}$$ 
Интегрирование тоже почленное. 
Причем при дифференцировании и интегрировании радиус сходимости не меняется.
\end{theor}
\textbf{Доказательство.}  Следует из соотвествующих теорем для функциональных
рядов. Последнее утверждение следует из формулы Коши-Адамара. 
$\square$\\ 
\textbf{Пример.} Вычислить сумму ряда $\sum\limits_{n=1}^{\infty} 
\frac{1}{n\cdot 2^n}$. Задания типа таких можно делать, используя 
свойства степенных рядов. Пусть $f(x)=\sum\limits_{n=1}^{\infty}\frac{x^n}{n}$.
Радиус сходимости $x\in[-1,1)$. Возьмем производную:
$f'(x)=\sum\limits_{n=1}^{\infty} x^{n-1}=\frac{1}{1-x}$. А вот теперь
проинтегрируем: $\int\limits^x_0\frac{dt}{1-t}=f(x)-f(0)$;
$f(x)=-\ln(1-x)+f(0)$. Значит, сумма искомого ряда равна $f(\frac{1}{2})=2$.
Цель этих телодвижений - привести к виду геометричсекой прогрессии, которую
легко посчитать. 








