\textbf{Пример.} Пусть $f(x)>0$ при  $x\geqslant 0$,
$\int\limits_{0}^{\infty} f(x)dx$ сходится.
Тогда $\forall \alpha>0$ интеграл $\int\limits_{0}^{\infty}f(y^\alpha x)dx$
сходится равномерно на $[y_0,\infty),~y_0>0$, и сходится неравномерно
на $(0,\infty)$.\\
\textbf{Решение.} 1. Методом оценки остатка: 
$R(b,y)=\int\limits_{b}^{\infty}f(y^\alpha x)dx=\frac{1}{y^\alpha}
\int\limits_{b}^{\infty} f(y^\alpha x)d(y^\alpha x)=\frac{1}{y^\alpha}
\int\limits_{by^\alpha}^{\infty}f(t)dt\leqslant \frac{1}{y^\alpha_0}
\int\limits_{by^\alpha}^{\infty} f(t)dt=r(b)\to 0$ при $b\to \infty$,
так как $\int\limits_{0}^{\infty} f(t)dt$ сходится.\\
2. Докажем неравномерную сходимость по супремум-критерию:
$\sup\limits_{y>0}\int\limits_{b}^{\infty} f(y^\alpha x)dx=
\sup\limits_{y>0}\frac{1}{y^\alpha}\int\limits_{by^\alpha}^{\infty} f(t)dt
\geqslant
\sup\limits_{0<y\leqslant 1}\frac{1}{y^\alpha}\int\limits_{by^\alpha}^{\infty}
f(t)dt\geqslant
\sup\limits_{0<y\leqslant 1}\frac{1}{y^\alpha}\int\limits_{b}^{\infty}f(t)dt=
\infty$ при $y\to \infty$. Тогда по супремум-критерию нет равномерной 
сходимости.

\textbf{Пример.} Интеграл Пуассона аналогично сходится равномерно на 
бесконечности, если интервал начинается не с нуля. 
\subsubsection{Свойства несобственных интегралов, зависящих от параметров}
\begin{enumerate}
    \item Равномерно сходящиеся интегралы образуют линейное пространство.
    \item Если равномерно сходится на множестве, то сходится и на его 
        подмножестве.
    \item Сходится равномерно на конечном объединении областей, 
        где сходится равномерно.
\end{enumerate}
\textbf{Пример.} Покажем, что свойство 3 нельзя обощить на объединения
бесконечного числа множеств. Так, $\int\limits_{0}^{\infty}e^{-x^2y}dx$
сходится равномерно на $E_n=[\frac{1}{n},\infty)$, но расходится на 
бесконечном объединении таких областей. 
\begin{theor}(критерий Коши)\\
    $\int\limits_{a}^{\infty}f(x,y)dx$ сходится равномерно на $Y$ тогда и 
    только тогда, когда 
     $$\forall \varepsilon>0~\exists b_0(\varepsilon)>a~\forall b_1,b_2>b_0~
 \forall y\in Y:\left|\int\limits_{b_1}^{b_2}f(x,y)dx\right|<\varepsilon$$
\end{theor}
\textbf{Доказательство.}  Фиксируем $\varepsilon>0$. Для 
$$\frac{\varepsilon}{2}~\exists b_0>a~\forall b>b_0~\forall y\in Y:
\left|\int\limits_{b_1}^{\infty}f(x,y)dx\right|<
\left|\frac{\varepsilon}{2}\right|$$
Пусть $b_1,b_2>b_0$, тогда $\left| \int\limits_{b_1}^{b_2}f(x,y)dx\right|=
\left|\int\limits_{b_1}^{\infty}f(x,y)dx-\int\limits_{b_2}^{\infty} f(x,y)dx
\right|\leqslant 
\left|\int\limits_{b_1}^{\infty}f(x,y)dx\right|
-\left|\int\limits_{b_2}^{\infty} f(x,y)dx\right|<
\frac{\varepsilon}{2}+\frac{\varepsilon}{2}=\varepsilon$.\\
Обратно, для
$$\frac{\varepsilon}{2}~\exists b_0>a~\forall b_1,b_2>b_0~\forall y\in Y:
\left|\int\limits_{b_1}^{b_2}f(x,y)dx\right|<
\left|\frac{\varepsilon}{2}\right|$$
Тогда 
$\left|\int\limits_{b_1}^{\infty}f(x,y)dx-\int\limits_{b_2}^{\infty} f(x,y)dx
\right|<\frac{\varepsilon}{2}$. Если $b_2\to \infty$, то
$$\forall y\in Y: \int\limits_{b_2}^{\infty}f(x,y)dx\to 0$$ 
(так как есть поточечная сходимость);
$\left|\int\limits_{b_1}^{\infty}f(x,y)dx\right|\leqslant\frac{\varepsilon}{2}
<\varepsilon$
Итак, выполняется определение. $\square$ \\

\begin{theor}
    (метод граничной точки)\\
    Пусть 1. $f(x,y)$ непрерывна на $[a,\infty)\times[c,d)$\\
    2. $\forall y\in (c,d)$ интеграл $\int\limits_{a}^{\infty} f(x,y)dx$ 
    сходится \\
    3. $\int\limits_{a}^{\infty}f(x,c)dx$ расходится.\\
    Тогда $\int\limits_{a}^{\infty}f(x,y)dx$ не сходится равномерно на 
    $(c,d)$
\end{theor}
\textbf{Доказательство.} Допустим, на $(c,d)$ есть равномерная сходимость.
Тогда по супремум-критерию
Самостоятельно от противного. 
$\square$ 

\begin{theor}
    (признак Вейерштрасса равномерной сходимости)\\
    Пусть\\
    1. $\forall x\geqslant a~\forall y\in Y:|f(x,y)|\leqslant g(x)$;\\
    2. $\int\limits_{a}^{\infty}g(x)dx$ сходится.\\
Тогда $\int\limits_{a}^{\infty}f(x,y)dx$ сходится равномерно на  $Y$.
\end{theor}
\textbf{Доказательство.}  Используем критерий Коши. Зафиксируем 
$\varepsilon>0$. Тогда интеграл сходится тогда и только тогда, когда
$$\forall \varepsilon>0~\exists b_0>a~\forall b_1,b_2>b_0:
\left| \int\limits_{b_1}^{b_2} g(x)dx \right|<\varepsilon$$
По условию $\forall y\in Y~\forall x\geqslant a:|f(x,y)|\leqslant g(x)$, 
откуда $g(x)\geqslant 0$. Значит,
$\left|\int\limits_{b_1}^{b_2}g(x)dx\right|=\int\limits_{b_1}^{b_2}g(x)dx$,
поэтому $\left| \int\limits_{b_1}^{b_2}f(x)dx \right| \leqslant 
\left| \int\limits_{b_1}^{b_2} |f(x)|dx \right| \leqslant 
\left| \int\limits_{b_1}^{b_2}g(x)dx\right| <\varepsilon$. По критерию Коши,
$\int\limits_{a}^{\infty}f(x,y)dx$ сходится равномерно на $Y$. $\square$ 


