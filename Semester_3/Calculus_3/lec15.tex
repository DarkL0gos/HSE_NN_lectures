\textbf{Пример 1.} $\int_{0}^{1} \frac{x^b-x^a}{\ln(x)}dx$, $0<a<b$.  
Имеем $\int\limits_{a}^{b} x^ydy=\frac{x^b-x^a}{\ln(x)}$, 
подынтегральная функция непрерывна на бруске $[0,1]\times[a,b]$, 
тогда интеграл равен $\int\limits_{0}^{1}dx\int\limits_{a}^{b}x^ydy=
\int\limits_{a}^{b}\frac{dy}{y+1}=\ln(b+1)-\ln(a+1)$.

\textbf{Пример 2.} Вычислить $\int\limits_{0}^{1}\sin(\ln(\frac{1}{x}))
\frac{x^b-x^a}{\ln(x)}=\int\limits_{0}^{1}dx \int\limits_{a}^{b}
\sin(\ln(\frac{1}{x}))x^ydy$. Функция $f(x,y)=\sin(\ln(\frac{1}{x}))x^y$ 
непрерывна на $[0,1]\times[a,b],~f(0,y)=0$. Тогда
$\int=\int\limits_{a}^{b}dy \int\limits_{0}^{1}\sin(\ln(\frac{1}{x}))x^ydx=
\begin{cases} t=\ln(\frac{1}{x})=-\ln(x)\\dx=-e^{-t}dt\end{cases}= 
\int\limits_{a}^{b} dy \int\limits_{0}^{\infty} \sin(t)e^{-ty}e^{-t}dt$. 
Внутренний интеграл возьмем по частям: $I=\int\limits_{0}^{\infty}
\sin(t)e^{-t(y+1)}dt=-\cos(t)e^{-t(y+1)}\big|_0^\infty
-(y+1)\int\limits_{0}^{\infty}cos(t)e^{-t(y+1)}dt,~I=1-(y+1)^2I$.
Значит, искомый интеграл равен 
$\int\limits_{a}^{b}\frac{dy}{(y+1)+1}=arctg\left( \frac{a-b}{1+ab}\right)$.
Домашка: тоже самое для косинуса.
\section{Несобственные интегралы, зависящие от параметра}
%Рассмотрим семейство функций $f(x,y),~x\in X,y\in Y$. Пусть
%$M\subset Y$ - множество сходимости.
%\begin{defin}
%$f(x,y)$ сходится поточечно к  $\varphi(x)$ на М при $x\to x_0$, если
% $$\forall y\in M~\forall \varepsilon>0~\exists \delta(\varepsilon,y)~
% \forall x\in X\cap U^\circ_\delta(x_0):
% |f(x,y)-\varphi(y)|<\varepsilon$$
%\end{defin}
%Определение предела: 
%$$\lim\limits_{x\to x_0}f(x)=A\iff \forall \varepsilon>0~\exists 
%\delta(\varepsilon)~\forall x\in X:0<|x-x_0|<\delta\implies|f(x)-A|
%<\varepsilon$$
%\begin{defin}
%$f(x,y)$ сходится равномерно к $\varphi(x)$ на Е при $x\to x_0$, если
%$$\forall \varepsilon>0~\exists \delta(\varepsilon)>0~\forall x\in X\cap
%U^\circ_\delta(x_0)~\forall y\in E:|f(x,y)-\varphi(x)|<\varepsilon$$
%\end{defin}
%\textbf{Пример.} $f(x,y)=\sin(y^x)$ - непрерывен (??)
Обозначим $F(y)=\int\limits_{a}^{\infty}f(x,y)dx,~y\in E$. 
\begin{defin}
%$\int\limits_{a}^{\infty}f(x,y)dx$ сходится поточечно к $F(y)$ на М, если
Если $\forall y\in E$ $F(y)$ сходится, то поточечная сходимость на Е - 
$$\forall \varepsilon>0~\exists b_0(\varepsilon,y)>a~
\forall b>b_0~\forall y\in E:\left| \int\limits_{b}^{\infty}f(x,y)dx\right|
<\varepsilon$$
\end{defin}

%Обозначим $F(b,y)=\int\limits_{a}^{b}f(x,y)dx$. Тогда $|F(b,y)-F(y)|=
%\left| \int\limits_{b}^{\infty} f(x,y)dx \right|$. $F(b,y)$ сходится к $F(y)$
%поточечно на  $M$ при  $b\to \infty$. Обозначим остаток интеграла
%$R(b,y)=\int\limits_{b}^{\infty}f(x,y)dx$. Остаток сходится поточечно 
%$R(b,y)\to 0~\forall y\in N$ при $b\to \infty$. 
\begin{defin}
$F(y)$ сходится равномерно на $E$, если
$$\forall \varepsilon>0~\exists b_0(\varepsilon)~\forall b>b_0~\forall y\in E:
\left| \int\limits_{b}^{\infty}f(x,y)dx\right|<\varepsilon$$
(то есть оценка не зависит от y).
\end{defin}
Как и с рядями, есть супремум-критерий.
\begin{theor}
    (супремум-критерий)\\
    Несобственный интеграл $\int\limits_{a}^{\infty} f(x,y)dx$, зависящий от
    параметра, сходится равномерно на $E$ тогда и только тогда, когда
    $$\lim\limits_{b\to \infty}\sup\limits_{y\in  E}\left| 
    \int\limits_{b}^{\infty}f(x,y)dx \right|=0$$
\end{theor}
\textbf{Доказательство.} Пусть интеграл сходится равномерно. Фиксируем
$\varepsilon>0$. Тогда по определению равномерной сходимости для
$${\varepsilon}>0~\exists b_0(\varepsilon)~\forall b>b_0~
\forall y\in E:
\left| \int\limits_{b}^{\infty}f(x,y)dx\right|<\varepsilon$$
Пусть 
$g(b)=\sup\limits_{y\in E}\left| \int\limits_{b}^{\infty}f(x,y)dx\right|$.
Тогда всилу произвольности $b$ имеем $g(b)<\varepsilon$, а в силу 
произвольности $\varepsilon$ имеем
$\lim\limits_{b \to \infty}g(b)=0$, что эквивалентно условию 
супремум-критерия.\\
Обратно, пусть $\lim\limits_{b\to \infty}\sup\limits_{y\in  E}\left| 
\int\limits_{b}^{\infty}f(x,y)dx \right|=0$. Тогда 
$$\forall \varepsilon>0~\exists b_0~\forall b>b_0:\sup\limits_{y\in E}
\left|\int\limits_{b}^{\infty}f(x,y)dx\right|<\varepsilon$$
Значит, выполняется неравенство 
$$\forall  y\in E:\left|\int\limits_{b}^{\infty}f(x,y)dx \right|\leqslant
\sup\left|\int\limits_{b}^{\infty}f(x,y)dx \right|$$ 
откуда следует равномерная сходимость. $\square$ 


\begin{theor} (метод оценки остатка)\\
Пусть интеграл $\int\limits_{a}^{\infty} f(x,y)dx$ сходится на $E$, и 
$r(b)$ - какая-то оценка остатка. Тогда если
 $|R(b,y)|\leqslant r(b)~\forall y\in E$ и $r(b)\to 0$ при  $b\to \infty$,
 тогда $\int\limits_{a}^{\infty}f(x,y)dx$ сходится равномерно на 
 $E$. Если же существует такая функция  $y(b)$, что $R(b,y(b))\to s\ne 0$,
 то интеграл не сходится равномерно на $X$.
\end{theor}
\textbf{Доказательство.} По теореме \ref{skhod_ost}, сходимость несобственного
интеграла эквивалентна сходимости любого из его остатков. $\square$


\textbf{Пример.} $F(y)=\int\limits_{0}^{\infty}ye^{-xy}dx$. Доказать, что 
сходимость равномерная при $[y_0,\infty),~y_0\geqslant0$, но на 
$(0,\infty)$ нет равномерной сходимости. Решение: пусть остаток
$R(b,y)=\int\limits_{b}^{\infty}ye^{-xy}dx=e^{-by}$. По методу оценки
остатка при оценке $r(b)=e^{-by_0}$ имеем равномерную сходимость.
Если мы возьмем $y=\frac{1}{b}$, то и $R(b,\frac{1}{b})=e^{-1}\ne0$, поэтому
нет равномерной сходимости. 



