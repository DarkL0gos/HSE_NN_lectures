\begin{theor}
    (признак Абеля равномерной сходимости функционального ряда)\\
Дан ряд $\sum\limits_{n=1}^{\infty} a_n(x)b_n(x)$ и $\forall x\in X$:\\
1. $|a_n(x)|\leqslant M=const$ для всех $n$;\\
2.  $\{a_n(x)\} $ мнонотонна;\\
3. $\sum\limits_{n=1}^{\infty} b_n(x)$ равномерно сходится на $X$;\\
Тогда исходный ряд равномерно сходится на  $X$.
\end{theor}
\textbf{Доказательство.}  По определению Коши. Фиксируем $\varepsilon>0$.
Так как ряд с общим членом $b_n$ сходится равномерно, то по критерию Коши для
$$\frac{\varepsilon}{3M}>0~\exists n_0(\varepsilon)~\forall n>n_0~\forall p\in
\mathbb{N}~ \forall x\in X:\left|\sum\limits_{k=n+1}^{n+p} b_k(x)\right|<
\frac{\varepsilon}{3M}$$ Тогда по неравенству Абеля 
$$\right|\sum\limits_{k=n+1}^{n+p} b_k(x)a_k(x)\left|\leqslant 
\frac{\varepsilon}{3M}
(|a_{n+1}|+2|a_{n+p}(x)|)<\frac{\varepsilon}{3M}\cdot 3M=\varepsilon$$
Тогда по критерию Коши этот ряд сходится равномерно на $X$. $\square$ 

\textbf{Пример.} Исследуем на равномерную сходимоcть ряд 
$\sum\limits_{n=1}^{\infty} \frac{\cos{nx}\sin{x}arctg{nx}}{\sqrt{n^2+x^2}}$. 
Алгоритм:\\
1. Арктангенс монотонен и ограничен.\\
2. Все остальное сходится по Дирихле.
\subsection{Свойства равномерно сходящихся рядов}
\begin{theor}
    (о непрерывности суммы равномерно сходящегося ряда)\\
    Дан ряд $\sum\limits_{n=1}^{\infty} a_n(x)$, причем \\
    1. Все функции непрерывны на множестве $X$;\\
2. $\sum\limits_{n=1}^{\infty} a_n(x)$ сходится равномерно к $S(x)$ на $X$;\\
Тогда $S(x)$ непрерывна на $X$. 
\end{theor}
\textbf{Доказательство.}  По условию, сумма из  $a_n(x)$ сходится равномерно
на  $X$ к  $S(x)$, то есть  $S_n(x)\rightrightarrows S(x)$ на  $X$, 
$S_n(x)$ непрерывна как сумма. Тогда по теореме о непрерывности предела
равномерно сходящейся последовательности, составленной из непрерывных
функций,  $S(x)$  непрерывна. Другая формулировка:  
$$\lim\limits_{x\to x_0}\sum\limits_{n=1}^{\infty} a_n(x) =
\sum\limits_{n=1}^{\infty}\lim\limits_{x\to x_0} a_n(x)
$$
(то есть можно поменять местами сумму и предел). $\square$ 

\textbf{Пример.} $\sum\limits_{n=1}^{\infty} \frac{\sin{nx}}{n}=f(x)$ - 
непрерывна на $(0,2\pi)$
\begin{theor}
(об интегрировании равномерно сходящегося ряда)\\
Пусть дан ряд $\sum\limits_{n=1}^{\infty} a_n(x)$, причем \\
 1. все функции непрерывны на отрезке $[a,b];$\\
 2. $\sum\limits_{n=1}^{\infty} a_n(x)$ сходится равномерно на $[a,b]$ к $s
 (x)$;\\
 Тогда $$\forall x,x_0\in[a,b]:~\int\limits^x_{x_0}\left( \sum\limits_{n=1}^
 {\infty} a_n(t) \right)dt=\sum\limits_{n=1}^{\infty} \left( 
\int\limits_{x_0}^{x}a_n(t)dt \right) $$ 
 (можно менять интеграл и сумму).
\end{theor}
\textbf{Доказательство.} Докажем, что $\int\limits^x_{x_0}S(t)dt=\sum\limits_{n=1}^{\infty} \int\limits^x_{x_0}a_n(t)dt$. По предыдущей теореме $S(t)$ 
непрерывна на  $[a,b]$, значит,интегрируема на нем по Риману. 
Обозначим  $\sigma_n(x)=\sum\limits_{k=1}^{n}\int\limits^x_{x_0}a_k(t)dt$ и
докажем, что $\sigma_n(x)\rightrightarrows\int\limits^x_{x_0}S(t)dt$.\\
Зафиксируем $\varepsilon>0$. По условию, $S_n(t)$ равномерно сходится на 
$[a,b]$ для  
$$\frac{\varepsilon}{b-a}>0~\exists n_0(\varepsilon)~\forall 
n>n_0~\forall x\in[a,b]:|S_n(t)-S(t)|<\frac{\varepsilon}{b-a}$$
Тогда
$\left|\sigma_n(x)-\int\limits^x_{x_0}S(t)dt\right|=
\left| \sum\limits_{k=1}^{n} \int\limits_{x_0}^{x} a_k(t)dt-
\int\limits_{x_0}^{x}S(t)dt\right|=\left| \int\limits_{x_0}^{x}(S_n(t)-S(t))dt
\right|\leqslant \left| \int\limits_{x_0}^{x}|S_n(t)-S(t)|dt\right| 
<\frac{\varepsilon}{b-a}\cdot |x-x_0|<\varepsilon$.
Значит, $\sigma_n(x)\rightrightarrows \int\limits_{x_0}^{x} S_n(t)dt$.
$\square$ 
\begin{theor}
(о дифференцировании равномерно сходящегося ряда)\\
Пусть дан ряд $\sum\limits_{n=1}^{\infty} a_n(x)$, причем \\
 1. Производные всех функций непрерывны на отрезке $[a,b];$\\
 2. $\sum\limits_{n=1}^{\infty} a_n(x)$ сходится на $[a,b]$ поточечно;\\
 3. Ряд из производных сходится равномерно на $[a,b]$ к  $S(x)$;\\
 Тогда 
$$\sum\limits_{n=a}^{\infty} a'_n(x)=
\left( \sum\limits_{n=1}^{\infty} a_n \right)'$$
то есть в ряде  можно менять производную и сумму, причем 
$\sum\limits_{n=1}^{\infty} a_n$ сходится равномерно.
\end{theor}
\textbf{Доказательство.} 
1. Используем предыдущую теорему. Тогда
$$\int\limits_{x_0}^x\left( \sum\limits_{n=1}^{\infty} a'_n(t) \right)dt=
\sum\limits_{n=1}^{\infty} \int\limits_{x_0}^xa'_n(t)dt$$
Получаем, что в равенстве
$\int\limits_{x_0}^xS(t)dt=\sum\limits_{n=1}^{\infty} (a_n(x)-a_n
(x_0))$ справа стоит число (в силу непрерывности функции), ряд из $a_n(x_0)$
сходится по условию, следовательно, ряд из $a_n(x)$ сходится.
Поэтому, дифференцируя равенство 
$\int\limits_{x_0}^{x} \sum\limits_{n=1}^{\infty} a_n(t)\,dt=
\sum\limits_{n=1}^{\infty} a_n(x)-\sum\limits_{n=1}^{\infty} a_n(x_0)$,
получаем первое утверждение теоремы.

Теперь покажем равномерную сходимость исходного ряда. 
Для этого покажем, что остаток 
ряда из производных $r_n(x)=\sum\limits_{k=n+1}^{\infty} a'_n(x)$
равномерно стремится к нулю. 
Из этого следует применимость теоремы об инетгировании: 
$\int\limits_{x_0}^{x}\sum\limits_{k=n+1}^{\infty} a'_k(t)\,dt=
\sum\limits_{k=n+1}^{\infty} \int\limits_{x_0}^{x} a'_k(t)dt=
\sum\limits_{k=n+1}^{\infty} (a_k(x)-a_k(x_0))$. Если ряд удовлетворяет 
теореме об интегрировании, то и его остатки тоже, значит,
$\int\limits_{x_0}^{x} r_n(t)dt=R_n(x)-R_n(x_0)$, откуда
$$R_n(x)=\int\limits_{x_0}^{x} r_n(t)dt+R_n(x_0)\quad(1)$$.
Зафиксируем $\varepsilon>0$. По условию, остаток обычного ряда стремится
к нулю: $R_n(x)\to0$. Тогда для 
$$\frac{\varepsilon}{2}>0~\exists n_1
(\varepsilon)~\forall n>n_1:|R_n(x_0)|<\frac{\varepsilon}{2}$$
Остаток ряда из производных равномерно стремится к нулю, тогда
для 
$$\frac{\varepsilon}{2(b-a)}>0~\exists n_2(\varepsilon)~\forall n>n_2~
\forall x\in[a,b]:|r_n(x)|<\frac{\varepsilon}{2(b-a)}$$
По формуле (1) получаем: 
$|R_n(x)|\leqslant \left| \int\limits_{x_0}^{x} r_n(t)dt \right|+
|R_n(x_0)|\leqslant \left|\left| \int\limits_{x_0}^{x} r_n(t)dt \right|+
|R_n(x_0)| \right|<\frac{\varepsilon}{2(b-a)}\cdot |x-x_0|+
\frac{\varepsilon}{2}=\varepsilon$. $\square$ 


