\subsection{Ряды Тейлора}
\begin{defin}
Пусть в некоторой окрестности $U(x_0)$ у функции существуют производные всех 
порядков. Тогда для функции $y=f(x)$ в точке $x_0$ существует ряд Тейлора:
$$f(x_0)=\sum\limits_{n=1}^{\infty} \frac{f^{(n)}(x_0)}{n!}(x-x_0)^n$$
\end{defin}
Если $x_0=0$, то ряд называется рядом Маклорена.
\begin{theor}
Если функция представляется в виде степенного ряда, то он совпадает с 
её рядом Тейлора. $f(x)=\sum\limits_{n=1}^{\infty} c_n(x-x_0)^n$.
\end{theor}
\textbf{Доказательство.}  Пусть $(x_0-R,x_0+R)$ - интервал сходимости ряда. 
Из разложения функции в ряд имеем  $f(x_0)=c_0$. Беря производную, получаем,
что  $f'(x_0)=c_1$. Дифференцируя дальше, получаем, что  $c_n=
\frac{f^{(n)}(x_0)}{n!}$. $\square$ \\
Если по произвольной функции составить ряд Тейлора, то совсем не обязательно,
что он сойдется к этой функции. Сейчас поясним:\\
\textbf{Пример.} Рассмотрим
$$f(x)=\begin{cases}
    e^{-\frac{1}{x^2}},~x\ne0\\0,~x=0    
\end{cases}$$
Очевидно (по индукции), что производная порядка $f^{(n)}(x)=e^{-\frac{1}{x^2}}
\cdot p\left( \frac{1}{x} \right)$, где $p(t)$ - многочлен. Посчитаем
производную в нуле; первая производная в нуле - нуль. По индукции получаем,
что все остальные производные тоже равны нулю. Значит, ряд Маклорена 
тождественно равен нулю, и сходится \textit{не к исходной функции, а
к тождественно нулевой}.
 \begin{theor}
     (достаточное условие сходимости ряда Тейлора)\\
Пусть $\exists h>0,~\exists M=const$ такие, что $\forall x\in \mathbb{N}~
\forall x\in(x_0-h,x_0+h): |f^{(n)}(x)|\leqslant M$. Тогда на всей
$h$-окрестности точки $x_0$ функция равна своему ряду Тейлора, 
причем он сходится равномерно на данном интервале. 
\end{theor}
\textbf{Доказательство.}  Разложим функцию $f(x)$ в ряд Тейлора и запишем  
остаток в форме Лагранжа: $r_n(x)=\frac{f^{(n+1)}(\xi)}{(n+1)!}(x-x_0)^{n+1}$,
$\xi\in(x_0,x)$ (лежит между ними).  Остаток по модуля меньше, чем
$M\cdot \frac{h^{n+1}}{(n+1)!}$ - значит, он равномерно сходится к нулю. 
Поэтому и сам ряд сходится равномерно на $(x_0-h,x_0+h)$. $\square$ 
\subsubsection{Ряды Маклорена для основных функций}
\begin{enumerate}
    \item $e^x=1+x+\frac{x^2}{2!}+...+\frac{x^n}{n!}...,~x\in\mathbb{R}$
\item $\sh(x)=x+\frac{x^3}{3!}+...+\frac{x^{2n+1}}{(2n+1)!}+...,~
x\in\mathbb{R}$
\item $\ch(x)=1+\frac{x^2}{2!}+...+\frac{x^{2n}}{(2n)!}+...,~x\in\mathbb{R}$
\item $\sin(x)=x-\frac{x^3}{3!}-...+(-1)^n\frac{x^{2n+1}}{(2n+1)!}+...,
    ~x\in \mathbb{R}$
\item $\cos(x)=1-\frac{x^2}{2!}+...+(-1)^n \frac{x^{2n}}{(2n)!}+...,~
    x\in\mathbb{R}$ 
\item $\ln(1+x)=1-\frac{x^2}{2}+\frac{x^3}{3}-...+(-1)^n
    \frac{x^n}{n}+...,~x\in(-1,1]$
\item $\ln(1-x)=,~x\in[-1,1)$
\item $\ln \frac{1+x}{1-x}=2 \sum\limits_{n=0}^{\infty} \frac{x^{2n+1}}
    {2n+1},~ x\in(-1,1)$ - в этой формуле функция принимает все положительные
значения, поэтому она круче. 
\item $(1+x)^\alpha=1+\alpha x+...+\frac{\alpha(\alpha-1)...(\alpha-n+1)}
    {n!}x^n+...$
\item $arctg(x)=x-\frac{x^3}{3}+\frac{x^5}{5}+...+(-1)^n \frac{x^{2n+1}}
    {2n+1}+...~,x\in[-1,1]$
\item $arcsin(x)=x+\sum\limits_{n=1}^{\infty} \frac{(2n-1)!!\cdot x^{2n+1}}
    {n!\cdot 2^n(2n+1)},~x\in(-1,1)$


\end{enumerate}
(Для логарифма) покажем, что остаток ряда стремится к нулю.\\
1. $x\in[0,1]:~r_n(x)=\frac{f^{(n+1)}(\xi)}{(n+1)!}x^{n+1}$. 
Подставим $\xi=x_0+\theta(x-x_0),~\theta=\theta(x,n)$.
При этом имем оценку $0\leqslant x\leqslant 1\leqslant 1+\theta x$.
Получим
$|r_n(x)|=\frac{1}{n+1}\cdot \left( \frac{x}{1+\theta x} \right)^{n+1}
\leqslant \frac{1}{n+1}$. Значит, остаток равномерно сходится к 0 на $[0,1]$.
Чтобы доказать равномерную сходимость на  $(-1,0)$, запишем остаток в 
форме Коши. Получим  $|r_n(x)|=\left( \frac{1-\theta}{1+\theta x} \right)^n
\cdot \frac{|x|^{n+1}}{1+\theta x}$. Первая дробь меньше 1, вторую 
оценим как $\frac{|x|^{n+1}}{1-{|x|}}$, что при фиксированном $x$ стремится 
к нулю.
Значит, мы можем писать разложение для логарифма!\\
\textbf{Пример.} $\sum\limits_{n=0}^{\infty} \frac{2^n}{n!}=e^2$\\
\textbf{Ряд $(1+x)^\alpha$}. Найдем радиус сходимости:
$R=\lim\limits_{n \to \infty} |\frac{a_n}{a_{n+1}}|=|\frac{n+1}{\alpha-n}|=1$.
Запишем остаток в форме Коши: $(1+x)^\alpha=1+\alpha x+...+\frac{\alpha
(\alpha-1)...(\alpha-n+1)}{n!}x^n+r_n,~r_n(x)=\frac{f^{(n+1)}(\theta x)}{n!}
(1-\theta)^nx^{n+1}$. Если остаток стремится к нулю, то и ряд сходится к 
данной функции. Пусть $r_n=A_n\cdot B_n\cdot C_n$, где $B_n(x)=(1+\theta x)^{
\alpha-1},~C_n(x)=\left( \frac{1-\theta}{1+\theta x} \right)^n,~A_n=
\frac{\alpha(\alpha-1)...(\alpha-n)}{n!}x^{n+1}$. $A_n\to0$ по признаку 
Даламбера, $|B_n(x)|\leqslant max \{(1-|x|)^{\alpha-1},(1+|x|)^{\alpha-1}\}$,
$C_n(x)<1$, значит, остаток стремится к нулю, и ряд сходится к функции.\\
\textbf{Задача.} Доказать, что в $x=1$ ряд сходится при $\alpha>-1$,
расходится при $\alpha\leqslant -1$. В точке $x=-1$ сходится абсолютно при
 $\alpha\geqslant0$, расходится при $\alpha<0$\\
Выражения для арксинуса и арктангенса получаются интегрированием разложния
их производных. \\
%Рассмотрим сходимость арксинуса на концах!!!!!!!!!!!!!!!!!!!!!!





