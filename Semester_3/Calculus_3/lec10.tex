\subsection{Использование степенных рядов}
Разложение функции в ряд - мощнейшая тема. Иногда в физике и других прикладных
областях делают так: берут сложную функцию, раскладывают её в ряд Тейлора и 
отбрасывают все члены, кроме первого. Дифференцирование обычно упрощает 
функцию, и зачастую такое упрощение имеет физический смысл (вспомним 
решение уравнения физического маятника).

\textbf{Пример.} Вычислим интеграл
$\int\limits_{0}^{1}e^{-x^2}dx$ с точностью до $0,001$. 
Разложим подынтегральную функцию в ряд Тейлора:
$e^{-x^2}=1-x^2+\frac{1}{2}x^4-\frac{1}{6}x^6+\frac{1}{24}x^8-
\frac{1}{120}x^{10}+...$. Интегрируя почленно и подставляя $x_0$, получаем 
$1-\frac{1}{3}+\frac{1}{25}-\frac{1}{6\cdot 7}+\frac{1}{27\cdot 9}-
\frac{1}{120\cdot 11}+...$. Чтобы достичь требуемой точности, необходимо
оценить остаток. Вспоминаем, что для знакочередующегося ряда оценка
дается первым членом остатка. Так как $\frac{1}{120\cdot 11}<\frac{1}{1000}$,
то $$\int\limits_{0}^{1}e^{-x^2}\approx
1-\frac{1}{3}+\frac{1}{25}-\frac{1}{6\cdot 7}+\frac{1}{27\cdot 9}$$

\textbf{Пример.} Вычислим $\ln3$ с точностью $0,001$. Представим его в виде
 $\ln3=\ln\frac{1+x}{1-x}$, откуда $x=\frac{1}{2}\in(-1,1)$ - входит
 в область сходимсоти, значит, мы можем написать разложение: 
 $\ln{3}=1+\frac{1}{3\cdot 2^2}+\frac{1}{5\cdot 2^4}+...+
 \frac{1}{(2n+1)2^{2n}}$. Этот ряд не знакочередующийся, поэтому
 придется оценивать остаток геометрической прогрессией:
 $r_n=\frac{1}{(2n+3)2^{2n+2}}+\frac{1}{(2n+5)2^{2n+4}}+...\leqslant 
 \frac{1}{2^{2n+2}}+\frac{1}{2^{2n+4}}...=\frac{1}{2^{n+2}(1-\frac{1}{4})}=
 \frac{1}{3\cdot 2^n}$. Требуемая точность достигается при $n=5$, поэтому
$$\ln 3\approx 1+\frac{1}{3\cdot 2^2}+\frac{1}{5\cdot 2^4}+
\frac{1}{7\cdot 2^6}+\frac{1}{9\cdot 2^8}\approx 1,099$$

\textbf{Пример.} Решение дифференциальных уравнений с помощью разложений в 
степенной ряд. Решим $y'' = 2x'y + 4y, y(0) = 0 , y'(0)=1$ -
диффренециальное уравнение с задачей Коши. 

Первый способ - метод неопределенных коэффициентов:
$$\begin{cases}
y = c_0 + c_1x + c_2x^2 + ... + c_nx^{n-1} ... |\cdot  4; y(0) = c_0 = 0\\
y' = c_1 + 2c_2x + 3c_3x^2+...+nc_nx^{n-1}, ... |\cdot  2x; y'(0)=c_1=1\\
y'' = 2c_2 + 6c_3x + ... + n(n-1)c_nx^{n-2}
\end{cases}$$
$x^0$ соответствует $2c_2=4c_0 \Rightarrow c_2 = 0; c_{2n} = 0; c_5 = 
\frac{1}{2};
c_7 = \frac{1}{3!}; \Rightarrow c_{2n+1} = \frac{1}{n!}$\\
$x^1$ соответствует $6c_3 = 2c_1 + 4 c_1 \Rightarrow c_3 = 1$\\
$x^n$ соответствует $(n+2)(n+1)c_{n+2} = 2nc_n + 4c_n \Rightarrow c_{n+2} = 
\frac{2}{n+1}c_n$\\
Мы получили 
$y = x + x^3 + \frac{1}{2!}x^5 + \frac{1}{3!}x^7 + ... + \frac{x^{2n+1}}{n!}
+ ... = x(1+x^2+\frac{x^4}{2!} + \frac{x^6}{3!}+...)$,
откуда решение диффура: 
$$y = xe^{x^2}$$

Второй способ - метод последовательного дифференцирования. 
Так как $y=\sum\limits_{n=0}^{\infty} \frac{y^{(n)}(0)}{n!}x^n$, 
то $y''(0)=1\cdot 2\cdot 0+4\cdot 0=0$, $y'''(x)=2xy''+6y'$ и так далее. 


