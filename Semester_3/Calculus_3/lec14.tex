\begin{theor}
    (о непрерывности интеграла)\\
    Если функция определена и непрерывна, 
\end{theor}
\textbf{Доказательство.}  
$\square$ \\

\begin{theor}
    (о дифференцируемости собственного интеграла, зависящего от параметра/
    правило Лейбница)\\
    Пусть $f(x,y)$ \\
    1. непрерывна на  $P=[a,b]\times[c,d]$;\\
    2. $\frac{\partial f}{\partial y}(x,y)$ непрерывна на $P$;\\
    Тогда:\\
    1. $F(y)=\int\limits_{a}^{b} f(x,y)dx$ дифференцируема на $[c,d]$;
    2. $F'(y)=\int\limits_{a}^{b}\frac{\partial f}{\partial y}(x,y)dx$
\end{theor}
\textbf{Доказательство.} Пусть $y\in[c,d],~y+h\in[c,d]$. 
Рассмотрим  $F(y+h)-F(y)=\int\limits_{a}^{b} (f(x,y+h)-f(x,y))dx$, 
значит, по теореме Лагранжа это равно $\int\limits_{a}^{b}
\frac{\partial f}{\partial y}(x,y+\theta h)h\,dx$, где $\theta\in(0,1)$. 
Дифференцируем:
$F'(y)=\lim\limits_{h \to 0}\frac{F(y+h)-F(y)}{h}=\lim\limits_{h \to 0}
\int\limits_{a}^{b} \frac{\partial f}{\partial y}(x,y+\theta h)dx$.
При $h\to 0$ делаем замену  $u=y+\theta h,~u\to y$. Тогда предел
 $\lim\limits_{h \to 0}
\int\limits_{a}^{b} \frac{\partial f}{\partial y}(x,u)dx=$
по теореме о предельном переходе!!!!!!!!!!
$\square$ \\
Следующая теорема обощает правило Лейбница:
\begin{theor} (обобщенное правило Лейбница)\\
    Пусть $f(x,y)$ непрерывна на  $D=\{(x,y)\mid a(y)\leqslant x\leqslant 
    b(y),c\leqslant y\leqslant d\}$, $\frac{\partial f}{\partial x}(x,y)$ 
    непрерывна на $D$ и $a'(y),b'(y)$ непрерывны на  $[c,d]$. Тогда
     $F(y)=\int\limits_{a(y)}^{b(y)}f(x,y)dx$ дифференцируема на 
     $y\in[c,d]$, причем  $F'(y)=\int\limits_{a(y)}^{b(y)} 
     \frac{\partial f}{\partial y} (x,y)dx+f(b(y),y)\cdot b'(y)-
     f(a(y),y)\cdot a'(y)$.
\end{theor}
\textbf{Доказательство.}  $F(y)=F(y,a(y),b(y))$. По правилу производной
сложной функции  $\frac{dF}{dy}=\frac{\partial F}{\partial y}+
\frac{\partial F}{\partial a}\cdot \frac{\partial a}{\partial y} 
\frac{\partial F}{\partial b}\cdot \frac{\partial b}{\partial y}=
\int\limits_{a(y)}^{b(y)}\frac{\partial F}{\partial y}(x,y)dx
+f(b(y),y)\cdot b'(y)-f(a(y),y)\cdot a'(y)$ $\square$ 
Дальше здеь была куча поясняющего текста (см фото 10.11.22 в 13620)

\textbf{Пример.} Посчитаем $F(a)=\int\limits_{0}^{\frac{\pi}{2}}
\frac{\ln(1+a^2\sin^2(x))}{\sin(x)}$. тут я отрубился

\begin{theor}
    (об интегрировании интеграла, зависящего от параметра)\\
    Пусть $f(x,y)$ непрерывна на  $P=[a,b]\times[c,d]$. 
    Тогда
     $$\int\limits_{c}^{d}dy\int\limits_{a}^{b} f(x,y)dx=
     \int\limits_{a}^{b}dy \int\limits_{c}^{d} f(x,y)$$
\end{theor}
\textbf{Доказательство.}  Введем функции $G(t)=\int\limits_{c}^{d}dy
\int\limits_{a}^{t}f(x,y)dy,~H(t)=\int\limits_{a}^{t}dx
\int\limits_{c}^{d}f(x,y)dy$. Докажем, что $G(b)=H(b)$ (что доказывает 
требуемое утверждение). Введем функцию  $g(t,y)=\int\limits_{a}^{t}f(x,y)dx$,
тогда $G(t)=\int\limits_{c}^{d}g(t,y)$ - применима теорема о дифференцировании
сложной функции: $\frac{\partial g}{\partial t}=\left( 
\int\limits_{a}^{t} f(x,y)dx\right)'_t=f(t,y)$ - непрерывна на $P$ по условию. 
Теперь докажем, что $g(t,y)$ непрерывна на $P$, для этого покажем, что 
 $\lim\limits_{\Delta t \to 0,\Delta y\to 0}\Delta g=0$. 
Имеем $\Delta g=g(t+\Delta t,y+\Delta y)-g(t,y)=\int\limits_{a}^{t+\Delta t}
f(x,y+\Delta y)dx-\int\limits_{a}^{t}f(x,y)dx=\int\limits_{a}^{t}(
f(x,y+\Delta y)-f(x,y))dx+\int\limits_{t}^{t+\Delta t}f(x,y+\Delta y)dx$. 
Так как $f(x,y)$ непрерывна на компакте $P$,  то она равномерно непрерывна
на $P$ и ограниченна константой  $M$. Зафиксируем  $\varepsilon>0$. 
Из равномерной непрерывности для 
$$\frac{\varepsilon}{2(b-a)}>0~\exists \delta_1>0~
\forall (x_1,y_1)\in P~\forall (x_2,y_2)\in P:\sqrt{(x_1-x_2)^2-(y_1-y_2)^2}<
\delta_1\implies|f(x_1,y_1)-f(x_2,y_2)|< \frac{\varepsilon}{2(b-a)}$$ 
Если $|\Delta y|<\delta$, то $|f(x,y+\Delta y)-f(x,y)|<
\frac{\varepsilon}{2(b-a)}$; тогда можно оценить интеграл:
$$\left| \int\limits_{a}^{t}(f(x,y+\Delta y)-f(x,y))dx\right|<
\frac{\varepsilon}{2}\cdot \frac{t-a}{b-a}\leqslant \frac{\varepsilon}{2}$$ 
Еcли $|\Delta t|< \frac{\varepsilon}{2M}$, то 

$\square$ \\












