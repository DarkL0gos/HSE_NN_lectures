\chapter{Интеграл, зависящий от параметра}
\section{Собственный интеграл, зависящий от параметра}
\begin{defin}
    Пусть функция $f(x,y)$ интегрируема по Риману (по $x$) 
    на отрезке $[a(y),b(y)]$
    при любом значении параметра  $y\in Y$. Тогда собственный интеграл - 
    $F(y)=\int\limits_{a(y)}^{b(y)}f(x,y)dx$.
\end{defin}

\begin{theor}
    (о непрерывности собственного интеграла, зависящего от параметра)\\
    Пусть\\
    1. $f(x,y)$ определена на  $D=[a(y),b(y)]\times [c,d]=X\times Y$;\\
    2. $a(y),b(y)$ непрерывны на  $[c,d]$, причем
    $a\leqslant a(y)\leqslant b(y)\leqslant b$;\\
    Тогда $F(y)=\int\limits_{a(y)}^{b(y)}f(x,y)dx$ непрерывна на $[c,d]$.
\end{theor}
\textbf{Доказательство.} Покажем, что $\forall y_0\in [c,d]:
\lim\limits_{y \to y_0}F(y)=F(y_0)$. Сделаем линейную замену
$x=a(y)+(b(y)-a(y))t$, $dx=(b(y)-a(y))dt$. Тогда
$$F(y)=\int\limits_{a(y)}^{b(y)}f(x,y)dx=\int\limits_{0}^{1}
f(a(y)+(b(y)-a(y))t,y)\cdot (b(y)-a(y))dt$$
Обозначим $g(t,y)=f(a(y)+(b(y)-a(y))t,y)\cdot (b(y)-a(y))$. Эта функция 
непрерывна на $P=[0,1]\times [c,d]$. Это множество - компакт, поэтому если 
функция непрерывна на компакте, то она и равномерно непрерывна на нем:
$$\forall \varepsilon>0~\exists \delta>0~\forall (t_1,y_1),(t_2,y_2)\in P:
\sqrt{(t_2-t_1)^2+(y_2-y_1)^2}<\delta\implies $$
$$\implies |g(t_1,y_1)-g(t_2,y_2)|<\varepsilon$$ 
Это означает, что
$$|F(y)-F(y_0)|=\left| \int\limits_{0}^{1}\big(g(t,y)-g(t,y_0)\big)dt\right|
\leqslant \int\limits_{0}^{1}|g(t,y)-g(t,y_0)|\leqslant 
$$
$$\leqslant \varepsilon\cdot \int\limits_{0}^{1}dt=\varepsilon$$
Этим доказано, что если $|y-y_0|<\delta$, то и $|g(t,y)-g(t,y_0)|<
\varepsilon$, то есть $\lim\limits_{y \to y_0}F(y)=F(y_0)$. $\square$ 

\begin{theor}
    (о предельном переходе под знаком собственного интеграла, зависящего от 
    параметра)\\
    Пусть\\
    1. $f(x,y)$ определена и непрерывна на  $D=[a(y),b(y)]\times [c,d]$;\\
    2. $a(y),b(y)\in C[c,d]$;\\
    Тогда 
    $$\lim\limits_{y \to y_0}\int\limits_{a(y)}^{b(y)}f(x,y)dx=
    \int\limits_{\lim\limits_{y \to y_0}a(y) }^{\lim\limits_{y \to y_0}b(y)}
   \lim\limits_{y \to y_0}f(x,y)dx=\int\limits_{a(y_0)}^{b(y_0)}f(x,y_0)dx$$
\end{theor}
\textbf{Доказательство.}  Доказано в предыдущей теореме. $\square$

\begin{theor}
    (о дифференцируемости собственного интеграла, зависящего от параметра /
    правило Лейбница)\\
    Пусть $f(x,y)$ \\
    1. непрерывна на  $P=[a,b]\times[c,d]$;\\
    2. $\frac{\partial f}{\partial y}(x,y)$ непрерывна на $P$;\\
    Тогда:\\
    1. $F(y)=\int\limits_{a}^{b} f(x,y)dx$ дифференцируема на $[c,d]$;\\
    2. $F'(y)=\int\limits_{a}^{b}\frac{\partial f}{\partial y}(x,y)dx$
\end{theor}
\textbf{Доказательство.} Пусть $y\in[c,d],~y+h\in[c,d]$. 
Рассмотрим  $F(y+h)-F(y)=\int\limits_{a}^{b} (f(x,y+h)-f(x,y))dx$, 
значит, по теореме Лагранжа это равно $\int\limits_{a}^{b}
\frac{\partial f}{\partial y}(x,y+\theta h)h\,dx$, где $\theta\in(0,1)$. 
Дифференцируем:
$F'(y)=\lim\limits_{h \to 0}\frac{F(y+h)-F(y)}{h}=\lim\limits_{h \to 0}
\int\limits_{a}^{b} \frac{\partial f}{\partial y}(x,y+\theta h)dx$.
При $h\to 0$ делаем замену  $u=y+\theta h,~u\to y$. Тогда предел
 $\lim\limits_{h \to 0}
\int\limits_{a}^{b} \frac{\partial f}{\partial y}(x,u)dx=$
по теореме о предельном переходе $=\int\limits_{a}^{b}
\frac{\partial F}{\partial y}(x,y)dx$. $\square$ 

Следующая теорема обощает правило Лейбница:
\begin{theor} (обобщенное правило Лейбница)\\
    Пусть\\
    1. $f(x,y)$ непрерывна на $D=\{(x,y)\mid a(y)\leqslant x\leqslant 
    b(y),c\leqslant y\leqslant d\}$;\\
    2. $\frac{\partial f}{\partial x}(x,y)$ непрерывна на $D$;\\
    3. $a'(y),b'(y)$ непрерывны на  $[c,d]$.\\
    Тогда $F(y)=\int\limits_{a(y)}^{b(y)}f(x,y)dx$ дифференцируема на 
     $y\in[c,d]$, причем  
     $$F'(y)=\int\limits_{a(y)}^{b(y)}\frac{\partial f}{\partial y}(x,y)dx
     +f(b(y),y)\cdot b'(y) - f(a(y),y)\cdot a'(y)$$
\end{theor}
\textbf{Доказательство.}  Рассмотрим $F(y)=F(y,a(y),b(y))$ как сложную
функцию. По правилу производной сложной функции
$$\frac{dF}{dy}=\frac{\partial F}{\partial y}+
\frac{\partial F}{\partial a}\cdot \frac{\partial a}{\partial y} + 
\frac{\partial F}{\partial b}\cdot \frac{\partial b}{\partial y}$$
Применяя правило Лейбница, получаем
$$\frac{dF}{dy} = \frac{d}{dy}\bigg(\int\limits_{a(y)}^{b(y)}f(x,y)dx\bigg) = 
\int\limits_{a(y)}^{b(y)}\frac{\partial f}{\partial y}(x,y)dx$$
Беря частную производную по $b$, получаем
 $$\frac{\partial F}{\partial b} = 
 \frac{d}{db(y)}\bigg(\int\limits_{a(y)}^{b(y)}f(x,y)dx\bigg) = 
 f\big(b(y),y\big)$$
 Аналогично, $\frac{\partial F}{\partial a}=-f\big(a(y),y\big)$. Из этих
 соотношений и получаем искомую формулу. $\square$

%Дальше здеь была куча поясняющего текста (см фото 10.11.22 в 13620)

%\textbf{Пример.} Посчитаем $F(a)=\int\limits_{0}^{\frac{\pi}{2}}
%\frac{\ln(1+a^2\sin^2(x))}{\sin(x)}$. тут я отрубился

\begin{theor}
    (об интегрировании интеграла, зависящего от параметра)\\
    Пусть $f(x,y)$ непрерывна на  $P=[a,b]\times[c,d]$. 
    Тогда
     $$\int\limits_{c}^{d}dy\int\limits_{a}^{b} f(x,y)dx=
     \int\limits_{a}^{b}dx \int\limits_{c}^{d} f(x,y)dy$$
\end{theor}
\textbf{Доказательство.}  Введем функции $G(t)=\int\limits_{c}^{d}dy
\int\limits_{a}^{t}f(x,y)dy,~H(t)=\int\limits_{a}^{t}dx
\int\limits_{c}^{d}f(x,y)dy$. Докажем, что $G(b)=H(b)$ (что доказывает 
требуемое утверждение). Введем функцию  $g(t,y)=\int\limits_{a}^{t}f(x,y)dx$,
тогда $G(t)=\int\limits_{c}^{d}g(t,y)$ - применима теорема о дифференцировании
сложной функции: $\frac{\partial g}{\partial t}=\left( 
\int\limits_{a}^{t} f(x,y)dx\right)'_t=f(t,y)$ - непрерывна на $P$ по условию. 
Теперь докажем, что $g(t,y)$ непрерывна на $P$, для этого покажем, что 
 $\lim\limits_{\Delta t \to 0,\Delta y\to 0}\Delta g=0$. 
Имеем $\Delta g=g(t+\Delta t,y+\Delta y)-g(t,y)=\int\limits_{a}^{t+\Delta t}
f(x,y+\Delta y)dx-\int\limits_{a}^{t}f(x,y)dx=\int\limits_{a}^{t}(
f(x,y+\Delta y)-f(x,y))dx+\int\limits_{t}^{t+\Delta t}f(x,y+\Delta y)dx$. 
Так как $f(x,y)$ непрерывна на компакте $P$,  то она равномерно непрерывна
на $P$ и ограниченна константой  $M$. Зафиксируем  $\varepsilon>0$. 
Из равномерной непрерывности для 
$$\frac{\varepsilon}{2(b-a)}>0~\exists \delta_1>0~
\forall (x_1,y_1)\in P~\forall (x_2,y_2)\in P:\sqrt{(x_1-x_2)^2-(y_1-y_2)^2}<
\delta_1$$ 
$$\implies|f(x_1,y_1)-f(x_2,y_2)|< \frac{\varepsilon}{2(b-a)}$$
Если $|\Delta y|<\delta$, то $|f(x,y+\Delta y)-f(x,y)|<
\frac{\varepsilon}{2(b-a)}$; тогда можно оценить интеграл:
$$\left| \int\limits_{a}^{t}(f(x,y+\Delta y)-f(x,y))dx\right|<
\frac{\varepsilon}{2}\cdot \frac{t-a}{b-a}\leqslant \frac{\varepsilon}{2}$$ 
Еcли $|\Delta t|< \frac{\varepsilon}{2M}$, то 
$$\left| \int\limits_{t}^{t+\Delta t}f(x,y+\Delta y)dx\right|\leqslant 
\left| \int\limits_{t}^{t+\Delta t}|f(x,y+\Delta y)|dx\right|\leqslant 
M\cdot \frac{\varepsilon}{2M}=\frac{\varepsilon}{2}$$
Пусть $\delta=\min \{\delta,\frac{\varepsilon}{2M}\}$, тогда
$$\forall \varepsilon>0~\exists \delta(\varepsilon)>0~\forall |\Delta t|
<\delta~\forall |\Delta y|<\delta:|\Delta g|<\varepsilon$$
Это означает, что $\lim\limits_{\substack{\Delta t\to 0\\
\Delta y\to 0}}\Delta g=0$. Заметим, что $g(t,y)$ непрерывна на  $P$.
Применим теорему о дифференцировании к функции $G(t)$: 
$$G'(t)=\int\limits_{c}^{d} \frac{\partial g}{\partial t}(t,y)dy=
\int\limits_{c}^{d}f(t,y)dy$$
С другой стороны,
$$H'(t)=\frac{\partial}{\partial t}\left( \int\limits_{a}^{t}dx
\int\limits_{c}^{d}f(x,y)dy \right)=\int\limits_{c}^{d}f(t,y)dy$$
Итак, мы получили, что $G'(t)=H'(t),~G(a)=H(a)=0$, откуда
$G(t)=H(t)\implies G(b)=H(b)$. $\square$ \\












