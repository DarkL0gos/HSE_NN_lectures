\documentclass[a4paper]{article}

%Общие настройки документа
\usepackage[14pt]{extsizes}                                         %Размер шрифта
\usepackage[left=2.5cm,right=2.5cm,top=2.5cm,bottom=3cm]{geometry}  %Поля страницы

%Настройки ссылок и гиперссылок
\usepackage{float}
%\usepackage{graphicx}
%\usepackage{hyperref}                 
%\usepackage{xcolor}
%\definecolor{linkcolor}{HTML}{799B03} % цвет ссылок
%\definecolor{urlcolor}{HTML}{799B03}  % цвет гиперссылок
%\hypersetup{pdfstartview=FitH,linkcolor=linkcolor,urlcolor=urlcolor,colorlinks=true}
%graphicspath{{\figures}}


%Пакеты символов
\usepackage{cmap}
\usepackage[T2A]{fontenc}
\usepackage[utf8]{inputenc}
\usepackage[russian]{babel}           
\usepackage{amsmath}
\usepackage{amssymb}
\usepackage{amsfonts}

%Новые команды 
\newtheorem{defin}{Определение}
\newtheorem{example}{Пример}
\newtheorem{zam}{Замечание}
\newtheorem{theor}{Теорема}

\author{}
\title{Практика из билетов}
\date{22.12.22}

\begin{document}
\maketitle
%\tableofcontents
%\newpage
\textbf{№1.}
$$\int\limits_{0}^{\infty}\frac{x\cos x-\sin x}{x^2}dx = $$ 
Особые точки - оба предела. 
$$= \int\limits_{0}^{\infty}\left( \frac{\sin x}{x} \right)'dx = 
\underbrace{
\int\limits_{0}^{\pi} \left( \frac{\sin x}{x} \right)'dx + 
\int\limits_{\pi}^{\infty} \left( \frac{\sin x}{x} \right)'dx}_\text{разбиваем,
чтобы было по одной особой точке} = 
$$
$$= \frac{\sin x}{x}\bigg|^\pi_0+\frac{\sin x}{x}\bigg|^\infty_\pi = 
0 - \lim\limits_{x \to 0}\frac{\sin x}{x}+\lim\limits_{x \to \infty} 
\frac{ \overbrace{\sin x}^\text{ограничен} }{\underbrace{x}_{\to 0}}
= 0-1+0 = -1 $$

\textbf{№2.} $$\int\limits_{0}^{2} \frac{dx}{1-x^2}$$ 
Имеем разрыв второго рода в точке 1. Посчитаем главное значение интеграла:
$$v.p.\int\limits_{0}^{2} \frac{dx}{1-x^2}=
\lim\limits_{\delta \to 0} \left( \int\limits_{a}^{c-\delta}f(x)dx -
\int\limits_{c+\delta}^{b}f(x)dx\right) = \frac{1}{2}\ln\bigg|
\frac{1+x}{1-x}\bigg|\Big|^2_0 = \frac{1}{2}\ln3
$$
Покажем, что интеграл расходится по отдельности:
$$\int\limits_{0}^{1}\frac{dx}{1-x^2}+\int\limits_{1}^{2}\frac{dx}{1-x^2}=
\lim\limits_{\delta \to +0} \int\limits_{0}^{1-\delta}\frac{dx}{1-x^2}+
\lim\limits_{\varepsilon\to +0} 
\int\limits_{1+\varepsilon}^{2}\frac{dx}{1-x^2}=
$$
$$=\lim\limits_{\delta \to +0}\frac{1}{2}\ln \frac{1+x}{1-x}\Big|^{1-\delta}_0
+\lim\limits_{\varepsilon \to +0}\frac{1}{2}\ln \frac{1+x}{1-x}
\Big|^2_{1+\varepsilon}=
\frac{1}{2}\left( \lim\limits_{\delta \to +0}\ln \frac{2-\delta}{\delta}+
\lim\limits_{\varepsilon \to +0}\ln \frac{2+\varepsilon}{\varepsilon}
\right)+
$$
$$+\frac{\ln3}{2}=\frac{\ln3}{2}+\frac{1}{2}
\lim\limits_{\substack{\delta \to+0\\
\varepsilon \to +0}} \ln\frac{\varepsilon}{\delta}$$
Предела не существует, поэтому интеграл расходится.

\end{document}
