\section{Связь признака Даламбера и Коши}
Если $\frac{a_n}{a_{n-1}}\leqslant q$ для 
всех n начиная с 1, то $a_n=a_1q^n$, откуда следует признак Коши. 
$$\sqrt[n]{a_n}\leqslant \sqrt[n]{a_1}\cdot q$$
Значит, Коши покрывает больше случаев. 
\section{Оценка погрешности приближения какой-то величины с помощью
положительного ряда}
$$\int^\infty_{n+1} f(x)dx<R_n\leqslant \int^\infty_nf(x)dx$$ 
Из доказательства интегрального признака
$$a_{k+1}<\int^{k+1}_kf(x)dx\leqslant a_k$$ 
$$\int^{k+1}_kf(x)dx\leqslant a_k\int^k_{k-1}f(x)dx$$ 
$$R_n=\sum\limits_{k=n+1}^{\infty} a_k$$ 
Итак, 
$$\int^\infty_{n+1}\leqslant R_n<\int^\infty_nf(x)dx$$
\textbf{Пример.} Вычислим с точностью до 0,001 ряд 
$\sum\limits_{n=1}^{\infty} \frac{1}{n^4}$. Ответ: $1,082\pm0,001$
(точный ответ $\frac{\pi^4}{90}$)
\section{Знакопеременные ряды}
Пусть теперь ряд знакопеременный.
\begin{defin}
Ряд сходится абсолютно, если сходится ряд из модулей. Ряд сходится условно,
если абсолютно расходится, но сам сходится. 
\end{defin}
\begin{theor}
Если ряд сходится абсолютно, то ряд сходится.
\end{theor}
\textbf{Доказательство.}  Следует напрямую из критерия Коши
и свойства модуля: $| |a_1|+...|a_n| |\geqslant|a_1+...+a_n|$.
$\square$ 
\begin{theor}
    (признак Лейбница для знакочередующихся рядов)\\
    Пусть ряд имеет вид $\sum\limits_{n=1}^{\infty} (-1)^nv_n$, 
    где $v_n>0$ и монотонно убывает. Тогда ряд сходится.  
    Более того, имеет место оценка погрешности $|R_n|\leqslant v_n$
\end{theor}
\textbf{Доказательство.}  1. Посчитаем частичную сумму для $2k:$
$$S_{2k}=v_1-v_2+...-v_{2k}$$ 
$$S_{2k+2}=S_{2k}+v_{2k+1}-v_{2k+2}$$ 
$$S_{2k+2}-S_{2k}=v_{2k+1}-v_{2k+2}$$ 
$$S_{2k}=v_1-(v_2-v_3)-(v_4-v_5)-...-(v_{2k-2}-v_{2k-1})-v_{2k}$$ 
Значит, эта последовательность возрастает и ограничена сверху, значит, у неё
есть конечный предел: $S_{2k}\leqslant u_1$
$$\lim\limits_{k \to \infty} S_{2k+1}=\lim\limits_{k \to \infty} (S_{2k}+
v_{2k+1})=S$$ 
Следовательно, 
$$\exists \lim\limits_{n \to \infty} S_n=S$$
Последовательность частичных сумм для нечетных чисел также убывает, 
доказательство аналогичное. \\
2. Докажем оценку погрешности. $|R_{2k}|=S-S_{2k}<S_{2k+1}-S_{2k}$. Итак,
$$|R_{2k}|\leqslant v_{2k+1}$$ 
$$R_{2k+1}=S_{2k+1}-S<S_{2k+1}-S_{2k+2}$$ 
$$|R_{2k+1}|\leqslant v_{2k+2}$$
$\square$ 
\subsection{Преобразование Абеля}
$$\sum\limits_{k=1}^{n} a_k b_k=\sum\limits_{k=1}^{n-1} (a_k-a_{k+1})B_k
+a_nB_n,~B_i=\sum\limits_{k=1}^{i} b_k$$ 
Доказательство. $b_k=B_{k}-B_{k-1},~k\in \{2,...,n\} $ ВСТАВКА

\begin{theor}
    (неравенство Абеля)\\
    Пусть последовательность монотонно возрастает или убывает. 
    И пусть $\exists M\forall k\in \{1...n\}|B_k|\leqslant M $.
    Тогда модуль конечной суммы $\leqslant M(|a_1|+2|a_n|)$ 
\end{theor}
\textbf{Доказательство.} Юра, допиши пж 
$\square$ 
\begin{theor}
    (признак Дирихле)


\end{theor}
\textbf{Доказательство.}  \
$\square$ 











