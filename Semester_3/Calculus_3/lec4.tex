\subsection{Знакопеременные ряды}
Переходим к исследованию рядов с произвольным знаком. Иногда бывает 
полезно рассмотреть этот ряд с числами постоянного знака, что мотивирует 
следующее определение.
\begin{defin}
Ряд сходится абсолютно, если сходится ряд, составленный из модулей членов
этого ряда. Ряд сходится условно, если он сходится, но расходится 
абсолютно. 
\end{defin}
\begin{theor}
Если ряд сходится абсолютно, то ряд сходится.
\end{theor}
\textbf{Доказательство.}  Следует напрямую из критерия Коши
и свойства модуля: $\left||a_1|+...|a_n|\right| \geqslant|a_1+...+a_n|$.
$\square$ 
\begin{theor}
    (признак Лейбница для знакочередующихся рядов)\\
    Пусть ряд имеет вид $\sum\limits_{n=1}^{\infty} (-1)^nv_n$, 
    где $v_n>0$ и монотонно убывает. Тогда ряд сходится.  
    Более того, имеет место оценка погрешности $|R_n|\leqslant v_n$
\end{theor}
\textbf{Доказательство.}  1. Посчитаем частичную сумму для $2k:$
$$S_{2k}=v_1-v_2+...-v_{2k}$$ 
$$S_{2k+2}=S_{2k}+v_{2k+1}-v_{2k+2}$$ 
$$S_{2k+2}-S_{2k}=v_{2k+1}-v_{2k+2}$$ 
$$S_{2k}=v_1-(v_2-v_3)-(v_4-v_5)-...-(v_{2k-2}-v_{2k-1})-v_{2k}$$ 
Значит, эта последовательность возрастает и ограничена сверху, значит, у неё
есть конечный предел: $S_{2k}\leqslant u_1$
$$\lim\limits_{k \to \infty} S_{2k+1}=\lim\limits_{k \to \infty} (S_{2k}+
v_{2k+1})=S$$ 
Следовательно, 
$$\exists \lim\limits_{n \to \infty} S_n=S$$
Последовательность частичных сумм для нечетных чисел также убывает, 
доказательство аналогичное. \\
2. Докажем оценку погрешности. $|R_{2k}|=S-S_{2k}<S_{2k+1}-S_{2k}$. Итак,
$$|R_{2k}|\leqslant v_{2k+1}$$ 
$$R_{2k+1}=S_{2k+1}-S<S_{2k+1}-S_{2k+2}$$ 
$$|R_{2k+1}|\leqslant v_{2k+2}$$
$\square$ 
\begin{theor} (Преобразование Абеля)\\
    Пусть $B_i=\sum\limits_{k=1}^{i} b_k$. Тогда
$$\sum\limits_{k=1}^{n} a_k b_k=\sum\limits_{k=1}^{n-1} (a_k-a_{k+1})B_k
+a_nB_n$$ 
\end{theor}
\textbf{Доказательство.} $b_k=B_{k}-B_{k-1},~k\in \{2,...,n\}$. Тогда
$\sum\limits_{k=1}^{n} a_kb_k=a_1b_1+\sum\limits_{k=2}^{n}a_k
\cdot (B_k-B_{k-1})=a_1b_1+\sum\limits_{k=2}^{n}a_kb_k-\sum\limits_{k=1}
^{n-1}a_{k+1}b_k=a_1b_1=\sum\limits_{k=2}^{n-1}(a_k-a_{k+1})\cdot B_k+
+a_nb_n-a_2b_1=\sum\limits_{k=1}^{n} a_k b_k=\sum\limits_{k=1}^{n-1} 
(a_k-a_{k+1})B_k+a_nB_n$
$\square$ 
\begin{theor}
    (неравенство Абеля)\\
    Пусть последовательность $a_n$ монотонно возрастает или убывает, 
    и пусть существует константа $M$ такая, что 
    $\forall k\in \{1...n\}:~|b_1+b_2+...b_n|\leqslant M$ (то есть 
    она ограничивает модуль частичных сумм ряда $B$). Тогда 
    $$\left| \sum\limits_{k=1}^{n}a_kb_k \right|\leqslant M(|a_1|+2|a_n|)$$
\end{theor}
\textbf{Доказательство.} Применим преобразование Абеля к ряду с общим 
членом $a_nb_n$: $\left| \sum\limits_{k=1}^{n}a_kb_k \right|=
\left| \sum\limits_{k+1}^{n-1}(a_k-a_{k+1})B_k+a_nB_n \right|\leqslant 
\left| \sum\limits_{k+1}^{n-1}(a_k-a_{k+1})B_k\right|+|a_n|\cdot|B_n|\leqslant 
M\cdot \left| \sum\limits_{k+1}^{n-1}(a_k-a_{k+1}) \right|+M|a_n|\leqslant 
M(|a_1|+2|a_n|)$. $\square$ 
\begin{theor} (признак Дирихле)\\
Пусть общий член ряда имеет вид $a_nb_n$
Тогда если:\\
1. Последовательность $a_n$ монотонна и $\lim\limits_{n \to \infty} a_n=0$;\\
2. Последовательность частичных сумм $b_n$ ограничена константой $B$;\\
Тогда ряд $\sum\limits_{n=1}^{\infty} a_nb_n$ сходится.
\end{theor}
\textbf{Доказательство.} Используем критерий Коши. 
Зафиксируем $\varepsilon>0$.
По условию, предел последовательности $a_n$ равен нулю, 
тогда для $$\frac{\varepsilon}{6B}>0~\exists n_0=n_0(\varepsilon)\in\mathbb{N}
~\forall n>n_0:|a_n|<\frac{\varepsilon}{6B}$$ 
Пусть $p\in\mathbb{N}$. Рассмотрим $\left| \sum\limits_{k=n+1}^{n+p}a_kb_k
\right|$. Подберем константу из неравенства Абеля:
$\left| \sum\limits_{k=n+1}^{n+i}b_k\right|=|B_{n+i}-B_n|\leqslant 
|B_{n+i}|+|B_n|\leqslant 2B=M$. Значит, из неравенства Абеля получаем
$\left|\sum\limits_{k=n+1}^{n+p}a_kb_k\right|\leqslant 2B(|a_{n+1}|+
2|a_{n+p}|)<2B\cdot (\frac{\varepsilon}{6B}+\frac{2\varepsilon}{6B})=
\varepsilon$. $\square$ 

\textbf{Пример.} $\sum\limits_{n=1}^{\infty} \frac{\sin{n\alpha}}{n}$. 
По признаку Дирихле ряд сходится, так как частичные суммы синуса 
арифметической прогрессии сходятся.

\begin{theor}
    (признак Абеля)\\
    Пусть общий член ряда имеет вид $a_nb_n$. Тогда если \\
    1. Последовательность  $a_n$ монотонна и ограничена константой $M$;\\
    2. Ряд  $\sum\limits_{n=1}^{\infty}b_n$ сходится.\\
Тогда ряд $\sum\limits_{n=1}^{\infty} a_nb_n$ сходится.
\end{theor}
\textbf{Доказательство.}  Докажем по критерию Коши. Зафиксируем 
$\varepsilon>0$. Так как сходится ряд с общим членом $b_n$, 
то по критерию Коши для 
$$\frac{\varepsilon}{3M}>0~\exists n_0~\forall n>n_0~\forall p\in\mathbb{N}:
\left| \sum\limits_{k=n+1}^{n+p} b_k \right| <\frac{\varepsilon}{3M}$$
Из неравенства Абеля получаем
$\left|\sum\limits_{k=n+1}^{n+p}a_k b_k\right| \leqslant
\frac{\varepsilon}{3M}(|a_{n+1}|+2|a_{n+p}|)<\frac{\varepsilon}{3M}(M+2M)
=\varepsilon$. Итак, 
$$\forall \varepsilon>0~\exists n_0~\forall n>n_0~\forall p\in\mathbb{N}:
\left| \sum\limits_{k=n+1}^{n+p} a_kb_k \right|<\varepsilon\quad\square$$
\textbf{Упражнение.} Доказать признак Абеля, используя признак Дирихле.\\
Решение. Условие означает, что признак Дирихле более общий, чем признак
Дирихле, поэтому покажем, что если ряд удовлетворяет условиям признака 
Абеля, то он удовлетворяет и признаку Дирихле. 

Пусть дан ряд $\sum\limits_{n=1}^{\infty} a_nb_n$. По 
условию признака Абеля, последовательность $a_n$ монотонна и ограниченна,
поэтому у её есть конечный предел $a$, %Также по условию 
%$\sum\limits_{n=1}^{\infty} b_n=b$,
поэтому исходный ряд можно представить в виде
$$a\sum\limits_{n=1}^{\infty} b_n+\sum\limits_{n=1}^{\infty} b_n(a_b-a)$$ 
Первый ряд сходится по условию, второй удовлетворяет признаку Дирихле. 

\textbf{Пример.} $\sum\limits_{n=2}^{\infty} (\sin{n\alpha}
\cos{\frac{\pi}{n}})/\ln{\ln{n}}$. Косинус монотонный и ограниченный, 
а все остальное сходится по Дирихле. Значит,ряд сходится по Абелю.

\begin{theor}
    (признак Даламбера для знакопеременных рядов)\\
    Пусть $a_n$ - общий член знакопеременного ряда, 
    и $\lim\limits_{n \to \infty} \frac{|a_{n+1}|}{|a_n|}=q$.
    Тогда:\\
    1. Если $0\leqslant q<1$, то ряд сходится абсолютно.\\
    2. Если $q>1$, то ряд расходится.\\
    Если $q=1$, ничего нельзя сказать.
\end{theor}
\textbf{Доказательство.} 
1. Следует из признака Даламбера для знакопостоянных рядов.\\
2. Пусть существует предел 
$\lim\limits_{n \to \infty} \frac{|a_{n+1}|}{|a_n|}=q>1$ Для
$$\varepsilon=q-1>0~\exists n_0~\forall n>n_0:\frac{|a_{n+1}|}{|a_n|}>1$$
откуда $\left|\frac{|a_{n+1}|}{|a_n|}-q\right|<\varepsilon$, 
поэтому $1<\left| \frac{a_{n+1}}{a_n} \right|<q+1$, значит не выполняется
необходимый признак. $\square$ 
\begin{theor}
    (признак Коши для знакопеременных рядов)\\
    Пусть $a_n$ - общий член знакопеременного ряда, 
    и $\lim\limits_{n \to \infty} \sqrt[n]{a_n}=q$.
    Тогда:\\
    1. Если $0\leqslant q<1$, то ряд сходится абсолютно.\\
    2. Если $q>1$, то ряд расходится.\\
    Если $q=1$, ничего нельзя сказать.

\end{theor}
\textbf{Доказательство.}
1. Следует из признака Коши для положительных рядов.\\
2. Проводится аналогично доказательству п.2 в признаке Даламбера.
$\square$ 

Признак сравнения для знакопеременных рядов не работает. Приведем 
пример: $a_n=\frac{(-1)^{n+1}}{n},~b_n=a_n+\frac{1}{(n+1)\ln(n+1)}$. 
Предел отношения таких рядов равен 1, то есть они эквивалентны, но вот
первый сходится, а второй - расходится %(см. общий пример с степенями и 
%логарифмами). 
\subsection{Свойства абсолютно сходящихся рядов}
\textbf{Лемма.} Если ряд сходится абсолютно, то модуль его суммы не 
превосходит суммы его модулей. Лемма следует из неравенства
$$\left| a_1+a_2+...+a_n \right|\leqslant |a_1|+|a_2|+...+|a_n|$$
\begin{theor} (о перестановках в абсолютно сходящемся ряде)\\
    Пусть дан ряд с общим членом $a_n$, и он сходится абсолютно.
    Пусть его сумма равна $S$, сумма из модулей равна $\overline{S}$. 
    Обозначим соответствующие частичные суммы как $S_n,~\overline{S_n}$.
    Рассмотрим ряд $\sum\limits_{n=1}^{\infty} a^*_n$ с переставленными
    членами исходного ряда, обозначим его сумму
    и частичную сумму как $S^*,~S^*_n$ (для ряда из модулей соответственно
    $\overline{S^*},~\overline{S^*_n}$). Тогда:\\
    1. $S^*$ существует и равна  $S$;\\
    2. Ряд из  $a^*_n$ сходится абсолютно.
    \end{theor}
\textbf{Доказательство.} 1.  
По условию, $\sum\limits_{n=1}^{\infty} |a_n|=\overline{S}$. Тогда
$$\forall \varepsilon>0~\exists n_0~\forall n>n_0:|\overline{S_n}-
\overline{S}|<\frac{\varepsilon}{2}$$ 
Из леммы следует, что $|S-S_n|<\frac{\varepsilon}{2}$. 
Перейдем к переставленному ряду. Выберем в нем такой номер $m_0$, 
чтобы частичная сумма $S^*_{m_0}$ содержала все слагаемые, входящие в 
$S_{n_0}$. Тогда для любого числа $m>m_0$ имеем
$$|S^*_{m}-S_{n_0}|<|\overline{S}|<\frac{\varepsilon}{2}$$
Тогда $|S^*_m-S|=|S^*_m-S_{n_0}+S_{n_0}-S|\leqslant 
|S^*_m-S_{n_0}|+|S_{n_0}-S|<\frac{\varepsilon}{2}+\frac{\varepsilon}{2}=
\varepsilon$. 
В итоге мы доказали сходимость ряда с переставленными членами и равенство 
его суммы сумме исходного ряда:
$$\forall \varepsilon>0~\exists m_0\in\mathbb{N}~\forall m>m_0:|S^*_m-S|
<\varepsilon$$ 
2. Абсолютная сходимость следует из таких же рассуждений для ряда с модулем. 
$\square$ 
