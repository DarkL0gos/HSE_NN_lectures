\subsection{Знакопеременные ряды}
Переходим к исследованию рядов с произвольным знаком. Иногда 

Теперь перейдем к случаю, когда общий член ряда является произведением 
каких-то штук. Здесь нас спасают признаки Дирихле и Абеля.
\begin{theor} (признак Дирихле)\\
Пусть общий член ряда имеет вид $a_nb_n$
Тогда если:\\
1. $a_n$ монотонна и $\lim\limits_{n \to \infty} a_n=0$;\\
2. Последовательность частичных сумм $b_n$ ограниченна;\\
Тогда ряд $\sum\limits_{n=1}^{\infty} a_nb_n$ сходится.
\end{theor}
\textbf{Доказательство.} Используем критерий Коши. 
Зафиксируем $\varepsilon>0$.
По условию, предел ряда А равен нулю, тогда для $\frac{\varepsilon}{6B}>0$,
$\exists n_0=n_0(\varepsilon)\in\mathbb{N}\forall n>n_0:
|a_n|<\frac{\varepsilon}{6B}$ 


$\square$ 

\textbf{Пример.} $\sum\limits_{n=1}^{\infty} \frac{\sin{n\alpha}}{n}$. 
По признаку Дирихле ряд сходится, так как частичные суммы синуса 
арифметической прогрессии сходятся.

\begin{theor}
    (признак Абеля)\\
    Пусть общий член ряда имеет вид $a_nb_n$. Тогда если \\
    1. Последовательность  $a_n$ монотонна и ограниченна\\
    2. Последовательность  $b_n$ сходится.
\end{theor}
\textbf{Доказательство.}  Докажем по критерию Коши. Зафиксируем 
$\varepsilon>0$. Так как сходится ряд $b_n$, то по критерию Коши для 
 $\frac{\varepsilon}{3M}>0$ найдется такой номер, начиная с которого модуль
 суммы p членов ряда $b_n$ меньше, чем эта штука. Из неравенства Абеля получим
$|\sum\limits_{k=n+1}^{n+p} a_k b_k\leqslant \frac{\varepsilon}{3M}$

$\square$ 
\textbf{Упражнение.} Доказать признак Абеля, используя признак Дирихле. \\
\textbf{Пример.} $\sum\limits_{n=2}^{\infty} (\sin{n\alpha}
\cos{\frac{\pi}{n}})/\ln{\ln{n}}$. Косинус монотонный и ограниченный, 
а все остальное сходится по Дирихле. Значит,ряд сходится по Абелю.\\





Сформулируем признаки Коши и Даламбера для знакопеременных рядов.
Доказательство чекаем в Фихтенгольце.
\begin{theor}
    (признак Даламбера)\\
    Пусть $a_n$ - общий член знакопеременного ряда. 
    Пусть  $\lim\limits_{n \to \infty} \frac{|a_{n+1}|}{|a_n|}=q$.
    Получаем классическую абсолютную сходимость. \\

\end{theor}
\textbf{Доказательство.} ПОЛНОСТЬЮ следует из признака Даламбера для 
знакопостоянных рядов. Единственно, что здесь нового - то, что при абсолютной
расходимости в признаке Даламбера будет и условная расходимость, поскольку не 
выполняется необходимое условие сходимости ряда. Для некоторого эпсилон...
$|\frac{|a_{n+1}|}{|a_n|}|-q|<\varepsilon$ 
$\square$ 
\begin{theor}
    (признак Коши) 
    Все аналогично. Из абсолютной расходимости следует расходимость.
\end{theor}
\textbf{Доказательство.}  \
$\square$ 
Признак сравнения для знакопеременных рядов не работает. Приведем 
контрпример: $a_n=\frac{(-1)^{n+1}}{n},~b_n=a_n+\frac{1}{(n+1)\ln(n+1)}$. 
Предел отношения таких рядов равен 1, то есть они эквивалентны, но вот
первый сходится, а второй - расходится (см. общий пример с степенями и 
логарифмами). 
\subsection{Свойства абсолютно сходящихся рядов}.
\textbf{Лемма.} Если рядсходится абсолютно, то модуль его суммы не превосходит
суммы его модулей.
\begin{theor}
    ()\\
    Пусть дан ряд с общим членом $a_n$, и он сходится абсолютно. Обозначим 
    его сумму,частичную сумму, сумму модулей и частичную сумму модулей как
     $S,S_n,\overline{S},\overline{S_n}$. Тогда, если переставить слагаемые, 
     новый ряд $a^*_n$ сходится абсолютно.
\end{theor}
\textbf{Доказательство.} Ещё один ворох обозначений: $\overline{S^*_n},S^*_n$.
Длялюбого эпсилон найдется номер такой, что $|\overline{S_n}-\overline{S}|<
\frac{\varepsilon}{2}$. Из леммы следует, что $|S-S_n|<\frac{\varepsilon}{2}$.
Перейдем к переставленному ряду. Выберем в нем номер, чтобы такая частичная
сумма содержала все слагаемые, входящие в $S_{n(\varepsilon)}$. Взяв 
любое число $m$ большее этого номера,  $|S^*_m-S_{n(\varepsilon)}|<
|\overline{S}|<\frac{\varepsilon}{2}$. В эту сумму ои все вошли. Остались
толкьо те, которые ???
$|S^*_m-S|=|S^*_m-S_{n(\varepsilon)}+S_{n(\varepsilon)}-S|\leqslant 
2\cdot \frac{\varepsilon}{2}$. Мы доказали сходимость ряда.
Абсолютная сходимость следует из таких же рассуждений для ряда с модулем. 


$\square$ 
