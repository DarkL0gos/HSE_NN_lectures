\chapter{Идеи и номера с практики}
Идеи Тимура, достойные того, чтобы быть запечатленными.
Те места, которые на слух отмечаются словами типа <<финт ушами>>, будут 
отмечаться знаком <<опасный поворот>>~\dbend в стиле Бурбаки (а не то, что
вы подумали). 

\section{Знакопостоянные несобственные интегралы}
Для знакопостоянных интегралов можно использовать признак сравнения.
Обычно сравнение происходит с обобщенной степенной функцией.
При этом имеется два различных типа особых точек: на бесконечности и 
с уходом на бесконечность в точке. Разберем подробнее.

\textbf{Пример 1.} Интеграл 
$$\int\limits_{1}^{\infty}\frac{1}{x^\alpha}dx$$ 
сходится при $\alpha>1$ и расходится при $\alpha\leqslant 1$.

\textbf{Пример 2.} Интеграл
$$\int\limits_{a}^{b}\frac{1}{(x-a)^\alpha}dx$$
сходится при $\alpha<1$ и расходится при $\alpha\geqslant 1$. 

\textbf{Пример.} Интеграл $\int\limits_{1}^{\infty} \frac{x^2dx}{x^4-x^2+1}$ 
сходится, поскольку подынтегральная функция эквивалентна $\frac{1}{x^2}$ - 
сходящейся штуке.


\textbf{Пример (№2374)}. Исследуем на сходимость в зависимости от параметров
интеграл
$$\int\limits_{1}^{\infty} \frac{1}{x^p\ln^q x}dx$$
Имеем 2 особые точки: 1 и $\infty$, поэтому разобъем область исследования
на две части и будет исследовать интеграл $\int\limits_{10}^{\infty}$.\\
Нам поторебуется следующий признак сравнения: для $\varepsilon>0$
$$\frac{1}{x^\varepsilon}<\ln^\alpha(x)<x^\varepsilon,~
x>\delta(\alpha,\varepsilon)$$ 
(доказательство через правило Лопиталя: действительно, 
$\lim\limits_{n \to \infty} \frac{\ln^\alpha(x)}{x^\varepsilon}=0$).\\
Значит, имеем
$$\frac{1}{x^{p+\varepsilon}}\leqslant \frac{1}{x^p\ln^qx}\leqslant 
\frac{1}{x^{p-\varepsilon}}$$ 
\dnote{Итак, интеграл сходится при $p>1+\varepsilon$ и расходится при
$p<1-\varepsilon$. Так как $\varepsilon$ вообще-то произвольный,
то и условие сходимости не должно зависеть от него; иначе говоря, 
интеграл сходится при $p>1$ и расходится $p<1$.}
Рассмотрим случай, когда $p=1$. Имеем 
$$\int\limits_{10}^{\infty} \frac{1}{x\ln^qx}dx = 
\begin{cases}\ln(x)=t\\dt=\frac{dx}{x}\end{cases} = 
\int\limits_{\ln 10}^{\infty}\frac{dt}{t^q}$$ 
Значит, этот интеграл сходится при $q>1$.
Соберем ответ:
$$\begin{cases}
    1.~ p>1 - \text{сходится;}\\
    2.~ p<1 - \text{расходится;}\\
    3.~ p=1,q>1 - \text{сходится;}\\
    4.~ p=1,q\leqslant 1 - \text{расходится.}
\end{cases}$$



\section{Знакопеременные несобственные интегралы}
Напомним, что для применения признаков Абеля и Дирихле в интеграле 
$\int\limits_{a}^{\infty} f(x)g(x)dx$, необходимо, чтобы $f(x)$ и $g'(x)$ были 
непрерывными функциями.

\textbf{Пример.} $$\int\limits_{1}^{\infty}\frac{\sin(x)}{x^\alpha}$$ 
Интеграл имеет одну особую точку: $+\infty$.\\
Сначала расмотрим абсолютную сходимость: 
$\frac{|\sin(x)|}{x^\alpha}\leqslant \frac{1}{x^\alpha}$, откуда
по признаку сравнения получаем, что интеграл сходится абсолютно при 
$\alpha>1$.\\
Рассмотрим обычную сходимость: интеграл удовлетворяет признаку Дирихле,
поскольку 
$\forall y>a: \int\limits_{a}^{y}\sin(x)dx=-\cos(y)+\cos(a)\leqslant 20$ и
$\frac{1}{x^\alpha}\to 0$ монотонно. Значит, интеграл сходится при 
$\alpha>0$.\\
Теперь рассмотрим расходимость интерала. Докажем условную сходимость на
$(0,1]$.
Оценим снизу увадратом синуса:
$$\frac{|\sin(x)|}{x^\alpha}\geqslant \frac{\sin^2(x)}{x^\alpha}=
\frac{1-\cos(2x)}{2x^\alpha}=\frac{1}{2x^\alpha}-\frac{\cos(2x)}{2x^\alpha}$$ 
Вторая дробь сходится по Дирихле, откуда весь интеграл расходится абсолютно
при $\alpha\leqslant 1$.\\
Осталось установить рассходимость при $\alpha\leqslant 0$. Вспомним определение
\textbf{предела по Гейне}:
$$\forall \{y_n\}\to 0: \lim\limits_{n \to \infty} \int\limits_{a}^{y_n}
f(x)dx\to const$$ 
Тогда интеграл можно предстваить в виде $\sum\limits_{n=1}^{\infty} 
\int\limits_{y_n}^{y_{n+1}}f(x)dx$. Найдем какую-нибудь последовательность,
на которой будет расходимость. Итак, пусть $y_n=\pi n$.\\
Теперь нам потребуется следующая 
\begin{theor}(о среднем)\\
Еcли $f(x)$ непрерывна и  $g(x)$ знакопостоянна, тогда
$$\int\limits_{a}^{b}=f(\xi)\cdot\int\limits_{a}^{b} g(x)dx,~\xi\in(a,b)$$
\end{theor}
Из теоремы получаем, что
$$\int\limits_{\pi n}^{\pi n+\pi} \frac{\sin(x)}{x^\alpha}dx = 
\frac{1}{\xi^\alpha_n}\int\limits_{\pi n}^{\pi n+\pi}\sin(x)dx = 
\frac{2\cdot (-1)^n}{\xi^\alpha_n}$$ 
Тогда интеграл равен
$$\sum\limits_{n=1}^{\infty}\frac{2\cdot (-1)^n}{\xi^\alpha_n}$$
Ряд расходится по необходимому признаку, поэтому интеграл расходится по 
опредлению Гейне.\\
Можно доказать то же самое по критерию Коши. Именно, при $\alpha\leqslant 0$:
$$\exists \varepsilon>0~\forall \delta~\exists y_1,y_2>\delta:
\left| \int\limits_{y_1}^{y_2} \frac{\sin(x)}{x^\alpha}\right|>
\varepsilon$$ 
Чтобы убить модули, выберем такие пределы интегрирования, на которых
синус знакопостоянен. Имеем 
$$\int\limits_{2\pi n}^{2\pi n+n} \frac{\sin(x)}{x^\alpha}dx=
\frac{1}{\xi^\alpha_n}\cdot 2,~2\pi n\leqslant \xi_n\leqslant 2\pi n+\pi$$
Подставив худший вариант, получаем $\frac{2}{(2\pi n)^\alpha}\geqslant 2$,
то есть расходимость.\\
Соберем ответ: 
$$\begin{cases}
    1.~ \alpha>1 - \text{сходится абсолютно;}\\
    2.~ 0<\alpha\leqslant 1 - \text{сходится условно;}\\
    3.~ \alpha\leqslant 0 - \text{расходится.}
\end{cases}$$
%17.11.22
Поговорим про суммы. Более-менее очевидно, что если 
$\int\limits_{a}^{b}f(x)dx$ и $\int\limits_{a}^{b}g(x)dx$ сходятся, 
то и $\int\limits_{a}^{b}(f(x)+g(x))dx$ сходится. Так же и для абсолютной
сходимости. Так, $f(x)=\frac{\sin(x)}{x},~g(x)=-\frac{\sin(x)}{x}$ - сходятся 
условно, но их сумма сходится абсолютно. 

\textbf{Пример (№2380в).} $\int\limits_{0}^{\infty}x^2\cos(e^x)dx$. Одна 
особая точка - $\infty$. Невероятно, но он сходится, так как пики косинуса 
становятся всё тоньше и тоньше. \\
\dnote{Идея: чтобы проинтегрировать, надо добавить что-то такое, что можно 
внести под дифференциал. Домножим и разделим подынтегральную функцию на 
экспоненту, получим $\frac{x^2e^x\cos(e^x)}{e^x}$, и проинтегрируем
$e^x\cos(e^x)$, а остальное выкинем из интеграла по теореме о 
среднем}.\\
Оценим монотонность: $(x^2e^{-x})'=2xe^{-x}-x^2e^{-x}$. При  $x>2$ 
производная отрицательна, значит, стремеление к нулю монотонно. 
Значит, по Дирихле он сходится, так как первообразная косинуса ограниченна.\\
Рассмотрим абсолютную сходимость: 
$$|x^2\cos(e^x)|\geqslant x^2\cos^2(e^x)=\frac{x^2}{2}+
\frac{x^22\cos(2e^x)}{2}$$
Здесь первая дробь расходится, вторая сходится аналогично самому интегралу,
то есть интеграл не сходится абсолютно. 

\textbf{Пример.}
Построим пример положительной функции, которая неограниченна, но интеграл
от неё сходится. Будем строить штуки с площадью $\frac{1}{2^n}$ 
интервалах $(n,n+1)$. Суммировав площади, получим, что интеграл сходится.\\
\dnote{Но по определению Гейне мы должны показать сходимость при любом выборе
последовательности! (а в данном случае мы взяли $x_n=n$). На самом деле, 
для знакоположительных функций при выборе любой последовательности 
пределы частичных сумм
$\sum\limits_{n=1}^{k} \int\limits_{n}^{n+1}f(x)dx $ одинаковы!}
Действительно, любую частичную сумму последовательности можно 
зажать между членами $x_{n}$ и $x_{n+1}$ последовательности $x_n=n$, а 
её предел одинаков.\\
Если мы возьмем знакопеременную функцию, то если она сходится при самой 
<<плохой>> последовательности, то она сходится при любой другой 
последовательности. Причем самая плохая последовательность состоит из тех 
точек, где функция меняет знак. Имеет место следующая
\begin{theor}
    Если $f(x)$ - знакопеременная функция и  $\{x_n\}$ - последовательность,
    состоящая из точек, где функция меняет знак, то из сходимости ряда
    $\sum\limits_{n=1}^{\infty} \int\limits_{x_n}^{x_{n+1}}f(x)dx$ 
    следует сходимость $\int\limits_{1}^{\infty}f(x)dx$
\end{theor}
Геометрический смысл: при данном выборе последовательности отрицательные и 
положительные члены имеют наибольший возможный размер. 

\textbf{Пример (№2380а)} $\int\limits_{0}^{\infty}x^p\sin(x^q)dx,~q\ne 0$. 
Две особые точки: $0$ и  $\infty$. Рассмотрим сначала на бесконечности.\\
1. Если $q<0$, то интеграл знакопостоянный:
$$x^p\sin(x^q)\sim x^{p+q}$$ 
Значит, интеграл сходится при $p+q<-1$.\\
2. Теперь займемся ситуацией, когда $q>1$, и интеграл знакопеременный. 
Применим идею идею из предыдущего номера:
домножим сверху и снизу на какую-нибудь штуку, в данном случае $qx^{q-1}$. 
Получим
$$\frac{x^p\sin(x^q)\cdot qx^{q-1}}{qx^{q-1}}=- \frac{x^{p-q+1}}{q}\cdot 
(\cos(x^q))'$$
Эта штука сходится по Дирихле при $p-q+1<0$, так как $-\frac{1}{q}x^{p-q+1}$ 
монотонно стремится к нулю.\\
3. Рассмотрим абсолютную сходимость:
$$|x^p\sin(x^q)|\leqslant x^p $$
Сходится абсолютно при $p<-1$.\\
4. Рассмотрим ситуацию, когда  $-1< p\leqslant -1+q$. Докажем, что 
здесь сходимость условная.
$$|x^p\sin(x^q)|\geqslant x^p\sin^2(x^q)=\frac{x^p}{2}-
\frac{x^p\cos(2x^q)}{2}$$
Первая дробь расходится, вторая дробь сходится при $p-q+1<0$ по аналогии
с самим интегралом. Значит, интеграл не сходится абсолютно.\\
5. Докажем, что интеграл расходится при $p\geqslant -1 + q$.
Рассмоторим последовательность, из точек, где синус меняет знак, и, согласно
теореме, оценим интеграл рядом 
$\sum\limits_{n=1}^{\infty}\int\limits_{(\pi n)^
{\frac{1}{q}}}^{(\pi n+\pi)^{\frac{1}{q}}}x^p\sin(x)dx$. 
Заметим, что $(\pi n)^{\frac{1}{q}}\to \infty$. Снова домножим на $qx^{q-1}$.
Тогда
$$\frac{1}{q}\xi_n^{p-q+1}\int\limits_{(\pi n)^{\frac{1}{q}}}^{(\pi n+\pi)^
{\frac{1}{q}}}\sin(x^q)qx^{q-1}dx=
\frac{1}{q}\xi_n^{p-q+1}\left( -\cos(x^q) \right)
\big|_{(\pi n)^{\frac{1}{q}}}^{(\pi n+\pi)^{\frac{1}{q}}}=
\frac{2(-1)^{n+1}}{q}\xi_n^{p-q+1}$$
Значит, ряд $\sum\limits_{n=1}^{\infty}\frac{2(-1)^{n+1}}{q}\xi_n^{p-q+1}$
расходится по необходимому признаку, так как $\xi_n^{p-q+1}\geqslant
(\pi n)^{\frac{p-q+1}{q}}\to \infty$.\\
Теперь рассмотрим интеграл в нуле. 
$$\int\limits_{0}^{1}x^p\sin(x^q)dx=\begin{cases}t=\frac{1}{x}\\
dt=-\frac{dx}{x^2}\\0\mapsto \infty\\1\mapsto 1 \end{cases}=
\int\limits_{1}^{\infty}\frac{1}{t^{p+2}}\sin\left(\frac{1}{t^q}\right)dt$$
Получили ситуацию один в один, только вместо $p$ и $q$ будет
$p+2$ и $-q$. 

\textbf{Пример (№2373).}
$\int\limits_{0}^{\infty}\frac{\sin(\ln(x))}{\sqrt{x}}dx$. Сначала исследуем
в нуле:
$$\left| \frac{\sin\ln(x)}{\sqrt{x}} \right|\leqslant \frac{1}{\sqrt{x}}$$ 
значит, сходится абсолютно.\\
Теперь исследуем на сходимость на бесконечности. Так как 
$(\cos\ln(x))'=- \frac{\sin\ln(x)}{x}$, представим функцию в виде 
$\frac{\sin\ln(x)}{x}\cdot \sqrt{x}$ и возьмем худшую последовательность
$x_n=e^{\pi n}$:
$$\int\limits_{e^{\pi n}}^{e^{\pi n+\pi}}\frac{\sin\ln(x)}{x}\cdot\sqrt{x}dx=
\sqrt{\xi_n}\cdot\int\limits_{e^{\pi n}}^{e^{\pi n+\pi}}\frac{\sin\ln(x)}{x}dx=
\sqrt{\xi_n}\cdot(\cos\ln(x))\big|_{e^{\pi n}}^{e^{\pi n+\pi}}=
$$ $$=\sqrt{\xi_n}(-1)^{-1}\cdot 2\to \infty$$
- интеграл расходится.

%21.11.22
%Если вы не задаете вопросов (по домашке), значит, вы её не делаете просто

\textbf{Пример (из Кудрявцева).} 
$\int\limits_{0}^{\infty}\sin
\left(\frac{\sin x}{\sqrt{x}}\right)\frac{dx}{\sqrt{x}}$. Имеем две особые 
точки: $0$ и  $\infty$. Рассмотрим в нуле и уберем синус оценкой:
$$\sin\left(\frac{\sin x}{\sqrt{x}}\right)\frac{1}{\sqrt{x}}\leqslant 
\frac{1}{\sqrt{x}}$$
сходится абсолютно.\\
\dnote{На бесконечности: разложим синус в ряд Тейлора до такого члена, который 
в итоге будет сходиться абсолютно:}
$$\frac{1}{\sqrt{x}}\left(\frac{\sin x}{\sqrt{x}}-
\frac{\sin^3x}{3!x^{\frac{3}{2}}}+o\left( \frac{\sin^3x}{x^{\frac{3}{2}}}
\right)\right)=\frac{\sin x}{x}-\frac{\sin^3x}{3!x^{\frac{5}{2}}}+
o\left( \frac{\sin^3}{x^2} \right) $$
Первая дробь сходится условно по Дирихле, вторая дробь сходится абсолютно,
о-малое от абсолютно сходящейся функции сходится абсолютно, поэтому все 
вместе сходится условно. 

\textbf{Пример.} $\int\limits_{2}^{\infty} \sqrt{x}\ln\left( 1-
\frac{\sin x^2}{x-1}\right)dx$. Одна особая точка - бесконечность. 
Разложим логарифм в ряд Тейлора:
$$\sqrt{x}\left( \frac{\sin x^2}{x-1}-\frac{\sin^2x^2}{2(x-1)^2}+
o\left( \frac{\sin^2x^2}{(x-1)^2} \right) \right)=
\frac{\sqrt{x} \sin x^2}{x-1}-\frac{\sqrt{x}\sin^2x^2}{2(x-1)^2}+
o\left( \frac{\sin^2x^2}{(x-1)^2} \right)$$
Первый член оценим по теореме о среднем, домножив на производную
аргумента синуса:
$$\frac{2x\sqrt{x}\sin x^2}{2x(x-1)}=\frac{\sqrt{x}(-\cos x^2)'}{2x(x-1)}$$ 
Косинус ограничен, все остальное монотонно стремится к нулю (можно взять 
производную и убедиться в этом), значит, первый член сходится условно по 
Дирихле. Теперь докажем, что абсолютно он расходится:
$$\left| \frac{\sqrt{x}\sin x^2}{x-1}\right|\geqslant
\frac{\sqrt{x}\sin^2x^2 }{|x-1|}=\frac{\sqrt{x}}{|x-1|}\cdot 
\frac{(1-\cos2x^2)}{2}=\frac{\sqrt{x}}{2|x-1|}-
\frac{\sqrt{x}\cos2x^2}{2|x-1|}$$ 
Вторая дробь сходится анлогично предыдущему пункту, первая дробь расходится,
значит, все вместе абсолютно расходится.\\
Теперь рассмотрим второй член разложения логарифма в ряд Тейлора:
$$\left| \frac{\sqrt{x}(-\sin^2x^2) }{2(x-1)^2} \right|\leqslant 
\frac{\sqrt{x} }{2(x-1)^2}\sim \frac{1}{2x^{\frac{3}{2}}}$$
Итак, второй член сходится абсолютно, значит, и о-малое от него сходится 
абсолютно. Так как первый член сходится условно, то интеграл сходится условно.

\textbf{Пример.} $\int\limits_{0}^{\infty}\frac{\sin(x+x^2)}{x^\alpha}dx$.
Имеем две особые точки: $0$ и  $\infty$.\\
1. Исследуем на сходимость в нуле:
$$\frac{\sin(x+x^2)}{x^\alpha}\sim \frac{1+x}{x^{\alpha-1}}\sim
\frac{1}{x^{\alpha-1}}$$
сходится при $\alpha<2$.\\
2. Иследуем на бесконечности. Начнем с абсолютной сходимости (обычная оценка
сверху):
$$\left|\frac{\sin(x+x^2)}{x^\alpha}\right|\leqslant \frac{1}{x^\alpha}$$
- сходится абсолютно при $\alpha>1$.\\
Исследуем на обычную сходимость. Домножим на производную аргумента косинуса:
$$\frac{(1+2x)\sin(x+x^2)}{(1+2x)x^\alpha}=
\frac{(-\cos(x+x^2))'}{(1+2x)x^{\alpha}}=\frac{(-\cos(x+x^2))'}{x^{\alpha+1}}
\cdot \frac{1}{2+\frac{1}{x}}$$
Заметим, что первая дробь сходится по признаку Дирихле при 
$\alpha+1>0$, 
а вторая монотонна и ограниченна, значит, все выражение сходится по 
признаку Абеля при $\alpha>-1$.\\
Теперь докажем, что ряд расходится абсолютно при $\alpha\in (-1,1]$;
оценим квадратом синуса, перейдем к косинусу и понизим степень:
$$\frac{|\sin(x+x^2)|}{x^{\alpha}}\geqslant \frac{1-\cos(2x+2x^2)}{2x^\alpha}=
\frac{1}{2x^\alpha}-\frac{\cos(2x+2x^2)}{2x^\alpha}$$
Первая дробь расходится, вторая сходится по Дирихле, значит, интеграл 
расходится.\\
Докажем расходимость при $\alpha\leqslant-1$. Используем для этого ряд из
нулей синуса. Корни синуса найдем из уравнения $x^2+x=\pi n$. Так как мы 
рассматриваем интеграл на $+\infty$, то нам нужен 
только положительный корень: обозначим его
$\sigma(n)=\frac{-1+\sqrt{1+4\pi n}}{2}$. В ряду домножим на ппроизводную
аргумента синуса и применим теорему о среднем:
$$\sum\limits_{n=1}^{\infty} \int\limits_{\sigma(n)}^{\sigma(n+1)}
\frac{(1+2x)\sin(x+x^2)}{(1+2x)x^{\alpha}}=\sum\limits_{n=1}^{\infty} 
\frac{1}{(1+2\xi_n)\xi_n^\alpha}\int\limits_{\sigma(n)}^{\sigma(n+1)}
(-\cos(x+x^2))'dx$$
Интеграл от производной косинуса в нулях синуса равен $2\cdot (-1)^n$, 
значит, общий член ряда имеет вид и оценивается по верхней границе
$$\frac{2\cdot (-1)^n}{(1+2\xi_n)\xi_n^\alpha}\geqslant
\frac{1}{(2\sigma(n+1)+1)(\sigma(n))^\alpha}$$
(мы взяли разные $\sigma$, поскольку мы хотим оценить снизу, а $\alpha<0$).
Подставляя значение корня, получаем расходимость ряда, и, как следствие,
расходимость интеграла.\\
Итак, соберем ответ: в нуле сходится при $\alpha<2$, на бесконечности 
сходится абсолютно на $(1,\infty)$, сходится на $(-1,1]$,
расходится на $(-\infty,-1]$. 

\textbf{Пример.} $\int\limits_{1}^{\infty}x^\alpha\sin\sin xdx$. Одна особая
точка - бесконечность. Абсолютная сходимость очевидна:
$$|x^\alpha\sin\sin x|\leqslant x^\alpha$$
- сходится при $\alpha<-1$.\\
Чтобы исследовать обычную сходимость, надо доказать ограниченность 
первообразной от $\sin\sin x$. Сделаем это через ряды (потому что через
домножение на интегрирующий множитель не получится):
$$\int\limits_{0}^{y} \sin\sin x dx=\int\limits_{0}^{\pi}...+
\int\limits_{\pi}^{2\pi}...+...+\int\limits_{2\pi k}^{y},$$ 
где $k=\left[ \frac{y}{2\pi} \right]$. Чтобы получить оценку, не зависящую 
от $y$, докажем, что сумма всех интегралов, кроме последнего, равна нулю:
$$\int\limits_{2\pi k}^{2\pi k+\pi}\sin\sin xdx+
\int\limits_{2\pi k+\pi}^{2\pi k+2\pi}\sin\sin xdx=
\begin{cases}t=t+\pi\\dx=dt\end{cases}=$$
Введем эту замену, чтобы привести второй интеграл к тем же пределам и 
сделать его отрицательным:
$$=\int\limits_{2\pi k}^{2\pi k+\pi}\sin\sin xdx\,+
\int\limits_{2\pi k}^{2\pi k+\pi}\sin\sin (t+\pi)dt=
\int\limits_{2\pi k}^{2\pi k+\pi}\sin\sin xdx\,+
\int\limits_{2\pi k}^{2\pi k+\pi}\sin(-\sin t) dt=0$$
Оценим последний интеграл:
$$\left|\,\int\limits_{2\pi k}^{y}\sin\sin xdx\right| \leqslant 
\int\limits_{2\pi k}^{y}|\sin x|dx\leqslant \int\limits_{2\pi k}^{y}1dx=
y-2\pi k\leqslant 2\pi$$
Значит, интеграл сходится по Дирихле при $\alpha<0$.\\
Докажем (обычную) расходимость при $\alpha\geqslant 0$. 
Снова используем ряды, но на 
этот раз будем использовать теорему о среднем:
$$\int\limits_{\pi n}^{\pi n+\pi}x^\alpha\sin\sin xdx=\xi^\alpha_n\cdot 
\int\limits_{\pi n}^{\pi n+\pi}\sin\sin xdx=\xi^\alpha_n(-1)^n
 \int\limits_{o}^{\pi}\sin\sin xdx$$
Интеграл $\int\limits_{o}^{\pi}\sin\sin xdx$ не зависит от  $n$ и положителен, 
обозначим его значение за  $\beta$. Тогда общий член ряда имеет вид
$$\xi^\alpha_n\cdot (-1)^n\cdot\beta\geqslant (\pi n+\pi)^\alpha\cdot(-1)^n$$
то есть ряд расходится по необходимому признаку, и сам интеграл расходится.\\
Наконец, докажем абсолютную расходимость при $\alpha\in[-1,0)$. Она 
устанавливается аналогично: так как
$$\int\limits_{\pi n}^{\pi n+\pi}|\sin\sin x|dx=
 \int\limits_{o}^{\pi}|\sin\sin x|dx=\beta$$
 то общий член ряда имеет вид $\xi^\alpha_n\beta$. Подставляя оценку
 $\xi_n=\pi n+\pi$, имеем расходимость ряда и соотвественно расходимость 
Мы  интеграла.\\
Соберем ответ: интеграл абсолютно сходится при $\alpha\in (-\infty,-1)$,
сходится условно при $\alpha\in [-1,0)$, расходится при 
$\alpha\in [0,\infty)$. 

\section{Интеграл, зависящий от параметра}
Будем рассматривать интеграл $F(y)=\int\limits_{a}^{b}f(x,y)dx,~y\in [c,d]$.
Мы будем использовать следующие нумерованные теоремы:\\
\textbf{Th 1.} Из непрерывности $f$ следует непрерывность  $F$ на $(c,d]$;\\
\textbf{Th 2.} $F$ можно интегрировать по $y$ и переставлять интегралы 
местами;\\
\textbf{Th 3.} Если $f$ и $f'_y$ непрерывны на  $[a,b]\times[c,d]$, тогда 
$F'(y)=\int\limits_{a}^{b}f'_y(x,y)dx$;\\
\textbf{Th 4.} 

\textbf{Пример (№3713в).} Найти $\lim\limits_{a \to 0}\int\limits_{0}^{2}
x^2\cos ax \,dx$. Сначала интегрируем по частям: 
$$\lim\limits_{a \to 0} \left(\frac{1}{a}\sin ax\cdot x\Big|_0^2-\frac{1}{a}
    \int\limits_{0}^{2}\sin ax\,dx\right)=\lim\limits_{a \to 0} 
\left( \frac{2\sin 2a}{a}+\frac{1}{a^2}\cos ax\Big|^2_0 \right) = 4-2=2$$
С другой стороны, так как $\lim\limits_{a \to 0} \cos ax=1$, то по 
теореме 1 получаем
$$\lim\limits_{a \to 0}\int\limits_{0}^{2} x\cos ax\,dx=\int\limits_{0}^{2}
x\,dx=2$$

\textbf{Пример (№3715).} Найти $\lim\limits_{y \to 0}\int\limits_{0}^{1} 
\frac{x}{y^2}e^{-\frac{x^2}{y^2}}dx$. Функция не определена при $y=0$, 
однако, посчитав предел по Лопиталю, можно увидеть, что предел в нуле равен
нулю. Доопределим функцию нулем при $y=0$ и поехали интегрировать. Заметим, 
что  $(e^{-\frac{x^2}{y^2}})'=-\frac{2x}{y^2}e^{-\frac{x^2}{y^2}}$, поэтому
$$\lim\limits_{y \to 0}-\frac{1}{2}\int\limits_{0}^{1} 
-\frac{2x}{y^2}e^{-\frac{x^2}{y^2}}dx= \lim\limits_{y \to 0} -\frac{1}{2}
e^{-\frac{x^2}{y^2}}\Big|_0^1=\lim\limits_{y \to 0}-\frac{1}{2}\left( 
e^{-\frac{1}{y^2}}-1\right) =\frac{1}{2}$$
\dnote{Но если бы мы воспользовались теоремой 1, то получили бы другой ответ!
Противоречие лишь кажущееся, так как для выполнения условий теоремы 1 
необходима непрерывность по двум переменным (то есть по всем путям, ведущим
 нулю). Доопределив функцию нулем, мы сделали её непрерывной по оси $Y$.}

 \textbf{Пример (№3734).}
 $\int\limits_{0}^{\frac{\pi}{2}} \frac{arctg(\alpha\cdot tg x)}{tg x}dx $.
Пусть $f(\alpha)=\int\limits_{0}^{\frac{\pi}{2}}
\frac{arctg(\alpha\cdot tg x)}{tg x}dx$.\\
Функция не определена в нуле и в $\frac{\pi}{2}$, поэтому она не непрерывна. 
Чтобы добить непрерывность, доопределим функцию:
$$f(x,\alpha)=\begin{cases}
    r!!!!!!!!!!!et\\
\alpha_0,~x=0\\
0,~x=\frac{\pi}{2}
\end{cases}$$
По теореме 3 можно дифференцировать, тогда 
$$f'_\alpha=
\int\limits_{0}^{\frac{\pi}{2}} \frac{\cos^2x\,dx}
{(1+(\alpha\cdot tgx)^2)\cos^2x}=
\int\limits_{0}^{\infty}\frac{du}{(1+(\alpha u)^2)(1+u^2)} $$
Берем по частям:
$$\frac{\alpha}{\alpha^2-1}\int\limits_{0}^{\infty} \frac{d(\alpha u)}{
1+(\alpha u)^2}-\frac{1}{\alpha^2-1}\int\limits_{0}^{\infty}\frac{du}{
1+u^2} =$$



%$$\frac{1}{(1=(\alpha u)^2)(1+u^2)}=\left( \frac{\alpha^2}{1+(\alpha u)^2} \right) $$

\textbf{Пример №3737.} $\int\limits_{0}^{1} \frac{x^b-x^a}{\ln x}dx,~a,b>0$.
Заметим, что $(x^b)'_b=x^b\ln x$, откуда $\frac{x^b-x^a}{\ln x}=
\int\limits_{a}^{b}x^ydy$.
Интеграл можно менять местами: 
$$\int\limits_{0}^{1} dx \int\limits_{a}^{b} x^ydy=\int\limits_{a}^{b} 
\left(  \right) dy=\int\limits_{a}^{b} \frac{dy}{y+1}=\ln(y+1)\Big|^b_a=
\ln\left( \frac{b+1}{a+1} \right) $$ 
Заметим , что теорему применять можно при ????

\section{Несобственный интеграл, зависящий от параметра}



Доказать равномерную сходимость на $0,a\leqslant \alpha\leqslant b$,
обычную сходимость $0\leqslant \alpha\leqslant b$
$I=\int\limits_{0}^{\infty}\alpha e^{-\alpha x}dx$


\textbf{Пример (№3755.2)}  $\int\limits_{1}^{\infty} \frac{dx}{x^\alpha}$ 
$1<\alpha_0\leqslant \alpha<\infty$
$1<\alpha<\infty$

\textbf{Пример (№37??)} $\int\limits_{0}^{\infty}\frac{dx}{(x-\alpha)^2+1}$ 
$0\leqslant \alpha<\infty$. Не сходится равномерно, ибо арктангенс (
если взять интеграл) имеем максимум в нуле. 


\section{Интегралы Эйлера}
Надо запомнить всего лишь 4 формулы. Основная идея - делание замен для того,
чтобы попасть в пределы интегрирования.

\textbf{№3843} $\int\limits_{0}^{1} \sqrt{x-x^2}dx=B(\frac{3}{2},\frac{3}{2})=
\frac{\Gamma(3/2)\Gamma(3/2)}{\Gamma(3)}=\frac{1}{8}\pi$ 

\textbf{№3849} $\int\limits_{0}^{1} \frac{dx}{\sqrt[n]{1-x^n}}$. 
Сведем к бета-функции заменой 
$t=1-x^n,~dx=-\frac{1}{n}(1-t)^{\frac{1}{n}-1}dt$. Имеем 
$$\int\limits_{0}^{1}t^{\frac{1}{n}}\cdot \frac{1}{n}\cdot 
(1-t)^{\frac{1}{n}-1}=\frac{1}{n}B\left( -\frac{1}{n}+1,\frac{1}{n} \right)
=\frac{1}{n} \frac{\pi}{\sin\pi n}$$

\textbf{№3851} $\int\limits_{0}^{\infty}\frac{x^{m-1}}{1+x^n}dx$.
Сделаем замену, чтобы подогнать пределы интегрирования под бета-функцию. 
Замена $t=\frac{1}{1+x^n}$. Пределы пересчитаются, тогда интеграл равен
$$\int\limits_{0}^{1}(\frac{1}{t}-1)^{\frac{m-1}{n}}t \frac{1}{n}
(\frac{1}{t}-1)^{\frac{1}{n}-1}\frac{1}{t^2}dt=
\frac{1}{n}\int\limits_{0}^{1}\left( \frac{1}{n}-1 \right)^{\frac{m-n}{n}}
\frac{1}{t}dt$$ !!!!!!

\textbf{№3856} 
$$\int\limits_{0}^{\frac{pi}{2}}\sin^mx\cos^nxdx=
\begin{cases}\sin x=t\\dt=\cos x\,dx\\ \cos x=\sqrt{1-t^2}   
\end{cases}=
\int\limits_{0}^{1}t^m(1-t^2)^{\frac{n-1}{2}}dt=$$
$$\begin{cases} u=t^2\\du=2t \,dt
\end{cases}
=\frac{1}{2}\int\limits_{0}^{1}u^{\frac{m-1}{2}}(1-u)^{\frac{n-1}{2}} 
$$

\textbf{Пример из Кудрявцева.} $\int\limits_{-2}^{2}\frac{dx}{-2
\sqrt[4]{(2+x)^3(2-x)}}$. Замена $t=2+x$. Имеем
$\int\limits_{0}^{4} \frac{dt}{t^{\frac{3}{4}}(4-t)^{\frac{1}{4}}}$.
Вынесем четверку, делая замену $u=\frac{t}{4}$. Тогда
$$\frac{4}{4^{\frac{1}{4}}\cdot 4^{\frac{3}{4}}}
\int\limits_{0}^{1}\frac{du}{u^{\frac{3}{4}}(1-u)^{\frac{1}{4}}}=
B\left( \frac{1}{4},\frac{3}{4} \right)=\frac{\pi}{\sin \frac{pi}{4}}=
\pi\sqrt{2}$$ 


\textbf{Пример посложнее.} $\int\limits_{1}^{2}\sqrt{\frac{x-1}{2-x}}
\frac{dx}{(x+3)^2}$. Сначала избавляемяс от второго множеителя, делая замену
$t=\frac{1}{x+3}$. Тогда $-\int\limits_{1/4}^{1/5}\sqrt{\frac{\frac{1}{t}-4}
{5-\frac{1}{t}}}dt=\int\limits_{\frac{1}{5}}^{\frac{1}{4}}
\sqrt{\frac{1-4t}{5t-1}}dt$. Теперь замена $u=4(5t-1)$:
$$\frac{1}{20}\int\limits_{0}^{1} \sqrt{\frac{4(1-u)}{5u}}du=
\frac{\sqrt{4}}{20\sqrt{5}}B\left( \frac{1}{2},\frac{3}{2} \right) =
\frac{\pi}{20\sqrt{5}}$$

\textbf{Пример (№3863).} $\int\limits_{0}^{\infty}\frac{x^{p-1}\ln x}{1+x}dx$.
%Чтобы убить логарифм, будем интегрировать по частям. 
Сначала посчитаем интеграл без логарифма: 
$$\int\limits_{0}^{\infty}\frac{x^{p-1}}{1+x}dx=
\begin{cases}t=\frac{1}{1+x}\\dx=-\frac{1}{t^2}dt   
\end{cases}=
\int\limits_{0}^{1} \frac{(\frac{1}{t}-1)^{p-1}}{\frac{1}{t}}\frac{1}{t^2}dt=
\int\limits_{0}^{1}(1-t)^{p-1}t^{-p}dt
$$
Значение этого интеграла $B(1-p,p)=\frac{\pi}{\sin\pi p}$ 

\dnote{Как убить логарифм? Возьмем производную от функции по параметру:
$f'_p=\left( \frac{\pi}{\sin\pi p}\right)'=- \frac{\cos \pi p}{\sin^2\pi p}
\pi^2$. Дифференцировать можно. потому что интеграл сходится равномерно.}

\textbf{пример.} $o$













