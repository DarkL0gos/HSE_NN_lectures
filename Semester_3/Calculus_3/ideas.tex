\chapter{Идеи и номера с практики}
Идеи Тимура, достойные того, чтобы быть запечатленными.
Те места, которые на слух отмечаются словами типа <<финт ушами>>, будут 
отмечаться знаком <<опасный поворот>> \dbend в стиле Бурбаки (а не то, что
вы подумали). 

\section{Знакопостоянные несобственные интегралы}
Для знакопеременных интегралов можно использовать признак сравнения.
Обычно сравнение происходит с обобщенной степенной функцией.
При этом имеется два различных типа особых точек: на бесконечности и 
с уходом на бесконечность в точке. Разберем подробнее.

\textbf{Пример 1.} Интеграл 
$$\int\limits_{1}^{\infty}\frac{1}{x^\alpha}dx$$ 
сходится при $\alpha>1$ и расходится при $\alpha\leqslant 1$.

\textbf{Пример 2.} Интеграл
$$\int\limits_{a}^{b}\frac{1}{(x-a)^\alpha}dx$$
сходится при $\alpha<1$ и расходится при $\alpha\geqslant 1$. 

\textbf{Пример.} Интеграл $\int\limits_{1}^{\infty} \frac{x^2dx}{x^4-x^2+1}$ 
сходится, поскольку подынтегральная функция эквивалентна $\frac{1}{x^2}$ - 
сходящейся штуке.


\textbf{Пример (№2374)}. Исследуем на сходимость в зависимости от параметров
интеграл
$$\int\limits_{1}^{\infty} \frac{1}{x^p\ln^q x}dx$$
Имеем 2 особые точки: 1 и $\infty$, поэтому разобъем область исследования
на две части и будет исследовать интеграл $\int\limits_{10}^{\infty}$.\\
Нам поторебуется следующий признак сравнения: для $\varepsilon>0$
$$\frac{1}{x^\varepsilon}<\ln^\alpha(x)<x^\varepsilon,~
x>\delta(\alpha,\varepsilon)$$ 
(доказательство через правило Лопиталя: действительно, 
$\lim\limits_{n \to \infty} \frac{\ln^\alpha(x)}{x^\varepsilon}=0$).\\
Значит, имеем
$$\frac{1}{x^{p+\varepsilon}}\leqslant \frac{1}{x^p\ln^qx}\leqslant 
\frac{1}{x^{p-\varepsilon}}$$ 
\dnote{Итак, интеграл сходится при $p>1+\varepsilon$ и расходится при
$p<1-\varepsilon$. Так как $\varepsilon$ вообще-то произвольный,
то и условие сходимости не должно зависеть от него; иначе говоря, 
интеграл сходится при $p>1$ и расходится $p<1$.}
Рассмотрим случай, когда $p=1$. Имеем 
$$\int\limits_{10}^{\infty} \frac{1}{x\ln^qx}dx = 
\begin{cases}\ln(x)=t\\dt=\frac{dx}{x}\end{cases} = 
\int\limits_{\ln 10}^{\infty}\frac{dt}{t^q}$$ 
Значит, этот интеграл сходится при $q>1$.
Соберем ответ:
$$\begin{cases}
    1.~ p>1 - \text{сходится;}\\
    2.~ p<1 - \text{расходится;}\\
    3.~ p=1,q>1 - \text{сходится;}\\
    4.~ p=1,q\leqslant 1 - \text{расходится.}
\end{cases}$$



\section{Знакопеременные несобственные интегралы}
Напомним, что для применения признаков Абеля и Дирихле в интеграле 
$\int\limits_{a}^{\infty} f(x)g(x)dx$, необходимо, чтобы $f(x)$ и $g'(x)$ были 
непрерывными функциями.

\textbf{Пример.} $$\int\limits_{1}^{\infty}\frac{\sin(x)}{x^\alpha}$$ 
Интеграл имеет одну особую точку: $+\infty$.\\
Сначала расмотрим абсолютную сходимость: 
$\frac{|\sin(x)|}{x^\alpha}\leqslant \frac{1}{x^\alpha}$, откуда
по признаку сравнения получаем, что интеграл сходится абсолютно при 
$\alpha>1$.\\
Рассмотрим обычную сходимость: интеграл удовлетворяет признаку Дирихле,
поскольку 
$\forall y>a: \int\limits_{a}^{y}\sin(x)dx=-\cos(y)+\cos(a)\leqslant 20$ и
$\frac{1}{x^\alpha}\to 0$ монотонно. Значит, интеграл сходится при 
$\alpha>0$.\\
Теперь рассмотрим расходимость интерала. Докажем условную сходимость на
$(0,1]$.
Оценим снизу увадратом синуса:
$$\frac{|\sin(x)|}{x^\alpha}\geqslant \frac{\sin^2(x)}{x^\alpha}=
\frac{1-\cos(2x)}{2x^\alpha}=\frac{1}{2x^\alpha}-\frac{\cos(2x)}{2x^\alpha}$$ 
Вторая дробь сходится по Дирихле, откуда весь интеграл расходится абсолютно
при $\alpha\leqslant 1$.\\
Осталось установить сходимость при $\alpha\leqslant 0$. Вспомним определение
\textbf{предела по Гейне}:
$$\forall \{y_n\}\to 0: \lim\limits_{n \to \infty} \int\limits_{a}^{y_n}
f(x)dx\to const$$ 
Тогда интеграл можно предстваить в виде $\sum\limits_{n=1}^{\infty} 
\int\limits_{y_n}^{y_{n+1}}f(x)dx$. Найдем какую-нибудь последовательность,
на которой будет расходимость. Итак, пусть $y_n=\pi n$.\\
Теперь нам потребуется следующая 
\begin{theor}(о среднем)\\
Еcли $f(x)$ непрерывна и  $g(x)$ знакопостоянна, тогда
$$\int\limits_{a}^{b}=f(\xi)\cdot\int\limits_{a}^{b} g(x)dx,~\xi\in(a,b)$$
\end{theor}
Из теоремы получаем, что
$$\int\limits_{\pi n}^{\pi n+\pi} \frac{\sin(x)}{x^\alpha}dx = 
\frac{1}{\xi^\alpha_n}\int\limits_{\pi n}^{\pi n+\pi}\sin(x)dx = 
\frac{2\cdot (-1)^n}{\xi^\alpha_n}$$ 
Тогда интеграл равен
$$\sum\limits_{n=1}^{\infty}\frac{2\cdot (-1)^n}{\xi^\alpha_n}$$
Ряд расходится по необходимому признаку, поэтому интеграл расходится по 
опредлению Гейне.\\
Можно доказать то же самое по критерию Коши. Именно, при $\alpha\leqslant 0$:
$$\exists \varepsilon>0~\forall \delta~\exists y_1,y_2>\delta:
\left| \int\limits_{y_1}^{y_2} \frac{\sin(x)}{x^\alpha}\right|>
\varepsilon$$ 
Чтобы убить модули, выберем такие пределы интегрирования, на которых
синус знакопостоянен. Имеем 
$$\int\limits_{2\pi n}^{2\pi n+n} \frac{\sin(x)}{x^\alpha}dx=
\frac{1}{\xi^\alpha_n}\cdot 2,~2\pi n\leqslant \xi_n\leqslant 2\pi n+\pi$$
Подставив худший вариант, получаем $\frac{2}{(2\pi n)^\alpha}\geqslant 2$,
то есть расходимость.\\
Соберем ответ: 
$$\begin{cases}
    1.~ \alpha>1 - \text{сходится абсолютно;}\\
    2.~ 0<\alpha\leqslant 1 - \text{сходится условно;}\\
    3.~ \alpha\leqslant 0 - \text{расходится.}
\end{cases}$$
%17.11.22
Поговорим про суммы. Более-менее очевидно, что если 
$\int\limits_{a}^{b}f(x)dx$ и $\int\limits_{a}^{b}g(x)dx$ сходятся, 
то и $\int\limits_{a}^{b}(f(x)+g(x))dx$ сходится. Так же и для абсолютной
сходимости. Так, $f(x)=\frac{\sin(x)}{x},~g(x)=-\frac{\sin(x)}{x}$ - сходятся 
условно, но их сумма сходится абсолютно. 

\textbf{Пример (№2380в).} $\int\limits_{0}^{\infty}x^2\cos(e^x)dx$. Одна 
особая точка - $\infty$. Невероятно, но он сходится, так как пики косинуса 
становятся всё тоньше и тоньше. \\
\dnote{Идея: чтобы проинтегрировать, надо добавить что-то такое, что можно 
внести под дифференциал. Домножим и разделим подынтегральную функцию на 
экспоненту, получим $\frac{x^2e^x\cos(e^x)}{e^x}$, и проинтегрируем
$\frac{\cos(e^x)}{e^x}$, а остальное выкинем из интеграла по теореме о 
среднем}.\\
Оценим монотонность: $(x^2e^{-x})'=2xe^{-x}-x^2e^{-x}$. При  $x>2$ 
производная отрицательна, значит, стремеление к нулю монотонно. 
Значит, по Дирихле он сходится, так как первообразная косинуса ограниченна.\\
Рассмотрим абсолютную сходимость: 
$$|x^2\cos(e^x)|\geqslant x^2\cos^2(e^x)=\frac{x^2}{2}+
\frac{x^22\cos(2e^x)}{2}$$
Здесь первая дробь расходится, вторая сходится аналогично самому интегралу,
то есть интеграл не сходится абсолютно. 

\textbf{Пример.}
Построим пример положительной функции, которая неограниченна, но интеграл
от неё сходится. Будем строить штуки с площадью $\frac{1}{2^n}$ 
интервалах $(n,n+1)$. Суммировав площади, получим, что интеграл сходится.\\
\dnote{Но по определению Гейне мы должны показать сходимость при любом выборе
последовательности! (а в данном случае мы взяли $x_n=n$). На самом деле, 
для знакоположительных функций при выборе любой последовательности 
пределы частичных сумм
$\sum\limits_{n=1}^{k} \int\limits_{n}^{n+1}f(x)dx $ одинаковы!}
Действительно, любую частичную сумму последовательности можно 
зажать между членами $x_{n}$ и $x_{n+1}$ последовательности $x_n=n$, а 
её предел одинаков.\\
Если мы возьмем знакопеременную функцию, то если она сходится при самой 
<<плохой>> последовательности, то она сходится при любой другой 
последовательности. Причем самая плохая последовательность состоит из тех 
точек, где функция меняет знак. Имеет место следующая
\begin{theor}
    Если $f(x)$ - знакопеременная функция и  $\{x_n\}$ - последовательность,
    состоящая из точек, где функция меняет знак, то из сходимости
    $\sum\limits_{n=1}^{\infty} \int\limits_{x_n}^{x_{n+1}}f(x)dx$ 
    следует сходимость $\int\limits_{1}^{\infty}f(x)dx$
\end{theor}
Геометрический смысл: при данном выборе последовательности отрицательные и 
положительные члены имеют наибольший возможный размер. 

\textbf{Пример (№2380а)} $\int\limits_{0}^{\infty}x^p\sin(x^q)dx,~q\ne 0$. 
Две особые точки: $0$ и  $\infty$. Рассмотрим сначала на бесконечности.\\
1. Если $q<0$, то интеграл знакопостоянный:
$$x^p\sin(x^q)\sim x^{p+q}$$ 
Значит, интеграл сходится при $p+q<-1$.\\
2. Теперь займемся ситуацией, когда $q>1$, и интеграл знакопеременный. 
Применим идею идею из предыдущего номера:
домножим сверху и снизу на какую-нибудь штуку, в данном случае $qx^{q-1}$. 
Получим
$$\frac{x^p\sin(x^q)\cdot qx^{q-1}}{qx^{q-1}}=- \frac{x^{p-q+1}}{q}\cdot 
(\cos(x^q))'$$
Эта штука сходится по Дирихле при $p-q+1<0$, так как $-\frac{1}{q}x^{p-q+1}$ 
монотонно стремится к нулю.\\
3. Рассмотрим абсолютную сходимость:
$$|x^p\sin(x^q)|\leqslant x^p $$
Сходится абсолютно при $p<-1$.\\
4. Рассмотрим ситуацию, когда  $-1< p\leqslant -1+q$. Докажем, что 
здесь сходимость условная.
$$|x^p\sin(x^q)|\geqslant x^p\sin^2(x^q)=\frac{x^p}{2}-
\frac{x^p\cos(2x^q)}{2}$$
Первая дробь расходится, вторая дробь сходится при $p-q+1<0$ по аналогии
с самим интегралом. Значит, интеграл не сходится абсолютно.\\
5. Докажем, что интеграл расходится при $p\geqslant -1 + q$.
Рассмоторим последовательность, из точек, где синус меняет знак, и, согласно
теореме, оценим интеграл рядом 
$\sum\limits_{n=1}^{\infty}\int\limits_{(\pi n)^
{\frac{1}{q}}}^{(\pi n+\pi)^{\frac{1}{q}}}x^p\sin(x)dx$. 
Заметим, что $(\pi n)^{\frac{1}{q}}\to \infty$. Снова домножим на $qx^{q-1}$.
Тогда
$$\frac{1}{q}\xi_n^{p-q+1}\int\limits_{(\pi n)^{\frac{1}{q}}}^{(\pi n+\pi)^
{\frac{1}{q}}}\sin(x^q)qx^{q-1}dx=
\frac{1}{q}\xi_n^{p-q+1}\left( -\cos(x^q) \right)
\big|_{(\pi n)^{\frac{1}{q}}}^{(\pi n+\pi)^{\frac{1}{q}}}=
\frac{2(-1)^{n+1}}{q}\xi_n^{p-q+1}$$
Значит, ряд $\sum\limits_{n=1}^{\infty}\frac{2(-1)^{n+1}}{q}\xi_n^{p-q+1}$
расходится по необходимому признаку, так как $\xi_n^{p-q+1}\geqslant
(\pi n)^{\frac{p-q+1}{q}}\to \infty$.\\
Теперь рассмотрим интеграл в нуле. 
$$\int\limits_{0}^{1}x^p\sin(x^q)dx=\begin{cases}t=\frac{1}{x}\\
dt=-\frac{dx}{x^2}\\0\mapsto \infty\\1\mapsto 1 \end{cases}=
\int\limits_{1}^{\infty}\frac{1}{t^{p+2}}\sin\left(\frac{1}{t^q}\right)dt$$
Получили ситуацию один в один, только вместо $p$ и $q$ будет
$p+2$ и $-q$. 

\textbf{Пример (№2373).}
$\int\limits_{0}^{\infty}\frac{\sin(\ln(x))}{\sqrt{x}}dx$. Сначала исследуем
в нуле:
$$\left| \frac{\sin\ln(x)}{\sqrt{x}} \right|\leqslant \frac{1}{\sqrt{x}}$$ 
значит, сходится абсолютно.\\
Теперь исследуем на сходимость на бесконечности. Так как 
$(\cos\ln(x))'=- \frac{\sin\ln(x)}{x}$, представим функцию в виде 
$\frac{\sin\ln(x)}{x}\cdot \sqrt{x}$ и возьмем худшую последовательность
$x_n=e^{\pi n}$:
$$\int\limits_{e^{\pi n}}^{e^{\pi n+\pi}}\frac{\sin\ln(x)}{x}\cdot\sqrt{x}dx=
\sqrt{\xi_n}\cdot\int\limits_{e^{\pi n}}^{e^{\pi n+\pi}}\frac{\sin\ln(x)}{x}dx=
\sqrt{\xi_n}\cdot(\cos\ln(x))\big|_{e^{\pi n}}^{e^{\pi n+\pi}}=
$$ $$=\sqrt{\xi_n}(-1)^{-1}\cdot 2\to \infty$$
- интеграл расходится.
















