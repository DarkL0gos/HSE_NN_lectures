\chapter{Идеи и номера с практики}
Идеи Тимура, достойные того, чтобы быть запечатленными.
Те места, которые на слух отмечаются словами типа <<финт ушами>>, будут 
отмечаться знаком <<опасный поворот>> \dbend в стиле Бурбаки (а не то, что
вы подумали). 

\section{Знакопостоянные несобственные интегралы}
Для знакопеременных интегралов можно использовать признак сравнения.
Обычно сравнение происходит с обобщенной степенной функцией.
При этом имеется два различных типа особых точек: на бесконечности и 
с уходом на бесконечность в точке. Разберем подробнее.

\textbf{Пример 1.} Интеграл 
$$\int\limits_{1}^{\infty}\frac{1}{x^\alpha}dx$$ 
сходится при $\alpha>1$ и расходится при $\alpha\leqslant 1$.

\textbf{Пример 2.} Интеграл
$$\int\limits_{a}^{b}\frac{1}{(x-a)^\alpha}dx$$
сходится при $\alpha<1$ и расходится при $\alpha\geqslant 1$. 

\textbf{Пример.} Интеграл $\int\limits_{1}^{\infty} \frac{x^2dx}{x^4-x^2+1}$ 
сходится, поскольку подынтегральная функция эквивалентна $\frac{1}{x^2}$ - 
сходящейся штуке.


\textbf{Пример (№2374)}. Исследуем на сходимость в зависимости от параметров
интеграл
$$\int\limits_{1}^{\infty} \frac{1}{x^p\ln^q x}dx$$
Имеем 2 особые точки: 1 и $\infty$, поэтому разобъем область исследования
на две части и будет исследовать интеграл $\int\limits_{10}^{\infty}$.\\
Нам поторебуется следующий признак сравнения: для $\varepsilon>0$
$$\frac{1}{x^\varepsilon}<\ln^\alpha(x)<x^\varepsilon,~
x>\delta(\alpha,\varepsilon)$$ 
(доказательство через правило Лопиталя: действительно, 
$\lim\limits_{n \to \infty} \frac{\ln^\alpha(x)}{x^\varepsilon}=0$).\\
Значит, имеем
$$\frac{1}{x^{p+\varepsilon}}\leqslant \frac{1}{x^p\ln^qx}\leqslant 
\frac{1}{x^{p-\varepsilon}}$$ 
\dnote{Итак, интеграл сходится при $p>1+\varepsilon$ и расходится при
$p<1-\varepsilon$. Так как $\varepsilon$ вообще-то произвольный,
то и условие сходимости не должно зависеть от него; иначе говоря, 
интеграл сходится при $p>1$ и расходится $p<1$.}
Рассмотрим случай, когда $p=1$. Имеем 
$$\int\limits_{10}^{\infty} \frac{1}{x\ln^qx}dx = 
\begin{cases}\ln(x)=t\\dt=\frac{dx}{x}\end{cases} = 
\int\limits_{\ln 10}^{\infty}\frac{dt}{t^q}$$ 
Значит, этот интеграл сходится при $q>1$.
Соберем ответ:
$$\begin{cases}
    1.~ p>1 - \text{сходится;}\\
    2.~ p<1 - \text{расходится;}\\
    3.~ p=1,q>1 - \text{сходится;}\\
    4.~ p=1,q\leqslant 1 - \text{расходится.}
\end{cases}$$



\section{Знакопеременные несобственные интегралы}
Напомним, что для применения признаков Абеля и Дирихле в интеграле 
$\int\limits_{a}^{\infty} f(x)g(x)dx$, необходимо, чтобы $f(x)$ и $g'(x)$ были 
непрерывными функциями.

\textbf{Пример.} $$\int\limits_{1}^{\infty}\frac{\sin(x)}{x^\alpha}$$ 
Интеграл имеет одну особую точку: $+\infty$.\\
Сначала расмотрим абсолютную сходимость: 
$\frac{|\sin(x)|}{x^\alpha}\leqslant \frac{1}{x^\alpha}$, откуда
по признаку сравнения получаем, что интеграл сходится абсолютно при 
$\alpha>1$.\\
Рассмотрим обычную сходимость: интеграл удовлетворяет признаку Дирихле,
поскольку 
$\forall y>a: \int\limits_{a}^{y}\sin(x)dx=-\cos(y)+\cos(a)\leqslant 20$ и
$\frac{1}{x^\alpha}\to 0$ монотонно. Значит, интеграл сходится при 
$\alpha>0$.\\
Теперь рассмотрим расходимость интерала. Докажем условную сходимость на
$(0,1]$.
Оценим снизу увадратом синуса:
$$\frac{|\sin(x)|}{x^\alpha}\geqslant \frac{\sin^2(x)}{x^\alpha}=
\frac{1-\cos(2x)}{2x^\alpha}=\frac{1}{2x^\alpha}-\frac{\cos(2x)}{2x^\alpha}$$ 
Вторая дробь сходится по Дирихле, откуда весь интеграл расходится абсолютно
при $\alpha\leqslant 1$.\\
Осталось установить сходимость при $\alpha\leqslant 0$. Вспомним определение
\textbf{предела по Гейне}:
$$\forall \{y_n\}\to 0: \lim\limits_{n \to \infty} \int\limits_{a}^{y_n}
f(x)dx\to const$$ 
Тогда интеграл можно предстваить в виде $\sum\limits_{n=1}^{\infty} 
\int\limits_{y_n}^{y_{n+1}}f(x)dx$. Найдем какую-нибудь последовательность,
на которой будет расходимость. Итак, пусть $y_n=\pi n$.\\
Теперь нам потребуется следующая 
\begin{theor}(о среднем)\\
Еcли $f(x)$ непрерывна и  $g(x)$ знакопостоянна, тогда
$$\int\limits_{a}^{b}=f(\xi)\cdot\int\limits_{a}^{b} g(x)dx,~\xi\in(a,b)$$
\end{theor}
Из теоремы получаем, что
$$\int\limits_{\pi n}^{\pi n+\pi} \frac{\sin(x)}{x^\alpha}dx = 
\frac{1}{\xi^\alpha_n}\int\limits_{\pi n}^{\pi n+\pi}\sin(x)dx = 
\frac{2\cdot (-1)^n}{\xi^\alpha_n}$$ 
Тогда интеграл равен
$$\sum\limits_{n=1}^{\infty}\frac{2\cdot (-1)^n}{\xi^\alpha_n}$$
Ряд расходится по необходимому признаку, поэтому интеграл расходится по 
опредлению Гейне.\\
Можно доказать то же самое по критерию Коши. Именно, при $\alpha\leqslant 0$:
$$\exists \varepsilon>0~\forall \delta~\exists y_1,y_2>\delta:
\left| \int\limits_{y_1}^{y_2} \frac{\sin(x)}{x^\alpha}\right|>
\varepsilon$$ 
Чтобы убить модули, выберем такие пределы интегрирования, на которых
синус знакопостоянен. Имеем 
$$\int\limits_{2\pi n}^{2\pi n+n} \frac{\sin(x)}{x^\alpha}dx=
\frac{1}{\xi^\alpha_n}\cdot 2,~2\pi n\leqslant \xi_n\leqslant 2\pi n+\pi$$
Подставив худший вариант, получаем $\frac{2}{(2\pi n)^\alpha}\geqslant 2$,
то есть расходимость.\\
Соберем ответ: 
$$\begin{cases}
    1.~ \alpha>1 - \text{сходится абсолютно;}\\
    2.~ 0<\alpha\leqslant 1 - \text{сходится условно;}\\
    3.~ \alpha\leqslant 0 - \text{расходится.}
\end{cases}$$



















