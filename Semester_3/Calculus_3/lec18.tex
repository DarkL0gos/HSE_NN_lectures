\begin{theor}
    (о дифференцировании несобственного интеграла, зависящего от параметра)\\
    Пусть\\
    1. $f(x,y)$ непрерывна на  $[a,\infty)\times Y$\\
    2. $\int\limits_{a}^{\infty} f(x,y)dx$ сходится  $\forall y\in Y$.\\
3. $\frac{\partial f}{\partial y}(x,y)$ непрерывна на $[a,\infty)\times Y$.\\
4. $\int\limits_{a}^{\infty} \frac{\partial f}{\partial y}(x,y)dx$ сходится
равномерно на $Y$.\\
    Тогда $\forall y\in Y$ 
$$\left( \int\limits_{a}^{\infty}f(x,y) dx \right)'_y=\int\limits_{a}^{\infty}
 \frac{\partial f}{\partial y}(x,y)dx$$

   % $$\int\limits_{c}^{d}dy \int\limits_{a}^{\infty}f(x,y)dx =
   % \int\limits_{a}^{\infty}dx \int\limits_{c}^{d}f(x,y)dy$$
\end{theor}
\textbf{Доказательство.} Так как выполняются условия теоремы об интегрировании
н.и.з.от п. для $\int\limits_{a}^{\infty}\frac{\partial f}{\partial y}(x,y)dx$,
то зафиксируем $y_0\in Y,~y\in \tilde Y$ без крайних точек, и тогда
$$\int\limits_{y_0}^{y}dy \int\limits_{a}^{\infty} 
\frac{\partial f}{\partial y}(x,y)dx=\int\limits_{a}^{\infty}dx
\int\limits_{y_0}^{y} \frac{\partial f}{\partial y}(x,y)dy=
\int\limits_{a}^{\infty}f(x,y)dx-\int\limits_{a}^{\infty}f(x,y_0)dx$$
Второй интеграл равен числу, поэтому
$$\left( \int\limits_{y_0}^{y} dy \int\limits_{a}^{\infty}
\frac{\partial f}{\partial y}(x,y) dx \right)'_y=
\int\limits_{a}^{\infty}\frac{\partial f}{\partial y}(x,y)dx 
=\left( \int\limits_{a}^{\infty} f(x,y)dx \right)'_y$$ $\square$ 
%готово

\textbf{Пример.} $\int\limits_{0}^{\infty} \frac{\sin\alpha x}{x}
e^{-\beta x}dx$, где $\alpha\in \mathbb{R}$, $\beta\in 0$. 
Легко проверяются условия теоремы об интегрировании, и мы можем 
привести интеграл к виду $\int\limits_{0}^{\infty}dx
\int\limits_{0}^{\alpha} \cos{xy}\cdot e^{-\beta x}dy$, который берется по
частям, ответ $arctg \frac{\alpha}{\beta}$. Так как арктангенс нечетный, 
то эта формула справедлива как для положительных, так и отрицательных
$\alpha$ (при условии $\beta>0$). Другой способ - по теореме о
дифференцировании. Снова обозначим $\Phi(\alpha)=
\int\limits_{0}^{\infty} \frac{\sin{\alpha x}}{x}e^{-\beta x}dx$. 
$\Phi(x,\alpha)$ непрерывна на $[0,\infty)\times \mathbb{R}$. Доопределим
функцию: $\Phi(0,\alpha)=\lim\limits_{x \to +0} \Phi=\alpha$. Снова легко
проверяются условия теоремы. Имеем $\Phi(\alpha)=
\int\limits_{0}^{\infty} \cos{\alpha x}e^{-\beta x}dx=\frac{\beta}{
    \beta^2+\alpha^2}$. Интегрируя и подставляя начально условие, снова 
    получаем арктангенс
\subsection{Вычисление некоторых классических интегралов}

\textbf{Интеграл Дирихле}
$$\boxed{
\int\limits_{0}^{\infty}\frac{\sin\alpha x}{x}dx=\frac{\pi}{2}\sign\alpha}$$
Пусть $\Phi(\beta)=\int\limits_{a}^{\infty}\frac{\sin\alpha x}{x}e^{
-\beta x}dx$.\\
1) $\varphi(x,b)=\frac{\sin\alpha x}{x}e^{-\beta x}$ непрерывна на 
$[0,\infty)\times [0,\infty)$, $\varphi(0,\beta)=\alpha$. \\
2) Докажем, что $\Phi(\beta)$ сходится равномерно на $[0,\infty)$.
Так как $\forall \alpha\in \mathbb{R}: \int\limits_{0}^{\infty}
\frac{\sin\alpha x}{x}dx$ сходится по признаку Дирихле, то сходимость
исходного интеграла равномерная (так как не зависит от $\beta$). 
Далее, $0\leqslant e^{-\beta x}\leqslant 1$ при $\beta\geqslant 0$,
$x\geqslant 0$. Значит, $e^{-\beta x}$ при данных условиях.
По признаку Абеля $\Phi(\beta)=\int\limits_{0}^{\infty}\frac{\sin\alpha x}{
x}e^{-\beta x}dx$ сходится равномерно. По теореме о предельном переходе под
знаком несобственного интеграла на $[0,\infty)$, имеем
$$\Phi(0)=\lim\limits_{\beta \to 0}\Phi(\beta)=\lim\limits_{\beta \to 0}
\arctg \frac{\alpha}{\beta}=\frac{\pi}{2}\sign\alpha$$

\textbf{Интеграл Лапласа}
$$\boxed{
    \int\limits_{0}^{\infty} \frac{\cos\alpha x}{1+x^2}dx=\frac{\pi}{2}
    e^{-|\alpha|}
}$$
Дифференцировать много раз не получится, так как мы придем к расходящемуся 
ряду. Нам нужем финт ушами, а именно прибавить $\frac{\pi}{2}$:
$$\Phi'(\alpha)+\frac{\pi}{2}=-\int\limits_{0}^{\infty} \frac{x\sin\alpha x}{
1+x^2}dx+\int\limits_{0}^{\infty} \frac{\sin\alpha x}{x}dx=
\int\limits_{0}^{\infty}\frac{1+x^2-x^2}{x(1+x^2)}\sin\alpha x\,dx$$
$$\Phi''(\alpha)=\left(\Phi'(\alpha)+\frac{\pi}{2}\right)'
=\int\limits_{0}^{\infty}
\left( \frac{\sin\alpha x}{x(1+x^2)}\right)'_\alpha dx$$
Внезапно, мы получили диффур $\Phi''(\alpha)=\Phi(\alpha)$. Общее решение
$\Phi(\alpha)=C_1e^{-\alpha}+C_2e_6{\alpha}$. Поскольку $\Phi$ ограничена,
то  $C_2=0$, а поскольку $\Phi(0)=\int\limits_{0}^{\infty}\frac{dx}{1+x^2}=
\frac{\pi}{2}$, то $C_1=\frac{\pi}{2}$. 





