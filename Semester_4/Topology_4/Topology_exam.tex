\documentclass[a4paper]{article}

%Общие настройки документа
\usepackage[14pt]{extsizes}                                         %Размер шрифта
\usepackage[left=2.5cm,right=2.5cm,top=2.5cm,bottom=3cm]{geometry}  %Поля страницы

%Настройки ссылок и гиперссылок
\usepackage{float}
\usepackage{graphicx}
\usepackage{hyperref}                 
\usepackage{xcolor}
\definecolor{linkcolor}{HTML}{799B03} % цвет ссылок
\definecolor{urlcolor}{HTML}{799B03}  % цвет гиперссылок
\hypersetup{pdfstartview=FitH,linkcolor=linkcolor,urlcolor=urlcolor,colorlinks=true}
\graphicspath{{figures/}}


%Пакеты символов
\usepackage{cmap}
\usepackage[T2A]{fontenc}
\usepackage[utf8]{inputenc}
\usepackage[russian]{babel}           
\usepackage{amsmath}
\usepackage{amssymb}
\usepackage{amsfonts}

%Новые команды 
\newtheorem{defin}{Определение}
\newtheorem{example}{Пример}
\newtheorem{zam}{Замечание}
\newtheorem{prop}{Утверждение}
\newtheorem{theor}{Теорема}

\author{Жукова}
\title{Вопросы к последнему экзамену}
\date{}

\begin{document}
\maketitle
%\tableofcontents
%\newpage
\section{Гомотопическая эквивалентность отображений есть отношение
эквивалентности на множестве всех непрерывных отображений двух
топологических пространств.}
\begin{defin}
Пусть $f,g\colon X\to Y$ --- непрерывные отображения топологических
пространств. Говорят, что $f$ гомотопно  $g$, если существует непрерывное
отображение $\Phi\colon X\times [0,1]\to Y$, такое что
$$\Phi(\cdot ,0) = f(\cdot )$$
$$\Phi(\cdot ,1) = g(\cdot )$$
\end{defin}
Обозначение: $f\sim g$


\begin{prop}
Гомотопность отображений --- отношение эквивалентности.
\end{prop}
\textbf{Доказательство.}















\end{document}
