\documentclass[a4paper]{article}

%Общие настройки документа
\usepackage[14pt]{extsizes}                                         %Размер шрифта
\usepackage[left=2.5cm,right=2.5cm,top=2.5cm,bottom=3cm]{geometry}  %Поля страницы

%Настройки ссылок и гиперссылок
\usepackage{float}
%\usepackage{graphicx}
\usepackage{hyperref}                 
\usepackage{xcolor}
\definecolor{linkcolor}{HTML}{799B03} % цвет ссылок
\definecolor{urlcolor}{HTML}{799B03}  % цвет гиперссылок
\hypersetup{pdfstartview=FitH,linkcolor=linkcolor,urlcolor=urlcolor,colorlinks=true}
%graphicspath{{\figures}}


%Пакеты символов
\usepackage{cmap}
\usepackage[T2A]{fontenc}
\usepackage[utf8]{inputenc}
\usepackage[russian]{babel}           
\usepackage{amsmath}
\usepackage{amssymb}
\usepackage{amsfonts}

%Новые команды 
\newtheorem{defin}{Определение}
\newtheorem{example}{Пример}
\newtheorem{zam}{Замечание}
\newtheorem{theor}{Теорема}
\newtheorem{lemma}{Лемма}

\author{Жукова Н. И.}
%{Топология - 4}
\title{Лекция 22. Эйлерова характеристика}
\date{09.03.2023}

\begin{document}
\maketitle
%\tableofcontents
%\newpage
%\begin{defin}
%Канонический многоугольник
%\end{defin}
%\begin{defin}
%Разбиение
%\end{defin}
%\begin{defin}
%Укрупнение
%\end{defin}
Пусть $X$ - замкнутая поверхность (связное компактное двумерное многообразие
без края). Рассмотрим конечное представление $\Pi = \{M_i\mid i \in I\}$ 
поверхности $X$.
\begin{defin}
Эйлеровой характеристикой представления П поверхности Х называется
число 
$$\chi(\Pi) = \alpha_0-\alpha_1+\alpha_2$$
где $\alpha_0$ - число классов склеиваемых вершин, $\alpha_1$ - число 
пар склеиваемых сторон, $\alpha_2$ - число многоугольников в представлении.
\end{defin}
\begin{theor}\label{euler}
 Эйлерова характеристика не меняется при элеемнтарных преобразованиях 
 поверхности.
 \end{theor}
 \textbf{Доказательство.} 
 Рассмотрим преобразование первого типа (1-укрупнение, т.е. удаление вершины).
 Что произошло:
$\alpha'_2 = \alpha_2$, $\alpha'_1 = \alpha_1 - 1$, $\alpha'_0 = 
\alpha_0-1$. Тогда
$$\chi(\Pi) = \alpha_0-\alpha_1+\alpha_2 = (\alpha_0-1) - (\alpha_1-1)+
\alpha_2 = 
\alpha'_0 - \alpha'_1 + \alpha'_2 = \chi(\Pi')
$$
То же верно и для обратной операции.

Теперь рассмотрим 2-разбиение (разъединение сторон). 
Тогда из $\Pi = \{M_i\}_{i\in I}$ получается
$\Pi' = \{M'_i\} = \{M_j\mid j\in I\setminus \{i\} \}\cup \{M_{i1},M_{i2}\}$.
Имеем 
$\alpha'_2 = \alpha_2+1$, $\alpha'_1 = \alpha_1 + 1$, $\alpha'_0 = \alpha_0$.
Получаем
$$\chi(\Pi) = \alpha_0-\alpha_1+\alpha_2 =
\alpha'_0 - \alpha'_1 + \alpha'_2 = \chi(\Pi')$$
Аналогично проверяется для обратной операции. 
$\square$ 
\begin{zam}
    При элементарных преобразованиях поверхность заменятся гомеоморфной,
    то естьне меняется с точки зрения топологии.
\end{zam}
\begin{theor}
Любое представление любой замкнутой поверхности можно привести 
к каноническому виду с помощью элеемнтарных преобразований.
\end{theor}
\textbf{План доказательства.}\\
1. Склейка многоугольников представления в один;\\
2. Уничтожение рядом стоящих сторон вида $...aa^{-1}...$;\\
3. Склейка всех вершин в одну;\\
4. Выделение плёнок;\\
5. Выделение ручек;\\
6. Приведение к каноническому многоугольнику;

\textbf{Доказательство.}\\
1. Пусть $X$ - любая замкнутая поверхность и 
$\Pi=\{M_i\mid i=\overline{1,k}\}$ --- её представление. Выберем $M_1$. 
Выберем сторону $a\subset M_1$, которая склеивается со стороной из
другого многоугольника $M_i,i\ne 1$. В противном случае, если  $M_1$
склеивалась бы только с собой, $\Pi$ не являлось бы связным правильным
семейством многоугольников, разбивающееся на компоненты связности
$\{M_1\}$ и $\{M_i\mid i = \overline{2,k}\}$. Итак, проведем склейку по $a$,
получим новое семейство  $\Pi' = \{M'_i\mid i = \overline{1,k-1}\}$. 
Мы снова получили представление поверхности. Повторяя эту процедуру 
конечное число ($k-1$) раз, мы придем к представлению поверхности
единственным многоугольником $M$.

2.1 Пусть других сторон нет, и мы имеем $M$ - двуугольник со схемой 
$aa^{-1}$. Это - канонический многоугольник (сфера). 

2.2 Пусть остался $M$ со схемой $...aa^{-1}...$. Тогда найдется вершина
$D$, не принадлежащая сторонам $a = AB$,  $a^{-1} = CB$. Проведем 2-разбиение,
создав сторону $b = BD$, а потом --- 2-укрупнение, склеив вершины $A$ и  $C$,
при этом вершина $B$ уйдет вглубь. 

3. Докажем, что всегда можно уменьшить число вершин, с которыми не 
склеивается конкретная вершина. Пусть $A$ не склеивается с $B$, 
елси там одна картинка, то укрупнение.
%%% Что значит уничтожить? Стереть с лица земли
Если другая картинка:
$\overline{AB} = x$, $\overline{BC}=y$.
Тогда $y\ne x,y\ne x^{-1}$. Найдется такая сторона, что $\overline{EF}=y$
(склеивается с $AB$). Отрежем треугольник $ABC$ и приклеим его к 
$FE$, поличим треугольник $FEA'$. Значит, число вершин многоугольника 
не изменилось, а число вершин, которые склеиваются с  $A$, увеличилось.
Мы придем к тому, что все вершины будут лежат в одном классе склеивания.


4. Пусть в схеме $\Pi = \{M\}$ имеется выражение $..c...c..$. Тогда 
говорят, что есть невыделенная плёнка Мёбиуса. Соединим начала сторон
$c$ стороной $x$, произведем  картинку. 
Получили $...xx..$. Это и будет выделенная пленка. Заметим, что 
ранее выделенные пленки не разрушаются. 

5. Предположим, что в схеме еcть выражение $..a..a^{-1}..$ и 
существует  $b$ такое, что  $..a..b..a^{-1}..b^{-1}..$.
Пусть $x$ соединяет начала ребер $b$. Разрежем по  $b$ И склеим по  $x$. 
После этого разрежем по  $y$, соединяющему начала $x$, и снова склем по 
$y$. Получили выделенную ручку:  $...yx^{-1}y^{-1}x... = yzy^{-1}z^{-1}$. 

6. Если в многоугольнике содержатся только выделенные пленки 
$a_1a_1...a_qa_q$, то $X\cong \mathbb{N}^2_q$ --- поверхность гомеоморфна 
сфере с $q$ пленками.
Если\\ $a_1b_1a_1^{-1}b_1^{-1}...a_pb_pa_p^{-1}b_p^{-1}$, то
$\mathbb{S}^2_p$ --- сфера с ручками.\\
Что будет, если у нас будут пленки и ручки? Покажем, что 
плёнку + ручку можно перевести в 3 плёнки. Пусть $...aba^{-1}b^{-1}..cc..$ 
--- схема поверхнсоти. Разрежем по $x$, соединяющей середины этих 
комбинаций букв. Склеив по  $c$, мы получим 3 невыделенные пленки.
Снова применим шаг 4. 

Таким образом, любое представление замкнутой поверхности 
элементарными преобразованиями приводится к некоторому каноническому
многоугольнику, так как при элементарных преобразованиях поверхность 
остается гомеоморфной самой себе. 
$\square$ \\
\textbf{Следствие.} Каждая поверхность гомеоморфна сфере с $n$ ручками
или  $k$ пленками Мёбиуса. 

Теперь покажем, что любое преобразование может привести только к одной
канонической поверхности.
\begin{lemma}\label{perehod}
    От любого представления замкнутой поверхности можно перейти 
    к любому другому представлению поверхности с помощью конечного числа
    элементарных преобразований. 
\end{lemma}
\textbf{Доказательство.} Пусть $\Pi = \{M_i\}_{i\in \overline{1,k}},
\Pi'=\{M_j\}_{j\in \overline{1,m}}$ --- два 
представления замкнутой поверхности $X$. Говорят, что эти представления
находятся в общем положении, если каждая вершина каждого многоугольника 
из $\Pi$ лежит внутри многоугольника из $\Pi'$, и наоборот, и всякая 
сторона каждого многоугольника из $\Pi$ пересекает каждую сторону 
каждого многоугольника из  $\Pi'$ по конечному множеству, возможно 
пустому (то есть стороны разных представлений не лежат друг на друге). 

Малой деформацией любые два представления можно привести к общему положению. 
Не нарушая общности, считаем, что $\Pi,\Pi'$ находятся в общем положении. 
Назовем измельчением представления $\Pi$ любое представление
$\Pi^*$, которое можно двумерными укрупнениями привести к  $\Pi$. 
Для представлений, находящихся в общем положении, существует общее 
измельчение, чьи вершины - вершины исходных представлений + 
точки пересечения сторон. Очевидно, от этого нового представления можно
перейти к любому из исходных с помощью укрупнений. Поэтому можно 
переходить и от $\Pi$ к  $\Pi'$, так как все обратимо. $\square$

\begin{defin}
Эйлерова характеристика поверхности Х называется эйлерова характеристика 
любого представления этой поверхности.
\end{defin}
Корректность определения: для каждой поверхности существует некоторое
представление $\Pi$ (например, триангуляция). Тогда положим
$\chi(X):=\chi(\Pi) = \alpha_0 - \alpha_1 + \alpha_2$. По лемме \ref{perehod}, 
между любыми представлениями можно переходить за конечное число 
элементарных преобразований, по теореме \ref{euler} эйлерова харатеристика
на меняется. Значит, эйлерова характеристика не зависит от представления.

\begin{theor}
Пусть Х --- каноническая поверхность. Тогда


\end{theor}
\textbf{Доказательство.}  
1. $\chi(\mathbb{S}^2) = 2$ (так как две вершины). \\
2. $\chi(\mathbb{S}^2_p),p\geqslant 1$ - сфера с ручками:
$a_1b_1a_1^{-1}b_1^{-1}...a_pb_pa_p^{-1}b_p^{-1}$.
Значит, $\chi = 2 - 2p$\\
3. $\chi(\mathbb{N}_q),q\geqslant 1$ - сфера с плёнками. 
Тогда $\chi = 2 - q$. $\square$ \\
\textbf{Следствие.} Эйлеровы хаарктеристикы сферы с $p$ ручками и сферы
с $q$ пленками совпадают тогда и только тогда, когда $q=2p$. 
\begin{theor}\label{euler_invariant}
Эйлерова характеристика --- топологический инвариант.
\end{theor}
\textbf{Доказательство.}  
Пусть $f\colon X\to Y$ --- гомеоморфизм замкнутых поверхностей, и пусть
$\Pi$ - представление для  $X$. Тогда под действием  $f$ 
представление  $\Pi$ перейдет в представление для  $Y$ (поскольку 
топологические многоугольники переходят в в топологические многоугольники). 
Значит, гомеоморфные поверхности имеют общее представление, то есть 
у них одинаковая эйлерова характеристика. Итак, эйлерова характеристика ---
топологический инвариант. $\square$ \\
\begin{zam}
    Эйлеровая характеристика --- не полный инвариант, так как она может не
    различать ориентируемые и неориентируемые поверхности. 
\end{zam}

% лекция 24, 23.03.2003
\section{Ориентрируемые поверхности}
\begin{defin}
Замкнутая поверхность $X$ называется ориентируемой, если
существует такое представление поверхности  
$\Pi = \{M_i\mid i = \overline{1,k}\}$, в котором можно задать ориентацию 
каждого многоугольника так, чтобы склеиваемые стороны проходились в 
противоположных направлениях, то есть в направлениях $a$ и  $a^{-1}$. В 
противном случае поверхность называется неориентируемой. 
\end{defin}
Ориентируемость --- свойство поверхности быть ориентируемой.
\begin{theor} (свойства ориентируемости)\\
1. Ориентируемость поверхности не зависит от выбора представления.\\
2. Ориентируемость --- топологический инвариант поверхности.
\end{theor}
\textbf{Доказательство.}  
1. Заметим, что ориентрируемость совхраняется при элементарных 
преобразованиях. Сначала рассмотрим одномерное укрупнение. 
Когда мы склеим две стороны, то при правильной склейке направление обхода
должно совпадать. Также и при разбиении.

По лемме \ref{perehod}, от любого представления к любому представлению 
можно перейти с помощью конечного числа элементарных преобразований, значит 
ориентируемость не зависит от представления. 

2. Пусть две поверхности гомеоморфны, и $f\colon X\to Y$ ---
гомеоморфизм замкнутых поверхностей. Согласно доказательству теоремы
\ref{euler_invariant}, у гомеоморфных поверхностей существует общее
представление. Значит, ориентируемость --- топологический инвариант.
$\square$ 
\begin{theor} (ориентируемость канонических поверхностей)\\
1. Сфера ориентриуема;\\
2. Сфера с $p\geqslant 1$ ручками ориентируема;\\
3. Сфера с $q\geqslant 1$ пленками неориентируема.\\
\end{theor}
\textbf{Доказательство.}  
1. У сферы существует ориентируемое представление:
$aa^{-1}$.\\
2. У сферы с ручками существует ориентируемое представление\\
$a_1b_1a_1^{-1}b_1^{-1}...a_pb_pa_p^{-1}b_p^{-1}$.\\
3. У сферы с пленками каноническое представление  
$a_1a_1...a_qa_q$ неориентируемое. По теореме об инвариантности 
ориентируемости, из этого следует неориентируемость поверхности.
$\square$ 

\textbf{Следствие.} $\mathbb{S}^2_p\not\cong \mathbb{N}^2_q$ 

\begin{theor} (о классификации замкнутых поверхностей)\\

    Полная система топологических инвариантов для замкнутой поверхности ---
    эйлерова характеристика и ориентируемость. Иначе говоря:
    $X\cong Y\iff \chi(X) = \chi(Y) \wedge$ у них совпадает ориентируемость. 
\end{theor}
\textbf{Доказательство.} Необходимость. Так как 
эйлерова характеристика и ориентируемость являются топологическими 
инвариантами, то они совпадают. Достаточность. 
Каждая замкнутая поверхность гомеоморфна одной из канонических, которые мы 
различаем по эйлеровой характеристике и ориентируемости.
$\square$ 
\begin{zam}
    Нечетная эйлерова характеристика однозначно указывает на неориентируемость,
    т.е. она полный инвариант. 
\end{zam}
\begin{zam}
    Максимальная эйлерова характеристика, равная 2, соответствует сфере.
    Эйлерова характеристика, равная 1, соответствует проективной плоскости.
    Эйлерова характеристика, равная 0, соответствует тору либо бутылке Клейна.
    Эйлерова характеристика, равная -1, соответствует кренделю (сфере с двумя
    ручками).
\end{zam}
\section{Классификация двумерных компактных многообраззий с краем}
Пусть $M$ --- связное двумерное компактное многообразие с непустоым краем
$\partial M\ne\varnothing$. Из компактности следует, что край имеет 
конечное число компонент связности.
Обозначим $\mathbb{S}^2_{p,n}$ сферу с $n$ "дырами" и $p$ ручками, 
$\mathbb{N}^2_{q,n}$ сферу с $n$ "дырами" и $q$ пленками.
\begin{defin}
    Многообразия $\mathbb{S}^2_{p\geqslant 0,n\in \mathbb{N}}$, 
$\mathbb{N}^2_{q\geqslant 0,n\in \mathbb{N}}$ называются каноническими
поверхностями с непустым краем.
\end{defin}
\begin{theor}
Любая связная компактная поверхность с непустым краем гомеоморфна одной из
следующих канонических поверхностей:\\
1. $\mathbb{S}^2_{0,n}$ ;\\
2. $\mathbb{S}^2_{p,n}$ ;\\
3. $\mathbb{N}^2_{q,n}$ ;\\


\end{theor}
\textbf{Доказательство.}  Заклеим "дыры", тогда мы получим одну из
канонических замкнутых поверхностей, для которых получена классификационная
теорема. 
$\square$ 

\section{Теорема Эйлера о многогранниках}
\begin{defin}
Многогранник называется выпуклым, если он лежит по одну сторону от
плоскости, содержащей любую его грань, в противном случае называется 
невыпуклым.
\end{defin}
\begin{theor} (Эйлер)\\
    $$B - P + \Gamma = 2$$ 
    где В --- число вершин, Р --- число ребер, Г --- число граней.
\end{theor}
\textbf{Доказательство.} Выпуклый многогранник гомеоморфен сфере, значит,
его эйлерова характеристика как замкнутой поверхности равна 2. С другой 
стороны, у многогранника есть представление в виде виде его граней. 
$\square$ \\
\textbf{Историческая справка.} Декарт в 1620 г. доказал, что сумма всех 
углов граней выпуклого многоугольника равна $360^0(P-\Gamma) = 
360^0(B - 2)$. 1750: Эйлер. 1811 --- Коши предложил более строгое
доказательство. 1812 Симон Люилье: при рассмотрении кристаллов обнаружил
эйлерову характеристику, отличную от 2. 20 век: при рассмотрении
симплициальных комплексов также возникает отличная от 2 эйлерова 
характеристика. 














\end{document}
