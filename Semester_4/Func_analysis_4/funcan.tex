\documentclass[a4paper]{article}

%Общие настройки документа
\usepackage[14pt]{extsizes}                                         %Размер шрифта
\usepackage[left=2.5cm,right=2.5cm,top=2.5cm,bottom=3cm]{geometry}  %Поля страницы

%Настройки ссылок и гиперссылок
\usepackage{float}
%\usepackage{graphicx}
%\usepackage{hyperref}                 
%\usepackage{xcolor}
%\definecolor{linkcolor}{HTML}{799B03} % цвет ссылок
%\definecolor{urlcolor}{HTML}{799B03}  % цвет гиперссылок
%\hypersetup{pdfstartview=FitH,linkcolor=linkcolor,urlcolor=urlcolor,colorlinks=true}
%graphicspath{{\figures}}


%Пакеты символов
\usepackage{cmap}
\usepackage[T2A]{fontenc}
\usepackage[utf8]{inputenc}
\usepackage[russian]{babel}           
\usepackage{amsmath}
\usepackage{amssymb}
\usepackage{amsfonts}

%Новые команды 
\newtheorem{defin}{Определение}
\newtheorem{example}{Пример}
\newtheorem{zam}{Замечание}
\newtheorem{theor}{Теорема}
\DeclareMathOperator{\Hom}{Hom}
\author{Олег Галкин}
\title{Лекции по функциональному анализу}
\date{20.01.22}

\begin{document}
\maketitle
\tableofcontents
\newpage
\section{Множества}
Зафиксируем универс $\mathcal{U}$.
\begin{defin}
Характеристическая функция (индикатор) множества 
$A\subset \mathcal{U}$:
$$I_A\colon\mathcal{U}\to \{0,1\};~I_A(x)=\begin{cases}
    1,x\in M\\0,x\notin M 
\end{cases}$$
\end{defin}
Обозначим за $\Hom(\mathcal{U},\{0,1\})$ множество всех отображений из 
универса в двухэлементое множество. Рассмотрим оператор
$$\varphi\colon\mathcal{P}(\mathcal{U})\to\Hom(\mathcal{U},\{0,1\});~
A\mapsto I_A$$
Мы построили функтор из категории подножеств универса в хом-категорию. 
\begin{theor}
$\varphi$ - биекция.
\end{theor}
\textbf{Доказательство.}  Инъективность следует из того, что различным 
множествам соответсвуют различные индикаторы. Докажем сюръективность. 
Пусть $I$ - произвольный индикатор. Сопоставим ему следующее множество:
 $$A_I=\{x\in \mathcal{U}\mid I(x)=1\}$$
$\square$ 

Как соотносятся операции над множествами и индикаторы?
Имеем $I_{A\cap B}=  I_A$, $I_{A\cup B} = I_A+A_B-I_AI_B$. Можно придумать 
и другие функции: например, $I_{A\cup B} = I_A^3+A_B^2-I_AI_B=\max\limits_{}$

Пусть $\varphi$ - оператор, $A_i$ - последовательность множеств. 
Построим по ней последовательность индикаторов $I_{A_i}$.
Предел функциональной последовательности 
$\lim\limits_{i \to \infty}I_{A_i}=I_A$ 
является индикаторов некоего множества $A$.
\begin{defin}
Множество А является пределом последовательности множеств $A_i$.
\end{defin}
\begin{defin}
Верхний предел множеств: 
\begin{equation}
    \varlimsup\limits_{n \to \infty} a_n = \lim\limits_{n \to \infty} 
    \sup\limits_{} A_n :=\bigcap\limits_{n=1}^\infty
    \bigcup\limits_{k=n}^\infty A_k
\end{equation}
\end{defin}

\textbf{Задача.} Доказать $\varliminf\limits_{n \to \infty} A_n\subset
 \varlimsup\limits_{n \to \infty}A_n$. Решение: по определению имеем
$$\varliminf\limits_{n \to \infty} A_n = \bigcup\limits_{n=1}^\infty
\bigcap\limits_{k=n}^\infty A_k=\left( A_1\cap ... \right)\cup
\left( A_2\cap... \right)\cup...$$
Если $x$ лежит в этом множестве, то  
$\exists  n:x\in \bigcap\limits_{k=n}^\infty A_k$. То есть 
$\forall  k\geqslant n_0:x\in A_k$. 
Теперь докажем, что этот $x\in \varlimsup\limits_{n \to \infty} A_n = 
(A_1\cup A_2\cup...)\cap...\cap(A_n\cup A_{n+1}\cup ...)\cap...$
%Это означает, что $\forall  n:x\in \bigcup\limits_{n=k}^\infty A_k$.
%Значит, $\exists  $
\begin{defin}
Нижний предел называется пределом последовательности (множеств),
если он совпадает с верхним пределом. 
\end{defin} 
\textbf{Задача.} Доказать по определению, что если
$A_1\subset A_2\subset A_3\subset ...$, то 
$\exists  \lim\limits_{n \to \infty} A_n$. Решение: найдем верхний и нижний
пределы:
$\varliminf\limits_{n \to \infty}A_n=\bigcup\limits_{n=1}^\infty
\bigcap\limits_{k=n}^\infty A_k = \bigcup\limits_{n=1}^\infty A_n$,
$\varlimsup\limits_{n \to \infty}\bigcap\limits_{n=1}^\infty
 \bigcup\limits_{k=n}^\infty A_k = \bigcap\limits_{n=1}^\infty
 \bigcup\limits_{k=1}^\infty A_k = \bigcup\limits_{k=1}^\infty A_k$ 
 (в последнем равенстве мы избавились от зависимости от $n$).
 
 \boxed{\text{Дз1: доказать в случае обратных включений}}.

 \textbf{Задача.} Найдем $\varliminf\limits_{n \to \infty} 
 [0,1+\frac{(-1)^n}{n}]$. Имеем
 $\varliminf\limits_{n \to \infty}\bigcup\limits_{n=1}^\infty
 \bigcap\limits_{k=n}^\infty A_k = \bigcup\limits_{n=1}^\infty B_n$.
 Заметим, что $B_n=B_{n-1}=A_n$ при нечетном $n$, поэтому

 \boxed{\text{ДЗ2: доказать }} 
 $C\left( \varliminf\limits_{n \to \infty} A_n \right)=
 \varlimsup\limits_{n \to \infty} CA_n $
 
\boxed{\text{ДЗ3: найти }}
 : ПУсть $A_1=A_3=A_5=...=[0,1]$, четные- $[-1,0]$. Найти верхний и 
 нижний предел. 

\boxed{\text{ДЗ4: доказать:  верхний предел последовательности 
множеств = элементы, лежащие в бесконечном числе множеств последовательности. 
Нижний предел: элементы, лежащие во всех множествах, начиная с некоторого
номера.}}

\section{Функции}
\begin{defin}
Функция $f\colon A\to B$ - это тройка  $(A,B,f)$, где А,В - множество,
f - правило, сопоставляющее каждому элементу множества А ровно один элемент
множества В. A - область определения.
\end{defin}
\begin{defin}
Область определения функции - $\{y\in B\mid \exists  x\in A:f(x)=y\} = 
\{f(x)\mid x\in A\}$. 
\end{defin}
\begin{defin}
Декартово (прямое) произведение - 
$A\times B=\{(x,y)\mid x\in A,y\in B\}$ 
Кортеж - $A_1\times A_2...\times A_n = \{\} $
\end{defin}
Как ввести счетное произведение множеств? 
$A_1\times A_2\times ...=\{(x_1,x_2,...)\mid x_i\in X_i\} = 
\prod\limits_{n=1}^{\infty} A_k = A^\mathbb{N}$. 
Как индексировать декартовы произведения множеством любой мощности?
Записывается как $\{(x_\gamma)_{\gamma\in \Gamma}\mid x_\gamma \in 
A_\gamma\} = 
\prod\limits_{\gamma \in \Gamma}^{}A_\gamma = A^\Gamma$.
\begin{theor}
Существует биекция между $A^\Gamma$ и $\Hom(\Gamma,A)$. 
\end{theor}
\textbf{Доказательство.}  Имеем $A^\Gamma = \prod\limits_{\gamma \in \Gamma}
A_\gamma$ 
Построим следующую биекцию: $\varphi\colon A^\Gamma\to\Hom(\Gamma,A)$,
$$(x_\gamma)_{\gamma\in \Gamma}\mapsto[\gamma\mapsto x_\gamma]$$
Почему это биекция?
$\square$ \\
\textbf{Пример.} Пусть $A=\mathbb{R}$, $\Gamma=\{1,2,3\}$. С одной стороны, 
мы имеем последовательности из 1,2,3, индексированные действительными 
числами, а с другой стороны - функции из множества $\{1,2,3\}$ в 
действительные числа. Тогда биекция $\varphi\colon\mathbb{R}^{\{1,2,3\}}\to
\Hom(\{1,2,3\},\mathbb{R})$ задается следующим образом:
$\varphi(x_1,x_2,x_3) = f:f(1)=x_1,f(2)=x_2,f(3)=x_3$

Из теоремы немедлено получаем следствие:
\begin{theor}
Существует биекция между $\{0,1\}^\Gamma$, $\Hom(\Gamma,\{0,1\})$ и 
$\mathcal{P}(\Gamma)$.
\end{theor}
\textbf{Доказательство.} Первая биекция - по теореме 1. Вторая биекция\\
$\psi\colon\Hom(\Gamma,\{0,1\})\to\mathcal{P}(\Gamma)$ строится так:
$$[f\colon\Gamma\to \{0,1\}]\mapsto \{\gamma\in\Gamma\mid f(\gamma)=1\}$$ 
Обратная к этой биекции сопоставляет множеству индикатор. $\square$ \\
\subsection{Свойства образов и прообразов}
Пусть $f\colon X\to Y$. 
\begin{defin}
$f^{-1}(y)=\{x\in X\mid f(x)=y\}$ - полный прообраз точки у.
Полный прообраз множества - объединение полных прообразов его точек.
\end{defin}
\begin{defin}
Образ множества - множество образов всех его элементов. 
\end{defin}
\textbf{Вопрос:} ДЗ образ пересечения = пересечение образов?
%%%%%%%%%%%%%%%%%%%%%%%%%%%%%%%%%%%%%%%%%%%%%%%%%%%
                   %04.02.23%
%%%%%%%%%%%%%%%%%%%%%%%%%%%%%%%%%%%%%%%%%%%%%%%%%%%


\textbf{Вопрос.} Счетеное произведение континуумов - континуум. \\
\textbf{Решение.} 
$\mathbb{R}^\mathbb{N}=\{(a_1,a_2,...)\mid a_i \in \mathbb{R}\}$. Так как
$\mathbb{R}\sim \Hom(\mathbb{N},\{0,1\})\sim \{0,1\}^\mathbb{N}\sim
 \{(x_1,x_2,...)\mid x_i\in \{0,1\}\}$, то $\mathbb{R}^\mathbb{N}\sim
\{(a_1,a_2,...)\mid a_i\in \{0,1\}^\mathbb{N}\}$. Покажем, что
$\{(a_1,a_2,..)\mid a_i\in \{0,1\}^\mathbb{N} \} \sim \{0,1\}^\mathbb{N}$. 
Расположив это все это в бесконечной матрице, используем змейку лол. 




Мы различаем $\Hom$-множества и  $B^A = \{(b_a)_{a\in A}\}$ - множество
индексированных последовательностей. С другой стороны, последовательность 
есть функция на множестве натуральных чисел. 

\section{Сравнение мощностей}
Пусть $\alpha=|A|,\beta = |B|$. 
\begin{defin}
Говорим, что $\alpha<\beta$, если $\alpha\ne \beta$ и
$\exists  B_1\subset B:A\sim B_1$. 
\end{defin}
\begin{defin}
Говорим, что $\alpha\leqslant \beta$, если $\alpha < \beta$ или 
$\alpha=\beta$
\end{defin}
В частности, из аксиомы выбора следует сравнимость любых множеств. 
Следует ли обратно?
\begin{theor}
Любые две мощности сравнимы.
\end{theor}
\textbf{Доказательство.} По аксиоме выбора, любые множества можно вполне
упорядочить (теорема Цермело). Из сравнения порядковых чисел 
следует, что либо $A\sim B_1\subset B$ (1), либо $B\sim A_1\subset A$ (2).
Возможны 
следующие варианты:\\
1. $A\sim B_1\subset B,~B\sim A_1\subset A$. Тогда по КШБ $A\sim B$.\\
2.  (1) и не (2). \\
3. не (1) и (2). 
4. не (1) и не (2) - не бывает. 
$\square$ \\

\subsection{Операции над мощностями}
Пусть $\alpha=|A|,\beta = |B|$.
\begin{defin} Определим следующие операции:\\
 $\alpha\cdot \beta:=|A\times B|$\\
 $\alpha+\beta:=|A\cup B|$\\
 $\alpha^\beta:=|A^B|$
\end{defin}
Докажем, что введенное определение корректно: именно, если
$|A_1|=|A|=\alpha,~|B_1|=|B|=\beta$, тогда:\\
1.  $|A_1\times B_1|=|A\times B|$;\\
2. $|A_1\cup B_1|=|A\cup B|$;\\
3. $|A_1^{B_1}|=|A^B|$.\\
\begin{theor}
Для мощностей выполнены свойства:\\
1. $\alpha \beta = \beta\alpha$\\
2. $(\alpha+\beta)\gamma = \alpha\gamma+\beta\gamma$\\
3. $\alpha^{\beta+\gamma} = \alpha^\beta\alpha^\gamma$\\
4. $(\alpha^\beta)^\gamma = \alpha^{\beta\gamma}$\\
5. $(\alpha\beta)^\gamma = \alpha^\gamma\beta^\gamma$
\end{theor}
\textbf{Доказательство.}\\
1. \\
2. \\
3. \\
4. По определению, $(\alpha^\beta)^\gamma = |(A^B)^C|$,
$\alpha^{\beta\gamma} = |A^{B\times C}|$, поэтому построим биекцию
$\varphi\colon (A^B)^C\to A^{B\times C}$. 

Вспоминаем, что $(A^B)^C = \{(d_c)_{c\in C}\mid d\in A^B\}$, 
$A^{B\times C} = \{(a_{(b,c)})_{(b,c)\in B\times C}\mid 
a_c\in A\}$. $\square$ \\
\boxed{\text{ДЗ №2 ДЗ №2 З Д З Д З Д ЗД З Д З Д З Д ЗД}}

\textbf{Примеры.} 
$2^{\aleph_0}=|\mathbb{R}|$

С одной стороны, $\aleph_0^{\aleph_0}\geqslant 2^{\aleph_0} = \mathfrak C$. 
С другой стороны,
$\aleph_0^{\aleph_0}\geqslant \mathfrak C^{\aleph_0}=\mathfrak C$, откуда
$\aleph_0^{\aleph_0} = \mathfrak C$. 

\begin{theor}
    (Кантор)\\
    $\mathcal{P}(X)\not\sim X$
\end{theor}
\begin{theor}
Если $|A|\geqslant 2$, то $\forall  B\ne \varnothing:|A^B|>B$.
\end{theor}
\textbf{Доказательство.} 
Очевидно из теоремы Кантора, что отношение здесь больше или равно. Исключим
раVенство - докажем, что нет биекции. Именно, нет биекции меZду 
$B$ и  НЕЙ - додокаZать. $\square$ \\


Пример: континум в степени континуум это 2 в степени континума. 

Если максимум существует, то он достигается. 

\newpage
\section{Теория меры}
\begin{defin}
Мера $\mu$ на семействе множеств  $\Sigma$ - функция  
$\mu\colon\Sigma\to\mathbb{R}$, такая что:\\
1. $\forall  A_1,...,A_n\in \Sigma:A_i\cap A_j\ne 0,\bigcup\limits_{i}A_i\in 
\Sigma:\mu(A_1\cup ,...,\cup A_n)=\mu(A_1)+...+\mu(A_n)$ -
конечная аддитивность;\\
2. $\mu(A)\geqslant 0$ - неотрицательность. 
\end{defin}
\begin{defin}
Мера называется счетно-аддитивной, если аддитивность для счетного числа 
множеств:
\end{defin}
$$\mu \left( \bigcup\limits_{i=1}^\infty A_i \right) = 
\sum\limits_{i=1}^{\infty} \mu(A_i)$$
\textbf{Пример.} $\Sigma = \{\langle a,b\rangle \cap \mathbb{Q}\mid 
a\leqslant b\}$, $\mu(\langle a,b\rangle \cap \mathbb{Q}) = b-a$. 
Мера точки - ноль. Имеем
$$\mu([0,1]\cap \mathbb{Q})=1$$ 
но
$$\mu([0,1]\cap \mathbb{Q}) = \sum\limits_{i=1}^{\infty}\mu(c_i)=0,~c_i\in 
[0,1]\cap \mathbb{Q}$$
 Значит, мера не счетно-аддитивна.
 \boxed{\text{ДЗ доказать конечно-аддитивность}}

\textbf{Пример.} $\mu(A\cap B)=\mu(A(\setminus B)\cap B) = 
\mu(A\setminus B)+\mu(B)$. Значит, 
$\mu(A\setminus B)=\mu(A)-\mu(A\cap B)$, 
$$\mu(A\cup B) = \mu(A)+\mu(B)-\mu(A\cap B)$$

\subsection{Сисетемы множеств}
\begin{defin}
Кольцо множеств $R$ --- непустое семейство множеств, причем $\forall  x,y
\in R:x\cap y,x\cup y,x\setminus y, x\triangle y \in R$
\end{defin}
\textbf{Пример.} Множество всех промежутков на прямой - не кольцо, так как
объединение не лежит в нем. Булеан множества - кольцо.
\begin{theor}
Булеан любого множества - кольцо относительно пересечения (умножение)
и симметрической разности (сложение).
\end{theor}
\textbf{Доказательство.}  ДЗДЗДЗДЗДЗДЗДЗЗДЗДЗДЗДЗДЗДЗДДЗДЗДЗДЗДЗДЗДЗДЗДЗД
$\square$ \\
\begin{theor}
Семейство множеств является кольцом множеств $\Leftrightarrow$ оно 
непусто и замнкуто относительно объединения и разности (либо относительно 
пересечения и симметрической разности).
\end{theor}
\textbf{Доказательство.} Докажем, что все операции, фигурирующиев в 
определении кольца множеств, выражаются черех объединение и разность.
Действительно, $A\triangle B = (A\setminus B)\cup (B\setminus A)$,
$A\cap B = (A\cup B)\setminus (A\trianle B)$.
есемейство множеств $R$ $\square$ \\

ДЗДЗДЗДЗДЗДДЗ 333333Полна ли система только с одной операцией?33333333

\begin{theor}
Кольцу множеств принадлежат конечные объединения и пересечения его элементов.
\end{theor}
\textbf{Доказательство.}  По индукции. 
$\square$ \\
\begin{theor}
Пересечение любых колец множеств - кольцо множеств.
\end{theor}
\textbf{Доказательство.} Объединение и пересечение, разность и симметрическая 
разность элементов лежат в каждом из множеств, а значит, и в пересечении 
колец. 
$\square$ \\
Объединение - не всегда кольцо: контрпример доставляют кольца
$\{A,\varnothing \},\{B,\varnothing \}$.
\begin{defin}
Произведение систем множеств - 
$$S_1\times S_2:=\{A_1\times A_2\mid A_1\in S_1,A_2\in S_2\}$$
\end{defin}
\begin{defin}
Если $S$ - система множеств, то порожденным её кольцом называется кольцо
 $R(S)$, такое что  $S\subset R(S)$ и $\forall  R_1:S\subset R_1$ имеет
 место $R(S)\subset R_1$. 
\end{defin}
ДЗДЗДЗДЗЗДЗДЗД найти порожденное кольцо $S=\{\{1,2,3\},\{2,3,4\}\}$
\begin{theor}
Порожденное кольцо существует для любого непустого семейства множеств 
и единственно. 
\end{theor}
\textbf{Доказательство.}  Пусть $X = \bigcup\limits_{A\in S}A$. 
$S\subset \mathcal{P}(X)$ - кольцо. Положим $\tilde R$ - пересечнние
всех колец, содержащих $S$. Покажем, что оно порожденное и 
единственное. 

Единственность. Пусть есть два минимальных кольца. Они лежат друг в друге.
$\square$ \\
\begin{defin}
Полукольцо множеств - семейство множеств $S$, если  замкнуто 
относительно пересечения и разность двух множеств должна представляться
в виде конечного объединения непересекающихся множеств.
\end{defin}
Пример - семейство всех промежутков на прямой. 

ДЗЗЗЗДЗДЗД ЗВЕЗДОЧКА Произведение множеств всех промежутков - кольцо или 
полукольцо?





\end{document}
