
\documentclass[12pt]{book}

\usepackage{amsmath, amssymb, amscd, russcorr, fancyhdr,
  epsfig, theorem, russlh} 

\usepackage[matrix,arrow]{xy}
\usepackage{psboxit} % для ведомостей

%\usepackage{epstopdf,hyperref,indentfirst}

\newcommand{\subs}{\section}
\newcommand{\arrow}{{\:\longrightarrow\:}}
\newcommand{\restrict}[1]{{\left|_{{\phantom{|}\!\!}_{#1}}\right.}}
\def\endproof{\hbox{\vrule width 5pt height 5pt depth 0pt}}


\makeatletter

\@ifundefined{Bbb}
     {\newcommand{\Bbb}[1]{{\mathbb #1}}}%
{}%     {\edef\Bbb#1{{\Bbb #1}}}



%%%%%%%%%%%%%%%%%%%%%%%%%%%%%%%%%%%%%%%%%%%%%%%%%%
%       Pagestyle                                %
%%%%%%%%%%%%%%%%%%%%%%%%%%%%%%%%%%%%%%%%%%%%%%%%%%

% Version 0.9,  03.05.2009 (для Шеня) 
% v. 0.9.1, добавил исправления от Шеня 08.05.2009
% v. 0.9.2, исправления от <ljr user=dimkir82> 21.05.2009
% v. 0.9.3, еще немного от <ljr user=dimkir82>, 21.05.2009
% v. 0.9.4, начал добавлять исправления от Прасолова (Нарита, 7 сент. 2009)
%           Также добавил задачи по анализу (Приложение).
% v. 0.9.5 Закончил исправления от Прасолова (15.09.2009)
% v. 0.9.6 Пара опечаток от анонимов в комментах (16.09.2009)
% v. 0.9.7 Исправления от Ивана Ремизова (18.10.2009)
% v. 0.9.8 Много исправлений от Прасолова и Ремизова
%     (22.11.2009)
% v. 0.9.8.1 Одна опечатка в опр. категории (05.02.2010)
% v. 1.0, куча исправлений от Прасолова (сделанных им в
%     декабре) и от <ljr user=measure0>.
% v. 1.0.1, почерпнуть! без д!
% v. 1.1, 27 июля 2012, добавлены опечатки из курса весны 2012
% v. 1.2, 1 октября 2012, локальная связность - поправил бредятину.


\newcommand{\version}{Миша Вербицкий, version 1.1,\ \   27.07.2012}

\setlength{\headheight}{15pt}
\pagestyle{fancy} \lfoot{\tiny Лекции и задачи по топологии} 
\cfoot{-- \thepage \ -- } \rfoot{\tiny \sc\version}
\rhead{{\small  }}




%%%%%%%%%%%%%%%%%%%%%%%%%%%%%%%%%%%%%%%
%  equations                          %
%%%%%%%%%%%%%%%%%%%%%%%%%%%%%%%%%%%%%%%

\renewcommand{\theequation}{\thesection.\arabic{equation}}

\@addtoreset{equation}{section}
\@addtoreset{footnote}{section}
\makeatother



%%%%%%%%%%%%%%%%%%%%%%%%%%%%%%%%%%%%%%%
%  listki-new.tex                     %
%%%%%%%%%%%%%%%%%%%%%%%%%%%%%%%%%%%%%%%

\renewcommand{\phi}{\varphi}
\renewcommand{\epsilon}{\varepsilon}
\renewcommand{\emptyset}{\varnothing}
\def\1{\sqrt{-1}}
\def\rad{\operatorname{\sf rad}}
\def\Alt{\operatorname{\sf Alt}}
\def\Sym{\operatorname{\sf Sym}}
\def\Id{{\operatorname{\sf Id}}}
\def\Hom{\operatorname{Hom}}
\def\Aut{\operatorname{Aut}}
\def\Vol{\operatorname{Vol}}
\def\Lie{\operatorname{Lie}}
\def\Map{\operatorname{Map}}
\newcommand{\End}{\operatorname{End}}
\newcommand{\Gal}{\operatorname{Gal}}
\newcommand{\Set}{\operatorname{Set}}
\newcommand{\Mat}{\operatorname{Mat}}
\newcommand{\Spec}{\operatorname{Spec}}
\newcommand{\ev}{\operatorname{\sf ev}}
\newcommand{\Ob}{\operatorname{{\cal O}b}}
\newcommand{\Mor}{\operatorname{{\cal M}or}}
\newcommand{\diam}{{\sf diam}}


\newcommand{\cac}{{\cal C}}
\def\Z{{\mathbb Z}}
\def\R{{\mathbb R}}
\def\C{{\mathbb C}}
\def\Q{{\mathbb Q}}
\def\N{{\mathbb N}}


\makeatletter
\begingroup
\gdef\th@upshape{\normalfont
  \def\@begintheorem##1##2{%
        \item[\hskip\labelsep \theorem@headerfont ##1\ ##2.]
%        \writevedomost{\string\mc{\thezadacha} \string & \string & \string 
%        \\ \string \hline}
   }%
\def\@opargbegintheorem##1##2##3{%
   \item[\hskip\labelsep \theorem@headerfont ##1\ ##2\ (##3).]
%   \writevedomost{\string\mc{\thezadacha ##3}  
%\string & \string \grd \string & \string\grp
%   \string \\ \string \hline}
    }}
\endgroup

\theoremstyle{upshape}

\newtheorem{zadacha}{Задача}[chapter]

\begingroup
\gdef\th@generic{\normalfont
  \def\@begintheorem##1##2{%
        \item[\hskip\labelsep \theorem@headerfont ##1\ ##2.]
     }%
\def\@opargbegintheorem##1##2##3{%
   \item[\hskip\labelsep \theorem@headerfont ##1\ ##2\ (##3).]
    }}
\endgroup

\theoremstyle{generic}

\newtheorem{teorema}{Теорема}[chapter]
\newtheorem{opredelenie}[teorema]{Определение}
\newtheorem{remark}[teorema]{Замечание}

\def\замечание{\begin{remark}}
\def\еза{\end{remark}}



\begingroup
\gdef\th@upshapenonumber{\normalfont
  \def\@begintheorem##1##2{%
        \item[\hskip\labelsep \theorem@headerfont ##1.]}%
\def\@opargbegintheorem##1##2##3{%
   \item[\hskip\labelsep \theorem@headerfont ##1\ (##3).]}}
\endgroup

\theoremstyle{upshapenonumber}

\newtheorem{ukazanie}{Указание}[section]
\newtheorem{zamechanie}{Замечание}[chapter]

\renewcommand{\labelenumi}{\ralph{enumi}.}


\newcommand{\listok}[2]{%
\setcounter{page}{1}
\renewcommand{\@oddhead}{\hfil #2 \hfil}
\renewcommand{\@evenhead}{\hfil #2 \hfil}
\section*{#2}
\refstepcounter{section}
\setcounter{section}{#1}
}



%%%%%%%%%%%%%%%%%%%%%%%%%%%%%%%%%%%%%
%   Chapter                         %
%%%%%%%%%%%%%%%%%%%%%%%%%%%%%%%%%%%%%

  \renewcommand{\chapter}{%
     \secdef\ChapterNumbered\ChapterStarred}

  \def\ChapterNumbered[#1]#2{\newpage
      \addcontentsline{toc}{chapter}{#1}
      \refstepcounter{chapter}
      {
\vspace{\baselineskip}
\chaptermark{#1}
          { {{\noindent\bf\huge #2\par}}}
      }
    \vspace{10pt}
}

  \newcommand{\ChapterStarred}[1]{%
      \refstepcounter{chapter}
      \chaptermark{#1}%
      \vspace{\baselineskip}}

\newcommand{\PartName}{}
  \renewcommand{\chaptermark}[1]{\markboth{{\bf \thechapter. \  #1}}{{\bfseries\sc\PartName}}}
  \renewcommand{\sectionmark}[1]{{}}
\makeatletter
\@addtoreset{chapter}{part}
\makeatother                             

\newcommand{\следствие}{%
     \refstepcounter{teorema}
     {\noindent\bf Следствие \thechapter.\arabic{teorema}:\ }}
\newcommand{\пример}{%
     \refstepcounter{teorema}
     {\noindent\bf Пример \thechapter.\arabic{teorema}:\ }}
\newcommand{\лемма}{%
     \refstepcounter{teorema}
     {\noindent\bf Лемма \thechapter.\arabic{teorema}:\ }}
\newcommand{\теорема}{%
     \refstepcounter{teorema}
     {\noindent\bf Теорема \thechapter.\arabic{teorema}:\ }}
\newcommand{\утверждение}{%
     \refstepcounter{teorema}
     {\noindent\bf Утверждение \thechapter.\arabic{teorema}:\ }}

\def\хфилл{\hfill}
\def\ноиндент{\noindent}
\def\бф{\bf}
\def\ем{\em}
\def\ит{\it}
\def\задача{\begin{zadacha}}
\def\ез{\end{zadacha}}
\def\указание{\begin{ukazanie}}
\def\еу{\end{ukazanie}}
\def\eu{\end{ukazanie}}
\def\определение{\begin{opredelenie}}
\def\ео{\end{opredelenie}}
\def\eo{\end{opredelenie}}
\def\goth{\mathfrak}
\def\енум{\begin{enumerate}} 
\def\ее{\end{enumerate}}
\def\ee{\end{enumerate}}
\def\итем{\item %
%    \writevedomost{\string & \ralph{enumi} \string
%      &\string & \string \\ \string \hline } 
}


\begin{document}

%%%%%%%%%%%%%%%%%%%%%%%%%%%%%%%%%%%%%%%%%%%%%%%%

{
\small
\renewcommand{\contentsname}{Содержание}
\tableofcontents
}
%%%%%%%%%%%%%%%%%%%%%%%%%%%%%%%%%%%%%%%%%%%%%%%%

%%%%%%%%%%%%%%%%%%%%%%%%%%%%%%%%%%%%%%%%%%%%%%%%%%%%%%%%%%%%

\chapter{Введение}
\label{_Intro_Section_}

%%%%%%%%%%%%%%%%%%%%%%%%%%%%%%%%%%%%%%%%%%%%%%%%%%%%%%%%%%%%

%%%%%%%%%%%%%%%%%%%%%%%%%%%%%%%%%%%%%%%%%%%%%%%%%%%%%%%%%%%%
\section{Краткое описание}
%%%%%%%%%%%%%%%%%%%%%%%%%%%%%%%%%%%%%%%%%%%%%%%%%%%%%%%%%%%%

Эта книга рассчитана на школьника или студента 
младших курсов, знакомого с основами математического мышления
(хорошего школьного учебника математики достаточно).

Можно читать ее по частям или целиком;
например, решать задачи, пропуская текст лекций.
"Геометрическая" часть задач и лекций (первый том) не очень
связана с алгебраической (второй том), а лекции дополняют
листки с задачами. Задачи разбиты на две группы
(простые задачи без звездочки и сложные --- со звездочкой),
можно решать либо все простые задачи, либо все сложные,
либо и те и другие.

Программа курса в немалой степени основана на 
программе матшколы, и содержит материал, который
в общих чертах известен хорошему матшкольнику.
Полный курс состоит из двух частей, алгебры и геометрии.
В этом томе читатель найдет задачи и лекции по геометрии 
и топологии. В приложении приводятся необходимые 
определения и задачи по основам анализа (определение 
поля вещественных чисел).

\хфилл

Геометрия (1-й том):

\begin{itemize}
\item 0. Метрические пространства. 
Последовательности Коши, пределы, пополнение
метрических пространств. Теорема Хопфа-Ринова.
Геодезические в полных метрических пространствах.
Векторные пространства с нормой.

\item 1. Теоретико-множественная топология
(определение непрерывных отображений, компактность,
отделимость, счетная база).

\item 2. Лемма Урысона и теорема о метризации
нормального топологического пространства со счетной базой.

\item 3.  Теорема Тихонова о компактности, равномерная
сходимость, теорема Арцела-Асколи. 
Конструкция кривой Пеано.

\item 4. Фундаментальная группа, свободные группы,
гомотопическая эквивалентность, накрытия Галуа,
конструкция универсального накрытия.
\end{itemize}


Алгебра (2-й том):

\begin{itemize}

\item 0. Группы, кольца, поля. Действительные и
комплексные числа. Теорема Евклида-Гаусса
об однозначности разложения на простые множители.
Решение простейших диофантовых уравнений.

\item 1. Конечномерные векторные пространства.
 Базис, размерность. Билинейные, полилинейные формы,
двойственные пространства. Определение тензорного 
произведения векторных пространств.
Симплектические и квадратичные формы. 

\item 2. Грассманова алгебра и определители.

\item 3.  Линейные операторы. Полупростота,
нильпотентность. Теорема Кэли-Гамильтона.
Жорданова нормальная форма.

\item 3. Алгебраические расширения полей. Артиновы
коммутативные алгебры. Расширения Галуа.

\item 4. Представления конечных групп.

\item 5. Основная теорема теории Галуа.

\end{itemize}



Последние 3-4 листка по геометрии и по алгебре
повторяют друг друга, местами дословно. Дело в том,
что группа Галуа устроена аналогично фундаментальной
группе, а накрытие топологического пространства --- 
конечному расширению полей. Пользуясь этой аналогией,
Гротендик построил фундаментальную группу,
пользуясь только алгебраическими методами
(этот раздел математики называется {\ем этальной
геометрией}). 

В. И. Арнольд прочел основанный на этой аналогии
курс теории Галуа в физико-математическом интернате 18;
впоследствии его лекции были записаны В. Б. Алексеевым
("Теорема Абеля в задачах и решениях"). 

В силу того, что методы топологии и алгебры
в этих разделах столь схожи, теорию Галуа, фундаментальную группу
и накрытия можно (и нужно) изучать по одному плану. 
Взаимовлияние алгебраических и геометрических идей ---
это магистральное направление всей математики (а в последнее
время --- и теоретической физики), а математик,
который владеет только одним из этих аппаратов,
не лучше инвалида. 

Материал этой книги должен быть в общих чертах 
известен хорошему матшкольнику и продвинутому 
первокурснику-математику. 

Кроме этого, первокурсник должен знать основы анализа;
их можно почерпнуть в учебнике В. А. Зорича
"Математический анализ" и в учебнике Лорана
Шварца "Анализ".

В этой книге анализа нет, потому что 
(в отличие от алгебры и геометрии) преподавание
анализа на первом курсе университета  ведется
весьма интенсивно, и начала анализа непрерывных
и гладких функций на прямой худо-бедно 
усваивает каждый студент. К тому же, изложить
математический анализ в задачах не так просто.

Соавтором и редактором листочков с задачами
был Дмитрий Каледин, которому автор безмерно благодарен.
Спасибо Марине Прохоровой за редакторскую работу над
задачами и А. Х. Шеню за ряд ценных замечаний. 
Структура книги отражает программу, составленную
А. Х. Шенем,  В. А. Гинзбургом и другими 
преподавателями маткласса 57 школы, где учился 
автор. Другим источником идей и вдохновения были
учебники "Теорема Абеля в задачах 
и решениях" Алексеева и "Теоремы и задачи 
функционального анализа" Кириллова и Гвишиани.




%%%%%%%%%%%%%%%%%%%%%%%%%%%%%%%%%%%%%%%%%%%%%%%%%%%%%%%%%%%%

\section{Матклассы: обучение по листочкам}

%%%%%%%%%%%%%%%%%%%%%%%%%%%%%%%%%%%%%%%%%%%%%%%%%%%%%%%%%%%%

В 1970-е в московских матшколах кристаллизовалась
необычная форма обучения математике. Ее возникновение
обыкновенно связывают с именем Н. Н. Константинова, 
который работал в 57, 91 и 179 школах. По этой
системе выучились сотни матклассов, и каждый 
преподаватель вносил нечто свое в программу и в подход к
обучению. Самым известным на настоящий момент практиком
матшкольного обучения по листочкам является
Б. М. Давидович, завуч московской школы 57;
автора этой книги учили А. Х. Шень, В. А. Гинзбург,
Б. П. Гейдман и А. Ю. Вайнтроб, и он благодарен им сверх 
всякой меры.

Здесь был бы уместен исторический очерк матшкольного
образования, но пока придется ограничиться этим куцым сообщением.
Автор заранее приносит извинения всем, кого он не упомянул.


\begin{figure}[ht]
\begin{center}
\epsfig{file=nn-konstantinov.eps,width=0.9\linewidth}\\
Николай Николаевич Константинов
\end{center}
\end{figure}


Система эта в канонической форме устроена так.
Обучение математике в матклассе разбито на два
параллельных предмета. Обычная математика (алгебра
и геометрия) преподается в рамках школьной программы
и сверх того, при этом форма обучения не отличается
от привычной чиновникам РОНО и проверяющим комиссиям.
Параллельно с этим, профессиональные математики,
аспиранты и студенты, не числящиеся формально
учителями, ведут уроки ``специальной математики'',
или же ``матанализа''. Часы делятся примерно поровну,
но само обучение ``специальной математике'' мало
соотносится со школьной программой, и занятия
устроены принципиально иначе.

На уроках ``специальной математики''
никто не стоит у доски с указкой и мелом;
все (или почти все) общение школьника с преподавателями
ведется за партой и тет-а-тет, либо в походах.
Школьникам выдается листок с задачами, обыкновенно ---
по одному или два в неделю; через какое-то время
после выдачи листочка, студенты должны ``сдать
задачи'', то есть рассказать их преподавателям 
на уроке. При такой системе преподавателей на
класс из 30 человек требуется где-то 5-10.

Задачи разбиты на задачи ``без звездочки'' (сдача
этих задач обязательна для всех) и более сложные
задачи, отмеченные одной или двумя звездочками.
Задачи с одной звездочкой должны быть доступны
самым продвинутым школьникам в их классе.
Задачи с двумя звездочками весьма сложны ---  
уровня студенческих научных олимпиад,
либо сложных (а часто и нерешенных)
научных проблем. Для индивидуального
обучения эта система весьма удобна ---
школьник может выбирать себе задачи
по плечу, решая либо сравнительно
простые задачи, доступные начинающим,
либо задачи со звездочкой, требующие
хорошего понимания материала.

Преподаватели подбираются из числа энтузиастов 
подобного обучения, профессиональных математиков
и студентов; в основном это --- выпускники матклассов. 
Они разъясняют школьникам непонятные места.
Также школьникам не возбраняется находить
решение задач в книжках либо (когда совсем припрет)
спрашивать у товарищей. Принято считать, что эта часть
обучения не менее важная, чем собственно
решение задач. Действительно, свободное 
обращение с литературой и способность рассказать
либо выслушать нечто математическое не менее важна,
чем решение задач.

Объем информации, усваиваемый школьником при такой системе,
вполне сравним с полученным из обычной школьной
системы обучения, несмотря на отсутствие ``уроков''
в обычном смысле. Теоретический материал размещается,
по возможности, в тексте задач, таким образом любой
школьник, успешно сдавший задачи, будет обязан
усвоить и освоить теорию.

На протяжении 1980-х программа матклассов сложилась
окончательно, и с тех пор остается практически неизменной.
В общих чертах идеализированная программа 
матшколы устроена примерно так. 

\begin{itemize}
\item Обучение ведется 
3 или 4 года. На первый год, школьники приучаются
обращению с множествами (элементарной теории множеств,
классам эквивалентности, отображениям, наложениям, вложениям 
и биекции, равномощности, счетным и континуальным множествам). 
Излагаются начала аксиоматического подхода.
Определяются понятия элементарной алгебры: группы, кольца 
и поля. Вводится алгоритм Евклида, его используют
для доказательства однозначности разложения на
множители в кольце целых чисел.

\item На второй год, школьники
изучают основы анализа (пределы, ряды, непрерывность
и дифференцируемость функций на прямой),
свойства логарифма и экспоненты. Излагается аксиоматическое
определение вещественных чисел (обыкновенно, через 
последовательности Коши). Проходят комплексные числа
и их геометрическую интерпретацию. Выводят из свойств
комплексных чисел тождества для тригонометрических 
функций, как обычные (формула косинусов и синусов),
так и необычные (формула для $\sin(nx)$ и т.д.).
 Также изучают
начала линейной алгебры (конечномерные пространства,
базис, размерность).

\item На третий год, школьники изучают основы 
теории метрических пространств (компактность,
пополнение) и топологии (аксиоматическое определение
топологического пространства, топологические
свойства метрических пространств, аксиомы
отделимости). В курсе алгебры школьники
усваивают определение p-адических чисел, 
классификацию конечных полей и элементы 
теории Галуа.
\end{itemize}

%
%\section{Зачем это}
%
%Матшкольники, которые попадают на Мехмат МГУ,
%первые 2 или 3 года ничего не делают, потому
%что они все знают. В принципе, практически
%все полезное, что дается математику в 
%современной университетской программе,
%содержится в вышеприведенном куррикулуме.
%В университетской программе содержится
%крайне много вещей неполезных, и их
%приходится учить старшекурсникам;
%так сложилось исторически.
%
%Программа матшколы очень правильная и удобная;
%знание приведенных выше предметов --- основное,
%что требуется в аспирантских экзаменов 
%хорошего западного университета. Также
%требуется знание основ многомерного анализа,
%но его гораздо труднее расказать посредством
%цикла задач, не прибегая к лекциям. Поскольку
%этот материал в школе дается не целиком,
%либо усваивается только частично,  программу 
%первого курса следует базировать на этой 
%же программе. 
%
%Можно прочесть этот материал в виде цикла лекций,
%но проще, наверное дать его первокурсникам в
%привычной форме цикла задач, и реализовать обучение
%по аналогии с матклассом.
%
%Есть ситуация, когда иначе поступить просто нельзя.
%
%Бывают случаи, когда часть студентов происходит
%из матшколы, где их обучили части этих предметов,
%а часть студентов знает только школьную программу.
%Невозможно прочесть лекции, которые будут понятны
%и интересны и той и другой группе. Поэтому обучение
%по необходимости должно быть индивидуально.
%
%Для такой ситуации и написана эта книга.
%
%Независимый Московский Университет ежегодно сталкивается
%с этой проблемой --- большая часть студентов матшкольники.
%Обыкновенно лекции читаются, исходя из того, что базовые
%понятия и большая часть матшкольной программы 
%студентам хорошо известны; это
%приводит к отсеву нематшкольной части студентов. При этом
%и матшкольники (как показывает опыт) по большей части 
%не знают базовых понятий, и отсеиваются немного погодя.
%
%%%%%%%%%%%%%%%%%%%%%%%%%%%%%%%%%%%%%%%%%%%%%%%%%
\section{Как читать эту книгу}
%%%%%%%%%%%%%%%%%%%%%%%%%%%%%%%%%%%%%%%%%%%%%%%%%

В этой книге есть две независимых части,
основанные на одной и той же программе:
цикл лекций и цикл задач. Они в немалой
степени повторяют друг друга, и их можно
читать независимо. По сути это два разных
курса, излагающих один и тот же материал.

Листочки составлены
таким образом, чтобы решение всех задач
со звездочкой из одного листка было
несколько менее трудоемко, чем решение
всех задач без звездочки из этого же листка.
Студенту имеет смысл прочесть все задачи
и усвоить их формулировку, затем решить
все задачи со звездочкой, если задачи
без звездочки для них не трудны и их
решение кажется бессмысленной
затратой труда. Задачи с восклицательным
знаком надо решать всем.

Таким образом, каждый листочек представляет
собой сразу два курса --- один для продвинутых
студентов, которые в общих чертах знают
программу, другой --- для начинающих.

Формально говоря,
для понимания листочков достаточно школьной
программы и знания основных определений
теории множеств (вложение, наложение,
ограничение отображения, классы эквивалентности).
Многие школьные учебники (например, учебник
Колмогорова) уже содержат все нужные
определения. 

Для решения некоторых задач со звездочкой
(особенно в конце курса геометрии) и хорошего 
понимания остального материала, необходимо
немного подробнее ознакомиться с теорией множеств,
в частности, научиться пользоваться 
леммой Цорна. Об этом см. главу
I.\ref{_AC_Chapter_}.

Остальные лекции читать
не обязательно, для владения материалом вполне 
достаточно прорешать задачи. С другой стороны, пытаться
решать задачи подряд и в изоляции от преподавателей
и студентов тоже не очень полезно --- всегда есть
риск застопориться на какой-то тривиальной 
вещи и застрять надолго. В нормальной учебной
обстановке, такая проблема решается просто:
надо спросить другого студента или преподавателя.
Если их нет, надо походить, подумать, почитать
книжку, попробовать изучить контекст непонятной
вещи, подумать о том, как исторически возникли
такие штуки в математике. Проще всего выяснить
это (если знать английский), сделав поиск
на нужные ключевые слова в Интернете.
Но на случай, если Интернет не работает,
или если это слишком трудно, или просто
чтоб отдохнула голова --- можно посмотреть
лекции, сопутствующие этому листочку. Они адресованы
студенту, которому задач недостаточно, но
читать их можно и независимо.

Также в лекциях содержится английский
перевод ключевых слов и краткий список 
литературы, полезной для данного 
предмета.


%%%%%%%%%%%%%%%%%%%%%%%%%%%%%%%%%%%%%%%%%%%%%%%%
\section{Важное замечание}
%%%%%%%%%%%%%%%%%%%%%%%%%%%%%%%%%%%%%%%%%%%%%%%%


В чтении книг по математике есть две проблемы,
которые не возникают при очном обучении.
Первая проблема называется опечатки: большинство
научных книг содержат опечатки, и немало.
Студент или школьник, не готовый к этому,
может провести несколько месяцев в попытках
придать смысл заявлению, которое смысла не
имеет, потому что сделано по ошибке.
Дорогие читатели! Никогда не думайте,
что автор умнее вас, и видит что-то,
чего вы не видите. В большинстве случаев,
дурак не читатель, а автор, который нечто важное
исказил, пропустил и напортил.

Никакого вреда в этом нет: чтение книги
должно быть занятием творческим; в идеале --- 
совместным творчеством автора и читателя.
Для этого некоторые авторы специально добавляют
в свои книги опечатки, чтобы студентам было
о чем задуматься. 

Вторая проблема, тоже весьма неприятная --- люди
любят читать книги подряд. При этом дойдя до места,
где непонятно, люди читают это место снова и снова,
до полного отупения. Это неправильно! Надо открыть
книжку на другом месте и читать там, а непонятное
место перечитать потом.

%%%%%%%%%%%%%%%%%%%%%%%%%%%%%%%%%%%%%%%%%%%%%%%%%%%%%%%%%%%%%%%%%%%%%%%%

\part{Основания математики}
\renewcommand{\PartName}{Часть I. Основания математики}

%%%%%%%%%%%%%%%%%%%%%%%%%%%%%%%%%%%%%%%%%%%%%%%%%%%%%%%%%%%%%%%%%%%%%%%%


%%%%%%%%%%%%%%%%%%%%%%%%%%%%%%%%%%%%%%%%%%%%%%%%

\chapter{Основания математики}

%%%%%%%%%%%%%%%%%%%%%%%%%%%%%%%%%%%%%%%%%%%%%%%%

%%%%%%%%%%%%%%%%%%%%%%%%%%%%%%%%%%%%%%%%%%%%%%%%
\section{О математической строгости}
%%%%%%%%%%%%%%%%%%%%%%%%%%%%%%%%%%%%%%%%%%%%%%%%

"Точные науки" отличаются от "социо-гуманитарных"
тем, что в точных науках утверждение можно проверить
воспроизводимым экспериментом либо наблюдением.
Роль воспроизводимого наблюдения в математике 
играет {\em доказательство}.

Идея доказательства восходит к древним грекам,
и дошла до наших дней практически без изменений.
Математическая теория строится {\bf аксиоматически}.
В основе лежат несколько утверждений --- аксиом, принятых 
как нечто абсолютно верное; все же остальное выводится 
из аксиом посредством формальной логики. 

При таком подходе, доказательству подлежат зачастую 
вещи, интуитивно очевидные. Именно это и называется 
{\em математической строгостью}.

На протяжении истории стандарты математической
строгости менялись довольно часто. Отказ от строгости
в пользу {\em интуиции}, {\em мышления по аналогии} и 
{\em эвристических
аргументов} --- дело не всегда вредное.
Многие современные математики считают, что 
принятые в конце XX века стандарты математической
строгости (восходящие к Гильберту и к французской группе 
Бурбаки) излишни. Немало об отрицательном
влиянии Гильберта и Бурбаки говорил 
В. И. Арнольд.

Отчасти он прав --- большинство эвристических
аргументов можно довести до математически
строгих доказательств посредством рутинной
(и не всегда полезной) работы. Эту точку зрения
лучше всего высказали сами Бурбаки ("Теория Множеств"). 

\begin{quote}
{\it ...Математик, желающий убедиться в полной правильности,
или, как говорят, ``строгости", доказательства или теории,
отнюдь не прибегает к одной из тех полных формализаций,
которыми мы сейчас располагаем, и даже большей частью
не пользуется частичными и неполными формализациями,
доставляемыми алгебраическим и другими подобными
исчислениями. Обыкновенно он довольствуется тем, что
приводит изложение к такому состоянию, когда его опыт
и чутье математика  говорят ему, что перевод на
формализованный язык был бы теперь лишь упражнением в
терпении. Если возникают сомнения, то, в конечном счете
они относятся именно к возможности прийти без
двусмысленности к такой формализации ---  употреблялось ли
одно и то же слово в разных смыслах в зависимости от
контекста, нарушались ли правила синтаксиса
бессознательным употреблением способов рассуждения, не
разрешаемых явно этими правилами, была ли, наконец,
совершена фактическая ошибка. Текст редактируется, все
больше и больше приближаясь к формализованному тексту,
пока, по мнению специалистов, дальнейшее продолжение этой
работы не станет излишним.}
\end{quote}


Излишняя увлеченность формальными методами, 
возможно, действительно вредна, но
в обучении студентов без математической строгости не
обойтись. Профессиональный математик способен легко
определить, когда эвристический аргумент можно
{\ем формализовать}, то есть довести до любой требуемой степени
математической строгости; но эту способность можно
приобрести, только упражняясь в получении 
формально строгих доказательств.

%%%%%%%%%%%%%%%%%%%%%%%%%%%%%%%%%%%%%%%%%%%%%%%%
\section{О формальном методе}
%%%%%%%%%%%%%%%%%%%%%%%%%%%%%%%%%%%%%%%%%%%%%%%%

Формальный метод восходит к Гильберту,
который надеялся, что с его помощью удастся обосновать
математику. Его надежды не оправдались из-за теорем
Гёделя о неполноте. Но и сейчас формальный метод остается простейшим
(и лучше всего развитым) методом построения оснований
математики.

\begin{figure}[ht]
\begin{center}
\epsfig{file=Hilbert1912.eps,width=0.56\linewidth}\\
{David Hilbert\\
(1862 --- 1943)}
\end{center}
\end{figure}



Формальная версия математики устроена так. 
Математическая теория описывает свойства
определенных объектов, с помощью аксиом
и правил вывода. Сущность этих объектов
с формальной точки зрения неинтересна:
по замечанию Гильберта, {\ем "следует добиться 
того, чтобы вместо точек, прямых и плоскостей с
равным успехом можно было говорить о столах, 
стульях и пивных кружках"}. Правила вывода
суть формальные операции над утверждениями;
верным (доказанным) называется такое 
утверждение, которое можно вывести из аксиом. 

Формальный метод подразумевает, что никакой
связи между математическим миром и миром, окружающим
нас, нет вовсе. В качестве базовых понятий и аксиом
можно брать что угодно. Для того, чтоб этот метод 
обоснования математики считался действенным, необходимо 
(как минимум) доказать, что из использованного 
набора аксиом нельзя получить противоречия:
иначе в этой теории будет верно {\ем любое}
утверждение. Действительно, "импликация
с ложной посылкой истинна". Это свойство
системы аксиом называется {\бф непротиворечивость}.

Также нужно доказать, что любое утверждение
можно доказать, либо опровергнуть, исходя из аксиом.
Это свойство теории называется {\бф полнота}. 
В противном случае формальное описание математических
объектов неадекватно их сущности, хотя это не так
просто видеть.

Дело в том, что математическим объектам можно 
приписать реальность безотносительно к аксиомам,
которые ими описываются. Скажем, аксиомы арифметики
описывают теорию чисел, науку о решении диофантовых
уравнений (уравнений в целых числах). 
Утверждение {\ем "полиномиальное уравнение $P(t_1, t_2, ..., t_n)=0$ 
не имеет целочисленных решений $t_1, ..., t_n$"} 
может выводиться из аксиом арифметики (аксиом
Пеано), а может и не выводиться. Во втором случае,
может случиться, что уравнение имеет решения.
Может случиться и так, что
из аксиом арифметики невозможно  вывести ни
наличия, ни отсутствия решений.\footnote{Это
утверждение известно как ``Десятая проблема Гильберта``.
Оно было доказано  Юрием Матиясевичем в 1970-м году.}


Когда в какой-то теории есть утверждение $Q$,
которое нельзя вывести из аксиом, и при этом 
нельзя вывести из аксиом его отрицание "не $Q$" --- эта
система аксиом называется {\bf неполной}.

"Математической реальности" такая система
аксиом, очевидно, неадекватна. Действительно, предположим,
что, исходя из аксиом Пеано, нельзя ни доказать,
ни опровергнуть утверждение $Q$ "полиномиальное
уравнение $P(t_1, t_2, ... t_n)=0$ 
не имеет целочисленных решений $t_1, ... t_n$".
В этой ситуации уравнение $P(t_1, t_2, ... t_n)=0$ 
таки не имеет решений, ибо, если бы такое решение
было, мы бы могли его подставить в уравнение,
и получить теорему "$Q$ ложно".

В этой ситуации разговор о числах, апеллирующий
к утилитарному пониманию числа, гораздо содержательнее
разговора о формальных аксиомах и следствиях.
Действительно, формально получить $Q$
как следствие аксиом нельзя, ибо теория неполна;
но $Q$ тем не менее верно, что ясно из невозможности
его опровергнуть (у уравнения либо нет решений, 
либо они есть --- третьего не дано).

Гильберт надеялся, что система аксиом Пеано
(и шире --- система аксиом теории множеств, лежавшей
в основе математики того времени) полна и непротиворечива.
Доказательство этого фактически доказало бы эквивалентность
формального метода и утилитарного (основанного
на естественно-научной интуиции) представления 
о числах и о математике.

Этого не случилось.

К концу 1920-х годов, формальная программа
Гильберта близилась к завершению. В 1930-м
году польский математик Альфред Тарский развил систему
аксиом для элементарной геометрии, более формальную
и строгую, чем у Гильберта, и доказал, что эта система
аксиом полна и непротиворечива.

Но в самом начале 1930-х, совершенно неожиданно,
Курт Гёдель доказал две теоремы о неполноте,
которые не оставили камня на камне от программы
Гильберта. 

Гёдель доказал, что ни одна достаточно
богатая (например, содержащая среди своих 
аксиом аксиомы Пеано) формальная теория не может быть полна; также
он доказал, что доказательство ее непротиворечивости
получить {\em невозможно.}

 \begin{figure}[ht]
\begin{center}
\epsfig{file=main_godel.eps,width=0.86\linewidth}\\
{Kurt G\"odel\\
(1906 --- 1978)}
\end{center}
\end{figure}


Формальный метод получил поражение, а 
аксиоматическое построение математики
было значительно дискредитировано.


%%%%%%%%%%%%%%%%%%%%%%%%%%%%%%%%%%%%%%%%%%%%%%%%
\section{Теория множеств и ее аксиоматизация}
%%%%%%%%%%%%%%%%%%%%%%%%%%%%%%%%%%%%%%%%%%%%%%%%


Дополнительную остроту кризису придавали 
{\em парадоксы теории множеств.}
Дело в том, что, в представлении Гильберта
и современных ему математиков, естественным
фундаментом для математики могла быть только
теория множеств, разработанная Кантором в конце
XIX века. Но в начале XX века в теории множеств
обнаружились неустранимые противоречия. 

"Наивная теория множеств" имеет дело с 
{\em множествами} --- любыми совокупностями
объектов, которые называются "элементами множества". 
Утверждение "$x$ является элементом $X$" записывается
$x\in X$. Отрицание его записывается
$x\notin X$.

Можно говорить о "множестве всех последовательностей
элементов данного множества",
"множестве всех букв алфавита", "множестве
всех слов из букв данного алфавита", и проделывать над такими множествами
естественные операции (пересечения, объединения
и т. п.). {\bf Подмножеством} множества
$X$ называется любое множество $X'$ такое,
что все элементы $X'$ являются элементами
$X$. Тот факт, что $X'$ является подмножеством $X$,
записывается $X'\subset X$. 

``Пустое множество" (обозначается $\emptyset$) 
не имеет элементов вовсе. 

Кантор не добивался абсолютной строгости
в теории множеств. Первая попытка построить
теорию множеств строго и аксиоматически
принадлежала Готтлобу Фреге, который 
предполагал, что получится логически вывести всю математику
из самоочевидных постулатов теории множеств (этот подход
к основаниям математики называется "логицизмом"). 
Построенную Фреге теорию называют "наивной теорией 
множеств". Она неверна (содержит противоречия).

Самое простое из противоречий наивной
теории множеств было обнаружено
Бертраном Расселом в 1901 году. 
Рассмотрим множество $A$ всех множеств
$X$ таких, что $X$ не является элементом $X$.
Будет ли $A$ принадлежать $A$? 
Если $A$ не принадлежит $A$, то $A$ должно
быть своим элементом. То есть формальным следствием
$A\notin A$ является $A\in A$. 

Этот парадокс --- форма хорошо известного
"парадокса цирюльника". В одном селе живет цирюльник X.,
который бреет всех жителей, кроме тех, которые бреют себя сами.
Бреет ли цирюльник X. сам себя? 

Со времен Рассела получено множество парадоксов
наивной теории множеств. Они все сводятся, грубо говоря,
к тому, что строится "слишком большое" множество,
которое и приводит к парадоксам.

Рассел и Уайтхед построили версию теории множеств, свободную
от известных парадоксов, но она оказалась неудобна.
Более удобная версия
аксиоматической теории множеств была разработана
Э. Цермело в 1908. В 1922 А. Френкель
дополнил эту систему аксиом; современная версия
аксиоматической теории множеств называется
{\ем система аксиом Цермело-Френкеля} (ZF).
Отличие этой версии теории множеств от
наивной было в том, что "излишне больших"
множеств Цермело-Френкель не допускали.
Их теория оперировала не "всеми" множествами,
а только теми, чье существование можно
доказать (вывести из аксиом). Такие множества
конструируются из других множеств посредством
набора четко определенных операций. И парадоксальные
объекты, такие, как "множество всех множеств",
в системе аксиом Цермело-Френкеля просто
{\em не существуют}.

Довольно долго
(вплоть до Гёделя) математики надеялись, 
что теория множеств Цермело-Френкеля полна и непротиворечива.
Сейчас ясно, что она неполна, и, возможно,
противоречива (во всяком случае, непротиворечивость
этой системы аксиом доказать невозможно, и это факт). 
Тем не менее, в большинстве версий "оснований математики"
математика базируется на теории множеств.

В обучении математике, нам приходится поступать 
так же. Отчасти это связано с тем, что альтернативные
подходы (конструктивная математика, теория
категорий и другие) труднее и менее известны.
А отчасти с тем, что базовые понятия 
теории множеств (отображения, произведение множеств,
подмножества, биекции, классы эквивалентности) необходимы
математику в любом случае.

Бурбаки комментируют возможность противоречий в основаниях
математики таким образом ("Теория множеств"). 

\begin{quote}
{\it За 40 лет с тех пор, как сформулировали с достаточной
точностью аксиомы Теории множеств и стали извлекать из них
следствия в самых разнообразных областях математики, еще
ни разу не встретилось противоречие, и можно с основанием
надеяться, что оно и не появится никогда.

Если бы дело и сложилось иначе, то, конечно, замеченное
противоречие было бы внутренне присуще самим принципам,
положенным в основание Теории множеств, а потому нужно
было бы видоизменить эти принципы, стараясь по возможности
не ставить под угрозу те части математики, которыми более
других дорожат. И ясно, достичь этого тем более легко, что
применение аксиоматического метода и формализованного
языка позволит формулировать эти принципы более четко и
отделять от них следствия более определенно. Впрочем,
приблизительно это и произошло недавно, когда устранили
парадоксы Теории множеств принятием формализованного
языка. Подобную ревизию следует предпринять и в случае,
когда этот язык окажется в свою очередь противоречивым.}
\end{quote}


%%%%%%%%%%%%%%%%%%%%%%%%%%%%%%%%%%%%%%%%%%%%%%%%
\section{Терминология и библиография}
%%%%%%%%%%%%%%%%%%%%%%%%%%%%%%%%%%%%%%%%%%%%%%%%

По-английски, теория множеств называется 
set theory, Цер\-ме\-ло-\-Френ\-кель --- Zermelo-Fraenkel.
В английской версии Википедии основания
математики изложены весьма подробно, особенно
экзотические и альтернативные версии.
Популярное изложение формального метода и
его истории есть в "Теории множеств" Бурбаки.
Полезный учебник по теории множеств --- 
"Теория множеств" Куратовского и Мостовского.
Биография Гильберта --- "Гильберт",
Констанс Рид. 

Натурфилософские аспекты
теорем Гёделя подробно обсуждаются
в книгах Р. Пенроуза "Новый разум 
императора" ("The Emperor's New Mind") 
и "Тени разума" ("The Shadows 
of the Mind").

%%%%%%%%%%%%%%%%%%%%%%%%%%%%%%%%%%%%%%%%%%%%%%%%

\chapter[Основные понятия теории 
множеств]{Основные понятия теории множеств}

%%%%%%%%%%%%%%%%%%%%%%%%%%%%%%%%%%%%%%%%%%%%%%%%

%%%%%%%%%%%%%%%%%%%%%%%%%%%%%%%%%%%%%%%%%%%%%%%%
\section{Обозначения теории множеств}
%%%%%%%%%%%%%%%%%%%%%%%%%%%%%%%%%%%%%%%%%%%%%%%%

Г. Кантор определял множество так:
"Множество есть многое, мыслимое нами как единое".
Говоря чуть более строго, "Множество -- это
единое имя для совокупности всех объектов, 
обладающих данным свойством" (тоже Кантор).

В наивной теории множеств,
множество есть "любая совокупность объектов, называемых
элементами множества". Во избежание противоречий,
которыми славится наивная теория множеств, это
определение придется несколько изменить, отбросив
"излишне большие" классы множеств.
Множества {\bf равны}, если они составлены из одинаковых элементов.


Множество $S$ следует понимать как своего рода черный ящик --- 
для каждого объекта $x$, этот черный ящик умеет отвечать
на вопрос: "принадлежит ли $x$ множеству $S$"? 
Больше ничего $S$ делать не умеет. Два множества
{\бф равны}, или же {\бф совпадают},
если они отвечают на этот вопрос одинаково.

Если $x$ принадлежит $S$, это обозначается $x\in S$,
или $S\ni x$, если не принадлежит, это обозначается $x\notin S$.
Это обозначение (как и большинство других обозначений
для логических операторов: $\exists$, $\cap$,
$\cup$ и так далее) придумал Джузеппе Пеано в конце XIX
века. В этих обозначениях, равенство множеств можно записать
так:
\[
S_1 = S_2 \Leftrightarrow 
\left (\forall x \ \ | \ \ x\in S_1 \Leftrightarrow x\in S_2 \right)
\]
"Множества $S_1$ и $S_2$ равны тогда и только тогда, когда для
любого $x$, $x\in S_1$ равносильно $x\in S_2$".
Значок $\forall$ обозначает "для каждого",
а $\Leftrightarrow$ --- эквивалентность 
утверждений ("тогда и только тогда",
"равносильно"). 

{\бф Подмножеством} множества $S$ называется 
множество $S_1$, целиком содержащееся в $S$.
Это обозначается $S_1\subset S$ (обозначение введено
Эрнстом Шредером в 1890). Используя логические обозначения,
это можно записать:
\[
S_1 \subset S \Leftrightarrow 
\left (\forall x \ \ | \ \ x\in S_1 \Rightarrow x\in S \right)
\]
В этой формуле
стрелка $\Rightarrow$ обозначает импликацию ("следовательно").
Иногда это же самое записывают $S\supset S_1$.
Если $S_2\subset S_1$ и $S_1\subset S_2$, множества
$S_1$ и $S_2$ равны. 

Подмножество $S_1\subset S$, не совпадающее с $S$,
называется {\бф собственным подмножеством}.
Это обозначается $S_1\subsetneq S$.

Пустое множество обозначают $\emptyset$;
это обозначение впервые появилось у Бурбаки, и 
принадлежит Андре Вейлю.

Множество всех подмножеств множества $S$ обозначается $2^S$.



\задача
Пусть $S$ --- конечное множество из $s$ элементов.
Докажите, что $2^S$ --- конечное множество из $2^s$
элементов
\ез


Если $S_1$ и $S_2$ --- два множества, можно говорить
об их {\бф объединении} $S_1\cup S_2$ (множестве всех элементов,
принадлежащих $S_1$ или $S_2$) и {\бф пересечении}
$S_1\cap S_2$
(множестве всех элементов,
принадлежащих $S_1$ и $S_2$).
Формально объединение определяется так:
\begin{equation}\label{_cup_formally_def_Equation_}
S_1\cup S_2= \{ x \ \  | \ \ (x\in S_1) \vee (x\in S_2)\}
\end{equation}
В этой формуле $\{ x \ \  | \ \  P(x) \}$
обозначает "множество всех $x$, удовлетворяющих $P(x)$''
(это обозначение принадлежит Кантору). 
Символы $\vee$ и $\wedge$ означают "или" и "и"
(использовать $\vee$ для обозначения дизъюнкции
первым стал Б. Рассел; вместо "и" Рассел 
использовал точку).


\задача
Запишите $S_1\cap S_2$ формулой, аналогичной 
\eqref{_cup_formally_def_Equation_}.
\ез

Если заданы множества $S_1$ и $S_2$, можно определить 
{\бф произведение} $S_1\times S_2$ --- множество всех
пар $(x, y)$, где $x\in S_1$, $y\in S_2$.

\hfill

{\бф Дополнение} множества $A$ до подмножества $B\subset A$ --- 
множество всех $a\in A$, которые не лежат в $B$.
Дополнение обозначается $A\backslash B$.

%%%%%%%%%%%%%%%%%%%%%%%%%%%%%%%%%%%%%%%%%%%%%%%%
\section{Соответствия и отображения}
\label{_Sootve_Otobra_Subsection_}
%%%%%%%%%%%%%%%%%%%%%%%%%%%%%%%%%%%%%%%%%%%%%%%%

Пусть $S_1$, $S_2$ --- множества. {\бф Соответствием}\footnote{Соответствие
иногда называют {\бф соотношением}, или {\бф бинарным отношением}. Это синонимы.}
называется любое подмножество $P\subset S_1 \times S_2$.
Если пара $(x, y)$ принадлежит $P$, то говорят 
"соответствие $P$ выполнено для пары $(x,y)$".
Это обозначается $P(x,y)$.

Соответствие обыкновенно задается 
какой-либо формулой: например, если 
$S_1=S_2=\R$ --- это множество вещественных чисел,
формулы $x< y$, $x> y$, $x\neq y$ 
задают соответствия. 

{\бф Отображением}, или {\бф функцией}
$f:\; S_1 \arrow S_2$ называется такое 
соответствие $\Gamma_f\subset S_1 \times S_2$, 
что для каждого $x\in S_1$ существует единственный
элемент $y\in S_2$ такой, что $(x, y) \in \Gamma_f$.
Множество $\Gamma_f$ в этой ситуации называется
{\бф графиком функции $f$}. {\бф Значением} функции
$f$ на элементе $x\in S_1$ называется
тот единственный $y\in S_2$, для
которого $(x, y) \in \Gamma_f$.
Это обозначается так: $y = f(x)$.

Функцию часто задают формулами: если 
$S_1$ и $S_2$ это множества вещественных чисел,
то формулы $x\arrow x^2$, $x\arrow |x|$,
$x\arrow x+2$ задают отображения.

Следует думать про функцию как про набор
правил, ставящих в соответствие каждому 
$x\in S_1$ некоторый элемент $S_2$.

Отображение из множества в себя называется
{\бф преобразованием} множества. {\бф Тождественное
преобразование} множества $S$ переводит
каждый $x\in S$ в себя. Его график называется
{\бф диагональю}. Тождественное преобразование
$S$ обозначается $\Id_S:\; S \arrow S$.

Если $f:\; S_1 \arrow S_2$, $g:\; S_2 \arrow S_3$
две функции, можно определить их {\бф композицию}
$f\circ g$, ставящую в соответствие $x \in S_1$
$g(f(x))\in S_3$. 

\задача
Пусть $f:\; S_1 \arrow S_2$, $g:\; S_2 \arrow S_3$,
$h:\; S_3 \arrow S_4$ --- три функции. Докажите,
что композиция удовлетворяет {\бф свойству ассоциативности}
\[ f\circ (g\circ h) = (f\circ g)\circ h
\]
\ез

\замечание
Иногда значок $\circ$ обозначает
композицию отображений, примененную в другом порядке:
не $g(f(x))$, а $f(g(x))$.
\еза

Если $f(x)=y$, то $y$ называется {\бф образом}
$x$, а $x$ --- {\бф прообразом} $y$. 

{\бф Образом $S_1$} (обозначается $f(S_1)$)
при отображении $f:\; S_1 \arrow S_2$ 
называется множество всех $y\in S_2$,
которые являются образами каких-то $x\in S_1$.

{\bf Прообраз} подмножества $R_2\subset S_2$ --- 
множество всех элементов $x\in S_1$, переходящих
в элементы $R_2$; {\бф образ подмножества} $R_1\subset S_1$ --- 
совокупность всех элементов, полученных как образы
элементов $R_1$. Образ подмножества обозначается
$f(R_1)$, прообраз --- $f^{-1}(R_2)$.

{\бф Ограничением} функции $f:\; S_1 \arrow S_2$ 
на подмножество $R_1\subset S_1$ называется
функция $f\restrict {R_1}:\; R_1 \arrow S_2$, которая
на каждом $x\in R_1$ принимает значение
$f(x)$.

Функция $f:\; S_1 \arrow S_2$ 
называется {\бф вложением}, или {\бф инъекцией},
или {\бф инъективным отображением}, если для 
разных $x, x' \in S_1$, их образы
не равны: $f(x) \neq f(x')$. Инъекцию часто
обозначают такой стрелочкой: $\hookrightarrow$.

Функция $f:\; S_1 \arrow S_2$ 
называется {\бф наложением}, или {\бф сюръекцией},
или {\бф сюръективным отображением}, если $f(S_1)= S_2$;
иначе говоря, каждый $y\in S_2$ является
образом какого-то элемента $S_1$.

Если функция $f:\; S_1 \arrow S_2$ --- одновременно
наложение и вложение, то она называется
{\бф биекцией}, или {\бф биективным отображением},
или {\бф взаимно-однозначным отображением},
или {\бф обратимой}. 

\задача 
Пусть $f:\; S_1 \arrow S_2$ --- биекция.
Докажите, что существует отображение
$g:\; S_2 \arrow S_1$ такое, что композиция
$f\circ g$ --- тождественное преобразование $S_1$,
а $g\circ f$ --- тождественное преобразование $S_2$.
\ез


Определение функций через подмножества произведения
принято в версии оснований математики, которая
базируется на теории множеств. В других версиях
оснований математики основным является понятие
функции, заданной формулой либо каким-то алгоритмом,
а множества определяются в терминах
функций. Даже в аксиоматической теории множеств
приведенное выше определение функции используется
наряду с формально-логическим "функция, заданная
формулой на языке Цермело-Френкеля" (высказывательная 
функция). Получить из "высказывательной функции"
функцию в смысле вышеприведенного определения
можно, воспользовавшись одной из аксиом 
("схемой подстановки").


%%%%%%%%%%%%%%%%%%%%%%%%%%%%%%%%%%%%%%%%%%%%%%%%
\section{Отношения эквивалентности}
%%%%%%%%%%%%%%%%%%%%%%%%%%%%%%%%%%%%%%%%%%%%%%%%

\определение
Пусть на множестве $S$ задано соотношение\footnote{Соотношение 
это то же самое, что и соответствие, то есть подмножество
в $S\times S$.} $\sim$,
удовлетворяющее следующим условиям.

\begin{description}
\item[Рефлексивность] Для любого $x\in S$, выполняется $x\sim x$.
\item[Симметричность] Для любых $x\in S, y\in S$, 
\[ 
 (x\sim y) \Leftrightarrow (y\sim x)
\]
\item[Транзитивность] Для любых $x, y, z\in S$, 
\[ 
 \bigg((x\sim y) \text{\ \ и \ \ } (y\sim z)\bigg)\Rightarrow (x\sim z).
\]
\end{description}

Такое соотношение называется {\бф отношением
эквивалентности}.
\ео 

\задача
Придумайте примеры соотношений, которые
\итем являются рефлексивными и симметричными, но не
транзитивными
\итем являются рефлексивными и транзитивными, но не
симметричными
\итем являются транзитивными и симметричными,
но не рефлексивными
\ез

\определение
Пусть $X, \sim$ --- множество, снабженное отношением
эквивалентности. {\бф Классом эквивалентности}
элемента $x\in X$ называется множество всех $y\in X$
таких, что $x\sim y$. {\бф Представителем класса} $S$
называется любой элемент $x\in X$, который лежит в этом классе.
\ео

\задача
Докажите, что в такой ситуации $X$ разбито в объединение
непересекающихся классов эквивалентности.
\ез

В силу этой задачи, на $X$ задано сюръективное
отображение $\pi:\; X \arrow Y$ в множество $Y$ классов эквивалентности,
которое переводит элемент в его класс эквивалентности.
Легко видеть, что любое сюръективное отображение
$\pi:\; X \arrow Y$ получается таким образом.




%%%%%%%%%%%%%%%%%%%%%%%%%%%%%%%%%%%%%%%%%%%%%%%%
\section{Аксиоматическая теория множеств}
%%%%%%%%%%%%%%%%%%%%%%%%%%%%%%%%%%%%%%%%%%%%%%%%

Наивная теория множеств приводит к парадоксам.
Источником этих парадоксов принято считать
несуществование "слишком больших множеств".
Аксиоматическая теория множеств имеет дело не
со "всеми" множествами, а только с теми, чье
существование может быть получено из аксиом.

Аксиоматическая  теория множеств требует продвинутого
логического аппарата. Большинство математиков
не помнят этих аксиом и ими не пользуются.
Вместо этого используется принцип "применением
естественных операций над множествами  получается множество".
Под "естественными операциями" понимаются
операции, описанные выше:
взятие пересечения, объединения, произведения,
подмножества и множества всех подмножеств. 

В канонической форме, система аксиом Цермело-Френкеля
(ZFC) состоит из 9 аксиом.\footnote{Довольно часто аксиомы 1-8
рассматриваются отдельно от аксиомы 9 (аксиомы выбора). Система
аксиом Цермело-Френкеля без аксиомы выбора обозначается аббревиатурой
ZF, с аксиомой выбора - ZFC, Zermelo-Fraenkel with Axiom of Choice.}

\begin{description}
\item[1. Аксиома объемности (экстенсиональности).]
Два множества $A$ и $B$ равны тогда и 
только тогда, когда они имеют одни и те же элементы:
\[
(\forall x (x\in A \Leftrightarrow x\in B)) \Rightarrow A=B
\]

\item[2. Аксиома пары.] Для любых множеств $А$ и $B$ существует
множество $C$ такое, что $A$ и $B$ являются его единственными
элементами. Множество $C$ обозначается $\{A,B\}$ и называется
{\bf неупорядоченной парой} $A$ и $B$. Если $A=B$, то $C$ состоит из
одного элемента.

\item[3. Аксиома выделения.] 
Пусть $\phi(p)$ --- свойство, которым может обладать
множество $p$. Для каждого множества $X$ существует
подмножество $X_\phi\subset X$, составленное из всех
элементов $x\in X$, удовлетворяющих свойству $\phi$.

\item[4. Аксиома объединения.] 
 Для любых множеств $A$ и $B$ существует их объединение $A\cup B$:
\[ 
A\cup B:= \{ x \ \  | \ \  (x\in A)\vee (x\in B) \}
\]

\end{description}

\замечание
Отметим, что пересечение $A\cap B$ существует в силу
аксиомы выделения.
\еза


\begin{description}
\item[5. Аксиома степени.] 
Для любого множества $S$ существует множество всех его подмножеств
$2^S$.

\end{description}

\замечание \label{_indukti_mno_Zamechanie_}
Конечные множества в аксиоматической теории
множеств строятся следующим образом. 
Множество из нуля элементов --- $\emptyset$.
Множество из одного элемента --- это $\{\emptyset\}$,
множество из двух элементов --- это
\[
\{ \{\emptyset\}, \emptyset\},
\]
множество из трех элементов --- 
\[
\{ \{ \{\emptyset\},\emptyset\}, \{\emptyset\}, \emptyset\},
\]
и так далее.  Аналогичным образом
можно построить и бесконечное множество.
Множество $S$ называется {\бф индуктивным},
если оно содержит $\emptyset$, и для любого
элемента $x\in S$, $S$ содержит $x\cup \{x\}$.
\еза
\begin{description}
\item[6. Аксиома бесконечного множества.]
\  \\ Существует индуктивное множество.
\end{description}

\определение
Пусть на языке теории множеств
Цермело-Френкеля задано свойство $\Phi(x, y)$, которым
может обладать упорядоченная пара множеств $x, y$. Предположим,
что для каждого $x$ есть не больше одного $y$ такого,
что $\Phi(x,y)$ выполняется. В такой ситуации,
$\Phi$ задает "функцию" из класса множеств $x$, 
для которых $y$ существует,
переводящую $x$ в этот (единственным
образом определенный) $y$.
Такая "функция" называется {\бф высказывательной
функцией}, а совокупность множеств, где она определена --- 
ее {\бф областью определения}. 

Отметим, что "высказывательные функции"
не являются функциями в обычном смысле этого слова
(см. раздел \ref{_Sootve_Otobra_Subsection_}).

Область определения
 высказывательной функции --- не обязательно множество.
Скажем, область определения тождественной
высказывательной функции
\[
\Phi(x,y) \Leftrightarrow (x=y)
\]
это все множества, но совокупность всех множеств
не образует множества (см. Замечание 
I.\ref{_set_of_sets_Zamechanie_}).
\ео

\begin{description}
\item[7. Схема подстановки для высказывательной функции $\Phi$.]
Пусть $\Phi(x,y)$ --- высказывательная функция, а $X$ --- 
множество, все элементы которого лежат в области
определения $\Phi$. Тогда существует множество
$Y$, составленное из всех $y$, для которых
существует $x\in X$ такое, что верно $\Phi(x,y)$.
\end{description}

\замечание
"Схема подстановки" --- не аксиома, а "правило вывода",
которое по каждой высказывательной функции строит
свою аксиому. Иначе говоря, "аксиома 7" --- это
 счетный набор аксиом, каждая заданная своей
высказывательной функцией.
\еза

\begin{description}
\item[8. Аксиома регулярности.]
Любое непустое множество $S$ содержит элемент $x\in S$
такой, что $x\cap S=\emptyset$.

\item[9. Аксиома выбора.]
Пусть $\phi:\; X \arrow Y$ сюръективное отображение
множеств. Тогда у $\phi $ есть {\бф сечение}
$\psi:\; Y\arrow X$, то есть отображение
$\psi:\; Y\arrow X$, удовлетворяющее $\psi\circ \phi =
\Id_Y$. Иначе говоря, $\psi$ ставит каждому элементу
$Y$ в соответствие некий элемент из прообраза $\phi^{-1}(y)$.
\end{description}

Аксиома выбора влечет ряд парадоксальных следствий.
Например, из нее следует, что единичный шар в $\R^3$
можно разбить на 5 конгруэнтных (одинаковых) 
частей, а затем сложить
из них два шара такого же размера
(парадокс Банаха-Тарского). Аксиома выбора
независима от остальных аксиом Цермело-Френкеля:
ее нельзя ни доказать, ни опровергнуть
в этой системе аксиом (это доказал Пол Коэн,
используя {\бф метод форсинга}).
Если система Цермело-Френкеля
непротиворечива, то она непротиворечива
в предположении, что аксиома выбора верна,
как и в предположении, что аксиома выбора неверна
(это доказал Гёдель).
Большинство альтернативных версий оснований математики
(конструктивизм, интуиционизм, ультрафинитизм) не
признает аксиому выбора и ее следствий. В большинстве
разделов математики без аксиомы выбора можно обойтись,
используя вместо аксиомы выбора какую-то ослабленную 
версию (см. I.\ref{_AC_alterna_Section_}), не влекущую парадоксов.
Поэтому многие ученые стараются избежать
аксиомы выбора и ее следствий, или, по крайней мере,
отмечают все случаи ее употребления. 

Доказательство, использующее аксиому выбора,
{\ем неконструктивно}, то есть использует 
математические объекты, которые невозможно
задать явно. Многие математики полагают,
что теорема существования верна только
если объект построен явно.

Аксиомы Цермело-Френкеля приводятся 
в этой главе для ознакомления.
Для большинства разделов математики наивной теории множеств
вполне достаточно, хотя необходимо помнить 
о возможности парадоксов и избегать их. 


%%%%%%%%%%%%%%%%%%%%%%%%%%%%%%%%%%%%%%%%%%%%%%%%
\section{Терминология и библиография}
%%%%%%%%%%%%%%%%%%%%%%%%%%%%%%%%%%%%%%%%%%%%%%%%


По-английски, отображение называется map, mapping, function,
образ и прообраз --- image, preimage.
Соотношение --- relation, соотношение эквивалентности --- 
equivalence relation. Биекция, инъекция и сюръекция --- bijection,
injection, surjection (эти слова изобретены группой Бурбаки).
Произведение множеств --- Cartesian product (в честь Декарта,
имя которого в латинских текстах писали Cartesius).
Аксиома выбора называется Axiom of choice (сокращается до AC). 
Система Цермело-Френкеля (ZFC) подробно освещается
в англоязычной Википедии. 

Большая коллекция
ссылок на страницы, посвященные аксиоме выбора, лежит тут\\
http://www.math.vanderbilt.edu/$\tilde\ $schectex/ccc/choice.html\\

Книга И. В. Ященко "Парадоксы теории множеств" -\\
http://www.mccme.ru/mmmf-lectures/books/books/books.php?book=20


%%%%%%%%%%%%%%%%%%%%%%%%%%%%%%%%%%%%%%%%%%%%%%%%

\chapter{Кардиналы и теорема Кантора}

%%%%%%%%%%%%%%%%%%%%%%%%%%%%%%%%%%%%%%%%%%%%%%%%

%%%%%%%%%%%%%%%%%%%%%%%%%%%%%%%%%%%%%%%%%%%%%%%%
\section{Теорема Кантора-Бернштейна-Шредера}
%%%%%%%%%%%%%%%%%%%%%%%%%%%%%%%%%%%%%%%%%%%%%%%%

Два множества $A$ и $B$ называются {\бф равномощными},
если между $A$ и $B$ существует взаимно-однозначное
соответствие. {\бф Счетное множество} есть множество,
равномощное множеству натуральных чисел.

Если множество $A$ равномощно подмножеству $B$,
говорится, что {\бф мощность $A$ не больше мощности $B$}.

\hfill

%%%%%%%%%%%%%%%%%%%%%%%%%%%%%%%%%%%%%%%%%%%%%%%%
{\бф Теорема Кантора-Бернштейна-Шредера.}
Пусть $A$ и $B$ такие множества, что $A$ равномощно
подмножеству $B$, а $B$ равномощно подмножеству $A$.
Тогда $A$ и $B$ равномощны.

\hfill


\begin{figure}[ht]
\begin{center}\ \\
\epsfig{file=Cantor-Bernstein.eps,width=0.5\linewidth}\\
{\small \em Разбиение $A$ и $B$ по числу прообразов}
\end{center}
\end{figure}


{\bf Доказательство:} 
Пусть $f:\; A \hookrightarrow B$ вложение
из $A$ в $B$, а $g :\; B \hookrightarrow A$
вложение из $B$ в $A$. Рассмотрим 
отображения $f\circ g:\; A \arrow A$
и $g\circ f:\; B \arrow B$. Обозначим
за $A_0$ дополнение $A\backslash g(B)$.
Определим индуктивно $A_i$
по формуле
$A_{i+1}= f\circ g(A_i)$.
Определим $A_\infty$ как пересечение
образов отображений $(f\circ g)^i$ для всех $i$
(за $(f\circ g)^i$ обозначается композиция $f\circ g$ с
собой $i$ раз). Легко видеть, что $A$ разбивается
в объединение непересекающихся подмножеств,
$A= A_0 \cup A_1 \cup ... \cup A_\infty$.

Применим аналогичную процедуру к $B$, получив
разбиение $B$ в объединение непересекающихся подмножеств,
$B= B_0 \cup B_1 \cup ... \cup B_\infty$.

Легко видеть, что $f$ задает биекцию из $A_\infty$ в
$B_\infty$. Действительно, каждый элемент из $A_\infty$
является образом элемента из $B_\infty$, и наоборот.
Также $f$ задает биективное отображение
из $A_0 \cup A_2 \cup A_4 \cup ...$ в $B_1 \cup B_3 \cup B_5 \cup ...$,
а $g$ задает биективное отображение из $B_0 \cup B_2 \cup B_4 \cup ...$ 
в $A_1 \cup A_3 \cup A_5 \cup ...$ (см. иллюстрацию). 


Мы разбили $A$ и $B$ на непересекающиеся 
подмножества, которые попарно биективны.
\endproof

\замечание
В математике значком \endproof \ \ или похожим
значком обозначается конец доказательства.
Этот знак называется "халмош" или "tombstone".
Его изобрел Пол Халмош, известный американский математик.
\еза

%%%%%%%%%%%%%%%%%%%%%%%%%%%%%%%%%%%%%%%%%%%%%%%%
\section{Мощность множества}
%%%%%%%%%%%%%%%%%%%%%%%%%%%%%%%%%%%%%%%%%%%%%%%%

\определение
Пусть на множестве $S$ задано соотношение $\succeq$.
Это соотношение называется {\бф отношением 
частичного порядка}, если выполнены следующие
аксиомы
\begin{description}
\item[(i)] (Рефлексивность) Для любого $a$, имеет место
$a\succeq a$
\item[(ii)] (Транзитивность) $a \succeq b$ и $b\succeq c$ влечет
$a \succeq c$
\item[(iii)] (Асимметричность) Если $a\succeq b$ и $b
\succeq a$, то $a=b$.
\end{description}
Если, в дополнение к тому, для любых двух $a, b$ имеет место
$a\succeq b$ либо $b
\succeq a$, то $\succeq$ называется {\бф отношением
линейного порядка}.
\ео

В качестве примера "отношения частичного порядка" 
рассмотрим соотношение "быть подмножеством" на множестве
всех подмножеств $X$. Ясно, что $A_1\subset A_2$ --- 
соотношение частичного порядка. Отношение
линейного порядка имеется на множестве
вещественных чисел ("меньше": $a\leq b$).

\hfill

\определение
Если множество $A$ допускает вложение в $B$,
говорится, что {\бф мощность $A$ меньше или равна
мощности $B$}. 
\ео

Легко видеть, что равномощность --- это отношение
эквивалентности. {\бф Кардинал}, или
{\бф кардинальное число} множества $A$ это
его мощность. На языке наивной теории множеств,
отношение равномощности разбивает все множества
на классы эквивалентности, пронумерованные 
{\бф кардиналами}. Иначе говоря, кардинал
$X$ --- это класс эквивалентности множеств, 
равномощных $X$.

С точки зрения аксиоматической теории
множеств, "класс эквивалентности множеств, 
равномощных $X$" множеством не является,
и говорить про него нельзя. Тем не менее,
"множество кардиналов, меньших данного"
определить можно. Это делается так.
Возьмем множество $X$. Рассмотрим
множество $2^X$ всех подмножеств $X$.
Равномощность задает на $2^X$
соотношение эквивалентности.
Множество классов эквивалентности
называется {\бф множеством кардиналов
меньших или равных $X$}. В дальнейшем
мы будем называть это множество 
"множеством кардиналов", предполагая,
что $X$ --- чрезвычайно большое множество,
мощность которого может быть при необходимости
увеличена настолько, насколько нужно.

\задача Выведите
из теоремы Кантора-Бернштейна-Шредера,
что "мощность $A$ меньше или равна
мощности $B$" задает на множестве
кардиналов отношение частичного порядка.
\ез

\замечание Кардиналы конечных
множеств называются {\бф конечными кардиналами}.
Легко видеть, что конечные кардиналы
взаимно однозначно соответствуют натуральным
числам. 
\еза

Используя аксиому выбора, легко доказать,
что множество кардиналов на самом деле
{\бф упорядочено}: для любых
множеств $X$ и $Y$, либо
$X$ вкладывается в $Y$, либо $Y$
вкладывается в $X$. 


%%%%%%%%%%%%%%%%%%%%%%%%%%%%%%%%%%%%%%%%%%%%%%%%
\section{Счетные множества}
%%%%%%%%%%%%%%%%%%%%%%%%%%%%%%%%%%%%%%%%%%%%%%%%

\определение
Множество называется {\бф счетным}, если
оно допускает взаимно однозначное соответствие
с множеством натуральных чисел.
\ео


\задача
Докажите, что следующие множества счетны
\итем Множество $\Z$ целых чисел
\итем Множество $\Q$ рациональных чисел
\итем Множество $\Z[t]$ полиномов с целыми коэффициентами.
\ез

Естественные операции с конечными множествами
(взятие произведения, взятие объединения
непересекающихся множеств и так далее)
как правило увеличивают мощность множества.
С бесконечными (например, счетными) множествами
все совершенно иначе --- естественные операции
не меняют их мощности.


\задача
Пусть $X$ --- счетное множество
Докажите, что $X\times X$ счетное. 
\ез

\задача
Пусть $X$, $Y$ --- непересекающиеся счетные
множества. Докажите, что $X\cup Y$ счетно.
\ез

Из аксиомы выбора легко вывести, 
что любое бесконечное множество содержит
счетное множество (докажите это). 
Из такого множества можно выкинуть
конечное подмножество, не меняя мощности.

\задача
Пусть $X$ --- бесконечное множество, которое
содержит счетное подмножество.
Рассмотрим конечное подмножество $Z\subset X$.
Докажите, что дополнение $X\backslash Z$
равномощно $X$. 
\ез



%%%%%%%%%%%%%%%%%%%%%%%%%%%%%%%%%%%%%%%%%%%%%%%%
\section{Диагональный метод Кантора}
%%%%%%%%%%%%%%%%%%%%%%%%%%%%%%%%%%%%%%%%%%%%%%%%

Если дано множество $X$, из него можно образовать
много других множеств, посредством естественных
операций. Примеры: взятие декартова квадрата
$X\arrow X\times X$, взятие двух копий множества:
$X \arrow X \times \{1,2\}$, умножение на натуральные
числа $X \arrow X \times {\Bbb N}$ и т.д.
Можно доказать, что эти операции не меняют
мощность множества, если оно бесконечно.
Из известных нам операций над множествами,
операция "взятия множества подмножеств" $X\mapsto 2^X$ стоит
особняком: как доказал Кантор, она всегда
увеличивает мощность множества.

\hfill

{\бф Теорема Кантора:}
Пусть $X$ --- непустое множество.
Тогда $X$ и $2^X$ неравномощны.

\hfill

{\бф Доказательство:}
Пусть $\phi:\; X \arrow 2^X$ --- биекция.
Рассмотрим множество $S$ всех $x\in X$, которые
не содержатся в $\phi(x)$. Пусть $\phi(x_0)=S$.
Зададимся вопросом --- лежит ли $x_0$ в $S$?
Если не лежит, то по определению $S$,
$x_0 \notin \phi(x_0)$ влечет $x_0\in S$ --- 
противоречие. Точно также, если 
$x_0$ лежит в $S$, имеем $x_0 \in \phi(x_0)$,
что влечет $x_0\notin S$. Следовательно,
биекция $\phi:\; X \arrow 2^X$ невозможна.
\endproof

\hfill

Континуум ---
это кардинальное число $2^\N$,
где $\N$ это множество натуральных чисел.
Множество, равномощное континууму, называется
{\бф континуальным}. 

Согласно теореме Кантора, мощность
континуума строго больше мощности $\N$.
Доказательство теоремы Кантора для
случая $X=\N$ можно наглядно изложить
следующим образом. 

Пусть $S\subset \N$ некоторое подмножество.
Рассмотрим последовательность $\{x_i\}$, $i=0,1,2, ...$,
где $x_i$ равен $1$, если $i\in S$ и 0, если
$i\notin S$. Это позволяет записывать
элементы $2^\N$ как последовательности
нулей и единиц. Ясно, что полученное
соответствие между элементами $2^\N$
и последовательностями взаимно однозначно.

Предположим, что задано взаимно однозначное
соответствие между $\N$ и $2^\N$. Это значит,
что есть последовательность последовательностей,
в которой найдется любая последовательность:
\[
\begin{array}{ccccc}
\lambda^0_0 & \lambda_1^0 & \lambda_2^0 & \lambda_3^0 & ...\\
\lambda^1_0 &\lambda_1^1 & \lambda_2^1 & \lambda_3^1 & ...\\
\lambda^2_0 &\lambda_1^2 & \lambda_2^2 & \lambda_3^2 & ...\\
\lambda^3_0 &\lambda_1^3 & \lambda_2^3 & \lambda_3^3 & ...
\end{array}
\]
Диагональ этой диаграммы --- 
это последовательность $\{\lambda^i_i\}$.

Рассмотрим последовательность $a_0, a_1, a_2, ...$,
полученную таким образом: если $\lambda^i_i$ равно 1,
то $a_i$ равно 0, в противном случае $a_i=1$.
Такая последовательность отличается
от каждой из $\{\lambda^i_*\}$ в $i$-м члене, 
а значит не совпадает ни с одной из последовательностей
$\{\lambda^i_*\}$. Поэтому отображение
$\N \arrow 2^\N$, относящее $i$ 
последовательность $\{\lambda^i_*\}$,
не может быть сюръективно.

Этот аргумент 
называется "диагональным методом Кантора";
он изобретен Кантором в 1891-м году. Несчетность
вещественных чисел Кантор доказал впервые 
в 1873-м, но совершенно другим способом.

Вещественное число от 0 до 1 можно
записать в двоичной системе счисления,
получив последовательность из нулей и единиц.
Некоторые последовательности соответствуют 
одним и тем же числам; например,
0.0011111111... и 0.01000000...

\задача
Докажите, что следующие множества равномощны континууму
\итем Множество $\R$ вещественных чисел
\итем Множество последовательностей из чисел 0,1,2,3
\итем Множество последовательностей целых чисел
\итем Множество функций $\phi:\; \Q \arrow \Q$
\ез

\указание
Воспользуйтесь теоремой Кантора-Бернштейна-Шредера.
\еу

\замечание
\label{_set_of_sets_Zamechanie_}
Из диагонального метода Кантора сразу следует,
что множество $S$ всех множеств не существует.
Действительно, множество всех подмножеств $S$ 
содержится в $S$. Значит, мощность $S$ не больше,
чем $2^S$, а такого
не бывает. На языке аксиоматической теории множеств
этот парадокс превращается в теорему --- 
"множества всех множеств не существует".
\еза



%%%%%%%%%%%%%%%%%%%%%%%%%%%%%%%%%%%%%%%%%%%%%%%%
\section{Континуум-гипотеза}
%%%%%%%%%%%%%%%%%%%%%%%%%%%%%%%%%%%%%%%%%%%%%%%%

Последние годы жизни Кантор провел, пытаясь доказать
{\ем гипотезу континуума}, которая утверждает, что
любое подмножество $\R$ либо континуально, либо счетно.
Обобщенная континуум-гипотеза утверждает, что
для любого бесконечного множества $X$,
не существует множества $Z$ мощности меньше 
$2^X$ и больше $X$. 

В 1940-м году Гёдель доказал, что гипотезу континуума
(и обобщенную гипотезу континуума) 
невозможно опровергнуть, пользуясь аксиомами
Цермело-Френкеля; иначе говоря, гипотеза континуума
может быть опровергнута только в том случае,
если эта система аксиом сама по себе противоречива.
Аргумент Гёделя весьма прост --- Гёдель
определил "конструктивный универсум",
набор множеств, полученных из пустого множества
естественными теоретико-множественными операциями,
и доказал, что в конструктивном универсуме
выполняются аксиомы Цермело-Френкеля. 
Мощность множества, полученного конструктивно, 
довольно легко вычислить явно, и проверить,
что кардиналов, промежуточных между $X$ и $2^X$,
не бывает.

В 1963-м году Пол Коэн доказал, что гипотезу
континуума (и обобщенную гипотезу континуума) 
невозможно вывести из этих аксиом,
то есть она {\ем недоказуема}.

Есть математики, считающие, что теория множеств
описывает "реально существующие объекты"; другие
математики считают, что теория множеств состоит
из формальных манипуляций, не имеющих основания
в "реальном мире". Эти два взгляда часто называют
"платоновским" и "формальным". Платон в книге 
"Государство" уподобил людей узникам, а 
материальный мир теням на стенах темного
узилища-пещеры. Светом, с точки зрения
Платона, является мудрость, доступная
лишь избранным. Материальный мир --- 
несовершенный отпечаток истинного мира
символов, идей и знания.

Для формалиста вопрос о "верности континуум-гипотезы"
после результатов Гёделя и Коэна не имеет смысла; 
для реалиста этот вопрос все еще осмысленный.

Гёдель был сторонником платоновской точки зрения
на математику; он считал, что континуум-гипотеза
неверна, и невозможность опровергнуть ее иллюстрирует
несовершенство системы аксиом Цермело-Френкеля.



%%%%%%%%%%%%%%%%%%%%%%%%%%%%%%%%%%%%%%%%%%%%%%%%
\section{Замечания}
%%%%%%%%%%%%%%%%%%%%%%%%%%%%%%%%%%%%%%%%%%%%%%%%

Теорема Кантора-Бернштейна-Шредера была
получена Кантором с помощью аксиомы выбора
и трансфинитной индукции. Доказательство,
приведенное выше, принадлежит Эрнсту 
Шредеру, одному из основателей математической
логики и изобретателю исчисления предикатов, 
и  Феликсу Бернштейну, ученику Кантора.
Поэтому эту теорему также называют теорема
Шредера-Бернштейна. Иногда ее называют
теорема Кантора-Бернштейна (доказательство
Шредера содержало ошибку, исправленную
Бернштейном в его диссертации). 

Интересно, что эту теорему
доказал Дедекинд в 1887 году, но его доказательство
было впервые опубликовано в собрании сочинений
Дедекинда (1932). 

Счетные множества по-английски называются
countable, или denumerable, соотношения 
частичного порядка --- partial order relations.

Чтобы избежать парадоксов наивной теории
множеств, нужно оперировать с множествами
"мощности не больше заданной", то есть
с множествами, которые допускают вложение
в заданное (очень большое) множество.
Это множество часто называется 
{\бф универсумом}. В 1920-х 
Джон фон Нойман (John von Neumann) 
сформулировал систему аксиом теории множеств,
основанную на понятии классов ("очень
больших множеств") и множеств 
("множеств небольшой мощности,
которыми можно оперировать в соответствии
с наивным представлением о множествах").
Эта система аксиом была упрощена и улучшена 
Паулем Бернайсом (Paul Bernays) в конце 1930-х,
и Гёделем в 1940-м. Она эквивалентна 
системе аксиом Цермело-Френкеля.

Для большинства
разделов современной математики
основанием является не теория множеств,
а теория категорий (о теории категорий
см. лекцию 14 и дальше).
Строгое построение теории 
категорий обыкновенно осуществляется 
на основе аксиоматики фон 
Ноймана-Бернайса-Гёделя
(NBG).

Аксиоматика NBG и описание ее применений
хорошо излагаются в англоязычной Википедии;
там же можно прочитать биографию
фон Ноймана с подробным очерком его 
работ по логике. 

Прекрасный учебник (а по совместительству
и задачник) наивной теории множеств --- первая часть книги
Н. К. Верещагина и А. Шеня "Лекции по математической 
логике и теории алгоритмов," доступная на сайте 
Независимого Университета: http://www.mccme.ru/free-books/

Доказательство Коэна недоказуемости континуум-гипотезы
можно найти в книжке "Теория множеств и
континуум-гипотеза", П. Дж. Коэн.

%%%%%%%%%%%%%%%%%%%%%%%%%%%%%%%%%%%%%%%%%%%%%%%%%%%%%%%%%%%%%%%%%%%%%%%%

\chapter{Аксиома выбора и ее приложения}
\label{_AC_Chapter_}

%%%%%%%%%%%%%%%%%%%%%%%%%%%%%%%%%%%%%%%%%%%%%%%%%%%%%%%%%%%%%%%%%%%%%%%%

%%%%%%%%%%%%%%%%%%%%%%%%%%%%%%%%%%%%%%%%%%%%%%%%

\section{Сечение отображения}

%%%%%%%%%%%%%%%%%%%%%%%%%%%%%%%%%%%%%%%%%%%%%%%%

Пусть $\phi:\; A \arrow B$ сюръективное отображение
множеств. {\бф Сечением} отображения $\phi$ называется
отображение $\psi:\; B \arrow A$, такое, что
$\psi\circ \phi=\Id_B$. 


\begin{figure}[ht]
\begin{center}\ \\
\epsfig{file=sechenie.eps,width=0.35\linewidth}\\
{\small \em Сечение сюръективного отображения}
\end{center}
\end{figure}

Аксиома выбора утверждает, что 
каждое сюръективное отображение имеет сечение.

%%%%%%%%%%%%%%%%%%%%%%%%%%%%%%%%%%%%%%%%%%%%%%%%

\section{Аксиоматическая теория множеств}

%%%%%%%%%%%%%%%%%%%%%%%%%%%%%%%%%%%%%%%%%%%%%%%%

Один из подходов к обоснованию математики 
называется {\бф формализмом}, и принадлежит он 
Давиду Гильберту. В начале XX-го века 
математика оказалась в кризисе, сотрясаемая
парадоксами, которые следовали
из ``наивной теории множеств'', придуманной Георгом
Кантором. Стало ясно, что обращение с произвольно
взятыми бесконечными множествами опасно и приводит
к противоречиям. В надежде избежать парадоксов, 
математики (Фреге, Гильберт, Цермело, Рассел и Уайтхед)
стали строить аксиоматическое обоснование
для теории множеств, надеясь, что аккуратное
следование логическим построениям поможет
избежать парадоксов. 

Система аксиом, предложенная Фреге, оказалось
противоречивой, основания математики по версии
Рассела-Уайтхеда --- слишком сложными, и к настоящему
времени канонической стала система аксиом, 
предложенная в 1908-м году Эрнстом Цермело,
и улучшенная в 1922 Адольфом Френкелем и 
(независимо от него) Торальфом Сколемом
(Adolf Fraenkel, Thoralf Skolem). 
В учебниках встречается немало 
версий системы Цермело-Френкеля,
но все они эквивалентны.

\begin{figure}[ht]
\begin{center}\ \\
\epsfig{file=Zermelo.eps,width=0.5\linewidth}\\
{Ernst Friedrich Ferdinand Zermelo\\
(1871 --- 1953)}
\end{center}
\end{figure}


Цермело провел 3 года (1905-1908),
пытаясь доказать, что его система аксиом
непротиворечива (не приводит к противоречию).
Вплоть до 1930-х годов, Гильберт с коллегами были
уверены, что в скором времени удастся доказать
непротиворечивость системы аксиом Цермело-Френкеля,
и аксиоматическая теория множеств станет фундаментом
для всей математики. 

В 1931-м году Курт Гёдель доказал, что 
непротиворечивость системы аксиом Цермело-Френкеля недоказуема.
Если точно, Гёдель доказал, что 
невозможно доказать непротиворечивость 
любой системы аксиом в математическом языке,
достаточно сильном, чтобы сформулировать
на нем утверждения арифметики, если из этой системы
аксиом следуют аксиомы Пеано (существование 
натуральных чисел и принцип индукции).
Еще точнее --- Гёдель доказал невозможность
доказательства средствами, допускающими 
формализацию в рамках того же самого 
математического языка.

Отмечу, что {\em опровергнуть} непротиворечивость
такой системы аксиом как раз весьма просто --
надо вывести из аксиом противоречие. Таким образом,
к примеру, была опровергнута система аксиом Готтлоба Фреге.

Многие математики убеждены в 
непротиворечивости системы аксиом Цермело-Френкеля;
другие считают, что в случае противоречий
будет изобретена новая система аксиом,
и осмысленную часть математических утверждений
с легкостью переведут на новый язык 
(и так до бесконечности, по мере обнаружения
противоречий). Именно этой точке зрения
следовала группа Бурбаки.

Есть немало математиков, считающих, что
аксиоматический метод порочен сам по себе.
Эта точка зрения была выдвинута голландским математиком
Брауэром (Luitzen Egbertus Jan Brouwer), знаменитым
топологом, в 1908-м году, в статье, провокативно
озаглавленной ``De onbetrouwbaarheid der 
logische principes'' --- ``О сомнительности
основ логики''. Брауэр считал, что классическая
(аристотелева) логика не может быть применима
к бесконечным множествам, и все математические
исследования, которые основаны на таких
применениях --- неправильны.
Особенно Брауэру не нравился принцип исключенного
третьего, на котором основаны популярные
доказательства ``от противного''. На протяжении 1920-х
годов Брауэр был редактором журнала Mathematische Annalen,
и он принципиально возвращал авторам все статьи, где
использовались доказательства от противного.

Брауэр называл свою философию ``интуиционизмом''.
В середине XX-го века русские математики, близкие к 
Маркову и Колмогорову, развили свою версию философии 
интуиционизма, под названием 
``конструктивизм'';\footnote{Надо отметить, что
лично Колмогоров не одобрял конструктивизма.}
как и интуиционисты, конструктивные математики
отрицают классическую математику, точнее --- 
все неявные доказательства существования.

С точки зрения конструктивиста, любой математический
объект должен быть задан явно. Например, действительное
число в конструктивной математике это алгоритм его
вычисления с любой точностью, плюс оценка скорости 
сходимости этого алгоритма.


Исследования по теории рекурсивных функций,
вдохновленные конструктивизмом, оказались очень
полезны в компьютерных науках, теории алгоритмов
и лингвистике.


%%%%%%%%%%%%%%%%%%%%%%%%%%%%%%%%%%%%%%%%%%%%%%%%

\section{Аксиома выбора и ее конкуренты}
\label{_AC_alterna_Section_}

%%%%%%%%%%%%%%%%%%%%%%%%%%%%%%%%%%%%%%%%%%%%%%%%

Большой части математиков (даже не следующих
экзотическим философиям) свойственно недоверие
к неявным построениям. Действительно, явный
пример числа (в виде ряда, к примеру)
гораздо удобнее иметь, чем теорему вида
``существует такое число, что...''.

Тем не менее, аргументами ``от противного''
и неявными построениями по необходимости 
пользуются почти все математики, потому что 
если от них отказаться, придется отказываться
от большого числа полезных утверждений,
не имеющих явного доказательства.

Самым патологическим примером ``неявной
конструкции'' является аксиома выбора,
которая утверждает существование объектов,
которые заведомо не могут быть построены
никакой явной конструкцией.

Ближе к концу 1920-х годов
оказалось, что аксиома выбора позволяет доказывать
(наряду с верными и полезными) утверждения
чрезвычайно сомнительные и противоречащие
физической интуиции. Так, в 1924-м 
году Банах и Тарский обнаружили парадоксальное
разложение шара в $\R^3$ в счетное количество
частей, из которых можно сложить два таких
же шара. Сейчас известно, что шар можно
разложить в 5 частей, и сложить из них два таких
же шара, изометрически передвигая эти части в  $\R^3$.

Это утверждение сродни мечте алхимиков получить
из небольшого куска золота очень большой, и 
вызывает столько же сомнений.

Впрочем, исключение аксиомы выбора не избавит
математику от противоречий: непротиворечивость
системы Цермело-Френкеля с аксиомой выбора\footnote{Эту систему
аксиом часто обозначают ZF+AC, или ZFC.} равносильна непротиворечивости
Цермело-Френкеля без нее.

Тем не менее, хорошим тоном считается
не пользоваться аксиомой выбора по возможности,
или пользоваться ею, всякий раз обозначая
факт неявности выбора.

В математике (кроме очень экзотических областей)
без аксиомы выбора довольно часто удается
обойтись. Основные утверждения, для которых
нужна аксиома выбора, такие.

1. Теорема о существовании максимальных идеалов
(см. лекцию 9).

2. Теорема о существовании базиса в любом бесконечномерном
векторном пространстве (``базиса Коши-Гамеля'').

3. Теорема Хана-Банаха о существовании замкнутой 
гиперплоскости, разделяющей два выпуклых, непересекающихся, замкнутых
подмножества в топологическом векторном пространстве.

4. Теорема Тихонова о компактности произведения
бесконечного количества компактов.

Аксиома выбора влечет
много утверждений, которые не нужны для доказательства
полезных теорем, и противоречат
интуиции --- наподобие парадокса Банаха-Тарского.
Основная проблема, которая следует из 
принятия аксиомы выбора --- существование
{\бф неизмеримых множеств}, множеств, для
которых не определена мера Лебега ({\bf мерой
Лебега} в математике называется 
понятие ``объема подмножества" в $\R^n$;
неизмеримые подмножетва --- подмножества,
для которых не определен объем).
Множества, которые фигурируют в парадоксе
Банаха-Тарского, очевидно неизмеримы: иначе
суммарный объем пяти частей, на которые
разбит шар, не мог бы равняться удвоенному
объему шара.

Существуют аксиомы теории множеств, которые
противоречат аксиоме выбора; самая известная
из них --- аксиома детерминированности
(axiom of determinacy),
из которой, среди прочего, следует,
что любое подмножество $\R^n$ измеримо.

Неизвестно, следует ли из непротиворечивости
ZF+AC непротиворечивость системы Цермело-Френкеля
с аксиомой детерминированности (ZF+AD). Другими
словами, ZF+AD может оказаться противоречива,
даже если ZF+AC непротиворечива. Поэтому
ZF+AD не нашла широкого употребления.

Другая причина, по которой аксиомы, отрицающие
аксиому выбора, не нашли употребления, такая.
Пусть $X$ и $Y$ --- два множества. Из аксиомы выбора
следует, что либо $X$ равномощно подмножеству $Y$, либо
$Y$ равномощно подмножеству $X$. Оказывается,
что это утверждение {\ем равносильно} аксиоме выбора.
Поэтому из любой аксиомы, отрицающей аксиому
выбора, следует существование таких множеств
$X$ и $Y$, что $X$ не равномощно подмножеству
$Y$, а $Y$ не равномощно подмножеству $X$.
Вместо изящной иерархии кардинальных
чисел,\footnote{{\bf Кардиналом}, или 
{\bf кардинальным
числом}, называется класс равномощных
множеств. Согласно теореме Кантора-Бернштейна, 
если одно множество равномощно подмножеству
другого, а другое --- подмножеству первого,
эти множества равномощны. Поэтому из двух
неравномощных множеств, одно равномощно
подмножеству другого, но не наоборот.
Это задает отношение порядка на кардинальных
числах, которое и называется {\бф иерархия
кардиналов}.}
построенной Кантором, мы получаем огромное
количество множеств, которые никак не соизмеримы
по мощности. 

Вместо использования аксиомы выбора
можно пользоваться ее слабой формой,
которая называется {\бф аксиома зависимого выбора}
(de\-pen\-dent choice). Аксиома зависимого выбора
утверждает следующее. Пусть дано  множество $X$,
и подмножество $V\subset X \times X$
такое, что $\pi_1(V)=X$, где $\pi_1$ --- проекция
на первый сомножитель. Тогда есть бесконечная
последовательность $\{x_i\}$
такая, что $(x_i, x_{i+1})\in V$, для любого $i$.

Система аксиом Цермело-Френкеля без аксиомы
выбора, но с зависимым выбором (ZF+DC)
достаточна для почти всех задач,
где необходим выбор, но недостаточна 
для утверждений 1-4, перечисленных выше. 
Также она следует из аксиомы детерминированности.

Любопытно, что непротиворечивость
Цермело-Френкеля равносильна
непротиворечивости системы аксиом,
состоящей из ZF+DC плюс аксиомы
"каждое подмножество $\R^n$
измеримо". Кроме того, DC следует 
из аксиомы детерминированности.

%%%%%%%%%%%%%%%%%%%%%%%%%%%%%%%%%%%%%%%%%%%%%%%%%%%%%%%%%%%%

\section{Вполне упорядоченные множества}

%%%%%%%%%%%%%%%%%%%%%%%%%%%%%%%%%%%%%%%%%%%%%%%%%%%%%%%%%%%%

\определение
Пусть $X$ --- множество, а $R\subset X\times X$ --- бинарное
отношение на множестве $X$, обозначенное $x_1 \prec x_2$.
Это отношение называется {\бф отношением частичного
порядка} (partial order), если верны следующие утверждения
\begin{description}
\item[транзитивность:] из $x\prec y$ и $y\prec z$ следует
$x\prec z$.
\item[асимметричность:] если $x\prec y$, то невозможно
$y\prec x$.
\end{description}
Множество $(X, \prec)$ с отношением частичного
порядка называется {\bf частично упорядоченным множеством}.
\ео

\определение
Пусть $(X, \prec)$ --- частично
упорядоченное множество. Если для каких-то $x, y\in X$ имеет
место  $x\prec y$ либо $y\prec x$, мы говорим,
что $x$ и $y$ {\бф сравнимы}.
Отношение $\prec$ называется {\бф отношением линейного
порядка} (total order), если любые два элемента сравнимы. Множество
$(X, \prec)$ с отношением линейного порядка
называется {\бф линейно упорядоченным множеством}, или {\бф цепью}.

Линейно упорядоченные множества также называются
{\бф монотонно упорядоченными}, или просто {\бф упорядоченными}.
\ео

Если $x=y$ или $x\prec y$, мы пишем $x\preccurlyeq y$.


\определение
Пусть $(X, \prec)$ --- линейно
упорядоченное множество, а $Y \subset X$ --- его
подмножество. Элемент $y_0\in Y$ называется {\бф
минимальным}, если  для любого $y\in Y$, 
имеем $y_0\preccurlyeq y$. Линейно упорядоченное
множество называется {\бф вполне упорядоченным}
(well-ordered set),
если любое его подмножество имеет минимальный
элемент. Отношение порядка на таком множестве
называется {\бф отношением полного порядка}.
\ео


\определение
{\бф Начальным элементом} вполне упорядоченного
множества называется его минимальный элемент.
{\бф Отрезком} линейно упорядоченного
множества $(X, \prec)$ называется подмножество 
$Y\subset X$ такое, что для любых $x, z\in Y$,
и любого $y\in X$ такого, что $x\prec y\prec z$,
имеем $y\in Y$. {\бф Начальным отрезком}
вполне упорядоченного
множества называется отрезок, содержащий
минимальный элемент. {\бф Начальным элементом}
отрезка называется его минимальный элемент.
\ео

\замечание
Пусть $X_0\subset X$ --- начальный отрезок 
вполне упорядоченного множества, $x_0$ его
начальный элемент, а $x$ --- минимальный
элемент в $X \backslash X_0$ (мы предполагаем,
что это множество непусто). 
Легко видеть, что $X_0$ --- множество всех
$y$ таких, что $x_0\preccurlyeq y\prec x$
(докажите это).
Мы обозначаем такой отрезок $[x_0, x[$.
\еза


Вполне упорядоченные множества изобрел Георг
Кантор, создатель теории множеств. Кантор
базировал свою теорию множеств на понятии
вполне упорядоченного множества и понятии ординала.


\begin{figure}[ht]
\begin{center}\ \\
\epsfig{file=Cantor.eps,width=0.45\linewidth}\\
{Georg Ferdinand Ludwig Philipp Cantor\\
(1845 --- 1918)}
\end{center}
\end{figure}

\определение
Два вполне упорядоченных множества называются
{\бф изоморфными}, если между ними есть
биекция, сохраняющая порядок. Классы 
изоморфизма\footnote{Классы изоморфизма суть классы эквивалентности
по соотношению ``изоморфизм''.}
вполне упорядоченных множеств называются
{\бф ординалами}, или же {\бф ординальными числами}.
\ео

\замечание
Ординалы можно складывать
(для этого надо взять объединение $X \bigsqcup Y$ двух
непересекающихся вполне упорядоченных множеств,
и положить $X \prec Y$). Это сложение некоммутативно,
но ассоциативно. Кроме того, ординалы можно умножать.  
Полный порядок на произведении $X\times Y$ задается так:
\[ (x,y) \prec (x', y') \text{ \ если\ } y\prec y',
\text{\ либо\ } y=y', x\prec x'.
\]
\еза

\задача
Докажите, что эти определения задают
вполне упорядоченные множества. Докажите
ассоциативность и умножения и сложения
и дистрибутивность слева сложения относительно
умножения. Приведите примеры, когда
сложение и умножение ординалов
не коммутативно и не дистрибутивно
справа.\footnote{Дистрибутивность слева:
$x\cdot (y+z) = x\cdot y +x\cdot z$. Дистрибутивность
справа: $(y+z)\cdot x = y\cdot x + z\cdot x$.}
\ез


%%%%%%%%%%%%%%%%%%%%%%%%%%%%%%%%%%%%%%%%%%%%%%%%
\теорема\label{_vpolne_upo_vlo_Teorema_}
Пусть $X$, $Y$ --- вполне
упорядоченные множества. Тогда 
$X$ изоморфно начальному отрезку $Y$,
либо $Y$ изоморфно начальному отрезку $X$.
Более того, такой изоморфизм определен
однозначно.

\хфилл

\ноиндент
{\бф Доказательство:}
Пусть $Z$ --- множество пар $(X_1, Y_1)$ 
изоморфных начальных отрезков $X$ и $Y$.

\хфилл

\ноиндент
{\бф Шаг 1.}
Изоморфизм начальных отрезков
$(X_1, Y_1)$ определяется однозначно
множеством $X_1$. В самом деле,
пусть существует два различных вложения
$\phi:\; X_1\arrow Y$ и 
$\phi':\; X_1\arrow Y$,
задающие изоморфизм $X_1$ и начального
отрезка $Y$. Обозначим через $x$ минимальный
элемент множества $X_1$, такой, что $\phi(x) \neq \phi'(x)$.
Тогда $\phi\restrict{[x_0, x[}= \phi'\restrict{[x_0, x[}$.
Поскольку $\phi$ и $\phi'$ --- изоморфизмы,
из этого следует, что $\phi(x) = \phi'(x)$.


\хфилл

\ноиндент
{\бф Шаг 2.}
Мы получили, что $Z$ упорядочено по включению
и это задает на $Z$ полный порядок (докажите). 
Пусть $x$ --- минимальный элемент
$X$, не принадлежащий $X_1$ для какого-то
$(X_1, Y_1)\in Z$. Если такого нет,
это значит, что $X$ изоморфен начальному
отрезку $Y$. Если $Y_1=Y$, мы все доказали.
В противном случае, начальный отрезок
$[x_0, x[$ изоморфен начальному отрезку
$[y_0, y[$, следовательно, отрезок
$[x_0, x]$ изоморфен $[y_0, y]$.
Мы пришли к противоречию. Это
доказывает Теорему \ref{_vpolne_upo_vlo_Teorema_}.
\endproof

\хфилл

Теоремой Цермело называется следующее утверждение,
доказанное  Эрнстом Цермело в 1904-м году.

\хфилл

\теорема
(Теорема Цермело, "well-ordering theorem")
Любое множество может быть вполне упорядочено.

\хфилл

Теорема Цермело равносильна аксиоме выбора;
мы докажем это немного погодя.



%%%%%%%%%%%%%%%%%%%%%%%%%%%%%%%%%%%%%%%%%%%%%%%%%%%%%%%%%%%%

\section{Лемма Цорна и теорема Цермело}

%%%%%%%%%%%%%%%%%%%%%%%%%%%%%%%%%%%%%%%%%%%%%%%%%%%%%%%%%%%%

Другое утверждение, также равносильное аксиоме выбора --- 
лемма Цорна. 

Пусть $(S, \prec)$ --- частично упорядоченное
множество. Элемент $x\in S$ называется
{\бф максимальным}, если не существует $y\in S$, 
такого, что $x\prec y$.
Для подмножества $S_1\subset S$ и $x\in S$, 
мы пишем $S_1 \preccurlyeq x$, если для каждого
$\xi \in S_1$ имеем $\xi \preccurlyeq x$.


\хфилл

\теорема
(лемма Цорна) Пусть $(S, \prec)$  --- частично упорядоченное
множество, причем для любого вполне упорядоченного 
подмножества\footnote{Довольно часто в утверждении леммы Цорна
пишут ``линейно упорядоченное подмножество'' вместо
вполне упорядоченного. Эти две формулировки равносильны.
Доказательство равносильности предоставлено читателю
в качестве нетрудного упражнения.}
$S_1\subset S$ найдется элемент $\xi\in S$
такой, что $S_1 \preccurlyeq \xi$. Тогда в $S$ найдется максимальный
элемент.

\хфилл


%%%%%%%%%%%%%%%%%%%%%%%%%%%%%%%%%%%%%%%%%%%%%%%%
\теорема\label{_AC_ZL_WOT_Teorema_}
Следующие утверждения равносильны:

\begin{description}
\item[ZL] Лемма Цорна.
\item[WOT] Теорема Цермело. 
\item[AC] Аксиома выбора.
\end{description}


\ноиндент
{\бф Доказательство:} 
Отметим, что полностью аккуратное доказательство
Теоремы \ref{_AC_ZL_WOT_Teorema_} требует
использования аксиоматической теории множеств.

\хфилл

\noindent
{\бф ZL $\Rightarrow$ WOT:}
Пусть $X$ --- любое множество, а $S$ --- множество
всех пар $(X_1, \prec)$, где $X_1\subset X$ --- подмножество,
а $\prec$ --- отношение полного порядка на $X_1$.

Рассмотрим отношение частичного порядка на $S$:
$(X_1, \prec) < (X_2, \prec)$, если $(X_1, \prec)$ это начальный
отрезок $(X_2, \prec)$. Легко видеть, что условие
леммы Цорна выполнено для $(S, <)$: если
$S_1\subset S$ вполне (и даже линейно) упорядочено, объединение 
$\Xi\subset X$ всех элементов $S_1$ с естественным отношением
порядка --- вполне упорядочено и удовлетворяет $S_1\leq\Xi$.
Поэтому в $S$ есть максимальный элемент
$(\Xi, \prec)$. Если $\Xi \neq X$, возьмем $\xi \in X \backslash \Xi$,
и определим отношение порядка на $\Xi_1:=\Xi \cup \{\xi\}$,
положив $\Xi \prec \xi$. Мы получим, что $\Xi_1$ --- вполне
упорядоченное множество, начальным
отрезком которого является $\Xi$,
а значит $\Xi$ не максимально. Мы пришли
к противоречию; следовательнo $\Xi=X$.
Поэтому $X$ вполне упорядочено.

\хфилл

\ноиндент
{\бф WOT $\Rightarrow$ AC:}
Пусть $X\stackrel\phi \arrow Y$ --- сюръекция, а $X$ вполне
упорядочено. Определим отображение $Y \stackrel\psi\arrow X$,
взяв за $\psi(y)$ минимальный (в смысле полного порядка)
элемент $\phi^{-1}(y)$. Легко видеть, что это сечение.

\хфилл

\ноиндент
{\бф AC $\Rightarrow$ ZL.} 

\хфилл

\ноиндент
{\bf  Шаг 1:}
Пусть $(S, \prec)$ --- частично упорядоченное множество,
удовлетворяющее условиям леммы Цорна, и не содержащее
максимального элемента. Тогда для каждого вполне
упорядоченного подмножества $S_1\subset S$, 
найдется $\xi \succcurlyeq S_1$. Поскольку $\xi$
не максимален, найдется $\xi_1\in S$ такой, что
$\xi_1 \succ \xi$, а значит, $\xi \succ S_1$.

\хфилл

\ноиндент
{\bf  Шаг 2:}
Пусть ${\goth S}$ --- множество
вполне упорядоченных подмножеств $S$,
а $\gamma:\; {\goth S}\arrow S$ --- отображение,
переводящее $S_1 \subset S$ в элемент $\xi\in S$,
удовлетворяющий $\xi \succ S_1$. 
Такое отображение существует в силу аксиомы выбора.
Для доказательства этого, рассмотрим множество
всех пар 
\[ {\goth R}:= \{ (S_1 \in {\goth S}, \xi \in S) \ | \  \xi \succ S_1\}
\]
Естественная проекция ${\goth R}\arrow {\goth S}$
сюръективна, что доказано на шаге 1. Сечение
${\goth R}\arrow {\goth S}$ в композиции с проекцией
${\goth R}\arrow S$ задаст искомое отображение $\gamma$.



\хфилл

\ноиндент
{\bf  Шаг 3:}
Пусть $\Theta$ --- множество вполне упорядоченных
подмножеств $P\subset S$ таких, что
для каждого $p\in P$, начальный отрезок
$[p_0, p[$ удовлетворяет $p = \gamma([p_0, p[)$.
Когда $p=p_0$, это значит, что
$p_0=\gamma(\emptyset)$.

Множество $\Theta$ вполне упорядочено по вложению.
В самом деле, возьмем $P, Q\in \Theta$,
и пусть $p$ --- минимальный элемент $P$
такой, что отрезок $[p_0,p[$ лежит в $Q$,
а $[p_0,p]$ уже не лежит в $Q$.
Поскольку $p=\gamma([p_0,p[)$, а
$[p_0,p[$ лежит в $Q$, из $p\notin Q$ следует,
что $Q=[p_0,p[$.

\хфилл

\ноиндент
{\bf  Шаг 4:}
Мы получили, что объединение
$P_\infty:=\bigcup_{P\in\Theta} P$ лежит в $\Theta$.
Это невозможно, потому что объединение
$P_\infty \cup \gamma(P_\infty)$ строго больше
$P_\infty$, и тоже лежит в $\Theta$.
Мы пришли к противоречию. Лемма Цорна доказана.
\endproof

{
%%%%%%%%%%%%%%%%%%%%%%%%%%%%%%%%%%%%%%%%%%%%%%%%%%%%%%%%%%%%%%%%%%%%%%%%

\part{Топология в задачах}

\renewcommand{\PartName}{Часть II. Задачи по топологии}

\renewcommand{\chaptermark}[1]{\markboth{{\bf
  #1}}{{\sc\PartName}}}

\def\замечание{\begin{zamechanie}}
\def\еза{\end{zamechanie}}




%%%%%%%%%%%%%%%%%%%%%%%%%%%%%%%%%%%%%%%%%%%%%%%%%%%%%%%%%%%%%%%%%%%%%%%%



Материалу первого листка предшествует
определение поля вещественных чисел. Поскольку это 
часть стандартного курса анализа (а часто и школьной
математики), я отнес их в конец книги, в приложения.

%%%%%%%%%%%%%%%%%%%%%%%%%%%%%%%%%%%%%%%%%%%%%%%%%%%%%%%%%%%%

\chapter{Листок 1: Метрические пространства и норма.}

%%%%%%%%%%%%%%%%%%%%%%%%%%%%%%%%%%%%%%%%%%%%%%%%%%%%%%%%%%%%


Для зачета по каждому листку надо сдать все задачи со звездочками,
либо все задачи без звездочек. Задачи с двумя звездочками
можно не сдавать. Сдавшим $k$ задач с двумя звездочками
разрешается не сдавать $2k$ задач со звездочками 
из того же листочка. Задачи, обозначенные (!),
следует сдавать всем.


В этом листочке предполагается знакомство с определением линейного
пространства и скалярного произведения (т.е. положительно
определенной билинейной симметричной формы), знакомство
с понятием кольца, поля, и определением поля вещественных чисел.

%%%%%%%%%%%%%%%%%%%%%%%%%%%%%%%%%%%%%%%%%%%%%%%%%%%%%%%%%%%%
\subs{Метрические пространства, выпуклые множества, норма.}
%%%%%%%%%%%%%%%%%%%%%%%%%%%%%%%%%%%%%%%%%%%%%%%%%%%%%%%%%%%%

\begin{opredelenie} Метрическое пространство есть множество $X$,
снабженное такой функцией $d:\; X \times X \to \R$, что
\begin{enumerate}
\item Для любых $x, y \in X$ имеем $d(x,y)\geq 0$, причем равенство
имеет место тогда и только тогда, когда $x=y$.

\item Симметричность: $d(x, y) = d (y, x)$

\item ``Неравенство треугольника'': для любых $x, y, z \in X$,
$$
d(x,z) \leq  d(x,y) + d(y,z).
$$
\end{enumerate}
Функция $d$, удовлетворяющая этим условиям, называется {\bf
метрикой}. Число $d(x,y)$ называется ``расстоянием между $x$ и
$y$''.
\end{opredelenie}

Если $x\in X$ --- точка, а $\epsilon$ --- вещественное число, то
множество
$$ 
B_\epsilon(x) = \{ y \in X \ \  | \ \  d(x,y)< \epsilon\}
$$
называется {\bf (открытый) шар радиуса $\epsilon$ с центром в $x$}.
Такой шар еще называется {\bf $\epsilon$-шар}.  Замкнутый шар
определяется как
$$ 
\overline B_\epsilon(x) = \{ y \in X \ \  | \ \  d(x,y)\leq \epsilon\}.
$$

\begin{zadacha} Рассмотрим любое подмножество в евклидовой плоскости
$\R^2$ с функцией $d$, заданной как $d(a, b)= |ab|$, где $|ab|$ --
длина отрезка $[a, b]$ на плоскости. Докажите, что это метрическое
пространство.
\end{zadacha}

\begin{zadacha} Рассмотрим такую функцию $d_\infty:\;\R^2\times \R^2
\to \R$:
$$
(x, y), (x', y') \mapsto \max (|x-x'|, |y-y'|).
$$
Докажите, что этo --- метрика. Опишите единичный шар с центром в нуле.
\end{zadacha}

\begin{zadacha} Рассмотрим такую функцию $d_1:\;\R^2\times \R^2 \to \R$:
$$
(x, y), (x', y') \mapsto |x-x'|+ |y-y'|.
$$
Докажите, что этo --- метрика. Опишите единичный шар с центром в нуле.
\end{zadacha}

\begin{zadacha}[*] Функция $f:\; [0, \infty[ \to [0, \infty[$ называется
{\bf выпуклой вверх}, если $f(\lambda x+ (1-\lambda) y)
\geq \lambda f(x) + (1-\lambda)f(y)$,
для любого вещественного $\lambda\in [0,1]$. 
Пусть $f$ --- такая функция, а $(X, d)$ --- метрическое пространство.
Предположим, что $f(\lambda)=0$ тогда и только тогда, когда $\lambda
= 0$. Докажите, что функция $d_f(x,y) = f(d(x,y))$ задает метрику на
$X$.
\end{zadacha}

\begin{zadacha} Пусть $V$ --- линейное пространство с 
положительно определенной билинейной симметричной формой $g(x,y)$ (в
дальнейшем мы будем называть такую форму {\bf скалярным
произведением}). Определим ``расстояние'' $d_g:\; V\times V \to \R$
как $d_g(x, y) = \sqrt{g(x-y, x-y)}$. Докажите, что $d(x,y)\geq 0$,
причем равенство имеет место тогда и только тогда, когда $x=y$.
\end{zadacha}

\begin{opredelenie} Пусть $x\in V$ --- вектор векторного пространства.
{\bf Параллельный перенос на вектор $x$} --- это отображение $P_x:\;
V \to V$, $y\mapsto y+x$.
\end{opredelenie}

\begin{zadacha} Докажите, что функция $d_g$ ``инвариантна
относительно параллельных переносов'', т.е.  $d_g(a, b) = d_g
(P_x(a), P_x(b))$.
\end{zadacha}

\begin{zadacha} Докажите, что  $d_g$ удовлетворяет 
неравенству треугольника:
$$
\sqrt{g(x-y,x-y)} \leq \sqrt{g(x,x)}+ \sqrt{g(y,y)}
$$
\end{zadacha}

\begin{ukazanie} Рассмотрим подпространство
$V_0 \subset V$, порожденное $x$ и $y$. Докажите, что оно 
либо одномерно, либо изоморфно,
как пространство со скалярным произведением, пространству $\R^2$ со
скалярным произведением $g((x,y), (x', y')) = xx' + yy'$.
Воспользуйтесь неравенством треугольника для $\R^2$.
\end{ukazanie}

\begin{zadacha}[!] Докажите, что $d_g$ --- это метрика.
\end{zadacha}

\begin{ukazanie} Пользуясь инвариантностью относительно
параллельных переносов, сведите эту задачу к предыдущей.
\end{ukazanie}

\begin{opredelenie} Пусть $V$ --- пространство со скалярным
произведением $g$, а $d_g$ --- метрика, построенная выше. Эта метрика
называется {\bf евклидовой.}
\end{opredelenie}

\begin{opredelenie} Пусть $V$ --- линейное пространство,
$P_x: \; V \to V$ --- параллельный перенос, а $V_1\subset V$ --
одномерное подпространство. Тогда образ $P_x(V_1)$ называется {\bf
прямой} в $V$.
\end{opredelenie}

\begin{zadacha} Даны две разные точки $x, y \in V$. Докажите, что
существует единственная прямая $V_{x,y}$, проходящая через $x$ и
$y$.
\end{zadacha}

\begin{opredelenie} Пусть $l$ прямая, проведенная через
точки $x$ и $y$, $a$ --- точка, лежащая на
$l$. Мы говорим, что $a$ лежит {\bf между} $x$, $y$,
если $d(x,a)+ d(a,y)= d(x,y)$.
{\bf Отрезок прямой между $x$ и $y$}
(обозначается $[x,y]$) есть множество всех точек прямой $V_{x,y}$,
которые ``лежат между'' $x$ и $y$. 
\end{opredelenie}

\begin{zadacha} Даны три разные точки на прямой. Докажите,
что одна (и только одна) из этих точек лежит между другими.
Докажите, что отрезок $[x,y]$ --- это множество всех
точек $z$ вида $a x + (1-a) y$, где $a \in [0, 1] \subset \R$.
\end{zadacha}

\begin{opredelenie} Пусть $V$ --- линейное пространство,
а $B\subset V$ --- некоторое подмножество. Говорят, что подможество
$B$ {\bf выпуклое}, если для любых $x, y \in V$, $B$ содержит все
точки отрезка $[x,y]$.
\end{opredelenie}

\begin{opredelenie} Пусть $V$ --- линейное пространство
над $\R$. {\bf Нормой} на $V$ называется такая функция $\rho:\; V
\to \R$, что выполняются следующие свойства.
\begin{enumerate}
\item Для любого $v\in V$ имеем $\rho(v) \geq 0$. Более того,
$\rho(v)>0$ для всех ненулевых $v$.

\item $\rho(\lambda v) = |\lambda| \rho(v)$

\item Для любых $v_1, v_2\in V$ выполнено $\rho(v_1+ v_2) \leq
\rho(v_1)+ \rho(v_2)$.
\end{enumerate}
Норму вектора часто обозначают $\vert x \vert$ или $|x|$. 
\end{opredelenie}


\begin{zadacha} Пусть $V$ --- линейное пространство
над $\R$, и пусть $\rho:\; V \to \R$ --- норма на $V$.  Рассмотрим
функцию $d_\rho:\; V \times V \to \R$, $d_\rho (x, y) =
\rho(x-y)$. Докажите, что это метрика на $V$.
\end{zadacha}

\begin{zadacha}[*] Пусть $d:\; V \times V \to \R$ --
метрика на $V$, инвариантная относительно параллельных
переносов. Предположим, что $d$ удовлетворяет условию
$$
d(\lambda x, \lambda y)= |\lambda| d(x,y)
$$
для всех $\lambda \in \R$. Докажите, что $d$ получается из нормы
$\rho:\; V \to \R$ по формуле $d (x, y) = \rho(x-y)$.
\end{zadacha}

\begin{zadacha} Пусть $V$ --- линейное пространство
над $\R$, а $\rho:\; V \to \R$ --- норма на $V$. Рассмотрим множество
$B_1(0)$ всех точек с нормой $\leq 1$. Докажите, что это множество
выпукло.
\end{zadacha}

\begin{opredelenie} Пусть $V$ --- векторное пространство
над $\R$, а $v$ ненулевой вектор. Тогда множество всех векторов вида
$\{ \lambda v, \ \ | \ \ \lambda >0\}$ называется {\bf лучом в $V$}.
\end{opredelenie}

\begin{opredelenie} {\bf Центральная симметрия} в $V$ --- это
отображение $x \mapsto -x$.
\end{opredelenie}

\begin{zadacha}[*] Пусть центрально симметричное выпуклое множество
$B\subset V$ не содержит лучей и пересекается с каждым лучом $\{
\lambda v, \ \ | \ \ \lambda >0\}$. Рассмотрим функцию
$$
v \overset{\rho}{\to} \sup\{\lambda \in \R^{>0}\ \ | \ \ \lambda^{-1}
v\notin B\}
$$
Докажите, что это норма на $V$. Докажите, что все нормы получаются
таким образом.
\end{zadacha}

\замечание
Эту функцию обыкновенно
называют ``функционал Минковского, построенный
по телу''.
\еза


\begin{zadacha} Пусть $G$ --- абелева группа, а 
$\nu:\; G \to \R$ функция, которая принимает неотрицательные
значения, и положительные значения для всех ненулевых $g\in G$.
Предположим, что $\nu(a+b) \leq \nu(a) + \nu(b)$, $\nu(0)=0$, а также
что $\nu(g) = \nu(-g)$ для всех $g\in G$.  Докажите, что
функция $d_\nu:\; G \times G \to \R$, $d_\nu(x, y) = \nu (x-y)$ --
это метрика на $G$.
\end{zadacha}

\begin{zadacha} Метрика $d$ на абелевой группе $G$ называется
{\bf трансляционно инвариантной}, если $d(x+g, y+g) = d(x,y)$ для всех $x, y, g
\in G$. Докажите, что любая трансляционно инвариантная метрика $d$ получена из
некоторой функции $\nu:\; G \to \R$ по формуле $d(x, y) = \nu
(x-y)$.
\end{zadacha}

\begin{opredelenie} 
Зафиксируем простое число $p\in \Z$. Рассмотрим функцию $\nu_p:\; \Z
\to \R$, которая ставит числу $n = p^k r$ ($r$ не делится на $p$) в
соответствие число $p^{-k}$, а $\nu_p(0)=0$. Эта функция называется {\bf
$p$-адическим нормированием на $\Z$}.
\end{opredelenie}

\begin{zadacha} Докажите, что функция $d_p(m, n) = \nu_p(n-m)$
задает метрику на $\Z$. Эта метрика называется {\bf $p$-адической
метрикой на $\Z$}.
\end{zadacha}

\begin{ukazanie} Проверьте соотношение $\nu_p(a+b) \leq \nu(a) +
\nu(b)$ и воспользуйтесь предыдущей задачей.
\end{ukazanie}

\begin{opredelenie} Пусть $R$ --- кольцо, а $\nu:\; R \to \R$ --
функция, которая принимает неотрицательные значения, и положительные
значения для всех ненулевых $r$. Предположим, что $\nu(r_1 r_2) =
\nu(r_1) \nu(r_2)$, а $\nu(r_1+r_2) \leq \nu(r_1) + \nu(r_2)$.
Тогда $\nu$ называется {\bf нормированием} кольца $R$. Кольцо,
снабженное нормированием, называется {\bf нормированное кольцо}.
\end{opredelenie}

\begin{zamechanie} Как видно из вышеприведенных задач,
нормирование на кольце $R$ определяет инвариантную метрику на $R$. В
дальнейшем любое нормированное кольцо будет рассматриваться как
метрическое пространство.
\end{zamechanie}

\begin{zadacha} Докажите, что $\nu_p$ --- нормирование на
кольце $\Z$. Определите нормирование на $\Q$, которое продолжает $\nu_p$.
\end{zadacha}

%%%%%%%%%%%%%%%%%%%%%%%%%%%%%%%%%%%%%%%%%%%%%%%%%%%%%%%%%%%%
\subs{{Полные метрические пространства.}}
%%%%%%%%%%%%%%%%%%%%%%%%%%%%%%%%%%%%%%%%%%%%%%%%%%%%%%%%%%%%

\begin{opredelenie} Пусть $(X, d)$ --- метрическое
пространство, а $\{a_i\}$ --- последовательность точек из
$X$. Последовательность $\{a_i\}$ называется {\bf
последовательностью Коши}, если для каждого $\epsilon>0$ найдется
$\epsilon$-шар в $X$, содержащий все $a_i$, кроме конечного числа.
\end{opredelenie}

\begin{zadacha} Пусть $\{a_i\}$, $\{b_i\}$ --- последовательности
Коши в $X$. Докажите, что $\{ d(a_i, b_i)\}$ --- последовательность
Коши в $\R$.
\end{zadacha}

\begin{opredelenie} Пусть $(X, d)$ --- метрическое
пространство, а $\{a_i\}$, $\{b_i\}$ --- последовательности Коши в
$X$. Последовательности $\{a_i\}$ и $\{b_i\}$ называются {\bf
эквивалентными}, если последовательность $a_0, b_0, a_1, b_1,\ldots$
-- последовательность Коши.
\end{opredelenie}

\begin{zadacha} Пусть $\{a_i\}$, $\{b_i\}$ --- последовательности
Коши в $X$. Докажите, что $\{a_i\}$, $\{b_i\}$ эквивалентны тогда и
только тогда, когда $\lim\limits_{i\to \infty} d(a_i, b_i) =0$.
\end{zadacha}

\begin{zadacha} Пусть $\{a_i\}$, $\{b_i\}$ --- эквивалентные
последовательности Коши в $X$, а $\{c_i\}$ --- еще одна
последовательность Коши. Докажите, что
$$
\lim\limits_{i\to \infty} d(a_i, c_i)=
   \lim\limits_{i\to \infty} d(b_i, c_i)
$$
\end{zadacha}

\begin{zadacha}[!] Пусть $(X, d)$ метрическое пространство, а
$\overline{X}$ --- множество классов эквивалентности
последовательностей Коши. Докажите, что функция
$$
\{a_i\}, \{b_i\} \mapsto \lim\limits_{i\to \infty} d(a_i, b_i)
$$
задает метрику на $\overline X$.
\end{zadacha}

\begin{opredelenie}\label{compl.bad.defn} 
В такой ситуации, $\overline{X}$ называется {\bf пополнением $X$}.
\end{opredelenie}

\begin{zadacha} Рассмотрим естественное отображение
$X \to \overline{X}$, $x \mapsto \{ x, x, x, x, ...\}$. Докажите, что
это вложение, которое сохраняет метрику.
\end{zadacha}

\begin{opredelenie} Пусть $A$ --- подмножество в $X$.
Элемент $c\in X$ называется {\bf предельной точкой} подмножества
$A$, если в любом открытом шаре, содержащем $c$, содержится
бесконечное количество элементов $A$.
\end{opredelenie}

\begin{zadacha} Дана последовательность Коши. Докажите, что
у нее не может быть больше одной предельной точки.
\end{zadacha}

\begin{opredelenie} Пусть $\{a_i\}$ --- последовательность
Коши. Мы говорим, что $\{a_i\}$ {\bf сходится к $x\in X$}, или {\bf
имеет предел в $x$} (пишется $\lim\limits_{i\to \infty} a_i =x$), если $x$
-- предельная точка $\{a_i\}$
\end{opredelenie}

\begin{opredelenie}
Метрическое пространство $(X,d)$ называется {\bf полным}, если любая
последовательность Коши в $X$ имеет предел.
\end{opredelenie}

\begin{zadacha}[!] Докажите, что пополнение 
метрического пространства полно.
\end{zadacha}

\begin{opredelenie} Подмножество $A \subset X$ метрического
пространства называется {\bf плотным}, если в каждом открытом шаре в
$X$ содержится элемент из $A$.
\end{opredelenie}

\begin{zadacha} Докажите, что $X$ плотно в $\overline{X}$.
\end{zadacha}

\begin{zadacha}[!] Пусть $R$ --- кольцо, снабженное нормированием $\nu$.
Постройте сложение и умножение на пополнении $R$ относительно
метрики, соответствующей нормированию. Докажите, что $\overline{R}$
снабжено нормированием, продолжающем нормирование на $R$.
\end{zadacha}

\begin{opredelenie} Нормированное кольцо $\overline{R}$ называется
{\bf пополнением $R$ относительно нормирования $\nu$.}
\end{opredelenie}

\begin{zadacha}[*] Пусть $R$ --- нормированное кольцо, 
а $\overline{R}$ его пополнение.  Предположим, что $R$ --
поле. Докажите, что $\overline{R}$ --- тоже поле.
\end{zadacha}

\begin{zadacha} Докажите, что $\R$ получено пополнением
$\Q$ относительно нормирования $q \mapsto |q|$. Можно ли это использовать в
качестве еще одного определения $\R$?
\end{zadacha}

\begin{opredelenie} Пополнение $\Z$ относительно нормирования
$\nu_p$ называется {\bf кольцо целых $p$-адических чисел}. Это
кольцо обозначается $\Z_p$.
\end{opredelenie}

\begin{zadacha} Пусть $(X, d)$ --- метрическое пространство,
а $\{a_i\}$ --- последовательность точек из $X$.  Предположим, что
ряд $\sum d(a_i, a_{i-1})$ сходится. Докажите, что $\{a_i\}$ --
последовательность Коши. Верно ли обратное?
\end{zadacha}

\begin{zadacha}[!] Докажите, что для любой последовательности целых
чисел $a_k$ ряд $\sum a_k p^k$ сходится в $\Z_p$.
\end{zadacha}

\begin{ukazanie} Воспользуйтесь предыдущей задачей.
\end{ukazanie}

\begin{zadacha} Докажите, что $(1-p) (\sum_{k=0}^\infty  p^k) =1$
в $\Z_p$.
\end{zadacha}

\begin{zadacha}[*] Докажите, что любое целое число, которое
не делится на $p$, обратимо в $\Z_p$.
\end{zadacha}

\begin{opredelenie} Пополнение $\Q$ относительно нормирования, полученного
продолжением $\nu_p$, обозначается $\Q_p$ и называется {\bf поле
$p$-адических чисел}.
\end{opredelenie}

\begin{zadacha}[*] Дано $x\in \Q_p$. Докажите, что
$x = \frac {x'}{p^k}$, где $x'\in \Z_p$.
\end{zadacha}

\begin{zadacha}[*]
Докажите, что $\lim\limits_{n\to \infty} \sqrt[n]{n}=1$
(здесь предел берется в $\R$, с обычной метрикой).
\end{zadacha}

\begin{opredelenie} Нормирование $\nu$ кольца $R$ называется
{\bf неархимедовым}, если $\nu (x+y) \leq \max (\nu(x), \nu(y))$
для всех $x,y$. В
противном случае нормирование называется {\bf архимедовым}.
\end{opredelenie}

\begin{zadacha}[*] Пусть $\nu$ --- нормирование в $\Q$.
Докажите, что $\nu$ неархимедово тогда и только тогда,
когда $\Z$ содержится в единичном шаре. 
\end{zadacha}

\begin{ukazanie} Воспользуйтесь пределом 
$\lim\limits_{n\to \infty} \sqrt[n]{n}=1$.
Оцените значение $\sqrt[n]{((\nu(x+y)^n)}$ для
больших $n$, воспользовавшись оценкой на
биномиальные коэффициенты: $\nu(C^k_n)\leq 1$.
\end{ukazanie}

\begin{zadacha}[*]\label{_prostoj_ideal_p_ad_Zadacha_}
Пусть $\nu$ --- неархимедово
нормирование в $\Z$.
Рассмотрим множество
${\mathfrak m} \subset \Z$, состоящее из всех целых $n$ с
$\nu(n)<1$. Выведите из неархимедовости, что $\mathfrak m$ это {\em идеал} в $\Z$
(идеал в кольце $R$ есть подмножество, замкнутое относительно
сложения и умножения на элементы из $R$). Докажите, что
идеал $\mathfrak m$
{\em простой} (простой идеал это такой идеал, что $xy \notin
{\mathfrak m}$ для всех $x, y \notin {\mathfrak m}$). 
\end{zadacha}


\begin{zadacha}[*] Докажите, что любой идеал в $\Z$ 
имеет вид \[ \{0, \pm1 m, \pm 2m, \pm3m, ...\}\] для некоторого $m\in \Z$.
Докажите, что любой простой идеал ${\mathfrak m}$ 
в $\Z$ имеет вид $\{0, \pm1 p, \pm 2p, \pm3p, ...\}$, где $p=0, 1$
либо $p$ простое.
\end{zadacha}

\begin{ukazanie} Воспользуйтесь алгоритмом Евклида.
\end{ukazanie}

\begin{zadacha}[*] Пусть $\nu$ --- неархимедово нормирование $\Q$, 
а \[ \mathfrak m = \{p, 2p, 3p, 4p, ...\}\] --- 
идеал, построенный в задаче \ref{_prostoj_ideal_p_ad_Zadacha_}. 
Докажите, что существует
такое вещественное число $\lambda>1$, что $\nu(n) = \lambda^{-k}$ для
каждого $n=p^k r$, $r\not\vdots p$.
\end{zadacha}

\begin{zadacha}[*] Пусть $\nu$ --- такое нормирование $\Q$,
что $\nu(2)\leq 1$. Докажите, что $\nu(a)< \log_2 (a)+1$
для любого целого $a>0$.
\end{zadacha}

\begin{ukazanie} Воспользуйтесь представлением числа в 
двоичной системе счисления.
\end{ukazanie}

\begin{zadacha}[*] Пусть $\nu$ --- такое нормирование $\Q$,
что $\nu(2)\leq 1$. Докажите, что $\nu(a)\leq 1$
для любого целого $a>0$ (т.е. $\nu$ неархимедово).
\end{zadacha}

\begin{ukazanie} Выведите из
$\lim\limits_{n\to \infty} \sqrt[n]{n}=1$ соотношение
$\lim\limits_{n\to \infty} \frac{\log n}{n}=0$. Воспользовавшись
предыдущей задачей, получите 
$\lim\limits_{N\to \infty} \nu(a^N)\leq 1.$
\end{ukazanie}

\begin{zadacha}[*] \label{_Cauchy_p-adic_Zadacha_}
Пусть $a_i$ --- последовательность
Коши рациональных чисел вида $\frac{x}{2^n}$ 
(``последовательность Коши'' здесь понимается в обычном
смысле, то есть как в вещественных числах). 
Предположим, что нормирование $\nu$ на $\Q$ архимедово.
Докажите, что $\nu(a_i)$ --- последовательность Коши.
\end{zadacha}

\begin{ukazanie}
Записав $x$ в двоичной системе счисления,
докажите, что \[ \nu(x/2^n)\leq
  \nu(2)^{\log_2(x)+1} /\nu(2)^n\leq \nu(2)^{\log_2(x+1)-n}.\]
\end{ukazanie}


\begin{zadacha}[*] Выведите из задачи 
\ref{_Cauchy_p-adic_Zadacha_}, что любое архимедово 
нормирование $\nu$ продолжается до непрерывной функции на $\R$,
которая удовлетворяет $\nu(xy)=\nu(x)\nu(y)$. Докажите, что
$\nu$ получается как $x \mapsto |x|^\lambda$ для какой-то константы
$\lambda>0$. Выразите $\lambda$ через $\nu(2)$.
\end{zadacha}

\begin{zadacha}[*] Для каких $\lambda>0$ функция $x \mapsto
|x|^\lambda$ задает нормирование на $\Q$?
\end{zadacha}

Мы получили полную классификацию нормирований на $\Q$: любое
нормирование получается как степень $p$-адического нормирования либо
модуля. Эта классификация называется {\bf теорема Островского}.


%%%%%%%%%%%%%%%%%%%%%%%%%%%%%%%%%%%%%%%%%%%%%%%%%%%%%%%%%%%%%%%%%%%%%%%%

\chapter{Листок 2: Топология метрических пространств.}

%%%%%%%%%%%%%%%%%%%%%%%%%%%%%%%%%%%%%%%%%%%%%%%%%%%%%%%%%%%%%%%%%%%%%%%%

\begin{opredelenie} Пусть $M$ --- метрическое пространство, $X\subset M$
подмножество. Подмножество $X$ называется {\bf открытым}, если оно
вместе с каждой точкой содержит некоторый $\epsilon$-шар с центром в
этой точке, и {\bf замкнутым}, если дополнение к $X$ открыто.
\end{opredelenie}

\begin{zadacha} Докажите, что $X$ открыто тогда и только тогда, когда
для каждой последовательности $\{a_i\}$, которая сходится к $x\in
X$, все $a_i$, кроме конечного числа, содержатся в $X$.
\end{zadacha}

\begin{zadacha} Докажите, что объединение любого количества
открытых множеств открыто. Докажите, что пересечение конечного числа
открытых множеств открыто.
\end{zadacha}

\begin{zadacha} Докажите, что замкнутый шар 
$$ 
\overline B_\epsilon(x) = \{ y \in X \ \ | \ \ d(x,y)\leq \epsilon\}
$$
всегда замкнут. 
\end{zadacha}

\begin{zadacha} Докажите, что множество замкнуто тогда и только
тогда, когда оно содержит все свои предельные точки.
\end{zadacha}

\begin{opredelenie} Замыкание множества $A\subset M$ 
есть объединение $A$ и всех предельных точек $A$.
\end{opredelenie}

\begin{zadacha} 
Дано метрическое пространство, а в нем 
открытый шар $B_\epsilon(x)$ и замкнутый шар
$\overline B_\epsilon(x)$. Всегда ли $\overline B_\epsilon(x)$ --- замыкание
$B_\epsilon(x)$? Докажите, что замыкание любого 
подмножества всегда замкнуто.
\end{zadacha}

\begin{zadacha} \label{_DISKRE_Zadacha_}
Пусть $A$ --- подмножество в $M$, 
не имеющее предельных точек (такое подмножество называется {\bf
дискретным}). Докажите, что $M \backslash A$ открыто.
\end{zadacha}

\begin{opredelenie} Пусть $M$ --- компактное метрическое
пространство, а $\epsilon > 0$ --- число.  Пусть $R\subset M$ таково,
что $M$ покрывается объединением всех $\epsilon$-шаров с центрами в
$R$. Тогда $R$ называется {\bf $\epsilon$-сетью}.
\end{opredelenie}

\begin{zadacha} Пусть каждая последовательность
в $M$ имеет предельную точку. Докажите, что 
для каждого $\epsilon >0$ в $M$ найдется конечная $\epsilon$-сеть.
\end{zadacha}

\begin{ukazanie} 
Пусть такой сети нет; тогда для каждого конечного множества
$R$ найдется точка $x$, отстоящая от $R$ больше, чем на $\epsilon$.
Присоединим $x$ к $R$, воспользуемся индукцией, и
мы получим бесконечное дискретное подмножество $M$.
\end{ukazanie}


\begin{opredelenie} Пусть $X\subset M$ --
подмножество, а $U_i\subset M$ --- набор открытых
подмножеств. Говорят, что $U_i$ --- {\bf покрытие $X$}, если $X
\subset \cup U_i$. Если из $\{U_i\}$ выкинуть какое-то количество
открытых множеств, и оно останется покрытием, то, что получится,
называется {\bf подпокрытие}.
\end{opredelenie}

 \begin{zadacha} \label{_shar_v_pokry_Zadacha_}
Пусть $M$ --- метрическое
  пространство, $S$ --- открытое покрытие $M$.
  Пусть каждая последовательность
  элементов $M$ имеет предельную точку.
  Докажите, что тогда существует такое
  $\epsilon>0$, что любой шар радиуса $<\epsilon$
  полностью содержится в одном из
  множеств покрытия $S$.
\end{zadacha}

\begin{ukazanie} 
Пусть для каждого $\epsilon$ найдется точка $x_\epsilon$,
такая, что соответствующий $\epsilon$-шар не содержится
целиком ни в одном из множеств покрытия. Возьмем
сходящуюся к нулю последовательность $\{\epsilon_i\}$,
и пусть $x$ --- предельная точка последовательности
$\{x_{\epsilon_i}\}$. Докажите, что $x$ не содержится ни в одном из
множеств покрытия $S$.
\end{ukazanie}

\begin{zadacha}[!]\label{comp.defn}
(теорема Гейне-Бореля) 
Пусть $X\subset M$ --- подмножество метрического
пространства. Докажите, что следующие условия равносильны
\begin{enumerate}
\item Каждая последовательность точек из $X$ имеет предельную точку
в $X$.

\item Каждое покрытие $X$ открытыми множествами имеет конечное
подпокрытие.
\end{enumerate}
\end{zadacha}

\begin{ukazanie} Чтобы вывести (а) из (б), воспользуйтесь
задачей \ref{_DISKRE_Zadacha_}. 
Чтобы вывести (б) из (а), возьмем любое 
покрытие $S$, число $\epsilon$ из задачи
\ref{_shar_v_pokry_Zadacha_} и конечную
$\epsilon$-сеть. Каждый из шаров
$\epsilon$-сети содержится в каком-то
из элементов $U_i\in S$. Докажите,
что $\{U_i\}$ --- конечное подпокрытие.
\end{ukazanie}

\begin{opredelenie} Пусть $M$, $M'$ --- метрические пространства,
а $f:\; M \to M'$ --- функция. Функция $f$ называется {\bf
непрерывной}, если $f$ переводит любую последовательность,
сходящуюся к $x$, в последовательность, сходящуюся к $f(x)$,
для каждого $x\in M$.
\end{opredelenie}

\begin{zadacha}[!] Пусть $X$ --- любое подмножество в $M$.
Докажите, что функция $f:\; M \to \R$, $x \overset f \mapsto
d(\{x\}, X)$, непрерывна, где $d(\{x\}, X)$ (расстояние
от $x$ до $X$) определяется как
$d(\{x\}, X):=\inf_{x'\in X}d(x, x')$.
\end{zadacha}

\begin{opredelenie} Пусть $M$ --- метрическое пространство, $X\subset M$
-- подмножество. Говорят, что подмножество $X$ {\bf компакт}, или
{\bf компактное множество}, если выполнено любое из условий
задачи~\ref{comp.defn}. Заметим, что это условия не зависит
от вложения $X\hookrightarrow M$, а зависит только от метрики 
на $X$.
\end{opredelenie}

\begin{zadacha}[!] Рассмотрим пополнение $\Z$ относительно
нормы $\nu_p$, определенное выше (оно называется ``кольцо целых
$p$-адических чисел'' и обозначается $\Z_p$). Докажите, что оно
компактно.
\end{zadacha}

\begin{ukazanie} Докажите, что любое $p$-адическое число
можно представить в виде $\sum a_i p^i$, где $a_i$ целое число от 0
до $p-1$.
\end{ukazanie}

\begin{zadacha} Докажите, что компактное 
подмножество $M$ всегда замкнуто.
\end{zadacha}

\begin{ukazanie} Докажите, что оно содержит все свои предельные
точки.
\end{ukazanie}

\begin{zadacha} Докажите, что замкнутое подмножество компакта всегда
компактно.
\end{zadacha}

\begin{zadacha} Докажите, что объединение конечного числа компактных
подмножеств компактно.
\end{zadacha}

\begin{zadacha}[!] Пусть $f:\; X \to \R$ --- непрерывная функция
на компакте. Докажите, что $f$ достигает максимума.
\end{zadacha}


%%%%%%%%%%%%%%%%%%%%%%%%%%%%%%%%%%%%%%%%%%%%%%%%%%%%%
\subsection*{Липшицевы функции}
%%%%%%%%%%%%%%%%%%%%%%%%%%%%%%%%%%%%%%%%%%%%%%%%%%%%%

\определение
Пусть $(M_1, d_1)$ и $(M_2, d_2)$ - метрические
пространства, а $C>0$ - вещественное число. 
Отображение $f:\; M_1 \arrow M_2$ называется {\bf 
$C$-липшицевым}, если для любых
$x, y\in M_1$, 
\[
d_2(f(x),f(y)) \leq C d_1 (x, y).
\]
Функция $M \arrow \R$ на метрическом пространстве
называется $C$-липшицевой, если соответствующее
отображение $C$-липшицево относительно естественной
метрики на $M$ и $\R$.
\ео

\задача
Докажите, что расстояние $d_z(x) := d(z,x)$ 
до фиксированной точки $z\in M$ - 1-липшицева функция.
\ез

\задача
Докажите, что липшицевы функции непрерывны.
\ез

\определение
Пусть $f_i:\; M \arrow \R$ -- последовательность
функций, таких, что $\lim_i f_i(x)=f(x)$ для какой-то
функции $f$. В таком случае говорится, что
{\бф $f_i$ поточечно сходится к $f$}.
\ео

\задача
Постройте последовательность $f_i$ непрерывных функций
на метрическом пространстве,
поточечно сходящуюся к разрывной функции $f$.
\ез

\задача
Пусть $f_i$ -- последовательность $C$-липшицевых функций,
поточечно сходящаяся к $f$. Докажите, что $f$ непрерывна.
\ез


%%%%%%%%%%%%%%%%%%%%%%%%%%%%%%%%%%%%%%%%%%%%%%%%%%%%%
\subsection*{Расстояние между подмножествами метрических пространств}
%%%%%%%%%%%%%%%%%%%%%%%%%%%%%%%%%%%%%%%%%%%%%%%%%%%%%


\определение
Пусть $X\subset M$ -- подмножество метрического пространства,
$y\in M$ точка. Определим $d(y, X):= \inf_{x\in X}d(x,y)$.
Это число называется {\бф расстоянием от $y$ до $X$}.
\ео


\задача
Пусть $X\subset M$ -- замкнутое подмножество,
а $y\notin X$. Докажите, что $d(y,X)>0$.
\ез

\задача
Пусть $X\subset M$, а $X_1$ -- множество всех точек
$y\in M$ таких, что $d(y,X)=0$. Докажите, что
$X_1$ есть замыкание $X$.
\ез

 
\определение
Пусть $X,X'\subset M$ -- подмножества метрического
пространства. Определим $d(X, X'):= \inf_{x\in X}d(x,X')$.
Это число называется {\бф расстоянием от $X$ до $X'$}.
\ео

\задача
Докажите, что $d(X, X')=d(X', X)$.
\ез

\задача
Докажите, что расстояние до множества задает 
1-липшицеву функцию $M\arrow \R$.
\ез


\задача
Пусть $f:\; X \to \R^{>0}$ -- непрерывная, положительная функция
на компакте. Докажите, что существует $\epsilon >0$ такой,
что $f>\epsilon$ на $X$.
\ез

\задача[!]
Пусть $X,X'\subset M$ -- непересекающиеся замкнутые
подмножества в $M$, причем $X$ компактно.
Докажите, что $d(X,X')>0$.
\ез

\указание 
Докажите, что $d(x,X'):\; X \arrow \R$ задает непрерывную,
положительную функцию на $X$, и воспользуйтесь
компактностью $X$, чтобы доказать, что ее минимум положителен.
\еу

\задача
Постройте два непересекающихся, замкнутых подмножества 
$X,X'\subset M$ в метрическом пространстве таких, что
$d(X,X')=0$.
\ез

\задача
Пусть $X,X'\subset M$ -- непересекающиеся компактные
подмножества в метрическом пространстве $M$. Докажите, что
у них есть непересекающиеся, открытые окрестности.
\ез


\begin{zadacha}[!] Пусть $X$, $Y$ --- два компактных подмножества
метрического пространства. Докажите, что в $X, Y$ есть такие точки
$x, y$, что $d(x,y) = d(X,Y)$.
\end{zadacha}

\begin{opredelenie} Подмножество $Z\subset M$ называется
ограниченным, если оно содержится в шаре $B_r(x)$ для каких-то
$r\in\R, x\in M$.
\end{opredelenie}

\begin{zadacha} Пусть $Z\subset M$ компактно.
Докажите, что оно ограниченно.
\end{zadacha}


%%%%%%%%%%%%%%%%%%%%%%%%%%%%%%%%%%%%%%%%%%%%%%%%%%%%%
\subsection*{Расстояние Хаусдорфа}
%%%%%%%%%%%%%%%%%%%%%%%%%%%%%%%%%%%%%%%%%%%%%%%%%%%%%


\begin{opredelenie} Пусть $M$ --- метрическое пространство,
а $X\subset M$ --- его подмножество. Объединение всех открытых
$\epsilon$-шаров с центрами во всех точках $X$ называется {\bf
$\epsilon$-окрестностью} $X$.
\end{opredelenie}

\begin{opredelenie} Пусть $M$ --- метрическое пространство, а $X$ и $Y$
-- ограниченные его подмножества. {\bf Расстояние Хаусдорфа}
$d_{H}(X,Y)$ есть инфимум всех $\epsilon$ таких, что $Y$ содержится
в $\epsilon$-окрестности $X$, а $X$ содержится в
$\epsilon$-окрестности $Y$.
\end{opredelenie}

\begin{zadacha}[!] Докажите, что расстояние Хаусдорфа
задает метрику на множестве $\cal M$ всех замкнутых ограниченных
подмножеств $M$.
\end{zadacha}

\begin{zadacha} Пусть $X$, $Y$ --- ограниченные подмножества $M$,
а $x\in X$. Докажите, что всегда $d_{H}(X, Y) \geq d(x, Y)$.
\end{zadacha}

\begin{zadacha}[*] Пусть $M$ --- полное метрическое
пространство. Докажите, что $\cal M$ тоже полно.
\end{zadacha}

\begin{ukazanie} Рассмотрим последовательность Коши
$\{ X_i\}$ подмножеств $M$. Пусть $\mathfrak S$ --- множество всех
последовательностей Коши $\{x_i\}$ с $x_i \in X_i$. Пусть $X$
множество предельных точек последовательностей из $\mathfrak
S$. Докажите, что $\{X_i\}$ сходится к $X$.
\end{ukazanie}

\begin{zadacha}[*] Пусть $\{ X_i\}$ --- последовательность Коши
компактных подмножеств в полном метрическом
пространстве $M$, а $X$ --- ее предел. Докажите, что $X$
компактен.
\end{zadacha}

\begin{ukazanie} Перейдя к подпоследовательности в $\{X_i\}$,
можно предположить, что $d_H(X_i, X_j)< 2^{-\min(i,j)}$.  Пусть
$\{x_i\}$ --- последовательность точек из $X$. Для каждого $X_j$
найдите такую последовательность $\{x_i(j)\in X_j\}$, что $d(x_i(j),
x_i)= d(x_i, X_j)$. Поскольку $X_j$ компактен, эта
последовательность всегда имеет предельную точку.  Выберем в
$\{x_i(0)\}$ предельную точку $x(0)$, и заменим $\{x_i\}$ на такую
его подпоследовательность, что $\{x_i(0)\}$ сходится к $x(0)$.
Потом заменим $\{x_i\}, i>0$ на такую подпоследовательность, чтобы
$\{x_i(1)\}$ сходилось к $x(1)$. На $k$-м шаге мы заменяем $\{x_i\},
i>k$ на подпоследовательность таким образом, чтобы $\{x_i(k)\}$
сходилось к $x(k)$. Докажите, что в результате получится такая
последовательность $\{x_i\}$, что $\{x_i(k)\}$ сходится к $x(k)$ для
всех $k$. Докажите, что эту операцию можно провести таким образом,
что $d(x_i(k), x(k)) < 2^{-i}$. Используя приведенную выше оценку
$d_H(X_i, X_j)< 2^{-\min(i,j)}$, докажите, что $d(x_i(k), x_i)<
2^{-\min(k,j)+2}$. Выведите из этого, что $\{x_i\}$ --
последовательность Коши.
\end{ukazanie}

\begin{zadacha}[!] $M$ компактно, $X\subset M$ --
любое подмножество. Докажите, что для каждого $\epsilon >0$ в $M$
найдется конечное множество $R$ такое, что $d_H(R, X)<\epsilon$.
(Это утверждение можно выразить так: ``$X$ допускает аппроксимацию
конечными множествами, с заданной наперед точностью'')
\end{zadacha}

\begin{ukazanie} Найдите в $X$ конечную $\epsilon$-сеть.
\end{ukazanie}

\begin{zadacha}[*] Пусть $M$ компактно. Докажите, что 
$\cal M$ тоже компактно.
\end{zadacha}

\begin{ukazanie} Воспользуйтесь предыдущей задачей.
\end{ukazanie}



%%%%%%%%%%%%%%%%%%%%%%%%%%%%%%%%%%%%%%%%%%%%%%%%%%%%%
\subs{Локально компактные метрические пространства}
%%%%%%%%%%%%%%%%%%%%%%%%%%%%%%%%%%%%%%%%%%%%%%%%%%%%%


\begin{opredelenie} Пусть $M$ --- метрическое пространство.
Говорят, что $M$ {\bf локально компактно}, если для любой точки
$x\in M$ существует такое число $\epsilon>0$, что замкнутый шар
$\overline{B}_\epsilon(x)$ компактен.
\end{opredelenie}


\задача
Докажите, что любое компактное метрическое пространство локально компактно.
\ез

\задача
Докажите, что $\R^n$ с обычной топологией локально компактно.
\ез

\задача[*]
Приведите пример полного, не локально компактного метрического
пространства.
\ез


\begin{zadacha}  Пусть $M$ --- локально компактное
метрическое пространство, $\overline{B}_\epsilon(x)$ --- замкнутый
шар, который компактен. Докажите, что $\overline{B}_\epsilon(x)$
содержится в открытом множестве $Z$, замыкание которого компактно.
\end{zadacha}

\begin{ukazanie} Покройте $\overline B_\epsilon(x)$ шарами,
замыкание которых компактно, и выберите конечное подпокрытие.
\end{ukazanie}

\begin{zadacha}[!] В условиях предыдущей задачи
докажите, что для какого-то $\epsilon'> 0$ замыкание
  открытого шара
${B}_{\epsilon+\epsilon'}(x)$ компактно.
\end{zadacha}

\begin{ukazanie} Возьмите $Z$ такое, как в предыдущей задаче. Возьмите
$\epsilon' = d (M \backslash Z, \overline B_\epsilon(x))$.
\end{ukazanie}

\begin{opredelenie} Пусть $(M, d)$ --- метрическое пространство.
Мы говорим, что $M$ {\bf удовлетворяет условию Хопфа-Ринова}, если
для любых двух точек $x, y\in M$ и таких чисел $r_1, r_2 >0$, что
$r_1+r_2 < d(x,y)$, имеем
$$
d(B_{r_1}(x), B_{r_2}(y)) = d(x,y) -r_1-r_2.
$$
\end{opredelenie}

\begin{zadacha}[*] Пусть $M$ --- полное локально компактное 
метрическое пространство, удовлетворяющее условию Хопфа-Ринова,
$x\in M$ --- точка, а $\epsilon>0$ --- такое число, что $\overline
B_{\epsilon'}(x)$ компактен для всех $\epsilon'<\epsilon$.
Докажите, что шар $\overline B_{\epsilon}(x)$ компактен.
\end{zadacha}

\begin{ukazanie} Пусть $\epsilon_i<\epsilon$ --- последовательность,
которая сходится к $\epsilon$. Пользуясь условием Хопфа-Ринова,
докажите, что $\{\overline B_{\epsilon_i}(x)\}$ --- последовательность
Коши в смысле метрики Хаусдорфа, и сходится к $\overline
B_{\epsilon}(x)$.  Воспользуйтесь тем, что, как доказано выше,
предел такой последовательности компактен.
\end{ukazanie}

\begin{zadacha}[*] \label{_Hopf_Rinow_1_Zadacha_}
(Теорема Хопфа-Ринова, I)
Пусть $M$ --- полное локально компактное метрическое пространство,
удовлетворяющее условию Хопфа-Ринова. Докажите, что каждый замкнутый
шар $\overline B_{\epsilon}(x)$ в $M$ компактен.
\end{zadacha}


\begin{zadacha}[*]
Придумайте пример полного 
локально компактного метрическое пространства, в котором
есть некомпактный замкнутый шар $\overline B_{\epsilon}(x)$.
\end{zadacha}

\замечание
Разумеется, такое пространство не может 
удовлетворять условию Хопфа-Ринова (Задача \ref{_Hopf_Rinow_1_Zadacha_}).
\еза


\begin{zadacha} Пусть $M$ --- такое метрическое пространство, что
каждый замкнутый шар $\overline B_{\epsilon}(x)$ в $M$
компактен. Докажите, что $M$ полно.
\end{zadacha}

\begin{zadacha}[*] Пусть $M$ --- локально компактное
полное метрическое пространство, удовлетворяющее условию
Хопфа-Ринова, $x, y\in M$. Предположим, что
> все замкнутые шары в $M$ компактны. Докажите, что есть такая точка $z\in M$,
что $d(x,z) = d (y, z)= \frac 1 2 d(x,y)$.
\end{zadacha}

\begin{zadacha}[*] Пусть $S$ --- множество всех рациональных 
чисел вида $\frac{n}{2^k}$, $n\in \Z$ на отрезке $[0,1]$. В условиях
предыдущей задачи, докажите, что существует такое отображение
$S\overset \xi\to M$, что $d(\xi(a), \xi(b)) = |a-b| d(x,y)$,
причем $\xi(0)=x$, а $\xi(1)=y$.
\end{zadacha}

\begin{zadacha}[*] (Теорема Хопфа-Ринова, II)
Пусть $M$ --- локально компактное полное метрическое пространство,
удовлетворяющее условию Хопфа-Ринова, $x, y\in M$. Докажите, что
отображение $\xi$ можно естественно продолжить на пополнение $S$
относительно стандартной метрики, получив такое отображение
$[0,1]\overset {\overline\xi}\to M$, что $\overline\xi(0)=x$,
$\overline\xi(1)=y$, и для всякой пары вещественных числа $a,b\in
[0,1]$ имеем $d((\overline\xi(a), \overline\xi(b)) = |a-b| d(x,y)$.
\end{zadacha}

\begin{zamechanie} Такое отображение называется {\bf
геодезическим}. Теорему Хопфа-Ринова можно сформулировать так --- для
любых двух точек в полном метрическом локально компактном
пространстве, удовлетворяющим условию Хопфа-Ринова, найдется
геодезическая, которая их соединяет.
\end{zamechanie}

\begin{opredelenie} 
Такое пространство называется {\bf геодезически связным}
\end{opredelenie}

\begin{zadacha}[*] Приведите пример метрического пространства,
которое не локально компактно, но тем не менее геодезически связно.
\end{zadacha}

\begin{zadacha} Пусть $V= \R^n$ --- векторное пространство со стандартной 
(евклидовой) метрикой. Докажите, что геодезические в $V$ --- это отрезки
(множества вида $a x + (1-a) y$, где $a$ пробегает отрезок $[0, 1]
\subset \R$, a $x, y \in V$).
\end{zadacha}

\begin{zadacha}[*] Пусть $d$ --- метрика на $\R^n$, ассоциированная
с нормой $(x_1, x_2, ... ) \mapsto \max |x_i|$. Докажите, что она
удовлетворяет условию Хопфа-Ринова. Докажите, что $\R^n$ с такой
метрикой геодезически связно. Опишите геодезические.
\end{zadacha}

\begin{zadacha}[*] Пусть $d$ --- метрика на $\R^n$, ассоциированная
с нормой $(x_1, x_2, ... ) \mapsto \sum |x_i|$. Докажите, что она
удовлетворяет условию Хопфа-Ринова. Докажите, что $\R^n$ с такой
метрикой геодезически связно. Опишите геодезические.
\end{zadacha}

\begin{zadacha}[*] Верно ли, что метрика $d$, определенная 
нормой, всегда удовлетворяет условию Хопфа-Ринова?
\end{zadacha}

\begin{opredelenie} Пусть $X$ --- метрическое пространство,
а $0<k<1$ --- вещественное число. Отображение $f:\; X \to X$
называется {\bf сжимающим с коэффициентом $k$}, если $k d(x,y) \geq 
d(f(x), f(y))$.
\end{opredelenie}

\begin{zadacha}[!] Пусть $X$ --- метрическое пространство,
а $f:\; X \to X$ --- сжимающее отображение. Докажите, что для каждого
$x \in X$ последовательность $\{a_i\}$, $a_0 :=x, a_1:=f(x),
a_2:=f(f(x)), a_3:=f(f(f(x))), ...$ --- последовательность Коши.
\end{zadacha}

\begin{ukazanie} Воспользуйтесь тем, что $d(a_i, a_{i+1}) = k^i d(x, f(x))$,
и выведите из этого сходимость ряда $\sum d(a_i, a_{i+1})$
\end{ukazanie}

\begin{zadacha}[!] (Теорема о сжимающих отображениях)
Пусть $X$ --- полное метрическое пространство, а $f:\; X \to X$ --
сжимающее отображение. Докажите, что $f$ имеет неподвижную точку
\end{zadacha}

\begin{ukazanie} Возьмите предел последовательности 
\[ x, f(x), f(f(x)), f(f(f(x))), \ldots.\]
\end{ukazanie}

%%%%%%%%%%%%%%%%%%%%%%%%%%%%%%%%%%%%%%%%%%%%%%%%%%%%%%%%%%%%

\chapter[Листок 3: Теоретико-\-множественная топология]{Листок 3: \\ Теоретико-\-множественная топология.}

%%%%%%%%%%%%%%%%%%%%%%%%%%%%%%%%%%%%%%%%%%%%%%%%%%%%%%%%%%%%

\begin{opredelenie}
Пусть дано пространство $M$, и выделен набор подмножеств $S\subset
M$, называемых {\bf открытыми подмножествами}. Пара $(M,S)$ 
(а также само $M$) называется
{\bf топологическим пространством}, если выполнены следующие
условия.
\begin{enumerate}
\renewcommand{\labelenumi}{\arabic{enumi}.}
\item Пустое множество и само $M$ открыты.

\item Объединение любого числа открытых подмножеств открыто.

\item Пересечение конечного числа открытых подмножеств открыто.
\end{enumerate}
Отображение $\phi:\; M \arrow M'$ топологических пространств
называется {\bf непрерывным}, если прообраз каждого открытого
множества открыт. Непрерывные отображения также называются {\bf
морфизмами} топологических пространств. {\bf Изоморфизм}
топологических пространств --- это такой морфизм $\phi:\; M
\arrow M'$, что существует морфизм $\psi:\; M' \arrow M$, обратный к
$\phi$ (т.е. $\phi\circ \psi $ и $\psi\circ \phi $ --- тождественные
морфизмы). Изоморфизм топологических пространств традиционно
называется {\bf гомеоморфизмом}.

Подмножество $Z\subset M$ называется {\bf замкнутым}, если его
дополнение открыто. {\bf Окрестность} точки $x\in M$ --- это любое
открытое подмножество $M$, которое ее содержит. {\bf Окрестность}
подмножества $Z\subset M$ --- это любое открытое подмножество $M$,
которое его содержит.
\end{opredelenie}

\begin{zadacha}
Докажите, что композиция непрерывных отображений
непрерывна.
\end{zadacha}

\begin{zadacha}[!]
Пусть $M$ --- некоторое множество, а $S$ --- множество всех
подмножеств $M$. Докажите, что $S$ задает на $M$ топологию. Эта
топология называется {\bf дискретной}. Опишите множество всех
непрерывных отображений из $M$ в заданное топологическое
пространство.
\end{zadacha}

\begin{zadacha}[!]
Пусть $M$ --- некоторое множество, а $S$ --- множество из двух
подмножеств $M$: пустого множества и самого $M$. Докажите, что $S$
задает на $M$ топологию.  Эта топология называется {\bf
кодискретной}. Опишите множество всех непрерывных отображений из $M$
в пространство с дискретной топологией.
\end{zadacha}

\begin{zadacha}
Постройте непрерывную биекцию топологических пространств, которая не
является гомеоморфизмом.
\end{zadacha}

\begin{zadacha} 
Дано подмножество $Z$ топологического пространства $M$.
Открытые подмножества в $Z$ задаются пересечениями вида
$Z\cap U$, где $U$ открыто в $Z$. 
Докажите, что это задает топологию на $Z$. Докажите, что
естественное вложение $Z\hookrightarrow M$ непрерывно.
\end{zadacha}

\begin{opredelenie}
Такая топология на $Z\subset M$ называется {\bf индуцированной с
$M$}. Подмножество любого топологического пространства мы будем
рассматривать как топологическое пространство с индуцированной
топологией.
\end{opredelenie}

\begin{opredelenie}
Пусть $M$ --- топологическое пространство, а $S_0$ --- такой набор
открытых множеств, что любое открытое множество можно получить как
объединение множеств из $S_0$. Тогда $S_0$ называется {\bf базой}
$M$.
\end{opredelenie}

\begin{zadacha}
Опишите все базы для $M$ с дискретной топологией; для $M$ с
кодискретной топологией.
\end{zadacha}

\begin{opredelenie}
Пусть $M$ --- метрическое пространство.  Напомним, что подмножество
$U\subset M$ называется {\bf открытым}, если для каждой точки $u\in
U$, $U$ содержит шар радиуса $\epsilon >0$ с центром в $u$.
\end{opredelenie}

\begin{zadacha} 
Докажите, что это определение задает топологию на метрическом
пространстве.
\end{zadacha}

\begin{opredelenie}
Топологическое пространство называется {\bf метризуемым} если 
его можно получить из метрического пространства вышеописанным способом.
\end{opredelenie}

\begin{zadacha} 
Докажите, что дискретное топологическое пространство метризуемо, а
кодискретное --- нет.
\end{zadacha}

\begin{zadacha} 
Докажите, что открытые шары в метрическом пространстве $M$ открыты.
Докажите, что открытые шары задают базу топологии на $M$.
\end{zadacha}

\begin{zadacha}[!]
Пусть $M$ --- топологическое пространство, а $S$, $S'$ --- две
топологии на $M$.  Предположим, что для каждой точки $m\in M$ и
окрестности $U'\ni m$, открытой в топологии $S'$, найдется окрестность
$U\ni m$, $U\subset U'$, открытая в топологии $S$. Докажите, что
тождественное отображение $(M, S) \overset{i}{\arrow} (M, S')$
непрерывно. Приведите пример, когда  $i$ не является
гомеоморфизмом.
\end{zadacha}

\begin{zamechanie}
В такой ситуации иногда говорится, что топология, заданная
$S'$, {\bf сильнее} топологии, заданной $S$. 
\end{zamechanie}

\begin{zadacha}  
Рассмотрим пространство $\R^n$ с нормой $\nu$, как в
листке 1. Эта норма задает метрику, а следовательно, и топологию
на $\R^n$. Обозначим эту топологию через $S_\nu$. Предположим, что
$\nu$, $\nu'$ --- такие две нормы, что для какой-то фиксированной
константы $C\in \R$ всегда имеем $C^{-1} \nu'(x) <\nu(x)< C
\nu'(x)$. Докажите, что тождественное отображение из $\R^n$ в себя
задает гомеоморфизм $(\R^n, S_\nu)\arrow (\R^n, S_{\nu'})$.
\end{zadacha}

\begin{ukazanie}
Воспользуйтесь предыдущей задачей.
\end{ukazanie}

\begin{zadacha}[*]
Предположим, что $\nu$, $\nu'$ --- такие две нормы на $\R^n$, что
тождественное отображение из $\R^n$ в себя задает гомеоморфизм
$(\R^n, S_\nu)\arrow (\R^n, S_{\nu'})$. Докажите, что найдется такая
константа $C$, что $C^{-1} \nu'(x) <\nu(x)< C \nu'(x)$.
\end{zadacha}

\begin{zadacha}[*]
Пусть $V$ --- конечномерное 
векторное пространство, наделенное положительно
определенной билинейной формой $g$. Рассмотрим $V$ как метрическое
пространство, с метрикой $d_g$, построенной в листке 1. 
Обозначим соответствующую топологию через $S_g$. Докажите, что
топология на $V$ не зависит от выбора $g$, то есть что для любых $g,
g'$, тождественное отображение из $V$ в себя задает гомеоморфизм
$(V, S_g)\arrow (V, S_{g'})$.
\end{zadacha}

\begin{zadacha}[**]
Пусть $V$ --- конечномерное пространство с нормой $\nu$.
Докажите, что топология $S_\nu$ не зависит от выбора нормы $\nu$:
тождественное отображение из $\R^n$ в себя всегда задает
гомеоморфизм $(\R^n, S_\nu)\arrow (\R^n, S_{\nu'})$.
Верно ли это, когда $V$ бесконечномерно?
\end{zadacha}

\begin{opredelenie}
Рассмотрим метрику $d$ на $\R^n$, заданную
нормой 
\[ 
|(\alpha_1, ... \alpha_n)|= \sqrt{\sum_i \alpha_i^2}.
\]
Топология на $\R^n$, связанная с $d$, называется {\bf естественной}.
{\bf Естественная топология} на подмножествах в $\R^n$ --- это
топология, индуцированная с $\R^n$.
\end{opredelenie}

\begin{zadacha} 
Рассмотрим $\R$ с естественной топологией.  Пусть $M$ пространство с
дискретной топологией, $M'$ --- пространство с кодискретной
топологией.  Найдите множество всех непрерывных отображений
\begin{enumerate}
\итем Из $\R$ в $M$

\итем Из $M$ в $\R$

\итем Из $M'$ в $\R$

\итем Из $\R$ в $M'$.
\end{enumerate}
\end{zadacha}

\begin{zadacha} 
Пусть $\phi:\; M \arrow M'$ --- некоторое отображение топологических
пространств.  Верно ли, что если $\phi$ непрерывно, то прообраз
любого замкнутого множества замкнут?  Верно ли, что если прообраз
любого замкнутого множества замкнут, то отображение непрерывно?
\end{zadacha}

\begin{zadacha}
Приведите пример такого непрерывного отображение топологических
пространств, что образ открытого множества не открыт. Приведите
пример такого непрерывного отображение топологических пространств,
что образ замкнутого множества не замкнут.
\end{zadacha}

\begin{opredelenie}
Пусть $M$ --- топологическое пространство, $Z\subset M$ --
произвольное подмножество, $\overline{Z}$ --- пересечение всех
замкнутых подмножеств $M$, содержащих $Z$.  Тогда $\overline{Z}$
называется {\bf замыканием} $Z$.
\end{opredelenie}

\begin{zadacha}
Докажите, что $\overline Z$ замкнуто.
\end{zadacha}

\begin{opredelenie}
Пусть $M$ --- топологическое пространство. Следующие условия Т0-Т4
называются {\bf условиями отделимости.}
\begin{enumerate}
\renewcommand{\labelenumi}{{\bf T\arabic{enumi}.}}
\setcounter{enumi}{-1}
\item Пусть даны любые две несовпадающие
точки $x,y \in M$, тогда по крайней мере
  у одной из них есть окрестность, которая не содержит другую.

\item Любая точка $M$ замкнута.

\item 
Любые две различные точки $x$, $y\in M$ обладают окрестностями
$U_x$, $U_y$, которые не пересекаются.

\item В $M$ верно Т1. Кроме того,
для любой точки $y\in M$, любая окрестность $U\ni y$ содержит
открытую окрестность $U'\ni y$, замыкание которой содержится в $U$.

\item  В $M$ верно Т1. Кроме того,
для любого замкнутого подмножества $Z\subset M$, любая окрестность
$U\supset Z$ содержит открытую окрестность $U'\supset Z$, замыкание
которой содержится в $U$.
\end{enumerate}
Условие Т2 известно как {\bf аксиома Хаусдорфа}. Топологическое
пространство, удовлетворяющее условию Т2, называется {\bf
хаусдорфовым}.
\end{opredelenie}

\begin{zadacha}
Докажите, что условие Т1 эквивалентно следующему: для любых двух
несовпадающих точек $x, y\in M$, найдется окрестность $y$, не
содержащая $x$.
\end{zadacha}

\begin{zadacha}
Докажите, что условие Т4 эквивалентно следующему: 
 у любых двух
непересекающихся замкнутых множеств $X, Y\subset M$, найдутся
непересекающиеся окрестности.
\end{zadacha}


\begin{zadacha} Пусть $M$ --- топологическое пространство. Определим
  на $M$ отношение эквивалентности следующим образом: $x$
  эквивалентно $y$ тогда и только тогда, когда $x \in
  \overline{\{y\}}$ и $y \in \overline{\{x\}}$. Обозначим множество
  классов эквивалентности через $M'$.
\begin{enumerate}
\итем Проверьте, что это действительно отношение эквивалентности.
Докажите, что $M$ удовлетворяет условию T0 тогда и только тогда,
когда $M=M'$.

\итем Скажем, что подмножество $U \subset M'$ открыто тогда и только
тогда, когда открыт его прообраз при отображении $M \to
M'$. Докажите, что это задает топологию на $M'$. Удовлетворяет ли
она условию T0?

\итем Докажите, что открытые подмножества в $M$ --- это в точности
  прообразы открытых подмножеств в $M'$.

\итем Пусть топология на $M$ кодискретная. Чему равно $M'$?
\end{enumerate}
\end{zadacha}

\begin{zadacha}
Выполняются ли условия Т0-Т4 в пространстве с дискретной топологией?
С кодискретной?
\end{zadacha}

\begin{zadacha}
Докажите, что условия Т0-Т4 выполняются в $\R$.
\end{zadacha}

\begin{zadacha}
Докажите, что условие T0 следует из T1, a Т1 следует из Т2.
\end{zadacha}

\begin{zadacha} 
Приведите пример пространства, не удовлетворяющего условию
Т1. Приведите пример нехаусдорфова пространства, где все точки
замкнуты.
\end{zadacha}

% \NewVedomost

\begin{zadacha}[*]
Приведите пример пространства, удовлетворяющего Т1, и такого, что
любые два непустых открытых множества пересекаются.
\end{zadacha}

\begin{zadacha}[*]
Докажите, что из Т3 следует Т2.
\end{zadacha}

\begin{zadacha}[*]
Приведите пример пространства, где выполняется Т3, но не выполняется
Т4.
\end{zadacha}

\begin{zadacha} 
Дано метризуемое топологическое пространство. Докажите, что в нем
выполнены условия Т1, Т2, Т3.
\end{zadacha}

\begin{zadacha}[*]
Дано метризуемое топологическое пространство. Докажите, что в нем
выполнено условие Т4.
\end{zadacha}

\begin{zadacha}[*]
Пусть множество $M$ конечно.
\begin{enumerate}
\итем Найдите все топологии на $M$, удовлетворяющие условию T1.
\итем Бывают ли на $M$ топологии, которые не удовлетворяют T1, но
  удовлетворяют T0?
\итем[**] Пусть $M$ состоит из $n$ точек.
Сколько разных топологий на $M$? Сколько из них
удовлетворяют T0?
\end{enumerate}
\end{zadacha}

\begin{opredelenie}
Множество $M$ называется {\bf частично упорядоченным}
(по-английски: ``poset'', ``partially ordered set''), если на нем
задано отношение $x \le y$ (``$x$ меньше либо равно $y$) с
такими  свойствами:
\begin{enumerate}
\renewcommand{\labelenumi}{\arabic{enumi}.}
\item Если $x \le y$, а $y \le z$, то $x \le z$.
\item Если $x \le y$ и $y \le x$, то $x=y$.
\end{enumerate}
\end{opredelenie}

\begin{zadacha}[*]
\begin{enumerate}
\итем Пусть $M$ --- частично-упорядоченное множество; будем говорить,
  что подмножество $S \subset M$ открыто, если вместе с любым
  элементом $x \in S$ оно содержит все $y \in M$, для которых $y \le
  x$. Докажите, что это задает топологию на $M$. Когда эта топология
  удовлетворяет свойству T0? а свойству T1?
\итем Пусть $M$ --- конечное множество, и на нем задана топология,
  удовлеворяющая свойству T0. Докажите, что она происходит из
  какого-то частичного порядка на $M$.
\end{enumerate}
\end{zadacha}

\begin{opredelenie}
Пусть $Z\subset M$ --- подмножество в топологическом пространстве.
Подмножество $Z$ называется {\bf плотным}, если $Z$ пересекается с
каждым непустым открытым подмножеством $M$.
\end{opredelenie}

\begin{zadacha}[!]
Докажите, что $Z$ плотно тогда и только тогда, когда замыкание
$\overline{Z}$ есть все $M$.
\end{zadacha}

\begin{zadacha}
Найдите все плотные подмножества в пространстве с дискретной
топологией; с кодискретной топологией.
\end{zadacha}

\begin{zadacha}
Докажите, что $\Q$ плотно в $\R$.
\end{zadacha}


\определение
Подмножество $Z$ в топологическом пространстве $M$ называется {\bf
нигде не плотным}, если для любого открытого подмножества $U\subset
M$, подмножество $Z\cap U$ не плотно в $U$. 
\ео

\задача
Докажите, что замыкание нигде не плотного подмножества
нигде не плотно.
\ез

\begin{zadacha}[!]
Докажите, что $Z$ нигде
не плотно тогда и только тогда, когда $M\backslash \overline{Z}$
плотно в $M$.
\end{zadacha}

\begin{zadacha}[*] 
Постройте континуальное нигде не плотное подмножество в отрезке
$[0, 1]$ с естественной топологией.
\end{zadacha}

\begin{zadacha}[*] 
Постройте континуальное нигде не плотное подмножество в отрезке
$[0, 1]$ с естественной топологией.
\end{zadacha}

\begin{zadacha}
Найдите все нигде не плотные подмножества в пространсте с дискретной
топологией; в пространстве с кодискретной топологией.
\end{zadacha}

\begin{opredelenie}
Пусть $M$ --- топологическое пространство, $x\in M$ --- произвольная
точка. База окрестностей $x$ --- это такой набор $B$ окрестностей
$x$, что любая окрестность $U\ni x$ содержит какую-то окрестность из
$B$.
\end{opredelenie}

\begin{zadacha}
Пусть в топологическом пространстве $M$ задан такой набор открытых
подмножеств $B$, что для любой точки $x\in M$, совокупность всех
$U\in B$, содержащих $x$, образует базу окрестностей $x$.  
Докажите, что $B$ --- база топологии $M$.
\end{zadacha}

\begin{opredelenie}
Пусть $M$ --- топологическое пространство. На $M$ можно наложить два
условия счетности. Если у каждой точки $M$ найдется счетная база
окрестностей, то говорят, что в $M$ {\bf выполняется первая аксиома
счетности}. Если у $M$ найдется счетная база открытых множеств, то
говорят, что для $M$ выполняется {\bf вторая аксиома счетности},
либо что $M$ --- {\bf пространство со счетной базой}. Если в $M$
найдется плотное счетное множество, то говорят, что $M$ {\bf
сепарабельно}.
\end{opredelenie}

\begin{zadacha}
Дано пространство $M$ с дискретной топологией.  Докажите, что в $M$
выполняется первая аксиома счетности.
\end{zadacha}

\begin{zadacha} 
Пусть топологическое пространство $M$ имеет счетную базу.  Докажите,
что оно сепарабельно.
\end{zadacha}

\begin{zadacha}[*]
Пусть метризуемое топологическое пространство $M$ 
сепарабельно. Докажите, что $M$ имеет
счетную базу. 
\end{zadacha}

\begin{zadacha}[!]
Дано метризуемое топологическое пространство.  Докажите, что оно
имеет счетную базу окрестностей в каждой точке.
\end{zadacha}

\begin{zadacha} 
Постройте несепарабельное метризуемое топологическое пространство.
\end{zadacha}

\begin{zadacha}[**]
Приведите пример счетного хаусдорфова пространства без счетной базы.
\end{zadacha}

\section{Топология и сходимость}

Топологические пространства были изобретены как язык, на котором
удобно говорить о непрерывных функциях. В листке 2 мы
определили непрерывную функцию как функцию, сохраняющую пределы
сходящихся последовательностей. К топологии можно подходить с
аксиоматической точки зрения, приведенной выше, либо с точки зрения
геометрической интуиции, определяя топологию на пространстве
посредством задания класса сходящихся последовательностей, а
непрерывные отображения --- как отображения, сохраняющие пределы.

Второй подход к топологии (при всех его очевидных преимуществах)
наталкивается на теоретико-множественные трудности --- если в нашем
пространстве нет счетной базы, приходится пользоваться полностью
упорядоченными несчетными последовательностями. В дальнейшем мы
будем работать в основном в пространствах со счетной базой
окрестностей точки, и в такой ситуации весьма удобно определять
топологию и непрерывность через пределы последовательностей.

\begin{opredelenie} 
Пусть $M$ --- топологическое пространство, $Z\subset M$ --
бесконечное подмножество. Точка $x\in M$ называется {\bf предельной
точкой} для $Z$, если в каждой окрестности $x$ содержится $z\in
Z$. {\bf Пределом} последовательности $\{ x_i\}$ называется такая
точка $x$, что в любой окрестности $x$ содержатся почти все $x_i$.
Последовательность называется {\bf сходящейся}, если у нее есть
предел. 
\end{opredelenie}

\begin{zadacha}
Найдите все сходящиеся последовательности в пространстве с
дискретной топологией; в пространстве с кодискретной топологией.
\end{zadacha}

\begin{zadacha}
Пусть $M$ --- хаусдорфово. Докажите, что у любой последовательности
 есть не более одного предела.
\end{zadacha}

\begin{zadacha}[*]
Верно ли обратное (т.е. вытекает ли хаусдорфовость из единственности
предела)?  А если в $M$ есть счетная база окрестностей точки? 
\end{zadacha}

\begin{zadacha} 
Пусть в $M$ предел любой последовательности единственен. Докажите,
что в $M$ выполнено условие отделимости Т1.
\end{zadacha}

\begin{zadacha}
Пусть задано непрерывное отображение $f:\; M \arrow M'$ и некоторое
подмножество $Z\subset M$. Докажите, что $f$ переводит предельные
точки $Z$ в предельные точки $f(Z)$.  Докажите, что $f$ переводит
пределы в пределы.
\end{zadacha}

\begin{zadacha}[!]
Пусть отображение переводит предельные точки любого 
множества в предельные точки его образа. Докажите, что оно непрерывно.
\end{zadacha}

\begin{zadacha} 
Пусть дано пространство $M$ с счетной базой окрестностей у каждой
точки.  Рассмотрим произвольное подмножество $Z\subset M$.
Докажите, что замыкание $Z$ есть множество пределов всех
последовательностей из $Z$.
\end{zadacha}

\begin{zadacha}[!]\label{lim.seq}
Пусть даны пространства $M$, $M'$ со счетной базой окрестностей у
каждой точки, и отображение $f:\; M \arrow M'$, сохраняющее пределы
последовательностей. Докажите, что $f$ непрерывно.
\end{zadacha}

\begin{ukazanie}
Воспользуйтесь предыдущей задачей.
\end{ukazanie}

\begin{zadacha}[*] 
А что, если в предыдущей задаче не требовать счетной базы
окрестностей точки для $M$? Для $M'$?

\end{zadacha}

%%%%%%%%%%%%%%%%%%%%%%%%%%%%%%%%%%%%%%%%%%%%%%%%

\chapter[Листок 4. Произведение пространств]{Листок 4.\\ Произведение пространств}

%%%%%%%%%%%%%%%%%%%%%%%%%%%%%%%%%%%%%%%%%%%%%%%%


\subsection*{База топологии}

\begin{opredelenie}
Пусть дано топологическое пространство $M$ и набор $B$ из открытых
подмножеств в $M$. Набор $B$ называется {\bf предбазой} для
топологии на $M$, если любое открытое множество можно получить
(возможно, бесконечным) объединением конечных пересечений открытых
подмножеств, принадлежащих $B$, и {\бф базой}, если
любое открытое множество можно получить
как объединение подмножеств. лежащих в $B$.
У топологии на $M$ есть {\бф счетная база},
если есть база топологии, состоящая из 
счетного набора подмножеств.
\end{opredelenie}

\begin{zadacha} Рассмотрим $\R$ с дискретной топологией. Докажите, что
в нем нет счетной предбазы.
\end{zadacha}

\begin{zadacha}[!]\label{count}
Пусть задано топологическое пространство $M$ со счетной
предбазой. Докажите, что у $M$ есть счетная база.
\end{zadacha}

\begin{zadacha}[*]
Дано конечное множество $M$, $|M|=2^n$, с дискретной топологией, а
$B$ --- предбаза в $M$. Докажите, что $|B| \geq n$. Найдите предбазу,
в которой $2n$ элементов.
\end{zadacha}

\begin{zadacha} 
Рассмотрим $\R$ с естественной топологией, и пусть $B$ --- множество
всех интервалов, у которых концы --- конечные двоичные дроби.
Докажите, что это база в топологии $\R$.
\end{zadacha}

\begin{zadacha}
Пусть дан набор подмножеств $B$ в множестве $M$, такой, что
$\cup B=M$.  Рассмотрим все
подмножества, которые можно получить из элементов $B$, 
а также $M$ и $\emptyset$ конечными пересечениями
и произвольными объединениями. Докажите, что получится топология на
$M$.
\end{zadacha}

\begin{opredelenie}
Такая топология называется {\bf топологией, заданной предбазой $B$}.
\end{opredelenie}

\begin{opredelenie}
Пусть $M_1$, $M_2$ --- топологические пространства.  Рассмотрим
топологию $S$ на $M_1 \times M_2$, заданную предбазой из подмножеств
вида $U_1\times M_2$,
$M_1\times U_2$, где $U_1$, $U_2$ открыты в $M_1$, $M_2$. Тогда
$(M_1\times M_2, S)$ называется {\bf произведением $M_1$ и $M_2$}.
\end{opredelenie}

\begin{zadacha}
Докажите, что естественная проекция $M_1 \times M_2\arrow M_1$
непрерывна. Докажите, что множества вида $U_1\times U_2$
задают базу в топологии на $M_1 \times M_2$.
\end{zadacha}

\begin{zadacha}\label{_product_nepre_Zadacha_}
Даны отображения топологических пространств
$X\overset{\gamma_1}{\arrow} M_1$, $X\overset{\gamma_2}{\arrow}
M_2$. Докажите, что они непрерывны тогда и только тогда, когда
произведение
\[
X\overset{\gamma_1\times \gamma_2}{\arrow} M_1\times M_2
\]
непрерывно.
\end{zadacha}

\begin{zadacha} 
Пусть $M_1$, $M_2$ удовлетворяет условию из списка,
приведенного ниже. Докажите, что $M_1\times M_2$ 
удовлетворяет тому же условию.
\begin{enumerate}
\итем Свойство отделимости Т1.

\итем[!] Условие Хаусдорфа (Т2).

\итем Свойство отделимости Т3.

\итем Сепарабельность.

\итем[!] Наличие счетной базы окрестностей у каждой точки.

\итем Наличие счетной базы.
\end{enumerate}
\end{zadacha}

\begin{zadacha}[**]
Верно ли это для аксиомы отделимости Т4? 
\end{zadacha}

\begin{opredelenie}
Отображение \[ x\overset{\Delta}{\arrow} (x,x)\in X\times X\]
называется {\bf диагональным вложением}, его образ --- {\bf
диагональю} в $X\times X$.
\end{opredelenie}

\begin{zadacha}
Докажите, что диагональное вложение является гомеоморфизмом на свой
образ (топология на $\Delta\subset X\times X$ предполагается
индуцированной с $X\times X$).
\end{zadacha}

\begin{ukazanie}
Воспользуйтесь задачей \ref{_product_nepre_Zadacha_}.
\end{ukazanie}

\begin{zadacha}
Докажите, что $X$ удовлетворяет условию Т1 тогда и только тогда,
когда диагональ является пересечением всех открытых множеств, ее
содержащих.
\end{zadacha}

\begin{zadacha}[!]
Докажите, что $X$ хаусдорфово тогда и только тогда, когда диагональ
замкнута в $X\times X$.
\end{zadacha}

\begin{zadacha}
Докажите, что топология на $X$ дискретна 
тогда и только тогда, когда диагональ открыта в $X\times X$.
\end{zadacha}

\begin{zadacha}
Пусть график $\Gamma\subset X\times Y$ отображения топологических
пространств $X\overset{\gamma}{\arrow} Y$ замкнут. Верно ли, что
$\gamma$ непрерывно?
\end{zadacha}

\begin{zadacha}[!]
Пусть $X\overset{\gamma}{\arrow} Y$ --- морфизм топологических
пространств, причем $Y$ хаусдорфово. Докажите, что график $\gamma$
замкнут.
\end{zadacha}

\begin{zadacha} 
Пусть $M_1$, $M_2$ --- метрические пространства, $M=M_1\times M_2$ --
их произведение, а $d$ --- одна из перечисленных ниже функций на
$M\times M$. Докажите, что $d$ задает метрику на $M$.
\begin{enumerate}
\итем $d((m_1, m_2), (m'_1, m'_2)) = d(m_1, m_1') + d(m_2, m_2')$

\итем $d((m_1, m_2), (m'_1, m'_2)) = \max(d(m_1, m_1'), d(m_2, m_2'))$

\итем[!] $d((m_1, m_2), (m'_1, m'_2)) = \sqrt{d(m_1,
m_1')^2+d(m_2, m_2')^2}$
\end{enumerate}
\end{zadacha}

\begin{zadacha}[!] 
Докажите, что три метрические структуры из предыдущей задачи задают
на $M_1\times M_2$ одну и ту же топологию. Докажите, что эта
топология эквивалентна топологии произведения на $M_1\times M_2$,
которое рассматривается как произведение топологических пространств.
\end{zadacha}

%%%%%%%%%%%%%%%%%%%%%%%%%%%%%%%%%%%%%%%%%%%%%%%%
\subs{Тихоновский куб и гильбертов куб}
%%%%%%%%%%%%%%%%%%%%%%%%%%%%%%%%%%%%%%%%%%%%%%%%

\begin{opredelenie}
Пусть $I$ --- некоторый набор индексов (возможно, несчетный), а
$M=X^I$ --- множество отображений из $I$ в фиксированное
топологическое пространство $X$. На $X^I$ можно смотреть как на
множество последовательностей точек $X$, индексированное $I$, либо
как на бесконечное произведение $X$ с собой.  Обозначим через $W(i,
U)\subset X^I$ множество всех отображений $I\arrow X$, переводящих
заданный индекс $i$ в элемент из подмножества $U\subset X$. Зададим
предбазу $B$ топологии на $X^I$ таким образом: $U\in B$, если
$U=W(i, U)$ для какого-то индекса $i\in I$ и какого-то открытого
подмножества $U\subset X$. Такая топология называется {\bf
  слабой}, или {\бф тихоновской}.
\end{opredelenie}

\begin{zadacha}[!]
Дана последовательность точек $\alpha_1, \alpha_2, \dots $ в
$X^I$. Докажите, что она сходится тогда и только тогда, когда
последовательность $\alpha_k(i)$ сходится для каждого индекса $i\in
I$.
\end{zadacha}

\begin{zamechanie}
Утверждение предыдущей задачи часто формулируют так: ``пространство
$X^I$ со слабой топологией есть множество отображений из $I$ в $X$,
с топологией поточечной сходимости''.
\end{zamechanie}

\begin{opredelenie}
Пусть $I$ --- некоторый набор индексов.  Пространство $[0,1]^I$ со
слабой топологией называется {\bf тихоновским кубом}.
\end{opredelenie}

\begin{zadacha} 
Пусть на топологическом пространстве $M$ задан набор непрерывных
функций $\alpha_i:\; M \arrow [0,1]$, проиндексированных набором
индексов $I$.  Докажите, что отображение
\[ 
\prod \alpha_i:\; m \arrow \prod_{i\in I}\alpha_i(m)
\]
в тихоновский куб $[0,1]^I$ непрерывно. 
\end{zadacha}

\begin{zadacha} 
Докажите, что любая точка тихоновского куба замкнута.
\end{zadacha}

\begin{zadacha}[*]
Докажите, что тихоновский куб удовлетворяет условиям Т2 и Т3.
\end{zadacha}

\begin{zadacha}[!]
Дан тихоновский куб $[0,1]^I$, где $I$ счетно.  Докажите, что у него
есть счетная база.
\end{zadacha}

\begin{ukazanie}
Докажите, что совокупность всех $U=W(i, ]a,b[)$ с рациональными
$a,b$ задает счетную предбазу в $[0,1]^I$, и воспользуйтесь
задачей~\ref{count}.
\end{ukazanie}

\begin{zadacha}[**]
 Пусть множество $I$ имеет мощность континуума
 или больше. Верно ли, что тихоновский куб
 $[0,1]^I$ несепарабелен?
\end{zadacha}

\begin{ukazanie}
Пусть задано счетное подмножество $W$ хаусдорфова
пространства. Докажите, что мощность замыкания $W$ 
не больше континуума.
\end{ukazanie}

\begin{zadacha}[!]
Рассмотрим множество $M=[0,1]^\N$ --- множество последовательностей
вещественных чисел в $[0,1]$, индексированных $\N$. Рассмотрим
функцию $d:\; M\times M\arrow \R$,
\[ 
d(\{\alpha_i\}, \{\beta_i\})= \sqrt{\sum i^{-2}|\alpha_i-\beta_i|^2}.
\]
Докажите, что эта функция корректно определена и задает
метрику на $[0,1]^\N$.
\end{zadacha}

\begin{opredelenie}
Метрическое пространство $[0,1]^\N$ с метрикой, построенной выше,
называется {\bf гильбертовым кубом}.
\end{opredelenie}

\begin{zadacha}[!]
Пусть задана последовательность $\{\alpha_i(n)\}$ точек в
$[0,1]^\N$.  Докажите, что она сходится в тихоновской топологии
тогда и только тогда, когда она сходится в топологии гильбертова
куба.
\end{zadacha}

\begin{zadacha}[*] 
Выведите из этого, что тождественное отображение задает гомеоморфизм
гильбертова куба и тихоновского куба.
\end{zadacha}

\begin{zamechanie}
Мы получили, что если множество индексов $I$ счетно, то тихоновский
куб $[0,1]^I$ метризуем.
\end{zamechanie}

\begin{zadacha}[*] 
Пусть множество индексов $I$ несчетно.  Будет ли тихоновский куб
$[0,1]^I$ метризуем?
\end{zadacha}


%%%%%%%%%%%%%%%%%%%%%%%%%%%%%%%%%%%%%%%%%%%%%%%%%%%%%%%%%%%%
\subs{Нормальные топологические пространства}
%%%%%%%%%%%%%%%%%%%%%%%%%%%%%%%%%%%%%%%%%%%%%%%%%%%%%%%%%%%%

\begin{opredelenie}
Пусть даны непересекающиеся замкнутые подмножества $A, B\subset M$
топологического пространства $M$. Непрерывная функция $f:\; M \arrow
[0,1]$ называется {\bf функцией Урысона}, если $f(A)=0, f(B)=1$.
\end{opredelenie}

 \определение
 Напомним, что топологическое пространство 
 {\бф нормально} (удовлетворяет условию отделимости Т4), если оно хаусдорфово, и
 для любых непересекающихся замкнутых подмножеств $A, B\subset
 M$ наидутся непересекающиеся окрестности.
 \ео  
 

\begin{zadacha}
Пусть для любых непересекающихся замкнутых подмножеств $A, B\subset
M$ существует функция Урысона, и верно условие Т1 (все точки замкнуты). 
Докажите, что $M$ нормально.
\end{zadacha}

\задача
Пусть $М$ -- метрическое пространство, $A\subset 
M$ -- замкнутое подмножество, а
$\phi_A(x)= \frac{d(x,A)}{d(x,A)+1}$.
Докажите, что $\phi_A$ непрерывно, принимает значения
в $[0,1[$, и  $\phi_A(z)=0\Leftrightarrow z\in A$.
\ез

\задача
Пусть $f, g$ -- непрерывные функции на топологическом
пространстве $M$. Докажите, что $\max(f,g)$ непрерывно.
\ез

\указание
Докажите, что $f\times g:\; M\arrow \R \times \R$
непрерывно, и функция $\max:\; \R\times \R \arrow \R$
тоже непрерывна. Тогда  $\max(f,g)$ задается
как композиция непрерывных отображений.
\еу

\задача
Пусть $М$ -- метрическое пространство, $A, B\subset
M$ -- непересекающиеся замкнутые подмножества, $\phi_A$,
$\phi_B$ -- функции, определенные выше, а 
$\psi_{AB}:=\frac{\phi_A}{\max(\phi_A,\phi_B)}$.
Докажите, что $0\leq \psi_{AB}\leq 1$, $\psi_{AB}\restrict A=0$,
$\psi_{AB}\restrict B=1$, причем $\psi_{AB}(z)=0\Leftrightarrow z\in A$.
\ез
\задача
В условиях предыдущей задачи, докажите, что 
$\frac 1 2 (\psi_{AB} + (1-\psi_{BA}))$ есть функция Урысона.
\ез


\задача
Докажите, что любое метрическое пространство нормально.
\ез



%%%%%%%%%%%%%%%%%%%%%%%%%%%%%%%%%%%%%%%%%%%%%%%%%%%%%%%%%%%%
\subs{Лемма Урысона и метризация топологических пространств}
%%%%%%%%%%%%%%%%%%%%%%%%%%%%%%%%%%%%%%%%%%%%%%%%%%%%%%%%%%%%

\begin{opredelenie}
Пусть даны непересекающиеся замкнутые подмножества $A, B\subset M$
топологического пространства $M$. Непрерывная функция $f:\; M \arrow
[0,1]$ называется {\bf функцией Урысона}, если $f(A)=0, f(B)=1$.
\end{opredelenie}


\begin{zadacha}[*]
Пусть $M$ нормально, а $A, B\subset M$ --
непересекающиеся замкнутые  подмножества. Докажите, что
можно найти последовательность окрестностей 
$U_{p/2^q}\supset A$, индексированную рациональными
числами вида $0<p/2^q<1$, и удовлетворяющую следующим
условиям:
\begin{enumerate}
\renewcommand{\labelenumi}{(\roman{enumi})}
\item для всех $p,q$, $B$ не пересекается с $U_{p/2^q}$.

\item Если $p_1/2^{q_1}< p_2/2^{q_2}$, то замыкание $U_{p_1/2^{q_1}}$
содержится в $U_{p_2/2^{q_2}}$.
\end{enumerate}
\end{zadacha}

\begin{ukazanie}
Воспользуйтесь индукцией.
\end{ukazanie}

\begin{zadacha}[*]
В условиях предыдущей задачи, определим функцию $f:\; M \arrow
[0,1]$ формулой
\[  
f(m) = \sup \left\{ p/2^{q} \ \ | \ \ m\notin U_{p/2^q}\right\}
\]
вне $A$ и положим $f$ равной нулю на $A$.
Докажите, что $f$ непрерывна и является функцией Урысона.
\end{zadacha}

\begin{ukazanie} 
Докажите, что отрезки вида $]p_1/2^{q_1}, p_2/2^{q_2}[$ задают предбазу
топологии в $[0,1]$. Докажите, что
\[ 
   f^{-1}(]p_1/2^{q_1}, p_2/2^{q_2}[) =
   U_{p_2/2^{q_2}}\backslash \overline{U_{p_1/2^{q_1}}}.
\]
Выведите из этого, что $f$ непрерывна.
\end{ukazanie}

\begin{zamechanie}
Мы получили следующую ``лемму Урысона'': если $M$ нормально,
то для любых двух непересекающихся замкнутых подмножеств
$M$ существует функция Урысона.
\end{zamechanie}

\begin{zadacha}[*]
Пусть $M$ --- хаусдорфово пространство со счетной базой $B$,
удовлетворяющее условию Т4, $I$ --- множество всех пар $U_1\subset U_2 \in
B$, таких, что $\overline U_1 \subset U_2$, а $F_{U_1,
U_2}$ --- функции Урысона, соответствующие непересекающимся
замкнутым множествам $\overline U_1$ и $M \backslash U_2$, 
а $F:\; M \arrow [0,1]^I$ --- отображение в тихоновский 
куб, заданное как $F(m) = \prod_{U_1, U_2 \in I} F_{U_1,U_2}(m)$.  
Докажите, что $F$ непрерывно и инъективно.
\end{zadacha}

\begin{zadacha}[*]
В условиях предыдущей задачи, обозначим через $G:\; F(M) \arrow M$
отображение, обратное $F$. Пусть дана последовательность точек
$\{x_i\}\subset M$ такая, что $F_{U_1, U_2}(x_i)$ сходится 
к $y\in F(M)I$ для любой пары
$(U_1, U_2)$ в $I$. Выведите из этого, что 
последовательность $\{x_i\}$ сходится к $x:=F^{-1}(y)$. 
Докажите, что $G$ непрерывно.
\end{zadacha}

\begin{zadacha}[*]
Докажите, что любое хаусдорфово топологическое пространство $M$ с
счетной базой, удовлетворяющее условию Т4  можно реализовать как топологическое
подпространство в гильбертовом кубе.
\end{zadacha}

\begin{zamechanie}
Мы получили следующую {\bf теорему о метризации}.  
Всякое нормальное топологическое пространство со счетной базой метризуемо.
\end{zamechanie}

\begin{zadacha} 
Докажите, что любое подмножество гильбертова куба нормально и
  со счетной базой.
\end{zadacha}

\begin{zadacha}
Любое ли метризуемое пространство --- нормально и со счетной базой?
\end{zadacha}

%%%%%%%%%%%%%%%%%%%%%%%%%%%%%%%%%%%%%%%%%%%%%%%%

\chapter{Листок 5: Компактность}

%%%%%%%%%%%%%%%%%%%%%%%%%%%%%%%%%%%%%%%%%%%%%%%%


\begin{opredelenie} 
Пусть $M$ --- топологическое пространство.
Назовем {\bf покрытием} $M$ любой набор открытых
подмножеств $U_i\subset M$ (возможно, бесконечный,
или даже несчетный), для которого $M= \bigcup U_i$.
Пространство $M$ называется {\bf компактным}, или просто {\bf компактом},
если из каждого открытого покрытия $M$ можно выбрать конечное
подпокрытие. Подмножество $Z \subset M$ топологического пространства
$M$ называется компактным, если оно компактно в индуцированной топологии.
\end{opredelenie}

\begin{zadacha}
Докажите, что отрезок $[0,1]$ компактен.
Когда компактно множество с дискретной топологией?
С кодискретной топологией? 
\end{zadacha}

\begin{zadacha}[*]
Пусть в $M$ задана такая топология: 
открытые множества это дополнения к конечным подмножествам
(такая топология называется {\bf кофинитной}). 
Найдите все компактные подмножества в $M$.
\end{zadacha}

\begin{zadacha}[!]
Пусть $Z$ компактно, a $Z' \subset Z$ замкнуто в $Z$.
Докажите, что $Z'$ тоже компактно. Следует ли из 
компактности подмножества его замкнутость?
\end{zadacha}

\begin{zadacha} 
Пусть топологическое пространство $M$ хаусдорфово, 
$Z$ --- произвольное подмножество
$M$, а $x\notin Z$ --- любая точка.
\begin{enumerate}
\итем  Докажите, что у $Z$
есть такое открытое покрытие $\{U_i\}$, что замыкание
каждого $U_i$ не содержит $x$. 

\итем[*] Приведите пример нехаусдорфова Т1-пространства, где
это не выполнено.
\end{enumerate}
\end{zadacha}

\begin{zadacha}[!]
Пусть $M$ хаусдорфово.
Докажите, что любое компактное подмножество в $M$ замкнуто.
\end{zadacha}

\begin{ukazanie}
Воспользуйтесь предыдущей задачей.
\end{ukazanie}

\begin{zadacha} 
Даны два компактных подмножества хаусдорфова пространства.
Докажите, что у них есть непересекающиеся открытые окрестности.
\end{zadacha}

\begin{zadacha}[!]
Дано компактное хаусдорфово топологическое пространство.
Докажите, что для него выполняется условие отделимости Т4.
\end{zadacha}


\begin{zadacha}[*]
Существует ли компактное, хаусдорфово, неметризуемое топологическое
пространство? 
\end{zadacha}

\begin{opredelenie}
Топологическое пространство называется {\bf локально
компактным}, если у любой точки найдется окрестность,
замыкание которой компактно.
\end{opredelenie}

\begin{zadacha} 
Дано локально компактное хаусдорфово топологическое пространство.
Докажите, что в нем выполнено условие Т3.
\end{zadacha}

\begin{zadacha}[**]
Существует ли локально компактное, хаусдорфово топологическое
пространство, в котором не выполнено первое условие счетности?
\end{zadacha}

\begin{zadacha}[**]
Существует ли счетное, хаусдорфово топологическое
пространство, которое не локально компактно?
\end{zadacha}

\begin{zadacha}
Дано хаусдорфово 
топологическое пространство $X$. Обозначим через $\widehat{X}$
множество $X \bigcup \{\infty\}$ ($X$, к которому добавили еще одну
точку, обозначенную как $\infty$) 
со следующей топологией: $U \subset \widehat{X}$
открыто либо если $\infty \in U$, а дополнение
к $U$ компактно как подмножество $X$, либо если $\infty
\not\in U$, и $U$ открыто как подмножество $X$. 
Докажите, что это действительно топология, и
пространство $\widehat{X}$ компактно.
\end{zadacha}

\begin{opredelenie}
Пространство $\widehat{X}$ называется {\bf одноточечной
    компактификацией} пространства $X$.
\end{opredelenie}

\begin{zadacha}[*]
Всегда ли $\widehat{X}$ хаусдорфово?
\end{zadacha}

\begin{zadacha}
Пусть $X = \R^n$ с естественной топологией. Докажите, что
  $\widehat{X}$ гомеоморфно $n$-мерной сфере.
\end{zadacha}


\begin{zadacha}
\label{_DISKR_Opredelenie_Zadacha_}
Дано хаусдорфово топологическое пространство $M$, и подмножество
$Z$ в нем. Докажите, что следующие условия эквивалентны.
\begin{enumerate}
\renewcommand{\labelenumi}{(\roman{enumi})}
\item У любой точки $z\in Z$ существует окрестность
$U\ni z$, не содержащая других точек из $Z$.

\item $M$ индуцирует на $Z$ дискретную топологию.

\item $Z$ не содержит своих предельных точек.
\end{enumerate}
\end{zadacha}

\begin{opredelenie}
Замкнутое подмножество $Z\subset M$, удовлетворяющее одному из условий
задачи \ref{_DISKR_Opredelenie_Zadacha_}, называется {\bf
дискретным}.
\end{opredelenie}

\begin{zadacha}
\label{_discre_nekompa_Zadacha_}
Пусть у хаусдорфова топологического пространства $Z\subset
  M$ есть бесконечное дискретное подмножество. Докажите, что
$M$ некомпактно.
\end{zadacha}

Пусть дан набор $Z_i$ подмножеств множества $M$. Будем говорить, что
этот набор множеств {\bf монотонный},
если для любых $Z_i$, $Z_j$ из нашего набора $Z_i \subset Z_j$ или
$Z_j \subset Z_i$.

\begin{zadacha}\label{cap.1}
Докажите, что если топологическое пространство $M$ компактно, то 
любой монотонный набор непустых замкнутых
подможеств $Z_i \subset M$ имеет непустое пересечение $\cap_i Z_i$.
\end{zadacha}



\begin{zadacha}\label{diskr}
Пусть $M$ --- хаусдорфово топологическое пространство
со счетной базой.
Докажите, что $M$ компактно тогда и только тогда,
когда у $M$ нет бесконечных дискретных подмножеств.
\end{zadacha}


\begin{ukazanie}
Если $M$ содержит бесконечное
дискретное подмножество, из задачи \ref{_discre_nekompa_Zadacha_}
следует, что $M$ некомпактно. Если, наоборот,
$M$ некомпактно, то у $M$ есть счетное покрытие
$S= \{U_i\}$, такое, что никакое конечное подмножество
$S$ не покрывает $M$. Заменив $U_i$ на объединение
всех $U_j, j\geq i$, можно считать, что
$U_1 \subset U_2\subset U_3 \subset ...$, причем
ни один из $U_i$ не содержит $M$. Взяв дополнения
получаем набор замкнутых подмножеств
$A_1\supset A_2\supset ...$, с нулевым
пересечением. Возьмите в каждом $A_i$
точку, докажите, что получится 
дискретное множество.
\end{ukazanie}


\begin{zadacha}[!]
Пусть $M$ --- хаусдорфово топологическое пространство со счетной
базой. Докажите, что $M$ компактно тогда и только тогда,
когда любая последовательность точек из $M$
имеет предельную точку.
\end{zadacha}

\замечание
Это свойство называется {\бф слабой секвенциальной
  компактностью}.
\еза

\задача[**] Существует ли слабо секвенциально компактное,
 некомпактное хаусдорфово пространство?
\ез 

\begin{zadacha}[*]
Дано топологическое пространство $M$, не обязательно хаусдорфово.
\begin{enumerate}
\item Может ли компактное подмножество $M$ содержать бесконечное
дискретное подмножество?

\item Может ли существовать ли некомпактное подмножество $M$,
не содержащее бесконечных дискретных подмножеств?

\item[**] Пусть $M$ хаусдорфово.
Существует ли некомпактное подмножество $M$,
не содержащее бесконечных дискретных 
подмножеств?

\end{enumerate}
\end{zadacha}

\begin{zadacha}[!]
Пусть $f:\; M \arrow N$ --- непрерывное отображение
топологических пространств.
Докажите, что для любого компактного подмножества
$Z\subset M$, $f(Z)$ всегда компактно.
\end{zadacha}

\begin{zadacha} 
Пусть дано подмножество $Z\subset\R$.
\begin{enumerate}
\item Докажите, что $Z$ компактно тогда
и только тогда, когда оно замкнуто и ограничено
(ограничено --- значит содержится в некотором 
отрезке $[a, b]$).

\item 
Докажите, что $Z$ компактно тогда и только
тогда, когда любое его подмножество имеет супремум
и инфимум в $Z$. 

\end{enumerate}
\end{zadacha}

\begin{zadacha}[!]
Пусть $f:\; M \arrow \R$ --
непрерывное отображение топологических прос\-т\-р\-анств. 
Докажите, что $f$ достигает максимума и минимума
на любом компактном подмножестве $M$.
\end{zadacha}

\begin{zadacha}[*]
Пусть дано некомпактное 
хаусдорфово топологическое пространство
со счетной базой, которое удовлетворяет 
свойству отделимости Т4.
Постройте непрерывную функцию $f:\; M \arrow \R$,
которая не достигает максимума.
\end{zadacha}

\begin{ukazanie}
Воспользуйтесь тем, что образ компакта - компакт.
\end{ukazanie}

\begin{zadacha} 
Пусть $f:\; M \arrow N$ --- непрерывное отображение
топологических пространств, $M$ компактно,
а $N$ хаусдорфово. Докажите, что 
$f$ переводит замкнутые множества
в замкнутые.
\end{zadacha}

\begin{zadacha} 
Пусть $f:\; M \arrow N$ --- непрерывное отображение
топологических пространств, $M$ компактно,
а $N$ хаусдорфово. Предположим, что $f$
взаимно однозначно. Докажите, что $f$ --
гомеоморфизм.
\end{zadacha}

\begin{zadacha} 
Придумайте такое непрерывное взаимно однозначное
отображение топологических пространств 
$f:\; M \arrow N$, что $M$
компактно, но $f$ --- не гомеоморфизм.
($N$ не хаусдорфово).
\end{zadacha}

%%%%%%%%%%%%%%%%%%%%%%%%%%%%%%%%%%%%%%%%%%%%%%%%
\subs{Компакты и произведения}
%%%%%%%%%%%%%%%%%%%%%%%%%%%%%%%%%%%%%%%%%%%%%%%%

\begin{opredelenie}
Непрерывное отображение $f:X \to Y$ топологических пространств называется
{\bf собственным}, если для каждого компактного $K \subset Y$ прообраз
$f^{-1}(K) \subset X$ компактен.
\end{opredelenie}

\begin{zadacha}[!]
Пусть пространство $Y$ хаусдорфово и имеет счетную базу
  окрестностей в точке. Докажите, 
что любое собственное отображение $f:X \to Y$ переводит замкнутые 
подмножества $X$ в замкнутые подмножества $Y$.
\end{zadacha}

\begin{ukazanie}
Пусть есть замкнутое $Z \subset X$, образ которого не замкнут. Выберите
последовательность точек $y_i \in f(Z)$, которая сходится к точке
$y \in Y$, не лежащей в $f(Z)$.
\end{ukazanie}

\begin{zadacha}[**]
Верно ли утверждение предыдущей задачи без предположения счетной базы?
\end{zadacha}


%\NewVedomost


\begin{zadacha}[!]
Пусть $X$, $Y$ --- компактные топологические пространства. Докажите,
что произведение $X \times Y$ компактно.
\end{zadacha}

\begin{ukazanie}
Воспользовавшись тем, что множества вида $U \times V$, $U$ 
открыто в $X$, $V$ открыто в $Y$ задают базу топологии на 
$X \times Y$, докажите сначала, что достаточно рассматривать 
покрытия $X \times Y$ множествами такого вида. Затем для каждой 
точки $y \in Y$ выберите конечное подпокрытие подмножества
$X \times \{y\} \subset X \times Y$, состоящее из каких-то
множеств $U_i \times V_i$, и заметьте, что множества 
$V_y = \cap V_i$ образуют открытое покрытие
пространства $Y$.
\end{ukazanie}

\begin{zadacha}
Дано подмножество $X\subset \R^n$. Докажите, что 
следующие свойства равносильны
\begin{enumerate}
\renewcommand{\labelenumi}{(\roman{enumi})}
\item $X$ компактно

\item $X$ замкнуто и ограниченно (т.е. содержится
в каком-то шаре).
\end{enumerate}
\end{zadacha}

\определение
Непрерывное отображение $f:\; X\arrow Y$
называется {\бф открытым}, если образ любого
открытого множества открыт.
\ео

\задача[*]
Пусть $f:\; X \arrow Y$ --- открытое отображение,
а слои $f^{-1}(y), y\in Y$ компактны. Всегда ли $f$
собственное?
\ез

%%%%%%%%%%%%%%%%%%%%%%%%%%%%%%%%%%%%%%%%%%%%%%%%
\subs{Теорема Тихонова}
%%%%%%%%%%%%%%%%%%%%%%%%%%%%%%%%%%%%%%%%%%%%%%%%

\begin{zadacha}
Пусть дана последовательность $a_i(n)$ отображений из $\N$
в $[0,1]$. Докажите, что можно выбрать
такую подпоследовательность $a_{i_1}, a_{i_2}, a_{i_3}, \dots$, 
что $\{a_{i_k}(n)\}$ сходится для любого $n$.
\end{zadacha}

\begin{zadacha}[!]
Выведите из этого, что тихоновский куб $[0,1]^\N$
компактен.
\end{zadacha}

\begin{zadacha}[*]
Дано топологическое пространство $M$.
Пусть задано такое (возможно, несчетное) множество $\{ V_{\alpha}\}$
покрытий $M$,
что каждое $V_{\alpha}$ 
содержит $V_{\alpha'}$ либо содержится в нем
(иначе говоря, $\{ V_{\alpha}\}$ --- набор
покрытий, получающихся друг из друга присоединением
каких-то элементов). Пусть из каждого $V_{\alpha}$ 
нельзя выбрать конечное подпокрытие.
Докажите, что из объединения всех
$V_{\alpha}$ тоже нельзя выбрать 
конечное подпокрытие.
\end{zadacha}

\begin{zadacha}[*] 
Используя лемму Цорна,
докажите, что у всякого некомпактного подмножества
$X\subset M$ найдется покрытие $\{ V_{\alpha}\}$, 
из которого нельзя выбрать конечного подпокрытия,
а если добавить к $\{ V_{\alpha}\}$ любое не содержащееся
в нем открытое множество, то из полученного покрытия
можно будет выбрать конечное подпокрытие.
\end{zadacha}

\begin{ukazanie}
Воспользуйтесь предыдущей задачей.
\end{ukazanie}

Мы будем называть такие покрытия {\bf максимальными}.

\begin{zadacha}[*]
Пусть дано максимальное покрытие 
$\{V_{\alpha}\}$ некомпактного топологического
  пространства $M$. Докажите, что если
  открытые множества $U_1$, $U_2$ не лежат в 
$\{V_{\alpha}\}$, и их пересечение непусто, то оно
  тоже не лежит в $\{ V_{\alpha}\}$. Докажите, что
  любое непустое конечное пересечение
  открытых множеств, не лежащих в 
$\{V_{\alpha}\}$, тоже не принадлежит $\{ V_{\alpha}\}$.
\end{zadacha}

\begin{ukazanie}
Воспользуйтесь предыдущей задачей.
\end{ukazanie}

\begin{zadacha}[*]
Пусть в топологическом пространстве $M$ задана
предбаза топологии $R$. Пусть дано некомпактное подмножество
$X\subset M$ и максимальное покрытие
$\{ V_{\alpha}\}$. Докажите, что в 
$\{ V_{\alpha}\}$ можно выбрать 
подпокрытие из $R$.
\end{zadacha}

\begin{ukazanie}
Воспользуйтесь предыдущей задачей.
\end{ukazanie}

\begin{zamechanie}
Мы получили следующую теорему
(теорема Александера о предбазе).
Пусть в топологическом пространстве $M$ задана
предбаза топологии $S$. Тогда подмножество
$X\subset M$ компактно тогда и только тогда,
когда из любого покрытия $X$ элементами из
$S$ можно выбрать конечное подпокрытие.
Теорема Александера использует аксиому выбора
и (как показал Дж. Л. Келли) эквивалентна ей.
\end{zamechanie}

\begin{zadacha}[*]
Выведите из этого, что
тихоновский куб $[0,1]^I$
компактен для любого множества $I$.
\end{zadacha}

\begin{ukazanie}
Рассмотрите предбазу для топологии на 
тихоновском кубе, составленную из 
подмножеств вида 
$[0,1]\times [0,1]\times \dots \times ]a,b[ \times [0,1]\times \dots$
(на одном месте стоит открытый интервал).
Воспользуйтесь теоремой Александера.
\end{ukazanie}

\begin{zamechanie} 
Компактность тихоновского куба эквивалентна такому
утверждению. Рассмотрим пространство $\Map(I, [0,1])$  отображений
из множества $I$ в отрезок $[0,1]$, с топологией
поточечной сходимости. Тогда  $\Map(I, [0,1])$ компактно.
В частности, из любой последовательности $\{a_{i}(x)\}$
отображений можно выбрать такую подпоследовательность $\{a_{i_k}(x)\}$, 
что $\{a_{i_k}(x)\}$ сходится в любом $x\in I$.
\end{zamechanie}

\begin{opredelenie}
Пусть $M$ --- топологическое пространство,
$I$ --- некоторое множество, а $M^I$ --- пространство
отображений из $I$ в $M$, то есть произведение $I$ копий
$M$. Для $x\in I$ и открытого множества
$U\subset M$ рассмотрим подмножество
$U(x)\subset M^I$, состоящее из всех
отображений, переводящих $x$ в $U$.  
Определим на $M^I$ топологию с предбазой,
состоящей из всех $U(x)$. Такая топология
называется {\bf тихоновской} (также {\bf слабой}
или {\bf топологией поточечной сходимости}).
\end{opredelenie}

\begin{zadacha}[*]
Пусть $M$ компактно. Выведите из
теоремы Александера, что $M^I$ с 
тихоновской топологией компактно.
\end{zadacha}


%%%%%%%%%%%%%%%%%%%%%%%%%%%%%%%%%%%%%%%%%%%%%%%%
\subs{Основная теорема алгебры}
%%%%%%%%%%%%%%%%%%%%%%%%%%%%%%%%%%%%%%%%%%%%%%%%


Пусть $P(x)= x^n + a_{n-1} x^{n-1} + \dots + a_0 $ --
полином 
положительной степени с комплексными коэффициентами. 
Мы рассматриваем $P$ как функцию из
$\C$ в $\C$. Как топологическое пространство, $\C$
отождествляется с $\R^2$. Мы хотим доказать, что
$P(x)=0$, для какого-то $x\in \C$.

\begin{zadacha}
Докажите, что $P$ непрерывен.
\end{zadacha}

\begin{zadacha}
Докажите, что есть $C$ такое, что для всех $|x| > C$, имеем
$\frac{|P(x)-x^n|}{|x^n|} <1/2$. 
\end{zadacha}

\указание
Возьмите $|x|> 2\max \left(1, \sum |a_i|\right)$.
\еу

\begin{zadacha}
Докажите, что есть $C$ такое, что для всех $|x| > C$
то $|P(x)| > R^n$.
\end{zadacha}

\указание
Возьмите
$|x|> 2R \max \left(1, \sum |a_i|\right)$,
\еу

\begin{zadacha}
Выведите из этого, что $|P|$ достигает 
локального минимума в точке $a\in \C$.
\end{zadacha}

\begin{ukazanie}
Мы приблизили многочлен $|P|$ многочленом $x^n$,
скорость роста которого нам известна.
Из этого мы вывели, что $|P(x)|>  R^n$,
когда $|x|$ достаточно велик.
Поэтому минимум $|P|$ на круге $|x|\leq R$
достигается внутри круга, а не на
его границе.
\end{ukazanie}

Для упрощения обозначений, мы будем в дальнейшем
предполагать, что минимум $|P|$ достигается в нуле.
Мы хотим доказать,
  что минимум $|P|$ равен нулю. Пусть это не
  так. Пусть $k$ --- самое маленькое число
  среди $1, 2, 3, \dots, n$, для которого $a_k\neq 0$.
  Домножив $P$ на $a_0^{-1}$, и сделав замену
  $x=z\sqrt[k]{a_k^{-1}}$, мы получим многочлен вида
 \[ Q(z) = 1 + z^k + b_{k+1} z^{k+1} + b_{k+2} z^{k+2} + \dots \]

\begin{zadacha}
       Докажите, что для любого
       комплексного $z$, такого, что $|z|< 1$, выполняется
       \[
       |Q(z)-1-z^k| < |z^{k+1}|\left (\sum |b_i|\right).
       \]
       \end{zadacha}

       \begin{zadacha}
       Докажите, что существует $\epsilon >0$ такой, что для любого
$z\in \C$, $|z|<\epsilon$, имеем
       \[
       \frac{|Q(z)-1-z^k| }{|z^{k}|}<\frac 1 2.
       \]
       \end{zadacha}

\указание
Возьмите 
$|z|< \frac 1 2 \left(\sum |b_i|\right)^{-1}$
и воспользуйтесь предыдущей задачей.
\еу

       \begin{zadacha}
       Выведите из этого, что для любого положительного
       вещественного $c< \epsilon$ 
       и любого комплексного $z$, для которого $z^k=-c$, выполняется
       \[
       \left|Q(z)-1 +c\right|< c/2.
       \]
       \end{zadacha}

  \begin{zamechanie} В окрестности нуля, мы
  приблизили $Q$ многочленом $1+ z^k$. Пользуясь этим приближением,
мы находим, что
  $|Q(\sqrt[k]{-\epsilon})|<|Q(0)|(1-\frac 1 2 \epsilon)$ для
  достаточно малых $\epsilon$. Следовательно, локальный минимум
многочлена это всегда 0.
\end{zamechanie}

       \begin{zadacha}[!]
       Докажите Основную Теорему Алгебры:
       каждый многочлен $P$ положительной степени имеет
       корень в $\C$.
       \end{zadacha}



%%%%%%%%%%%%%%%%%%%%%%%%%%%%%%%%%%%%%%%%%%%%%%%%


\chapter{Листок 6: Поточечная и равномерная сходимость}

%%%%%%%%%%%%%%%%%%%%%%%%%%%%%%%%%%%%%%%%%%%%%%%%


На протяжении этого листка, разрешается
пользоваться следующей формой теоремы
Тихонова. Пусть $X$ --- компактное топологическое
пространство, $I$ --- произвольное множество,
а $X^I$ --- пространство отображений из $I$ в $X$,
с топологией поточечной сходимости. Тогда $X^I$
компактно. 

\begin{zadacha}
Рассмотрим пространство функций из отрезка в отрезок,
с топологией поточечной сходимости. Докажите, что предел
непрерывных функций может быть разрывен.
\end{zadacha}

\begin{opredelenie}
Пусть $X$, $Y$ --- метрические пространства,
а $\{f_\alpha\}$ --- набор непрерывных отображений
из $X$ в $Y$. Множество $\{f_\alpha\}$ называется
{\bf равномерно непрерывным}, если 
для каждого $\epsilon$ найдется такое $\delta$, 
что образ любого $\delta$-шара
под действием любого $f_\alpha$ содержится
в некотором $\epsilon$-шаре (возможно, зависящим от $f_\alpha$).
\end{opredelenie}

\begin{zadacha}
Пусть дано отображение $f:\; X \arrow Y$ 
метрических пространств,
которое переводит последовательности Коши в
последовательности Коши. Докажите, что 
$f$ непрерывно как отображение топологических
пространств. Всякое ли непрерывное
отображение переводит последовательности Коши в
последовательности Коши?
\end{zadacha}

\begin{zadacha}[!]
Пусть $X$, $Y$ --- метрические пространства,
а $\{f_i\}$ --- равномерно непрерывная
последовательность непрерывных отображений
из $X$ в $Y$. Предположим, что $\{f_i\}$ 
сходится к $f$ в топологии поточечной
сходимости. Докажите, что $f$ непрерывна.
\end{zadacha}

\begin{ukazanie} 
Докажите, что $f$ равномерно непрерывна,
с теми же самыми числами $\epsilon$, $\delta$,
что и $\{f_i\}$, и воспользуйтесь предыдущей задачей.
\end{ukazanie}

Зафиксируем компактные метрические пространства $X$, $Y$, 
и пусть $\Map(X,Y)$ --- множество непрерывных отображений 
из $X$ в $Y$. 

\begin{zadacha} 
Для любых $f, g\in \Map(X,Y)$ определим число $d_{\sup}(f,g)$
как $\sup_{x\in X} d(f(x), g(x))$. Докажите,
что $d_{\sup}(f,g)$ корректно определено и задает
метрику на $\Map(X,Y)$.
\end{zadacha}

\begin{opredelenie}
Эта метрика называется $\sup$-метрикой на
$\Map(X,Y)$.
\end{opredelenie}

\begin{zadacha}[!]
Предположим, что равномерно непрерывная 
последовательность отображений $\{f_i\}\subset \Map(X,Y)$ поточечно 
сходится к $f$. Докажите, что она сходится к $f$
в топологии, заданной $\sup$-метрикой.
\end{zadacha}

\begin{ukazanie}
Пусть $\sup_{x\in X} d(f(x), f_i(x))>C$
для любого $i$. Найдите сходящуюся
последовательность
таких $\{x_i\}$, что $d(f(x_i), f_i(x_i))> C$,
и пусть $x$ --- ее предел.
В силу равномерной непрерывности, 
$d(f_i(x_i), f_i(x))$ стремится к нулю. 
Воспользовавшись неравенством треугольника
\[ 
d(f_i(x), f(x)) + d(f_i(x_i), f_i(x)) \geq d(f(x), f_i(x_i)),
\]
получите противоречие.
\end{ukazanie}

\begin{zadacha}[!] 
(теорема Арцела-Асколи)
Пусть дано замкнутое (в смысле $\sup$-метрики) и
равномерно непрерывное множество отображений
$\Psi\subset\Map(X,Y)$. Докажите, что $\Psi$ компактно.
\end{zadacha}

\begin{ukazanie}
Воспользуйтесь теоремой Тихонова и предыдущей задачей. Как 
и говорилось ранее, предполагается, что $X$ и $Y$ компактны!
\end{ukazanie}

\begin{zadacha}[**]
Придумайте независимое от теоремы Тихонова
(а следовательно, аксиомы выбора) доказательство
теоремы Арцела-Асколи.
\end{zadacha}

\begin{zadacha}[*]
Пусть задано компактное подмножество $K\subset X$
и открытое подмножество $V\subset Y$.
Обозначим через $U(K,V)\subset \Map(X,Y)$ множество
всех отображений переводящих $K$ в $V$.
Рассмотрим топологию на $\Map(X,Y)$,
заданную предбазой из всех $U(K,V)$.
Докажите, что та же самая топология
задается $\sup$-метрикой.
\end{zadacha}

\begin{opredelenie}
Эта топология на  $\Map(X,Y)$ называется 
{\bf компактно-открытой топологией}, или
{\bf топологией равномерной сходимости}.
\end{opredelenie}

\begin{zadacha}
Докажите, что топология поточечной сходимости
слабее топологии равномерной сходимости; другими
словами, что тождественное отображение из $\Map(X,Y)$
с топологией равномерной сходимости в 
$\Map(X,Y)$ с топологией поточечной сходимости
непрерывно.
\end{zadacha}

\begin{opredelenie}
Пусть $Z$ --- подмножество метрического пространства
$M$. {\bf Диаметром} $Z$ называется число 
$\diam(Z):= \sup_{x,y\in Z} d(x,y)$.
\end{opredelenie}

\begin{zadacha} 
Пусть $f \in \Map(X,Y)$ --- непрерывное отображение, $X$ компактно,
$\epsilon$ --- вещественное число, а $\delta(f, \epsilon)$ --
супремум $\diam(f(B))$
по всем $\epsilon$-шарам $B$ в $X$.
Докажите, что $\lim\limits_{\epsilon \arrow 0}\delta(f, \epsilon)=0$.
\end{zadacha}

\begin{ukazanie} 
Пусть задана такая сходящаяся к нулю последовательность
$\epsilon_i$, что какого-то набора точек $x_i\in X$ и
положительной константы $C$ имеем $\diam
f(B_{\epsilon_i}(x_i)))>C$. Рассмотрим предельную точку $x$ 
последовательности $\{x_i\}$. Тогда
в каждом $\epsilon$-шаре вокруг $x$ содержится
$B_{\epsilon_i}(x_i)$ (для достаточно большого $i$),
из чего следует, что образ этого $\epsilon$-шара
имеет диаметр больше $C$. Значит, $f$ не непрерывна.
\end{ukazanie}

\begin{zadacha}[!]
Пусть задано непрерывное отображение $f \in \Map(X,Y)$, и $X$ компактно.
Докажите, что $f$ равномерно непрерывна.
\end{zadacha}

\begin{ukazanie}
Это утверждение тавтологически эквивалентно
\[ \lim\limits_{\epsilon \arrow 0}\delta(f, \epsilon)=0.\]
\end{ukazanie}

\begin{zadacha} 
Пусть дано подмножество $\Psi \subset\Map(X,Y)$.
Докажите, что $\Psi$ равномерно непрерывно
тогда и только тогда, когда 
\[ \lim\limits_{\epsilon\arrow 0}\sup_{f\in\Psi}\delta(f, \epsilon)=0.\]
\end{zadacha}

\begin{zadacha}[*]
Пусть $d_{\sup}(f,g) <\gamma$. Докажите, что
$\delta(f, \epsilon)< \delta(g, \epsilon)+\gamma$.
\end{zadacha}

\begin{zadacha}[*]
Пусть $\{f_i\}$ --- последовательность Коши в $(\Map(X,Y),d_{\sup})$.
Докажите, что она равномерно непрерывна.
\end{zadacha}

\begin{ukazanie}
Нам нужно доказать, что 
\[
\lim\limits_{\epsilon\arrow 0}\sup_{i}\delta(f_i,\epsilon)=0
\]
Воспользoвавшись предыдущей задачей, убедитесь,
что для всех $f_i$, лежащих в каком-то $\gamma$-шаре 
в $(\Map(X,Y),d_{\sup})$, числа $\delta(f_i, \epsilon)$
отличаются не больше, чем на $\gamma$. Выведите из этого,
что $\sup_{i}\delta(f_i,\epsilon) <\delta(f_N,\epsilon)+\gamma$
для фиксированного $N$, а следовательно,
\[
\sup_{i}\delta(f_i,\epsilon)< \gamma + \max_{i\leq N}\delta(f_i,\epsilon)
\]
Предел этого выражения при $\epsilon \arrow 0$ не
больше $\gamma$, поскольку все $f_i$ равномерно непрерывны. 
\end{ukazanie}

\begin{zadacha}[*]
Докажите, что метрическое пространство $(\Map(X,Y),d_{\sup})$ полное.
\end{zadacha}

%\begin{ukazanie}
%Это утверждение непосредственно
%следует из приведенных выше задач.
%\end{ukazanie}
%
% Ну и на фиг такие
% указания? студенты
% в таких случаях
% пишут: читай
% указание, и
% говорят, что вот
% оно решение
%

\begin{zadacha}[*]
Является ли $(\Map(X,Y),d_{\sup})$ локально компактным?
\end{zadacha}

%%%%%%%%%%%%%%%%%%%%%%%%%%%%%%%%%%%%%%%%%%%%%%%%
\subs{Кривая Пеано}
%%%%%%%%%%%%%%%%%%%%%%%%%%%%%%%%%%%%%%%%%%%%%%%%

Пусть дан отрезок $[a,b]$. Отображение
$[a,b] \overset{f}{\arrow} \R^n$ называется {\bf линейным},
если $f (\lambda a + (1-\lambda) b)= \lambda f (a) +(1-\lambda) f(b)$,
для любого $0<\lambda<1$. Отображение называется
{\bf кусочно линейным}, если отрезок разбит
на подотрезки $[a, a_1], [a_1, a_2], [a_2, a_3], \ldots$,
и $f$ линеен на каждом из этих подотрезков.
Образ кусочно линейного отображения это, очевидно,
ломаная. 

Пусть дано кусочно линейное
отображение $f$ из отрезка $[0, 1]$ в квадрат
$[0, 1]\times [0, 1]$ со следующим свойством:
все сегменты ломаной $f([0, 1])$ параллельны
прямой $x=y$ либо прямой $x=-y$.
\begin{center}
\epsfxsize .45\linewidth
\epsfbox{geom8pic0.eps}
\end{center}
Иными словами, для каждого подотрезка 
$[a, a_1]$, на котором $f$ линеен, $f$ отображает
$[a, a_1]$ в диагональ некоторого квадрата $Q$, со сторонами,
параллельными осям координат. Пусть ${\cal P}l$ --
пространство таких кусочно-линейных отображений.

Определим операцию $\mu$, которая делает из
кусочно-линейного отображения $f\in {\cal P}l$ с
$k$ линейными сегментами кусочно-линейное отображение
с $4k$ линейными сегментами:
%$$
%\begin{CD}
%\mbox{\epsfig{file=geom8pic1.eps,width=0.25\linewidth}}
%@>\mu>>
%\epsfig{file=geom8pic2.eps,width=0.25\linewidth}
%@>\mu>>
%\epsfig{file=geom8pic3.eps,width=0.25\linewidth}
%\end{CD}
%$$
\begin{center}
\epsfig{file=geom8pic1.eps,width=0.25\linewidth}
$\overset{\mu}{\longrightarrow}$
\epsfig{file=geom8pic2.eps,width=0.25\linewidth}
$\overset{\mu}{\longrightarrow}$
\epsfig{file=geom8pic3.eps,width=0.25\linewidth}
\end{center}
Определим $\mu(f)$ следующим образом.
\begin{enumerate}
\renewcommand{\labelenumi}{\arabic{enumi}.}
\item Обозначим через $a_0, a_1, \dots, a_k$ 
концы сегментов, на которых функция $f$ линейная.
Тогда $\mu(f)$ отображает $a_i$ в $f(a_i)$.

\item Разобьем каждый из сегментов 
$[a_i, a_{i+1}]$ на четыре равные части:
\[ 
  [b_{4i},b_{4i+1}], [b_{4i+1},b_{4i+2}], 
  [b_{4i+2},b_{4i+3}], [b_{4i+3},b_{4i+4}].
\]
$\mu(f)$ отображает $[b_{4i},b_{4i+1}]$
линейно в $[f(a_i), f\left(\frac{a_i+a_{i+1}}2\right)]$,
а $[b_{4i+3},b_{4i+4}]$ в $[f\left(\frac{a_i+a_{i+1}}2\right), f(a_{i+1})]$.

\item Рассмотрим квадрат, диагональю которого
является отрезок $[f(a_i), f(a_{i+1})]$, и перенумеруем
его вершины по часовой стрелке: $f(a_i), A, f(a_{i+1}), B$.
Тогда $\mu(f)$ отображает $[b_{4i+1},b_{4i+2}]$
линейно в $[f\left(\frac{a_i+a_{i+1}}2\right), B]$,
а $[b_{4i+2},b_{4i+3}]$ в $[B, f\left(\frac{a_i+a_{i+1}}2\right)]$.
\end{enumerate}
Мы получаем такую ломаную:
\begin{center}
\epsfxsize .45\linewidth
\epsfbox{geom8pic4.eps}
\end{center}

\begin{zadacha}
Рассмотрим отрезок и квадрат как метрические
пространства со стандартной метрикой.
Пусть $f\in {\cal P}l$, и самый большой прямолинейный
сегмент $[f(a_i), f(a_{i+1})]$ соответствующей ломаной
имеет длину $k$. Тогда $d_{\sup}(f, \mu(f)) \leq\frac{k}{\sqrt 2}$.
\end{zadacha}

\begin{zadacha} 
Пусть $f\in {\cal P}l$, и самый большой прямолинейный
сегмент $[f(a_i), f(a_{i+1})]$ соответствующей ломаной
имеет длину $k$. Тогда самый большой прямолинейный сегмент
в $\mu(f)$ имеет длину $k/2$.
\end{zadacha}

\begin{zadacha}
Пусть $f_0\in {\cal P}l$, $f_1 = \mu(f_0), \dots, f_n = \mu(f_{n-1})$,
а самый большой прямолинейный сегмент в ломаной, соответствующей
$f_0$, имеет длину $k$. Докажите, что 
\[
d_{\sup}(f_n, f_{n+1}) < \frac k {2^n \sqrt 2}
\]
\end{zadacha}

\begin{zadacha}[!]
Докажите, что $\{ f_i\}$ --- последовательность Коши
в метрике $d_{\sup}$.
\end{zadacha}

\begin{zadacha} 
Пусть $f\in {\cal P}l$, и для всех прямолинейных сегментов
$[a_i, a_{i+1}]$ в $f$, длина \[ [f(a_i), f(a_{i+1})]\]
не больше, чем $\rho(a_{i+1}-a_i),$ где $\rho$ --- какое-то
положительное вещественное число. 
Докажите, что $\delta(f, \epsilon)\leq \rho\epsilon$,
где $\delta(f, \epsilon)$ --- функция, определенная выше.
\end{zadacha}

\begin{zadacha}
Пусть $f_0\in {\cal P}l$, $f_1 = \mu(f_0), \dots, f_n = \mu(f_{n-1})$,
и для всех прямолинейных сегментов
$[a_i, a_{i+1}]$ в $f_0$, длина $[f(a_i), f(a_{i+1})]$
не больше, чем $\rho(a_{i+1}-a_i)$.
Докажите, что $\delta(f_n, \epsilon)\leq \rho 2^n\epsilon$.
\end{zadacha}

\begin{zadacha}
Пусть $f\in {\cal P}l$, а самый большой прямолинейный
сегмент $[f(a_i), f(a_{i+1})]$ соответствующей ломаной
имеет длину $k$. Докажите, что 
$\delta(\mu(f), \epsilon) \leq 2\frac{k}{\sqrt 2}+\delta(f, \epsilon)$.
\end{zadacha}

\begin{zadacha} 
Пусть $f_0\in {\cal P}l$, $f_1 = \mu(f_0), \dots, f_n = \mu(f_{n-1})$,
а самый большой прямолинейный сегмент в ломаной, соответствующей
$f_0$, имеет длину $k$.
Докажите, что 
\begin{equation}\label{_delta_Equation_}
\delta(f_n, \epsilon) \leq 4\frac{k}{2^{n-m}\sqrt 2}+\rho 2^m\epsilon
\end{equation}
для любых $n$, $m$ ($n>m$)
\end{zadacha}

\begin{zadacha}
В предыдущей задаче, возьмем $\epsilon < 2^{-2m}$,
$n > 2m$. Выведите из \eqref{_delta_Equation_} 
\[
\delta(f_n, \epsilon) \leq  \frac{4k\sqrt 2+\rho}{2^{-m}}.
\]
Докажите, что для произвольного $i$
\[
\delta(f_i, \epsilon) \leq  \max \left(\frac{4k\sqrt
2+\rho}{2^{-m}},  \rho 2^{2m}\epsilon\right).
\]
\end{zadacha}

\begin{zadacha}[!]
Пусть $f_0$ линейно отображает $[0, 1/2]$ в отрезок
$[(0,0), (1,1)]$, a $[1/2, 1]$ --- в отрезок $[ (1,1), (0,0)]$.
Докажите, что множество $\{ f_i\}$ равномерно непрерывно.
\end{zadacha}

\begin{ukazanie}
Выведите из предыдущей задачи, что 
$\lim\limits_{\epsilon\arrow 0}\sup_i(\delta(f_i, \epsilon))=0$.
\end{ukazanie}

\begin{zadacha} 
Выведите из теоремы Арцела-Асколи, что
предел $\lim f_i$ (в $\sup$-метрике) существует
и является непрерывной функцией 
${\cal P}:\; [0,1] \arrow [0,1]\times [0,1]$.
\end{zadacha}

\begin{opredelenie}
Определенная выше функция
${\cal P}$ называется {\bf кривая Пеано}.
\end{opredelenie}

\begin{zadacha} 
Найдите ${\cal P}(q)$, где $q=\frac {a}{2^n}$ ($a\in \Z$) --
двоично-рациональное число. 
\end{zadacha}

\begin{zadacha}
Пусть $Q_2$ --- множество двоично-рациональных чисел.
Докажите, что ${\cal P}(Q_2)$ плотно в
квадрате. 
\end{zadacha}

\begin{zadacha}[!]
Докажите, что образ ${\cal P}$ --
это весь квадрат.
\end{zadacha}

\begin{ukazanie}
Воспользуйтесь тем, что образ компакта компактен.
\end{ukazanie}

\begin{zadacha}[!]
Можно ли сюръективно и непрерывно 
отобразить $[0,1]$ на куб? На куб с выколотой точкой?
\end{zadacha}

%%%%%%%%%%%%%%%%%%%%%%%%%%%%%%%%%%%%%%%%%%%%%%%%

\chapter{Листок 7: Связность}

%%%%%%%%%%%%%%%%%%%%%%%%%%%%%%%%%%%%%%%%%%%%%%%%

\begin{opredelenie}
Пусть дано топологическое пространство $M$.
Подмножество $W\subset M$ называется {\bf открытозамкнутым},
если оно открыто и замкнуто. $M$ называется {\bf связным},
если любое открытозамкнутое подмножество $M$
это либо $\emptyset$, либо само $M$.
Подмножество $Z\subset M$ называется
{\bf связным}, если оно связно в индуцированной
топологии.
\end{opredelenie}

\begin{zadacha}
Связно ли $\R$?
\end{zadacha}

\begin{zadacha}[!]
Пусть $X$, $Y$ связные. Докажите, что $X\times Y$ связно.
\end{zadacha}

\begin{ukazanie}
Пусть в $X\times Y$ есть открытозамкнутое подмножество.
Рассмотрим пересечение $U\cap X\times \{y\}$. Докажите, что
$X\times \{y\}$ (с индуцированной топологией) гомеоморфно
$X$, а $U\cap X\times \{y\}$ открытозамкнуто там.
\end{ukazanie}

\begin{zadacha}
Связно ли $\R^n$ (с естественной топологией)?
\end{zadacha}

\begin{zadacha}
Пусть в топологическом пространстве
$M$ любые две точки $x, y$ можно ``соединить путем'',
то есть найти такое непрерывное отображение
$[0,1] \stackrel \phi \arrow M$, что
$\phi(0)=x$, $\phi(1)=y$. Докажите, что
$M$ связно.
\end{zadacha}

\begin{zamechanie} В такой ситуации $M$ называется
{\bf линейно связным}. \end{zamechanie}

\begin{zadacha}
Выкинем точку из окружности или плоскости.
Докажите, что получится связное пространство.
\end{zadacha}

\begin{zadacha}[!]
\begin{enumerate} \итем Выкинем конечное число точек из $\R^2$.
Докажите, что получится связное пространство.

\итем Выкинем точку из интервала. 
Докажите, что получится несвязное пространство.
\end{enumerate}
\end{zadacha}

\begin{zadacha}[!]
Докажите, что следующие пространства попарно
негомеоморфны: $\R$, $\R^2$, окружность.
\end{zadacha}

\begin{zadacha}[!]
Докажите, что
отрезок, интервал и полуинтервал попарно негомеоморфны.
\end{zadacha}

\begin{zadacha}
Дано непрерывное отображение
$f:\; X \arrow Y$. Пусть $X$ связно.
Докажите, что образ $f$ связен.
\end{zadacha}

\begin{zadacha}[!]
Дано связное подмножество в отрезке $[0,1]$.
Докажите, что это интервал, полуинтервал или отрезок.
\end{zadacha}

\begin{zadacha}
Дано непрерывное отображение
$f:\; X \arrow \R$. Пусть $X$ связно,
а $f$ принимает и положительные, и отрицательные
значения. Докажите, что $f$ где-то зануляется.
\end{zadacha}

\begin{zadacha}[*]
Пусть дано метризуемое
счетное связное пространство $M$. Докажите,
что $M$ это точка.
\end{zadacha}

\begin{zadacha} Пусть даны связные 
подмножества топологического пространства $M$, 
пересечение которых непусто.
Докажите, что их объединение связно.
\end{zadacha}


\begin{zadacha}[!]
Пусть $x\in M$ --- точка в топологическом пространстве,
а $W$ --- объединение всех связных подмножеств, которые
ее содержат. Докажите, что $W$ связно.
\end{zadacha}


\begin{opredelenie}
В такой ситуации, $W$ называется {\bf компонентой
связности} точки $x$ (или просто {\bf компонентой связности}).
\end{opredelenie}

\begin{zadacha}
Докажите, что связное подмножество $W\subset M$ есть компонента
связности тогда и только тогда, когда любое связное
подмножество, содержащее $W$, с ним совпадает.
\end{zadacha}

\begin{zadacha}
Докажите, что $M$ разбивается в объединение
непересекающихся компонент связности.
\end{zadacha}

\begin{zadacha}
Докажите, что все компоненты связности $M$ замкнуты.
\end{zadacha}

%%%%%%%%%%%%%%%%%%%%%%%%%%%%%%%%%%%%%%%%%%%%%%%%
\subs{Вполне несвязные пространства}
%%%%%%%%%%%%%%%%%%%%%%%%%%%%%%%%%%%%%%%%%%%%%%%%

\begin{opredelenie}
Топологическое пространство $M$ называется {\bf вполне несвязным},
если любая компонента связности $M$ состоит из одной точки.
\end{opredelenie}

\begin{zadacha}
Докажите, что множество рациональных чисел
(с топологией, индуцированной с $\R$) вполне несвязно.
Докажите, что оно не дискретно.
\end{zadacha}

\begin{zadacha}[*]
Докажите, что пространство $p$-адических чисел
вполне несвязно.
\end{zadacha}

\begin{zadacha}[*]
Докажите, что произведение вполне несвязных
пространств вполне несвязно.
\end{zadacha}

\begin{zadacha} 
Пусть дано хаусдорфово топологическое пространство с предбазой $S$.
Пусть все элементы $S$ открытозамкнуты. Докажите, что
$S$ вполне несвязно.
\end{zadacha}

\begin{zadacha}[!]
Рассмотрим множество $\{0, 1\}$
с дискретной топологией, и пусть $\{0, 1\}^I$ --
произведение $I$ копий $\{0, 1\}$ с тихоновской
топологией, где $I$ --- произвольный набор индексов.
Докажите, что $\{0, 1\}^I$ вполне несвязно.
\end{zadacha}

\begin{ukazanie}
Воспользуйтесь предыдущей задачей.
\end{ukazanie}

\begin{zadacha}[*]
Пусть дано компактное, хаусдорфово
топологическое пространство $M$,  пусть $M_1$ --
множество компонент связности $M$, а
$M\overset{\pi}{\arrow} M_1$ --- естественная проекция (точка
переходит в свою компоненту связности).
Введем на $M_1$ такую топологию --
подмножество $U\subset M_1$ открыто, если $\pi^{-1}(U) \subset M$
открыто.
Докажите, что $M_1$ вполне несвязно.
Докажите, что любое непрерывное отображение
$M\overset{\pi_2}{\arrow} M_2$
из $M$ во вполне несвязное пространство $M_2$
раскладывается в композицию непрерывных отображений
$M\overset{\pi}{\arrow} M_1 \arrow  M_2$
(в таком случае говорится, что
``отображение $\pi_2$ пропускается через $\pi$'').
\end{zadacha}

\begin{zadacha} 
Пусть дано открытое подмножество компактного 
пространства $U$ и набор замкнутых
подмножеств $\{K_i\}$, пересечение которых содержится в $U$.
Докажите, что из $\{K_i\}$ можно выбрать конечный
поднабор, пересечение элементов которого содержится в $U$.
\end{zadacha}

\begin{zadacha}[*] Пусть дано
вполне несвязное компактное хаусдорфово
топологическое пространство $M$.
Докажите, что каждая точка $x\in M$ является пересечением
всех открытозамкнутых подмножеств $M$, которые
ее содержат.
\end{zadacha}

\begin{ukazanie}
Пусть $P$ --- пересечение всех открытозамкнутых подмножеств
$M$, которые содержат $x$. Очевидно, что оно
замкнуто. Докажите, что оно равно $\{x\}$ либо несвязно.
Если оно несвязно, $P$
распадается в объединение двух непустых непересекающихся замкнутых
подмножеств $P_1$, $P_2$. Воспользовавшись тем,
что в компактном хаусдорфовом пространстве выполняется Т4
(докажите это), найдем у $P_1$, $P_2$ непересекающиеся
открытые окрестности $U_1$, $U_2$.
Выведите из предыдущей задачи, что в $U_1 \cup U_2$
содержится открытозамкнутое подмножество $W\subset M$,
содержащее $x$.
Докажите, что $W\cap U_i$ открытозамкнуты, и 
выведите из этого, что $P$ это $\{x\}$.
\end{ukazanie}

\begin{zadacha}[*]
Пусть дано вполне несвязное компактное хаусдорфово
топологическое пространство $M$. Докажите, что
открытозамкнутые множества образуют 
базу топологии $M$.
\end{zadacha}


\begin{ukazanie}
Пусть дано открытое подмножество $U\subset M$ и в нем
точка $x$. Возьмем у каждой точки $M\backslash U$
открытозамкную окрестность, не содержащую $x$
(докажите, что это можно сделать).
Мы получим покрытие $\{U_\alpha\}$ 
множества $M\backslash U$.
Поскольку  $M\backslash U$ компактно, 
из $\{U_\alpha\}$ можно выбрать конечное
подпокрытие $U_1, ... U_n$. Докажите, что
дополнение к $\cup U_i$ открытозамкнуто,
содержит $x$ и содержится в $U$.
\end{ukazanie}

\begin{zadacha}[*]
Пусть дано вполне несвязное компактное хаусдорфово
топологическое пространство $M$, и пусть $x, y \in M$ --- две различные
точки. Докажите, что $M$ допускает непрерывное отображение
в $\{0, 1\}$ (с дискретной топологией) такое, чтo
$x$ переходит в $0$, а $y$ --- в $1$.
\end{zadacha}

\begin{zadacha}[*]
Пусть дано вполне несвязное компактное хаусдорфово
топологическое пространство $M$, и пусть $I$ --- множество
всех непрерывных отображений $M$ в $\{0, 1\}$.
Определите естественное отображение
$M \arrow \{0, 1\}^I$. Докажите, что
это непрерывное вложение, и что образ $M$ замкнут.
\end{zadacha}

\begin{zadacha}[*]
Пусть $M$ --- компактное хаусдорфово топологическое пространство.
Докажите, что следующие утверждения равносильны
\begin{enumerate}
\renewcommand{\labelenumi}{(\roman{enumi})}
\item $M$ вполне несвязно 

\item $M$ может быть вложено в $\{0, 1\}^I$
для какого-то множества индексов $I$.
\end{enumerate}
\end{zadacha}

\begin{zamechanie}
Напомним, что если компакт $M$ допускает непрерывное инъективное отображение 
$f:M \to X$ в хаусдорфово пространство $X$, то $f$ есть гомеоморфизм
между $M$ и $f(M) \subset X$ с индуцированной топологией.
\end{zamechanie}


%%%%%%%%%%%%%%%%%%%%%%%%%%%%%%%%%%%%%%%%%%%%%%%%%%%%%%%%%%%%

\chapter[Листок 8: Фундаментальная группа и пространство петель]%
{Листок 8: Фундаментальная\\ группа и пространство петель}

%%%%%%%%%%%%%%%%%%%%%%%%%%%%%%%%%%%%%%%%%%%%%%%%%%%%%%%%%%%%

%%%%%%%%%%%%%%%%%%%%%%%%%%%%%%%%%%%%%%%%%%%%%%%%
\subs{Линейная связность}
%%%%%%%%%%%%%%%%%%%%%%%%%%%%%%%%%%%%%%%%%%%%%%%%

\begin{opredelenie}
Пусть $M$ --- топологическое пространство. 
Напомним, что {\bf путем} в $M$ называется
непрерывное отображение $[a, b] \stackrel \phi \arrow M$.
В этом случае говорится, что путь 
$\phi$ {\bf соединяет точки $\phi(a)$ и $\phi(b)$}.
$M$ называется {\bf линейно связным}, если любые
две точки $M$ можно соединить путем $[a, b] \stackrel \phi \arrow M$.
\end{opredelenie}

\begin{zadacha}
Пусть $a$, $b$, $c$ лежат в $M$, причем 
$a$ можно соединить путем с $b$, а $b$ с $c$.
Докажите, что $a$ можно соединить путем с $c$.
\end{zadacha}

\begin{zadacha} 
Выведите из этого, что объединение 
линейно связных подмножеств $M$, содержащих 
выбранную точку $x\in M$, линейно связно.
\end{zadacha}

\begin{opredelenie}
Объединение всех линейно связных
подмножеств, содержащих какую-то фиксированную точку $x$, называется 
{\bf компонентой линейной связности} $M$.
\end{opredelenie}

\begin{zadacha} 
Рассмотрим следующее подмножество $X\subset \R^2$:
график функции $\sin(1/t)$, объединенный с отрезком
$[(0,1), (0,-1)]$. Докажите, что $X$ локально
компактно, связно, и не линейно связно.
Найдите компоненты линейной связности.
\end{zadacha}

\begin{zadacha}[*]
Найдите компактное и связное метризуемое
топологическое пространство, имеющее
бесконечное количество компонент линейной связности.
\end{zadacha}

\begin{opredelenie}
Пусть $\{M_\alpha\}$ --- набор топологических пространств,
индексированный множеством ${\mathfrak A}$.
Несвязное объединение $\bigsqcup_{\alpha \in \mathfrak A} M_\alpha$ --
это топологическое пространство, точками которого являются пары 
$(\alpha,m) \ \ | \ \  \alpha\in {\mathfrak A}, m\in M_\alpha$,
а база топологии задается открытыми множествами во всех 
$M_\alpha$.
\end{opredelenie}

\begin{zadacha}
Докажите, что несвязное объединение одноточечных  пространств
дискретно. Докажите, что естественная проекция
$\bigsqcup_{\alpha \in \mathfrak A} M_\alpha \arrow {\mathfrak A}$
на ${\mathfrak A}$ с дискретной топологией непрерывна.
\end{zadacha}


\begin{opredelenie}
Топологическое пространство $M$
называется локально связным 
(локально линейно связным), если каждая 
окрестность точки $x\in M$ содержится в связной
(линейно связном) окрестности этой точки.
\end{opredelenie}

\begin{zadacha} 
Пусть дано топологическое пространство $M$.
Докажите, что если $M$ локально связно (локально линейно связно)
то $M$ представляется в виде
несвязного объединения своих компонент связности (линейной связности).
\end{zadacha}

\begin{zadacha}
Докажите, что связное пространство линейно связно,
если оно локально линейно связно.
\end{zadacha}

\begin{zadacha}
Пусть дано открытое подмножество в $\R^n$. Докажите, что оно
локально линейно связно.
\end{zadacha}

\begin{zadacha}[**]
Пусть $\omega$ --- первый континуальный ординал,
а $\phi:\; [0,1] \arrow \omega$ --- соответствующая биекция.
Пусть $X \subset [0,1]\times [0,1]$ --
подмножество квадрата, состоящее из всех
$x,y$ таких, что $\phi(x) > \phi(y)$.
Докажите, что $X$ связно. Докажите, что
линейно связные компоненты $X$ --- либо
точки, либо сегменты горизонтальных 
отрезков.
\end{zadacha}

\begin{ukazanie}
Докажите, что пересечение $X$ с любым
вертикальным отрезком нигде не плотно. 
Пусть $V\subset [0,1]\times [0,1]$ --- связное
замкнутое подмножество квадрата, содержащееся 
в $X$. Докажите, что $V$ пересекается с каждым вертикальным 
отрезком не более чем в одной точке. 
Значит, $V$ это график непрерывного отображения
$\gamma:\; [a,b] \arrow [0,1]$,
такого, что $\phi(\gamma(a))< \phi(a)$.
Докажите, что такое отображение постоянно.
\end{ukazanie}

%%%%%%%%%%%%%%%%%%%%%%%%%%%%%%%%%%%%%%%%%%%%%%%%
\subs{Геодезическая связность}
%%%%%%%%%%%%%%%%%%%%%%%%%%%%%%%%%%%%%%%%%%%%%%%%

\begin{opredelenie}
Пусть $M$ --- полное локально компактное
метрическое пространство. Напомним, что 
{\bf геодезической} в $M$ называется
отображение $[a, b] \arrow M$, которое
сохраняет метрику. Говорят, что $M$ {\bf геодезически
связно}, если любые две точки можно соединить
геодезической. Естественно, геодезически
связное пространство линейно связно.
\end{opredelenie}

\begin{opredelenie}
Пусть $M$ --- полное локально компактное
метрическое пространство. Говорят, что $M$
{\bf липшицево связно}, с константой Липшица $C\geq 1$,
если для любых $x, y \in M$ и любого $\epsilon >0$
найдется такая последовательность точек $x_1=x, x_2,\dots, x_n=y$, что 
$d(x_i, x_{i+1})< \epsilon$, а 
$\sum_i d(x_i, x_{i+1}) \leq C d(x,y)$.
Иначе говоря, мы можем расставить $n$ точек
между $x$ и $y$ таким образом, что они отстоят
друг от друга не больше, чем на $\epsilon$,
а длина ломаной, составленной из этих 
точек, не больше $C d(x,y)$.
\end{opredelenie}

\begin{zadacha}[*]
Докажите, что метрическое пространство
$M$ геодезически связно $\Leftrightarrow$ 
$M$ липшицево связно, с константой Липшица 1.
\end{zadacha}

\begin{ukazanie}
Это теорема Хопфа-Ринова.
\end{ukazanie}

\begin{zadacha}[!]
Пусть $(M, d)$ --- липшицево связное метрическое пространство,
с константой $C$. Определим функцию $d_h:\; M\times M \arrow \R$ как
\[
\lim\limits_{\epsilon\arrow 0} \inf\left(\sum d(x_i, x_{i+1})\right),
\]
где $\inf$ берется по всем таким последовательностям 
$x_1=x, x_2, \dots, x_n=y$, что $d(x_i, x_{i+1})< \epsilon$.
Докажите, что
$d(x, y) \leq d_h(x,y) \leq Cd(x,y)$
для любых $x, y \in M$. Докажите,
что $d_h$ --- метрика, и что $(M, d)$ гомеоморфно
$(M, d_h)$.
\end{zadacha}

\begin{zadacha}[*]
Докажите, что 
$(М, d_h)$ липшицево связное, с любой константой
$C>1$. 
\end{zadacha}

\begin{zadacha}[*]
Докажите, что $(М, d_h)$
удовлетворяет условию Хопфа-Ринова (а следовательно,
геодезически связно).
\end{zadacha}

\begin{opredelenie}
Напомним, что отображение $[a,b]\stackrel\phi\arrow M$ 
{\bf удовлетворяет условию Липшица, с константой $C>0$},
если $d(\phi(x),\phi(y)) \leq C|x-y|$, для любых
$x, y\in [a,b]$. Легко видеть, что липшицево 
отображение непрерывно.
\end{opredelenie}

\begin{zadacha}[*]
Пусть $M$ --- локально компактное полное метрическое пространство.
Докажите, что $M$ липшицево связное с константой $C$,
тогда и только тогда,  когда любые две точки можно 
соединить путем, который удовлетворяет условию Липшица
с той же самой константой. 
\end{zadacha}

\begin{ukazanie} 
Воспользуйтесь предыдущей задачей и неравенством
$d(x, y) \leq d_h(x,y) \leq Cd(x,y)$.
\end{ukazanie}

\begin{zamechanie}
Мы получили, что липшицево связное метрическое
пространство линейно связно.
\end{zamechanie}

\begin{zadacha}  
Рассмотрим окружность $S$ на плоскости, 
с индуцированной метрикой. Докажите, что
$S$ липшицево связно, с константой $\frac \pi 2$.
\end{zadacha}

\begin{zadacha}[*]
Докажите, что $\frac \pi 2$ --- наименьшая из
констант, для которых окружность с такой метрикой 
липшицево связна.
\end{zadacha}

\begin{zadacha}[**]
Рассмотрим отображение $]0, \infty[ \arrow \R^2$,
заданное в полярных координатах функцией
$\theta = 1/x$, $r=x$ (это спираль,
которая наматывается вокруг нуля, с шагом $\frac 1 {2\pi n}$).
Пусть $X$ --- замыкание графика этого отображения
(оно, очевидно, состоит из этого графика и нуля).
Докажите, что $X$ линейно связно. Докажите, что
$X$ не липшицево связно, какую бы константу $C$ мы не взяли.
\end{zadacha}

\begin{zadacha}[*]
Пусть $M$ --- локально компактное полное метрическое пространство.
Обозначим через $S_\epsilon(x)$ сферу радиуса $\epsilon$ с центром в $x$.
Докажите, что следующие условия равносильны.
\begin{enumerate}
\renewcommand{\labelenumi}{(\roman{enumi})}
\item $M$ липшицево связное, с константой $C$

\item для любых $x, y \in M$ и любых $r_1, r_2 >0$, для которых $r_1 + r_2 \leq 1$,
расстояние между сферами $S_{dr_1}(x)$, $S_{dr_2}(y)$
не больше $C d (1-r_1-r_2)$, где $d= d(x,y)$.
\end{enumerate}
\end{zadacha}


\begin{ukazanie}
Чтобы вывести из липшицевой связности 
(ii), проведите через $x, y$ кривую Липшица. 
Из (ii) липшицева связность следует непосредственно.
Расстояние от точки $x$ до сферы $S_{d(1-C^{-1}\epsilon)}(y)$
не больше $\epsilon$; возьмем в качестве $x_2$
точку сферы, реализующую это расстояние (что
возможно, поскольку сфера, по теореме Хопфа-Ринова, 
компактна) и применим индукцию.
\end{ukazanie}

\begin{zamechanie} 
Напомним, что условие Хопфа-Ринова 
(в одной из версий) состоит в том, что
расстояние между сферами $S_{dr_1}(x)$, $S_{dr_2}(y)$
равно $d(1-r_1-r_2)$.  
\end{zamechanie}

%%%%%%%%%%%%%%%%%%%%%%%%%%%%%%%%%%%%%%%%%%%%%%%%%%%%%%%%%%%%
\subs{Пространство петель}
%%%%%%%%%%%%%%%%%%%%%%%%%%%%%%%%%%%%%%%%%%%%%%%%%%%%%%%%%%%%

\begin{opredelenie}
Пусть $(M, x)$ --- топологическое пространство с отмеченной
точкой. Рассмотрим множество $\Omega(M, x)$ путей
$[0, 1] \stackrel \phi \arrow M$, $\phi(0) = \phi(1) = x$,
с открыто-компактной топологией (предбаза этой топологии --- 
множества $U(K, W)$ отображений, переводящих
заданный компакт $K\subset [0,1]$ в заданное
открытое множество $W\subset M$). 
Тогда $\Omega(M, x)$ называется 
{\bf пространством петель} для $(M,x)$.
\end{opredelenie}

\begin{zadacha}[!]
Пусть $M$ метризуемо. Докажите, что 
$\Omega(M, x)$ тоже метризуемо, и метрика
задается по формуле 
\[ d(\gamma, \gamma') = \sup_{x\in[0,1]}d(\gamma(x),
\gamma'(x)).
\]
\end{zadacha}

\begin{zadacha}
Пусть $(M, x)$ --- пространство с отмеченной точкой,
$M_0$ --- компонента связности отмеченной точки $x$, а $M_1$ --- компонента
линейной связности $x$. Докажите, что
$\Omega(M, x)=\Omega(M_0, x)=\Omega(M_1, x)$.
\end{zadacha}

\begin{zadacha}
Пусть $X, Y$ --- компакты, а $\cal W$  --- пространство
отображений из $X$ в $M$, снабженное открыто-компактной топологией.
Постройте биекцию между непрерывными отображениями
из $Y$ в $\cal W$ и непрерывными отображениями
$X \times Y \arrow M$.
\end{zadacha}


\begin{zadacha}[!]
Пусть $\gamma, \gamma'\in \Omega(M,x)$ --- точки в
пространстве петель. Постройте биекцию между следующими
множествами:
\begin{enumerate}
\renewcommand{\labelenumi}{(\roman{enumi})}
\item Пути $\Gamma:\; [0, 1] \arrow \Omega(M,x)$,
соединяющие $\gamma$ и $\gamma'$.

\item Непрерывные отображения $\Psi$ из квадрата
$[0,1]\times [0,1]$ в $M$, переводящие $\{1\}\times [0,1]$
в $x$, и такие, что 
$\Psi\restrict{[0,1]\times \{0\}} = \gamma$,
$\Psi\restrict{[0,1]\times \{1\}} = \gamma'$.
\end{enumerate}
\end{zadacha}

\begin{opredelenie}
Пути $\gamma, \gamma'\in \Omega(M,x)$, для
которых такое отображение 
$\Psi:\; [0,1]\times [0,1]\arrow M$ существует,
называются {\bf гомотопными}, а $\Psi$ --- связывающей их 
{\bf гомотопией}.
\end{opredelenie}

\begin{zadacha} 
Докажите, что множество всех петель, гомотопных
$\gamma\in \Omega(M,x)$ --- это компонента линейной
связности $\gamma\in \Omega(M,x)$.
\end{zadacha}

\begin{zadacha}
Докажите, что гомотопия петель является отношением
эквивалентности.
\end{zadacha}

\begin{zamechanie} 
Гомотопные петли также называют {\bf гомотопически
эквивалентными}.
\end{zamechanie}

\begin{opredelenie}
Пусть $(M, x)$ линейно связно. Множество классов 
гомотопической эк\-ви\-ва\-лент\-ности петель обозначается
через $\pi_1(M, x)$.
\end{opredelenie}

\begin{zadacha}[*]
Пусть $M\subset \R^2$ --- объединение отрезка 
$[(0,1), (0,-1)]$ и сегментов окружностей диаметра
$3, 4, 5, \dots$, соединяющих $(0,1)$ и $(0,-1)$.
\begin{center}
\epsfig{file=geom10b.eps,width=0.5\linewidth}
\end{center}
Докажите, что $M$ линейно связно. Докажите, что для любого $x\in M$
$\Omega(M,x)$ не локально линейно связно.
\end{zadacha}

\begin{zadacha}[*] \label{_edi_geode_Zadacha_}
Пусть $(M,d)$ --- такое геодезически связное 
локально компактное метрическое пространство,
что для некоторого $\delta >0$ и любых
точек $x, y\in M$, $d(x,y)<\delta$, 
геодезическая, соединяющая $x$ и $y$, единственна.
Пусть $\Delta_\delta\subset M\times M$ --
множество пар $x, y\in M$, $d(x,y)<\delta$.
Рассмотрим отображение $\Delta_\delta\arrow M$,
ставящее паре точек середину соединяющей
их геодезической. Докажите, что оно непрерывно.
\end{zadacha}

\begin{ukazanie}
Пусть $\{(x_i, y_i)\}$ --- последовательность таких
пар, сходящихся к $(x,y)$, а 
$\{z_i\}$ --- последовательность середин геодезических.
В силу локальной компактности, у $\{z_i\}$
есть предельные точки и нет бесконечных 
дискретных подмножеств. Любая предельная точка 
$\{z_i\}$ будет серединой геодезической,
соединяющей $x$ и $y$. Следовательно, у 
$\{z_i\}$ есть единственная предельная точка.
\end{ukazanie}

\begin{zadacha}[*]
Рассмотрим отображение 
$\Delta_\delta\otimes [0,1] \stackrel \Psi \arrow M$, 
ставящее паре точек $x, y\in M$, $d(x,y)=d$, 
и $t\in [0,1]$ в соответствие точку
$\gamma_{x,y}\left(\frac t d\right)$,
где $\gamma_{x,y}$ --- геодезическая,
соединяющая $x$ и $y$ (если эти точки 
совпадают, положим $\Psi(x,y,t)=x$).
Докажите, что это отображение непрерывно.
\end{zadacha}

\begin{ukazanie}
Воспользуйтесь предыдущей задачей и конструкцией
геодезической как предела середин отрезков,
которая приводилась в доказательстве
теоремы Хопфа-Ринова.
\end{ukazanie}

\begin{opredelenie}
Пусть $M$ --- метрическое пространство.
Путь $\gamma:\; [0,1] \arrow M$ 
называется {\bf кусочно-геодезическим},
если отрезок $[0,1]$ разбит на подотрезки
$[0, a_1]$, $[a_1, a_2]$, $\ldots$, $[a_n, 1]$,
и на каждом из этих отрезков $\gamma$
удовлетворяет $d (\gamma(x), \gamma(y)) = \lambda_i|x-y|$,
для какой-то константы $\lambda_i$

Иначе говоря, кусочно геодезический путь представляет
собой ломаную, каждый отрезок которой --
геодезическая (с точностью до линейной замены переменных).
\end{opredelenie}

\begin{zamechanie}
Если $M$ --- открытое подмножество в $\R^n$,
с естественной метрикой, то геодезические,
как было доказано в листке 4, это отрезки прямой.
Таким образом, кусочно геодезические
пути --- это ломаные. Такие отображения также
называются {\bf кусочно-линейными}.
\end{zamechanie}

\begin{zadacha}[*]
В условиях задачи \ref{_edi_geode_Zadacha_},
рассмотрим $\Omega(M,x)$ как метрическое
пространство (с $\sup$-метрикой). 
Докажите, что любая петля $\gamma\in\Omega(M,x)$
гомотопна кусочно геодезической, причем 
гомотопию можно выбрать в любой $\epsilon$-окрестности
$B_\epsilon(\gamma)\subset \Omega(M,x)$.
\end{zadacha}

\begin{zadacha}[*]
Выведите из этого, что $\Omega(M,x)$
локально линейно связно.
\end{zadacha}

\begin{zamechanie}
В такой ситуации, $\pi_1(M, x)$ --- множество
связных компонент $\Omega(M, x)$.
\end{zamechanie}

\begin{zadacha} 
Пусть $M$ --- открытое подмножество в $\R^n$.
Докажите, что $\Omega(M, x)$ локально линейно связно.
\end{zadacha}

\begin{ukazanie}
Докажите, что любую петлю можно прогомотопировать (в 
произвольно малой $\epsilon$-окрестности) в кусочно
линейную.
\end{ukazanie}

%\NewVedomost


%%%%%%%%%%%%%%%%%%%%%%%%%%%%%%%%%%%%%%%%%%%%%%%%
\subs{Фундаментальная группа}
%%%%%%%%%%%%%%%%%%%%%%%%%%%%%%%%%%%%%%%%%%%%%%%%

\begin{zadacha}\label{_proizvede_Zadacha_}
Пусть даны петли $\gamma_1, \gamma_2 \in \Omega(M, x)$.
Рассмотрим петлю $\gamma_1 \gamma_2\in \Omega(M, x)$, 
которая задается следующим образом:
\[
\gamma_1 \gamma_2(\lambda) = 
\begin{cases}
\gamma_1(2\lambda) &\qquad \lambda\in [0, 1/2],\\
\gamma_2(2\lambda-1) &\qquad \lambda\in [1/2,1].
\end{cases}
\]
Докажите, что класс гомотопии $\gamma_1\gamma_2$ зависит
только от классов гомотопии $\gamma_1$, $\gamma_2$: если 
$\gamma_1\sim\gamma_1'$, $\gamma_2\sim\gamma_2'$,
то $\gamma_1\gamma_1'\sim \gamma_2\gamma_2'$.
\end{zadacha}

\begin{zadacha}
Докажите, что $(\gamma_1\gamma_2)\gamma_3$ гомотопно
$\gamma_1 (\gamma_2\gamma_3)$.
\end{zadacha}

\begin{zadacha} 
Пусть дана петля $\gamma \in \Omega(M, x)$.
Обозначим через $\gamma^{-1}$ петлю $\gamma^{-1}(x) = \gamma(1-x)$.
Докажите, что петли $\gamma \gamma^{-1}$ и $\gamma^{-1} \gamma$
гомотопны тривиальной петле $[0,1]\arrow x$.
\end{zadacha}

\begin{zamechanie}
Петли, которые гомотопны тривиальной петле, называются
{\bf гомотопными нулю}.
\end{zamechanie}

\begin{zadacha}[!]
Докажите, что операция $\gamma_1, \gamma_2\arrow \gamma_1\gamma_2$
задает на $\pi_1(M, x)$ структуру группы.
\end{zadacha}

\begin{opredelenie} 
Эта группа называется {\bf фундаментальной группой} $M$.
\end{opredelenie}

\begin{zadacha}
Пусть $X\stackrel f \arrow Y$ --- непрерывное отображение
линейно связных пространств,
а $x\in X$ --- произвольная точка. 
Рассмотрим соответствующее отображение
$$
\Omega(X, x) \overset{\check{f}}{\arrow} \Omega(Y, f(y)),
\qquad \gamma \mapsto \gamma\circ f.
$$
Докажите, что $\check f$ переводит
гомотопные пути в гомотопные,
и индуцирует гомоморфизм фундаментальных групп.
\end{zadacha}

\begin{zadacha}
Пусть $M$ --- линейно связное 
топологическое пространство, а $x, y \in M$ --- две точки.
Рассмотрим пространство $\Omega(M, x, y)$
путей $[0,1]\arrow M$, соединяющих
$x$ и $y$, с открыто-компактной топологией.
Как и выше, пути называются гомотопными
(гомотопически эквивалентными)
если они лежат в одной компоненте линейной
связности $\Omega(M, x, y)$.
Определим операцию
$\Omega(M, x, y) \times \Omega(M, y, z)\arrow \Omega(M,x, z)$,
$\gamma_1, \gamma_2 \mapsto \gamma_1\gamma_2$
той же формулой, которая приводится в задаче
\ref{_proizvede_Zadacha_}. Докажите, что
это отображение непрерывно, и переводит
гомотопные пути в гомотопные.
\end{zadacha}

\begin{zadacha}[!]
Пусть $x, y \in M$, а $\gamma_{xy}[0,1] \arrow M$ --
путь, соединяющий $x$ и $y$. Определим
$\gamma_{xy}^{-1}$ формулой 
$\gamma_{xy}^{-1}(\lambda) = \gamma_{xy}(1-\lambda).$
Рассмотрим отображение $\Omega(M,x) \arrow \Omega(M, y)$, 
$\gamma \mapsto \gamma_{xy}^{-1}\gamma \gamma_{xy}$
и $\Omega(M,y) \arrow \Omega(M, x)$, 
$\gamma \mapsto \gamma_{xy}\gamma \gamma_{xy}^{-1}$.
Докажите, что эти отображения переводят гомотопные пути
в гомотопные. Пусть $f$, $g$ --- соответствующие
отображения на фундаментальных группах.
Докажите, что $f$ и $g$ взаимно обратны, 
и индуцируют изоморфизм групп 
$\pi_1(M, x)\stackrel{\phi_{\gamma_{xy}}}\arrow \pi_1(M, y)$.
\end{zadacha}

\begin{zamechanie}
Как видно из следующей задачи, если $\pi_1(M)$ не абелева, то
полученный изоморфизм $\pi_1(M, x)\cong \pi_1(M, y)$
нетривиально зависит от выбора пути $\gamma_{xy}$.
Тем не менее, когда зависимость от отмеченной
точки не важна, фундаментальную группу $M$ обозначают
просто $\pi_1(M)$. Это обозначение не вполне
корректно.
\end{zamechanie}

\begin{zadacha}[!]
В условиях предыдущей задачи,
пусть $x=y$, а $\gamma_{xx}$ --- некоторый путь. 
Докажите, что полученный выше 
изоморфизм \[ \pi_1(M, x)\stackrel{\phi_{\gamma_{xx}}}\arrow \pi_1(M, x)\]
выражается через $\gamma_{xx}$ так:
$\gamma \arrow \gamma_{xx} \gamma \gamma_{xx}^{-1}$.
\end{zadacha}

%%%%%%%%%%%%%%%%%%%%%%%%%%%%%%%%%%%%%%%%%%%%%%%%
\subs{Односвязные пространства}
%%%%%%%%%%%%%%%%%%%%%%%%%%%%%%%%%%%%%%%%%%%%%%%%


\begin{opredelenie}
Пусть $M$ --- линейно связное топологическое
пространство. Говорят, что $M$ {\bf односвязно},
если все петли на $M$ стягиваемы, т.е. 
если $\pi_1(M) = \{1\}$.
\end{opredelenie}

\begin{zadacha} Докажите, что $\R^n$ односвязно.
\end{zadacha}

\begin{opredelenie}
Пусть $(M, x)$ --- топологическое пространство с отмеченной точкой,
а $M \times [0,1] \stackrel \phi\arrow M$ --- такое непрерывное отображение, 
что $\phi(M\times \{1\}) = \{x\}$,
а $\phi \restrict{M\times \{0\}}$
задает тождественное отображение
из $M= M\times \{0\}$ в $M$. Тогда $(M,x)$
называется {\bf стягиваемым}.
В такой ситуации, говорится, что
$\phi$ {\bf задает гомотопию между тождественным
отображением и проекцией $M\arrow \{x\}$.}
\end{opredelenie}

\begin{zadacha}[!]
Пусть $(M, x)$ линейно связно и стягиваемо.
Докажите, что для любой
точки $y\in M$ пространство $(M, y)$ стягиваемо.
\end{zadacha}

\begin{ukazanie} 
Пусть $M \times [0,1] \stackrel \phi\arrow M$ --
гомотопия между тождественным отображением 
и проекцией в $\{x\}$, а $[1,0]\stackrel\gamma\arrow M$ --
путь, соединяющий $x$ и $y$. Возьмите отображение
$M \times [0,1] \overset{\phi_1}{\arrow} M$,
переводящее $(m, t)$ в $\phi(m, 2t)$
для $t\in [0,1/2]$ и $(m, t)$ в
$\gamma(2t-1)$ для $t\in [1/2,0]$.
\end{ukazanie}

\begin{zadacha}
Докажите, что стягиваемое топологическое пространство линейно связно.
\end{zadacha}

\begin{zamechanie}
Из двух вышеприведенных задач ясно, что
стягиваемость $(M, x)$ не зависит от
выбора $x$. В дальнейшем мы будем
говорить просто ``$M$ стягиваемо''.
\end{zamechanie}

\begin{zadacha} 
Докажите, что стягиваемое пространство односвязно.
\end{zadacha}

\begin{zadacha}[!]
Пусть $V\subset \R^n$ --- {\bf звездчатое подмножество}
$(\R^n, x)$, то есть такое подможество, что любая прямая, проходящая
через $x\in \R^n$, пересекается с $V$ по связному множеству,
а $x\in V$. Докажите, что $V$ стягиваемо.
\end{zadacha}

\begin{zadacha}
Пусть $V\subset \R^n$ --- выпуклое подмножество.
Докажите, что оно стягиваемо.
\end{zadacha}

\begin{opredelenie}
Пусть $N\subset M$ --- топологическое пространство
и его подмножество. {\bf Деформационной ретракцией}
$M$ к $N$ называется такое непрерывное отображение
$M \times [0,1] \stackrel \phi\arrow M$,
что $\phi(M\times \{1\}) \subset N$,
причем ограничение этого отображения на $N$
тождественное,  а $\phi \restrict{M\times \{0\}}$
задает тождественное отображение. В этом случае
$N$ называется {\bf деформационным ретрактом} $M$.
\end{opredelenie}

\begin{zadacha}[!]
Пусть $N\subset M$ --- деформационный 
ретракт, $n\in N$ --- точка в $N$.
Докажите, что естественное отображение 
$\pi_1(N, n)\arrow \pi_1(M, n)$ --
изоморфизм.
\end{zadacha}

\begin{opredelenie}
Пусть $M$ --- топологическое пространство, 
а $\sim$ --- соотношение эквивалентности. 
Множество классов эквивалентности обозначается,
как всегда, через $M/\sim$. 
На $M/\sim$ вводится {\bf топология фактора}:
открытые подмножества
$M/\sim$ --- это такие подмножества, прообраз
которых в $M$ открыт.
В частности, если на $M$ действует группа $G$, то возникает
естественное отношение эквивалентности: $x \sim y$ если
существует такое $g \in G$, что $g \cdot x = y$.
Фактор $M$ по этому отношению эквивалентности называется
{\bf факторпространством $M$ по действию $G$}, и обозначается
$M/G$. Классы эквивалентности называются {\bf $G$-орбитами}
в $M$.
\end{opredelenie}

\begin{zadacha}\label{_hausdo_kone_Zadacha_}
Пусть $M$ --- хаусдорфово топологическое пространство,
а $\{x_1, \dots, x_n\} \subset M$ и $\{y_1, \dots, y_m\} \subset M$ 
-- два непересекающихся конечных подмножества. Докажите, что у 
подмножеств $\{x_1, \dots, x_n\}$ и  $\{y_1,\dots, y_m\}$ найдутся
непересекающиеся
окрестности. 
\end{zadacha}

\begin{zadacha}[!]
Пусть $M$ --- хаусдорфово  топологическое пространство, а 
$G$ --- конечная группа, которая действует на $M$
гомеоморфизмами. Рассмотрим
факторпространство $M/G$ с топологией
фактора. Докажите, что $M/G$ хаусдорфово.
\end{zadacha}

\begin{ukazanie}
Пусть $x,y$ --- две точки, не принадлежащие
одной и той же $G$-орбите. Найдите у $x$, $y$
непересекающиеся $G$-инвариантные окрестности. Для этого
примените \ref{_hausdo_kone_Zadacha_} к орбитам
$Gx$, $Gy$, получите окрестности $U$, $U'$,
и возьмите $\bigcap_{g\in G} gU$, $\bigcap_{g\in G} gU'$.
\end{ukazanie}

\begin{zadacha}[*]
Приведите пример,
когда $M$ хаусдорфово, а $M/G$ 
нехаусдорфово (и группа, соответственно, не конечна).
\end{zadacha}

\begin{opredelenie}
Пусть $\Gamma$ --- некоторый граф, то есть
набор данных вида ``множество вершин'' $\{\cal V\}$,
``множество ребер'' $\{\cal R\}$, и сведений о том,
какие вершины являются концами каких ребер.
\begin{center}
\epsfig{file=geom10a.eps,width=0.5\linewidth}
\end{center}
Более строго, $\Gamma$ можно определить как пару множеств
${\cal V}$, $\cal R$ и сюръективное отображение
$\{\cal R\}\times\{0,1\} \stackrel\Gamma\arrow \{\cal V\}$. 
Введем на $\{\cal R\}\times [0,1]$,
отношение эквивалентности, определенное следующим
образом: концы двух ребер эквивалентны, если они примыкают
к одной и той же вершине, остальные точки 
эквивалентны сами себе (и только).
Фактор  $\{\cal R\}\times [0,1]$ по этому
отношению эквивалентности называется
{\bf топологическим пространством графа}. 
\end{opredelenie}

\begin{zadacha}
Докажите, что топологическое пространство любого
графа хаусдорфово.
\end{zadacha}

\begin{zadacha}
Граф называется {\bf связным}, если любая вершина соединена
с любой другой конечной цепочкой ребер. Докажите, что топологическое 
пространство связного графа линейно связно. 
\end{zadacha}

\begin{zadacha}[**]
Пусть дан граф  с бесконечным множеством вершин.
Докажите, что в графе найдется бесконечное подмножество
вершин, попарно соединенных ребрами, либо бесконечное подмножество
вершин, попарно несоединенных.
\end{zadacha}

\begin{zadacha}[!]
Пусть $\Gamma$ --- связный граф, у которого
$n$ вершин и $n-1$ ребро (такой граф называется {\bf деревом}).
\begin{center}
\epsfig{file=geom10c.eps,width=0.5\linewidth}
\end{center}
Докажите, что его топологическое пространство $M_\Gamma$ стягиваемо.
\end{zadacha}

\begin{zadacha}[*]
Пусть $\Gamma$ --- такой бесконечный  связный граф, что любой связный конечный
подграф $\Gamma$ --- дерево. Докажите, что $\pi_1(M_\Gamma)=\{1\}$.
\end{zadacha}

\begin{zadacha}[*]
Пусть $S^n$ --- $n$-мерная сфера ($n> 1$). Докажите, что
$S^n$ односвязна.
\end{zadacha}

\begin{ukazanie}
Воспользуйтесь учением о геодезической связности.
\end{ukazanie}

%%%%%%%%%%%%%%%%%%%%%%%%%%%%%%%%%%%%%%%%%%%%%%%%
\subs{Накрытия}
%%%%%%%%%%%%%%%%%%%%%%%%%%%%%%%%%%%%%%%%%%%%%%%%

\begin{opredelenie}
Пусть $\tilde M \stackrel \pi \arrow M$ --
непрерывное отображение топологических пространств.
Отображение $\pi$ называется {\bf накрытием}, если
у каждой точки есть такая окрестность $U$,
что $\pi^{-1}(U)$ изоморфно произведению
$U$ и дискретного топологического пространства 
$K$, причем стандартное отображение
$\pi^{-1}(U)\stackrel \pi \arrow U$
совпадает с естественной проекцией
$\pi^{-1}(U) = U \times K \arrow U$.
В этом случае также говорится, что
$\tilde M$ {\bf накрывает} $M$.
\end{opredelenie}

Мы рассматриваем окружность
$S^1$ как фактор $S^1 = \R/\Z$.
Это задает естественную групповую структуру на $S^1$.

\begin{zadacha} 
Пусть $n$ --- ненулевое целое число.
Рассмотрим естественное
отображение $S^1 \arrow S^1$, $t \arrow nt$.
Докажите, что это накрытие.
\end{zadacha}

\begin{zadacha} 
Докажите, что естественная проекция
$\R \arrow  S^1 = \R/\Z$ --- накрытие.
\end{zadacha}

\begin{zadacha}
Докажите, что естественная проекция
$\R^n \arrow (S^1)^n$ это накрытие
\end{zadacha}

\begin{zadacha} 
Рассмотрим фактор
$S^n \arrow S^n/\{\pm 1\}= \R P^n$
сферы по центральной симметрии, с 
естественной топологией (открытые множества --- 
это такие, прообраз которых открыт). Докажите, что
это накрытие.
\end{zadacha}

\begin{zadacha}
Пусть $\tilde M \stackrel \pi \arrow M$ --
накрытие, а $\tilde M'\subset \tilde M$ --- подпространство,
которое тоже накрывает $M$. Докажите, что
$\tilde M'$ открыто и замкнуто в $\tilde M$.
\end{zadacha}

\begin{zadacha}
Пусть $\tilde M \stackrel \pi \arrow M$ --
накрытие, а $M$ локально линейно связно. Докажите, что
$\tilde M$ локально линейно связно. Докажите, что 
любая компонента линейной связности в $\tilde M$
накрывает $M$.
\end{zadacha}

\begin{zadacha}[!] \label{_nakry_line_svya_Zadacha_}
Пусть $\tilde M \stackrel \pi \arrow M$ --
накрытие, а $M$ локально линейно связно. Докажите,
что $\tilde M$ связно тогда и только тогда, когда оно
линейно связно.
\end{zadacha}

\begin{opredelenie} 
Пусть $\gamma:\; [a, b] \arrow M$ --- некоторый
путь, а $\tilde M \stackrel \pi \arrow M$ --
накрытие $M$. Отображение 
$\tilde \gamma:\; [a, b] \arrow \tilde M$
называется {\bf поднятием $\gamma$}, если
$\tilde \gamma\circ \pi = \gamma$.
\end{opredelenie}

\begin{zadacha}[!]
Пусть $\tilde M \stackrel \pi \arrow M$ --
накрытие, а $\gamma:\; [a, b] \arrow M$ --
путь, соединяющий $x$ и $y$.
Докажите, что для каждого $\tilde x\in \pi^{-1}(\{x\})$,
существует и единственно поднятие $\tilde \gamma$,
переводящее $a$  в $\tilde x$.
\end{zadacha}

\begin{zadacha}[!] \label{_konec_puti_Zadacha_}
Докажите, что гомотопные пути поднимаются 
до гомотопных путей, а $\tilde \gamma(y)\in \pi^{-1}(\{y\})$
однозначно определяется классом гомотопии $\gamma$ в
$\Omega(M, x, y)$ и точкой $\tilde x$.
\end{zadacha}

\begin{zamechanie}
Обозначим через $\pi_1(M, x, y)$ множество классов
гомотопии путей из $x$ в $y$. Мы получили отображение 
\[ \pi^{-1}(\{x\})\times \pi_1(M, x, y) \stackrel \Psi\arrow \pi^{-1}(\{y\})
\]
\end{zamechanie}


\begin{opredelenie}
Пусть $\tilde M \stackrel \pi \arrow M$ --
накрытие, а $M$ линейно связно. Пространство $\tilde M$ называется
{\bf универсальным накрытием}, если оно связно и односвязно.
\end{opredelenie}

\begin{zamechanie} 
Односвязность была определена только для линейно
связных пространств. Но это ничему не мешает, поскольку
из задачи \ref{_nakry_line_svya_Zadacha_} 
следует, что $\tilde M$ линейно связно.
\end{zamechanie}

\begin{zadacha}[!]
Пусть $\tilde M\stackrel \pi  \arrow M$ --
универсальное накрытие. Зафиксируем
$x\in M$ и $\tilde x\in \pi^{-1}(\{x\})$.
Рассмотрим отображение 
$\pi_1(M, x) \stackrel \psi \arrow \pi^{-1}(\{x\})$,
построенное в \ref{_konec_puti_Zadacha_},
$\psi(\gamma) = \Psi(\tilde x, \gamma)$.
Докажите, что это биекция.
\end{zadacha}

\begin{zadacha} 
Докажите, что $\pi_1(S^1) = \Z$.
\end{zadacha}

\begin{zadacha} Докажите, что $\pi_1((S^1)^n) = \Z^n$.
\end{zadacha}

\begin{zadacha}[*]
Докажите, что при $n>1$ имеем $\pi_1(\R P^n)= \Z/2\Z$.
\end{zadacha}

\begin{zadacha} 
Найдите фундаментальные группы всех букв 
русского алфавита, кроме ``ф'' и ``В''
(точнее, графов, смоделированных на этих буквах).
\end{zadacha}

\begin{zadacha}[*]
Дан конечный связный граф, у которого
$n$ ребер и $n$ вершин. Пусть $M$ --- его топологическое
пространство. Докажите, что $\pi_1(M)=\Z$.
\end{zadacha}

%%%%%%%%%%%%%%%%%%%%%%%%%%%%%%%%%%%%%%%%%%%%%%%%%%%%%%%%%%%%

\chapter{Листок 9: Накрытия Галуа}

%%%%%%%%%%%%%%%%%%%%%%%%%%%%%%%%%%%%%%%%%%%%%%%%%%%%%%%%%%%%

Наука о накрытиях Галуа, про которую
рассказывается в этом листке, весьма похожа
на теорию Галуа алгебраических расширений полей.
Это не случайно. В алгебраической
геометрии методы топологии и дифференциальной 
геометрии применяются к объектам алгебраической 
и теоретико-числовой природы. 

Гротендик определил фундаментальную
группу алгебраического многообразия 
таким образом, что группа Галуа
и фундаментальная группа топологического
пространства оказались частными случаями
более общей конструкции. При изучении
накрытий и расширений полей, а также 
фундаментальной группы и группы Галуа, 
очень полезно держать в голове, что 
это похожие вещи. 

Все топологические пространства в этом
листке предполагаются хаусдорфовыми и локально связными.


\begin{zadacha}
Пусть $\tilde M \stackrel \pi \arrow M$ --
накрытие, а $M_1$ --- связная компонента $\tilde M$.
Докажите, что $\pi(M_1)$ --- связная компонента в $M$. 
\end{zadacha}

\begin{zadacha}[!]
Пусть $\tilde M \stackrel \pi \arrow M$ --
накрытие, причем $\tilde M$ и  $M$ связны и непусты,
а $\pi$ инъективно. Докажите, что $\pi$ --
гомеоморфизм.
\end{zadacha}

\begin{opredelenie}
Пусть $\tilde M \stackrel \pi \arrow M$, 
$\tilde M'\stackrel {\pi'} \arrow M$ --- накрытия.
{\bf Морфизмом накрытий}
называется непрерывное отображение 
$\phi:\; \tilde M\arrow \tilde M'$,
согласованное с проекцией в $M$ --- иначе говоря, такое, что
$\phi\circ \pi' = \pi$.
Множество морфизмов между накрытиями обозначается $\Mor(\tilde M, \tilde M')$.
Изоморфизмом накрытий называется морфизм,
который обратим, причем таким образом,
что $\phi^{-1}\circ \phi=\Id$, $\phi\circ \phi^{-1}=\Id$.
\end{opredelenie}

\begin{zadacha}[!]
Пусть $\phi:\; \tilde M\arrow \tilde M'$ --
морфизм накрытий. Докажите, что 
$\phi:\; \tilde M\arrow \tilde M'$ --
накрытие.
\end{zadacha}

\задача
Пусть $\tilde M \stackrel\pi\arrow M$ --- непрерывное отображение.
\енум
\итем Пусть $\pi$ --- накрытие
Докажите, что у каждой точки $x\in \tilde M$ есть окрестность
$U$ такая, что проекция $\pi:\; U \arrow \pi(U)$ --- гомеоморфизм.
\итем[!] Пусть у каждой точки $x\in \tilde M$ есть окрестность
$U$ такая, что проекция $\pi:\; U \arrow \pi(U)$ --- гомеоморфизм.
Всегда ли $\pi$ --- накрытие?
\ее
\ез

\begin{zadacha} 
Пусть $M$ локально связно, а $\tilde M \stackrel \pi \arrow M$ --
накрытие. Докажите, что $\tilde M$ локально связно.
\end{zadacha}

\begin{zadacha}
Пусть $\tilde M \stackrel \pi \arrow M$,
$\tilde M' \stackrel {\pi'} \arrow M$ --- накрытия,
а $\tilde M'\bigsqcup \tilde M$ --- их несвязная
сумма. Докажите, что это тоже накрытие $M$.
\end{zadacha}

\begin{zadacha}
Пусть $M$ связно, а $\tilde M \stackrel \pi \arrow M$ --
накрытие. Докажите, что 
$\tilde M\cong \bigsqcup_{\alpha\in I} \tilde M_\alpha$,
где $\{\tilde M_\alpha\}$ --- множество компонент связности
$\tilde M$, рассмотренных как накрытия $M$.
\end{zadacha}

\begin{opredelenie}
{\bf Расщеплением} накрытия 
$\tilde M \stackrel \pi \arrow M$
называется изоморфизм $\tilde M$ и накрытия
вида $\tilde M \cong V \times M$, где 
$V$ --- множество с дискретной топологией.
\end{opredelenie}

\begin{zadacha}
Пусть  $\tilde M \stackrel \pi \arrow M$ --
накрытие связного пространства $M$. Докажите, что
$\pi$ расщепляется тогда и только тогда, когда
все связные компоненты $\tilde M$ изоморфны $M$.
\end{zadacha}


%%%%%%%%%%%%%%%%%%%%%%%%%%%%%%%%%%%%
\subs{Накрытия Галуа}
%%%%%%%%%%%%%%%%%%%%%%%%%%%%%%%%%%%%

\begin{zadacha}[!]
Пусть $M_1 \stackrel {\pi_1} \arrow M$,
$M_2 \stackrel {\pi_2} \arrow M$ --- накрытия.
Рассмотрим следующее подмножество в $M_1\times M_2$
\[ M_1\times_M M_2:= 
\{ (m_1, m_2)\in M_1\times M_2 \ \ | \ \ \pi_1(m_1)= \pi_2(m_2)\}.
\]
Мы рассматриваем $M_1\times_M M_2$ как топологическое
пространство (с топологией, индуцированной из $M_1\times M_2$).
Докажите, что естественное отображение 
$M_1\times_M M_2\arrow M$ --
это накрытие.
\end{zadacha}

\begin{opredelenie}
Пространство $M_1\times_M M_2$ вместе с естественным отображением
в $M$ называется {\bf произведением
накрытий $M_1$, $M_2$}. Аналогичным образом 
определяется произведение любого конечного числа накрытий.
\end{opredelenie}

\begin{zamechanie}
Если пользоваться аналогией между расширениями
полей и накрытиями, несвязные объединения накрытий
соответствуют прямой сумме полупростых
артиновых колец, а произведения --- 
тензорным произведениям.
\end{zamechanie}

\begin{zadacha} 
Пусть $M_1$, $M_2$, $M_3$ --- накрытия $M$. Докажите, что
морфизмы из $M_3$ в $M_1\times M_2$
взаимно однозначно соответствуют
парам морфизмов  $\phi_1:\; M_3\arrow M_1$, 
$\phi_2:\; M_2\arrow M_1$.
\end{zadacha}

\begin{zadacha}
Рассмотрим $\R$ как накрытие $S^1$.
Сколько связных компонент у $\R\times_{S^1}\R$?
\end{zadacha}

\begin{opredelenie}
Пусть $M_1\stackrel\phi\arrow M_2$ --- морфизм между двумя накрытиями $M$.
Определим {\bf график $\phi$}
как подмножество в $M_1\times_M M_2$,
состоящее из пар вида $(m, \phi(m))$
для всех $m\in M_1$.
\end{opredelenie}

\begin{zadacha}[!] 
Пусть $M_1\stackrel\phi\arrow M_2$ --- морфизм
между двумя накрытиями $M$, а $\Gamma_\phi$ --- его график.
Докажите, что $\Gamma_\phi$ открыто и замкнуто в 
$M_1\stackrel\phi\arrow M_2$.
\end{zadacha}

\begin{zadacha}
Пусть $[\tilde M:M]$ --- накрытие, причем
$M$ и $\tilde M$ связны (такое накрытие называется {\bf связным}). 
Пусть $X\subset \tilde M\times_M\tilde M$ --- 
связная компонента. Докажите,  что $X$ тогда и только тогда
является графиком автоморфизма 
$\nu:\; \tilde M \arrow \tilde M$, 
когда проекция на первую компоненту задает изоморфизм
$X\cong \tilde M$.
\end{zadacha}

\begin{zadacha}[!]\label{_Mor_v_pro_Zadacha_}
Пусть $[\tilde M:M]$ --- связное накрытие.
Рассмотрим проекцию (по первому аргументу)
$\tilde M\times_M\tilde M \arrow \tilde M$ как накрытие
$\tilde M$. Постройте взаимно однозначное соответствие
между  $\Mor_{\tilde M}(\tilde M, \tilde M\times_M\tilde M)$
и множеством автоморфизмов $\tilde M$ над $M$.
\end{zadacha}

\begin{ukazanie}
Воспользуйтесь предыдущей задачей.
\end{ukazanie}

\begin{opredelenie} 
Пусть $[\tilde M:M]$ --- накрытие, причем
$M$ и $\tilde M$ связны. Тогда
$[\tilde M:M]$ называется {\bf накрытием Галуа},
если накрытие $\tilde M\times_M\tilde M \arrow \tilde M$
расщепляется. В такой ситуации группа автоморфизмов
 $\tilde M$ над $M$ называется {\bf группой Галуа
накрытия $[\tilde M:M]$} (обозначается 
$\Gal([\tilde M:M])$). Иногда группа Галуа
накрытия называется {\bf группой монодромии},
а по-английски --- {\bf deck transformation group}
(группа перелистывания колоды).
\end{opredelenie}

\begin{zadacha}[!]
Пусть $M$ связно, а $[\tilde M:M]$ --- такое накрытие Галуа, 
что у каждой точки $M$ есть ровно $n$ прообразов
(такое накрытие называется $n$-листным). Докажите, что 
у группы Галуа $[\tilde M:M]$ ровно $n$ элементов.
\end{zadacha}

\begin{ukazanie} Докажите, что $[\tilde M\times_M\tilde M:\tilde M]$
тоже $n$-листное, и воспользуйтесь предыдущей задачей.
\end{ukazanie}

\begin{opredelenie} Пусть группа $G$ действует 
на множестве $S$. Действие называется {\bf свободным},
если для любых $g\in G$,  $s\in S$, $s\neq gs$, если 
$g\neq 1$. Действие называется {\bf транзитивным},
если для любых двух $s_1$, $s_2\in S$, найдется $g\in G$
такой, что $g(s_1)=s_2$. 
\end{opredelenie}

\begin{zadacha}
Пусть $\tilde M \stackrel\pi\arrow M$ --
накрытие, а $G=\Aut_M(\tilde M)$ --- его
группа автоморфизмов. Предположим, что $M$ связно.
Докажите, что для любого $x\in M$ группа $G$ действует свободно
на $\pi^{-1}(x)$.
\end{zadacha}

\begin{zadacha}[!]
Пусть $\tilde M \stackrel\pi\arrow M$ --
накрытие Галуа, а $x\in M$ --- любая точка.
Докажите, что $\Gal([\tilde M:M])$
действует на $\pi^{-1}(x)$ свободно и транзитивно.
\end{zadacha}

\begin{ukazanie}
Установите взаимно однозначное 
соответствие между $\pi^{-1}(X)$
и множеством связных компонент
$\tilde M\times_M\tilde M$ и примените
задачу \ref{_Mor_v_pro_Zadacha_}.
\end{ukazanie}

\begin{zadacha}[!]\label{_transi_na_sloe_Zadacha_}
Пусть $\tilde M \stackrel\pi\arrow M$ --
накрытие, а $x\in M$ --- любая точка. 
Докажите, что $\Aut_M(\tilde M)$
тогда и только тогда транзитивно действует на $\pi^{-1}(x)$,
когда $[\tilde M: M]$ --- накрытие Галуа.
\end{zadacha}

\begin{zadacha} 
Рассмотрим накрытие
$\R^n \arrow \R^n/\Z^n \cong (S^1)^n$.
Докажите, что это накрытие Галуа.
\end{zadacha}

\begin{zadacha} 
Зафиксируем $n\in \Z$.
Рассмотрим $n$-листное накрытие
$S^1 \arrow S^1$, $t \mapsto nt$.
Докажите, что это накрытие Галуа.
\end{zadacha}

\begin{opredelenie} Пусть $M$ --- топологическое пространство, а
$G$ --- группа, действующая на $M$ непрерывными преобразованиями. 
Рассмотрим пространство $G$-орбит
$M/G$. Напомним (см. листок Топология 10), что на $M/G$ 
следующим образом вводится топология:
подмножество $M/G$ открыто тогда и только тогда, когда
его прообраз в $M$ открыт. Множество $M/G$ с этой топологией
называется {\bf факторпространством} $M$ по действию $G$.
\end{opredelenie}

\begin{zadacha}[!]
Пусть $[\tilde M:M]$ --- накрытие, а $G\subset \Aut_M(\tilde M)$ 
действует на $[\tilde M:M]$ автоморфизмами. Докажите, что
это действие свободно, а факторпространство $\tilde M/G$
хаусдорфово и накрывает $M$.
\end{zadacha}

\begin{zamechanie}
Фактор по $G$ играет в теории накрытий Галуа 
ту же роль, что $G$-инварианты в теории расширений Галуа.
\end{zamechanie}

\begin{zadacha}[!]
Пусть $[\tilde M:M]$ --- накрытие,
а $G$ --- его группа автоморфизмов. Докажите, что $\tilde M/G$
изоморфно $M$ тогда и только тогда, когда $[\tilde M:M]$ --
накрытие Галуа.
\end{zadacha}

\begin{ukazanie}
Воспользуйтесь задачей \ref{_transi_na_sloe_Zadacha_}.
\end{ukazanie}

\begin{zadacha}\label{_kompo_rasshe_Zadacha_}
Пусть $M_1\stackrel{\phi_1}\arrow M_2 \stackrel{\phi_2}\arrow M_3$
-- последовательность накрытий, причем
$\phi_i$ сюръективны, а их композиция расщепляется.
Докажите, что $\phi_i$ расщепляются.
\end{zadacha}


В развитие аналогии с теорией Галуа, 
накрытия вида $\tilde M \stackrel \pi \arrow M$
будут в дальнейшем обозначаться $[\tilde M:M]$.


\begin{zadacha}[!]\label{_M_1_times_promezhu_Zadacha_}
Пусть $M_1\arrow M_2 \arrow M_3$ --
последовательность накрытий, причем все $M_i$ связны, а
$[M_1:M_3]$ --- накрытие Галуа. Докажите, что 
$M_1 \times_{M_3} M_2$ расщепляется как расслоение над $M_1$.
\end{zadacha}

\begin{ukazanie} 
Воспользуйтесь  задачей \ref{_kompo_rasshe_Zadacha_}, применив ее к
последовательности
\[ 
  M_1\times_{M_3}M_1\arrow M_1\times_{M_3}M_2 \arrow M_1\times_{M_3}M_3.
\]
\end{ukazanie}

\begin{zadacha}[!]
Пусть $M_1\arrow M_2 \arrow M_3$ --
последовательность накрытий, причем $[M_1:M_3]$
-- накрытие Галуа. Докажите, что
$[M_1:M_2]$ --- накрытие Галуа.
\end{zadacha}

\begin{ukazanie}
Воспользуйтесь  задачей \ref{_kompo_rasshe_Zadacha_}.
\end{ukazanie}

\begin{zadacha} 
Пусть $M_1\arrow M_2 \arrow M_3$ --
последовательность накрытий. Докажите, что 
\[ M_1\times_{M_3} M_1 \cong 
   M_1\times_{M_2} (M_2\times_{M_3}M_2)\times_{M_2}M_1.
\]
\end{zadacha}

\begin{zadacha} Выведите из этого следующее:
если $M_1\arrow M_2 \arrow M_3$ --
последовательность накрытий, причем $[M_1:M_2]$
и $[M_2:M_3]$ --- накрытия Галуа, то $[M_1:M_3]$ --- тоже
накрытие Галуа.
\end{zadacha}

\begin{zadacha} 
Пусть $[\tilde M:M]$ --- накрытие,
$G$ --- его группа Галуа, а $G'\subset G$ --- ее подгруппа.
Рассмотрим фактор $\tilde M/G'$. Докажите, что
$[\tilde M: \tilde M/G']$ --- накрытие Галуа,
с группой Галуа $G'$.
\end{zadacha}

\begin{opredelenie}
Пусть $\tilde M \arrow M$ --- накрытие. 
{\bf Факторнакрытием} $[\tilde M:M]$
называется накрытие $\tilde M'\arrow M$,
заданное вместе с последовательностью накрытий
$\tilde M \arrow \tilde M' \arrow M$, где
$\tilde M \arrow \tilde M'$ сюръективно.
\end{opredelenie}

\begin{zadacha}[!]
(основная теорема теории Галуа)
Пусть $[\tilde M:M]$ --- накрытие Галуа с группой
Галуа $G$. Рассмотрим соответствие, сопоставляющее
подгруппе $G'\subset G$
факторнакрытие $[\tilde M/G':M]$.
Докажите, что это соответствие
устанавливает биекцию между 
множеством подгрупп и множеством
классов изоморфизма факторнакрытий.
\end{zadacha}

\begin{zadacha} 
Пусть $M_1\arrow M_2 \arrow M_3$ --- последовательность
накрытий, причем $[M_1:M_3]$ --- накрытие Галуа.
Рассмотрим естественную проекцию 
\[ M_1 \times_{M_3} M_1 \stackrel\Psi\arrow  M_2 \times_{M_3} M_2.
\]
Пусть $g\in \Gal([M_1:M_3])$, а $e_g\subset M_1 \times_{M_3} M_1$
--  компонента связности 
$\{(m, g(m))\}$ в $M_1 \times_{M_3} M_1$.
Докажите, что $g\in \Gal([M_1:M_2])\subset \Gal([M_1:M_3])$
тогда и только тогда, когда при проекции в
$M_2 \times_{M_3} M_2$ компонента $e_g$
переходит в диагональную компоненту.
\end{zadacha}

\begin{zadacha} 
Пусть $M_1\arrow M_2 \arrow M_3$ --- последовательность
накрытий Галуа. Докажите, что естественная проекция
\[ M_1 \times_{M_3} M_1 \stackrel\Psi\arrow  M_2 \times_{M_3} M_2
\]
задает сюръективный гомоморфизм 
$\Gal([M_1:M_3])\stackrel\psi\arrow \Gal([M_2:M_3])$. 
Докажите, что $\ker\psi=\Gal([M_1:M_2])$.
\end{zadacha}

\begin{ukazanie}
Воспользуйтесь тем, что 
группа Галуа $\Gal([M_i:M_3])$ отождествляется
с множеством связных компонент
$M_i\times_{M_3}M_i$, и примените 
предыдущую задачу.  
\end{ukazanie}

\begin{zadacha}[!]
Пусть $\tilde M \arrow M$ --- накрытие Галуа,
а $G' \arrow \tilde M/G'$ --- биективное соответствие
между факторнакрытиями и подгруппами 
в группе Галуа, построенное выше.
Докажите, что $G'$ является нормальной подгруппой
тогда и только тогда, когда $[\tilde M/G':M]$ --- накрытие Галуа.
\end{zadacha}

%\NewVedomost

%%%%%%%%%%%%%%%%%%%%%%%%%%%%%%%%%%%%%%%%%%%%%%%%%%%%%%%%%%%%
\subs{Накрытия линейно связных пространств}
%%%%%%%%%%%%%%%%%%%%%%%%%%%%%%%%%%%%%%%%%%%%%%%%%%%%%%%%%%%%

\begin{opredelenie}
Пусть $M$ --- метрическое пространство. Напомним, что
{\bf геодезической} в $M$ называется такой путь
$[a,b] \stackrel\gamma\arrow M$, что $d(\gamma(x), \gamma(y))=|x-y|$.
{\bf Длина} геодезической --- это расстояние между ее концами.
Путь называется {\bf кусочно геодезическим}, если
его можно разбить в объединение конечного числа 
геодезических сегментов. {\bf Длина} кусочно
геодезического пути определяется как сумма длин
составляющих этот путь геодезических отрезков.
Мы обозначаем длину пути $\gamma$ через $|\gamma|$.
\end{opredelenie}

\begin{opredelenie}
Пусть $\Gamma$ --- граф, а $M_\Gamma$ --- его топологическое
пространство. Мы говорим, что $\Gamma$ {\bf связен},
если его топологическое пространство связно. 
\end{opredelenie}

\begin{zadacha}[!]
Докажите, что граф связен
тогда и только тогда, когда любые две вершины
соединяются конечной последовательностью ребер.
Докажите, что связный граф линейно связен.
\end{zadacha}

\begin{zadacha}[!]
Пусть $\Gamma$ --- связный граф. 
По построению, на каждом ребре
$r_\alpha\subset M_\gamma$ графа введены 
координаты, отождествляющие его с $[0,1]$.
Пусть $\gamma$ --- кусочно линейный путь в $\Gamma_M$,
то есть путь, составленный из конечного числа отрезков вида 
$[a_i,b_{i}]\stackrel{\phi_i}\arrow [\lambda_i, \mu_i]\subset r_\alpha$, 
где $\phi_i$ линейна. Определим $|\gamma|:= \sum|\lambda_i,\mu_i|$, 
как сумму длин всех отрезков, составляющих этот путь. 
Определим $d(x,y):= \inf |\gamma|$, где $\gamma$ пробегает
все кусочно линейные пути, ведущие из $x$ в $y$. 
Докажите, что $d(x,y)$ задает метрику, и $M_\Gamma$
геодезически связен.
\end{zadacha}

\begin{opredelenie}
Эта метрика называется
{\bf стандартной метрикой на 
топологическом пространстве графа}.
\end{opredelenie}

\begin{opredelenie}
Геодезически связное многообразие $M$ называется
{\bf звездчатым}, если любые две точки $M$ соединяются
единственной геодезической.
\end{opredelenie}

\begin{zadacha} 
Докажите, что любое выпуклое подмножество в $\R^n$
(со стандартной метрикой) звездчатое.
\end{zadacha}

\begin{zadacha}[*]
Найдите на $M=\R^2$ такую метрику, что $M$ геодезически
связно, а из любой точки в любую идет бесконечно
много геодезических.
\end{zadacha}

\begin{zadacha}[*]
Пусть $\Gamma$ --- дерево, то есть 
конечный связный граф, у которого
$n$ вершин и $n-1$ ребро. Докажите, что 
$M_\Gamma$ со стандартной метрикой
звездчатое.
\end{zadacha}

\begin{zadacha}[*]
Пусть $\Gamma$ --- такой конечный граф, что $\Gamma_M$
звездчатое. Докажите, что $\Gamma$ --- дерево.
\end{zadacha}

\begin{zadacha}[!]
Пусть $M$ --- локально компактное,
геодезически связное пространство, 
$\tilde M\stackrel\pi\arrow M$ --- связное 
накрытие, а $x$ и $y$ --- две точки в $\tilde M$. 
Рассмотрим множество $S_{x,y}$ всех путей на $\tilde M$,
соединяющих $x$ и $y$, проекция которых в $M$ кусочно геодезична.
Рассмотрим следующую функцию на $\tilde M \times \tilde M$
\[ \tilde d(x,y) = \inf_{\gamma\in S_{x,y}}|\pi(\gamma)|.\]
Докажите, что это метрика. Докажите, что 
$\tilde d(x,y)\geq d(\pi(x),\pi(y))$. 
\end{zadacha}

\begin{zadacha}[*]
В условиях предыдущей задачи, докажите, что
$\tilde M$ геодезически связно.
\end{zadacha}

\begin{zadacha}  Пусть $M$ --- геодезически связное 
метрическое пространство, а $\tilde M\arrow M$ --- его
накрытие. Докажите, что связная компонента прообраза 
геодезической --- геодезическая в $(\tilde M, \tilde d)$. 
\end{zadacha}

\begin{ukazanie} 
Докажите, что прообраз геодезической 
является геодезической в окрестности каждой точки.
Затем воспользуйтесь неравенством
$\tilde d(x,y)\geq d(\pi(x),\pi(y))$.
\end{ukazanie}

\begin{zadacha}[!]
Пусть  $(M, d)$ --- звездчатое метрическое пространство,
а $\tilde M \stackrel\pi \arrow M$ --- его связное накрытие. 
Пусть, кроме того, $x\in \tilde M$ --- любая точка, а $U_x$ --- множество
точек $y\in M$, которые можно соединить с $x$
геодезической. Докажите, что $U_x$ открыто
и замкнуто в $\tilde M$, и что 
$(U_x, \tilde d)$ звездчатое.
Выведите из этого, что естественная
проекция $\tilde M \stackrel\pi \arrow M$ --
изометрия и гомеоморфизм.
\end{zadacha}

\begin{ukazanie}
Воспользуйтесь предыдущей задачей.
\end{ukazanie}

\begin{zadacha}
Пусть $M=[0,1]\times [0,1]$ --- квадрат, а 
$\tilde M \arrow M$ --- его связное накрытие. Докажите, что
это гомеоморфизм.
\end{zadacha}

\begin{zadacha}
Пусть $M$ --- линейно связное и односвязное пространство,
а $\tilde M \stackrel\pi\arrow M$ --- связное накрытие. Докажите, что
это гомеоморфизм.
\end{zadacha}

\begin{ukazanie}
Докажите, что $\tilde M$ линейно связно.
Пусть $x, y\in \pi^{-1}(x_0)$ --- две точки, а
$\tilde \gamma$ --- путь, который их соединяет.
Тогда $\gamma:=\pi(\tilde\gamma)$ это петля.
Поскольку $M$ односвязно, $\gamma$
продолжается до отображения из квадрата в
$X\subset M$ (докажите это).
Рассмотрим прообраз этого квадрата в 
$\tilde M$, и пусть $\tilde X$ компонента прообраза, 
которая содержит $\tilde\gamma$. Воспользовавшись
предыдущей задачей, докажите, что 
$\tilde X\stackrel\pi\arrow X$
это гомеоморфизм, и выведите из этого,
что $x=y$.
\end{ukazanie}

\begin{zadacha}
В условиях предыдущей задачи, докажите, что
любое накрытие $M$ расщепляется.
\end{zadacha}

\begin{opredelenie}
Пусть $M$ --- любое (не обязательно линейно связное)
связное топологическое пространство. $M$ 
называется {\bf односвязным}, если 
любое накрытие $M$ расщепляется.
\end{opredelenie}

\begin{zamechanie} 
В силу предыдущей задачи, это определение
согласовано с определением односвязности
для линейно связных топологических пространств,
данным в листке Топология 8.
\end{zamechanie}

\begin{opredelenie}
Пусть $M$ связно.
Накрытие $\tilde M \arrow M$ 
называется {\bf универсальным}, если оно односвязно.
\end{opredelenie}

\begin{zadacha}[!] 
\енум
\итем Докажите, что 
универсальное накрытие есть накрытие Галуа.
\итем Докажите, что универсальное накрытие единственно
с точностью до изоморфизма.
\ее
\end{zadacha}

\begin{ukazanie} Пусть $\tilde M$, $\tilde M'$ --- два
универсальных накрытия $M$. Поскольку 
$\tilde M\times_M\tilde M'$ является накрытием
$\tilde M$, $\tilde M'$, оно расщепляется над $\tilde M$,
$\tilde M'$. Это значит, что любая связная компонента
$\tilde M\times_M\tilde M'$ изоморфно проектируется в 
$\tilde M$, $\tilde M'$.
\end{ukazanie}


\begin{zadacha}
Пусть $M_1\arrow M_2$ и $M_2\arrow M_3$ --- накрытия.
\begin{enumerate}
\итем[**] Верно ли, что композиция $M_1\arrow M_3$ --- тоже накрытие?

\итем[!] Пусть у каждой точки $M_3$ есть односвязная окрестность.
Докажите, что  $M_1\arrow M_3$ --- накрытие.
\end{enumerate}
\end{zadacha}


%%%%%%%%%%%%%%%%%%%%%%%%%%%%%%%%%%%%%%%%%%%%%%%%
\subs{Существование универсального накрытия}
%%%%%%%%%%%%%%%%%%%%%%%%%%%%%%%%%%%%%%%%%%%%%%%%

\begin{zadacha}
Пусть $M$ линейно связно, 
$\tilde M \overset{\pi}{\arrow} M$ --- связное накрытие, а 
$x\in M$ --- любая точка. 
Докажите, что мощность множества $\pi^{-1}(x)$ 
не больше, чем мощность $\pi_1(M)$.
\end{zadacha}

\begin{zadacha}
Докажите, что 
мощность $\pi^{-1}(x)$  не больше, чем 
мощность множества $M^{[0,1]}$ отображений из $[0,1]$ в $M$.
\end{zadacha}


\begin{zadacha}[*]
Пусть $\tilde M \overset{\pi}{\arrow} M$ --- связное
накрытие связного $M$, а $x\in M$ любая точка. 
Докажите, что
мощность $\pi^{-1}(x)$ не больше, чем
$|2^{2S}|$, где $|2^{2S}|$ --- мощность множества
подмножеств $S\times S$.
\end{zadacha}

\begin{ukazanie}
Выберем $x_1, x_2\in \pi^{-1}(x)$. Докажите, что
найдется набор таких связных открытых множеств $\{\tilde U_\alpha\}\in \pi^{-1}(S)$,
что $\tilde U_{\alpha_0}$ пересекается с объединением
всех $\tilde U_\alpha$, не равных $U_{\alpha_0}$, причем
\[ \{x_1 ,x_2\}= \pi^{-1}(x)\cap \left(\bigcup \tilde U_\alpha\right).\]
Сужая базу $S$, если необходимо, можно предположить,
что $\pi$ расщепляется над  $\pi(U_\alpha)$ для всех
$\alpha$. Докажите, что $x_2$ задается однозначно,
если задано $x_1$, $\{\pi(U_\alpha)\}$, и отмечено,
какие из $U_{\alpha}$ пересекаются.
\end{ukazanie}

\begin{zadacha} 
Пусть $M$ связное, а $V$ --- множество заданной ниже мощности.
Обозначим через ${\cal R}$ множество всех топологий,
заданных на каком-то подмножестве 
$X\subset M \times V$ таким образом, что 
естественная проекция $X\arrow M$
является накрытием. Докажите, что
любое связное накрытие $M$ изоморфно какому-то
элементу ${\cal R}$, если
\begin{enumerate}
\итем $M$ линейно связно, а мощность $V$ 
равна $|M^{[0,1]}|$

\итем[*] Мощность 
$V$ равна $|2^{2S}|$, где $S$ --- база топологии в $M$.
\end{enumerate}
\end{zadacha}

\begin{zamechanie}
Эта задача позволяет говорить о ``множестве классов
изоморфизма накрытий''. Напомним, что не все
математические объекты являются множествами; так,
множеством не является класс всех множеств.
Чтобы доказать, что какой-то класс является
множеством, надо ограничить его мощность.
\end{zamechanie}

\begin{opredelenie}
Пусть $\{M_\alpha\stackrel{\pi_\alpha}\arrow M\}$ --- набор
отображений на $M$,
проиндексированный набором индексов $I$ (возможно,
бесконечным, или даже несчетным).
Рассмотрим множество всех таких 
$(m_{\alpha_1}, m_{\alpha_1},\dots)\in \prod M_{\alpha}$,
что $\pi_{\alpha}(m_\alpha)=m$ для какого-то 
$m\in M$. Это множество называется {\bf расслоенным
произведением $\{M_{\alpha}\}$} и обозначается
$\prod_M M_{\alpha}$.
\end{opredelenie}

\begin{zadacha}
Пусть $M$ --- топологическое пространство, а
$\{M_\alpha\stackrel{\pi_\alpha}\arrow M\}$ --- набор его накрытий.
Введем на $\prod_M M_{\alpha}$ топологию следующим образом. Пусть
$U\subset M$ открыто, а $\{U_\alpha\subset M_{\alpha}\}$ --
набор открытых множеств, накрывающих $U$. Докажите, что
множества вида $\prod_U U_\alpha\subset \prod_M M_{\alpha}$
задают базу топологии на $\prod_M M_{\alpha}$.
Докажите, что $\prod_M M_{\alpha}$ хаусдорфово
\begin{enumerate}
\итем[*] Верно ли, что естественная проекция 
$\prod_M M_{\alpha}\arrow M$ --- накрытие?

\итем[!] Предположим, что у каждой точки $M$
найдется односвязная окрестность. Докажите, что
естественная проекция $\prod_M M_{\alpha}\arrow M$ --- накрытие.
\end{enumerate}
\end{zadacha}

\begin{opredelenie}
В такой ситуации $\prod_M M_{\alpha}$ называется
{\bf расслоенным произведением $M_{\alpha}$ над $M$}, либо
просто произведением накрытий
$M_\alpha\stackrel{\pi_\alpha}\arrow M$.
\end{opredelenie}

\begin{zadacha} \label{_rascheplya_product_Zadacha_}
Пусть все накрытия $\{M_\alpha\stackrel{\pi_\alpha}\arrow M\}$
расщепляются. Докажите, что $\prod_M M_{\alpha}$ тоже расщепляется.
\end{zadacha}

\begin{zadacha}[!]
Пусть $\{M_\alpha\stackrel{\pi_\alpha}\arrow M\}$
-- накрытия Галуа. Докажите, что любая компонента связности
их произведения над $M$ --- тоже накрытие Галуа.
\end{zadacha}

\begin{ukazanie}
Воспользуйтесь задачей \ref{_rascheplya_product_Zadacha_}.
\end{ukazanie}

\begin{zadacha} 
Пусть $\tilde M$ --- накрытие $M$.
Постройте естественную биекцию между множествами
$\Mor(\prod_M M_{\alpha}, \tilde M)$
и $\prod \Mor(M_{\alpha}, \tilde M)$
\end{zadacha}

\begin{zadacha}[*]
Пусть $\{M_\alpha\stackrel{\pi_\alpha}\arrow M\}$ --
множество всех накрытий $S^1 \arrow S^1$, $t\arrow nt$,
проиндексированных $n\in \Z$. Докажите, что любая
связная компонента $\prod_M M_{\alpha}$ изоморфна
$\R\arrow \R/\Z = S^1$.
\end{zadacha}

\begin{zadacha} 
Пусть $\tilde M\stackrel\pi\arrow M$ --- накрытие, причем $\tilde M$ и $M$
связные, $x\in M$, $x_1, x_2 \in \pi^{-1}(x)$, $W$ --- компонента
связности $\tilde M\times_M\tilde M$, содержащая $x_1\times x_2$,
а $W_1$ --- компонента связности $\tilde M\times_M\tilde M\times_M\tilde M$, 
содержащая $x_1\times x_2\times x_2$. Докажите, что
естественная проекция $W_1 \arrow W$ (забывание третьего
аргумента) это изоморфизм. 
\end{zadacha}

\begin{zadacha} 
В такой же ситуации, пусть 
$\{x_\alpha\}$ --- набор точек в $\pi^{-1}(x)$,
проиндексированных $\alpha\in I$, $W$ --
соответствующая компонента в расслоенном 
произведении $\prod_{M,I}\tilde M$ $I$ копий $\tilde M$, а $W_1$ --- 
компонента в 
$\left(\prod_{M,I}\tilde M\right)\times_M\tilde M$,
содержащая $\{x_\alpha\}$ и $x_0$, причем $x_0\in \{x_\alpha\}$.
Докажите, что естественная проекция $W_1 \arrow W$ это
изоморфизм.
\end{zadacha}

\begin{zadacha}[!]
Пусть $\tilde M \stackrel \pi \arrow M$ --
связное накрытие, а $x\in M$. Рассмотрим произведение 
$\prod_{M,\{\pi^{-1}(x)\}}\tilde M$ 
$\tilde M$ с собой, проиндексированное множеством 
$\pi^{-1}(x)$, и пусть $\tilde M_G$ --- связная компонента 
в $\prod_{M,\{\pi^{-1}(x)\}}\tilde M$, содержащая
произведение всех $x_\alpha \in \{\pi^{-1}(x)\}$.
Докажите, что $\tilde M_G\times_M \tilde M$
расщепляется над $\tilde M_G$. Докажите, что
$\tilde M_G \arrow M$ --- накрытие Галуа.
\end{zadacha}

\begin{zamechanie}
Мы доказали, что любое
накрытие является фактор-накрытием накрытия Галуа.
\end{zamechanie}

\begin{zadacha}  
Пусть $M$ --- связное топологическое пространство, 
${\cal R}$ --- множество всех классов изоморфизма связных
накрытий $M$, $\{M_\alpha\stackrel{\pi_\alpha}\arrow M\}$ --
соответствующий набор накрытий, а 
$\tilde M\subset \prod_M M_{\alpha}$ --
компонента связности их произведения.
Докажите, что для каждого связного 
накрытия $\tilde M' \arrow \tilde M$
найдется сюръективный морфизм накрытий
$\tilde M \arrow \tilde M'$. 
\end{zadacha}

\begin{ukazanie} Воспользуйтесь предыдущей задачей.
\end{ukazanie}

\begin{zadacha}
В условиях предыдущей задачи,
докажите, что $\tilde M$ --- накрытие Галуа.
\end{zadacha}

\begin{zadacha}[!]
Выведите из этого, что для любого $\tilde M \arrow M$, накрытие
$\tilde M \times_M \tilde M'\arrow \tilde M$ расщепляется.
\end{zadacha}

\begin{ukazanie} Воспользуйтесь
\ref{_M_1_times_promezhu_Zadacha_}.
\end{ukazanie}

\begin{zadacha}[!]
Пусть $M$ --- любое связное
топологическое пространство, а 
$\tilde M \arrow M$ --- накрытие Галуа, построенное выше.
Докажите, что $\tilde M$ односвязно.
\end{zadacha}

\begin{zamechanie}
Мы получили, что у любого связного
топологического пространства найдется
универсальное накрытие. Как было выше
доказано, универсальное накрытие единственно.
\end{zamechanie}

\begin{zadacha}[!]
Пусть $M$ линейно связно, $\tilde M$ --- его универсальное
накрытие, а $\Gal([\tilde M:M])$ --- соответствующая группа
Галуа. Докажите, что $\Gal([\tilde M:M])$ изоморфно
фундаментальной группе $M$.
\end{zadacha}

\begin{opredelenie}
{\bf Фундаментальной группой} топологического
пространства $M$ называется группа $\pi_1(M):=\Gal([\tilde M:M])$,
где $\tilde M$ --- универсальное накрытие.
\end{opredelenie}

\begin{opredelenie} 
Подгруппы $G_1, G_2\subset G$ называются {\bf
сопряженными}, если найдется такой $g\in G$,
что $G_1$ переводится в $G_2$ автоморфизмом
$x\mapsto x^g$.
\end{opredelenie}

\begin{zadacha}[*]
Пусть $M_1\arrow M$ --- некоторое накрытие, а 
$\tilde M \arrow M_1\arrow M$ --- универсальное накрытие.
Рассмотрим подгруппу $G_1\subset \Gal([\tilde M:M])=\pi_1(M)$, 
полученную в результате применения 
основной теоремы теории Галуа. Докажите, что это соответствие
задает биекцию между классами изоморфизма накрытий $M$
и классами сопряженности подгрупп в $\pi_1(M)$. 
\end{zadacha}

\begin{zadacha}[!]
Найдите все накрытия окружности, с точностью до изоморфизма. Постройте их явно.
\end{zadacha}

\begin{zadacha}[*]
Пусть $M$ --- связное топологическое пространство,
все компоненты линейной связности которого односвязны.
Может ли оно иметь нетривиальную фундаментальную группу?
\end{zadacha}

\begin{zadacha}[*]
Пусть $B$ --- множество полиномов 
$P(t)= t^n+ a_{n-1}t^{n-1}+ a_{n-2}t^{n-2}+ \dots + a_0$
над $\C$, у которых все корни разные, а $B_1$ --- множество
всех наборов $(x_1,\dots, x_n)\in \C^n$ попарно различных чисел
$x_i \in \C$. Введем на $B$ и $B_1$ естественную
топологию подмножества в $\C^n$. Рассмотрим отображение
$B_1\stackrel\pi\arrow B$, $(x_1,\dots, x_n)\arrow \prod(t-x_i)$.
Докажите, что $\pi$ --- накрытие Галуа. Найдите его
группу Галуа.
\end{zadacha}

\begin{zadacha}[*]
Постройте связное накрытие, которое не будет накрытием Галуа.
\end{zadacha}

%%%%%%%%%%%%%%%%%%%%%%%%%%%%%%%%%%%%%%%%%%%%%%%%%%%%%%%%%%%%

\chapter[Листок 10: Фундаментальная группа и гомотопии]%
{Листок 10: Фундаментальная\\ группа  и гомотопии}

%%%%%%%%%%%%%%%%%%%%%%%%%%%%%%%%%%%%%%%%%%%%%%%%%%%%%%%%%%%%

%%%%%%%%%%%%%%%%%%%%%%%%%%%%%%%%%%%%%%%%%%%%%%%%
\subs{Гомотопии}
%%%%%%%%%%%%%%%%%%%%%%%%%%%%%%%%%%%%%%%%%%%%%%%%

Все топологические пространства в этом листочке
предполагаются локально линейно связными и хаусдорфовыми,
если не оговорено противного.

\begin{opredelenie}
Пусть $f_1, f_2:\; X \arrow Y$ --- непрерывные отображения
топологических пространств. Напомним, что {\bf гомотопией}
между $f_1$ и $f_2$ называется такое непрерывное отображение
$F:\; [0,1]\times X \arrow Y$, что
$F\restrict{\{0\}\times X}$ равно $f_1$, а
$F\restrict{\{1\}\times X}$ равно $f_2$. 
\end{opredelenie}

\begin{zadacha}
Докажите, что гомотопные отображения индуцируют
один и тот же гомоморфизм $\pi_1(X)\arrow \pi_1(Y)$.
\end{zadacha}

\begin{opredelenie}
Пусть $f:\; X \arrow Y$, $g:\; Y \arrow X$ --
непрерывные отображения топологических пространств,
причем $f\circ g$ и $g\circ f$ гомотопны тождественным
отображениям из $X$ в $X$ и из $Y$ в $Y$. 
Такие отображения называются {\bf гомотопическими
эквивалентностями}, а $X$ и $Y$ --- {\bf гомотопически
эквивалентными}.
\end{opredelenie}

\begin{zadacha} 
Докажите, что композиция гомотопических эквивалентностей
отображений есть гомотопическая эквивалентность.
Докажите, что гомотопическая эквивалентность пространств
есть отношение эквивалентности.
\end{zadacha}

\begin{zadacha}[!]
Пусть $f:\; X \arrow Y$ --- гомотопическая эквивалентность.
Докажите, что $f$ индуцирует изоморфизм фундаментальных групп.
\end{zadacha}

\begin{zadacha}
Пусть $X\subset Y$ --- 
деформационный ретракт. Докажите, что $X$ и $Y$
гомотопически эквивалентны.
\end{zadacha}

\begin{zadacha}[!]
Пусть $X$ --- топологическое пространство. Докажите, что
$X$ стягиваемо тогда и только тогда, когда оно
гомотопически эквивалентно точке.
\end{zadacha}

\begin{zadacha}[!]
Дан связный граф $\Gamma$, у которого
$n$ ребер и $n$ вершин. Докажите, что его топологическое
пространство гомотопически эквивалентно окружности.
\end{zadacha}

\begin{zadacha}[!]
Пусть $M$ --- связное топологическое
пространство, а $x,x', y, y'\in M$ --- любые точки. 
Докажите, что соответствующие пространства путей 
$\Omega(M,x,x')$ и $\Omega(M,y,y')$ гомотопически
эквивалентны.
\end{zadacha}

\begin{ukazanie} 
Выберите путь $\gamma_{xy}$, соединяющий
$x$ и $y$ и путь $\gamma_{x'y'}$, соединяющий
$x'$ и $y'$. Пусть $\gamma^{-1}_{xy}(t)=\gamma_{xy}(1-t)$ 
и $\gamma^{-1}_{x'y'}(t)=\gamma_{x'y'}(1-t)$. 
Рассмотрим отображение 
$f:\Omega(M,x,x')\arrow\Omega(M,y,y')$
переводящее любой путь $\gamma\in \Omega(M,x,x')$
в композицию $\gamma^{-1}_{xy}\gamma\gamma_{x'y'}$, и аналогичное
отображение 
$g:\Omega(M,y,y')\arrow\Omega(M,x,x')$,
переводящее $\gamma\in \Omega(M,y,y')$ в 
$\gamma_{xy}\gamma\gamma^{-1}_{x'y'}$.
Докажите, что $fg$ гомотопно тождественному
и $gf$ гомотопно тождественному.
\end{ukazanie}

%%%%%%%%%%%%%%%%%%%%%%%%%%%%%%%%%%%%%%%%%%%%%%%%%%%%%%%%%%%%
\subs{Пространство путей на локально стягиваемых пространствах}
%%%%%%%%%%%%%%%%%%%%%%%%%%%%%%%%%%%%%%%%%%%%%%%%%%%%%%%%%%%%

\begin{opredelenie}
Пусть $M$ --- топологическое пространство. $M$
называется {\bf локально стягиваемым}, если 
у каждой точки есть стягиваемая окрестность.
\end{opredelenie}

\begin{zadacha} Пусть
$M$ --- локально стягиваемое топологическое пространство.
Докажите, что $M$ локально линейно связно.
\end{zadacha}

\begin{zadacha}[*]
Пусть $M$ --- такое геодезически связное метрическое пространство,
что для какого-то $\delta>0$ любые две точки,
отстоящие на расстояние $<\delta$, соединяются
единственной геодезической. Докажите, что $M$
локально стягиваемо.
\end{zadacha}

\begin{zadacha}
Докажите, что любой граф локально стягиваем.
\end{zadacha}

\begin{opredelenie} 
Топологическое пространство $M$ 
называется {\bf многообразием размерности $n$}, если 
у любой точки найдется окрестность, гомеоморфная
открытому шару в $\R^n$.
\end{opredelenie}

\begin{zamechanie}
Многообразия, очевидно, локально стягиваемы.
\end{zamechanie}

\begin{zadacha}[!]
Докажите, что сфера $S^n$ --- это многообразие.
\end{zadacha}

\begin{ukazanie} 
Воспользуйтесь стереографической проекцией.
\end{ukazanie}

\begin{zadacha}
Пусть $M$ стягиваемое, $x,y\in M$. Докажите,
что все пути $\gamma\in \Omega(M, x,y)$ гомотопны.
\end{zadacha}

\begin{zadacha}[!]
Пусть $\gamma\in \Omega(M, x,y)$ --- путь в локально стягиваемом
пространстве $M$, а $\{U_\alpha\}$ --- множество стягиваемых открытых
множеств на $M$. Выберем в $\{U_\alpha\}$ конечное подмножество,
покрывающее $\gamma$ (это можно сделать, потому что $\gamma$
компактен).  Пусть $V_1,\dots,V_n$ --- соответствующее покрытие
$[0,1]$ связными интервалами, где каждый $V_i$ является связной
компонентой $\gamma^{-1}(U_i)$, а все $U_i$ стягиваемы. Упорядочим
$V_i$ таким образом, что $V_i$ и $V_{i+1}$ пересекаются в точке
$t_i$, и пусть $a_i := \gamma(t_i)$. Докажите, что любой путь
$\gamma'\in \Omega(M, x,y)$, такой, что $\gamma'(t_i)=a_i$, и
$\gamma'([t_i, t_{i+1}])\subset U_i$, гомотопен $\gamma$.
\end{zadacha}

\begin{ukazanie} Воспользуйтесь предыдущей задачей.
\end{ukazanie}

\begin{zadacha}[!] \label{_gomoto_bli_petli_Zadacha_}
Пусть $M$ --- локально стягиваемое топологическое
пространство, а $\gamma\in \Omega(M, x,y)$ --- некоторый
путь. Докажите, что у $\gamma$ найдется такая окрестность
${\mathcal U}\subset \Omega(M, x,y)$, что все 
$\gamma'\in {\mathcal U}$ гомотопны.
\end{zadacha}

\begin{ukazanie}
Воспользуйтесь предыдущей задачей.
\end{ukazanie}

\begin{zamechanie} Заметим, что на компактных 
многообразиях размерности $>1$ существует петли,
задаваемые сюръективным отображением; 
пример такой петли легко построить
тем же методом, что кривую Пеано.
\end{zamechanie}

\begin{zadacha}[!]
Пусть $M$ это многообразие (например, сфера) 
размерности больше $1$, а $\gamma\in \Omega(M, x)$ --- 
петля. Докажите, что $\gamma$ гомотопна петле, которая 
не сюръективна.
\end{zadacha}

\begin{ukazanie}
Воспользуйтесь предыдущей задачей.
\end{ukazanie}

\begin{zadacha}[!]
Пусть $n>1$. Докажите, что $n$-мерная сфера односвязна.
\end{zadacha}

\begin{ukazanie}
Пусть $\gamma$ --- петля на сфере.
Воспользовавшись предыдущей задачей, прогомотопируйте
$\gamma$ в петлю, которая отображает 
$[0,1]$ в $S^n\backslash \{x\}$, где $x$ некоторая
точка. Докажите, что сфера без точки гомеоморфна
$\R^n$, в частности стягиваема.
\end{ukazanie}

\begin{zadacha}[*] \label{_puti_iz_styagi_Zadacha_}
Пусть $M$ стягиваемо, а
$F:\; M\times [0,1]\arrow M$ --- гомотопия тождественного
отображения в постоянное отображение $M \to y \in M$. 
Рассмотрим следующее отображение
$M\arrow \Omega(M, y, *)$, $t, m \arrow F(m, t)$ 
($t\in[0,1]$, $m\in M$). Докажите, что оно непрерывно.
\end{zadacha}

\begin{zadacha} 
Пусть $M$ локально стягиваемо, $x, y\in M$ --- две точки,
$\gamma\in \Omega(M, x, y)$ --- некоторый путь. Докажите, что
у $\gamma$ есть такая окрестность ${\mathcal U}\in \Omega(M, x, *)$,
что все пути $\gamma'\in {\mathcal U}$, соединяющие
$x$ и $a$, гомотопны в $\Omega(M, x, a)$.
\end{zadacha}

\begin{zadacha}[*]
Пусть $M$ --- локально стягиваемое топологическое
пространство, $x \in M$ --- точка, а $\Omega(M, x, *)$ 
-- множество всех путей,
начинающихся в точке $x$, снабженное открытокомпактной топологией. 
Рассмотрим такое отношение эквивалентности на 
$\Omega(M,x, *)$: $\gamma\sim\gamma'$, если
$\gamma$ и $\gamma'$ соединяют $x$ и $y$,
и гомотопны в $\Omega(M, x, y)$. Рассмотрим
$\Omega(M, x, *)/\sim$, с топологией фактора.
Выберем стягиваемую окрестность $U_y\ni y$,
и пусть $U_y\stackrel F \arrow \Omega(U_y, y, *)$ --
отображение, построенное в задаче 
\ref{_puti_iz_styagi_Zadacha_}. 
Пусть $\gamma\in \Omega(M, x, y)$ --
некоторый путь, а $U_y \stackrel\Psi\arrow \Omega(M, x, *)$ --
отображение, ставящее $a\in U_y$ путь
$\gamma F(a)$ (то есть путь, заданный
на $[0, 1/2]$ как $t \arrow \gamma(2t)$,
и на $[1/2,1]$ как $F(a, 2t-1)$.
Докажите, что (для достаточно маленького 
$U_y$) $\Psi$ в композиции с 
$\Omega(M, x, *)\stackrel\pi\arrow\Omega(M, x, *)/\sim$ --
это гомеоморфизм $U_y$ на некоторое открытое
подмножество в $\Omega(M, x, *)/\sim$.
\end{zadacha}

\begin{ukazanie}
Непрерывность $\Psi\circ\pi$  
очевидна по конструкции, а инъективность
следует из предыдущей задачи. Чтобы убедиться, что
$\Psi\circ\pi$  задает гомеоморфизм $U_y$ на
$\Psi\circ\pi(U_y)$, нам нужно доказать, что
$\Psi\circ\pi$ переводит открытые множества
в открытые. Это ясно из того, что
естественное отображение $\Omega(M, x, *)/\sim\arrow M$,
$\gamma'\arrow \gamma'(1)$, непрерывно,
и индуцирует гомеоморфизм $U_y$ на образ.
\end{ukazanie}

\begin{zadacha}[*]
Рассмотрим отображение $\Omega(M, x, *)/\sim\arrow M$,
ставящее в соответствие пути $\gamma\in \Omega(M, x, y)$
точку  $y=\gamma(1)$. Докажите, что это накрытие.
\end{zadacha}

\begin{ukazanie}
Воспользуйтесь предыдущей задачей.
\end{ukazanie}

\begin{zadacha}[!]
Докажите, что $\Omega(M, x, *)$ стягиваемо.
\end{zadacha}

\begin{zadacha}[*]
Пусть $\gamma$ --- путь в $\Omega(M, x, *)/\sim$.
Докажите, что $\gamma$ гомотопен образу некоторого пути из
$\Omega(M, x, *)$.
\end{zadacha}

\begin{ukazanie}
Докажите, что $\gamma$ можно поднять до пути 
в $\Omega(M, x, *)$ локально, и воспользуйтесь
тем, что для каждой точки в $\Omega(M, x, *)/\sim$ ее
прообраз в $\Omega(M, x, *)$ связен.
\end{ukazanie}

\begin{zadacha}[*]
Выведите из этого, что $\Omega(M, x, *)/\sim$ односвязно.
\end{zadacha}

\begin{zamechanie}
Пусть $(M, x)$ --- локально стягиваемое топологическое
пространство с отмеченной точкой. 
Универсальное накрытие $M$ можно таким образом
отождествить с множеством пар ($y\in M$, класс гомотопии
путей $\gamma\in \Omega(M,x,y)$).
\end{zamechanie}

%%%%%%%%%%%%%%%%%%%%%%%%%%%%%%%%%%%%%%%%%%%%%%%%
\subs{Свободная группа и букет}
%%%%%%%%%%%%%%%%%%%%%%%%%%%%%%%%%%%%%%%%%%%%%%%%

\begin{opredelenie}
Пусть $(M_1, x_1)$, $(M_2, x_2)$, $(M_3, x_3)$, $\ldots$ --
набор (возможно, бесконечный) связных топологических
пространств с отмеченной точкой. Рассмотрим
факторпространство несвязного объединения всех 
$(M_\alpha, x_\alpha)$ по соотношению эквивалентности
$\{x_1\}\sim \{x_2\}\sim \{x_3\}\sim \dots$
Это факторпространство называется
{\bf букетом}, обозначается 
$\bigvee_\alpha (M_\alpha,x_\alpha)$.
Также букет обозначается 
$(M_1, x_1)\vee(M_2, x_2)\vee(M_3, x_3)\vee \dots$
\end{opredelenie}

\begin{zadacha} 
Пусть все $M_\alpha$ связные (линейно связные,
хаусдофовы). Докажите, что их букет связен (линейно 
связен, хаусдорфов).
\end{zadacha}

\begin{zadacha}[!]
Пусть все $M_\alpha$ связные и односвязные.
Докажите, что их букет односвязен.
\end{zadacha}

\begin{zadacha}[!]
Пусть $\Gamma$ --- связный граф, у которого $n$ вершин и
$n+k-1$ ребер. Докажите, что его топологическое
пространство $M_\Gamma$ гомотопически эквивалентно
букету $k$ окружностей.
\end{zadacha}

\begin{ukazanie} 
Пусть у $\Gamma$ есть ребро $r$, соединяющее две разные
вершины $v_1, v_2$,  Рассмотрим граф $\Gamma'$,
у которого $n-1$ вершин и $n+k-2$ ребер, полученный
из $\Gamma$ следующим образом. Из $\Gamma$
выкидывается ребро $r$, а вершины $v_1$ и $v_2$
склеиваются в одну. Докажите, что $M_\Gamma$
и $M_{\Gamma'}$ гомотопически эквивалентны.
\end{ukazanie}

\begin{opredelenie}
Зададим множество $\{a_1, a_2, \dots\}$ мощности 
$N$ ($N$ может быть как конечным кардиналом, так
и бесконечным). {\bf $N$-арное дерево} $D_N$ --- это бесконечный
граф, который определяется следующим образом.
Вершины $D_N$ --- конечные последовательности
из символов $a_i$. Ребрами соединяются
вершины, соответствующие $A_1A_2 \dots A_k$
и $A_1A_2 \dots A_kA_{k+1}$ (все $A_i$ принадлежат
$\{a_1, a_2, \dots\}$).
\end{opredelenie}

\begin{zadacha} Докажите, что в каждую вершину 
$D_N$ входят $N+1$ ребер.
\end{zadacha}

\begin{zadacha}[!]
Пусть $M_N$ --- топологическое пространство 
$N$-арного дерева, с естественной метрикой, построенной в
начале этого листка.  Докажите, что $M_N$ является звездчатым
(любые две точки соединяются единственной геодезической).
Докажите, что оно стягиваемо.
\end{zadacha}

\begin{zadacha}[!]
Рассмотрим $2N-1$-арное дерево.
Раскрасим его ребра в $N$ цветов, таким образом,
что к каждой вершине сходится по 2 ребра каждого цвета.
Рассмотрим букет из $N$ окружностей, и раскрасим
каждую из окружностей в свой цвет. 
Рассмотрим отображение из $M_{2N-1}$ в букет из $N$
окружностей, переводящее вершины графа 
в вершины букета, а ребро цвета $a_i$ в 
окружность такого же цвета. Докажите, что
это универсальное накрытие.
\end{zadacha}

\begin{zadacha} 
Пусть $\{a_1, a_2, \dots\}$ --- множество мощности 
$N$, а ${\mathcal W}$ --- множество конечных последовательностей
(``слов'') из символов $a_i$, $a_j^{-1}$, в которых нигде 
не встречаются подряд $a_ia_i^{-1}$, а также
$a_i^{-1}a_i$. Последовательность длины 
0 обозначается $e$. Мы умножаем слова, записывая
одно за другим и зачеркивая последовательно
все $a_ia_i^{-1}$, $a_i^{-1}a_i$, которые
встречаются подряд. Докажите, что ${\mathcal W}$
образует группу.
\end{zadacha}

\begin{opredelenie}
Эта группа называется {\bf свободной группой, порожденной 
образующими $\{a_1, a_2, \dots\}$}, обозначается $F_N$.
\end{opredelenie}

\begin{zadacha}
Докажите, что $F_1$ изоморфно $\Z$.
\end{zadacha}

\begin{zadacha}[!]
Пусть $G$ --- любая группа, а
$\{g_1, g_2, \dots \}$ --- набор элементов
из $G$, пронумерованный $\{a_1, a_2, \dots\}$.
Докажите, что существует единственный гомоморфизм $F_N\arrow G$,
переводящий $a_i$ в $g_i$.
\end{zadacha}

\begin{zadacha}[!] Постройте свободное действие $F_N$ 
на топологическом пространстве $M_{2N-1}$ $2N-1$-арного дерева,
транзитивное на вершинах.
\end{zadacha}

\begin{zadacha}[!] Докажите, что 
$M_{2N-1}/F_N$ --- букет $N$ окружностей,
а фундаментальная группа букета свободна.
\end{zadacha}

\begin{zadacha}[!]
Докажите, что любой (возможно, бесконечный) граф гомотопически
эквивалентен букету окружностей.
\end{zadacha}

\begin{zadacha}[!]
Выведите из этого, что любая подгруппа свободной группы
свободна.
\end{zadacha}

\begin{ukazanie}
Воспользуйтесь теорией Галуа для накрытий.
\end{ukazanie}

\begin{zadacha}[*] 
Пусть $G_1,G_2,\dots$ --- какой-то набор
групп. Рассмотрим множество ${\mathcal W}$ 
конечных последовательностей неединичных 
элементов из разных $G_i$, таких, что 
элементы одной и той же группы
нигде не идут подряд. Если дана
любая последовательность $A$ элементов
из $G_i$, из нее можно получить элемент
${\mathcal W}$ следующим способом.
Если в $A$ идут подряд два элемента
из $G_i$, мы их перемножаем и заменяем
эти два элемента на произведение. Если
в $A$ встречается единица одний из групп,
мы ее вычеркиваем. Повторим эту процедуру
столько раз, сколько нужно, чтобы получить
элемент из ${\mathcal W}$. Элементы ${\mathcal W}$
можно перемножать, записав одно слово после
другого и применив вышеописанную процедуру.
Докажите, что получится группа.
\end{zadacha}

\begin{opredelenie}
Эта группа называется {\bf свободным произведением 
групп $G_1$, $G_2$, $\dots$}.
\end{opredelenie}

\begin{zadacha} Докажите, что свободная
группа от $N$ образующих --- это свободное
произведение $N$ копий $\Z$.
\end{zadacha}

\begin{zadacha} Докажите, что свободное произведение
свободных групп свободно.
\end{zadacha}

\begin{zadacha}[*]
Пусть $(M_1, x_1), (M_2, x_2), (M_3, x_3), \dots$ --
набор связных топологических пространств с отмеченной
точкой. Докажите, что фундаментальная группа
букета $\pi_1(\bigvee_\alpha(M_\alpha,x_\alpha))$
изоморфна свободному произведению групп $\pi_1(M_1, x_1),
\pi_1(M_2, x_2), \pi_1(M_3, x_3), \dots$.
\end{zadacha}


}


%%%%%%%%%%%%%%%%%%%%%%%%%%%%%%%%%%%%%%%%%%%%%%%%%%%%%%%%%%%%%%%%%%%%%%%%

\part{Лекции по топологии}

\renewcommand{\PartName}{Часть III. Лекции по топологии}

\renewcommand{\chaptermark}[1]{\markboth{{\bf
  #1}}{{\sc\PartName}}}

%%%%%%%%%%%%%%%%%%%%%%%%%%%%%%%%%%%%%%%%%%%%%%%%%%%%%%%%%%%%%%%%%%%%%%%%


%%%%%%%%%%%%%%%%%%%%%%%%%%%%%%%%%%%%%%%%%%%%%%%%%%%%%%%%%%%%%%%%%%%%%%%%
\chapter{Лекция 1: метрика, пополнение,
  $p$-адические числа}


%%%%%%%%%%%%%%%%%%%%%%%%%%%%%%%%%%%%%%%%%%%%%%%%%%%%%%%%%%%%%%%%%%%%%%%%

\section{Метрические пространства и пополнение}

%%%%%%%%%%%%%%%%%%%%%%%%%%%%%%%%%%%%%%%%%%%%%%%%%%%%%%%%%%%%%%%%%%%%%%%%

Определение и свойства вещественных чисел см.
в приложении в конце этой книги.

Обозначим через $\R^{\geq 0}$ множество всех неотрицательных
вещественных чисел. 

\определение
Пусть $M$ --- множество. {\бф Метрикой} на $M$ называется
функция $d:\; M\times M\arrow \R^{\geq 0}$, удовлетворяющая
следующим условиям
\begin{description}
\item[Невырожденность:]\ \ $d(x,y)=0$ тогда и только тогда,
когда $x=y$.
\item[Симметричность:]\ \  $d(x,y)=d(y,x)$
\item[Неравенство треугольника:]\ \  $d(x,y) \leq d(x, z) + d(z,y)$
\end{description}
для любых точек $x,y,z\in M$.
\ео

Определение метрики весьма точно соответствует
интуитивному представлению о "расстоянии". Аксиоматическое
определение метрического пространства дал Морис
Фреше в 1906-м году, но сам термин "метрическое
пространство" (metrischer Raum) принадлежит Хаусдорфу.
Слово "расстояние" часто используют как синоним "метрике",
особенно в конструкциях типа "расстояние от $x$ до $y$".
Также говорят "расстояние от $x$ до $y$ в метрике $d$".


\begin{figure}[ht]
\begin{center}
\epsfig{file=Frechet.eps,width=0.5\linewidth}\\
Maurice Fr\'echet\\
(1878 --- 1973) 
\end{center}
\end{figure}

\hfill

\пример
Дурацкий пример метрического пространства:
$M$ любое множество, а $d(x,y)=1$ для любых
 точек $x\neq y$. Проверьте аксиомы.

\хфилл

\пример
$\R^n$, со стандартным расстоянием
$d(x,y) = |x-y|$ --- метрическое пространство. Проверьте аксиомы.


\определение
Пусть $M, N$ --- метрические пространства.
Вложение $M \stackrel\iota\hookrightarrow N$ называется
{\бф изометрическим вложением}, если $\iota$ сохраняет
расстояния: $d_M(x,y) = d_N(\iota(x), \iota(y))$,
для любых $x,y\in M$. {\бф Изометричные пространства} --- пространства,
между которыми есть биекция, сохраняющая расстояния.
\ео

\определение
Пусть $x\in M$ точка в метрическом пространстве. Открытый
{\бф $\epsilon$-шар} $B_\epsilon(x)$ с центром в $x$ --- множество всех точек,
отстоящих от $x$ меньше, чем на $\epsilon$:
\[
B_\epsilon(x)= \{ y\in M\ \ | \ \  d(x,y) < \epsilon\}
\]
\ео

\определение
Пусть $M$ --- метрическое пространство. 
Последовательность $\{\alpha_i\}$ точек из $M$ называется
{\бф последовательностью Коши}, если для каждого 
$\epsilon>0$, все элементы последовательности 
$\{\alpha_i\}$, кроме конечного числа, содержатся в некотором
$\epsilon$-шаре. Последовательности Коши 
$\{\alpha_i\}$,  $\{\beta_i\}$ называются 
{\бф эквивалентными}, если последовательность
$\alpha_0, \beta_0, \alpha_1, \beta_1, ...$
является последовательностью Коши. (Докажите, 
что это отношение эквивалентности.)
\ео

\определение
Говорят, что последовательность Коши $\{\alpha_i\}$
{\бф сходится к $x\in M$}, если
$\alpha_0, x, \alpha_1, x, \alpha_2, ...$ --- 
последовательность Коши. В этом случае также говорят,
что $x$ --- это {\бф предел} последовательности
$\{\alpha_i\}$. Метрическое пространство $M$
называется {\бф полным}, если у любой
последовательности Коши есть предел.
\ео

Свойства последовательностей Коши и предела известны 
многим  из школьного курса анализа. В школе
и в МГУ обыкновенно изучают последовательности
в $\R$, но доказательства легко переносятся 
на случай произвольного метрического пространства.

Среди прочего, верно следующее (проверьте).


\begin{description}
\item[(i)]\ \ Предел последовательности Коши единственный
(если существует).
\item[(ii)] \ \ Подпоследовательность последовательности
Коши --- снова последовательность Коши. Последовательность Коши
эквивалентна любой своей подпоследовательности.
\item[(iii)] \ \ Если переставить элементы последовательности
Коши $\{\alpha_i\}$ произвольным образом, получится последовательность
Коши, эквивалентная $\{\alpha_i\}$.
\end{description}

\определение
{\бф Диаметр} множества $X\subset M$ есть
$\sup_{x,y \in X} d(x,y)$. 
\ео

Легко видеть, что диаметр
$\epsilon$-шара не больше $2\epsilon$ (проверьте).
С другой стороны, для каждой точки $x\in M$,
и любых точек $y,z$, с $d(y,z)< 2\epsilon$,
верно 
\begin{equation}\label{_do_shara_dist_Equation_}
| d(x, z)-d(y, z)| < 2\epsilon,
\end{equation} что
следует из неравенства треугольника.
Из \eqref{_do_shara_dist_Equation_} 
вытекает, что для любой последовательности
Коши $\{\alpha_i\}$, последовательность вещественных
чисел $d(x, \alpha_i)$ --- тоже Коши:
если $\{\alpha_i\}$ содержится в $\epsilon$-шаре, то
$d(x, \alpha_i)$ содержится в отрезке длины $2\epsilon$.
Похожий аргумент доказывает, что для любых
последовательностей Коши $\{\alpha_i\}, \{\beta_i\}$, последовательность
$\{d(\alpha_i, \beta_i)\}$ --- тоже  Коши.
Более того, если $\{\alpha_i\}$,  $\{\beta_i\}$
не экивалентны, то предел $\{d(\alpha_i, \beta_i)\}$
ненулевой. Проверьте каждое из этих утверждений!

Пусть дано метрическое пространство $M$. Обозначим через
$\bar M$ множество классов эквивалентности
последовательностей Коши в $M$. Определим
на $\bar M$ метрику формулой 
\[
d(\{\alpha_i\}, \{\beta_i\}) := \lim d(\alpha_i, \beta_i).
\]
Докажите, что это метрическое пространство.

\определение
Множество классов эквивалентности
последовательностей Коши в $M$
с метрикой, определенной выше, называется
{\бф пополнением} $M$.
\ео

Эта конструкция хорошо известна 
большинству студентов, ибо таким образом
в курсе анализа определяется множество $\R$ вещественных
чисел (как множество классов эквивалентности 
последовательностей Коши из $\Q$).  Пополнение
метрического пространства впервые появилось у Хаусдорфа,
в монографии ``Grundz\"uge einer Theorie der geordneten Mengen, '' (1914).

Пополнение является полным метрическим пространством.
Чтобы в этом убедиться, возьмем последовательность 
\[ 
  \{\alpha_i(0)\}, \{\alpha_i(1)\}, \{\alpha_i(2)\}, ...
\]
последовательностей Коши в $M$. Для доказательства
полноты $\bar M$, нам нужно предъявить последовательность Коши
$\{\beta_i\}$ элементов $M$ такую, что 
$\{\alpha_i(0)\}$, $\{\alpha_i(1)\}$, $\{\alpha_i(2)\}$, ... 
сходится к $\{\beta_i\}$. Первое, что приходит
в голову --- взять диагональную последовательность
$\beta_i:= \alpha_i(i)$. Этот аргумент не работает,
потому что на $i$-м месте в последовательности
Коши может стоять что угодно. Надо для каждого $N$
заменить  $\{\alpha_i(N)\}$ на подпоследовательность,
которая сходится очень быстро, например, такую, что
$\alpha_i(N), \alpha_{i+1}(N), \alpha_{i+2}(N), ...$
содержится в шаре радиуса $2^{-i}$. Тогда 
$\alpha_k(k), \alpha_{k+1}(k+1), \alpha_{k+2}(k+2), ...$
содержится в шаре радиуса 
\[ 2^{-k+1} +  2^{-k} + 2^{-k-1} + ...  < 2^{-k+2}
\]
то есть является последовательностью Коши. При этом,
$\{\alpha_i(N)\}$ отстоит от $\{\alpha_i(i)\}$
не больше, чем на $2^{-N+1}$. Следовательно,
$\{\alpha_i(i)\}$ --- это предел 
$\{\alpha_i(0)\}$, $\{\alpha_i(1)\}$, $\{\alpha_i(2)\}$, ...

Мы доказали существование пополнения.

\хфилл

\пример
Пространство с метрикой $d(x,y)=1$ для любых
 $x\neq y$ полно (докажите).


\хфилл


\пример
Пространство $\Q$ с обычной метрикой неполно, и его
пополнение --
это $\R$. К сожалению, буквально эту конструкцию для определения
$\R$ использовать нельзя, потому что метрика на метрическом
пространстве принимает значения в $\R$. Поэтому приходится
сначала определять $\R$ как множество классов
эквивалентности  последовательностей
Коши в $\Q$, а затем определять пополнение метрического
пространства, повторяя эту же самую конструкцию еще раз.

\хфилл

\пример
Пространство $\R^n$ с обычной метрикой полно (докажите).


%%%%%%%%%%%%%%%%%%%%%%%%%%%%%%%%%%%%%%%%%%%%%%%%%%%%%%%%%%%%

\section{Нормирование на группах и кольцах}

%%%%%%%%%%%%%%%%%%%%%%%%%%%%%%%%%%%%%%%%%%%%%%%%%%%%%%%%%%%%

Метрику можно вводить на различных алгебраических
объектах --- группах, кольцах, полях и так далее.
Делается это следующим образом.

\определение
Пусть $G$ --- абелева группа, а $d$ --- метрика на $G$.
Мы будем использовать обозначение $x, y \arrow x+y$
для групповой операции в абелевых группах.
Говорят, что $(G, +, d)$ метрическая группа,
если операция $x\arrow -x$ взятия обратного элемента
есть изометрия, и операция $x \arrow x +g$ есть
изометрия для любого $g\in G$. В этом случае
также говорят что метрика {\бф согласована с групповой
структурой,} или что метрика {\бф инвариантна.}
\ео

\определение
Функция $\nu:\; G \arrow \R^{\geq 0}$
называется {\бф нормой на группе} 
если 
\begin{description}
\item[(i)]\ \ $\nu(g) = \nu(-g)$, $\nu(0)=0$.
\item[(ii)] \ \ $\nu(g)>0$ для любого $g\neq 0$.
\item[(iii)] \ \ $\nu (g+g') \leq \nu(g) + \nu(g')$, для
любых $g, g'\in G$.
\end{description}
\ео

Легко видеть, что для любой нормы $\nu$ функция
$d_\nu(x,y):= \nu(x-y)$ задает метрику на $G$,
согласованную с групповой структурой. Обратно,
любая такая метрика задает норму $\nu(x):= d(x,0)$
(проверьте).

Множество последовательностей Коши в
метрической группе с операцией почленного сложения
образует группу, а  последовательности Коши, эквивалентные 
нулю --- подгруппу этой группы. Это легко видеть из
следующего соображения. Пусть $A, B\subset G$ --- подмножества в группе.
Множество всех сумм вида $\{ a+b\ \ |  \ \ a\in A, b\in B\}$
обозначается $A+B$. Легко видеть, что сумма двух шаров
радиуса $\epsilon, \epsilon'$ содержится в шаре
радиуса $\epsilon+\epsilon'$. Следовательно,
для любых последовательнпстей Коши $\{a_i\}$
$\{b_i\}$, все члены суммы $\{a_i+b_i\}$,
кроме конечного числа, содержатся в шаре
сколь угодно малого наперед заданного радиуса.

Факторгруппа 
группы последовательностей Коши по 
подгруппе последовательностей Коши, эквивалентных
нулю --- это пополнение $G$. Таким образом, пополнение метрической
группы есть снова метрическая группа.
Эта конструкция хорошо известна для
группы рациональных чисел по 
сложению; таким образом строится
аддитивная структура на множестве
вещественных чисел.


\определение
Пусть $A$ --- кольцо, с ассоциативным, коммутативным
умножением, а $\nu:\; A \arrow \R^{\geq 0}$
функция на $A$. $\nu$ называется нормой на кольце, если выполнены
следующие условия
\begin{description}
\item[(i)]\ \ Рассмотрим $A$ как группу, с групповым
законом, заданным сложением. Тогда $\nu$ --- это норма.
Иначе говоря, $\nu(g)>0$ для каждого $g\neq 0$, $\nu(0)=0$, 
$\nu(g) = \nu(-g)$, и $\nu (g+g') \leq \nu(g) + \nu(g')$.
\item[(ii)] \ \ Норма мультипликативна: $\nu(xy) = \nu(x)\nu(y)$.
\end{description}
\ео

Примером нормы на кольце является
отображение $t \arrow |t|$, определенное
на $\Q$ и на $\R$. Другим примером является 
дискретная норма: $\nu(t) =0$, если $t=0$, и 1 в
противном случае. Она мультипликативна на любом
кольце таком, что из $xy =0$ следует $x=0$ или $y=0$
(такие кольца называются {\бф кольцами без делителей нуля}).

Кольцо с нормой наделено инвариантной метрикой,
построенной по формуле $d_\nu(x,y) = \nu(x-y)$.
Множество последовательностей Коши в таком кольце,
с операциями почленного сложения и умножения,
образует кольцо, а последовательности, эквивалентные
нулю --- идеал в этом кольце.\footnote{Напомним, что
идеал $I$ в кольце $A$ --- это подгруппа по сложению в $A$,
такая, что для любых $a\in A, \iota \in I$ произведение 
$a\iota$ лежит в $I$. Факторгруппа по идеалу наделена 
естественной структурой кольца. Обратное тоже верно:
для любого сюръективного гомоморфизма колец
$A\stackrel \phi\arrow A_1$, ядро $\phi$
является идеалом.} Фактор по этому идеалу
есть пополнение $R$ по метрике $d_\nu$.
Мы получили, что пополнение кольца по метрике,
заданной нормой --- снова кольцо. 

Аналогичную процедуру можно провести с полем.
Почленного деления последовательности Коши 
на последовательность Коши не получится, 
потому что в ненулевой последовательности
Коши могут содержаться элементы, равные нулю.
Но каждую последовательность Коши, не
эквивалентную нулю, можно заменить
на подпоследовательность, не содержащую
нулей, а такие подпоследовательности можно делить
почленно. То, что частное снова будет последовательностью
Коши, проверяется за 3-4 строки вычислений.

Из этого следует, что пополнение поля с нормой --
это поле. Именно таким образом, исходя из
множества последовательностей Коши в $\Q$,
строится поле $\R$.

%%%%%%%%%%%%%%%%%%%%%%%%%%%%%%%%%%%%%%%%%%%%%%%%%%%%%%%%%%%%

\section{Целые $p$-адические числа: неархимедова геометрия}

%%%%%%%%%%%%%%%%%%%%%%%%%%%%%%%%%%%%%%%%%%%%%%%%%%%%%%%%%%%%

Начиная с интересных нормирований на кольцах,
можно получать очень полезные алгебраические объекты. 
Зафиксируем простое число $p$.

\определение
Пусть $x\in \Z$ представимо в виде $x= p^\alpha x_1$, где
$x_1$ не делится на $p$, а $\alpha \in \Z^{\geq 0}$. 
Тогда {\бф $p$-адическая норма} $\nu_p(x)$ равна
$p^{-\alpha}$. Положим $\nu_p(0)=0$. Проверьте,
что это норма. Пополнение $\Z$ относительно такой
нормы называется {\бф кольцом целых $p$-адических чисел},
обозначается $\Z_p$. 
\ео

Аналогичная конструкция, примененная к $\Q$, даст
пополнение $\Q_p$, которое называется {\бф полем 
$p$-адических чисел.} Любое рациональное число $a \in \Q$
можно представить в виде $a = p^\alpha\frac m n$,
где $n$, $m$ взаимно просты с $p$, а $\alpha$ --- 
целое число, однозначно заданное разложением
числителя и знаменателя $a$ на простые множители.
Определим $\nu(a):= p^{-\alpha}$. Докажите, что это
нормирование на поле $\Q$. 

$p$-адическая норма задает метрику, обычным способом:
$d(x,y) = \nu(x-y)$. Эта метрика довольно замечательна
геометрически, ибо обладает свойством {\бф
неархимедовости}:
\[
d(x,y) \leq \max (d(x,z), d(y, z)).
\]
Из этого условия следует аксиома треугольника,
но оно сильнее.
Для нормы, то же самое записывается в виде
\[
\nu(x+y) \leq \max(\nu(x), \nu(y)).
\]
Проверьте, что эти условия равносильны.
Проверьте это неравенство для $p$-адической нормы.

Неархимедовость метрического пространства
равносильна такому условию. Пусть задан треугольник
$x, y, z$, с длинами сторон $a, b, c$. Тогда
две из сторон равны, а третья меньше каждой из них.
Действительно, пусть $a$ --- самая длинная сторона.
Из неархимедовости следует, что $a \leq \max(b, c)$,
поэтому $b$ или $c$ имеет такую же длину, а 
третья сторона (самая маленькая) такая же, или меньше.

Геометрически, это свойство можно
переговорить так: любой треугольник в неархимедовом пространстве
равнобедренный, и его основание меньше 
двух других сторон.

Если $B_\epsilon(a)$ --- $\epsilon$-шар в неархимедовом
пространстве, с центром в $a$, а $x, y$ --- две его точки,
то $d(x,y) \leq \max( d(x,a), d(y, a)) \leq \epsilon$.
Поэтому $\epsilon$-шар с центром в любой точке
$B_\epsilon(a)$ совпадает с $B_\epsilon(a)$.
В неархимедовом пространстве, любая точка
шара является его центром.

Неархимедова геометрия довольно полезна в теории чисел,
алгебраической геометрии и других науках. 
Например, в физике высоких энергий, некоторые версии
теории струн развивают, исходя из того, что
физическое пространство в очень малых (квантовых) 
масштабах имеет геометрию, приближающуюся
к неархимедовой. 

%%%%%%%%%%%%%%%%%%%%%%%%%%%%%%%%%%%%%%%%%%%%%%%%%%%%%%%%%%%%

\section{Арифметика $p$-адических чисел}

%%%%%%%%%%%%%%%%%%%%%%%%%%%%%%%%%%%%%%%%%%%%%%%%%%%%%%%%%%%%


Легко видеть, что в любой полной группе с инвариантной
метрикой, заданной нормой $\nu$, ряд вида $\sum g_i$ 
сходится, если сходится соответствующий ряд из норм 
$\sum\nu(g_i)$ (проверьте). Поскольку
$\nu(p^i z) \leq p^{-i}$, для любой последовательности
целых чисел $z_i$, ряд $\sum_{i=0}^\infty z_i p^{i}$
сходится к целому $p$-адическому числу. Действительно,
\[
\sum _{i=0}^\infty \nu(z_i p^{i}) \leq \sum _{i=0}^\infty
p^{-i} = \frac {p}{p-1}
\]
(геометрическая прогрессия).

В частности, сходится ряд
\[
1 + p + p^2 + p^3 + ...
\]
Дробное число $\frac {1}{1-p}$ является целым
$p$-адическим! Действительно, сумма этого ряда,
будучи умножена на $1-p$, дает 1 (проверьте это).

Если два целых числа $a, b$ принадлежат $\epsilon$-шару, с
$2\epsilon < p^{-i}$, разность $a-b$ делится на $p^i$
(проверьте). Записав элементы последовательности Коши
$\{\alpha_i\}$ целых чисел в $p$-ичной системе счисления, мы получим 
нечто вроде
\begin{equation*}\begin{array}{c}
0000000000000000000000000000000000020\\
0000000000093275091374509172340957210\\
0000000000000026381637617631863181610\\
0000000000007927931793719279129881610\\
0000000000000000009812038102829881610\\
0000082739812739127397038102829881610\\
0003719237912723927397038102829881610\\
7213719237912723927397038102829881610\\
\end{array}
\end{equation*}
Из этой таблицы наглядно видно,
что соответствующая последовательность цифр стабилизируется:
на $i$-м месте, начиная с какого-то момента, стоит одна
и та же цифра. Пределом ее будет, очевидно, сумма вида
$\sum_{i=0}^\infty z_i p^{i}$, где $0\leq z_i\leq p-1$ --- 
$i$-я цифра с конца, в $p$-ичном представлении 
$\alpha_N$, для достаточно большого $N$.

Как и вещественные числа, $p$-адические числа 
можно складывать и умножать в столбик,
не забывая переносить переполнение в следующий регистр.
Продумайте эту процедуру, самостоятельно посчитайте
произведение и сумму каких-нибудь $p$-адических чисел.

Для каждого $n$, не делящегося на $p$, 
уравнение $nx =1 \mod p$ имеет целое решение.
Пусть $v := 1-nx$. Очевидно,  $\nu(v) \leq \frac 1 p$,
и поэтому сумма вида $1+ v + v^2 + v^3+ ...$
сходится. Поскольку $(1+ v + v^2 + v^3+ ...)(1-v)=1$,
имеем $1+ v + v^2 + v^3+ ...= \frac 1 {xn}$, 
поэтому $\frac 1 n = x + vx + v^2 x + v^3 x + ...$.
Таким образом, в кольце целых $p$-адических чисел
определено деление на любое $n$, взаимно простое с $p$.

Из этого видно, что рациональное число $a\in \Q$
является целым $p$-адическим тогда и только тогда,
когда $a= \frac m n$, и $n$ взаимно просто с $p$.
Другими словами, рациональное число $a\in \Q$
является целым $p$-адическим тогда и только тогда,
когда $\nu(a) \leq 1$. Целые $p$-адические числа
это шар радиуса 1 в $\Q_p$, с центром
в любом целом числе, например, в нуле (центром
шара в неархимедовом метрическом пространстве
является любая его точка).

В кольце $p$-адических чисел можно совершать
и более сложные алгебраические операции, например,
вычислить квадратный корень. Из разложения Тэйлора следует, 
что сумма ряда
\begin{equation}\label{_koren_Equation_}
S := \sum_{i=0}^\infty \frac{(-1)^i (2i)! x^i}{(1-2i)(i!)^2 4^i N^{2i-1}}
\end{equation}
удовлетворяет $S^2 = N^2 +x$, если этот ряд сходится.
Из $S^2 = N^2 +x$ и единственности разложения
в ряд Тэйлора следует ряд комбинаторных
тождеств (по одному для каждой степени $x$), 
которые можно усмотреть непосредственно.
Воспользовавшись этими комбинаторными тождествами,
получим, что из абсолютной сходимости 
\eqref{_koren_Equation_} в каком-нибудь нормированном
поле (например, в $p$-адическом) следует, что
его сумма тоже удовлетворяет $S^2 = N^2 +x$.

Если $p$ нечетно, и взаимно просто с $N$,
$\frac{1} {4^iN^{2i-1}}$ --- целое $p$-адическое число.
Частное $\frac{(2i)!}{(i!)^2}$ целое, потому что это
биномиальный коэффициент. Получаем, что норма $i$-го члена
этой суммы оценивается через
\begin{equation}\label{_koren_ocenka_Equation_}
\nu(\xi_i) \leq \nu\left(\frac {x^i}{1-2i}\right)\leq (1-2i) p^{-i}
\end{equation}
(здесь мы используем неравенство
\[
\nu \left(\frac 1 k\right) \leq k
\]
верное для любого целого $k$; докажите его).
Используя равенство \eqref{_koren_ocenka_Equation_},
и абсолютную сходимость ряда $\sum (2i-1) p^{-i}$
(докажите), мы получаем, что ряд \eqref{_koren_Equation_}
сходится. Поэтому в кольце $p$-адических чисел, для
нечетного $p$, можно вычислять квадратный корень из числа вида
$N^2 +x$, где $x$ делится на $p$, а $N$ взаимно
просто с $p$.

\задача
Какие квадратные уравнения можно решить в $\Z_p$?
А какие --- в $\Q_p$? 
\ез

Это рассуждение можно обобщить для произвольного
алгебраического уравнения. Знаменитая лемма Гензеля
(Hensel's lemma) утверждает, что любое полиномиальное
уравнение вида $P(x)=0$ с целыми коэффициентами
имеет решение в $\Z_p$, если $P(a) = 0 \mod p$
для какого-то целого числа $a$, и $P'(a) \neq 0 \mod p$.
Здесь $P'$ обозначает производную многочлена $P$.
Лемма Гензеля доказывается рекурсивно, решением
системы уравнений вида
\[
P(a_i) = 0 \mod p^{i+1}, \ \  a_i-a_{i-1} = 0 \mod p^i.
\]

\задача
Докажите лемму Гензеля.
\ез


\begin{figure}[ht]
\begin{center}
\epsfig{file=Hensel.eps,width=0.4\linewidth}\\
Kurt Hensel\\
(1861 --- 1941) 
\end{center}
\end{figure}


$p$-адические числа изобрел в 1897 году Курт Гензель,
который руководствовался идеями Куммера. 
Гензель, ученик Кронекера,  был внуком сестры
композитора Мендельсона. Он надеялся, посредством
$p$-адических чисел, решать вопросы теории чисел,
и немало решил их. Впрочем, метрика и сходимость
$p$-адических чисел была совершенно непонятна
Гензелю и его современникам.

Гензель доказал, что любое вещественное
число можно представить как сумму ряда,
который будет сходиться в $\Z_p$; изучая
$p$-адическую сумму этого ряда, он доказал
несколько ошибочных теорем о вещественных числах.
Например, Гензель представил $e^p$ как сумму
сходящегося $p$-адического ряда, и вывел
отсюда неправильное доказательство 
трансцендентности числа $e$. 

$p$-адические числа были концептуально не поняты 
вплоть до 1910-х годов, когда Фреше и Рисс (Riesz)
изобрели метрические пространства, обосновав
неясные рассуждения Гензеля.

В 1912-м году, венгерский математик Йожеф
Кюршак (J\'ozsef K\"ur\-sch\'ak, 1864 --- 1933)
изобрел валюации (нормы) на кольце, обобщив 
$p$-адические нормы. В 1917-м году, Александр
Маркович Островский \\(1893-1986), ученик Гензеля,
дал полную классификацию норм на поле $\Q$.
Оказалось, что нормы на поле $\Q$ исчерпываются $p$-адическими
(с точностью до возведения в степень) и обычной
(евклидовой) нормой. Набросок доказательства
теоремы Островского приведен в задачах. 

%%%%%%%%%%%%%%%%%%%%%%%%%%%%%%%%%%%%%%%%%%%%%%%%
\section{Библиография, замечания}
%%%%%%%%%%%%%%%%%%%%%%%%%%%%%%%%%%%%%%%%%%%%%%%%

 $p$-адические числа --- центральное понятие большинства
курсов теории чисел. Теория метрических пространств
и пополнение вводятся в начале многих хороших
курсов анализа, например, Зорича, Лорана Шварца,
и Кириллова-Гвишиани; также их изучают в матшкольном
курсе анализа и геометрии. Матричные группы над
$p$-адическими полями чрезвычайно важны в теории
представлений. В. С. Владимиров
и его соавторы написали много трудов о применении
$p$-адического анализа в математической физике 
и теории струн.

Вот некоторые книжки, 
которые могут пригодиться.

\begin{itemize}

\item  Коблиц Н., {\ем   
p-адические числа, p-адический анализ и дзета-\-функ\-ции}, --- 
М.: Мир, 1982.

\item  Серр, Ж.-П., {\ем Курс Арифметики}, --- М.: Мир, 1972,\\
http://ega-math.narod.ru/Books/Arithm.htm

\item Кириллов А.А., Гвишиани А.Д. 
{\ем Теоремы и задачи функционального анализа}, --- М.: Наука, 1979

\item Электронная библиотека учебников по $p$-адическим числам:\\
http://www.fen.bilkent.edu.tr/~franz/LN/LN-padic.html
\end{itemize}


%%%%%%%%%%%%%%%%%%%%%%%%%%%%%%%%%%%%%%%%%%%%%%%%

\chapter{Лекция 2: нормирования в векторных пространствах}

%%%%%%%%%%%%%%%%%%%%%%%%%%%%%%%%%%%%%%%%%%%%%%%%


%%%%%%%%%%%%%%%%%%%%%%%%%%%%%%%%%%%%%%%%%%%%%%%%

\section{Примеры нормированных пространств}

%%%%%%%%%%%%%%%%%%%%%%%%%%%%%%%%%%%%%%%%%%%%%%%%

В этом разделе, все векторные пространства предполагаются
заданными над $\R$.

\определение
Пусть $V$ --- векторное пространство над $\R$, а 
$\nu:\; V \arrow \R^{\geq 0}$ функция со значениями
в неотрицательных числах. $\nu$ называется {\бф нормой} на
$V$, если имеет место следующее 
\begin{description}
\item[Невырожденность:] $\nu(v)>0$, если $v\neq 0$,
\item[Неравенство треугольника:] $\nu (v+v') \leq \nu(v) + \nu(v')$.
\item[Инвариантность относительно гомотетии:] \ \ $\nu(\lambda v) = |\lambda| \nu(v)$,
\end{description}
для любых $v$, $v'\in V$, и  любого $\lambda\in \R$.
В такой ситуации $V$ называется {\бф нормированным пространством}.
\ео

Заметим, что из инвариантности относительно гомотетии
следует, что $\nu(0)=0$, и $\nu(-x)=\nu(x)$. Поэтому
$\nu$ является нормой на группе $V$.

\хфилл


\пример
\begin{itemize}
\item $V= \R^n$. Для каждого вектора $v = (x_1, x_2, ..., x_n)$
определим \[ |v|_{L^\infty}:= \max|x_i|.\] Докажите, что это норма.
\item $V= \R^n$. Для каждого вектора $v = (x_1, x_2, ..., x_n)$
определим \[ |v|_{L^1}:= \sum |x_i|.\] Докажите, что это норма.
\item $V= \R^n$. Для каждого вектора $v = (x_1, x_2, ..., x_n)$
определим \[ |v|_{L^2}:= \sqrt{\sum x_i^2}.\] 
\item На одномерном пространстве норма единственна
с точностью до умножения на число: $x\arrow c |x|$. 
\end{itemize}

Норма $|\cdot|_{L^2}$ называется обычной, или {\бф евклидовой нормой}
на векторном пространстве. Неравенство треугольника для евклидовой
нормы называется {\бф неравенством Коши-Буняковского}. В нерусскоязычной
литературе оно же называется {\бф неравенство Коши-\-Шварца} 
(Cau\-chy-\-Schwarz inequality). Чтоб его доказать, возьмем два ненулевых, неколлинеарных 
вектора $x, y$ в векторном пространстве с положительно 
определенным скалярным произведением $g$; неравенство 
треугольника для $|\cdot|_{L^2}$ будет следовать
из неравенства 
\[ 
   \sqrt{ g(x,x)} + \sqrt{g(y,y)} \geq \sqrt{ g(x+y,x+y)}.
\]
Возведя обе части в квадрат и раскрыв скобки, получим, что
это неравенство равносильно такому:
\begin{equation}\label{_C_B_inequa_Equation_}
\sqrt{ g(x,x)g(y,y)} \geq  g(x,y).
\end{equation}
С другой стороны, из $g(x-\lambda y,x-\lambda y)> 0$ следует,
что квадратичный полином
\[
   P(\lambda):=  g(x,x)-2\lambda g(x,y) + \lambda^2 g(y,y),
\]
не имеет корней. Значит, его дискриминант
$D= g(x,y)^2-g(x,x)g(y,y)$ отрицателен.
Это доказывает \eqref{_C_B_inequa_Equation_}.



\begin{figure}[ht]
\begin{center}
\epsfig{file=Bouniakowsky.eps,width=0.55\linewidth}\\
Виктор Яковлевич Буняковский\\
(1804-1889)
\end{center}
\end{figure}


В качестве альтернативного метода
заметим, что на двумерном пространстве, порожденном
$x,y$, можно ввести координаты таким образом,
что квадратичная форма $g$ будет стандартной
$g((x_1, x_2), (y_1, y_2)) = x_1y_1 + x_2 y_2$.
Тогда неравенство треугольника следует из
того, что (как известно из школьной планиметрии)
\[ g(x,y)= |x||y|\cos\alpha \leq |x||y| = \sqrt{ g(x,x)g(y,y)}
\]
где $\alpha$ есть угол между векторами $\vec x$ и $\vec y$.

Норма на пространстве задает метрику по формуле
\[ d_\nu(x,y) = \nu(x-y).\]
Та же самая формула используется
для колец, полей, групп и т. д.




\определение
Напомним, что подмножество $Z$ векторного пространства
$V$ называется {\бф выпуклым}, если для любых точек
$x, y\in Z$, $Z$ содержит отрезок $[x,y]$ целиком.
\ео


\утверждение
Пусть $\nu$ --- норма на векторном пространстве.
Тогда единичный шар с центром в нуле
\[
B_1(0):= \{ x\in V\ \  \nu(x) < 1\}
\]
выпуклый.

\хфилл

{\бф Доказательство:} Напомним, что
отрезок $[x,y]$ --- это множество точек вида
$\lambda x + (1-\lambda)y$, где $0\leq \lambda\leq 1$ --- 
вещественное число. Можно считать это определением
отрезка. Выпуклость шара $B_1(0)$ означает, что
\[
\nu(\lambda x + (1-\lambda)y) < 1
\]
для любых $x, y$ таких, что $\nu(x)< 1, \nu(y) <1$, 
и $\lambda\in [0,1]$.
В силу неравенства треугольника и мультипликативности
нормы, имеем
\[
\nu(\lambda x + (1-\lambda)y) \leq 
\nu(\lambda x) + \nu((1-\lambda)y)\leq \lambda\nu(x) +
(1-\lambda)\nu(y)\leq 1
\]
(последнее неравенство следует из того, что $\nu(x)< 1,
\nu(y) <1$). \endproof

\hfill


\begin{figure}[ht]
\begin{center}
единичный шар в $\R^2$\\
\begin{tabular}{cc}
\epsfig{file=l1-ball.eps,width=0.3\linewidth} & \epsfig{file=l-infty-ball.eps,width=0.3\linewidth}\\
в $L^1$-норме & в $L^\infty$-норме
\end{tabular} \\
\epsfig{file=l2-ball.eps,width=0.3\linewidth}\\
в $L^2$-норме (евклидовой)
\end{center}
\end{figure}

 

Единичный шар в нормах $|\cdot|_{L^1}$, 
$|\cdot|_{L^\infty}$, $|\cdot|_{L^2}$, легко нарисовать
(для $\R^2$), и наглядно убедиться в его выпуклости.



\хфилл

Евклидова норма выделена из всех прочих тем,
что у нее группа изометрий самая большая.
Чтобы доказать это (и даже сформулировать),
необходимо разобраться с тем, что такое "размер" (размерность)
группы. Но даже на картинке выше видно, что у единичной
сферы в $|\cdot|_{L^2}$ нет выделенных частей,
и группа движений (изометрий) действует на ней
транзитивно (переводя любую точку в любую), 
а в единичной сфере для $|\cdot|_{L^1}$ и $|\cdot|_{L^\infty}$
особыми точками являются углы квадратов.

\hfill

Нормы $L^1$, $L^2$, $L^\infty$ --- часть непрерывной системы
норм на $\R^n$, которые определяются следующим образом.
Пусть $v= (x_1, ..., x_n)$ --- точка $\R^n$, а $p$ вещественное
число, $p\geq 1$.  Определим $L^p$-норму формулой
\[ |v|_{L^p}:= \sqrt[p]{\sum |x_i|^p}.\] 
Неравенство треугольника для этой нормы называется
{\бф неравенством Минковского}; его доказательство
довольно трудоемко. 

\хфилл

Для пространства непрерывных функций на отрезке (или другом
компактном множестве $M$), можно определить $L^p$-нормы,
для $p\geq  1$, формулой
\[
|f|_{L^p}= \sqrt[p]{\int_M |f|^p}.
\]
и
\[
|f|_{L^\infty}= \sup_M f
\]
Супремум $|f|$ конечен, и интегралы $|f|^p$
определены, потому что непрерывная функция достигает
максимума на отрезке (и любом компакте), а значит ограничена. 
Неравенство треугольника для $L^1$, $L^2$ и $L^\infty$-нормы
в этой ситуации доказывается элементарно (докажите). 



%%%%%%%%%%%%%%%%%%%%%%%%%%%%%%%%%%%%%%%%%%%%%%%%

\section{Непрерывные отображения}

%%%%%%%%%%%%%%%%%%%%%%%%%%%%%%%%%%%%%%%%%%%%%%%%



\определение
Подмножество $Z\subset M$ метрического пространства
называется {\бф открытым}, если 
верны следующие равносильные условия (докажите равносильность).
\begin{description}
\item[(i)] оно является объединением
$\epsilon$-шаров 
\item[(ii)] вместе с каждой точкой $z\in Z$, $Z$ содержит целиком
некоторый $\epsilon$-шар с центром в этой точке.
\end{description}
\ео

\определение
Пусть $\{z_i\}$ --- последовательность точек в метрическом
пространстве $(M,d)$. Мы говорим, что $z_i$ {\бф сходится к $z$},
если $\lim\limits_{i\arrow \infty} d(z_i, z) =0$
\ео

\определение
Пусть $(M_1, d_1)$ и $(M_2, d_2)$ --- метрические пространства,
а $f:\; M_1 \arrow M_2$ --- некоторое отображение.
Оно называется {\бф непрерывным}, если верны
следующие равносильные условия (докажите равносильность).
\begin{description}
\item[(i)] Отображение $f$ {\бф сохраняет пределы}: 
если последовательность $\{z_i\}$ сходится
к $z$, то $\{f(z_i)\}$ сходится к $f(z)$.
\item[(ii)] Для каждой $z\in M$ и каждого $\epsilon >0$ найдется
$\delta >0$ такое, что из $d_1(x, z) < \delta$ следует
$d_2(f(x), f(z)) <\epsilon$.
\item[(iii)] Прообраз любого открытого множества открыт.
\end{description}
\ео

Композиция непрерывных отображений очевидно непрерывна.

\замечание 
Непрерывное отображение совершенно не обязано переводить
последовательности Коши в последовательности
Коши. Рассмотрим, например, отображение 
$\Q \stackrel \phi\arrow \{0, 1\}$,
переводящее все числа $> \sqrt 2$ в 1,
 а все числа меньше $\sqrt 2$ в 0.
Открытых подмножеств в множестве 
$\{0, 1\}$ 4 штуки: \{0\}, \{1\}, пустое
множество и все $\{0, 1\}$; легко видеть,
что прообраз каждого из них открыт.

\еза

\пример\label{_distance_cont_Primer_}
Пусть $z$ --- точка метрического пространства $M$.
Тогда $x \stackrel {d_z} \arrow d(z,x)$ является непрерывным отображением
из $M$ в $\R$ с евклидовой метрикой. Действительно,
из неравенства треугольника сразу следует, что
$d_z(x)-d_z(y) \leq d(x,y)$.

\определение
Отображение метрических пространств называется
{\бф гомеоморфизмом}, если оно непрерывно, биективно,
и обратное ему тоже непрерывно.
\ео

Две нормы $\nu$ и $\nu'$ на векторном пространстве $V$
называются {\бф эквивалентными}, если тождественное
отображение задает гомеоморфизм $(V, \nu) \arrow (V,\nu')$.

Непрерывность тождественного отображения $(V, \nu)\arrow (V,\nu')$
значит, что некоторый открытый шар в норме $\nu'$
содержит открытый шар в норме $\nu$. Если первый шар
имеет радиус $r$, второй шар --- радиус $s$, то шар
радиуса 1 в $(V, \nu')$ содержит шар радиуса $s/r$
в норме $\nu$. Иначе говоря, из $\nu'(x) \leq 1$
следует $\nu(x)\leq s/r$. Это равносильно такому
неравенству: $\frac {\nu(x)}{\nu'(x)}\leq \frac s r$.
Из непрерывности $(V, \nu')\arrow (V,\nu)$
следует противоположное неравенство, с другим
коэффициентом. Мы получили такое утверждение. 

\hfill

\утверждение
Пусть $\nu$ и $\nu'$ --- нормы на векторном пространстве $V$.
Эти нормы эквивалентны тогда и только тогда, когда
существуют положительные числа $C_1$, $C_2$, такие, что 
для любого $x\in V$ выполнены
неравенства
\begin{equation}\label{_equi_metri_Equation_}
C_1 \nu(х) \leq \nu'(x) \leq C_2 \nu(x).
\end{equation}

\endproof

\хфилл

Докажем следующую полезную теорему.

\hfill

%%%%%%%%%%%%%%%%%%%%%%%%%%%%%%%%%%%%%%%%%%%%%%%%
\теорема
На конечномерном пространстве все нормы эквивалентны.

\хфилл

Заметим, что эквивалентность норм $L^1$, $L^2$, $L^\infty$ 
на $\R^2$ вполне очевидна из чертежа, на котором нарисован 
единичный шар (см. выше).

Пусть $\nu$ --- произвольная норма на $V=\R^n$, $|\cdot|_{L^1}$ --- $L^1$-норма,
а $x_1, ..., x_n$ --- стандартный базис в $V$. Воспользовавшись неравенством
треугольника, получаем
\[
\nu(z) \leq \sum_i |\lambda_i| \nu(x_i) \leq
\max_i \nu(x_i) \sum_i |\lambda_i| = C |z|_{L^1},
\]
где $z=\sum_i \lambda_i x_i$, а $C= \max_i \nu(x_i)$.
Это дает одно из двух неравенств \eqref{_equi_metri_Equation_}, 
нужных для эквивалентности норм. 
Мы получили, что тождественное отображение
$(V, |\cdot|_{L^1})\stackrel\Id\arrow (V,\nu)$
непрерывно.

Чтобы доказать, что обратное отображение
тоже непрерывно, воспользуемся компактностью.
Подробнее про компактность я расскажу
одной из следующих лекций. 

Напомним, что (секвенциально) компактным подмножеством в
метрическом пространстве называется множество, в котором из каждой
последовательности можно выбрать сходящуюся
подпоследовательность. Замкнутым подмножеством
называется такое подмножество, дополнение
до которого открыто.  Из курса анализа 
известно, что в $\R^n$ со стандартной
метрикой каждое замкнутое, 
ограниченное \footnote{Ограниченное 
значит "содержащееся в каком-то шаре".}
подмножество компактно.
Также известно, что непрерывная
функция на компакте принимает
максимум и минимум. Мы передокажем
эти утверждения в одной из следующих лекций. 

Функция $\nu:\; V \arrow \R$ непрерывна относительно
метрики, заданной $\nu$ (Пример \ref{_distance_cont_Primer_}).
В силу непрерывности $(V, |\cdot|_{L^1})\stackrel\Id\arrow (V,\nu)$
эта функция непрерывна на $V$ с $L^1$-метрикой.
Поэтому она достигает минимума $C_1$ на единичной сфере
относительно этой метрики.\footnote{Нетрудно видеть,
что единичная сфера в $L^1$-метрике является границей правильного
гипер-октаэдра с вершинами в центрах граней единичного куба.} 
Поскольку $\nu$ положительна
на ненулевых векторах, $C_1$ тоже
положительно. Мы получили неравенство
$\nu(v) \geq C_1 |v|_{L^1}$.
Неравенство $\nu(v) \geq C |v|_{L^1}$
получено чуть выше. Мы доказали эквивалентность
норм на $V$.
\endproof

%%%%%%%%%%%%%%%%%%%%%%%%%%%%%%%%%%%%%%%%%%%%%%%%

\section{Выпуклые множества и норма}

%%%%%%%%%%%%%%%%%%%%%%%%%%%%%%%%%%%%%%%%%%%%%%%%

Пусть $V$ --- конечномерное векторное пространство,
$\nu$ норма на нем, а $B$ --- единичный шар.
Как мы видели, $B$ ограниченное (содержится
в евклидовом шаре большого радиуса), открытое,
выпуклое множество. Оказывается, каждое такое
множество задает некоторую норму.

\теорема
Пусть $V$ --- конечномерное векторное пространство,
а $B$ --- непустое открытое, выпуклое, ограниченное
подмножество в $V$. Предположим, что $B$
центрально-симметрично, то есть для каждого $v\in B$
точка $-v$ тоже лежит в $B$. Тогда $B$ является
единичным шаром для какой-то нормы.

\хфилл

{\бф Доказательство:}
Для любых множеств $A,B\subset V$, определим 
$A+B\subset V$ как множество всех векторов вида
$a+b$, $a\in A, b\in B$. Для $\lambda \in \R$,
определим $\lambda A$ как множество всех векторов
вида $\lambda a$, $a\in A$.

Выпуклость $B$ равносильна условию 
$\lambda B + (1-\lambda) B\subset B$, где $\lambda\in[0,1]$
произвольное число от 0 до 1 (докажите это).
Поэтому для выпуклого $B$ мы имеем
$\alpha B + \beta B \subset (\alpha + \beta) B$.

Определим функцию $\nu_B:\; V \arrow \R$ формулой
\[ 
  \nu_B(x) := \inf_\rho \ \{ \rho > 0\ \  | \ \ x \in \rho B\}
\]
Поскольку $B$ открыт и содержит 0, это множество непусто,
а поскольку ограничен --- для ненулевого $x$, 
$\nu_B(x)\neq 0$. Условие 
$\nu_B(\lambda x) = |\lambda| \nu_B(x)$ следует
прямо из определения. Наконец, неравенство
треугольника вытекает из 
$\alpha B + \beta B \subset (\alpha + \beta) B$.
Действительно, пусть $x, y$ такие векторы, что
$x\in (\alpha+\epsilon) B$ и $y\in (\beta+\epsilon) B$. Тогда 
$x+y \in (\alpha + \beta+2\epsilon) B$. 
Поэтому для любого $\epsilon >0$,
из $\nu_B(x) = \alpha$, 
$\nu_B(y) = \alpha$ следует
$\nu(x+y) \leq \alpha+\beta+2\epsilon$. Это 
доказывает неравенство треугольника.
Мы построили норму $\nu_B$ по выпуклому,
ограниченному, открытому, центрально-симметричному
множеству $B$. Легко видеть, что единичный
шар в $\nu_B$ это и есть $B$. \endproof

%%%%%%%%%%%%%%%%%%%%%%%%%%%%%%%%%%%%%%%%%%%%%%%%

\section{История, замечания}

%%%%%%%%%%%%%%%%%%%%%%%%%%%%%%%%%%%%%%%%%%%%%%%%

Основы теории нормированных векторных
пространств излагаются в любом приличном
учебнике анализа. Особенно хорош для этой цели
двухтомный "Анализ" Лорана Шварца и учебник Зорича.

Неравенство Коши-Буняковского
в конечномерных векторных пространствах доказал 
Огюстен Коши (Augustin Cauchy)
в 1821-м году. В пространствах функций это неравенство
доказал Виктор Яковлевич Буняковский, в 1859-м;
в 1888-м результат Буняковского передоказал 
Герман Шварц (Hermann Amandus Schwarz), ученик
Вейерштрасса, один из основателей комплексного анализа,
в честь которого названа лемма Шварца из комплексного
анализа, производная Шварца, и много других вещей.

\begin{figure}[ht]
\begin{center}
\epsfig{file=Banach.eps,width=0.40\linewidth}\\
Stefan Banach\\
(1892 --- 1945) 
\end{center}
\end{figure}


Понятие метрического пространства
изобрел Фреше, в его диссертации 1906-го года. 
В 1909-м году, Фридьеш Рисс  (Frigyes Riesz)
обнаружил, что понятие метрики для изучения
топологии пространства необязательно, и предложил
определение топологии, основанное на понятии замыкания.
Его система аксиом была непохожа на современную, 
но в 1914-м году Хаусдорф опубликовал монографию
``Grundz\"uge einer Theorie der geordneten Mengen,'' где 
развил аксиоматическую теорию топологических пространств, 
практически тождественную современной. 

Любопытно, что многие идеи монографии Хаусдорфа можно
обнаружить в его литературных и философских работах,
опубликованных в конце XIX века под псевдонимом Пол Монгре
(Paul Mongr\'e). 

Термин "компактное пространство" также принадлежит
Фреше. Метрика $L^p$ определена Риссом, в 1910-м году.

Понятие нормированного пространства определил и детально
исследовал Банах в начале 1920-х годов. 

%%%%%%%%%%%%%%%%%%%%%%%%%%%%%%%%%%%%%%%%%%%%%%%%%%%%%%%%%%%%%%%%%%%%%%%%

\chapter{Лекция 3: Компакты в метрических пространствах}

%%%%%%%%%%%%%%%%%%%%%%%%%%%%%%%%%%%%%%%%%%%%%%%%%%%%%%%%%%%%%%%%%%%%%%%%



%%%%%%%%%%%%%%%%%%%%%%%%%%%%%%%%%%%%%%%%%%%%%%%%%%%%%%%%%%%%%%%%%%%%%%%%

\section{Теорема Гейне-Бореля}

%%%%%%%%%%%%%%%%%%%%%%%%%%%%%%%%%%%%%%%%%%%%%%%%%%%%%%%%%%%%%%%%%%%%%%%%

Пусть $M$ --- метрическое пространство.
Напомним, что открытое подмножество $M$  --это 
объединение любого числа открытых шаров. 
Ясно, что объединение любого числа открытых подмножеств
открыто.

\определение
Набор открытых подмножеств $\{U_\alpha\}$ в $M$
называется {\бф покрытием $M$}, если $M = \bigcup_\alpha U_\alpha$.
{\бф Подпокрытием} покрытия $\{U_\alpha\}$
называется такое подмножество $\{U_\alpha\}$,
которое тоже является покрытием.
\ео

\определение
Пространство $M$ называется {\бф компактным},
если из любого покрытия $M$ можно выбрать конечное
подпокрытие. 
\ео


\определение
Пространство $M$ называется {\бф секвенциально компактным},
если любая последовательность точек $\{x_i\}$ в $M$
имеет сходящуюся подпоследовательность.
\ео

\хфилл

\теорема
(Теорема Гейне-Бореля)\\
Метрическое пространство компактно тогда и только тогда,
когда оно секвенциально компактно.

\хфилл

\noindent
{\бф Доказательство}\\
Вывести из обычной компактности секвенциальную
ничего не стоит.  Напомним, что {\бф окрестностью}
точки $x\in M$ называется любое открытое множество,
содержащее $x$. Точка $x$ является предельной точкой
последовательности $\{x_i\}$ тогда и только тогда, когда в любой
окрестности $x$ содержится  элемент  $\{x_i\}$. 
Пусть $\{x_i\}$ --- последовательность,
не имеющая предельных точек.

Для каждого $x\notin \{x_i\}$,
некоторая окрестность $x$ не содержит элементов
$\{x_i\}$. Объединение $U$ всех таких окрестностей 
открыто, и совпадает с дополнением $M \backslash \{ x_i\}$.
Взяв у каждого $x_i$ окрестность $U_i$, которая не содержит
других элементов $\{x_i\}$, получим покрытие
$\{U, U_i\}$, которое не содержит конечного подпокрытия.

Доказательство импликации\footnote{Заключение вида ``из утверждения
$A$ следует утверждение $B$" называется импликацией.}
\[\text{
(секвекциальная компактность) $\Rightarrow$ (обычная компактность)}
\]
более трудоемко. 


\хфилл

\лемма
Пусть $M$ --- секвенциально компактное метрическое пространство,
а $f:\; M \arrow \R$ непрерывная функция. Тогда супремум
и инфимум $f$ на $M$ конечен. Более того, есть точки
$x$, $y$ такие, что
\[
f(x) = \sup_{z\in M} f(z), \ \ \ \ f(y) = \inf_{z\in M} f(z).
\]
{\bf Доказательство:} Пусть $\{x_i\}$ последовательность
точек таких, что 
\[
\lim f(x_i) = \sup_{z\in M} f(z).
\]
Из секвенциальной компактности следует, что $\{x_i\}$ 
содержит сходящуюся подпоследовательность $\{x_i'\}$. Обозначим
ее предел через $x$. Поскольку $f$ непрерывно, $f$ сохраняет 
пределы последовательностей, и поэтому $\lim f(x_i')= f(x)$.
Мы доказали, что $f(x)= \sup_{z\in M} f(z)$.
Доказательство для инфимума аналогично.
\endproof

\хфилл

В метрической геометрии весьма полезно следующее понятие.

\определение 
Пусть $(M_1, d_1)$ и $(M_2, d_2)$ --- метрические
пространства, а $C>0$ --- вещественное число. 
Отображение $f:\; M_1 \arrow M_2$ называется {\бф
$C$-липшицевым} ($C$-Lipshitz map) если для любых
$x, y\in M_1$, имеет место неравенство
\[
d_2(f(x),f(y)) \leq C d_1 (x, y).
\]
Функция $f:\; M \arrow \R$
на метрическом пространстве называется
{\bf $C$-липшицевой}, если она $C$-липшицева как отображение
из $M$ в $\R$ с обычной метрикой.
Отображения, которые являются липшицевыми
для ка\-кой-\-то константы $C$, называюстя
{\бф липшицевыми}, или {\бф непрерывными
по Липшицу} (Lipschitz continuous).
\ео

Липшицевы отображения называются липшицевыми в честь
немецкого математика Рудольфа Липшица, ученика Дирихле. 
Легко видеть, что они непрерывны (докажите).


\begin{figure}[ht]
\begin{center}
\epsfig{file=Lipschitz.eps,width=0.66\linewidth}\\
Rudolf Otto Sigismund Lipschitz \\
(1832 --- 1903)
\end{center}
\end{figure}

Расстояние $d_z(x) := d(z,x)$ 
до фиксированной точки $z\in M$ является
примером липшицевой функции. Действительно,
$|d_z(x)-d_z(y)| \leq d(x,y)$, что следует из неравенства
треугольника (докажите это).

Другим примером $C$-липшицевой функции является
дифференцируемая функция $f:\; R \arrow \R$,
с $|f'|\leq C$ (докажите).


Напомним, что $B_\rho(x)$
обозначает открытый шар радиуса $\rho$ с центром в $x$.

\хфилл

\лемма
Пусть $M$ --- метрическое пространство, а ${\cal U}:= \{ U_\alpha\}$
его покрытие. 
Рассмотрим функцию
\begin{align*}
\rho_{\cal U}(x) := \sup_\rho \  &\{\rho \in \R \ \ |\ \ B_\rho(x) \text{\ \ содержится}\\ \ &\ \ \  
\text{в одном из элементов покрытия ${\cal U}$}\}.
\end{align*}
Тогда $\rho_{\cal U}$ 1-липшицева.

\хфилл

\noindent
{\бф Доказательство:} Пусть $y\in M$ --- точка, отстоящая
от $x$ на $\epsilon$, а $\rho_{\cal U}(x)> \rho_0$.
В этом случае, открытый шар $B_{\rho_0}(x)$ целиком содержится
в одном из элементов покрытия ${\cal U}$.
Из неравенства треугольника следует, что
$B_{\rho_0-\epsilon}(y)\subset B_{\rho_0}(x)$
(докажите). Поэтому 
\[ 
  \rho_{\cal U}(y)\geq \rho_{\cal U}(x)-d(x,y)
\]
Выписывая аналогичное неравенство для пары $(y,x)$, получаем 
\[
d(x,y) \geq |\rho_{\cal U}(y)-\rho_{\cal U}(x)|.
\]
\endproof

\hfill

Следующее понятие также чрезвычайно полезно.


\определение
 {\бф $\epsilon$-сетью} в 
метрическом пространстве $M$ называется
такое подмножество $V \subset M$, что $M$
лежит в объединении всех $\epsilon$-шаров с центрами в $V$.
$\epsilon$-сеть называется {\бф конечной},
если $V$ конечно. 
\ео

Легко убедиться, что в секвенциальном
компакте есть конечная $\epsilon$-сеть,
для каждого $\epsilon >0$. Действительно, 
если такой сети нет, найдется бесконечная
последовательность точек $\{x_i\}$ таких, что
никакой $x_i$ не лежит в объединении $\epsilon$-шаров
с центрами во всех предыдущих. Но такая последовательность
не может содержать сходящейся подпоследовательности,
потому что $d(x_i, x_j)\geq \epsilon$ для всех $i\neq j$.

\hfill

Вернемся к доказательству теоремы Гейне-Бореля.

Пусть $M$ --- секвенциально компактное 
метрическое пространство, а ${\cal U}:= \{ U_\alpha\}$
некоторое покрытие. Для доказательства
Гейне-Бореля, нужно показать, что у ${\cal U}$ 
есть конечное подпокрытие конечно.

Рассмотрим функцию 
$\rho_{\cal U}:\; M \arrow \R$, определенную выше.
Поскольку непрерывная функция на секвенциальном
компакте принимает минимум, а $\rho_{\cal U}$
во всех точках положительна, имеем 
$\rho_{\cal U}\geq{\rho_{\min}} >0$.
Это значит, что для каждой точки $x\in M$,
${\rho_{\min}}$-шар с центром в $x$ содержится
в одном из элементов покрытия ${\cal U}$.

Возьмем в $M$ конечную $\rho_{\min}$-сеть $V$
и пусть $\{B_i\}$ --- шары с центрами в точках $V$,
радиуса $\rho_{\min}$. По определению $\rho_{\min}$,
каждый из этих шаров содержится
в некотором элементе $U_i$ покрытия ${\cal U}$.
Мы получаем
\[
M = \bigcup B_i \subset \bigcup U_i,
\]
а значит, $\{U_i\}$ --- конечное подпокрытие ${\cal U}$.
Мы доказали теорему Гейне-Бореля. \endproof

\hfill

Исторически, теорему Гейне-Бореля формулировали
так: ``каждое замкнутое, ограниченное подмножество в $\R^n$
компактно''. Обобщение ее на метрические пространства 
принадлежит, вероятно, Александрову и Урысону,
определившим компактные пространства (они называли
их ``бикомпактные'') в терминах покрытий и подпокрытий.


\begin{figure}[ht]
\begin{center}
\ \\ \ \\
\epsfig{file=Borel.eps,width=0.6\linewidth}\\
\'Emile Borel\\
(1871 --- 1956)
\end{center}
\end{figure}

Самое раннее доказательство теоремы Гейне-Бореля 
принадлежит Дирихле (1862), для подмножеств прямой $\R$.
Ученик Дирихле  Генрих Эдуард Гейне (Heinrich Eduard Heine, 1821-1881)
опубликовал доказательство (для подмножеств 
прямой) в 1872-м году. В 1895-м году
 Эмиль Борель  доказал, что любое {\ем счетное} покрытие
замкнутого, ограниченного подмножества $\R^n$
имеет конечное подпокрытие. Для несчетных покрытий,
доказательство было получено пятью годами позже
Шенфлисом (Arthur Moritz Sch\"onflies, 1900) и 
Лебегом (1898, опубликовано в 1904). Во 
франкоязычной литературе теорему Гейне-Бореля
называют ``теорема Бореля-Лебега" (Th\'eor\`eme de
Borel-Lebesgue).




%%%%%%%%%%%%%%%%%%%%%%%%%%%%%%%%%%%%%%%%%%%%%%%%%%%%%%%%%%%%

\section{Историческое отступление: \\работы Хаусдорфа}

%%%%%%%%%%%%%%%%%%%%%%%%%%%%%%%%%%%%%%%%%%%%%%%%%%%%%%%%%%%%


Метрические пространства были изобретены Фреше
в 1906-м году для изучения топологических свойств 
пространств функций, но на эвристическом уровне понятия топологического 
пространства и непрерывности встречались еще у Римана. 
Риман и Фреше понимали, что к пространствам функций можно применять
те же геометрические приемы, что и к геометрическим
объектам. Фреше надеялся, что теория метрических
пространств станет удобным фундаментом для 
объединения геометрии и теории функций.
В работах Банаха по функциональному
анализу эта надежда вполне оправдалась, 
но в геометрии понятие метрического пространства 
было не слишком употребительно вплоть до 1980-х.

В конце XIX века, топологическую природу
подмножеств прямой изучал Кантор. Ему принадлежат
понятия замкнутого и открытого множества, и ряд
полезных классификационных теорем. 

\begin{figure}[ht]
\begin{center}
\epsfig{file=300px-Hausdorff_1913-1921.eps,width=0.55\linewidth}\\
Felix Hausdorff \\
(1868 --- 1942) 
\end{center}
\end{figure}

В 1914-м году Феликс Хаусдорф опубликовал  книгу
``Grundz\"uge der Mengenlehre'' (Основы теории множеств).
Развивая достижения Фреше и Рисса, Хаусдорф
определил понятие топологического пространства,
набором чрезвычайно простых и удобных аксиом,
почерпнутых в аксиоматике Гильберта, придуманной
тем для евклидовой геометрии взамен недостаточно
строгой аксиоматики Евклида.

\хфилл

\noindent
Напомним, что $2^M$ обозначает множество всех подмножеств в $M$.

\определение
Пусть $M$ --- множество, а ${\cal U} \subset 2^M$
набор подмножеств, называемых {\бф открытыми}. 
Мы говорим, что ${\cal U}$ {\бф задает топологию}
на $M$, если 
\begin{description}
\item[(i)] Любое объединение открытых подмножеств открыто
\item[(ii)] Конечное пересечение открытых подмножеств открыто
\item[(iii)] $M$ и пустое множество $\emptyset$ открыты.
\end{description}
В такой ситуации $M$ называется {\бф топологическим пространством}.
\ео

Оказалось, что этой простой формальной структуры
вполне достаточно для определения ключевых понятий
топологии множеств: непрерывного отображения,
предела, замыкания, точек концентрации и так далее.
Подробнее об этом я расскажу в следующих лекциях.

В первой половине XX века математики весьма интересовались
формальными следствиями простых аксиоматических систем,
и изучение топологии на много лет свелось к изучению
общих (как правило, весьма экзотических) 
топологических пространств. Эту деятельность
называют ``общая топология" (point-set topology).
В arxiv.org ``общей топологии" посвящена отдельная
категория math.GN; там можно посмотреть все статьи
за какой-нибудь год, например {\tt \scriptsize http://arxiv.org/list/math.GN/07} . 

Их немного: мало кто занимается сейчас общей топологией.

\хфилл

Метрика на пространстве задает топологию на нем
(открытые множества можно определить как объединения
открытых шаров). Топологическое пространство,
которое получается таким образом, называется 
{\бф метризуемым}. Далеко не все теоремы, которые
верны в метризуемых пространствах, верны в общей
ситуации.

В частности, неверна теорема Гейне-Бореля
(равносильность секвенциальной компактности и обычной).
Также неверна равносильность непрерывности и
секвенциальной непрерывности: даже если отображение 
сохраняет пределы последовательностей (это называется 
``секвенциальная непрерывность"), прообраз
открытого множества не обязательно открыт.
Из обычной непрерывности (прообраз открытых
множеств открыт) можно вывести секвенциальную,
но не наоборот.

В такой ситуации довольно часто говорят
``обычная непрерывность {\ем сильнее} секвенциальной."
``Одно предположение сильнее другого" значит 
``из первого предположения можно вывести второе".


%%%%%%%%%%%%%%%%%%%%%%%%%%%%%%%%%%%%%%%%%%%%%%%%%%%%%%%%%%%%

\section{Расстояние Хаусдорфа}

%%%%%%%%%%%%%%%%%%%%%%%%%%%%%%%%%%%%%%%%%%%%%%%%%%%%%%%%%%%%

Пусть $M$ --- метрическое пространство.
Напомним, что подмножество $Z\subset M$ называется
{\бф замкнутым}, если его дополнение открыто, и
{\бф ограниченным}, если оно содержится в шаре $B_C(x)$,
для какого-то $C>0$.

Расстояние от точки до множества определяется так:
\[
d(x, Z) = \inf_{z\in Z} d(x, z).
\]
Это непрерывная, 1-липшицева функция на $M$.
Для замкнутого $Z\subset M$, $d(x, Z)=0$ тогда и только
тогда, когда $x \in Z$ (докажите). 

Для двух замкнутых, ограниченных 
подмножеств $Z_1, Z_2$ в метрическом
пространстве $M$, определим {\bf расстояние Хаусдорфа}
$d_H(X,Y)$ формулой
\[
  d_H(Z_1, Z_2):= \max \left( \sup_{x\in Z_1} d(x, Z_2), 
 \sup_{x\in Z_2} d(x, Z_1)\right).
\]
\утверждение
$d_H$ задает метрику на множестве всех 
замкнутых, ограниченных подмножеств $M$.

\хфилл

\ноиндент
{\бф Доказательство:} 
Супремум $\sup_{x\in Z_1} d(x, Z_2)$
конечен, потому что $Z_1$ и $Z_2$ содержатся в некотором
шаре $B_C(z)$, а расстояние между точками $B_C(z)$
ограничено $2C$ (докажите). Если этот супремум равен нулю,
это значит, что все точки $Z_1$ лежат в $Z_2$ и наоборот;
это доказывает положительность метрики. Симметричность
очевидна. Осталось доказать неравенство треугольника.

Для подмножества $Z\subset M$, и $\epsilon> 0$, 
определим $Z(\epsilon)$ ({\bf $\epsilon$-окрестность $Z$}) 
как объединение всех $\epsilon$-шаров с центром в $z\in Z$. Определение
расстояния Хаусдорфа можно переписать так:
\[
  d(Z_1, Z_2):= \inf_\epsilon \ \left \{ \epsilon \in \R\ \  | \ \
  Z_1\subset Z_2(\epsilon), Z_2 \subset Z_1(\epsilon)\right\}.
\]
``расстояние Хаусдорфа от $Z_1$ до $Z_2$ есть инфимум всех
$\epsilon$ таких, что $Z_1$ лежит в $\epsilon$-окрестности
$Z_2$, а $Z_2$ --- в $\epsilon$-окрестности $Z_1$".

Легко видеть, что 
\begin{equation}\label{_epsilon_okr_Equation_}
Z(\epsilon)(\epsilon') \subset Z(\epsilon+\epsilon').
\end{equation}

Если $d_H(Z_1, Z_2) < \epsilon$ и $d_H(Z_2, Z_3) < \epsilon'$,
то $Z_1$ лежит в $\epsilon$-окрестности $Z_2$, а $Z_2$ в 
$\epsilon'$-окрестности $Z_3$. Из 
$Z_2\subset Z_3(\epsilon')$, неравенства  
треугольника и \eqref{_epsilon_okr_Equation_} следует
$Z_2(\epsilon) \subset Z_3(\epsilon'+ \epsilon)$.
Получаем: \[ Z_1\subset Z_2(\epsilon)\subset Z_3(\epsilon'+ \epsilon).\]
Тот же самый аргумент, примененный к $Z_3$, $Z_2$ и $Z_1$,
дает \[ Z_1\subset Z_2(\epsilon)\subset Z_3(\epsilon'+ \epsilon)\]
Поэтому из $d_H(Z_1, Z_2) < \epsilon$ и $d_H(Z_2, Z_3) < \epsilon'$
следует $d_H(Z_1, Z_3)< \epsilon'+ \epsilon$. Это и есть
неравенство треугольника для $d_H$. Мы доказали,
что расстояние Хаусдорфа --- метрика. \endproof


%%%%%%%%%%%%%%%%%%%%%%%%%%%%%%%%%%%%%%%%%%%%%%%%%%%%%%%%%%%%

\section{$\epsilon$-сети}

%%%%%%%%%%%%%%%%%%%%%%%%%%%%%%%%%%%%%%%%%%%%%%%%%%%%%%%%%%%%

Весьма удобный критерий компактности можно получить,
воспользовавшись понятием $\epsilon$-сети.


Напомним, что {\бф $\epsilon$-сетью} в 
метрическом пространстве $M$ называется
такое подмножество $V \subset M$, что $M$
лежит в объединении всех $\epsilon$-шаров с центрами в $V$.
$\epsilon$-сеть называется {\бф конечной},
если $V$ конечно. 


%%%%%%%%%%%%%%%%%%%%%%%%%%%%%%%%%%%%%%%%%%%%%%%%
\замечание
Пусть $M$ --- метрическое пространство, а
$\epsilon >0$ --- вещественное число.
Подмножество $V\subset M$ называется {\бф
$\epsilon$-сетью}, если верно любое из следующих
равносильных утверждений (докажите равносильность)
\begin{description}
\item[(i)] $V(\epsilon) \supset M$
\item[(ii)] $d_H(M,V) < \epsilon$
\item[(iii)] $V$ --- это $\epsilon$-сеть.
\end{description}
\еза

\хфилл


%%%%%%%%%%%%%%%%%%%%%%%%%%%%%%%%%%%%%%%%%%%%%%%%
\утверждение
Пусть $M$ --- полное метрическое пространство.
Тогда $M$ компактно тогда и только тогда, когда
для каждого $\epsilon >0$ в $M$ найдется конечная
$\epsilon$-сеть. 

\хфилл

\ноиндент
{\бф Доказательство:} \\
Наличие конечной $\epsilon$-сети в каждом
компакте $M$ очевидно. Возьмем в качестве
покрытия $M$ множество всех $\epsilon$-шаров.
У него есть конечное подпокрытие $\{ B_\epsilon(x_i)\}$.
По определению, множество $\{x_i\}$ образует конечную
$\epsilon$-сеть.

Пусть, наоборот, в $M$ найдется конечная
$2^{-N}$-сеть, для любого $N$.
Возьмем произвольную 
последовательность $\{x_i\}$,
и пусть $V_1$ --- конечная $2^{0}$-сеть.
Тогда бесконечное множество элементов
$\{x_i\}$ лежат в некотором $2^{0}$-шаре.
Выкинем из $\{x_i\}$ все, которые там не лежат. 
Перейдем к конечной $2^{-1}$-сети,
и выкинем из $\{x_i\}$ все элементы, которые
не лежат в некотором $2^{-1}$-шаре, кроме 
первого. Поступим так же для всех $i$:
на $i$-м шаге выкинем из  $\{x_i\}$ все 
элементы, не лежащие в некотором  $2^{-i}$-шаре,
кроме первых $i$. Этот процесс стабилизируется:
начиная от $N$-го шага, все элементы последовательности
вплоть до $N$-го выбраны и не меняются. 
Полученная таким образом последовательность является,
последовательностью Коши (докажите), а значит --- сходится. \endproof




\hfill

В доказательстве теоремы Хопфа-Ринова в
лекции 4 нам понадобится следующий простой результат.
Заметим, что любое компактное подмножество
метрического пространства замкнуто и ограничено
(докажите). А значит, метрика Хаусдорфа определена
на компактных подмножествах.

\хфилл

%%%%%%%%%%%%%%%%%%%%%%%%%%%%%%%%%%%%%%%%%%%%%%%%%%%%%%%%%%%%
\утверждение \label{_predel_compa_Utverzhdenie_}
Пусть $\{Z_i\}$ --- последовательность компактных 
подмножеств в полном метрическом пространстве $M$.
Предположим, что $\{Z_i\}$ --- последовательность Коши
в метрике Хаусдорфа, а $Z$ ее предел. Тогда $Z$
тоже компактно.

\хфилл

\ноиндент
{\бф Доказательство:} \\
Поскольку $Z$ --- замкнутое подмножество $M$, оно полно.
Для доказательства компактности $Z$, мы построим
в $Z$ конечную $3\epsilon$-сеть, для любого наперед заданного
значения $\epsilon$. Возьмем конечную
$\epsilon$-сеть $x_0, ..., x_k$ в $Z_i$, для $d_H(Z_i, Z) <\epsilon$,
и пусть $z_1, ..., z_k$ --- точки в $Z$ такие, что
$d(z_i, x_i)< \epsilon$ (такие точки существуют,
потому что $d_H(Z_i, Z) <\epsilon$).
Обозначим через $V$ множество $\{x_0, ..., x_k\}$.
Тогда $V(\epsilon)$ содержит  $x_0, ..., x_k$.
По определению $\epsilon$-сети, из этого следует,
что $V(2\epsilon)$ содержит $Z_i$. А коль скоро
$d_H(Z_i, Z) <\epsilon$, $Z_i(\epsilon) \supset Z$,
и значит, $V(3\epsilon)\supset Z$.
Мы получили, что $V$ есть $3\epsilon$-сеть.
\endproof

%%%%%%%%%%%%%%%%%%%%%%%%%%%%%%%%%%%%%%%%%%%%%%%%

\section{Историческое отступление: \\
расстояние Громова-Хаусдорфа}

%%%%%%%%%%%%%%%%%%%%%%%%%%%%%%%%%%%%%%%%%%%%%%%%

В 1920-е годы, общую топологию немало изучали
в Москве П. С. Урысон (рано погибший, к сожалению),
П. С. Александров и их школа, в Польше Казимир Куратовский
(Kazimierz Kuratowski), Альфред Тарский (Alfred Tarski)
и Вацлав Серпинский (Wac\l aw Sierpi\'nski). 
Среди прочего, их интересовала {\бф проблема метризации}:
найти, какие топологические пространства получаются
из метрических забыванием метрики. Довольно скоро эти исследования 
исчерпали себя, и метрическими пространствами
(вне функционального анализа) практически не занимались.

В дифференциальной геометрии много изучали 
геометрию римановых многообразий (метрических
пространств, локально гомеоморфных $\R^n$, и с 
метрикой, которая в первом приближении евклидова, и 
гладко зависит от точки). Более общими пространствами 
дифференциальные геометры практически не интересовались.

Замечательным исключением были работы 
А. Д. Александрова и математиков его школы
среди которых наиболее знаменит Михаил Громов.



\begin{figure}[ht]
\begin{center}
\epsfig{file=Gromov.eps,width=0.5\linewidth}\\
Михаил Громов \\
(р. 23 декабря 1943)
\end{center}
\end{figure}


Громов определил метрику на множестве всех компактных
метрических пространств. Пусть $Z_1, Z_2$ --- компактные
метрические пространства, изометрически вложенные
в третье метрическое пространство $M$ (такие вложения
существуют по теореме Урысона). Рассмотрим инфимум
$\inf d_H(Z_1, Z_2)$ для всех таких вложений. 
Немножко повозившись, можно доказать, что это
действительно метрика.

В такой ситуации можно говорить о сходимости и пределе
метрических пространств ("предел Громова-Хаусдорфа"). 
Эта идея оказалась неожиданно
полезной в топологии и дифференциальной геометрии.
Довольно большие классы пространств оказались
компактными в метрике, заданной Громовым-Хаусдорфом;
из этого удалось вывести много важных ограничений
на топологические инварианты многообразий.

В 2000-е годы геометрия метрических пространств
получила дополнительный толчок. Григорий Перельман, изучая эволюцию
сложного диф\-фе\-рен\-ци\-аль\-но-\-ге\-о\-метрического уравнения
("потока Риччи"), смог классифицировать
вырождения решений этого уравнения. Он
обнаружил, что в громовском пределе 
пространство, на котором оно определено,
становится из многообразия негладким
метрическим пространством. Оказалось, 
что это предельное метрическое пространство 
устроено довольно просто. Вырезав из него
особые точки и доклеив их пленкой, Перельман снова 
получил гладкое многообразие, и продолжил на
нем эволюцию потока Риччи, получив
в пределе многообразие, где этот
поток стабилен. Такие пространства (``с постоянной
кривизной Риччи") были давно классифицированы. 
Таким образом Перельман доказал гипотезу Пуанкаре,
и получил классификацию трехмерных
многообразий.


%%%%%%%%%%%%%%%%%%%%%%%%%%%%%%%%%%%%%%%%%%%%%%%%%%%%%%%%%%%%%%%%%%%%%%%%

\chapter[Лекция 4: Внутренняя метрика]{Лекция 4:\\ Внутренняя метрика}

%%%%%%%%%%%%%%%%%%%%%%%%%%%%%%%%%%%%%%%%%%%%%%%%%%%%%%%%%%%%%%%%%%%%%%%%


%%%%%%%%%%%%%%%%%%%%%%%%%%%%%%%%%%%%%%%%%%%%%%%%

\section{Пространство с внутренней метрикой}

%%%%%%%%%%%%%%%%%%%%%%%%%%%%%%%%%%%%%%%%%%%%%%%%

Пусть $M$ метрическое пространство,
а $\gamma:\; [0,\alpha] \arrow M$ --- непрерывное отображение
из отрезка. Такое отображение
называется {\бф путем из $\gamma(0)$ в $\gamma(\alpha)$},
а $\gamma(0)$ и $\gamma(\alpha)$ --- {\бф началом} и {\бф концом} пути
$\gamma$, а также его {\бф концами}.

Рассмотрим разбиение отрезка $[0,\alpha]$ в объединение
меньших отрезочков, 
\[  [0,\alpha]= [0,х_1] \cup [x_1, x_2] \cup ... \cup [x_{n-1}, \alpha].
\]
Для простоты обозначим $x_0:= 0, x_n := \alpha$.
Пусть
\[
L_\gamma(x_1, ... x_{n-1})= \sum_{i=0}^{n-1} d(\gamma(x_i), \gamma(x_{i+1})).
\] 
\определение
{\бф Длиной} пути $\gamma$ называется супремум
\[ L(\gamma):=\sup_{x_1, ..., x_{n-1}}L_\gamma(x_1, ... x_{n-1}),
\]
взятый по всем разбиениям отрезка.
\ео

Конечно, такой супремум может быть равен
бесконечности, но для $C$-липшицева пути $\gamma:\; [0,\alpha] \arrow M$
длина $\gamma$ не превосходит $C\alpha$
(докажите). Также ясно, что $L(\gamma)\geq d(x,y)$,
где $x,y$ --- концы пути $\gamma$ (выведите это из 
неравенства треугольника).

Предположим, что для любых точек $x,y$ метрического
пространства $M$, найдется путь $\gamma$ 
конечной длины, соединяющий $x$ и $y$.
Легко видеть, что в таком случае инфимум
$L(\gamma)$ (``длина кратчайшего пути от $x$ до $y$") 
по всем таким путям задает метрику на $M$ (докажите). 

Метрика на $M$ называется {\bf внутренней метрикой}
(``intrinsic metric"), если этот инфимум равен $d(x,y)$.

Пусть $M$ --- метрическое пространство,
такое, что между любыми двумя точками $M$
есть путь конечной длины. Легко видеть, что функция,
ставящая в соответствие паре точек $x,y$
в $M$ инфимум длины путей из $x$ в $y$,
является внутренней метрикой (проверьте это).

\hfill

Внутренняя метрика отличается следующим свойством.

\хфилл


{\бф Условие Хопфа-Ринова:}\\
Пусть $M$ --- пространство с внутренней метрикой.
Для любых точек $x, y \in M$, и любого положительного
$r < d(x,y)$, имеем
\[
d(y, B_r(x)) = d(x,y) -r.
\]
Докажем это равенство.

\хфилл

Заметим, что неравенство $d(y, B_r(x)) \geq d(x,y) -r$
имеет место в произвольном метрическом пространстве,
и следует из неравенства треугольника (докажите). Поэтому для
доказательства условия Хопфа-Ринова нужно доказать 
неравенство $d(y, B_r(x)) \leq d(x,y) -r$.

Пусть $\gamma:\; [0, 1]\arrow M$ --- путь
из $x$ в $y$, длины $d(x,y)+\epsilon$. Такой путь
существует, потому что метрика внутренняя. Из определения
$L(\gamma)$ ясно, что 
\begin{equation}\label{_d_gamma_bound_Equation_}
d(x, z) + d(z,y) \leq d(x,y)+\epsilon.
\end{equation}
для любого $z$ в образе $\gamma$ (докажите).

Рассмотрим функцию $d_\gamma:\; [0,1] \arrow \R$,
$d_\gamma(t):=d(x, \gamma(t))$. 
Поскольку $d_\gamma$
непрерывна, и $d_\gamma(0)=0, d_\gamma(1) = d(x,y)+\epsilon$,
каждая точка отрезка $[0, d(x,y)+\epsilon]$
имеет прообраз. Возьмем $t_0$ такой, что
$d_\gamma(t_0)= r-\epsilon$. Тогда $z:=\gamma(t_0)$
лежит в шаре $B_r(x)$. В силу \eqref{_d_gamma_bound_Equation_}, 
имеем
\[ 
  d(y, z) \leq d(x,y) +\epsilon-d(x, z) = 
  d(x,y)+\epsilon- d_\gamma(t_0) = d(x,y)-r+2\epsilon.
\]
Поэтому $d(y, B_r(x)) \leq d(x,y)-r+2\epsilon$,
для любого $\epsilon >0$. 
Мы доказали условие Хопфа-Ринова. \endproof

\хфилл

Заметим, что не любое метрическое пространство,
удовлетворяющее условию Хопфа-Ринова, обладает
внутренней метрикой. Легко видеть, что никаких
непостоянных непрерывных отображений из отрезка
в рациональные числа нет, между тем рациональные
числа с обычной метрикой удовлетворяют 
условию Хопфа-Ринова (докажите это).


%%%%%%%%%%%%%%%%%%%%%%%%%%%%%%%%%%%%%%%%%%%%%%%%%%%%%%%%%%%%

\section{Локально компактные \\метрические пространства}

%%%%%%%%%%%%%%%%%%%%%%%%%%%%%%%%%%%%%%%%%%%%%%%%%%%%%%%%%%%%

Напомним, что {\бф открытыми множествами}
в метрическом пространстве называются произвольные
объединения открытых шаров, а {\бф замкнутыми множествами} --- их 
дополнения. 

Пусть $M$ --- метрическое пространство, $x\in M$ точка, $r>0$
вещественное число. Множество
\[
  \bar B_r(x):= \{ y\in M \ \ | \ \  d(x, y) \leq r\} 
\]
называется {\бф замкнутым шаром радиуса $r$ с центром в $x$}.
Это множество замкнуто. Действительно, для каждой
точки $y\notin \bar B_r(x)$, $d(x,y) > r$,
и открытый шар $B_\epsilon(y)$ не пересекается с
$\bar B_r(x)$ для любого $\epsilon \leq d(x,y) -r$
(выведите это из неравенства треугольника). 

Напомним, что {\бф замыканием} множества $Z$ называется
множество предельных точек всех последовательностей,
лежащих в $Z$. Легко видеть, что замыкание множества
замкнуто (докажите). 

Вообще говоря, замыкание открытого шара
$B_r(x)$ не равно замкнутому шару $\bar B_r(x)$. Возьмем в качестве
$M$ пространство с метрикой $d(x,y)=1$ для всех $x\neq y$;
тогда открытый шар $B_1(x)$ это точка $x$, а замкнутый шар --
все $M$.

Если $M$ удовлетворяет условию Хопфа-Ринова, то
замыкание $B_r(x)$ --- это $\bar B_r(x)$.
Действительно, возьмем любую точку 
$y\in \bar B_r(x)$.
Из условия Хопфа-Ринова легко вывести, что
$d(y, B_r(x))=0$ (выведите это). Но тогда некоторая
последовательность точек $B_r(x)$ сходится
к $y$, а значит, $y$ лежит в замыкании $B_r(x)$.

\определение
Метрическое пространство $M$ называется {\бф локально
компактным}, если для каждой точки $x\in M$ найдется
число $r>0$ такое, что замкнутый шар $\bar B_r(x)$
компактен.
\ео

\хфилл

Напомним, что {\бф ограниченным подмножеством}
метрического\\ пространства называется подмножество,
которое содержится в каком-то шаре $B_C(x)$.

\hfill

\теорема
\label{_H_R_1_Theorem_}
(теорема Хопфа-Ринова, часть 1)
Пусть $M$ --- локально компактное метрическое 
пространство с условием Хопфа-Ринова. Тогда следующие
утверждения равносильны:
\begin{description}
\item[(i)] $M$ полно
\item[(ii)] Любое замкнутое, ограниченное
подмножество $M$ компактно.
\end{description}


\ноиндент
{\бф Доказательство:}\\
Следствие (ii) $\Rightarrow$ (i) вполне очевидно.
Действительно, любая последовательность Коши $\{x_i\}$ содержится
в замкнутом шаре, который компактен по условию (ii),
значит $\{x_i\}$ сходится.

Осталось доказать, что в полном 
локально компактном метрическом пространстве с условием
Хопфа-Ринова любое замкнутое, ограниченное
подмножество компактно.

\хфилл

\ноиндент
{\бф Шаг 1.}\\
Поскольку замкнутое подмножество компакта компактно
(докажите), для утверждения теоремы Хопфа-Ринова 
достаточно доказать, что любой замкнутый шар в $M$
компактен. 

\хфилл

\ноиндент
{\бф Шаг 2.}\\
Рассмотрим функцию $\rho:\; M \arrow \R$,
\[
\rho(x):= \sup_\rho\{\rho\in \R \ \ |\ \ \bar B_\rho(x)\ \ \text{компактен}\}.
\]
Если $\rho$ бесконечно в одной точке $x$, это значит,
что любой замкнутый шар с центром в $x$ компактен.
Из этого легко следует, что все замкнутые шары в $M$
компактны (докажите это). Поэтому можно считать,
что функция $\rho$ везде конечна.

Легко видеть, что $\rho$ 1-липшицева, а следовательно
непрерывна. Действительно, 
$\bar B_{\rho-d(x,y)}(y)\subset\bar B_\rho(x)$
(выведите это из неравенства треугольника). 
Поэтому $|\rho(x)-\rho(y)| \leq d(x,y)$.

\хфилл

\ноиндент
{\бф Шаг 3.}\\
Докажем теперь, что замкнутый шар $\bar B_{\rho(x)}(x)$
компактен. А приори, это может быть и не так,
ведь в определении $\rho(x)$ используется супремум,
поэтому из этого определения следует лишь то, что
$\bar B_r(x)$ компактен для всех $r< {\rho(x)}$.

Из условия Хопфа-Ринова вытекает
$B_r(x)(\epsilon) = B_{r+\epsilon}(x)$,
где $Z(\epsilon)$ обозначает объединение
всех открытых $\epsilon$-шаров с центрами в $Z$.
Это позволяет вычислить расстояние Хаусдорфа между шарами
$\bar B_r(x)$ и $\bar B_{r+\epsilon}(x)$:
\[
  d_H(\bar B_r(x),\bar  B_{r+\epsilon}(x))=\epsilon.
\]
Из этого очевидно, что
для каждой последовательности $\{r_i\}$, сходящейся к $r$,
последовательность замкнутых шаров $\bar B_{r_i}(x)$
является последовательностью Коши (в смысле метрики
Хаусдорфа), и сходится к $\bar B_r(x)$. 

Возьмем последовательность $r_i< \rho(x)$, сходящуюся к $\rho(x)$.\\
Замкнутый шар $\bar B_{\rho(x)}(x)$ получается
как предел последовательности Коши $\bar B_{r_i}(x)$
компактных шаров. По утверждению, доказанному
в предыдущей лекции с помощью $\epsilon$-сетей,
такой предел всегда компактен. 

\хфилл

\ноиндент
{\бф Шаг 4.}\\
Воспользовавшись компактностью $\bar B_{\rho(x)}(x)$,
мы докажем, что шар \\$\bar B_{\rho(x)+\epsilon}(x)$
компактен для достаточно малого $\epsilon >0$.
Таким образом мы придем к противоречию.

Рассмотрим ограничение функции $\rho$ на 
$\bar B_{\rho(x)}(x)$. Поскольку $\rho$
непрерывна и положительна, а шар $\bar B_{\rho(x)}(x)$
компактен, \[ \rho\restrict {\bar B_{\rho(x)}(x)} \geq 2\epsilon >0,\]
для какого-то положительного $\epsilon\in \R$.
Поэтому каждый замкнутый $2\epsilon$-шар с центром в 
$z\in \bar B_{\rho(x)}(x)$ компактен.

Пусть $V$ --- конечная $\epsilon$-сеть в 
$\bar B_{\rho(x)}(x)$. Тогда 
$\bar B_{\rho(x)}(x)\subset V(\epsilon)$, а
\[
\bar B_{\rho(x)}(x)(\epsilon) \subset
V(\epsilon)(\epsilon)= V(2\epsilon).
\]
Мы получили, что шар $\bar B_{\rho(x)+\epsilon}(x)$
лежит в объединении замкнутых $2\epsilon$-шаров с 
центрами в точках $V$. Эти шары компактны, а 
поскольку конечное объединение компактов компактно (докажите это),
$\bar B_{\rho(x)+\epsilon}(x)$ является замкнутым подмножеством
компакта. Значит, $\rho(x)$ --- не супремум всех 
$\rho$, для которых $\bar B_\rho(x)$ компактно:
мы пришли к противоречию. Теорема \ref{_H_R_1_Theorem_}
доказана. \endproof



%%%%%%%%%%%%%%%%%%%%%%%%%%%%%%%%%%%%%%%%%%%%%%%%

\section{Геодезические в метрическом пространстве}

%%%%%%%%%%%%%%%%%%%%%%%%%%%%%%%%%%%%%%%%%%%%%%%%


\определение
Пусть $M$ --- метрическое пространство с внутренней
метрикой. Непрерывное отображение $\gamma:\; [0,\alpha]\arrow M$
называется {\бф кратчайшей}, если его длина равна 
$d(\gamma(0), \gamma(\alpha))$. 
\ео


Любой отрезок кратчайшей --- снова кратчайшая.
Действительно, если $\gamma(t_1), \gamma(t_2)$
можно соединить путем $\gamma'$, более коротким, чем
$\gamma$, тогда $\gamma(0), \gamma(\alpha)$
можно соединить путем, идущим от 0 до $t_1$
по $\gamma$, от $t_1$ до $t_2$ по $\gamma'$
и от $t_2$ до $\alpha$ по $\gamma$; этот путь
будет, очевидно, короче исходного.

\хфилл

Напомним, что гомеоморфизмом метрических пространств
называется непрерывная биекция, обратное отображение
к которому --- тоже непрерывно.

Если $\phi:\; [0,\alpha] \arrow [0,\alpha]$ --- гомеоморфизм,
а $\gamma$ --- путь из $x$ в $y$, композиция 
$\phi \circ\gamma$ --- тоже путь из $x$ в $y$.
В такой ситуации, $\phi \circ\gamma$ 
называется {\бф репараметризацией} пути $\gamma$.
Легко видеть, что длина пути не меняется при
его репараметризации (докажите). Поэтому
репараметризация кратчайшей --- снова кратчайшая.

Пути, которые получены посредством репараметризации,
называются {\бф эквивалентными с точностью до репараметризации},
а выбор пути в классе эквивалентности --- {\бф параметризацией}.

\определение
Пусть $\gamma:\; [0,\alpha]\arrow M$ --- кратчайшая,
соединяющая $x$ и $y$, причем $d(\gamma(x), \gamma(y))= |x-y|$
для любых $x,y$. Такая кратчайшая называется {\бф кратчайшей
геодезической}, а соответствующая параметризация --- 
{\бф геодезической параметризацией}. 
Очевидно, кратчайшая геодезическая задает
изометрическое вложение $[0,\alpha]\arrow M$.
\ео


\хфилл

%%%%%%%%%%%%%%%%%%%%%%%%%%%%%%%%%%%%%%%%%%%%%%%%
\утверждение
\label{_geode_sushche_Utverzhdenie_}
Пусть $\gamma:\; [0,\alpha]\arrow M$ --- кратчайшая,
соединяющая $x$ и $y$, причем $d(x,y)=\alpha$.
Тогда у $\gamma$ существует геодезическая
параметризация. 

\хфилл

\ноиндент
{\бф Доказательство:}
Утверждение \ref{_geode_sushche_Utverzhdenie_} 
легко увидеть из простых физических соображений.
Представьте себе велосипедиста,
который едет по дороге с переменной скоростью. Пусть
$\gamma(t)$ --- координата велосипедиста. Возьмем
вместо $t$ расстояние, которое велосипедист уже проехал;
полученная траектория (зависящая уже не от времени,
а от параметра "расстояние от начала") и
является кратчайшей геодезической.

Для доказательства Утверждения \ref{_geode_sushche_Utverzhdenie_},
нам понадобится несколько предварительных замечаний.

\хфилл

\лемма
Пусть $M\stackrel\phi\arrow N$ --- непрерывное отображение. Тогда
образ компактного подмножества $Z$ --- компакт.

\хфилл

\ноиндент
{\бф Доказательство:} Возьмем покрытие $\phi(Z)$
открытыми подмножествами; его прообраз дает открытое
покрытие $Z$, и там можно выбрать конечное подпокрытие
в силу компактности.
\endproof

\hfill

Если $M$ компактно, то любая непрерывная биекция
из $M$ в $N$ является гомеоморфизмом. Чтобы в этом
убедиться, достаточно проверить, что образ открытого
множества является открытым; в силу биективности, 
это эквивалентно тому, что образ замкнутого множества
замкнут. Но образ компакта при непрерывном отображении
всегда компактен, значит образ любого замкнутого подмножества
замкнут.

\хфилл

Вернемся к доказательству Утверждения \ref{_geode_sushche_Utverzhdenie_}.
Рассмотрим отображение $\phi:\; [0,\alpha]\arrow [0,\alpha]$,
$\phi(t)= d(x, \gamma(t))$. Это отображение непрерывно,
потому что функция $d_x(y) = d(x,y)$ непрерывна
(она липшицева; проверьте это). Поскольку каждый
отрезок кратчайшей --- кратчайшая, оно биективно:
действительно, $d(x, \gamma(t))$ равен длине пути
$\gamma\restrict{[0,t]}$, а значит эта функция
монотонно возрастает. Непрерывное, биективное
отображение из компакта в компакт --- гомеоморфизм,
как мы только что доказали. Поэтому 
$\gamma' =\phi^{-1}\circ\gamma$ является репараметризацией
$\gamma$. Для $t\in [0,\alpha]$,
\[ 
   d(x, \gamma(\phi^{-1}(t)))=\phi(\phi^{-1}(t))=t,
\]
а значит, $\gamma'$ --- геодезическая. 

Теоремой Хопфа-Ринова называется утверждение
о компактности ограниченного замкнутого подмножества в
локально компактном пространстве с внутренней метрикой
(Теорема \ref{_H_R_1_Theorem_}). Также теоремой
Хопфа-Ринова называется утверждение о наличии геодезических 
в полном, локально компактном пространстве с внутренней
метрикой.

\хфилл

\теорема
\label{_H_R_2_Theorem_}
(теорема Хопфа-Ринова, часть 2)
Пусть $M$ --- локально компактное, полное метрическое пространство
с условием Хопфа-Ринова, а $x_0, x_1\in M$ произвольные точки,
с $d(x_0, x_1)=\alpha$. Тогда существует кратчайшая геодезическая
$\gamma:\; [0,\alpha]\arrow M$, соединяющая $x_0$ и $x_1$.
В частности, $M$ является пространством с внутренней метрикой.

\хфилл

\ноиндент
{\бф Доказательство:} 
В силу условия Хопфа-Ринова, $d(x_0, \bar B_{\alpha/2}(x_1))=\alpha/2$.
Функция $d_{x_0}(y):= d(x_o,y)$ непрерывна, а шар $\bar B_{\alpha/2}(x_1)$
компактен по уже доказанной теореме Хопфа-Ринова. Поэтому
в $\bar B_{\alpha/2}(x_1)$ есть точка $x_{\frac 1 2}$ такая, что
$d(х_0, х_{\frac 1 2}) = d(х_{\frac 1 2}, x_1)= \alpha/2$.
Аналогичный арумент, примененный к паре $x_0, х_{\frac 1 2}$,
доказывает, что найдется точка $x_{\frac 14}$ такая, что
$d(х_0, х_{\frac 1 4}) = d(х_{\frac 1 4}, x_{\frac 1 2})=\alpha/4$.
Повторяя это до бесконечности, мы получим для
каждого рационального числа $\lambda =
\frac{n}{2^k}$,\footnote{Такие рациональные числа
называются {\бф двоично-рациональными}.}
$0\leq \lambda\leq 1$ точку $x_\lambda\in M$, причем
$d(x_\lambda, x_{\lambda'})= |\lambda-\lambda'|\alpha$.
Это задает изометрическое отображение из множества
чисел вида $\lambda\alpha$ ($\lambda$ двоично-рациональное)
на отрезке $[0, \alpha]$ в $M$. Поскольку $M$ полно,
можно продолжить это отображение до отображения
пополнений. Получим изометрическое отображение
из отрезка $[0, \alpha]$ в $M$. Это и есть 
кратчайшая геодезическая. Мы доказали 
Теорему \ref{_H_R_2_Theorem_}.



%%%%%%%%%%%%%%%%%%%%%%%%%%%%%%%%%%%%%%%%%%%%%%%%%%%%%%%%%%%%

\section{История, терминология, литература}

%%%%%%%%%%%%%%%%%%%%%%%%%%%%%%%%%%%%%%%%%%%%%%%%%%%%%%%%%%%%

Пространства с внутренней метрикой возникли в
работах Хайнца Хопфа в 1930-е годы. Хопф изучал
топологию римановых многообразий, и обнаружил,
что многие локальные результаты, полученные
из анализа (существование геодезических,
локальная компактность и так далее) верны
глобально, и позволяют получить много 
информации о топологии многообразия.




\begin{figure}[ht]
\begin{center}
\epsfig{file=Hopf.eps,width=0.6\linewidth}\\
Heinz Hopf \\
(1894-1971)
\end{center}
\end{figure}

Многообразие есть топологическое пространство, локально
(в ок\-рес\-тности каждой точки) гомеоморфное $\R^n$.
Такие гомеоморфизмы называются {\бф картами},
совокупность всех карт --- {\бф атласом} на многообразии.
С каждым атласом связаны отображения перехода от 
одной карты к многообразию и к другой карте;
все это отображения из открытых подмножеств в $\R^n$ 
в открытые подмножества в $\R^n$. Если они все гладкие
(бесконечно дифференцируемые), многообразие называется
{\бф гладким}. Гладкое многообразие называется
{\бф римановым}, если в каждом касательном пространстве
задана метрика (положительно определенное скалярное
произведение). Интегрируя эту метрику по гладкому пути, 
можно получить функционал длины гладкого пути;
расстояние между точками $x,y$ риманова многообразия
определяется как инфимум длины по всем гладким 
путям из $x$ в $y$. Эта метрика по построению внутренняя,
а $\R^n$ очевидно локально компактно. Таким 
образом, доказанные выше теоремы можно применить 
к римановым многообразиям.

В последние 20-30 лет основные результаты
в топологии (доказательство гипотезы Пуанкаре, 
инварианты Дональдсона и Зайберга-Уиттена) происходят 
из римановой геометрии, то есть геометрии римановых
многообразий.

Хопф и его ученик Вилли Ринов (Willi Rinow, 1907-1979) получили
теорему Хопфа-Ринова в 1931-м году, для римановых
многообразий. Ее обобщение для локально компактных метрических
пространств принадлежит Стефану Кон-Фоссену
(Stephan Cohn-Vossen, 1902-1936).

По римановой геометрии есть огромное количество
литературы, по большей части совершенно нечитабельной.
Лично мне была полезна книжка Милнора ``Теория Морса",
и ``Эйнштейновы многообразия" Артура Бессе.
Геометрия пространств с внутренней метрикой восходит,
по большей части, к Громову, и к математикам школы 
А. Д. Александрова (Ю. Бураго, 
Г. Перельман, Д. Бураго, С. Иванов). 

Вот небольшой список литературы, которая может оказаться полезной.

\begin{itemize}

\item Милнор, Дж., {\em Теория Морса},  М.: Мир, 1965 г.

\item Бессе, А., {\em Многообразия Эйнштейна}, М.: Наука, 1990.

\item Бураго Д.Ю., Бураго Ю.Д., Иванов С.В.,
{\em Курс метрической геометрии}, Ижевск:
издательство "Институт компьютерных исследований," 
2004.

\item Громов М. {\ем Гиперболические группы},
Ижевск: Институт компьютерных исследований, 2002, 160 стр.

\item Громов М. {\em Знак и геометрический смысл
кривизны}, Ижевск: НИЦ "Регулярная и хаотическая 
динамика", 2000, 128 стр.

\item Gromov M. {\em Metric structures for 
Riemannian and non-Riemannian spaces,}
Progress in Math., 152, Birkh\"auser  (1999).

\item Труды Г. Перельмана, А. Петрунина и других авторов на
странице А. Петрунина\\
{\tt http://www.math.psu.edu/petrunin/papers/papers.html}

\end{itemize}

%%%%%%%%%%%%%%%%%%%%%%%%%%%%%%%%%%%%%%%%%%%%%%%%%%%%%%%%%%%%%%%%%%%%%%%%

\chapter{Лекция 5: Основы общей топологии}

%%%%%%%%%%%%%%%%%%%%%%%%%%%%%%%%%%%%%%%%%%%%%%%%%%%%%%%%%%%%%%%%%%%%%%%%

\section{Топологическое пространство}

Определение топологического пространства,
употребляемое и по сей день, принадлежит Хаусдорфу.

\hfill

\noindent
Напомним, что $2^M$ обозначает множество всех подмножеств в $M$.

\определение
Пусть $M$ --- множество, а ${\cal U} \subset 2^M$
набор подмножеств, называемых {\бф открытыми}. 
Мы говорим, что ${\cal U}$ {\бф задает топологию}
на $M$, если 
\begin{description}
\item[(i)] Любое объединение открытых подмножеств открыто
\item[(ii)] Конечное пересечение открытых подмножеств открыто
\item[(iii)] $M$ и пустое множество $\emptyset$ открыты.
\end{description}
В такой ситуации $M$ называется {\бф топологическим пространством}.
\ео

\задача
Проверьте, что метрическое пространство, с
обычным понятием открытого множества, удовлетворяет
этим аксиомам.
\ез

\определение
Замкнутым множеством называется множество, 
дополнение которого открыто.
\ео

Заметим, что вместо аксиом, использующих открытые
множества, можно было бы выбрать аксиомы, 
основанные на замкнутости. Получается весьма
похожая система аксиом: (i) ``пересечение любого количества
замкнутых множеств замкнуто'', (ii) ``объединение
конечного числа замкнутых множеств замкнуто",
(iii) ``$M$ и пустое множество $\emptyset$ замкнуты".

\определение
{\бф Окрестностью} подмножества $Z\subset M$ называется
любое открытое множество, содержащее $Z$. {\бф Замыканием}
подмножества $Z\subset M$ называется пересечение
всех замкнутых подмножеств, содержащих $Z$.
\ео

\задача
Докажите, что замыкание любого множества замкнуто.
Найдите замыкание $\Q$ в $\R$.
\ез

\определение
Подмножество $M$ называется {\бф всюду плотным},
если замыкание его совпадает с $M$. Оно называется
{\бф нигде не плотным}, если его замыкание
не содержит непустых открытых подмножеств $M$.
\ео

\задача
Приведите пример нигде не плотного, континуального
подмножества отрезка (в обычной топологии).
\ез

\определение
Пусть $M$, $N$ топологическое пространство.
Отображение $\phi:\; M \arrow N$ называется
{\бф непрерывным}, если прообраз любого открытого множества открыт.
\ео

\определение
{\бф Пределом} последовательности $\{x_i\}$ в $M$
называется такая точка $x\in M$, что в любой окрестности
$x$ содержатся почти все элементы $\{x_i\}$.
\ео

\задача
Придумайте пример пространства, в котором
предел не единственный. Докажите, что
образ предела, при непрерывном отображении --- всегда предел.
\ез

\определение
Пусть $M$ --- топологическое пространство.
{\бф Базой топологии} (base of topology) на $M$ называется
набор ${\cal U}\subset 2^M$ подмножеств $M$,
состоящий из открытых множеств, и такой,
что любое открытое подмножество $M$ получено
объединением набора элементов ${\cal U}$.
\ео

\определение
Пусть $Z\subset M$ --- подмножество топологического пространства 
$M$. Подмножества вида $U\cap Z$, где $U$ открыто в $M$,
задают топологию на $Z$ (докажите). Эта топология
называется {\бф индуцированной с $M$}.
\ео 


%%%%%%%%%%%%%%%%%%%%%%%%%%%%%%%%%%%%%%%%%%%%%%%%
\section{Аксиомы Хаусдорфа}
%%%%%%%%%%%%%%%%%%%%%%%%%%%%%%%%%%%%%%%%%%%%%%%%

В определении топологического пространства, данном
Хаусдорфом, требовалось еще одно условие: условие
{\ем отделимости}. Впоследствии оказалось, что неотделимые
топологические пространства встречаются весьма часто,
и это условие стали рассматривать как дополнительную аксиому.

\определение
Топологическое пространство $M$ называется {\бф
отделимым}, или {\бф хаусдорфовым} (separated, Hausdorff),
если любые две точки $x\neq y \in M$ имеют
непересекающиеся окрестности $U\ni x, V \ni y$.
\ео

\задача
Докажите, что в хаусдорфовом топологическом
прост\-ранс\-тве предел последовательности единственен.
\ез

В алгебраической геометрии важную роль играет топология
Зариского. Пусть $R$ --- кольцо, $\Spec(R)$ --- множество
его простых идеалов, a $f \in R$ --- любой элемент. 
Обозначим через $A_f$ подмножество в  $\Spec(R)$,
состоящее из всех идеалов, которые не содержат $f$.
Рассмотрим на $\Spec(R)$ топологию, где
база открытых множеств состоит из $A_f$, для
всех $f\in R$. Эта топология называется 
{\бф топологией Зариского}, а $\Spec(R)$ --- 
{\бф спектром} кольца.

\begin{figure}[ht]
\begin{center}
\epsfig{file=Oscar_Zariski.eps,width=0.4\linewidth}\\
Oscar Zariski \\
(1899 --- 1986)
\end{center}
\end{figure}


\задача
Рассмотрим пространство $\Spec(\Z)$,
с топологией Зариского. Докажите, что оно нехаусдорфово.
\ез




В 1920-е годы математики придумали
целую линейку аксиом, $T_0$-$T_6$, которые называются
 {\ем аксиомы отделимости} (separation axioms). 
Каждая из них сильнее всех предыдущих (докажите это). 
Довольно часто, аксиомы $T_0$-$T_6$ также называют
"аксиомами Хаусдорфа". 


\определение\ \\
\begin{description}
\item[$T_0$] (Аксиома Колмогорова) Для любых двух точек $x\neq y \in M$, 
у одной есть окрестность, не содержащая другую точку.

\item[$T_1$] (Аксиома Фреше)
Для любых двух точек $x\neq y \in M$, 
у $x$ есть окрестность, не содержащая $y$.
Равносильная формулировка: все точки $M$ являются
замкнутыми множествами (докажите равносильность).

\item[$T_2$] (аксиома Хаусдорфа)
У любых двух точек $x\neq y \in M$
есть непересекающиеся окрестности.

\item[$T_{2\frac 1 2}$] (аксиома Урысона)
У любых двух точек $x\neq y \in M$
есть окрестности, замыкания которых не пересекаются.

\item[$T_3$] В $M$ выполняется аксиома $T_0$.
К тому же, для любого замкнутого множества
$Z\subset M$, и любой точки $x\notin Z$,
у $Z$ и $x$ есть непересекающиеся окрестности.

\item[$T_4$] В $M$ выполняется аксиома $T_1$.
К тому же, любые два непересекающихся замкнутых
подмножества $M$ имеют непересекающиеся окрестности.

\item[$T_5$] В любом подмножестве $M$, взятом с 
индуцированной топологией, выполняется
аксиома $T_4$.

\item[$T_6$] В $M$ выполняется аксиома $T_4$.
К тому же, каждое замкнутое множество можно
получить как счетное пересечение открытых.
\end{description}
\ео

Кроме этих, существует немало других аксиом отделимости,
но они не очень употребительны. Впрочем, аксиомы
$T_0$-$T_6$ тоже употребляются весьма редко, кроме 
аксиомы Хаусдорфа $T_2$.


В любом метрическом пространстве $M$ выполняется
$T_6$ (а значит, и все остальные аксиомы из списка).
Действительно, $\epsilon$-окрестность любого множества
открыта (докажите), и каждое замкнутое подмножество 
метрического пространства получается
как пересечение своих $\epsilon$-окрестностей.
Чтобы убедиться, что в $M$ выполняется $T_4$,
возьмем замкнутые, непересекающиеся множества
$Z_1$, $Z_2$ в $M$, и для каждой точки
$x\in Z_1$ возьмем шар $B_{\frac r 3}(x)$,
где $r= d(x, Z_2)$. Объединение всех таких
шаров открыто, и не пересекается с окрестностью 
$Z_2$, полученной таким же образом.


\задача
Пусть $Z_1$, $Z_2$ --- непересекающиеся
замкнутые подмножества пространства, где верно $T_4$.
Докажите, что у $Z_1, Z_2$ есть окрестности, замыкания
которых не пересекаются.
\ез

{\small Докажем следствие $T_6 \Rightarrow T_5$. Пусть $Z\subset
M$ --- любое подмножество, а $K, K'\subset Z$ непересекающиеся
подмножества, которые замкнуты в $Z$. Обозначим их
замыкания в $M$ как $\bar K$, $\bar K'$. Легко видеть,
что $K$ не пересекается с $\bar K'$, а 
$K'$ не пересекается с $\bar K$. Обозначим
за $K''$ пересечение $\bar K\cap \bar K'$.
Тогда $K''$ получается как пересечение
счетного семейства открытых множеств $U_i$.
 Возьмем у $K\backslash U_0$, $K'\backslash U_0$ непересекающиеся
окрестности $V_0, V'_0$. Воспользовавшись  предыдущей
задачей, можно предположить, что замыкания 
$\bar V_0, \bar V'_0$ не пересекаются.
Применив $T_4$ к $\bar V_0\cup K\backslash U_1$,
$\bar V'_0\cup K'\backslash U_1$, получим
окрестности $V_1$, $V'_1$ замкнутых подмножеств
$K\backslash U_1$, $K'\backslash U_1$, замыкания
которых не пересекаются. Применив индукцию,
получим систему открытых множеств
$V_0 \subset V_1\subset V_2 \subset ...$,
$V_0' \subset V_1'\subset V_2' \subset ...$
которые не пересекаются, причем
$V_i\supset K\backslash U_i$ и $V_i'\supset K'\backslash U_i$.
Объединение всех $V_i$ содержит $K$, открыто,
и не пересекается с объединением всех $V'_i$, которое
содержит $K'$.}

\hfill

Если пропустить предыдущий абзац, никакой беды в этом не будет,
в дальнейшем этот аргумент не используется.

\задача[*]
Для каждой из аксиом $T_i$, 
придумайте примеры про\-ст\-ранств, 
в которых она выполнена, а предыдущая
не выполнена.
\ез

\задача 
Докажите, что в пространстве $\Spec(R)$ с топологией
Зариского выполнена аксиома $T_0$, а $T_1$ не выполнена
для $\Spec(\Z)$.
\ез

%%%%%%%%%%%%%%%%%%%%%%%%%%%%%%%%%%%%%%%%%%%%%%%%%%%%%%%%%%%%
\section{Аксиомы счетности}
%%%%%%%%%%%%%%%%%%%%%%%%%%%%%%%%%%%%%%%%%%%%%%%%%%%%%%%%%%%%


\определение
Пусть $M$ --- топологическое пространство, а $x\in M$ --- точка.
Набор окрестностей $\{U_\alpha\ni x\}$ называется
{\bf базой окрестностей в точке} ("local base"), если каждая
окрестность $x$ содержит какую-то из окрестностей $U_\alpha$.
\ео

\определение
Топологическое пространство {\бф обладает счетной базой в точке},
если у каждой точки есть счетная база окрестностей.
Это условие также называется {\бф первой аксиомой счетности}
(first axiom of countability).
\ео

Метрическое пространство удовлетворяет этому условию
(докажите).

Для пространства $M$ со счетной базой окрестностей в точке,
$Z\subset M$ замкнуто тогда и только тогда, когда оно содержит
предельные точки всех последовательностей. Доказательство
вполне аналогично доказательству этого факта для
метрических пространств (докажите). Поэтому
для пространств со счетной базой окрестностей в точке,
непрерывность можно определять через пределы
последовательностей, как это делается для метрических
пространств.


\определение
Топологическое пространство {\бф обладает счетной базой},
если у него есть счетная база открытых множеств.
Это условие также называется {\бф второй аксиомой
счетности} (second axiom of countability). 
\ео

По-английски, пространства, удовлетворяющие
первой и второй аксиоме счетности, часто называют
first-countable, second-countable.

В топологии и алгебраической геометрии, слово
"сепарабельность" --- синоним отделимости. В анализе
бесконечномерных пространств, "сепарабельно" значит
"содержит плотное, счетное подмножество". Не 
перепутайте!

\задача
Докажите, что пространство со счетной базой в точке
содержит плотное, счетное подмножество тогда и только 
тогда, когда у него есть счетная база.
\ез

%%%%%%%%%%%%%%%%%%%%%%%%%%%%%%%%%%%%%%%%%%%%%%%%%%%%%%%%%%%%%%%%%%%%%%%%

\chapter{Лекция 6: Произведение пространств}[Лекция 6: \\Произведение пространств]

%%%%%%%%%%%%%%%%%%%%%%%%%%%%%%%%%%%%%%%%%%%%%%%%%%%%%%%%%%%%%%%%%%%%%%%%

%%%%%%%%%%%%%%%%%%%%%%%%%%%%%%%%%%%%%%%%%%%%%%%%%%%%%%%%%%%%
\section{Свойства произведения}
%%%%%%%%%%%%%%%%%%%%%%%%%%%%%%%%%%%%%%%%%%%%%%%%%%%%%%%%%%%%

\определение
Пусть $M$ --- топологическое пространство.
Напом\-ним, что {\бф базой топологии} на $M$ называется
такой набор открытых подмножеств $\{U_\alpha\}\subset
2^M$, что любое открытое множество получается как 
объединение элементов $U_\alpha$.
{\бф Предбазой топологии} на $M$ называется
такой набор открытых подмножеств $\{U_\alpha\}\subset
2^M$, что любое открытое множество получается как 
объединение (возможно, бесконечное) и 
конечное пересечение элементов из набора 
$\{U_\alpha\}$.
\ео

\замечание
Любой набор множеств $\{U_\alpha\}\subset
2^M$, такой, что \[ \bigcup_\alpha U_\alpha=M,\] является
предбазой некоторой топологии на $M$. Определим  топологию на $M$
таким образом, что открытые множества получаются
объединениями и конечными пересечениями элементов
$\{U_\alpha\}$. Проверьте, что это топология.
\еза

\замечание
По той же самой причине, базой некоторой топологии на $M$
является любой набор множеств $\{U_\alpha\}\subset
2^M$, который замкнут относительно конечных
пересечений\footnote{{\bf Замкнутость $\{U_\alpha\}$ относительно
конечных пересечений} означает, что для любого конечного
набора $U_1, ... U_k \in \{U_\alpha\}$,
пересечение $\bigcap_i U_i$ лежит в $\{U_\alpha\}$.}
и удовлетворяет $\bigcup_\alpha U_\alpha=M$ (проверьте).
\еза

Пусть $M, M'$ --- топологические пространства.
Пусть ${\cal U}\subset 2^{M\times M'}$ --- набор подмножеств
$M\times M'$, состоящий из всех подмножеств
вида $U\times U'$, где $U\subset M$, $U'\subset M'$
открыты. Очевидно, ${\cal U}$ 
замкнуто относительно взятия конечных пересечений
(проверьте это). В силу вышеизложенного,
оно является базой топологии на $M\times M'$.

\определение
Рассмотрим $M\times M'$ с топологией, заданной базой
открытых множеств вида $U\times U'$, где $U\subset M$, $U'\subset M'$
открыты. Это топологическое пространство называется
{\бф произведением $M_1$ и $M_2$}.
\ео

\задача
Докажите, что произведение пространств,
удовлетворяющих первой (второй) аксиоме счетности,
снова удовлетворяет первой (второй) аксиоме счетности.
\ез


Напомним, что хаусдорфово топологическое пространство --
такое пространство, что любые две точки его имеют
непересекающиеся ок\-рес\-тности. 
Произведение хаусдорфовых топологических
пространств снова хаусдорфово. Действительно,
пусть $(x,x')$ и $(y, y')$ --- две разные точки в $M\times M'$.
Тогда либо $x\neq y$, либо  $x'\neq y'$. Предположим,
что верно первое. Возьмем непересекающиеся
окрестности $U, V$ у $x, y$, тогда открытые 
множества $U\times M'$, $V\times M'$ содержат
$(x,x')$ и $(y, y')$ и не пересекаются.

\определение
Отображение $M \arrow M\times M$, $x\stackrel \Delta \arrow (x,x)$
называется {\бф диагональным вложением}, а его
образ --- {\бф диагональю}. Докажите, что $\Delta$ непрерывно.
\ео

\задача
Докажите, что пространство $M$ является
хаусдорфовым тогда и только тогда, когда 
диагональ --- замкнутое подмножество в $M \times M$.
\ез

Произведение нескольких топологических пространств
определяется индуктивно:
$(M_1\times M_2)\times M_3\times ...$.
Докажите, что результат не зависит от порядка
расстановки скобок. 

%%%%%%%%%%%%%%%%%%%%%%%%%%%%%%%%%%%%%%%%%%%%%%%%

\section{Отображения в $M\times M'$}

%%%%%%%%%%%%%%%%%%%%%%%%%%%%%%%%%%%%%%%%%%%%%%%%


\определение
Пусть на множестве $M$ заданы две топологии: ${\cal U}_1$
и ${\cal U}_2$. Говорится, что {\бф ${\cal U}_1$ слабее
  ${\cal U}_2$}, а {\bf ${\cal U}_2$ сильнее ${\cal U}_1$},
если тождественное отображение $(M, {\cal U}_2) \arrow (M,
{\cal U}_1)$
непрерывно.
\ео


Чем {\ем слабее} топология, тем {\ем больше} 
сходящихся последовательностей, {\ем меньше}
непрерывных функций и {\ем меньше} открытых множеств;
на каждом множестве, кодискретная топология --- самая слабая,
а дискретная --- самая сильная.

Очевидно, топология произведения --- слабейшая 
топология на $M\times M'$ такая, что проекции
$M\times M'\stackrel \pi\arrow M$, $M\times M'\stackrel {\pi'}\arrow M'$
непрерывны. Действительно, предбаза топологии
на $M\times M'$ порождена прообразами
открытых множеств при этих проекциях.

Топологию произведения можно охарактеризовать следующим
образом.

\хфилл

%%%%%%%%%%%%%%%%%%%%%%%%%%%%%%%%%%%%%%%%%%%%%%%%%%%%%%%%%%%%
\утверждение\label{_product_functiorial_Utverzhdenie_}
Пусть $M, M', X$ --- топологические пространства,
a $M\times M'\stackrel \pi\arrow M$, $M\times M'\stackrel
{\pi'}\arrow M'$ --
отображения проекции.
Тогда отображение $X \stackrel \phi\arrow M\times M'$
непрерывно тогда и только тогда,
когда непрерывны композиции $\phi \circ \pi$, $\phi \circ\pi'$.
Это задает биекцию между множествами
\begin{equation}\label{_nepre_bie_Equation_}
 \left\{ \begin{minipage}{0.3\linewidth}
{пары непрерывных \\отображений \\ $X\stackrel \phi \arrow
M$, $X\stackrel {\phi'}\arrow M'$}\end{minipage} \right\} 
\leftrightarrow 
\left\{ \begin{minipage}{0.2\linewidth}{непрерывные\\отображения \\ $X\arrow M\times M'$}\end{minipage}\right\}
\end{equation}
{\бф Доказательство:}\\
Биекция \eqref{_nepre_bie_Equation_}
строится так: паре $(\phi, \phi')$ 
ставится в соответствие отображение
$x\stackrel \Phi \arrow (\phi(x), \phi'(x))$. 
Если $\Phi$ непрерывно, то $\phi$, $\phi'$
тоже непрерывны, потому что они получены
композицией $\Phi$ с проекцией. С другой стороны,
\[ 
  \Phi^{-1}(U\times U') = \phi^{-1}(U) \cap
  {\phi'}^{-1}(U')
\]
(проверьте это). Следовательно, $\Phi^{-1}(V)$
открыто для любого открытого $V\subset M\times M'$,
если $\phi, \phi'$ непрерывны.
\endproof

\hfill

Пусть $f:\; X \arrow Y$ --- любое отображение.
{\бф График $f$} --- это множество всех пар
вида $(x, f(x))\in X\times Y$. Диагональ является
графиком тождественного отображения. 


%%%%%%%%%%%%%%%%%%%%%%%%%%%%%%%%%%%%%%%%%%%%%%%%

\section{Произведение метрических пространств}

%%%%%%%%%%%%%%%%%%%%%%%%%%%%%%%%%%%%%%%%%%%%%%%%

Обозначим, как и раньше, за $\R^{\geq 0}$
множество всех неотрицательных чисел.

\хфилл

\утверждение
Пусть $(M, d)$ и $(M', d')$ --- метрические
пространства, а $\rho:\; (\R^{\geq 0})^2\arrow \R^{\geq 0}$ функция,
удовлетворяющая следующим  условиям: 
\begin{description}
\item[невырожденность:] $\rho(\lambda, \mu)=0$ $\Leftrightarrow$ $\lambda=\mu=0$
\item[субаддитивность:] $\rho(x+y) \leq \rho(x)+\rho(y)$
\item[монотонность:] $\rho (a, b)\geq \rho(a_1, b_1)$, если $a\geq a_1$,
а $b\geq b_1$.
\end{description}
Тогда
\[
d_\rho((x, x'), (y, y')):= \rho(d(x,y), d'(x',y'))
\]
задает метрику на $M\times M'$.

\хфилл

{\бф Доказательство:}
Симметричность $d_\rho$ следует прямо из определения,
а невырожденность --- из невырожденности $\rho$.
Неравенство треугольника выводится так:
\begin{align*}
d_\rho((x, x'), (z, z')) & = 
\rho(d(x,z), d'(x', z'))
\\ \ &\leq \rho(d(x,y)+ d(y,z), d'(x',y')+ d'(y',z'))\\ \ &\leq 
\rho(d(x,y), d'(x',y')) + \rho(d(y, z), d'(y',z'))\\ \ & = 
d_\rho((x, x'), (y, y'))+ d_\rho((y, y'), (z, z')).
\end{align*}
(первое неравенство следует из монотонности,
второе из субаддитивности). \endproof

\хфилл

Рассмотрим функцию $\rho_2(x,y) = \sqrt{x^2 + y^2}$.
Эта функция монотонна, субаддитивна и невырождена
(проверьте), а поэтому задает метрику $d_{\rho_2}$
на произведении метрических пространств.
Такая метрика называется {\бф метрикой произведения}. 
Легко видеть, что $\R \times \R \times \R \times ...$
изометрично $\R^n$.

Вместо $\sqrt{x^2 + y^2}$ можно рассматривать
другие функции, например $\rho_\infty(x,y): = \max(x,y)$
и $\rho_1(x,y) := x+y$. Проверьте, что эти функции
тоже задают метрику на $M\times M'$.

\задача
Пусть $a, b \in \R^{\geq 0}$.
Докажите следующие неравенства.
\[
\sqrt{a^2+ b^2}\leq a+b \leq 2 \sqrt{a^2+ b^2}.
\]
\[
\max(a,b) \leq a+b \leq 2 \max(a,b).
\]
\ез

Из первого неравенства следует, что тождественное отображение
\begin{equation}\label{_id_onprod_Equation_}
(M\times M', d_{\rho_2}) \arrow (M\times M', d_{\rho_1})
\end{equation}
липшицево, и обратное ему тоже липшицево. Значит,
\eqref{_id_onprod_Equation_}
является гомеоморфизмом. Из второго неравенства
следует, что тождественное отображение
$(M\times M', d_{\rho_\infty}) \arrow (M\times M',
d_{\rho_1})$ тоже является гомеоморфизмом.

\задача[*]
Будет ли тождественное отображение
\[ (M\times M', d_{\rho_2}) \arrow (M\times M', d_{\rho})\]
липшицевым для любой функции
$\rho:\; (\R^{\geq 0})^2\arrow \R^{\geq 0}$,
которая монотонна, выпукла и невырождена?
Будет ли оно всегда гомеоморфизмом?
\ез

Из определения ясно, что открытые шары
в $(M\times M', d_{\rho_\infty})$ имеют вид
$B_r(x) \times B_r(x')$. Такие шары, очевидно,
открыты в топологии произведения, и поэтому тождественное
отображение \[ M\times M'\arrow (M\times M', d_{\rho_\infty})\]
является непрерывным. Обратное отображение
непрерывно в силу Ут\-вер\-ждения
\ref{_product_functiorial_Utverzhdenie_}, 
поскольку непрерывны проекции
$(M\times M', d_{\rho_\infty}) \arrow M$
и $(M\times M', d_{\rho_\infty}) \arrow M'$
(обе эти проекции липшицевы). 


%%%%%%%%%%%%%%%%%%%%%%%%%%%%%%%%%%%%%%%%%%%%%%%%%%%%%%%%%%%%

\section{Полуметрики и полунормы}

%%%%%%%%%%%%%%%%%%%%%%%%%%%%%%%%%%%%%%%%%%%%%%%%%%%%%%%%%%%%


\определение
Пусть $d:\; M \times M \arrow \R^{\geq 0}$
функция, удовлетворяющая следующим условиям
\begin{description}
\item[Рефлексивность:] $d(x,x)=0$
\item[Симметричность:]\ \  $d(x,y)=d(y,x)$
\item[Неравенство треугольника:]\ \  $d(x,y) \leq d(x, z) + d(z,y)$
\end{description}
для любых точек $x,y,z\in M$.
Тогда $d$ называется {\бф полуметрикой} ("semimetric").
\ео

От определения метрики это отличается только
отсутствием условия невырожденности: $d(x,y)=0$
$\Rightarrow$ $x=y$.

Отметим, что условие $d(x,y)=0$ задает на $M$
отношение эквивалентности, что следует из неравенства
треугольника (проверьте). Более того, если $d(x,y)=0$,
то
\[
d(z,x) + d(x,y) \geq d(y,z), \ \ d(z,y) + d(y,x) \geq d(z,x),
\]
в силу неравенства треугольника. Первое неравенство дает
$d(z,x)  \geq d(y,z)$, а второе --- дает $d(z,x)  \leq d(y,z)$,
поскольку $d(x,y)=0$. Получаем, что $d$ корректно определена на 
множестве $\underline M$ классов эквивалентности по отношению $d(x,y)=0$.
Эта функция является метрикой (проверьте). Мы получили
следующее утверждение

\хфилл

%%%%%%%%%%%%%%%%%%%%%%%%%%%%%%%%%%%%%%%%%%%%%%%%%%%%%%%%%%%%
\утверждение\label{_metrika_iz_polumetriki_Utverzhdenie_}
каждое пространство $(M,d)$ с полуметрикой наделено сюръективным
отображением $\pi:\; M \arrow \underline M$
в метрическое пространство $(\underline M, \underline d)$,
при этом 
\begin{equation}\label{_semimetric_Equation_}
d(x,y) = \underline d(\pi(x), \pi(y)).
\end{equation}
\endproof

\hfill

Начав с произвольного отображения 
$\pi:\; M \arrow \underline M$ из множества $M$ 
в метрическое пространство $(\underline M, \underline d)$,
определим на $M$ функцию $d$ по формуле
\eqref{_semimetric_Equation_}. Это будет полуметрика
(проверьте). Из Утверждения
\ref{_metrika_iz_polumetriki_Utverzhdenie_}
следует, что любая полуметрика получается таким образом.

В геометрии линейных пространств,
похожим образом определяются полунормы.


\определение
Пусть $V$ --- векторное пространство над $\R$, а 
$\nu:\; V \arrow \R^{\geq 0}$ функция со значениями
в неотрицательных числах. Функция $\nu$ называется {\бф полунормой} на
$V$, если имеет место следующее 
\begin{description}
\item[Неравенство треугольника:] $\nu (v+v') \leq \nu(v) + \nu(v')$.
\item[Инвариантность относительно гомотетии:] \ \ 
$\nu(\lambda v) = |\lambda| \nu(v)$,
\end{description}
для любых $v$, $v'\in V$, и  любого $\lambda\in \R$.
\ео

Векторное пространство с полунормой наделено полуметрикой,
по формуле $d(x,y) = \nu(x-y)$. Множество векторов,
удовлетворяющих $\nu(x)=0$, называется {\бф
нуль-пространством} полунормы (nullspace).
Применив конструкцию, описанную в Утверждении
\ref{_metrika_iz_polumetriki_Utverzhdenie_},
мы получим, что отображение $V\arrow \underline V$
это отображение $V$ в его факторпространство по
нуль-пространству, а $\underline V$ --- нормированное
векторное пространство.


\определение
Пусть $M$ --- множество, наделенное семейством
полуметрик $\{d_\alpha\}$. {\бф Открытым шаром в полуметрике $d_\alpha$}
называется множество
\[ 
  B_{r,d_\alpha}(x) = \{ y\in M \ \ | \ \ d(x,y) < r\}.
\]
{\бф Топология, заданная семейством полуметрик} --- это
топология, построенная по предбазе $B_{r,d_\alpha}(x)$,
для всех $x\in M, d_\alpha \in \{d_\alpha\}, r\in \R^{>0}$.
\ео

\хфилл


%%%%%%%%%%%%%%%%%%%%%%%%%%%%%%%%%%%%%%%%%%%%%%%%%%%%%%%%%%%%

\section{Тихоновская топология}

%%%%%%%%%%%%%%%%%%%%%%%%%%%%%%%%%%%%%%%%%%%%%%%%%%%%%%%%%%%%

Обозначим через $\Map(A,B)$ множество всех отображений из 
множества $A$ в $B$.

Пусть $M$ --- некоторое множество, а ${\goth I}$ --- набор
индексов (не обязательно конечный или счетный), а 
$M^{\goth I}$ --- произведение $M$ на себя ${\goth I}$ раз.
Можно думать про $M^{\goth I}$ как про множество
последовательностей, индексированных элементами из ${\cal
I}$, или как про множество отображений $\Map({\goth I}, M)$.

\hfill


Пусть $M$ топологическое пространство, ${\goth I}$ --- набор
индексов, $\alpha \in {\goth I}$, 
а $\pi_\alpha:\; M^{\goth I} \arrow M$ --- проекция из 
$M^{\goth I}$ на компоненту с индексом $\alpha$.
Если мы отождествим  $M^{\goth I}$  с $\Map({\goth I}, M)$,
то $\pi_\alpha$ переводит отображение $f:\; {\goth I}\arrow M$
в $f(\alpha)$.

Для какого-то набора индексов $\alpha_1, ..., \alpha_n \in {\goth I}$,
обозначим через \[ \pi_{\alpha_1, ..., \alpha_n}:\; M^{\goth I}\arrow M^n\]
проекцию $M^{\goth I}$ на произведение компонент с
индексами $\alpha_1, ..., \alpha_n$.

\хфилл

\утверждение
В этих условиях, пусть ${\cal U}$ --- слабейшая топология на $M^{\goth I}$,
в которой непрерывны все отображения $\pi_\alpha$.
Тогда ${\cal U}$ задается предбазой вида 
\[ \{ \pi_\alpha^{-1}(U)\ \ | \ \ \alpha \in {\goth I}, 
U \subset M,\ \  \text{$U$ открыто}\}.
\]
Кроме того, ${\cal U}$ задается базой вида 
\[
 \{ \pi_{\alpha_1, ..., \alpha_n}^{-1}(U_1\times U_2 \times ... \times U_n)\ \ 
| \ \ \alpha_1, ..., \alpha_n  \in {\goth I}, 
\ \ U_i \subset M,\ \  \text{$U_i$ открыты}\}.
\]
{\бф Доказательство:} 
Множества $\pi_\alpha^{-1}(U)$ 
открыты в силу непрерывности $\pi_\alpha$,
и порождают слабейшую топологию, в которой
все $\pi_\alpha$ непрерывны. Пересечение
таких множеств имеет вид 
\[
\bigcap_i \pi_{\alpha_i}^{-1}(U_i)=
 \pi_{\alpha_1, ..., \alpha_n}^{-1}(U_1\times U_2 \times ... \times U_n),
\]
а конечные пересечения множеств из предбазы образуют базу
(докажите это).
\endproof

\определение
Построенная выше топология на $M^{\goth I}$
называется {\бф тихоновской топологией}, или 
{\бф топологией произведения}.
\ео

\задача
Пусть $M$ --- хаусдорфово топологическое пространство.
Докажите, что $M^{\goth I}$ хаусдорфово.
\ез

\хфилл

\утверждение
Пусть $(M,d)$ --- метрическое пространство. 
Для каждого индекса $\alpha \in {\goth I}$,
определим полуметрику $d_\alpha$ на $M^{\goth I}$
по формуле $d_\alpha(x,y) = d(\pi_\alpha(x), \pi_\alpha(y))$.
Рассмотрим топологию ${\cal U}_1$, определенную на $M$ системой
полунорм $d_\alpha$. Тогда эта топология совпадает с
тихоновской.

\хфилл

\ноиндент
{\бф Доказательство:}
Предбазой для топологии ${\cal U}_1$
являются множество открытых шаров вида $B_{r,d_\alpha}(x)$.
Очевидно, \[ B_{r,d_\alpha}(x)=\pi_\alpha^{-1}(U), \]
где $U= B_r(\pi_\alpha(x))$ --- это открытый шар в 
$M$ (докажите это). Поэтому предбаза для
топологии ${\cal U}_1$ является предбазой для
тихоновской топологии, и эти две топологии совпадают.
\endproof

\определение
Пусть $(M, \{d_\alpha\})$ пространство с семейством
полуметрик, а $\{x_i\}$ --- последовательность точек в $M$.
Мы говорим, что $\{x_i\}$ --- {\бф последовательность Коши
относительно этого семейства полуметрик}, если 
для каждого индекса $\alpha$, и для каждого $\epsilon >0$,
почти все элементы $\{x_i\}$ лежат в $\epsilon$-шаре 
$B_{\epsilon,d_\alpha}(x)$. Мы говорим, что $(M, \{d_\alpha\})$
{\бф полно}, если каждая последовательность Коши 
имеет предел в топологии, заданной полуметриками.
\ео

\хфилл

\теорема\label{_prod_polno_Teorema_}
Пусть $M$ --- полное метрическое
пространство, а ${\goth I}$ --- некоторый набор индексов.
Рассмотрим $M^{\goth I}$ с тихоновской топологией, и
полуметриками, заданными выше. Тогда $M^{\goth I}$ 
полно.

\хфилл

{\бф Доказательство:} Пусть $\{x_i\}\in M^{\goth I}$ --- 
последовательность Коши. Отождествляя $M^{\goth I}$ 
с $\Map({\goth I}, M)$, как выше, мы можем рассматривать
$x_i$ как отображения ${\goth I}\stackrel{x_i}\arrow M$.
По определению, $\{x_i\}$ является последовательностью
Коши тогда и только тогда, когда для каждого индекса
$\alpha \in {\goth I}$, образы $\{x_i(\alpha)\}$
задают последовательность Коши в $M$.
Поскольку $M$ полно, последовательность Коши 
$\{x_i(\alpha)\}$ сходится к элементу $x(\alpha)\in M$.
Это задает отображение ${\goth I}\stackrel{x}\arrow M$,
которое и будет пределом $\{x_i\}$.
\endproof

\hfill

Из этого доказательства ясно,
что последовательность \[ \{x_i\}\in \Map({\goth I}, M)\]
сходится к ${\goth I}\stackrel{x}\arrow M$ в 
тихоновской топологии тогда и только тогда, когда последовательность
$\{x_i(\alpha)\}$ сходится для любого индекса $\alpha$.
Поэтому тихоновскую топологию называют еще
{\бф топологией поточечной сходимости},
и {\бф топологией почленной сходимости}.

Когда $M=\R$, а ${\goth I}=\N$ (множество натуральных чисел), 
$M^{\goth I}$ --- это множество последовательностей
вещественных чисел, с топологией почленной сходимости.
Эту топологию часто называют {\бф слабой топологией}.

Слабую топологию открыл венгерский математик Фридьеш Рисс.

\begin{figure}[ht]
\begin{center}\ \\
\epsfig{file=Frigyes_Riesz.eps,width=0.5\linewidth}\\
{Frigyes Riesz\\
(1880 --- 1956)}
\end{center}
\end{figure}

Довольно рано стало ясно, что ее невозможно задать 
никакой нормой. Обнаружив это, Рисс придумал
в 1909-м году определение топологического пространства
(независимое от нормы и метрики), основанное
на понятии замыкания. Таким образом появилось
первое определение топологического пространства.
Впоследствии идеи Рисса развил Хаусдорф, получив
современное определение топологического пространства.


%%%%%%%%%%%%%%%%%%%%%%%%%%%%%%%%%%%%%%%%%%%%%%%%%%%%%%%%%%%%

\section{Пространства Фреше}

%%%%%%%%%%%%%%%%%%%%%%%%%%%%%%%%%%%%%%%%%%%%%%%%%%%%%%%%%%%%

\определение
Пусть $V$ векторное пространство, на котором задана
топология ${\cal U}$. Пространство $(V,  {\cal U})$
называется {\бф топологическим векторным пространством},
если  $x,y \arrow x+y$ непрерывно как отображение
$V \times V \arrow V$, и для любого ненулевого $\lambda\in \R$,
отображение $x \arrow \lambda x$ задает гомеоморфизм
из $V$ в $V$.
\ео

Топология на топологическом векторном пространстве
не обязательно задается нормой (даже если у пространства
есть счетная база). Во многих случаях, топология задается 
не нормой, а системой полунорм. Есть и более экзотические 
способы задания топологии.

\определение
Пространство $V$ с системой полунорм $\{d_\alpha\}$ называется
{\бф пространством Фреше} ("Fr\'echet space"), если эта система задает на $V$
хаусдорфову топологию, и $V$ полно как пространство с 
семейством полунорм.
\ео

Напомним, что векторное пространство с нормой
называется {\бф банаховым}, если оно полно как
метрическое пространство. Банахово пространство
является (дурацким) примером пространства Фреше.
Другим примером пространства Фреше является
пространство $\R^\N$ последовательностей вещественных
чисел, с топологией почленной сходимости
(оно полно, как следует из Теоремы
\ref{_prod_polno_Teorema_}). 
Можно доказать, что эта топология на 
$\R^\N$ не может быть задана никакой нормой.


%%%%%%%%%%%%%%%%%%%%%%%%%%%%%%%%%%%%%%%%%%%%%%%%

\section{Тихоновский куб и гильбертов куб}

%%%%%%%%%%%%%%%%%%%%%%%%%%%%%%%%%%%%%%%%%%%%%%%%


Рассмотрим отрезок $[0,1]$, и пусть $[0,1]^{\N}$ --
множество последовательностей точек из $[0,1]$.
Для $\{x_i\}, \{y_i\}\in [0,1]^{\N}$, определим
\[ 
  d_h(\{x_i\}, \{y_i\}):= \sqrt{\sum^\infty_{i=1} \frac{|x_i-y_i|^2}{i^2}}
\]
Легко видеть, что $d_h$ задает метрику на $[0,1]^{\N}$ 
(докажите).

\определение
Построенное таким образом метрическое
про\-с\-т\-ранство $([0,1]^{\N}, d_h)$ называется
{\бф гильбертов куб} ("Hilbert cube").
\ео

\определение
Пространство $[0,1]^{\N}$ с топологией произведения
называется {\бф тихоновский куб} ("Tychonoff cube").
\ео

\хфилл

%%%%%%%%%%%%%%%%%%%%%%%%%%%%%%%%%%%%%%%%%%%%%%%%
\теорема\label{_Hilb_Tych_Theorem_}
Тождественное отображение
задает гомеоморфизм между тихоновским
кубом и гильбертовым кубом.

\хфилл

\noindent
{\бф Доказательство:}\\
\noindent
{\бф Шаг 1.} Пусть $I_h$ --- это гильбертов куб, $I_t$ --- тихоновский
куб. Обозначим через $\pi_n:\; [0,1]^{\N} \arrow [0,1]$ проекцию
$\{x_i\}\arrow x_n$. Поскольку отображение \[ \pi_n:\; I_h\arrow [0,1]\]
$n^2$-липшицево (докажите), проекции $\pi_n$  непрерывны на $I_h$.

\хфилл

\noindent
{\бф Шаг 2.} Для любого топологического пространства
$X$, отображение $X \stackrel \phi\arrow I_t$ 
непрерывно тогда и только тогда, когда композиция $\phi\circ \pi_i$
непрерывна для всех $i$
(это следует из определения тихоновского куба;
см. также доказательство 
Утверждения \ref{_product_functiorial_Utverzhdenie_}).

Поскольку проекции $\pi_n:\; I_h\arrow [0,1]$ 
непрерывны (Шаг 1), тождественное отображение 
$I_h \arrow I_t$
непрерывно.

\хфилл

\noindent
{\бф Шаг 3.} \\
{\bf Лемма:}
Пусть $(M, \{d_\alpha\})$ --- пространство с топологией,
заданной системой полуметрик $\{d_\alpha\}$,
а $d_\alpha(x, \cdot):\; M \arrow \R$
функция, которая переводит $y$ в $d_\alpha(x, y)$.
Тогда функция $d_\alpha(x, \cdot)$ непрерывна.

\хфилл

\noindent
{\бф Доказательство:} Открытый шар
$B_{r,d_\alpha}(x)$ открыт по определению.
Замкнутый шар 
\[
\bar B_{r,d_\alpha}(x)= \{ y \in M \ \ | \ \ d_\alpha(x, y) \leq r\}
\]
замкнут, поскольку каждая точка $y$, которая ему не принадлежит,
принадлежит открытому шару $B_{\epsilon,d_\alpha}(x)$,
для любого $\epsilon < d(y,x) -r$. 
Поэтому прообраз любого открытого отрезка
\[
  d_\alpha(x, \cdot)^{-1}(]\alpha, \beta[)
\]
открыт (это дополнение открытого шара до замкнутого).


\hfill

\noindent
{\бф Шаг 4.} 
Пусть ${\goth v}=\{x_i\}\in [0,1]^{\N}$ --- любая последовательность.
Рассмотрим функцию
$\mu_{\goth v}:\; I_t \arrow \R$,
\[
\mu_{\goth v}(\{y_i\}):= \sum_{k=1}^\infty \frac {d_k(\{x_i\}, \{y_i\})}{k^2}
\]
где $d_k(\{x_i\}, \{y_i\})= |x_k- y_k|$.
Поскольку $d_k$ это $k$-я полуметрика, используемая
для определения тихоновской топологии, 
 функция \[ \{y_i\}\arrow d_k(\{x_i\}, \{y_i\})\]
непрерывна в силу леммы, доказанной в Шаге 3.
Поэтому функция $\mu_{\goth v}$ тоже непрерывна.

\хфилл

\noindent
{\бф Шаг 5.}\\ 
\ноиндент
{\бф Лемма:}
Пусть $f:\; X \arrow M$ --- отображение из топологического
пространства в метрическое пространство $(M, d)$.
Для любой точки $z\in M$, рассмотрим функцию $d_z:\; M
\arrow \R^{\geq 0}$,
$d_z(x):= d(z,x)$. Тогда $f$ непрерывна тогда и только
тогда, когда композиции $f\circ d_z:\; Z \arrow \R$ непрерывны
для всех $z$.

\хфилл

\ноиндент
{\бф Доказательство:}
Функции $d_z$ непрерывны, ибо они 1-липшицевы.
Поэтому если $f$ непрерывна, то композиции $f\circ d_z$ тоже
непрерывны. С другой стороны, 
\[ (f\circ d_z)^{-1}([0, \alpha[) = f^{-1}(B_\alpha(z)),\]
поэтому из непрерывности $f\circ d_z$ следует
открытость $f^{-1}(B_\alpha(z))$, что влечет
непрерывность $f$.



\хфилл

\noindent
{\бф Шаг 6.} Рассмотрим тождественное отображение
$I_t\stackrel\iota \arrow I_h$. Непрерывность $\iota^{-1}$
доказана на шаге 1, поэтому чтобы убедиться, что
$\iota$ это гомеоморфизм, достаточно доказать, что
это отображение непрерывно. В силу леммы из Шага 5,
для этого достаточно доказать, что функция
$d_h({\goth v}, \cdot):\; I_t \arrow \R$ непрерывна,
для любого ${\goth v}=\{x_i\}\in [0,1]^\N$.
Эта функция может быть явно записана как 
\[
\mu_{\goth v}(\{y_i\}):= \sum_{k=1}^\infty \frac {d_k(\{x_i\}, \{y_i\})}{k^2}
\]
и на шаге 4 мы доказали, что она непрерывна.
Теорема \ref{_Hilb_Tych_Theorem_} 
доказана. Тихоновский куб гомеоморфен гильбертову!
\endproof



%%%%%%%%%%%%%%%%%%%%%%%%%%%%%%%%%%%%%%%%%%%%%%%%

\section{История, замечания}

%%%%%%%%%%%%%%%%%%%%%%%%%%%%%%%%%%%%%%%%%%%%%%%%

Я называл "тихоновским кубом" пространство
$[0,1]^\N$ --- произведение счетного числа
интервалов. Но никто не мешает нам взять в
качестве множества индексов любое 
множество $A$, например несчетное. 
Пространство $[0,1]^A$ с тихоновской 
топологией тоже называется
тихоновским кубом. Оно компактно.



Для $[0,1]^\N$ этот удивительный факт,
который называется теоремой Тихонова,
можно вывести из гомеоморфизма $[0,1]^\N$
и гильбертова куба. Но для общего $A$
доказательство компактности $[0,1]^A$  
имеет другую природу, и
кажется совершенно неправдоподобным.


\begin{figure}[ht]
\begin{center}
\epsfig{file=Tikhonov.eps,width=0.55\linewidth}\\
{Андрей Николаевич Тихонов\\
(1906 --- 1993)}
\end{center}
\end{figure}


Тихонов доказал свою теорему в 1924-м году (ему
было тогда 18 лет). В том же 1924-м году он доказал
теорему о метризации --- любое хаусдорфово топологическое
пространство со счетной базой, которое нормально
(то есть удовлетворяет аксиоме Хаусдорфа Т4)
метризуемо. Этот результат был немедленно
(в 1925-м году) опубликован в Mathematische Annalen,
тогда же вошел в переиздание учебника Хаусдорфа и стал
широко известен. 
 
Теорема Тихонова не была опубликована вплоть до 1930-го
года, а ее полная версия (любое произведение компактов
компактно), также известная Тихонову, не была опубликована 
им никогда: ее доказал независимо от Тихонова 
чешский математик Эдуард Чех (Eduard \v Cech) в 1937-м году. 

Дело в том, что среди 
старших товарищей Тихонова много лет никто не 
верил, что подобное утверждение может быть вообще 
верно.
Теорема Тихонова (и его работа
1930-го года с публикацией) --- самая цитируемая
и знаменитая теорема общей топологии. Но в
1920-х и начале 1930-х, когда Тихонов занимался
чистой математикой, он был знаменит в основном
работами о метризации.

Причина этого, видимо, лежит в парадоксальности
самого понятия тихоновской топологии на произведении
топологических пространств.
Сейчас свежесть и необычность этого определения
не ощущается, но в 1924-м году оно
было в полной мере революционным. 

%%%%%%%%%%%%%%%%%%%%%%%%%%%%%%%%%%%%%%%%%%%%%%%%%%%%%%%%%%%%%%%%%%%%%%%%

\chapter[Лекция 7: Теорема о метризации]{Лекция 7:\\ Теорема о метризации}

%%%%%%%%%%%%%%%%%%%%%%%%%%%%%%%%%%%%%%%%%%%%%%%%%%%%%%%%%%%%%%%%%%%%%%%%

%%%%%%%%%%%%%%%%%%%%%%%%%%%%%%%%%%%%%%%%%%%%%%%%%%%%%%%%%%%%
\section{Нормальные топологические пространства}
%%%%%%%%%%%%%%%%%%%%%%%%%%%%%%%%%%%%%%%%%%%%%%%%%%%%%%%%%%%%

\определение
Топологическое пространство $M$ называется
{\бф нормальным}, если любые два непересекающихся замкнутых подмножества
$A$ и $B\subset M$ имеют непересекающиеся окрестности.
\ео

\замечание
Напомним, что {\ем аксиома Хаусдорфа} утверждает что любые
две разные точки топологического пространства 
имеют непересекающиеся окрестности.
Если все точки пространства замкнуты, то из нормальности вытекает
хаусдорфовость. Пространство удовлетворяет аксиоме
Т4, если оно нормально и хаусдорфово.
\еза

\определение
Пусть $U,V$ --- два подмножества топологического
пространства, причем замыкание $U$ лежит в $V$. Это 
отношение обозначается так: $U \Subset V$.
\ео

\замечание 
Нормальность топологического пространства
равносильнa такому свойству. Пусть $U\supset A$ --
окрестность замкнутого множества $A$. Тогда 
есть окрестность $V \supset A$ такая, что
$V\Subset U$. Возьмем $B:= M\backslash U$,
и воспользуемся определением нормальности.
Это даст окрестность $U\supset A$, которая
не пересекается с некоторой окрестностью 
$W\supset B$. Дополнение к $W$ замкнуто,
содержит $V$, и содержится в $U$, поэтому
замыкание $V$ тоже содержится в $U$.
\еза

\begin{figure}[ht]
\begin{center}\ \\
\epsfig{file=normalnost.eps,width=0.5\linewidth}\\
{\small \em Непересекающиеся окрестности замкнутых множеств}
\end{center}
\end{figure}

\замечание\label{_exist_interme_okre_Zamechanie_}
Пусть $U_0\Subset U_1$ --- два открытых множества в нормальном
топологическом пространстве. Тогда существует открытое множество $U_{1/2}$,
с $U_0\Subset U_{1/2} \Subset U_1$. Действительно,
возьмем в качестве $A$ замыкание $U_0$, и воспользуемся
предыдущим замечанием.
\еза



\замечание
Пусть $M$ --- метрическое пространство. 
Тогда $M$ нормально и хаусдорфово. Действительно,
пусть $A$, $B$ --- непересекающиеся замкнутые множества.
Возьмем открытые множества
\[
V:= \bigcup_{z\in A} B_{\frac 1 2 d(z, B)}(z), \ \ 
W:= \bigcup_{z\in B} B_{\frac 1 2 d(z, A)}(z).
\]
Они не пересекаются (проверьте это).
\еза



%%%%%%%%%%%%%%%%%%%%%%%%%%%%%%%%%%%%%%%%%%%%%%%%%%%%%%%%%%%%
\section{Функции Урысона}
%%%%%%%%%%%%%%%%%%%%%%%%%%%%%%%%%%%%%%%%%%%%%%%%%%%%%%%%%%%%

\определение 
Пусть $A$, $B$ --- непересекающиеся замкнутые множества
в топологическом пространстве $M$. Непрерывная функция
$\phi:\; M \arrow [0,1]$ называется {\бф функцией Урысона}
(Urysohn function),
если $\phi \restrict A =0$, $\phi\restrict B =1$.
\ео

\хфилл

\замечание 
В метрическом пространстве функцию Урысона можно построить,
воспользовавшись формулой
\[
\phi(x) = \min\left(1, \frac{d(x,A)}{d(x,B)}\right).
\]
\еза

\хфилл

\теорема 
(Лемма Урысона)
Пусть $M$ --- топологическое пространство.
$M$ нормально тогда и только тогда, когда 
для любых двух непересекающихся замкнутых
множеств существует функция Урысона.

\хфилл

Доказательство леммы Урысона довольно просто. 
Обозначим через $U_1$ дополнение к $B$, за $U_0$ обозначим $A$.
Возьмем $U_{1/2}$ такое, что 
$U_0\Subset U_{1/2} \Subset U_1$, потом возьмем
$U_{1/4}$ такое, что $U_0\Subset U_{1/4}\Subset U_{1/2}$
и $U_{3/4}$ такое, что $U_{1/2} \Subset U_{3/4}\Subset U_1$.
Воспользуемся индукцией. Получим, что для каждого
двоично-рационального числа\footnote{Напомним, что
двоично-рациональным числом называется рациональное
число вида $\frac n {2^m}$, где $n,m$ целые.} 
$\lambda\in [0,1]$ выбрано множество $U_\lambda$, открытое при $\lambda>0$,
причем для $\lambda < \mu$, имеем $U_\lambda \Subset U_\mu$.

Рассмотрим функцию $\phi:\; U_1 \arrow [0,1]$
\[
\phi(x):= \inf \{ \lambda \ \ |\ \ x\in U_\lambda\}.
\]
Продолжим эту функцию на $M$, доопределив ее
посредством $\phi\restrict B=1$.
Легко видеть, что $\phi\restrict A=0$.
Чтобы доказать, что $\phi$ --- функция Урысона, надо 
убедиться в ее непрерывности. Для этого достаточно
проверить, что $\phi^{-1}([0,\alpha[)$ открыто, а
$\phi^{-1}([0,\beta])$ замкнуто, для любых
$\alpha, \beta \in [0,1]$.

Имеем 
\[ \phi^{-1}([0,\alpha[) = \bigcup_{\lambda < \alpha} U_\lambda\]
(проверьте это). Это множество очевидно открыто. 
Аналогично,
\[
\phi^{-1}([0,\beta])=\bigcap_{\alpha > \beta} \phi^{-1}([0,\alpha[) =
\bigcap_{\alpha > \beta} \left(\bigcup_{\lambda < \alpha} U_\lambda\right).
\]
Это дает
\[
\phi^{-1}([0,\beta]) =\bigcap_{\lambda > \beta} U_\lambda.
\]
Обозначим через $\overline{U_\lambda}$ замыкание $U_\lambda.$
Поскольку 
\[ U_{\lambda}\subset 
\overline{U_\lambda}\subset U_{\lambda+\frac {\lambda-\beta}2}\]
для каждого $\lambda > \beta$, имеем
\[ 
\bigcap_{\lambda > \beta} U_\lambda \subset 
\bigcap_{\lambda > \beta}\overline{U_\lambda} \subset 
\bigcap_{\lambda > \beta}U_{\lambda+\frac {\lambda-\beta}2}=
\bigcap_{\lambda > \beta} U_\lambda.
\]
Мы получили 
\[
\phi^{-1}([0,\beta])= \bigcap_{\lambda > \beta}\overline{U_\lambda},
\]
а это множество, очевидно, замкнуто. 
Мы доказали лемму Урысона.

\замечание
Если в пространстве верна лемма Урысона, оно нормально
(докажите).
\еза

%%%%%%%%%%%%%%%%%%%%%%%%%%%%%%%%%%%%%%%%%%%%%%%%

\section{"Создатель советской топологии"}

%%%%%%%%%%%%%%%%%%%%%%%%%%%%%%%%%%%%%%%%%%%%%%%%

Основные результаты общей топологии принадлежат
"московской школе топологии" --- Урысону, Александрову и 
Тихонову, которые провели начало и середину 1920-х годов,
решая проблему метризации топологических пространств,
и изучая компактные пространства.

П. С. Александров был учеником Н. Н. Лузина,
специалиста по теории функций действительного
переменного. В конце 1910-х годов, Лузин занимался
общими вопросами теории множеств. 
В 1917-м году Лузин предложил Александрову
доказать общую форму кон\-ти\-ну\-ум-\-ги\-по\-тезы
(о несуществовании множеств мощности, промежуточной 
между $X$ и $2^X$). Сейчас известно, что
ни континуум-гипотеза, ни ее отрицание 
не следует из аксиом теории множеств.
Александров провел немало времени,
записывая доказательство этого утверждения,
как оказалось --- неправильное. После того,
этой катастрофы, Александров бросил
математику, и стал режиссером.

Проведя несколько лет в занятиях театром
и литературной деятельностью, Александров вернулся
в Москву и поступил в аспирантуру. Там он встретился
с П. С. Урысоном, студентом на 2 года его младше.

После возвращения в Москву Александров продолжил занятия
функциями действительного переменного, популярные
в Москве. Урысон принадлежал к другому поколению.
Начав свое образование как физик, он опубликовал
свою первую статью, о рентгеновском свечении,
 в 1915-м году, 17-ти лет, но в скором времени
заинтересовался математикой. Защитив диссертацию
по интегральным уравнениям, Урысон, по совету
Д. Ф. Егорова, занялся топологией, и провел
1921 и 1922 годы, разрабатывая теорию размерности
для общего метрического пространства. 

Урысон пытался объединить абстрактные топологические
конструкции Хаусдорфа с геометрическими идеями,
почерпнутыми у Пуанкаре, и немало преуспел в этом --- 
сам предмет теоремы о метризации состоит в нахождении
геометрической структуры на абстрактном топологическом
пространстве.

В 1921-22 годах Урысон читал в Московском Университете
курс, под названием "Топология континуума", на котором 
приобщал московскую математическую общественность
к топологии, и приобщил; П. С. Александров
называет Урысона "создателем советской топологии".
Тогда же Урысон стал, на краткое время, научным 
руководителем А. Н. Колмогорова (Колмогорову было 
18 лет, но он успел прославиться, найдя контрпример
к одной из задач Н. Н. Лузина). 

К исследованиям Урысона 
вскоре присоединился и сам Александров,
и в 1923-м году Александров и Урысон написали
совместную работу, где дали определение компактного
пространства (в терминологии того времени "бикомпактного";
русские специалисты по общей топологии до сих пор
иногда используют это слово).

1923-й и 1924-й годы Александров и Урысон 
провели в поездках заграницу. Деньги на эти
поездки они зарабатывали, читая в Москве,
Воронеже, Смоленске и других городах публичные
лекции по теории относительности Эйштейна.
Выгодное соотношение курса червонца к 
западным валютам привело к тому, что денег
от 20 лекций хватало для полугодичной поездки
по научным центрам Европы, и длительных
пешеходных экспедиций.

За 2 года, оставшиеся до его смерти в 1924-м году
(Урысон утонул в море у берегов Бретани), Урысон
доказал теорему о метризации нормальных
пространств со счетной базой; для доказательства
этой теоремы он изобрел лемму, названную его именем.

\begin{figure}[ht]
\begin{center}\ \\
\epsfig{file=Urysohn.eps,width=0.7\linewidth}\\
{Павел Самуилович Урысон\\
(1898 --- 1924)}
\end{center}
\end{figure}


Последней работой Урысона была
статья об универсальном метрическом пространстве
(универсальном пространстве Урысона), в которое 
изометрически вкладывается любое
метрическое пространство ограниченной
мощности и диаметра. 

Лемма Урысона считается важнейшим результатом
общей топологии, наряду с теоремой
Тихонова о компактности произведения
компактных пространств.
Условие нормальности, которое требуется
в лемме Урысона, совсем не ограничительно:
как будет видно из следующей лекции,
любое компактное, хаусдорфово пространство
нормально. 

Из леммы Урысона следует чрезвычайно
полезная {\ем теорема Титце о продолжении}
(Tietze extension theorem): если $A\subset M$ --- замкнутое
подмножество нормального топологического пространства,
а $f:\; A \arrow [0,1]$ --- непрерывная функция на $A$,
ее можно продолжить до непрерывной функции
$F:\; M \arrow [0,1]$ такой, что $F\restrict A =f$.

Для доказательства этой теоремы требуется равномерная
сходимость функций: надо записать $f$
как сумму ряда, составленного из функций 
$f_i:\;M \arrow [0, \alpha_i]$, $\sum \alpha_i < 1$,
постоянных на замкнутых подмножествах $A$,
и продолжить каждую из этих функций до функции
из $M$ в $[0, \alpha_i]$ по лемме Урысона.

%%%%%%%%%%%%%%%%%%%%%%%%%%%%%%%%%%%%%%%%%%%%%%%%%%%%%%%%%%%%%%%%%%%%%%%%

\section{Нормальные пространства \\и нуль-множества}

%%%%%%%%%%%%%%%%%%%%%%%%%%%%%%%%%%%%%%%%%%%%%%%%%%%%%%%%%%%%%%%%%%%%%%%%


Пусть $M$ --- топологическое пространство.
Напомним, что $M$ {\бф удовлетворяет аксиоме Т6,}
если оно хаусдорфово, нормально и каждое замкнутое подмножество
в $M$ получается как пересечение счетного числа
своих окрестностей.


\хфилл

\утверждение \label{_T_6_Utverzhdenie_}
Любое хаусдорфово, нормальное 
пространство $M$ со счетной базой 
удовлетворяет Т6.  

\хфилл

\noindent
{\бф Доказательство} \\
{\бф Шаг 1:} Пусть $\{U_i\}$ --- множество всех элементов
из счетной базы топологии $M$ таких, что
замыкание каждого $U_i$ не пересекается с $A$. Выведите
из нормальности, что $\bigcup_i U_i= M\backslash A$.

\хфилл

\ноиндент
{\бф Шаг 2:}
Возьмем, для каждого $U_i$, открытое множество
$V_i:= M \backslash \overline{U_i}$, где $\overline{U_i}$ --
замыкание $U_i$. Пересечение всех $V_i$ содержит $A$,
поскольку $\overline{U_i}$ не пересекаются с $A$.
С другой стороны, 
\[
\bigcap V_i \subset \bigcap_i (M \backslash U_i) = 
M \backslash\bigcup_i U_i = A.
\]
Мы доказали, что в $M$ выполняется аксиома Т6.
\endproof

\hfill

%%%%%%%%%%%%%%%%%%%%%%%%%%%%%%%%%%%%
\теорема\label{_T6_sche_Theorem_}
Пусть $M$ --- нормальное топологическое
пространство, а $A\subset M$ --- замкнутое подмножество.
Тогда следующие свойства равносильны.

\begin{description}
\item[(i)]
Множество $A$ можно получить как пересечение счетного числа
открытых  окрестностей $W_i \supset A$.
\item[(ii)] Существует непрерывная функция 
$f:\; M \arrow [0,1]$ такая, что $A = f^{-1}(0)$.
\end{description}

\noindent
{\bf Доказательство:}
Из утверждения (ii) легко следует (i).
Действительно, $A= \bigcap_i f^{-1}([0, \frac 1 {2^i}[)$,
а все эти множества открыты.
Следствие (i) $\Rightarrow$ (ii) можно получить, немного 
видоизменив аргумент, доказывающий лемму Урысона.
Возьмем в качестве $B$ пустое множество, и пусть
$V_1\supset V_2 \supset V_3 \supset ...$ --- последовательность
открытых множеств, дающая $\bigcap_i V_i =A$.

Будем строить набор открытых множеств
$U_{\frac m {2^n}}$, $U_\lambda\Subset U_\mu$ для всех
$\lambda <\mu$, таким образом, чтобы множество
$U_{\frac 1 {2^n}}$ содержалось в $V_n$.
Это можно осуществить, заменив на каждом шаге
выбранное $U_{\frac 1 {2^n}}$ 
на $U_{\frac 1 {2^n}}\cap V_n$. Такая замена
корректна, ибо все, что требуется от $U_{\frac 1 {2^n}}$
условиями конструкции Урысона
-- это $A \subset U_{\frac 1 {2^n}} \Subset U_{\frac 1 {2^{n-1}}}$,
а при замене $U_{\frac 1 {2^n}}$ 
на $U_{\frac 1 {2^n}}\cap V_n$, это условие сохраняется.

Пусть 
\[
f(x):= \inf \{ \lambda \ \ |\ \ x\in U_\lambda\} 
\]
функция Урысона, построенная по набору $\{U_\lambda\}$,
и принимающая значение 0 на $A$.
Тогда $f^{-1}([0, \frac 1 {2^i}[)\subset V_i$,
а значит 
\[ f^{-1}(0) = 
   \bigcap_i f^{-1}\left (\left[0, \frac 1{2^i}\right[\right)\subset \bigcap_i
   V_i = A.
\]
\endproof

\определение
Пусть $A\subset M$ --- замкнутое подмножество,
такое, что для некоторой непрерывной функции 
$f:\; M \arrow [0,1]$ имеем $A = f^{-1}(0)$.
Тогда $A$ называется {\бф нуль-множеством}.
\ео

Таким образом, аксиома Т6 для топологического
пространства $M$ равносильна тому, что $M$ нормально,
хаусдорфово, а всякое замкнутое подмножество $M$
является нуль-множеством.

%%%%%%%%%%%%%%%%%%%%%%%%%%%%%%%%%%%%%%%%%%%%%%%%

\section{Теорема Урысона о метризации}

%%%%%%%%%%%%%%%%%%%%%%%%%%%%%%%%%%%%%%%%%%%%%%%%

Пусть $M$ --- нормальное, хаусдорфово топологическое пространство.
Тогда для любых двух разных точек $x, y \in M$ найдется
функция Урысона $f_{x,y}:\; M \arrow [0,1]$, принимающая 
0 на $x$, и 1 на $y$.

Возьмем в качестве множества индексов 
множество ${\goth I} = M \times M \backslash \Delta$,
где $\Delta$ это диагональ. Функции $f_{x,y}$
задают отображение
\begin{equation} \label{_embe_via_Ury_Equation_}
\begin{CD}
M @>{\prod_{(x,y)\in {\goth I}} f_{x,y}}>>  [0,1]^{\goth I}
\end{CD}
\end{equation}
в тихоновский куб; по определению топологии произведения,
оно непрерывно. Поскольку $f_{x,y}(x) \neq f_{x,y}(y)$,
отображение $F:= \prod_{(x,y)\in {\goth I}} f_{x,y}$
инъективно.

Это утверждение не очень полезно.
Из инъективности $F$ не следует, что $M$
гомеоморфно образу $F$. Также, тихоновский куб
$[0,1]^{\goth I}$ не метризуем, если
${\goth I}$ несчетно.

Немного видоизменив этот аргумент, можно добиться
и того, и другого.


\хфилл

%%%%%%%%%%%%%%%%%%%%%%%%%%%%%%%%%%%%%%%%%%%%%%%%
\теорема\label{_Urysohn_Metrization_}
Пусть $M$ --- нормальное, хаусдорфово топологическое
пространство со счетной базой. Тогда существует
непрерывное, инъективное вложение 
$M \stackrel \Phi \hookrightarrow [0,1]^{\N}$ в
счетное произведение отрезков.
Более того, $\Phi$ является гомеоморфизмом $M$
на его образ.

\hfill

\замечание
Из теоремы \ref{_Urysohn_Metrization_}
немедленно следует теорема Урысона о метризации:
любое нормальное, хаусдорфово топологическое
пространство $M$ со счетной базой метризуемо.
Действительно, $M$ гомеоморфно подмножеству
тихоновского куба $[0,1]^{\N}$, а тихоновский куб
метризуем, ибо он гомеоморфен гильбертову
кубу (об этом см. предыдущую лекцию).
\еза

\хфилл

\noindent
{\бф Доказательство теоремы \ref{_Urysohn_Metrization_}.}\\
Пусть $\{U_i\}$ --- счетная база топологии $M$, а 
$A_i:= M \backslash U_i$. Поскольку $M$ --- нормальное, 
хаусдорфово пространство со счетной базой, каждое
$A_i$ является нуль-множеством. И в самом деле,
из Утверждения \ref{_T_6_Utverzhdenie_}
следует, что в $M$ выполнено условие Т6.
Из Теоремы \ref{_T6_sche_Theorem_} следует,
что в такой ситуации существует функция
$f_i:\; M \arrow [0,1]$, такая, что $f_i^{-1}(0)=A_i$.

Возьмем следующее отображение из $M$ в тихоновский куб:
\[
M \stackrel{\prod_i f_i}\arrow [0,1]^{\N}.
\]
Поскольку $U_i$ это база хаусдорфовой топологии, 
для любых двух точек $x\neq y$ существует $A_i$, которое содержит 
$x$ и не содержит $y$ (докажите это). Поскольку $f_i^{-1}(0)=A_i$,
соответствующая функция Урысона удовлетворяет
$f_i(x)=0, f_i(y)\neq 0$. Поэтому $\Phi:=\prod_i f_i$ 
инъективно.

Для доказательства Теоремы \ref{_Urysohn_Metrization_},
осталось убедиться, что $\Phi$ --- это гомеоморфизм $M$ на
его образ.  А приори, это может быть и не так. Например,
если $M$ --- пространство с дискретной топологией,
любое отображение $M \arrow [0,1]^{\N}$ непрерывно,
но не любое вложение --- гомеоморфизм.

Чтобы $\Phi$ было гомеоморфизмом на его образ, нужно, чтобы
любое открытое множество в $M$ получалось как прообраз
$\Phi^{-1}(U)$, для какого-то открытого $U\subset [0,1]^{\N}$. 
Для открытых множеств, принадлежащих базе $U_i$, это верно,
ибо $U_i = f_i^{-1}(]0,1])$, а множество $U_i$  будет
прообразом открытого множества вида
\[
[0,1]\times [0,1] \times \cdots \times ]0,1] \times [0,1]\times \cdots \subset [0,1]^{\N}.
\] 
Поскольку каждое открытое множество в $M$ 
получается объединением $U_i$, все открытые множества
в $M$ --- прообразы открытых подмножеств в $[0,1]^{\N}$.
Поэтому $\Phi$ --- гомеоморфизм. Мы доказали теорему Урысона.
\endproof

\хфилл

%%%%%%%%%%%%%%%%%%%%%%%%%%%%%%%%%%%%%%%%%%%%%%%%%%%%%%%%%%%%

\section{Теоремы о метризуемости}

%%%%%%%%%%%%%%%%%%%%%%%%%%%%%%%%%%%%%%%%%%%%%%%%%%%%%%%%%%%%

Теорема Урысона дает полную характеризацию метризуемых 
топологических пространств со счетной базой. Пространство
со счетной базой является метризуемым тогда и только
тогда, когда оно нормально и хаусдорфово.


\hfill

Для пространств, не имеющих счетной базы,
простого критерия метризуемости нет.

Существует много версий метризационной
теоремы, которые не требуют счетной базы.

Самая полезная из них принадлежит Ю. М. Смирнову (1951),
и требует определения паракомпактности. В дальнейшем
это понятие использоваться не будет.

\хфилл

%%%%%%%%%%%%%%%%%%%%%%%%%%%%%%%%%%%%%%%%%%%%%%%%
\определение
Покрытие $\{U_\alpha\}$ топологического пространства $M$ 
называется {\бф локально конечным}, если любая точка
$M$ лежит в конечном числе элементов $\{U_\alpha\}$
Покрытие $\{V_\beta\}$ называется {\бф измельчением}
покрытия $\{U_\alpha\}$, если каждый элемент $\{V_\beta\}$
лежит в каком-то элементе $\{U_\alpha\}$.
Топологическое пространство $M$ называется {\бф
паракомпактным}, если каждое покрытие $M$
имеет локально конечное измельчение.
\ео

\хфилл

Английский математик Артур Харольд Стоун
(Arthur Harold Stone), изобретатель 
флексагона, доказал в 1948-м году,
что все метрические пространства паракомпактны.
Теорема Смирнова о метризации утверждает,
наоборот, что любое паракомпактное топологическое
пространство метризуемо, если оно {\ем локально
метризуемо}, то есть если у каждой точки
есть метризуемая окрестность. Таким образом
удается ответить на вопрос о метризуемости
многообразий (топологических пространств,
которые локально гомеоморфны $\R^n$).


%%%%%%%%%%%%%%%%%%%%%%%%%%%%%%%%%%%%%%%%%%%%%%%%%%%%%%%%%%%%%%%%%%%%%%%%

\chapter{Лекция 8: Компакты}

%%%%%%%%%%%%%%%%%%%%%%%%%%%%%%%%%%%%%%%%%%%%%%%%%%%%%%%%%%%%%%%%%%%%%%%%

%%%%%%%%%%%%%%%%%%%%%%%%%%%%%%%%%%%%%%%%%%%%%%%%%%%%%%%%%%%%

\section{Компакты и слабо секвенциально компактные пространства}

%%%%%%%%%%%%%%%%%%%%%%%%%%%%%%%%%%%%%%%%%%%%%%%%%%%%%%%%%%%%

Следующее определение хорошо всем знакомо.

\определение
Топологическое пространство $M$ называется
{\бф компактным}, если каждое открытое покрытие $M$
имеет конечное подпокрытие.
\ео

Термин "компакт" введен Фреше, а современное понятие
компакта принадлежит П. С. Урысону и П. С. Александрову.

\begin{figure}[ht]
\begin{center}\ \\
\epsfig{file=Alexandrov.eps,width=0.6\linewidth}\\
{Павел Сергеевич Александров\\
(1896 --- 1982)}
\end{center}
\end{figure}

\замечание
Замкнутое подмножество компакта очевидно компак\-тно
(докажите). Обратное, вообще говоря, неверно.
В отличие от ситуации, известной из метрической
топологии, компактные подмножества не обязательно
замкнуты. Замкнутость компакта следует из хаусдорфовости.
\еза

Действительно, пусть $X \subset M$ --- компактное
подмножество, а $z$ --- точка его замыкания, которая
не лежит в $Z$. Воспользовавшись хаусдорфовостью, 
найдем у каждой точки $x\in X$ 
окрестность $U_х \ni x$, замыкание которой
$\overline{U_x}$ не содержит $z$. Выбрав из $\{U_x\}$
конечное подпокрытие, получим покрытие $U_i$, $i =1,2,3, ..., n$
множества $X$, причем для всех $i$,
замыкание $\overline{U_i}$ не содержит $z$.
Воспользуемся равенством
\[\overline{\bigcup_{i=1}^n U_i} = \bigcup_{i=1}^n \overline {U_i}
\]
(докажите его). Получаем, что $\bigcup_{i=1}^n \overline {U_i}$
замкнуто и не содержит $z$. 

\замечание
Мы доказали,
что всякое компактное подмножество в хаусдорфовом пространстве
замкнуто. В нехаусдорфовом пространстве это не всегда
верно (докажите).
\еза

Следующий элементарный факт чрезвычайно важен.

\хфилл

\утверждение
Пусть $f:\; X \arrow Y$ --- непрерывное отображение.
Тогда образ компактного подмножества $X$ компактен. 

\хфилл

\ноиндент
{\бф Доказательство.}
Пусть $Z\subset X$ --- компактное подмножество,
а $f(Z)$ --- его образ. Возьмем открытое покрытие $\{U_\alpha\}$
множества $f(Z)$. Прообразы элементов этого  покрытия
образуют покрытие $\{f(U_\alpha)\}$
компактного множества $Z$; выбрать из него конечное
подпокрытие значит выбрать конечное покрытие из
$\{U_\alpha\}$.\endproof



\определение
Топологическое пространство $M$ называется
{\бф слабо секвенциально компактным}, если каждая последовательность
$\{x_i\}$ в $M$ имеет предельную точку (точку, в любой окрестности
которой содержится бесконечное количество членов
$\{x_i\}$).\footnote{Такие точки еще называются {\бф
точками накопления.}}
\ео

\замечание
Для метрических пространств, слабая секвенциальная компактность
совпадает с обычной, как следует из теоремы Гейне-Бореля. 
\еза

\утверждение
Слабая секвенциальная компактность вытекает из обычной.

\хфилл

\ноиндент {\бф Доказательство:}
Пусть $M$ --- компактное топологическое пространство.
Из компактности легко выводится, что последовательность
вложенных, непустых, замкнутых подмножеств
\[
A_1 \supset A_2 \supset \cdots
\]
всегда имеет непустое пересечение.
Пусть $\{ x_i\}$ --- последовательность точек $M$,
$R_n$ --- множество $\{x_n, x_{n+1}, x_{n+2}, ...\}$,
a $\overline{R_n}$ --- его замыкание.  Тогда пересечение 
$\bigcap_i \overline{R_i}$ непусто. Ясно, что это
пересечение состоит из предельных точек последовательности
$\{ x_i\}$. Значит, $M$ слабо секвенциально компактно.
\endproof

\определение
Топологическое пространство $M$ называется
{\бф счетно компактным}, если
из любого счетного покрытия $M$ можно выбрать
конечное подпокрытие.
\ео

\утверждение
Пусть $M$ --- топологическое пространство, удовлетворяющее
аксиоме Хаусдорфа Т1 (все точки $M$ замкнуты).
Тогда для $M$ слабая секвенциальная компактность
равносильна счетной. 

\хфилл

\ноиндент
{\бф Доказательство:}
Пусть $M$ счетно компактно, а
$\{ x_i\}$ --- последовательность, не имеющая
предельных точек. Положим $U_n:= M\backslash \{x_1,
... x_{n-1}\}$. В силу Т1 и отсутствия у $\{x_i\}$ 
предельных точек, все $U_i$ открыты. 
Тогда $\{U_i\}$ --- счетное покрытие
$M$, не имеющее конечного подпокрытия. Если же,
наоборот, $M$ слабо секвенциально компактно, а $\{U_i\}$ --
счетное покрытие, из которого нельзя выбрать 
конечного подпокрытия, рассмотрим 
$V_n:= \bigcup_{i=1}^n U_i$. Легко видеть,
что $\{V_i\}$ --- тоже покрытие, из которого 
нельзя выбрать конечного подпокрытия,
причем $V_{i-1}\subset V_i$
Выкинув совпадающие $V_i$, можно
считать, что $V_{i-1}\subsetneq V_i$.
Пусть последовательность $\{x_i\}$
выбрана таким образом, что $x_i \in V_i\backslash V_{i-1}$,
а $x$ --- ее предельная точка. Предположим, что $x\in V_N$.
Тогда бесконечное число элементов $\{x_i\}$ лежит в
$V_N$, что невозможно по построению этой последовательности.
\endproof


\хфилл

Любая непрерывная функция $f:\; M \arrow \R$
на счетном компакте принимает максимум и минимум. 
Действительно, пусть $\{x_i\}$ --- последовательность
точек, такая, что 
\[ 
  \lim\limits_{i\arrow \infty} f(x_i) = \sup_{x\in M} f(x),
\]
а $x$ --- предельная точка этой последовательности.
Поскольку $f$ непрерывно, $f(x)=\sup_{x\in M} f(x)$
(докажите).

Мы получили такую цепочку импликаций, верных
для любого топологического пространства:
\[
 \boxed{ \begin{minipage}{0.19\linewidth}
{компактность}\end{minipage}} \ \Rightarrow\ 
\boxed{ \begin{minipage}{0.23\linewidth}
{слабая \\секвенциальная\\ компактность}\end{minipage}} \ \Rightarrow\ 
\boxed{ \begin{minipage}{0.33\linewidth}
{непрерывные функции достигают минимума и максимума}\end{minipage} }
\]

%%%%%%%%%%%%%%%%%%%%%%%%%%%%%%%%%%%%%%%%%%%%%%%%%%%%%%%%%%%%

\section{Компакты и нормальные пространства}

%%%%%%%%%%%%%%%%%%%%%%%%%%%%%%%%%%%%%%%%%%%%%%%%%%%%%%%%%%%%

Следующая простая теорема чрезвычайно полезна, потому что из нее
вытекает существование функций Урысона на компактах.

\хфилл

\теорема
Пусть $M$ --- компактное, хаусдорфово
топологическое пространство. Тогда $M$ нормально.

\хфилл

\noindent
{\бф Доказательство.}\\
{\бф Шаг 1:} Докажем, что в $M$ выполнена аксиома Т3,
то есть для каждого замкнутого множества 
$A\subset M$, и точки $x\notin A$, существуют непересекающиеся
окрестности $A$ и $x$. Это равносильно тому,
что у $A$ есть окрестность, замыкание которой
не содержит $x$.

Для всех $z\in A$, выберем окрестность
$U_z\ni z$, замыкание которой $\overline {U_z}$ не содержит $x$ (такая
окрестность существует в силу хаусдорфовости --- докажите).
Поскольку $A$ компактно, а $U_z$ --- открытое покрытие
$A$, из него можно выбрать конечное подпокрытие
$U_1, ..., U_n$. Замыкание множества $\bigcup U_i$
не содержит $x$, потому что 
\begin{equation}\label{_zamy_kone_union_Equation_}
\overline{\bigcup_{i=1}^n  U_i} = \bigcup_{i=1}^n \overline {U_i}.
\end{equation}

\хфилл

\ноиндент
{\бф Шаг 2:} Тот же самый аргумент позволяет
вывести из T3 и компактности T4. Пусть $A$ и $B$ --- 
два непересекающихся замкнутых подмножества $M$.
Нам нужно найти окрестность $A$ такую, что
ее замыкание не пересекается с $B$. У каждой
точки $z\in A$ есть окрестность $U_z$, замыкание которой
не пересекается с $B$, в силу T3. Выбрав из $\{U_z\}$
конечное подпокрытие, как на предыдущем шаге,
мы получим конечное покрытие $A$ множествами $U_1, ..., U_n$
такими, что $\overline {U_i}\cap B=\emptyset$.
Снова воспользуемся \eqref{_zamy_kone_union_Equation_},
и получим, что $\bigcup U_i$ --- окрестность
$A$, замыкание которой не пересекается с $B$.
\endproof

\хфилл

Для компакта проблема метризации
дополнительно упрощается следующим простым 
и чрезвычайно важным наблюдением.

\хфилл

\утверждение \label{_nepre_vlo_comp_Utverzhdenie_}
Пусть $f:\; X \arrow Y$ --- непрерывное вложение
хаусдорфовых топологических пространств, причем $X$ --- компакт.
Рассмотрим образ $f(X) \subset Y$ как топологическое
пространство, с индуцированной топологией.
Тогда $f$ --- это гомеоморфизм.

\хфилл

\ноиндент
{\бф Доказательство:}
Для доказательства, нам нужно убедиться,
что образ открытого множества открыт в $f(X)$.
Это равносильно тому, что
образ замкнутого множества замкнут.

Как мы только что видели, образ компактного подмножества
всегда компактен. Поскольку $X$ компактно и хаусдорфово,
компактность подмножества равносильна его замкнутости.
Поэтому образ любого замкнутого подмножества $A\subset X$
компактен в $Y$, а следовательно, замкнут.
\endproof

\хфилл

Мы использовали этот факт несколько лекций назад,
при доказательстве того, что непрерывное
взаимно однозначное отображение из отрезка
в отрезок --- это гомеоморфизм.

\замечание
Легко видеть, что из Утверждения \ref{_nepre_vlo_comp_Utverzhdenie_}
и леммы Урысона следует, что всякое компактное, хаусдорфово топологическое
пространство $M$ гомеоморфно подмножеству тихоновского куба $[0,1]^{\goth I}$,
где ${\goth I}= M \times M \backslash \Delta$.
Непрерывное вложение $M \stackrel\Phi\arrow  [0,1]^{\goth I}$ было построено в
\eqref{_embe_via_Ury_Equation_} с использованием леммы Урысона
и нормальности $M$, доказанной для компактов чуть выше.
В силу Утверждения \ref{_nepre_vlo_comp_Utverzhdenie_},
отображение $\Phi$ является гомеоморфизмом $M$ на его образ.
\еза

%%%%%%%%%%%%%%%%%%%%%%%%%%%%%%%%%%%%%%%%%%%%%%%%%%%%%%%%%%%%%%%%%%%%%%%%

\chapter{Лекция 9: Произведение компактов}

%%%%%%%%%%%%%%%%%%%%%%%%%%%%%%%%%%%%%%%%%%%%%%%%%%%%%%%%%%%%%%%%%%%%%%%%


%%%%%%%%%%%%%%%%%%%%%%%%%%%%%%%%%%%%%%%%%%%%%%%%%%%%%%%%%%%%

\section{Открытые, замкнутые, собственные отображения}

%%%%%%%%%%%%%%%%%%%%%%%%%%%%%%%%%%%%%%%%%%%%%%%%%%%%%%%%%%%%

\определение
Пусть $f:\; X \arrow Y$ --- непрерывное отображение
топологических пространств. Отображение $f$ называется
{\бф собственным}, если прообраз любого компакта --- компакт,
 {\бф открытым}, если образ любого открытого множества открыт,
и {\бф замкнутым}, если образ любого замкнутого множества
замкнут.
\ео


\замечание
Основной пример собственного отображения --
непрерывное отображение из компакта $X$ в хаусдорфово
топологическое пространство $Y$. Действительно, прообраз 
компактного (следовательно, замкнутого) подмножества $Y$
замкнут в $X$, то есть компактен.

Кроме того, это отображение замкнуто. Действительно,
 любое замкнутое подмножество $X$ компактно,
образ компакта компактен, а компактное
подмножество хаусдорфова пространства
замкнуто.
\еза


\замечание
Пример открытого отображения.
Для любых топологических пространств $X$ и $Y$,
проекция $X \times Y \stackrel \pi\arrow Y$ открыта. 
Для проверки этого, достаточно убедиться, что
$\pi(W)$ открыт для любого множества из базы
топологии на $X\times Y$.
Выбрав базу из множеств вида $U\times V$,
где $U, V$ открыты в $X, Y$, мы получим
что $\pi(W)=V$ открыто.
\еза


\задача
Приведите пример непрерывного отображения
хаусдорфовых пространств, которое
\итем замкнуто, но не открыто
\итем открыто, но не замкнуто
\ез

\замечание
Компактность топологического пространства $M$ равносильна
следующему свойству (докажите).
Пусть $\{A_\alpha\}$ --- набор
замкнутых подмножеств $M$, такой, что любое конечное
подмножество $A_1, A_2, ... , A_n \subset \{A_\alpha\}$
имеет общую точку. Тогда все $A_i$ имеют общую точку.
Действительно,  $\bigcap_\alpha A_\alpha=\emptyset$
тогда и только тогда, когда $\{ M\backslash A_\alpha\}$ 
-- покрытие $M$.
\еза


Пусть $f:\; X \arrow Y$ --- непрерывное отображение
Напомним, что для любой точки $y\in Y$, прообраз
$f^{-1}(y)$ называется {\бф слоем} $f$.

\хфилл


%%%%%%%%%%%%%%%%%%%%%%%%%%%%%%%%%%%%%%%%%%%%%%%%%%%%%%%%%%%%
\теорема\label{_sobstve_zamknu_Teorema_}
Пусть $f:\; X \arrow Y$ --- замкнутое, непрерывное отображение,
причем все слои $f$ компактны. Тогда $f$ --- собственное.

\хфилл

\ноиндент
{\бф Доказательство:} 
Пусть $K\subset Y$ --- компакт.
Для доказательства \ref{_sobstve_zamknu_Teorema_},
достаточно убедиться, что $f^{-1}(K)$ --- компакт.
Заменив $Y$ на $K$, а $X$ на $f^{-1}(K)$, 
можно считать $Y$ компактом.

\хфилл

\ноиндент
{\бф Шаг 1:} Пусть $\{A_\alpha\}$ --
набор замкнутых подмножеств в $X$ такой, что любое конечное
подмножество $A_1, A_2, ... , A_n \subset \{A_\alpha\}$
имеет общую точку. Добавив к $\{A_\alpha\}$
все конечные пересечения элементов $\{A_\alpha\}$,
получим набор замкнутых подмножеств $X$,
обладающий тем же свойством. Будем считать,
что $\{A_\alpha\}$ содержит все конечные
пересечения своих элементов.

\хфилл

\ноиндент
{\бф Шаг 2:}
Поскольку $Y$ компактно, а все $f(A_\alpha)$
замкнуты, $\{ f(A_\alpha)\}$ имеет 
общую точку $y\in Y$ (докажите). 

\хфилл

\ноиндент
{\бф Шаг 3:} Рассмотрим слой $f^{-1}(y) \subset X$.
Пусть  $A_1, A_2, ... , A_n \subset \{A_\alpha\}$.
Любое конечное пересечение $\bigcap_i A_i$
лежит в наборе $\{A_\alpha\}$, а
$f(A_\alpha)\ni y$ для всех $\alpha$.
Значит, $\bigcap_i A_i$ пересекается с $f^{-1}(y)$.
Мы получили, что любое конечное подмножество
набора $\{A_\alpha\cap f^{-1}(y)\}$
имеет общую точку. Поскольку $f^{-1}(y)$
компактен, из этого следует, что  набор
$\{A_\alpha\cap f^{-1}(y)\}$ имеет
общую точку. Это доказывает Теорему \ref{_sobstve_zamknu_Teorema_}. 
\endproof


%%%%%%%%%%%%%%%%%%%%%%%%%%%%%%%%%%%%%%%%%%%%%%%%%%%%%%%%%%%%

\section{Конечные произведения компактов}

%%%%%%%%%%%%%%%%%%%%%%%%%%%%%%%%%%%%%%%%%%%%%%%%%%%%%%%%%%%%

Теорема Тихонова утверждает, что любое (даже бесконечное)
произведение компактов компактно. Для бесконечных
произведений, ее доказательство требует аксиомы выбора.
Как доказал Джон Келли (John L. Kelley) в 1950-м году,
теорема Тихонова равносильна аксиоме выбора.

Для конечных произведений, аксиома выбора не нужна,
и компактность произведения компактов можно доказать
непосредственно. К сожалению, доказательство
получается чуть более сложным.


Пусть $X$, $Y$ --- компакты, а
$\pi:\; X \times Y \arrow Y$ --- проекция.
Слои $\pi$ гомеоморфны $X$, и поэтому компактны.
Таким образом, компактность произведения $X\times Y$
вытекает из Теоремы \ref{_sobstve_zamknu_Teorema_},
и следующего утверждения.

\хфилл

\утверждение\label{_zamk_proe_Predlozhenie_}
Пусть $X$, $Y$ --- топологические пространства,
причем $X$ компактно, а $\pi:\; X \times Y \arrow Y$ --
проекция. Тогда отображение $\pi$ замкнуто.

\хфилл

\ноиндент
{\бф Доказательство:}
Пусть $Z\subset X\times Y$ --- замкнутое подмножество,
а $y$ --- предельная точка множества $\pi(Z)$.
Точка $y$ лежит в $\pi(Z)$ тогда и только тогда, когда
$\pi^{-1}(y)\cap Z\neq\emptyset$.
Если $\pi(Z)$ незамкнуто, для какой-то предельной
точки имеем  $\pi^{-1}(y)\cap Z=\emptyset$.

В этом случае, у каждой точки $(x,y)\in
\pi^{-1}(y)$ есть окрестность $U_{x,y}$, не пересекающаяся с $Z$.
Выбрав $U_{x,y}$ в базе топологии, можно считать, что
$U_{x,y} = V_{x,y} \times W_{x,y}$, где $V_{x,y}\subset X$ --- окрестность
$x\in X$, а $W_{x,y}\subset Y$ --- окрестность $y\in Y$.

Множество $\{U_{x,y}\}$ задает покрытие $\pi^{-1}(y)$.
Выберем у него конечное подпокрытие $\{V_i \times W_i\}$.
Тогда $\{V_i\}$ составляет покрытие $X$. Поэтому имеем
\[ X \times \bigcap_i W_i\subset \bigcup_i V_i \times W_i
\] 
(проверьте). Следовательно, множество $X \times \bigcap_i W_i$ 
не пересекает $Z$. Поэтому $y$ имеет окрестность
$\bigcap_i W_i$, не пересекающую $\pi(Z)$,
а значит, не является предельной точкой.
Мы получили, что $\pi(Z)$ замкнуто. \endproof

\begin{figure}[ht]
\begin{center}\ \\
\epsfig{file=zamknu.eps,width=0.8\linewidth}\\
{\small \em Проекция с компактным слоем замкнута}
\end{center}
\end{figure}

\hfill

\замечание 
Если $X$ некомпактно, Утверждение \ref{_zamk_proe_Predlozhenie_}
неверно. Действительно, рассмотрим проекцию $\R \times \R
\arrow \R$, и гиперболу, то есть замкнутое подмножество 
$Z\subset \R \times \R$, состоящее из пар $\{ (x,y) \ |\ xy=1\}$.
Легко видеть, что $\pi(Z)$ незамкнуто (докажите).
\еза


\hfill

Сравнивая Утверждение \ref{_zamk_proe_Predlozhenie_}
и Теорему \ref{_sobstve_zamknu_Teorema_}, мы получаем,
что произведение компактных пространств компактно.

%%%%%%%%%%%%%%%%%%%%%%%%%%%%%%%%%%%%%%%%%%%%%%%%

\section{Максимальные идеалы в кольцах}

%%%%%%%%%%%%%%%%%%%%%%%%%%%%%%%%%%%%%%%%%%%%%%%%

Все кольца в этой лекции предполагаются коммутативными
(с коммутативным умножением) и с единицей (элементом
$1\in R$ таким, что $1\cdot x =x $ для каждого $x$).

Напомним, что {\бф идеалом} в кольце $R$ называется
подмножество $I\subset R$, которое является подгруппой
по сложению, и к тому же удовлетворяет следующему.
Для каждого $x\in R, \gamma\in I,$ произведение 
$x\gamma$ также лежит в $I$. Это свойство
записывается так: $RI \subset I$.

Напомним, что {\бф гомоморфизмом колец}
называется отображение $R_1\stackrel \phi \arrow R_2$, которое
переводит 1 в 1, 0 в 0, и согласовано со сложением и умножением
(то есть удовлетворяет $\phi(x+y) = \phi(x) + \phi(y)$,
$\phi(xy) = \phi(x) \phi(y)$).

{\бф Ядром гомоморфизма} $R_1\stackrel \phi \arrow R_2$
называется множество всех элементов $R_1$, переходящих в 0.

Легко видеть, что ядро гомоморфизма --- идеал (проверьте).
Для каждого идеала $I \subset R$, факторгруппа $R/I$
наделяется естественной структурой кольца (проверьте).
Таким образом, идеалы в кольце --- подмножества, которые
могут быть ядром гомоморфизма $R \arrow R_1$.


\определение
Пусть $R$ --- кольцо, а $S\subset R$ --- набор элементов
$R$. Рассмотрим множество $I\subset R$, состоящее
из всех линейных комбинаций вида 
\[
\sum_{i=1}^n \lambda_i s_i,
\]
где $s_i \in S$, а $\lambda_i \in R$.
Легко видеть, что $I$ это идеал.
Этот идеал называеся {\бф идеалом,
порожденным элементами $S$.}
\ео

\замечание \label{_bez_idea_pole_Zamechanie_}
Пусть $r\in R$ --- любой элемент, а $r R$ --- порожденное
им подмножество $R$. Легко видеть, что $rR\neq R$ 
тогда и только тогда, когда когда $r$ не 
обратим\footnote{Напомним, что $r\in R$ называется
{\бф обратимым в кольце $R$}, если $rr_1 =1$, для
какого-то $r_1 \in R$.} в $R$ (проверьте это).

Поэтому любое кольцо $R$, в котором любой идеал
равен $R$ либо 0, является полем (докажите).
\еза

\определение
Пусть $R$ --- кольцо, а $I\subsetneqq R$ --- идеал.
Идеал $I$ называется {\бф максимальным}, 
если не существует идеала $I_1$ с 
$I \subsetneqq I_1 \subsetneqq R$.
\ео

\утверждение
Идеал $I\subset R$ максимален тогда и только тогда,
когда факторкольцо $R/I$ --- поле.

\хфилл

\ноиндент
{\бф Доказательство:} Легко видеть, что идеалы 
$I_1 \supset I$ находятся во вза\-им\-но-\-од\-нозначном
соответствии с идеалами кольца $R/I$ (проверьте это). 
Отсутствие идеалов  $I \subsetneqq I_1 \subsetneqq R$
равносильно тому, что в  $R/I$ любой
идеал равен 0 либо $R$. В силу 
Замечания \ref{_bez_idea_pole_Zamechanie_},
это равносильно тому, что $R/I$ --- поле.
\endproof


\определение
Напомним, что идеал $I \subset R$ кольца $R$ называется 
{\бф простым}, если для любых $x, y\in R$, из
$xy\in I$ следует, что $x\in I$ либо $y\in I$.
\ео

\задача
Докажите, что любой максимальный идеал --- простой.
\ез


Напомним, что набор подмножеств $S_1 \subset 2^M$
называется  {\бф монотонным}, или {\бф вложенным}, если для любых
$p, q \in S_1$, либо $p\subset q$,
либо $q \subset p$. {\бф Максимальным
элементом} набора подмножеств $S \subset 2^M$
называется такой  элемент $s\in S$, что для любого 
$r\in S$, из $r \supset s$ следует 
$r=s$.

Напомним, что лемма Цорна (Zorn's Lemma) --- следующее
утверждение теории множеств, равносильное аксиоме выбора.
Пусть $S=\{S_\alpha\}\subset 2^M$ --- набор подмножеств множества $M$,
которые удовлетворяют такому свойству: для любого
монотонного поднабора $\{S_\beta\}\subset S$, объединение
$\bigcup_\beta S_\beta$ тоже лежит в $S$. Тогда в $S$
есть максимальный элемент.

Из леммы Цорна немедленно следует существование
максимальных идеалов.

\hfill

\теорема
Пусть $I \subsetneqq R$ --- идеал в кольце.
Тогда существует максимальный идеал $I_1 \supset I$.

\хфилл

\ноиндент
{\бф Доказательство:} Пусть $S \subset 2^R$ --- множество
идеалов, содержащих $I$, и не равных $R$. Легко видеть,
что для вложенного набора идеалов $\{I_\alpha\}\subset S$,
объединение $\bigcup_\alpha I_\alpha$ --- идеал,
не равный $R$. Действительно, объединение
любого набора вложенных идеалов --- снова идеал 
(проверьте это). Объединение $\bigcup_\alpha I_\alpha$
равно $R$, если $1\in R$ лежит в каком-то из
$I_\alpha$, но это невозможно, потому что
$I_\alpha \subsetneqq R$. Применив
лемму Цорна к $S \subset 2^R$,
получим, что в $S$ существует
максимальный элемент. Он и будет
максимальным идеалом, содержащим $I$.
\endproof


\begin{figure}[ht]
\begin{center}\ \\
\epsfig{file=Zorn.eps,width=0.53\linewidth}\\
{Max August Zorn\\
(1906 --- 1993)}
\end{center}
\end{figure}



%%%%%%%%%%%%%%%%%%%%%%%%%%%%%%%%%%%%%%%%%%%%%%%%

\section{Лемма Цорна: история, замечания}

%%%%%%%%%%%%%%%%%%%%%%%%%%%%%%%%%%%%%%%%%%%%%%%%

Лемма Цорна была доказана Максом Цорном в 1935-м году.
Цорн был алгебраистом, учеником Эмиля Артина (Emil
Artin). Абстрактная
алгебра в 1930-е годы развивалась весьма бурно 
(трудами Эмми Нетер, Эмиля Артина и Бартеля
Ван дер Вардена, среди прочих), но для строгих 
доказательств приходилось постоянно прибегать к теории
множеств. К тому времени эта наука
была немало дискредитирована парадоксами.
Примерно тогда же, Гёдель доказал, что невозможно
доказать ее непротиворечивость, и теория
множеств оказалась неожиданно шатким
фундаментом для математики. 

Для доказательства существования
максимальных идеалов до Цорна использовалась
теорема Цермело о существовании
полного порядка на любом множестве 
(в англоязычной литературе
эта теорема известна как "well-ordering principle").
К теореме Цермело математики традиционно
относятся с большим недоверием.
("The Axiom of Choice is obviously true, the well-ordering
principle obviously false, and who can tell about Zorn's
lemma?" --- шутка, приписываемая Джерри Бонэ).

Цорн предложил строить абстрактную алгебру
аксиоматически, не прибегая к сложным
конструкциям вроде теоремы Цермело, 
и предложил лемму Цорна
(которую он называл "принципом максимума")
в качестве одной из аксиом. Основным
(практически единственным) применением
этой леммы в алгебре является теорема
о существовании максимальных идеалов.

Впрочем, нетрудно доказать, что 
теорема о существовании максимальных идеалов
равносильна аксиоме выбора.

Название "лемма Цорна" впервые
использовано американским математиком
Джоном Тьюки (John Tukey).

Помимо алгебры, Цорн занимался теорией чисел и
функциональным анализом. Среди прочего, ему
принадлежит аксиоматическое построение
алгебры октав ("octonions", "Cayley numbers"), 
и доказательство того, что квадрат любого 
бесконечного множества равномощен этому множеству.


%%%%%%%%%%%%%%%%%%%%%%%%%%%%%%%%%%%%%%%%%%%%%%%%

\section{Кольцо подмножеств и ультрафильтры}

%%%%%%%%%%%%%%%%%%%%%%%%%%%%%%%%%%%%%%%%%%%%%%%%


\определение
Пусть $A, B\subset M$ --- подмножества $M$.
Определим {\бф симметрическую разность}
$A \triangle B$ формулой
\[
A \triangle B := (A \cup B) \backslash (A \cap B).
\]
\ео

\определение
Пусть $S \subset 2^M$ --- набор подмножеств $M$.
Мы говорим, что $S$ {\бф замкнут относительно
конечных пересечений и симметрических разностей},
если пересечение и симметрическая разность любых
элементов $S$ снова лежит в $S$. Если $S$ 
замкнуто относительно конечных пересечений и
симметрических разностей, и к тому же 
содержит $M$, мы говорим, что $S$ --
{\бф кольцо подмножеств} $M$.
\ео

\замечание
Легко видеть, что $2^M$ является кольцом.
\еза


\определение
Пусть $\nu \subset M$ --- подмножество
$M$. Рассмотрим функцию $\chi_\nu:\; M \arrow \{0,1\}$,
\[
\chi_nu(x) = \begin{cases} 1, & \ \ \text{ если $x\in \nu$,}\\
0, & \ \ \text{ если $x\notin \nu$.}
\end{cases}
\]
Эта функция называется {\бф характеристической функцией}
подмножества $\nu\subset M$.  Отождествив $\{0,1\}$
с полем ${\Bbb F}_2$ остатков по модулю 2, можно
считать, что $\chi_\nu$ --- функция со значениями в ${\Bbb F}_2$.

Эта конструкция отождествляет $2^M$ с множеством
 функций \[ M \arrow \{0,1\}.\] В дальнейшем, мы будем
отождествлять подмножества и соответствующие им
характеристические функции.
\ео


Пусть $S \subset 2^M$ --- набор подмножеств $M$.
Для каждого $a\in S$, рассмотрим его характеристическую
функцию $\chi_а:\; M \arrow {\Bbb F}_2$.
Такие функции можно складывать и умножать
почленно. 

\хфилл

\утверждение
Mножество функций \[ R_S:= \{ \chi_а:\; M \arrow {\Bbb F}_2\ | \  a\in S \}\]
образует кольцо относительно почленного
сложения и умножения тогда и только тогда, когда 
$S$ --- кольцо подмножеств $M$.

\хфилл

\ноиндент 
{\бф Доказательство:}
Для любых $\nu, \rho \subset M$, имеем
\[
\chi_\nu + \chi_\rho = \chi_{\nu\triangle \rho}, \ \ 
\chi_\nu \cdot \chi_\rho = \chi_{\nu\cap \rho},
\]
поэтому замкнутость $S$ относительно 
конечных пересечений и симметрических разностей
равносильна замкнутости $R_S$ относительно
сложения и умножения. Наличие в этом множестве
нуля очевидно, потому что $X \triangle X = \emptyset$,
а $\chi_\emptyset$ равно нулю. Наличие в
этом множестве единицы следует из того,
что $M \in S$, а $\chi_M =1$.
\endproof

\хфилл

Мы построили арифметические операции (сложение,
умножение) на любом кольце подмножеств $S \subset 2^M$.

\хфилл

\замечание
Теорема о существовании максимальных идеалов,
будучи примененной к кольцу подмножеств, дает сюръективный
гомоморфизм колец $R_S \arrow k$, где $k$ --- некоторое
поле. Легко видеть, что все элементы $R_S$ удовлетворяют
$a^2=a$ (такие элементы называются {\бф идемпотентами}.
Поэтому все элементы $k$ --- тоже идемпотенты.
Согласно теореме Безу, многочлен $P(x)$ степени $i$ 
имеет не больше $i$ корней в поле $k$ (докажите это). Поскольку все элементы $k$
являются корнями квадратного многочлена $x^2-x=0$,
$k$ --- поле из двух элементов.
\еза

\hfill

\замечание\label{_dop_ultrafilter_Zamechanie_}
Пусть $I \subset R\subset 2^M$ --- максимальный идеал
в кольце подмножеств, $\phi:\; R \arrow {\Bbb F}_2$ --
проекция $R \arrow R/I$, а $A \in R$ --- какой-то элемент.
Поскольку $\chi_M=1$, $A$ либо $M\backslash A$ принадлежит $I$.
Действительно, если $A$ не принадлежит $I$, то
$\phi(A)=1$, а значит \[ \phi(M\backslash A) = 1-\phi(A)=0.\]
\еза

\определение
Пусть $M$ --- множество, $2^M$ --- кольцо всех подмножеств $M$,
а $I$ --- максимальный идеал в $2^M$. {\бф Ультрафильтром} на $M$
называется множество всех $X\subset M$, не лежащих в $I$.
\ео





\определение
Пусть $x\in M$ --- точка. Рассмотрим гомоморфизм 
$2^M \arrow {\Bbb F}_2$, ставящий функции $\chi:\; M \arrow {\Bbb F}_2$
ее значение $\chi(x)$. Ядро этого отображения, очевидно,
максимальный идеал. Дополнение к такому идеалу называется
{\бф главным ультрафильтром}. 
Главный ультрафильтр состоит из множества всех
$X\subset M$, содержащих $x$:
\[
\bigcap_{B\notin I} B = \{x\}, \ \ \bigcap_{A\in I} A = М
\backslash \{x\}
\]
(последнее равенство следует из того, что 
$A \in I \Leftrightarrow (M \backslash A) \notin I$,
что ясно из Замечания \ref{_dop_ultrafilter_Zamechanie_}).
\ео

\замечание
Многие математики считают, что понятие ультрафильтра
парадоксально, и использовать ультрафильтры не следует,
наравне с теоремой Цермело и другими экзотическими
следствиями аксиомы выбора.
Действительно, ультрафильтры, кроме главных,
невозможно построить явно. Если у вас понятие 
ультрафильтра вызывает отторжение, пропустите
конец этого раздела, и забудьте про ультрафильтры.
\еза


Ультрафильтры можно определить аксиоматически,
что видно из следующей задачи.

\задача
Пусть $M$ --- множество, а ${\cal U} \subset 2^M$ --- набор
его подмножеств. Докажите, что следующие утверждения
равносильны.
\begin{description}
 \item[(i)] ${\cal U}$ это ультрафильтр.
\item[(ii)] Выполнены следующие свойства.
\begin{enumerate}
\renewcommand{\labelenumi}{{\bf \Alph{enumi}.}}
\item если $A\subset B$, $A\in {\cal U}$, то $B \in {\cal U}$.
\item Для любого $A \subset M$, либо $A$, либо
$M\backslash A$ лежат в ${\cal U}$ (но не одновременно).
\item Если $A, B \in {\cal U}$, то $A \cap B\in {\cal U}$.
\end{enumerate}
\end{description}
\ез

Ультрафильтры были введены в 1937-м году
Анри Картаном, одним из основателей Бурбаки,
и широко использовались в трактатах Николя Бурбаки.

\begin{figure}[ht]
\begin{center}\ \\
\epsfig{file=Henri_Cartan.eps,width=0.50\linewidth}\\
{\small Henri Cartan, 1996 \\
(1904 --- 2008)}
\end{center}
\end{figure}



\замечание\label{_ideal_podmno_Zamechanie_}
Пусть $S\subset 2^M$ --- набор подмножеств.
Рассмотрим идеал в $2^M$, порожденный $S$.
Он равен $2^M$ тогда и только тогда, когда 
$M$ можно получить как объединение конечного 
поднабора в $S$ (докажите).
\еза


Пусть теперь $S=\{X_\alpha\}\subset 2^M$ --- набор подмножеств
$M$ таких, что $\bigcup_\alpha X_\alpha = M$, но никакое конечное
подмножество $S$ не дает в объединении $M$. Так, к
примеру, если $M= \Z$, можно взять в качестве $S$
множество всех конечных подмножеств $\Z$.



Идеал, порожденный $S$, не равен $M$ в силу 
Замечания \ref{_ideal_podmno_Zamechanie_}.
Поэтому он содержится в некотором максимальном
идеале $I$. Поскольку 
\[ 
 \bigcup_{A\in I} A \subset \bigcup_\alpha X_\alpha = M,
\]
этот ультрафильтр не главный.



\определение
{\бф Аддитивной мерой} на кольце множеств 
$S\subset 2^M$ называется отображение
$\mu:\; S \arrow \R^{\geq 0}$ такое, что 
$\mu (A \cup B) = \mu(A) + \mu(B)$ для любых
непересекающихся множеств $A, B \in S$.
\ео

\задача
Пусть ${\cal U} \subset 2^M$ --- некоторое подмножество. 
Рассмотрим отображение $\mu:\; 2^M \arrow \{0,1\}$,
переводящее $A$ в 1, если $A\in {\cal U}$, и в 0,
если $A\notin {\cal U}$. Докажите, что 
$\mu$ является аддитивной мерой тогда и только
тогда, когда ${\cal U}$ --- ультрафильтр.
\ез


Ультрафильтры придумал Анри Картан, в 1937-м году,
следуя идеям Клода Шевалле, хотя Шевалле впоследствии отказался
от всех прав на это изобретение. Учебники Бурбаки по основам 
топологии (в целом, весьма неудачные) используют ультрафильтры
и многие другие экзотические конструкции, после них 
практически не употреблявшиеся. В отличие от других
изобретений Бурбаки, которые вообще никому не
понадобились, понятие ультрафильтра оказалось
полезно в логике, общей топологии и некоторых 
разделах алгебры.



%%%%%%%%%%%%%%%%%%%%%%%%%%%%%%%%%%%%%%%%%%%%%%%%%%%%%%%%%%%%

\section{Теорема Александера о предбазе}

%%%%%%%%%%%%%%%%%%%%%%%%%%%%%%%%%%%%%%%%%%%%%%%%%%%%%%%%%%%%

\утверждение
Пусть $M$ --- топологическое пространство,
а ${\cal V}\subset 2^M$ --- покрытие $M$.
Рассмотрим идеал $I$ в $2^M$,
порожденный ${\cal V}$. Тогда
следующие утверждения равносильны.
\begin{description}
\item[(i)] Из ${\cal V}$ можно выбрать конечное
подпокрытие 

\item[(ii)] $I=2^M$.
\end{description}

\noindent
{\bf Доказательство:} Если $U_1, ..., U_n$ --- конечное подпокрытие,
то объединение $\bigcup U_i=M$ выражается
через пересечения и симметрические разности,
а значит, принадлежит $I$. Мы получаем $1\in I$,
что влечет $I=2^M$.

Если же $I=2^M$, имеем
\[
1 = \sum_{i=1}^n \lambda_i U_i,
\]
где $\lambda_i\in 2^M$, a $U_i \in {\cal V}$.
На языке множеств это равенство переписывается
\[
M = (\lambda_1 \cap U_1) \triangle (\lambda_2 \cap U_2) \triangle ... \triangle (\lambda_n \cap U_n)
\]
Поскольку $A \triangle B\subset A \cup B$, имеем
\[
M = (\lambda_1 \cap U_1) \triangle (\lambda_2 \cap U_2) \triangle ... \triangle (\lambda_n \cap U_n) \subset \bigcup_{i=1}^n U_i,
\]
значит, в ${\cal V}$ найдется конечное подпокрытие.
\endproof

\хфилл

\теорема
(теорема Александера о предбазе, ``Alexander subbase theorem'')
Пусть $M$ --- топологическое пространство с предбазой
$\{U_\alpha\}$. Предположим, что любое покрытие $M$
элементами $\{U_\alpha\}$ имеет конечное подпокрытие.
Тогда $M$ компактно.

\хфилл

\ноиндент
{\бф Доказательство:} 
Пусть $M$ некомпактно,
и пусть ${\cal P}$ --- покрытие $M$, не допускающее конечного
подпокрытия. Рассмотрим идеал $I$ в $2^M$, порожденный
${\cal P}$. Абзацем выше доказано, что $I\neq 2^M$.
Пусть $I_m\subset 2^M$ --- максимальный идеал, содержащий $I$.

\хфилл

\ноиндент 
{\бф Шаг 1.} 
Обозначим через ${\cal U}$ множество элементов 
предбазы $\{U_\alpha\}$, содержащихся в $I_m$.
Докажем, что ${\cal U}$ --- это покрытие $M$.
Поскольку $I_m$ это идеал, вместе с любым
 множеством $I_m$ содержит все
его подмножества. По условию,
$I_m$ содержит открытое покрытие $M$.
Поэтому для  для каждой точки $x\in M$, и каждой базы
топологии на $M$ найдется элемент базы,
который содержит $x$ и содержится в  $I_m$.
Взяв в качестве базы конечные
пересечения $U_\alpha$, мы получим
\[
x \in \bigcap_i U_i \in I_m,
\]
где $U_i$ лежат в предбазе.
Получаем $\bigcap_i U_i\in I_m$.
Поскольку $I_m$ максимальный идеал, 
$I_m$ --- простой идеал. Значит,
из $\bigcap_i U_i\in I_m$  следует,
что хотя бы один из $U_i$ лежит в $I_m$.
Мы получили, что $x\in U_i \in {\cal U}$.
Значит, ${\cal U}$ --- покрытие.


\хфилл

\ноиндент 
{\бф Шаг 2.}
На предыдущем шаге,
мы получили, что ${\cal U}$ --- 
покрытие $M$ элементами предбазы.
Поскольку $I_m\neq 2^M$, а ${\cal U}\subset I_m$.
никакой конечный набор элементов
${\cal U}$ не дает в объединении $M$.
Мы пришли к противоречию с условиями теоремы
Александера о предбазе. Следовательно, $M$ компактно.
\endproof



\begin{figure}[ht]
\begin{center}\ \\
\epsfig{file=Alexander.eps,width=0.50\linewidth}\\
{James Waddell Alexander\\
(1888 --- 1971)}
\end{center}
\end{figure}

\замечание
Отметим, что теорема Александера о предбазе влечет
теорему Тихонова о компактности произведения
(об этом ниже). Поэтому она равносильна аксиоме выбора.
\еза


Американский математик Джеймс Александер прославился
в основном как один из основателей современной алгебраической
топологии. Ему принадлежит изобретение симплициальных пространств
и когомологий. Кроме того, Александер изучал
теорию узлов, и немало ее развил. Он 
определил инвариант узлов, который сейчас
называется его именем (инвариант Александера).

Александер происходил из очень влиятельной
американской семьи и был миллионером. Несмотря 
на это, он был чрезвычайно левых взглядов. Когда 
в 1950-х годах в Америке начались гонения на социалистов,
Александер стал одной из громких жертв преследований; в 1951-м году
Александеру пришлось уйти из Принстонского Университета
и Institute of Advanced Studies, где он работал.
Александер дожил до 1971-го года фактическим
отшельником, не появляясь на людях. Единственное
публичное выступление Александера случилось в 1954-м году --- 
он подписал открытое письмо в защиту физика Роберта
Оппенгеймера, директора Манхэттенского проекта,
которого в 1954-м году выгнали с работы за 
политику.

Также Александер был знаменитым альпинистом.

%%%%%%%%%%%%%%%%%%%%%%%%%%%%%%%%%%%%%%%%%%%%%%%%

\section{Теорема Тихонова о компактности}

%%%%%%%%%%%%%%%%%%%%%%%%%%%%%%%%%%%%%%%%%%%%%%%%

Пусть $\{M_\alpha\}$ --- набор множеств.
Рассмотрим объединение всех этих множеств
(которые считаются непересекающимися).
Оно обозначается \\ $\bigsqcup M_\alpha$.


\определение
Пусть $\{M_\alpha\}$ --- набор топологических пространств, 
проиндексированный множеством индексов ${\goth I}$. 
Напомним, что {\бф произведение $M_\alpha$} 
это множество отображений из ${\goth I}$ в $\bigsqcup
M_\alpha$, ставящих в соответствие каждому индексу
$\alpha \in {\goth I}$ точку пространства $M_\alpha$.
На $\prod_\alpha M_\alpha$ вводится {\бф тихоновская топология},
заданная следующей предбазой. Для каждой пары
$\alpha \in {\goth I}$, и открытого множества
$U \subset M_\alpha$, рассмотрим
подмножество 
\[
M_{\alpha_1} \times ... \times U_\alpha \times ... \subset 
M_{\alpha_1} \times ... \times M_\alpha \times ...
\]
(произведение набора $\{M_\alpha\}$, где
элемент $M_\alpha$ заменили на $U_\alpha$,
а все остальные оставили как есть).
Обозначим это подмножество за ${\cal F}_{\alpha, U}$.
Топология, заданная такой предбазой,
называется {\бф тихоновской топологией},
или {\бф топологией произведения}.
\ео


\begin{figure}[ht]
\begin{center}\ \\
\epsfig{file=pokry-subbase.eps,width=0.65\linewidth}\\
{\small \em Элементы предбазы, покрывающие часть
произведения $M_1\times M_2$}
\end{center}
\end{figure}

\теорема 
(теорема Тихонова о компактности)
Пусть $\{M_\alpha\}$ --- набор компактных 
топологических пространств. Тогда $\prod_\alpha M_\alpha$
тоже компактно.


\ноиндент
{\бф Доказательство:}
Теорема Тихонова сразу следует из теоремы Александера о
предбазе. Пусть $\{ {\cal F}_{\alpha, U_{\alpha, \xi}}\}$ --
набор элементов предбазы. Легко видеть, что
\[
\bigcup_{\alpha, \xi} {\cal F}_{\alpha, U_{\alpha, \xi}} =
\bigcup_\alpha  {\cal F}_{\alpha, W_\alpha},
\]
где $W_\alpha= \bigcup_\xi U_{\alpha, \xi}$.
Если для всех $\alpha$, $W_\alpha \neq M_\alpha$,
выберем точку $x_\alpha \in M_\alpha\backslash W_\alpha$.
Очевидно, точка $\prod_\alpha x_\alpha$ не лежит в 
$\bigcup_\alpha  {\cal F}_{\alpha, W_\alpha}$.




Мы получаем, что $\{ {\cal F}_{\alpha, U_{\alpha, \xi}}\}$
является покрытием, если и только если $\bigcup_\xi U_{\alpha, \xi}= M_\alpha$,
для какого-то $\alpha$. Это значит, что $U_{\alpha, \xi}$
является открытым покрытием $M_\alpha$. Поскольку
$M_\alpha$ компактно, в $U_{\alpha, \xi}$ есть
конечное подпокрытие $U_{\alpha, i}$.
Тогда ${\cal F}_{\alpha, U_{\alpha, i}}$
будет конечным покрытием $\prod_\alpha M_\alpha$.
Мы доказали, что из любого покрытия $\prod_\alpha M_\alpha$
элементами предбазы можно выбрать конечное подпокрытие.
Теперь из теоремы Александера вытекает, что $\prod_\alpha M_\alpha$
компактно. \endproof

%%%%%%%%%%%%%%%%%%%%%%%%%%%%%%%%%%%%%%%%%%%%%%%%%%%%%%%%%%%%%%%%%%%%%%%%

\chapter[Лекция 10: Равномерная  сходимость]{Лекция 10:\\ Равномерная  сходимость}

%%%%%%%%%%%%%%%%%%%%%%%%%%%%%%%%%%%%%%%%%%%%%%%%%%%%%%%%%%%%%%%%%%%%%%%%


%%%%%%%%%%%%%%%%%%%%%%%%%%%%%%%%%%%%%%%%%%%%%%%%%%%%%%%%%%%%

\section{Банаховы пространства}

%%%%%%%%%%%%%%%%%%%%%%%%%%%%%%%%%%%%%%%%%%%%%%%%%%%%%%%%%%%%

\определение
Пусть $(V,\nu)$ --- пространство с нормой.
Напомним, что $(V, \nu)$ называется {\бф банаховым},
если оно полно, как метрическое пространство.
\ео

\замечание
Предположим, что $(V, \nu_2)$ --- конечномерное
пространство с евклидовой нормой, заданной
как $\nu_2(v) = g(v,v)$, где $g$ --- билинейная
симметричная, невырожденная форма.
 В Лекции 2 мы доказали, что тождественный
изоморфизм $(V, \nu)\stackrel {\Id}\arrow (V, \nu_2)$
является липшицевым, и обратное ему отображение
тоже липшицево (такие отображения называются
{\бф билипшицевыми}). Из этого следует, что 
$(V, \nu)\stackrel {\Id}\arrow (V, \nu_2)$
переводит последовательности Коши в последовательности
Коши. Поэтому $(V,\nu)$ полно.
\еза

Мы доказали следующее утверждение

\хфилл

\утверждение
Любое конечномерное
нормированное пространство\footnote{{\бф Нормированное
пространство} --- пространство с нормой.}  --- банахово.

\хфилл

Отметим также, что единичный шар в $(V, \nu)$
компактен. Действительно, он является замкнутым
и ограниченным подмножеством в евклидовом пространстве $(V, \nu_2)$,
что следует из билипшицевости тождественного
отображения $(V, \nu)\stackrel {\Id}\arrow (V, \nu_2)$.

\хфилл


Следующая полезная теорема была доказана Ф. Риссом.

\хфилл

Пусть $(V,\nu)$ --- нормированное пространство,
а $\bar B_1\subset V$ --- единичный замкнутый шар в $V$.
Если $\bar B_1$ компактен, то $V$ конечномерно.

\хфилл

\ноиндент
{\бф Доказательство.} {\bf Шаг 1:}
Если $\bar B_1$ компактен, из
покрытия $\bar B_1$ открытыми шарами радиуса $\frac 1 2$ можно 
выбрать конечное подпокрытие. Пусть $\{x_1, ..., x_n\}$
центры шаров, которые составляют это подпокрытие.
Тогда для каждого $v\in V$, $|v|\leq 1$, для какого-то
$x_i$ имеет место $|v-x_i|< \frac 1 2$.

\хфилл

\ноиндент
{\bf Шаг 2:}
Следовательно, для каждого $w\in V$, такого, что 
$|w|\leq \lambda$, и для какого-то $x_i\in \{x_1, ..., x_n\}$ 
имеет место $|w-\lambda x_i|< \frac 1 2\lambda$.


\хфилл

\ноиндент
{\bf Шаг 3:} Возьмем какой-то $v\in V$, $|v|\leq 1$.
Выберем $x_{i_1}\in \{x_1, ..., x_n\}$ такой, что
$|v-x_{i_1}|< \frac 1 2$. Применив утверждение
предыдущего шага к $w=v-x_{i_1}$, $\lambda=1/2$,
получим, что
\[ |v-x_{i_1}-\frac 1 2 x_{i_2}| < \frac 1 4,
\]
для какого-то $x_{i_2}\in \{x_1, ..., x_n\}$.
Применим утверждение предыдущего шага 
к $w =v- \sum_{k=1}^{n-1} \frac 1 {2^{k-1}} x_{i_k}$, $\lambda=1/2^{n-1}$,
воспользуемся индукцией, получим
\[
\bigg | v- \sum_{k=1}^n \frac 1 {2^{k-1}} x_{i_k}\bigg| <
\frac 1 {2^n}.
\]

\хфилл

\ноиндент
{\bf Шаг 4:} Мы доказали, что 
\[ v = \sum_{k=1}^\infty \frac 1 {2^{k-1}} x_{i_k}. 
\]
Следовательно, $v$ принадлежит линейной оболочке
векторов $\{x_1, ..., x_n\}$, которая таким образом
должна содержать $V$ целиком. Поэтому $V$ конечномерно.
\endproof

\хфилл

Пусть $M$ --- топологическое пространство.
Напомним, что функцию $f:\; M \arrow \R$ называют
{\бф ограниченной}, если $\sup_{x\in M} |f(x)| < \infty$.
Пусть $C_b(M)$ --- пространство непрерывных, ограниченных,
вещественнозначных функций на $M$. Пусть
\[ |f|:= \sup_{x\in M} |f(x)|.\]
Легко видеть, что это норма (докажите). 
Такая норма называется {\бф $L^\infty$-нормой,}
или же {\бф $\sup$-нормой} на пространстве
непрерывных, ограниченных,
вещественнозначных функций. 

\определение
Топология, определяемая $\sup$-нормой
на пространстве непрерывных, ограниченных функций
$C_b(M)$, называется {\бф топологией равномерной
сходимости}. Последовательность функций,
которая сходится в такой топологии,
называется {\бф равномерно сходящейся.}
\ео




%%%%%%%%%%%%%%%%%%%%%%%%%%%%%%%%%%%%%%%%%%%%%%%%
\теорема\label{_sup_norma_Banach_Theorem_}
Предел равномерно сходящейся последовательности
непрерывных функций непрерывен. Более того,
пространство $C_b(M)$ с $\sup$-нормой --- банахово.

\хфилл

\ноиндент
{\бф Доказательство.} {\бф Шаг 1.}
Пусть $\{f_i\}$ --- последовательность Коши 
ограниченных функций. Поскольку 
\[ |f_i(y)-f_j(y)| \leq \sup_{x\in M} |f_i(x)-f_j(x)|,\]
для каждой точки $y\in M$, $\{f_i(y)\}$ ---
последовательность
Коши. Поэтому $f_i$ поточечно сходятся к функции $f:\; M \arrow \R$.

\хфилл

\ноиндент
{\бф Шаг 2.} Пусть 
$\phi:\; M \arrow \R$ --- непрерывная функция, а 
$\Gamma_{\phi, \epsilon}\subset M \times \R$ это объединение всех
$\epsilon$-отрезков вида $m\times [\phi(m)-\epsilon, \phi(m)-\epsilon]$.
Можно думать про $\Gamma_{\phi, \epsilon}$ как про объединение графиков
функций $\phi+c$, где $c\in[-\epsilon, \epsilon]$.
Докажем, что множество $\Gamma_{\phi, \epsilon}$
замкнуто. Для этого рассмотрим отображение
$\Psi=\phi\times \Id:\; M \times \R\arrow \R \times \R$.
Легко видеть, что $\Gamma_{\phi, \epsilon} = \Psi^{-1}(W)$,
где $W$ --- замкнутое подмножество в $\R\times \R$,
составленное из всех пар $(x, y)$, таких, что $|x-y| \leq \epsilon$.
Поскольку $W$ замкнуто, а $\Psi$ непрерывна, $\Gamma_{\phi, \epsilon}$
также замкнуто.

\begin{figure}[ht]
\begin{center}\ \\
\epsfig{file=grafik.eps,width=0.9\linewidth}\\
{\small \em Множество $\Gamma_{f, \epsilon}$ получено как $\Psi^{-1}(W)$}
\end{center}
\end{figure}


\хфилл

\ноиндент
{\бф Шаг 3.} Для каждого $i$, обозначим через $\epsilon_i$
число $|f-f_i|$. График $\Gamma_f$ лежит в 
замкнутом множестве $\Gamma_{f_i, \epsilon_i}$.
Поскольку последовательность $\{\epsilon_i\}$
сходится к нулю, имеем
\[
\Gamma_f = \bigcap_i \Gamma_{f_i, \epsilon_i}.
\]
График функции $\Gamma_f$ является пересечением
замкнутых множеств, поэтому он замкнут.

\хфилл

\ноиндент
{\бф Шаг 4.} 
Поскольку функция $f$ ограниченна,
можно рассматривать ее как функцию
$f:\; M \arrow [-C, C]$.
Пусть $\pi:\; M\times [-C, C] \arrow M$ --
обычная проекция. Легко видеть, что 
$f^{-1}([a,b])= \pi(\Gamma_f\cap M\times [a,b])$.
Поскольку проекция $\pi:\; M\times [-C, C] \arrow M$
имеет компактные слои, она замкнута (это было
доказано на прошлой лекции). Поэтому  
$f^{-1}([a,b])= \pi(\Gamma_f\cap M\times [a,b])$
замкнуто, а значит, прообраз любого открытого
интервала в $[-C, C]$ открыт. Поскольку открытые
интервалы являются базой топологии, $f$
непрерывна. Мы доказали, что $\{f_i\}$ 
равномерно сходится
к непрерывной функции. \endproof

%%%%%%%%%%%%%%%%%%%%%%%%%%%%%%%%%%%%%%%%%%%%%%%%

\section{Примеры пространств Фреше}

%%%%%%%%%%%%%%%%%%%%%%%%%%%%%%%%%%%%%%%%%%%%%%%%

Пусть $V$ --- топологическое векторное пространство,
с хаусдорфовой топологией, заданной системой
полунорм $\{\nu_\alpha\}$. Напомним, что
$V$ называется {\бф пространством Фреше},
если каждая последовательность $\{x_i\}$,
которая является последовательностью Коши
относительно всех полунорм $\nu_\alpha$,
сходится к $x\in V$.

Пусть $M$ --- локально компактное топологическое
пространство, а $V$ --- пространство непрерывных
функций на $M$. Для каждого компактного подмножества
$K \subset M$, рассмотрим полунорму на $V$,
\[
|f|_K:=\sup_{x\in K} |f(x)|.
\]
Поскольку $K$ компактно, а $f$ непрерывно,
имеем $\sup_{x\in K} |f(x)|< \infty$.

Эта система полунорм задает на $V$ топологию,
которая называется
{\бф топология равномерной сходимости на компактах}. 
Легко видеть, что $V$ является пространством Фреше
(докажите).

Другой пример пространства Фреше получается,
если рассмотреть пространство $C^\infty([0,1])$
гладких функций на отрезке. Рассмотрим, для каждого $n$,
норму $|f|_{C^n}$, определенную следующим образом:
\[ |f|_{C^0}= \sup_{x\in [0,1]} |f(x)|, \ \ 
|f|_{C^1}= \sup_{x\in [0,1]} |f(x)|+ |f'(x)|, ...,
\] 
\[
|f|_{C^n}:= \sup_{x\in [0,1]} \sum_{i=0}^n |f^{(i)}(x)| .
\]
Легко видеть, что
\[
|\phi|_{C^n} \geq |\phi^{(k)}|_{C^{n-k}}
\]
для любой $n$-кратно дифференцируемой функции $\phi$.
(проверьте это). В частности, $\{f_i^{(k)}\}$ --- последовательность
Коши в $C^{n-k}$-топологии, если $\{f_i\}$ --- последовательность
Коши в $C^n$.

Поскольку предел по $C^0$-норме
непрерывен, как было доказано выше, предел по
$C^k$-норме --- $k$ раз дифференцируемая функция.
Действительно, $f_i$ будет $k$-кратной первообразной
для $f_i^{(k)}$, значит, $f$ будет $k$-кратной первообразной
для $f^{(k)}:= \lim f_i^{(k)}$ (чтобы получить
это, докажите, что взятие первообразной
перестановочно с взятием предела по равномерной
сходимости).

\замечание
Мы получили, что пространство $C^k([0,1])$ $k$ раз дифференцируемых
функций на $[0,1]$  полно относительно нормы
$|\cdot|_{C^k}$. Действительно, пусть $\{f_i\}$ --
последовательность Коши относительно этой
нормы. Поскольку $|\cdot|_{C^k} \geq |\cdot|_{C^0}$,
из Теоремы \ref{_sup_norma_Banach_Theorem_} следует, что
любая последовательность Коши в норме $|\cdot|_{C^k}$
сходится к непрерывной функции. В силу вышесказанного,
эта функция $k$ раз дифференцируема.
\еза



Рассмотрим пространство $C^\infty([0,1])$
с системой норм, заданных $C^i$, $i=0,1, 2, ...$.
В силу вышесказанного, если последовательность 
$\{f_i\}$ сходится во всех этих нормах, предел
этой последовательности гладкий. 
Поэтому $C^\infty([0,1])$ --- пространство Фреше.


%%%%%%%%%%%%%%%%%%%%%%%%%%%%%%%%%%%%%%%%%%%%%%%%

\section{$\sup$-метрика на пространстве отображений}

%%%%%%%%%%%%%%%%%%%%%%%%%%%%%%%%%%%%%%%%%%%%%%%%

\определение
Пусть $X$ --- топологическое пространство, $Y$ --
метрическое пространство. Отображение $f:\; X \arrow Y$
называется {\бф ограниченным}, если $f(X)$ лежит
в шаре $B_r(y)$ для какого-то $y\in Y$,
и $r\in \R$.
\ео

На множестве $\Map_b(X,Y)$ ограниченных отображений
из $X$ в $Y$ определена {\бф $\sup$-метрика} формулой
$d(f_1, f_2):= \sup_{x\in X} d(f_1(x), f_2(x))$
(проверьте, что это метрика). Эта же метрика
определяется на $C_b(M)$ посредством $\sup$-нормы
(проверьте). Тот же самый аргумент, что доказывает
Теорему \ref{_sup_norma_Banach_Theorem_} о банаховости
$C_b(M)$, доказывает полноту пространства $C_b(X,Y)$
непрерывных, ограниченных отображений. Для доказательства,
надо убедиться, что последовательность Коши отображений
$\{f_i\}\in C_b(X,Y)$ сходится поточечно к отображению
$f\in\Map_b(X,Y)$, а затем воспользоваться 
замкнутостью графика $f$, чтоб доказать его 
непрерывность. Для этого используется такая
лемма.

\хфилл


%%%%%%%%%%%%%%%%%%%%%%%%%%%%%%%%%%%%%%%%%%%%%%%%
\лемма\label{_zamknu_gra_Lemma_}
Пусть $f:\; X \arrow Y$ --- отображение топологических
пространств, причем график $\Gamma_f\subset X\times Y$
замкнут. Предположим, что  $Y$ компактно.
Тогда $f$ --- непрерывно.


\хфилл

\ноиндент
{\бф Доказательство:} Пусть $\pi_X:\; X \times Y\arrow X$ --- 
отображение проекции. Для каждого замкнутого подмножества
$A\subset Y$, 
\[ f^{-1}(A) = \pi_X(\Gamma_f\cap X \times A).
\]
Если $Y$ компактно, то отображение $\pi_X$ замкнуто 
(Лекция 9), поэтому множество \[ \pi_X(\Gamma_f\cap X \times A)\] тоже замкнуто,
а значит, $f$ непрерывно (докажите). \endproof



\хфилл


%%%%%%%%%%%%%%%%%%%%%%%%%%%%%%%%%%%%%%%%%%%%%%%%
\теорема\label{_oto_polno_Teorema_}
Пусть $X$ --- топологическое пространство, $Y$ --- полное
метрическое пространство, а $C_b(X,Y)$ --- пространство
непрерывных, ограниченных отображений, с $\sup$-метрикой. 
Предположим, что любой замкнутый шар в $Y$ компактен.
Тогда $C_b(X,Y)$ полно.


\ноиндент
{\бф Доказательство Теоремы \ref{_oto_polno_Teorema_}.} \\{\бф Шаг 1:} 
Поскольку $d(f_1(x), f_2(x))\leq d(f_1, f_2)$, любая
последовательность Коши $\{f_i\}$ отображений 
поточечно сходится к ограниченному отображению
$f \in \Map_b(X,Y)$. Для доказательства полноты
$C_b(X,Y)$ осталось проверить, что $f$ непрерывно.

\hfill

\noindent
{\бф Шаг 2:} 
Предположим, что любой замкнутый шар в $Y$ компактен.
Тогда для любой последовательность Коши  
\[ \{f_i\} \in \Map_b(X,Y),\] $\{f_i\}$ целиком лежит
в каком-то замкнутом шаре. Поэтому можно считать
$Y$ компактным, а значит, выполнено условие Леммы
\ref{_zamknu_gra_Lemma_}. Поэтому
для доказательства непрерывности $f$ достаточно
убедиться в том, что его график замкнут.


\hfill

\noindent
{\бф Шаг 3:} Пусть $\phi:\; X\arrow Y$ --- непрерывное
отображение, где $X$ --- топологическое пространство,
а $Y$ --- метрическое пространство.
Обозначим через $\Gamma_{\phi, \epsilon}\subset X \times Y $ множество
\[
\Gamma_{\phi, \epsilon}=
\{ (x,y) \in X\times Y\ \ |\ \ d(f(x), y) \leq \epsilon\}.
\]
Докажем, что $\Gamma_{\phi, \epsilon}$ замкнуто. В самом
деле, пусть $\Psi:\; X \times Y \arrow Y\times Y$ 
отображает $(x,y)$ в $(\phi(x),y)$, а $A_\epsilon \subset
Y \times Y$ --- множество всех пар $(y_1, y_2)$ таких, 
что $d(y_1,y_2) \leq\epsilon$.
Множество $A_\epsilon$, очевидно, замкнуто (проверьте), а 
$\Gamma_{\phi, \epsilon}= \Phi^{-1}(A_\epsilon)$,
значит, оно тоже замкнуто.


\hfill

\noindent
{\бф Шаг 4:}
Пусть $\epsilon_i = d(f, f_i)$. Тогда
\[ 
  \Gamma_f= \bigcap_i \Gamma_{f_i, \epsilon_i}
\]
(проверьте). Значит, $\Gamma_f$ --- пересечение замкнутых
множеств и оно замкнуто.
В силу Шага 2, из этого следует непрерывность $f$.
Мы доказали Теорему \ref{_oto_polno_Teorema_}.
\endproof

%%%%%%%%%%%%%%%%%%%%%%%%%%%%%%%%%%%%%%%%%%%%%%%%%%%%%%%%%%%%

\section{История, замечания}

%%%%%%%%%%%%%%%%%%%%%%%%%%%%%%%%%%%%%%%%%%%%%%%%%%%%%%%%%%%%

В 1821-м году Огюстен Коши опубликовал неправильное доказательство
того, что поточечный предел непрерывных функций непрерывен.
В скором времени, Абель и Фурье нашли контрпримеры к этому
утверждению, a Дирихле обнаружил ошибку в доказательстве
Коши.

Определение равномерной сходимости принадлежит,
судя по всему, Кристофу Гудерманну (Christoph Gudermann). 
В 1841-м году ученик Гудерманна Карл Вейерштрасс
опубликовал определение равномерной сходимости, и придумал немецкий
термин "gleichm\"a\ss ig konvergent", который переводится
на русский как "равномерная сходимость" (по-английски --
uniform convergence).

\begin{figure}[ht]
\begin{center}\ \\
\epsfig{file=Weierstrass.eps,width=0.85\linewidth}\\
{Karl Theodor Wilhelm Weierstra\ss\\
(1815 --- 1897)}
\end{center}
\end{figure}

%%%%%%%%%%%%%%%%%%%%%%%%%%%%%%%%%%%%%%%%%%%%%%%%%%%%%%%%%%%%%%%%%%%%%%%%

\chapter{Лекция 11:  Пространство непрерывных отображений}

%%%%%%%%%%%%%%%%%%%%%%%%%%%%%%%%%%%%%%%%%%%%%%%%%%%%%%%%%%%%%%%%%%%%%%%%


%%%%%%%%%%%%%%%%%%%%%%%%%%%%%%%%%%%%%%%%%%%%%%%%%%%%%%%%%%%%

\section{Топология равномерной сходимости на $C(X,Y)$}

%%%%%%%%%%%%%%%%%%%%%%%%%%%%%%%%%%%%%%%%%%%%%%%%%%%%%%%%%%%%


Пусть $X$ --- компактное топологическое пространство,
$Y$ --- метрическое пространство, а $C(X,Y)$ --
множество непрерывных отображений из $X$ в $Y$.
Напомним, что на $C(X,Y)$ определена {\бф $\sup$-метрика}
по формуле
\[
d(f, f') = \sup_{x\in X}d(f(x), f'(x)).
\]
Конечность супремума следует из компактности $X$
(докажите).

Напомним, что {\бф окрестность} подмножества
$Z$ топологического пространства --- это открытое
множество, которое содержит $Z$.

Пусть $\Delta_\epsilon$ обозначает 
окрестность диагонали $\Delta$ в $Y\times Y$, заданную формулой
\[
\Delta_\epsilon = \{ (y,y') \in Y \times Y \ | \  d(y, y')
< \epsilon \}.
\]
Для каждого непрерывного отображения 
$\phi:\; X \arrow Y$, рассмотрим 
$\Phi= \phi \times \Id_Y:\; X \times Y \arrow Y \times Y$.
Легко видеть, что график $\Gamma_\phi$ получается как
$\Gamma_\phi = \phi^{-1}(\Delta)$.


\begin{figure}[ht]
\begin{center}\ \\
\epsfig{file=grafik.eps,width=0.9\linewidth}\\
{\small \em Множество $\Gamma_{\phi, \epsilon}$ получено как $\Psi^{-1}(W)$}
\end{center}
\end{figure}
Пусть $\Gamma_{\phi, \epsilon}:=\Phi^{-1}(\Delta_\epsilon)$ --
окрестность $\Gamma_\phi$, полученная как прообраз $\Delta_\epsilon$.

\замечание\label{_Gamma_f_epsilon_neigh_Zamechanie_}
На прошлой лекции мы доказали, что
$d(\phi, \phi') < \epsilon$ тогда и только тогда,
когда график $\Gamma_{\phi'}$ целиком лежит в 
$\Gamma_{\phi, \epsilon}$.
\еза

\определение
Пусть
$X \times C(X,Y) \stackrel{\ev}\arrow Y$,
переводит пару $(x, \phi)$ в $\phi(x)$.
Это отображение называется
{\бф отображение эвалюации} (вычисления).
\ео

Основное утверждение этой лекции --- следующая теорема.

\хфилл

%%%%%%%%%%%%%%%%%%%%%%%%%%%%%%%%%%%%%%%%%%%%%%%%
\теорема\label{_eval_C(X,Y)_Teorema_}
Пусть $X$ --- компактное топологическое пространство,
$Y$ --- метрическое пространство, а $C(X,Y)$ --
множество непрерывных отображений из $X$ в $Y$,
с топологией равномерной сходимости.
Предположим, что у $X$ есть счетная база окрестностей в точке.
Тогда отображение эвалюации 
$X \times C(X,Y) \stackrel{\ev}\arrow Y$
непрерывно. Более того, топология равномерной
сходимости --- самая слабая топология, в
которой $\ev$ непрерывно.

\замечание
Из Теоремы \ref{_eval_C(X,Y)_Teorema_}
сразу следует, что топология равномерной
сходимости на $C(X,Y)$ целиком определяется
топологической структурой $X$ и $Y$.
\еза

\замечание
Напомним, что отображение топологических
прост\-ранств называется {\бф секвенциально непрерывным},
если оно переводит пределы последовательностей
в пределы последовательностей. Секвенциальная
непрерывность $\ev$ немедленно следует (проверьте) 
из неравенства треугольника
\[
d(f_i(x_i), f(x)) \leq d(f_i(x_i), f(x_i)) + d(f(x_i), f(x)),
\]
где $\{f_i\}\in C(X,Y)$ --- последовательность
функций, равномерно сходящихся к $f$, а $\{x_i\}\in X$ --
последовательность точек, сходящихся к $x$.
\еза

\замечание
Для пространств со счетной базой окрестностей в точке, 
секвенциальная непрерывность равносильна обычной
(докажите это). Поэтому $X \times C(X,Y) \stackrel{\ev}\arrow Y$
непрерывно, в предположении, что у $X$ есть счетная
база окрестностей в точке (докажите).
\еза

Мы завершим доказательство Теоремы 
\ref{_eval_C(X,Y)_Teorema_}
в конце следующей секции.

%%%%%%%%%%%%%%%%%%%%%%%%%%%%%%%%%%%%%%%%%%%%%%%%%%%%%%%%%%%%

\section{Tопология, заданная окрестностями графика}

%%%%%%%%%%%%%%%%%%%%%%%%%%%%%%%%%%%%%%%%%%%%%%%%%%%%%%%%%%%%

Пусть $X$, $Y$ --- топологические пространства,
а $C(X,Y)$ --- множество всех непрерывных отображений
из $X$ в $Y$.

Пусть $W$ --- подмножество
в $X\times Y$. Рассмотрим множество ${\goth S}_W$ в $C(X, Y)$,
состоящее из всех непрерывных отображений
$f\in C(X,Y)$ таких, что график $\Gamma_{f}$ лежит в $W$.
Пусть $C'(X,Y)$ --- пространство непрерывных функций,
с топологией, базой которой является множество всех
${\goth S}_W$, где $W$ --- объединение конечного числа открытых подмножеств
$U \times V\subset X\subset Y$ с подмножествами вида $К \times Y$,
где $K \subset X$ замкнут.

Рассмотрим отображение 
$X \times C'(X,Y)\stackrel \Phi\arrow X \times Y$,
переводящее $(x, \phi)$ в $(x, \phi(x))$.
Для каждого открытого множества 
$U\times V \subset X \times Y$,
пусть
\[ W:=U\times V \cup (X \backslash U) \times Y.\]

Из определения ${\goth S}_W$ сразу следует, что
\[
\Phi^{-1}(U\times V) = U \times {\goth S}_W.
\]
Действительно, для каждого $\phi\in C(X,Y)$,
образ $\phi(U)$ лежит в $V$ тогда и только тогда,
когда график $\phi$ лежит в $W$.
Следовательно, отображение
$\Phi:\; X \times C'(X,Y)\arrow X \times Y$
непрерывно. Поскольку отображение эвалюации
получается как композиция $\Phi$ и проекции,
мы получаем, что $X \times C'(X,Y) \stackrel{\ev}\arrow Y$
непрерывно.

По определению,
база открытых множеств в $C'(X,Y)$ порождена
множествами вида  ${\goth S}_W$, где
$W= U \times V \cup (X \backslash U) \times Y$.
Эта база получается из прообразов
$\pi(\Phi^{-1}(A))$, где 
$\pi:\; X \times C'(X,Y) \arrow C'(X,Y)$ --- естественная
проекция, а $A=U \times V $ открыто в $X\times Y$. Поэтому
топология $C'(X,Y)$ есть слабейшая топология, в которой
$\Phi$ непрерывно. 

\замечание
Непрерывность 
$X \times C(X,Y)\stackrel \Phi\arrow X \times Y$ в
какой-то топологии на $C(X,Y)$ равносильна
непрерывности $\ev$ в этой же самой топологии.
Действительно, $\ev$ получается композицией $\Phi$
и проекции, а $\Phi$ получается как произведение
$\ev$ и проекции $X \times C(X,Y)\arrow X$.
\еза

Мы получили, что $C'(X,Y)$ --- слабейшая
топология, в которой непрерывно отображение $\ev$.
Для доказательства Теоремы \ref{_eval_C(X,Y)_Teorema_}
осталось доказать, что топология $C'(X,Y)$
эквивалентна топологии $C(X,Y)$.

Пусть $X$ --- компактное топологическое пространство
со счетной базой окрестностей в точке
а $Y$ --- метрическое пространство.
Для доказательства Теоремы \ref{_eval_C(X,Y)_Teorema_},
рассмотрим тождественное отображение
\[C'(X,Y) \stackrel \Id \arrow C(X,Y).\] База открытых множеств
в $C(X,Y)$ состоит из множеств вида 
${\goth S}_{\Gamma_{\phi, \epsilon}}$
(Замечание \ref{_Gamma_f_epsilon_neigh_Zamechanie_}).
Такие множества открыты в $C'(X,Y)$, по
определению топологии на $C'(X,Y)$.
Поэтому $C'(X,Y) \stackrel \Id \arrow C(X,Y)$
непрерывно, то есть топология, определенная ${\goth S}_W$,
сильнее, чем топология равномерной сходимости.

С другой стороны, поскольку  $C'(X,Y)$ --- слабейшая
топология, в которой непрерывно отображение
эвалюации, а $X \times C(X,Y) \stackrel{\ev}\arrow Y$
непрерывно, топология $C'(X,Y)$ слабее,
чем топология $C(X,Y)$. Мы доказали,
что эти топологии эквивалентны.

\hfill

\утверждение
Пусть $Y$ --- компактное метрическое пространство,
$Z$ --- метрическое пространство, а $X$ компактно.
Тогда отображение композиции
$C(X,Y) \times C(Y,Z) \stackrel A \arrow C(X,Z)$ непрерывно.

\хфилл

\ноиндент
{\бф Доказательство:}
Отображение $C(X,Y) \times C(Y,Z)\stackrel A \arrow
C(X,Z)$ непрерывно тогда и только тогда, когда 
\[ X\times C(X,Y) \times C(Y,Z)\stackrel {\Id_X \times A} \arrow
X \times C(X,Z)
\]
непрерывно. По Теореме \ref{_eval_C(X,Y)_Teorema_},
база топологии на $X \times C(X,Z)$
порождается $\ev^{-1}(U)$, где $U\subset Z$ --- открытое
множество. Поэтому для доказательства непрерывности
композиции, достаточно доказать, что
отображение $\ev^2(X\times C(X,Y) \times C(Y,Z)) \arrow Z$,
$(x, \phi, \psi) \arrow \psi(\phi(x))$ непрерывно.
С другой стороны, $\ev^2$ получается применением
эвалюации два раза: первый раз
$(x, \phi, \psi) \arrow (\phi(x), \psi)$,
а второй раз --- $(\phi(x), \psi) \arrow \psi(\phi(x))$,
и каждая из этих эвалюаций непрерывна, что следует
из Теоремы \ref{_eval_C(X,Y)_Teorema_}. \endproof







%%%%%%%%%%%%%%%%%%%%%%%%%%%%%%%%%%%%%%%%%%%%%%%%

\section{Замечания}

%%%%%%%%%%%%%%%%%%%%%%%%%%%%%%%%%%%%%%%%%%%%%%%%

Топология на $C(X,Y)$, определенная выше, 
является примерном {\бф от\-кры\-то-компактной топологии},
которая определена для любого локально компактного $X$ 
и любого (не обязательно метризуемого) $Y$.
Для компактного подмножества $K \subset X$ и
открытого $U\subset Y$, пусть ${\cal V}(K, U)$ --
множество отображений, переводящих $K$ в $U$.
Открыто-компактная топология (compact-open topology)
на $C(X,Y)$ --- топология, заданная базой ${\cal V}(K, U)$.


\begin{figure}[ht]
\begin{center}
\epsfig{file=Fox.eps,width=0.56\linewidth}\\
{Ralph Hartzler Fox\\
(1913 --- 1973)}
\end{center}
\end{figure}


\задача[*]
Докажите, что открыто-компактная топология
эквивалентна топологии равномерной сходимости,
если $X$ компактно, а $Y$ метризуемо.
\ез

\задача[*]
Докажите, что отображение композиции
$C(X,Y) \times C(Y,Z) \arrow C(X,Z)$ непрерывно,
для открыто-компактной топологии, если $X$, $Y$, $Z$ 
хаусдорфовы.
\ез

Открыто-компактную топологию изобрел в 1945-м году
Ральф Фокс, ученик Соломона Лефшеца, который 
был научным руководителем множества знаменитых математиков,
в частности Джона Милнора и Барри Мазура.


%%%%%%%%%%%%%%%%%%%%%%%%%%%%%%%%%%%%%%%%%%%%%%%%%%%%%%%%%%%%%%%%%%%%%%%%

\chapter[Лекция 12: Связные пространства]{Лекция 12:\\ Связные пространства}

%%%%%%%%%%%%%%%%%%%%%%%%%%%%%%%%%%%%%%%%%%%%%%%%%%%%%%%%%%%%%%%%%%%%%%%%

\section{Свойства связных подмножеств}

%%%%%%%%%%%%%%%%%%%%%%%%%%%%%%%%%%%%%%%%%%%%%%%%%%%%%%%%%%%%%%%%%%%%%%%%

\определение
Пусть $M$ ---  топологическое пространство,
a $A\subset M$ --- его подмножество, которое
открыто и замкнуто. Тогда $A$ называется
{\бф открыто-замкнутым} (clopen).
\ео

\замечание
Очевидно, $M$ и $\emptyset$ открыто-замкнуты.
Если у $M$ есть открыто-замкнутое подмножество
$U$, не равное $M$ и $\emptyset$, то $M$ можно разбить
в объединение двух непересекающихся, непустых, открытых
подмножеств, $U$ и $M \backslash U$. Обратное тоже
верно (докажите).
\еза


\определение
Пусть $M$ --- топологическое пространство,
а $X \subset M$ --- его подмножество, рассмотренное
как топологическое пространство с индуцированной
топологией. Тогда $X$ называется {\бф связным}
(connected), если
верны следующие равносильные условия
\begin{description}
\item[(i)] $X$ не содержит открыто-замкнутых подмножеств,
кроме $X$ и $\emptyset$.
\item[(ii)] $X$ не может быть разбито в объединение
двух непересекающихся, непустых, открытых подмножеств
\end{description}
\ео

\утверждение
Связное подмножество отрезка $[0,1]$ --- это отрезок,
интервал или полуинтервал.\footnote{Точка и пустое
множество считаются частными случаями отрезка и интервала.}

\хфилл

\ноиндент
{\бф Доказательство:} 
Пусть $X\subset [0,1]$ --- подмножество, не являющееся 
отрезком, интервалом или полуинтервалом.
Тогда в $[0, 1] \backslash X$ содержится 
точка $\alpha$ такая, что в $X$ есть точка
$x> \alpha$ и точка $y < \alpha$.
В этом случае, $X$ разбивается в объединение
двух непустых, открытых подмножеств
$X \cap [0, \alpha[$ и $X \cap ]\alpha, 0]$.

Чтобы доказать, что отрезок, интервал или
полуинтервал $I$ связны, возьмем разбиение $I$
в объединение двух непересекающихся, открытых подмножеств
$X$ и $Y$, и пусть $x\in X$ --- любая точка.
Для простоты, положим, что $I$ --- интервал
(доказательство для отрезка и полуинтервала аналогично).
Пусть $I_x$ --- объединение всех интервалов,
лежащих в $X$ и содержащих $x$. Тогда $I_x$
это интервал $]\alpha, \beta[\subset I$, 
причем либо $\alpha$, либо $\beta$
не является концом $I$. Пусть, к примеру,
$\alpha$ это не конец $I$. Тогда $\alpha$ 
содержится в $Y$. Поскольку $Y$ 
открыто, из этого должно следовать, что 
$Y$ содержит некоторую окрестность точки $\alpha$,
но это невозможно, потому что $\alpha$ является
предельной точкой $X$.
\endproof


\хфилл

\замечание
Замыкание $\bar Z$ связного подмножества $Z$ всегда связно.
Действительно, если $\bar Z$ удалось разбить в объединение
непустых непересекающихся открытых подмножеств $U$ и $V$,
$U \cap Z$ и $V\cap Z$ открыты и не
пересекаются. Поскольку $Z$ плотно в $\bar Z$, эти 
множества непусты (докажите).
\еза

\замечание
Пусть $X$ связно, а $f:\; X \arrow Y$ --- 
непрерывное отображение. Тогда $f(Y)$ связно.
Действительно, если $f(Y)$ разбито в объединение
непустых непересекающихся открытых подмножеств,
их прообразы --- непересекающиеся открытые подмножества $Y$
\еза

\следствие
Если $f:\; X \arrow \R$ --- непрерывная функция
на связном множестве то $f(X)$ это отрезок,
интервал или полуинтервал. В частности,
$f$ принимает все промежуточные значения
между $f(x_1)$ и $f(x_2)$. 


\begin{figure}[ht]
\begin{center}\ \\
\epsfig{file=Poincare.eps,width=0.60\linewidth}\\
{Jules Henri Poincar\'e\\
(1854 --- 1912)}
\end{center}
\end{figure}



\hfill

Понятие связности изучали многие математики в XIX веке, но
математически строгое определение связного множества первым
ввел Анри Пуанкаре.

%%%%%%%%%%%%%%%%%%%%%%%%%%%%%%%%%%%%%%%%%%%%%%%%

\section{Компоненты связности}

%%%%%%%%%%%%%%%%%%%%%%%%%%%%%%%%%%%%%%%%%%%%%%%%

\замечание
Пусть $X$ и $Y$ --- два пересекающихся связных подмножества
топологического пространства $M$. Тогда $X \cup Y$ тоже связно
(докажите это).
\еза


\определение
Пусть $x\in M$ --- точка топологического пространства,
а $A_x$ --- объединение всех связных подмножеств $M$,
содержащих $x$. В силу вышесказанного, $A_x$ связно.
Множество $A_x$ называется {\бф компонентой связности}
точки $x$.
\ео

\замечание
Каждое топологическое пространство является
неперескающимся объединением своих компонент связности.
\еза


\замечание
Если $M$ представлено в виде объединения
непустых, непересекающихся открытых подмножеств $U$ и $V$,
каждая компонента связности $M$ содержится
целиком в $U$ или в $V$ (докажите).
\еза

Напомним, что непрерывное отображение называется
{\бф открытым}, если оно переводит открытые множества
в открытые.


\замечание
Пусть $f:\; X \arrow Y$ --- открытое отображение
со связными слоями. Предположим, что
$X$ представлено в виде объединения
непустых, непересекающихся открытых подмножеств $U$ и $V$.
Поскольку слои $f$ связны, подмножества $U$ и $V$ содержат каждый
слой целиком. Поэтому $U = f^{-1}(U_1)$, $V = f^{-1}(V_1)$,
причем $U_1$ и $V_1$ не пересекаются.
Поскольку $f$ открыто, $U_1$ и $V_1$ открыты
и непусты. 
\еза

\следствие
Пусть $f:\; X \arrow Y$ --- открытое отображение
со связными слоями, и $Y$ связно. Тогда $X$ связно.
\endproof

\замечание 
Из этого следует, что 
произведение связных пространств $X$ и $Y$ связно.
Действительно, отображение проекции $\pi:\; X\times Y\arrow Y$
открыто (докажите)
и имеет связные слои.
\еза

\замечание
Если  $f:\; X \arrow Y$ не открыто, то связность
$X$ не вытекает из связности $Y$ и слоев $f$.
В качестве примера, рассмотрим естественное
отображение из $\R$ с дискретной топологией в
$\R$ с обычной топологией. Слои этого отображения --
точки, образ связен, но $\R$ с дискретной
топологией, очевидно, несвязно.
\еза


\замечание
Компоненты связности топологического пространства
замкнуты, потому что замыкание связного множества
связно. 
\еза

\определение
Топологическое пространство называется
{\бф вполне несвязным}, если все его связные
подмножества --- точки.
\ео

\утверждение
Пусть $M$ --- хаусдорфово топологическое пространство, у которого
есть база топологии, состоящая из открыто-замкнутых
множеств. Тогда $M$ вполне несвязно.

\хфилл

\ноиндент
{\бф Доказательство:} 
Пусть $m \in M$. В силу хаусдорфовости, у каждой точки
$m_1\neq m$ есть открытозамкнутая окрестность, не
содержащая $m$. Поэтому $\{m\}$ --- пересечение
открыто-замкнутых множеств, $\{m\} = \bigcap U_\alpha$. 
Любое связное подмножество
$Z\subset M$, содержащее $m$, содержится в каждом из этих 
открыто-замкнутых множеств целиком (докажите). Поэтому 
$Z \subset \bigcap U_\alpha=\{m\}$.
\endproof

\хфилл

\следствие
Пространство $\Z_p$ $p$-адических чисел
вполне несвязно. Действительно, каждый открытый
шар в $\Z_p$ замкнут (докажите это).

\задача
Все ли открытые подмножества $\Z_p$ открытозамкнуты?
\ез


\замечание 
Пусть $M$ --- топологическое пространство,
а \\ $f:\; M \arrow X$ --- произвольное отображение.
Рассмотрим на $X$ такую топологию: $U\subset X$ открыто,
если $f^{-1}(U)$ открыто. Это сильнейшая топология
на $X$ такая, что $f$ непрерывно. 
\еза


\определение
Пусть $M$ --- топологическое пространство,
а $\underline M$ --- множество компонент связности $M$.
Рассмотрим на $\underline M$ сильнейшую топологию,
в которой естественная проекция 
$\pi:\; M \arrow \underline M$
непрерывна. Тогда $\underline M$ называется
{\бф пространством компонент связности $M$}.
\ео

%%%%%%%%%%%%%%%%%%%%%%%%%%%%%%%%%%%%%%%%%%%%%%%%%%%%%%%%%%%%

\section{Линейная связность}

%%%%%%%%%%%%%%%%%%%%%%%%%%%%%%%%%%%%%%%%%%%%%%%%%%%%%%%%%%%%


\определение
Пространство $M$ называется {\бф линейно связным},
если для любых двух точек $x, y\in M$ найдется
непрерывное отображение $\gamma:\; [0,1]\arrow M$
такое, что $x$ и $y$ лежат в образе $\gamma$.
\ео

\замечание
Из линейной связности следует связность.
Действительно, отрезок $[0,1]$ связен,
образ связного множества связен, объединение
пересекающихся связных множеств связно,
и поэтому линейно связное пространство
состоит из одной-единственной компоненты связности.
\еза

\begin{figure}[ht]
\begin{center}\ \\
\epsfig{file=sin-1x.eps,width=0.8\linewidth}\\
{\small  График $f(x)= \sin(1/x)$}
\end{center}
\end{figure}

Рассмотрим график $\Gamma_f$ отображения $f(x)= \sin(1/x)$,
\[ f:\; ]0, 1]\arrow [-1,1].\]


Это множество незамкнуто; легко видеть, что его замыканием
$\bar\Gamma_f$ будет объединение $\Gamma_f$ и отрезка 
$\{0\}\times [-1,1]$.

Пространство $\Gamma_f$ гомеоморфно полуинтервалу $]0, 1]$
(гомеоморфизм задается проекцией на ось абсцисс).
Поэтому $\bar\Gamma_f$ тоже связно, как замыкание
связного множества.

\хфилл

\утверждение
Построенное таким образом множество
 $\bar\Gamma_f$ не является линейно связным.


\хфилл

\ноиндент
{\бф Доказательство:}
Пусть $\gamma:\; [0,1] \arrow \bar\Gamma_f$ --
отображение из отрезка, образ которого не лежит
целиком в $\{0\}\times [-1,1]$. Докажем, что
образ $\gamma$ не пересекает $\{0\}\times [-1,1]$.
Пусть $U \subset [0,1]$ --- открытое множество,
полученное как $\gamma^{-1}(\Gamma_f)$,
$U_0$ --- какая-то компонента связности $U$,
$\bar U_0$ --- ее замыкание, 
а $Z= \gamma(\bar U_0)$. Предположим,
что $Z$ пересекает $\{0\}\times [-1,1]$
(в противном случае $\gamma([0,1])$ лежит
в $\Gamma_f$, и мы все доказали).

\хфилл


\ноиндент
{\бф Шаг 1:} Поскольку $\bar U_0$ компактно,
$Z$ тоже компактно, а следовательно замкнуто.

\хфилл

\ноиндент
{\бф Шаг 2:}
Поскольку замыкание $\gamma(U_0)$ пересекает
$\{0\}\times [-1,1]$, оно не содержится
в $\Gamma_f$. Следовательно, $\gamma(U_0)$
содержит подмножество вида 
\[
\{ (\alpha, \sin(1/\alpha)) \ | \  0< \alpha< c\},
\]
и его замыкание $Z$ содержит отрезок $\{0\}\times [-1,1]$ целиком.


\хфилл

\ноиндент
{\бф Шаг 3:} Поскольку $Z\backslash \gamma(U_0)$
содержит $\{0\}\times [-1,1]$,
образ $\gamma(\bar U_0 \backslash U_0)$ содержит отрезок
$\{0\}\times [-1,1]$. Но это невозможно, потому что
множество $\bar U_0 \backslash U_0$  конечно.
\endproof

%%%%%%%%%%%%%%%%%%%%%%%%%%%%%%%%%%%%%%%%%%%%%%%%%%%%%%%%%%%%%%%%%%%%%%%%

\chapter{Лекция 13: Вполне несвязные пространства}

%%%%%%%%%%%%%%%%%%%%%%%%%%%%%%%%%%%%%%%%%%%%%%%%%%%%%%%%%%%%%%%%%%%%%%%%


%%%%%%%%%%%%%%%%%%%%%%%%%%%%%%%%%%%%%%%%%%%%%%%%%%%%%%%%%%%%

\section{Примеры вполне несвязных пространств}

%%%%%%%%%%%%%%%%%%%%%%%%%%%%%%%%%%%%%%%%%%%%%%%%%%%%%%%%%%%%

Пусть $M$ ---  топологическое пространство. Напомним, что
$M$ называется {\бф вполне несвязным} (totally
disconnected), если любое подмножество
$M$, взятое с индуцированной топологией, несвязно,
если оно содержит больше одной точки $M$.
 

Напомним, что
подмножество $U \subset M$ называется {\бф открытозамкнутым}
(clopen), если оно одновременно открыто и замкнуто. 

\замечание
Конечное пересечение, конечное объединение,
дополнение открытозамкнутых подмножеств
снова открытозамкнуто.
\еза

\замечание
Пусть в топологическом пространстве $M$ есть
предбаза из открытозамкнутых множеств. Тогда в $M$
есть база из открытозамкнутых множеств. Действительно,
базу можно получить из предбазы взятием конечных
пересечений (докажите).
\еза


\замечание
Предположим, что у хаусдорфова топологического
пространства $M$ есть база топологии, состоящая из
открытозамкнутых множеств. Тогда оно вполне несвязно.
Действительно, в этом случае у каждого подмножества
$Z\subset M$ есть база из открытозамкнутых множеств.
В случае, когда $Z$ содержит более одной точки, $Z$ содержит непустое и не 
совпадающее с $Z$ открытозамкнутое множество (выведите это из 
хаусдорфовости $M$).
\еза


\замечание
Пусть $M$ --- топологическое пространство, у которого
есть база из открытозамкнутых множеств. 
Тогда тихоновское произведение $M^I$ 
для любого набора индексов $I$ имеет базу из 
открытозамкнутых множеств (докажите). Поэтому
оно вполне несвязно.
\еза

\задача
Докажите, что произведение вполне несвязных
прост\-ранств вполне несвязно.
\ез

\пример
Пусть $M=\{0,1\}$ --- множество из двух точек, с дискретной
топологией (такое множество называется {\бф двоеточием}).
Произведение любого числа двоеточий компактно
(по теореме Тихонова), хаусдорфово и вполне несвязно.


\замечание
Предположим, что $(M, d)$ --- метрическое пространство,
причем метрика $d:\; M \times M \arrow \R$ 
не принимает значений в интервале $]\alpha, \beta[$,
где $\beta> \alpha$. Тогда замкнутый шар 
$\bar B_\alpha(x)$ открытозамкнут. Действительно,
$\bar B_\alpha(x) = B_{\alpha+\epsilon}(x)$,
для любого $\epsilon$ такого, что $\alpha +\epsilon
\in]\alpha, \beta[$, а шар $ B_{\alpha+\epsilon}(x)$
открыт.
\еза

\пример
Из этого немедленно следует, что пространство $\Z_p$
$p$-адических целых чисел вполне несвязно. Действительно,
$p$-адическая метрика принимает значения в множестве $\{p^{s}, s\in \Z\}$,
и в силу предыдущего замечания любой шар в $\Z_p$
открытозамкнут (докажите).

\задача
Докажите, что $\Z_2$ гомеоморфно $\{0,1\}^\N$.
\ез

\пример
Легко видеть, что $\Q$ (множество рациональных чисел,
с естественной топологией) вполне несвязно (докажите это)
и некомпактно.

\задача
Найдите все компактные подмножества в $\Q$.
\ез 


%%%%%%%%%%%%%%%%%%%%%%%%%%%%%%%%%%%%%%%%%%%%%%%%%%%%%%%%%%%%%%%%%%%%%%%%

\section{Пространства Стоуна}

%%%%%%%%%%%%%%%%%%%%%%%%%%%%%%%%%%%%%%%%%%%%%%%%%%%%%%%%%%%%%%%%%%%%%%%%

\определение
Компактное, вполне несвязное, хаусдорфово
топологическое пространство называется
{\бф пространством Стоуна} \\ (Stone space).
\ео



\begin{figure}[ht]
\begin{center}
\epsfig{file=Stone1.eps,width=0.6\linewidth}\\
{Marshall Harvey Stone\\
(1903 --- 1989)}
\end{center}
\end{figure}


%%%%%%%%%%%%%%%%%%%%%%%%%%%%%%%%%%%%%%%%%%%%%%%%
\лемма\label{_Okrestnosti_otkrytoza_neper_Lemma_}
Пусть $M$ --- пространство Стоуна, а $x,y \in M$ --- две разные точки.
Тогда у $x$ и $y$ есть непересекающиеся, открытозамкнутые окрестности.

\хфилл

\ноиндент
{\бф Доказательство:} 
Пусть $Z=\bigcap_\alpha U_\alpha$ --- пересечение всех 
открытозамкнутых подмножеств, содержащих $x$. Если $Z$
не содержит $y$, какое-то открытозамкнутое подмножество
$U\ni x$ не содержит $y$, тогда его дополнение $M\backslash U$ 
содержит $y$, и мы получим, что у $x$ и $y$ есть
 непересекающиеся открытозамкнутые окрестности. Поэтому достаточно
доказать, что $Z=\{x\}$.

\хфилл

\ноиндент
{\бф Шаг 1:}
Предположим, что $Z\neq \{x\}$. Поскольку  $Z$ --- пересечение
замкнутых подмножеств $M$, оно замкнуто. 
Поскольку $Z$ несвязно, $Z$ есть объединение 
непересекающихся подмножеств $Z = Z_x\bigsqcup Z'$,
замкнутых в $Z$,\footnote{Обозначение
$X = Y \bigsqcup Z$ используется, чтобы указать,
что $X$ есть объединение непересекающихся множеств $Y$ и $Z$.}
где $Z_x$ содержит $x$, а $Z'$ не содержит $x$.
 Но коль скоро $Z$ замкнуто, $Z_x$, $Z'$ замкнуты
в $M$ (проверьте это). 

\хфилл

\ноиндент
{\бф Шаг 2:}
Поскольку $M$ компактно и хаусдорфово,
$M$ нормально, то есть любые два замкнутых, непересекающихся
подмножества $M$ имеют непересекающиеся окрестности
(Лекция 8). Применив это к $Z_x, Z'$, получим
непересекающиеся окрестности $U_x\supset Z_x, U'\supset
Z'$. 


\begin{figure}[ht]
\begin{center}\ \\
\epsfig{file=otkrytoza.eps,width=0.5\linewidth}\\
{\small  Непересекающиеся окрестности $Z_x$, $Z'$}
\end{center}
\end{figure}



\хфилл

\ноиндент
{\бф Шаг 3:}
Обозначим через $K$ дополнение $M \backslash (U_x \cup U')$
Поскольку $Z=\bigcap_\alpha U_\alpha$,
имеем
\[
K\subset \bigcup_\alpha (M\backslash U_\alpha).
\]
Поскольку $K$ замкнуто, а $M$ компактно, $K$ тоже 
компактно. Поэтому открытое покрытие 
$K\subset\bigcup_\alpha (M\backslash U_\alpha)$ имеет конечное
подпокрытие:
\[
K \subset \bigcup_{i=1}^n (M\backslash U_i)
\]
где все $U_i$ открытозамкнуты, и содержат $Z$.
Из этого следует, что $U:=\bigcup_{i=1}^n U_i$
открытозамкнуто, содержит $x$, и содержится в 
$M \backslash K=U_x \cup U'$.


\хфилл

\ноиндент
{\бф Шаг 4:} $U= (U \cap U_x) \bigsqcup (U \cap U')$.
Поскольку $U_x, U'$ открыты, $U \cap U_x$ открыто и замкнуто
в $U$. Следовательно, $V=U \cap U_x$ открыто и замкнуто в $M$.
Это множество не пересекается с $Z'$, и содержит $x$, по
построению. Значит, пересечение всех открытозамкнутых
окрестностей $x$ не  не пересекается с $Z'$. Мы пришли 
к противоречию, и доказали, что $Z=\{x\}$. Лемма
\ref{_Okrestnosti_otkrytoza_neper_Lemma_} доказана.
\endproof


\хфилл

Из этой леммы немедленно следует

\хфилл


\теорема
Пусть $M$ --- пространство Стоуна. Тогда у $M$ есть база
топологии, состоящая из открытозамкнутых подмножеств.

\хфилл

\ноиндент
{\бф Доказательство:} Пусть ${\cal T}$ --- топология на
$M$, а ${\cal T}_1$ --- топология, полученная 
из базы, состоящей из всех открытозамкнутых
подмножеств $M$. Естественное отображение
\[
(M, {\cal T}) \stackrel \Id \arrow (M, {\cal T}_1)
\]
биективно и непрерывно (проверьте). В силу
Леммы \ref{_Okrestnosti_otkrytoza_neper_Lemma_},
пространство  $(M, {\cal T}_1)$ хаусдорфово (докажите это).
Непрерывная биекция из компактного пространства 
в хаусдорфово является гомеоморфизмом,
следовательно, топология ${\cal T}$ 
экивалентна ${\cal T}_1$. \endproof


\замечание \label{_razdelya_tochki_vpolne_nesvyazno_Zamechanie_}
Из Леммы \ref{_Okrestnosti_otkrytoza_neper_Lemma_}
немедленно вытекает, что для любой пары точек
$x, y \in M$, существует непрерывная функция
\[ f:\; M \arrow \{0,1\},\] принимающая значение
1 на $x$ и 0 на $y$.
\еза

Пусть $R=C(M, {\Bbb F}_2)$ --- кольцо непрерывных
функций на $M$ со значениями в ${\Bbb F}_2=\{0,1\}$.
Рассмотрим отображение
\[
M \stackrel {\Psi}\arrow  \{0,1\}^R,
\]
переводящее $m$ в $\prod_{\phi\in R}\phi(m)$.
Оно непрерывно, что следует из определения 
тихоновской топологии (проверьте это).
Из Замечания \ref{_razdelya_tochki_vpolne_nesvyazno_Zamechanie_}
следует, что $\Psi$ инъективно; поскольку $M$
компактно, из этого вытекает, что $\Psi$ ---  это
гомеоморфизм на его образ. Мы получили следующую
теорему

\хфилл

\теорема
Пусть $M$ --- пространство Стоуна. Тогда $M$ гомеоморфно
замкнутому подмножеству в тихоновском произведении $\{0,1\}^I$,
для какого-то набора индексов $I$.

%%%%%%%%%%%%%%%%%%%%%%%%%%%%%%%%%%%%%%%%%%%%%%%%%%%%%%%%%%%%%%%%%%%%%%%%

\chapter[Лекция 14: Теорема Стоуна и теория
  категорий]{Лекция 14: Теорема Стоуна о
    представлении булевых колец и теория категорий}

%%%%%%%%%%%%%%%%%%%%%%%%%%%%%%%%%%%%%%%%%%%%%%%%%%%%%%%%%%%%%%%%%%%%%%%%

%%%%%%%%%%%%%%%%%%%%%%%%%%%%%%%%%%%%%%%%%%%%%%%%%%%%%%%%%%%%

\section{Категории}

%%%%%%%%%%%%%%%%%%%%%%%%%%%%%%%%%%%%%%%%%%%%%%%%%%%%%%%%%%%%

\определение
{\бф Категорией} $\cac$ называется набор данных ("объектов
категории", "морфизмов между объектами" и так далее), 
удовлетворяющих аксиомам, приведенным ниже.

\hfill

\ноиндент
{\бф  Данные:}
\begin{description}
\item[Объекты:] Множество $\Ob(\cac)$ объектов $\cac$
(иногда рассматривают не множество, а {\ем класс}
$\Ob(\cac)$, который может и не быть множеством, например,
класс всех множеств, или класс всех линейных пространств).

\item[Морфизмы:] Для любых $X, Y \in \Ob(\cac)$, задано
множество $\Mor(X,Y)$ {\бф морфизмов} из $X$ в $Y$.

\item[Композиция морфизмов:] Если
$\phi\in \Mor(X,Y), \psi \in \Mor(Y,Z)$, 
задан морфизм $\phi\circ \psi \in \Mor(X, Z)$,
который называется {\бф композицией морфизмов}.

\item[Тождественный морфизм:] Для каждого $A\in \Ob(\cac)$
задан морфизм $\Id_A \in \Mor(A,A)$.
\end{description}

\ноиндент
Эти данные удовлетворяют следующим аксиомам.
\begin{description}
\item[Ассоциативность композиции:] 
$\phi_1\circ(\phi_2\circ\phi_3)=(\phi_1\circ\phi_2)\circ\phi_3$.
\item[Свойства тождественного морфизма:]
Для любого морфизма $\phi\in \Mor(X,Y)$,
$\Id_X\circ \phi = \phi = \phi\circ \Id_Y$.
\end{description}
\ео

Практически любая математическая структура является
категорией. Например, категория множеств (морфизмы --
произвольные отображения), категория линейных пространств 
(морфизмы --- линейные отображения), категории колец,
полей, групп (морфизмы --- гомоморфизмы), категория
топологических пространств (морфизмы --- непрерывные
отображения) и так далее. 

Категории сами образуют категорию; морфизмами
этой категории являются {\бф функторы}.

\определение
Пусть $\cac_1, \cac_2$ --- категории. {\бф Ковариантным функтором}
из $\cac_1$ в $\cac_2$ называется следующий набор данных.
\begin{description}
\item[(i)] Отображение $F:\; \Ob(\cac_1) \arrow \Ob(\cac_2)$,
ставящее в соответствие объектам $\cac_1$ объекты $\cac_2$.
\item[(ii)] Отображение морфизмов $F:\; \Mor(X,Y) \arrow \Mor(F(X), F(Y))$,
определенное для любой пары объектов $X, Y \in \Ob(\cac_1)$.
\end{description}
Эти даные {\ем определяют функтор
из $\cac_1$ в $\cac_2$}, если $F(\phi) \circ F(\psi) =
F(\phi\circ\psi)$, и $F(\Id_X) = \Id_{F(X)}$.
\ео

Легко видеть, что композиция функторов --- тоже функтор;
таким образом, категория всех категорий --- тоже категория.

\хфилл

\пример
Любая "естественная операция" на математических объектах --- это
функтор. Например, отображение $X \arrow 2^X$ на категории
множеств, или отображение $M \arrow M^I$ на топологических
пространствах, для заданного набора индексов
$I$, или отображение $V \arrow V \oplus V$
на линейных пространствах. Другим примером функтора
является {\бф тождественный функтор} из категории
в себя. Еще один пример функтора --- отображение,
ставящее в соответствие топологическому пространству
его пространство связных компонент. 

\определение
Если задана категория $\cac$, можно определить \\ {\бф двойственную
категорию} ("opposite category") 
$\cac^{op}$. Множество объектов в $\cac^{op}$ --- то же самое,
что и в $\cac$, а $\Mor_{\cac^{op}}(A,B)= \Mor_{\cac}(B,A)$.
Соответственно, композиция $\phi\circ \psi$ в $\cac$ дает
композицию $\psi^{op}\circ \phi^{op}$ в $\cac^{op}$.
Проверьте, что это категория.
\ео

\определение
{\бф Контравариантный функтор из $\cac_1$ в $\cac_2$} 
это функтор из $\cac_1^{op}$ в $\cac_2$.
\ео

\пример
Примером контравариантного функтора является
отображение, ставящее векторному пространству 
$V$ в соответствие двойственное пространство $V^*$.
Другим примером контравариантного функтора является отображение
из топологических пространств в кольца, ставящее
в соответствие топологическому пространству $M$ кольцо
непрерывных $\R$-\-зна\-ч\-ных функций на $M$
(проверьте, что это функтор).

\определение
Пусть $X, Y\in \Ob(\cac)$ --- объекты категории $\cac$.
Морфизм $\phi\in \Mor(X,Y)$ называется {\бф изоморфизмом},
если существует $\psi\in \Mor(Y,X)$ такой, что
$\phi \circ \psi = \Id_X$ и $\psi\circ\phi = \Id_Y$.
В таком случае, объекты $X$ и $Y$ называются
{\бф изоморфными}.
\ео

\замечание
Пусть $X\in \Ob(\cac)$ --- объект категории $\cac$.
Тогда отображение $Y\arrow \Mor(X,Y)$ задает ковариантный
функтор из $\cac$ в категорию $\Set$ множеств (проверьте это),
а отображение $Y\arrow \Mor(Y, X)$ задает контравариантный
функтор из $\cac$ в  $\Set$ (проверьте). Такие функторы
называются {\бф представимыми}.
\еза

\определение
Два функтора $F, G:\;\cac_1\arrow \cac_2$ 
называются {\бф эквивалентными}, если для каждого 
$X \in \Ob(\cac_1)$ задан изоморфизм $\Psi_X:\; F(X) \arrow
G(X)$, для любого морфизма $\phi\in \Mor(X,Y)$,
\begin{equation}\label{_equi_fu_Equation_}
 F(\phi) \circ \Psi_Y= \Psi_X\circ G(\phi).
\end{equation}
\ео

\замечание
Подобные коммутационные отношения принято \\изображать
{\бф коммутативными диаграммами}. Так, к примеру,\\
\eqref{_equi_fu_Equation_} можно записать следующей
коммутативной диаграммой
\begin{equation}\label{_de_Rham_Equation_} 
\begin{CD}
F(X) @>{F(\phi)}>> F(Y)\\
@V{\Psi_X}VV @VV{\Psi_Y}V\\
G(X) @>{G(\phi)}>> G(Y)
\end{CD}
\end{equation}
\еза



\задача
Пусть $\cac \stackrel F\arrow \Set$ --- функтор из категории $\cac$ в
категорию множеств. Предположим, что $F$ {\em представим:}
$F$ эквивалентен функтору $Y \arrow \Mor(X,Y)$. Докажите, что тогда $X$ определен
однозначно с точностью до изоморфизма.
\ез


\определение
Функтор $F:\; \cac_1 \arrow \cac_2$ называется
{\бф эквивалентностью категорий}, если
найдутся функторы $G, G':\; \cac_2 \arrow \cac_1$
такие, что $F\circ G$ эквивалентен тождественному
функтору на $\cac_1$, а $G' \circ F$ эквивалентен
тождественному функтору на $\cac_2$.
\ео

\замечание
Можно проверить, что это равносильно следующему:
$F$ задает биекцию на классах изоморфизма
объектов, и биекцию $\Mor(X,Y) \arrow \Mor(F(X), F(Y))$.
\еза



\begin{figure}[ht]
\begin{center}
\epsfig{file=Maclane.eps,width=0.65\linewidth}\\
{Saunders Mac Lane\\
(1909-2005)}
\end{center}
\end{figure}

С точки зрения теории категорий, эквивалентные категории
неразличимы.



%%%%%%%%%%%%%%%%%%%%%%%%%%%%%%%%%%%%%%%%%%%%%%%%%%%%%%%%%%%%

\section{Теория категорий: история, замечания}

%%%%%%%%%%%%%%%%%%%%%%%%%%%%%%%%%%%%%%%%%%%%%%%%%%%%%%%%%%%%

Категории были изобретены топологами Самуэлем Эйленбергом
и Сондерсом Маклейном в 1942-45 годах, для употребления
в топологии. Эйленберг и Маклейн заметили, что
алгебраические инварианты топологических пространств
(такие, как фундаментальная группа) являются
функторами. Также функторами являются
различные естественные конструкции в топологии,
например, "пространство петель", сопоставляющее
пространству $M$ пространство отображений
$C(S^1, M)$ (Лекция 11). Эйленберг и Маклейн
обнаружили, что алгебраическую топологию
гораздо проще изучать, если абстрагироваться
от геометрической стороны дела. Несмотря
на кажущуюся абстрактность, условие
функториальности того или иного 
отображения во многих случаях достаточно 
для доказательства единственности и для его 
явного вычисления, и дополнительные геометрические
детали только затрудняют работу.

Основным алгебраическим инвариантом топологического
пространства является {\бф группа когомологий}; к середине
1940-х, топологи знали с десяток разных геометрических определений
когомологий, но работать с ними было неловко, потому что
эквивалентность доказывать не умели. Эйленберг и Маклейн,
а в 1950-е --- Эйленберг и Стинрод определили когомологии
в терминах категорий, таким образом, что проверить различные 
свойства когомологий оказалось очень просто.

В конце 1950-х теория категорий легла в основу
алгебраической геометрии, разработанной А. Гротендиком,
Ж. Дьедонне и группой Бурбаки, в книге
"\'El\'ements de g\'eom\'etrie alg\'ebrique" и других книгах, 
и стала основным языком современной математики, без
которого не обходится ни одна область, развившаяся
после 1960-х. 

В терминах категорий можно определить довольно много
математических конструкций, не прибегая к "множествам",
и их "элементам". Например, произведение объектов
$A$ и $B$ категории $\cac$ можно определить
как такой объект $A\times B$, что 
функтор $X \arrow \Mor(X, A\times B)$ из $\cac$ в категорию множеств
эквивалентен функтору $X \arrow \Mor(X, A) \times \Mor(X, B)$.


\begin{figure}[ht]
\begin{center}
\epsfig{file=Eilenberg.eps,width=0.5\linewidth}\\
{Samuel Eilenberg \\
(1913-1998)}
\end{center}
\end{figure}


Многие математики предлагают отказаться от теории
множеств в изложении основ математики, и использовать
вместо нее аксиоматическую теорию
категорий.\footnote{Первым
тут был, видимо, Ловир (F. W. Lawvere), предложивший
категории в качестве фундамента математики
в своей диссертации в 1963-м году; с тех пор
эту точку зрения разнообразно развивали и
логики, и специалисты по категориям.}
Впрочем, сама теория категорий не
свободна от теоретико-множественных трудностей,
которые связаны с тем, что $\Ob(\cac)$ для 
большинства категорий (для категории множеств,
например) --- не множество, а класс.

Следуя Гротендику и Вердье,
теоретико-множественные трудности в определении
категории обыкновенно обходят следующим способом.
{\бф Малой категорией} называется такая категория,
у которой $\Ob(\cac)$ и $\Mor(X,Y)$ --- множества.
Вместо категории "всех" (множеств, пространств, колец
и так далее) рассматривают категорию множеств, пространств, колец
и так далее, с мощностью, ограниченной некоторым
(раз и навсегда выбранным) кардиналом, который
называется {\бф универсумом Гротендика},
и определяется аксиоматически.
Пользуясь аксиомами универсума,
легко видеть, что каждая категория
с объектами малой мощности эквивалентна 
малой категории.

\begin{figure}[ht]
\begin{center}
\epsfig{file=Grothendieck.eps,width=0.5\linewidth}\\
{Alexander Grothendieck \\
(род. 28 марта 1928)}
\end{center}
\end{figure}

К сожалению, существование универсума 
равносильно существованию {\ем сильно недостижимых кардиналов}
в теории множеств. Подобная аксиома независима
от аксиом ZFC (Цермело-Френкеля плюс аксиома выбора).
Непротиворечивость ZFC  равносильна непротиворечивости 
ZFC, где дополнительно постулировано несуществование сильно
недостижимых кардиналов. Из существования сильно
недостижимых кардиналов можно вывести 
непротиворечивость ZFC в рамках самой ZFC,
поэтому непротиворечивость аксиомы универсума 
строго сильнее, чем ZFC.



%%%%%%%%%%%%%%%%%%%%%%%%%%%%%%%%%%%%%%%%%%%%%%%%%%%%%%%%%%%%

\section{Булевы кольца и булевы алгебры}

%%%%%%%%%%%%%%%%%%%%%%%%%%%%%%%%%%%%%%%%%%%%%%%%%%%%%%%%%%%%

\определение
{\бф Идемпотент} в кольце --- элемент, удовлетворяющий 
соотношению $x^2=x$.
{\бф Булево кольцо} --- кольцо (коммутативное, с единицей),
где любой элемент является идемпотентом.
\ео

\замечание
Отметим, что в булевом кольце выполнено соотношение
$2x=0$ для любого $x$.  Действительно,
\[
  0 = (x+1)^2 -x-1 = (x^2-x) + 2x = 2x.
\]
\еза

Булевы кольца чрезвычайно важны в логике и информатике,
ибо категория булевых колец эквивалентна категории булевых алгебр,
то есть алгебр логических высказываний.

\определение
Пусть $A$ --- множество, наделенное бинарными \\ операциями
$\land$ (коньюнкция, "и"), $\lor$ (дизъюнкция, "или"),
и $\lnot$ (негация, "не"), и выделены элементы
1 ("истина") и 0 ("ложь"). $A$ называется {\бф булевой
алгеброй}, если выполнены следующие условия.

\begin{description}
\item[ассоциативность:] $a\land (b\land c)= (a\land b)\land c$,\ \  
$a\lor (b\lor c)= (a\lor b)\lor c$
\item[коммутативность:] $a\land b=b\land a$,\ \  $a\lor b=b\lor a$ 
\item[дистрибутивность:] $a\land (b\lor c) =(a\land  b) \lor (a\land c)$, \ \ 
$a\lor (b\land c) =(a\lor  b) \land (a\lor c)$
\item[абсорбция:] $a\lor (a\land b) = a$, \ \ $a\land (a\lor b) = a$
\item[дополнительность:] $a\lor \lnot a = 1$,\ \  $a\land \lnot a = 0$.
\end{description}
\ео

\замечание
Этим условиям, очевидно, удовлетворяет любой набор логических 
утверждений, которые могут принимать значения ``истинно'' и ``ложно''.
Логические операции ("и", "или", "не")
превращают такой набор утверждений
в булеву алгебру.
\еза

\замечание
Пусть $A$ --- булева алгебра.
Зададим на $A$ умножение и сложение следующим образом:
$a\cdot b := a \land b$, $a + b := (a\lor b) \land \lnot (a \land b)$
(сумма соответствует "симметрической разности", или,
что то же самое, "исключающему или").
Полученные операции удовлетворяют аксиомам кольца (это ясно
из ассоциативности, коммутативности и дистрибутивности; проверьте). 
Ноль и 1 играют роль 0 и 1, а соотношение
$a^2 =a$ следует из аксиомы дополнительности $a +\lnot a =1$, которая 
(после домножения на $a$) влечет
\[
а\land a + а\land\lnot a=a
\]
применив $а\land\lnot a=0$ (дополнительность), обретем $a^2 =a$.
Мы получили функтор $F$ из категории булевых алгебр в категорию
булевых колец.
\еза

\теорема\label{_bulevy_algebra_kolca_ekvi_Teorema_}
Этот функтор --- эквивалентность категорий.

\хфилл

\ноиндент
{\бф Доказательство:}
Чтобы доказать, что $F:\; \cac \arrow \cac_1$ эквивалентность,
надо построить обратный функтор $G$, то есть из каждого
булева кольца произвести булеву алгебру. Делается
это весьма просто: если $A$ --- булево кольцо,
операции $\lnot, \land, \lor$ определяются формулами
$\lnot a = 1-a$, $a\land b = ab$, $a\lor b = ab +a +b$.
Аксиомы булевой алгебры проверяются непосредственно
(проверьте их), а взаимная обратность $F$ и $G$
очевидна из конструкции. \endproof

\хфилл

\пример
Пусть $M$ --- любое топологическое пространство,
а $R= C(M, {\Bbb F}_2)$ --- кольцо непрерывных
функций на $M$ со значениями в поле ${\Bbb F}_2$
из двух элементов. Тогда $R$ --- булево кольцо.
Оно называется {\бф кольцом открытозамкнутых
подмножеств $M$}, а соответствующая булева алгебра --
{\бф алгеброй открытозамкнутых
подмножеств}.

\хфилл

Следующая простая лемма известна из Лекции 9.

\хфилл

\лемма
Пусть $R$ --- булево кольцо, а $I\subset R$ --- максимальный
идеал в $R$. Тогда $R/I\cong {\Bbb F}_2$.

\хфилл

\ноиндент
{\бф Доказательство:} 
Поскольку $R$ состоит из идемпотентов, поле $R/I$
тоже состоит из идемпотентов. Это значит, что
все его элементы являются корнями многочлена
$x^2-x=0$. По теореме Безу, многочлен степени $m$ в поле
не может иметь больше $m$ разных корней, значит,
$R/I$ состоит из двух элементов. \endproof

\замечание\label{_prostye_maks_v_bul_Zamechanie_}
Похожий аргумент доказывает, что 
каждый простой идеал в булевом кольце --- максимальный.
Действительно, пусть $I\subset R$ --- простой идеал.
Тогда $R/I$ --- кольцо без делителей нуля, все элементы
которого --- идемпотенты. Для любого $x \in R/I$,
имеем $x(1-x)=0$, значит, или $x$ или $1-x$ равны 0.
\еза

%%%%%%%%%%%%%%%%%%%%%%%%%%%%%%%%%%%%%%%%%%%%%%%%

\section{Спектр Зариского для булева кольца}

%%%%%%%%%%%%%%%%%%%%%%%%%%%%%%%%%%%%%%%%%%%%%%%%

\определение
Пусть $R$ --- булево кольцо, а $\Spec(R)$ --
множество максимальных (или простых; в силу 
Замечания \ref{_prostye_maks_v_bul_Zamechanie_},
все простые идеалы в $R$ максимальны) идеалов 
в $R$, снабженное {\бф топологией Зариского}.
Напомним, что база открытых множеств в топологии
Зариского состоит из множеств 
\[
  A_f:= \{{\goth m}\in \Spec(R) \ | \ f \notin {\goth m}\},
\]
где  $f \in R$ --- какой-то элемент $R$. Пространство
$\Spec(R)$ называется {\бф спектром}, или
{\бф пространством Зариского} кольца $R$.
\ео

\замечание \label{_A_f_coprod_A_1-f_Zamechanie_}
Для любого $f\in R$, имеем
$\Spec(R)= A_f \bigsqcup A_{1-f}$.
Действительно,  $f\notin {\goth m}$
равносильно $(1-f)\in {\goth m}$, потому что
$R/{\goth m}= {\Bbb F}_2$.
\еза

\замечание
Из этого немедленно вытекает, что
все множества $A_f$ открытозамкнуты, и
значит, $\Spec(R)$ вполне несвязно
(докажите).
\еза

\замечание
Если $x\neq y\in \Spec(R)$ --- максимальные идеалы,
найдется $f \in x$, $f\notin y$. В этом случае,
$A_f$, $A_{1-f}$ --- непересекающиеся окрестности
$x$ и $y$. Мы получили, что $\Spec(R)$
хаусдорфово.
\еза 


\лемма
Пусть $R$ --- кольцо, а $\Spec(R)$ --
его пространство Зариского. Тогда $\Spec(R)$
компактно.

\хфилл

\ноиндент
{\бф Доказательство:} Пусть $M = \bigcup_\alpha A_{f_\alpha}$ --- покрытие
$M=\Spec(R)$. Для доказательства Леммы достаточно убедиться,
что в $\bigcup_\alpha A_{f_\alpha}$ найдется конечное подпокрытие
(проверьте). 
Обозначим $V_{f}:= M \backslash A_{f}$.
Это множество состоит из всех идеалов, содержащих $f$.
Пусть $I$ --- идеал, порожденный всеми $f_\alpha$.
Очевидно,
\[
M = \bigcup_\alpha A_{f_\alpha}  \ \  \Leftrightarrow \ \ \emptyset 
= \bigcap_\alpha V_{f_\alpha}.
\]
 Поэтому никакой максимальный 
идеал не содержит всех $f_\alpha$. Значит, $I$ не содержится
в максимальном идеале, и поэтому $1\in I$. Из этого следует, что
$1$ выражается в виде линейной комбинации конечного
числа $f_\alpha$:
\[
1 = \sum_{i=1}^n \lambda_i f_i, \ \lambda_i \in R.
\]
Мы получаем, что 
\[ \emptyset = \bigcap_{i=1}^n V_{f_i},\]
а поэтому $M=\bigcup_{i=1}^n A_{f_i}$.
Мы доказали, что $M=\Spec(R)$ компактно. \endproof

\hfill

\замечание
Мы доказали, что $\Spec(R)$ для любого булева кольца --
хаусдорфово, компактное, вполне несвязное топологическое
пространство. Оно называется {\бф пространство Стоуна}
булева кольца. 
\еза

\замечание
Соответствие \[ R \arrow \Spec(R)\] задает контравариантный
функтор из категории булевых колец в категорию
пространств Стоуна\footnote{Напомним, что пространства Стоуна --- хаусдорфовы, компактные, вполне несвязные топологические
пространства.} (докажите это).
\еза

Оказывается, что каждое пространство Стоуна $M$ можно
получить таким образом. Из Утверждения
\ref{_Spec(R)_gomeo_M_Utverzhdenie_},
доказанного ниже, следует, что $M$ гомеоморфно спектру
кольца $C(M, {\Bbb F}_2)$ непрерывных функций на $M$ 
со значениями в ${\Bbb F}_2 = \{0,1\}$.

\хфилл

%%%%%%%%%%%%%%%%%%%%%%%%%%%%%%%%%%%%%%%%%%%%%%%%
\лемма\label{_obshchij_nul_Lemma_}
Пусть $M$ --- пространство Стоуна, а $R= C(M, {\Bbb F}_2)$. 
Пусть ${\goth m}\subsetneqq R$ --- некоторый идеал.
Тогда все функции $f\in {\goth m}$ имеют общий 
нуль в $M$ (точку, где все эти функции зануляются). 


\хфилл

\ноиндент
{\бф Доказательство.} {\бф Шаг 1.} 
 Если у ${\goth m}$ нет общего нуля, то 
\[ M = \bigcup_{f\in {\goth m}}f^{-1}(1).\]
Мы получили открытое покрытие $M$. Поскольку
$M$ компактно, из него можно выбрать конечное подпокрытие.
Поэтому есть конечный набор $f_1, ..., f_n \in {\goth m}$
такой, что $M = \bigcup_{i=1}^n f_i^{-1}(1)$.


\хфилл

\ноиндент
{\бф Шаг 2.} 
Каждый элемент $f\in R$ имеет вид $\chi_U$, где $U = f^{-1}(1)$,
а $\chi_U$ --- характеристическая функция $U$. 
Поэтому есть такой набор открытых множеств $U_i$
с $\chi_{U_i}\in {\goth m}$, что $M = \bigcup_{i=1}^n U_i$.


\хфилл

\ноиндент
{\бф Шаг 3.} $\chi_{U\cup V} = \chi_U + \chi_V + \chi_U
\chi_V$. Пусть теперь $W = \bigcup_{i=1}^n U_i$.
Воспользовавшись индукцией, получим, что $\chi_{W} =
P(\chi_{U_1}, \chi_{U_2}, ... )$, где $P$ --- полином 
без свободного члена.


\хфилл

\ноиндент
{\бф Шаг 4.} Мы получили, что 
$1 = \chi_M = P(\chi_{U_1}, \chi_{U_2}, ... ) = P(f_1, f_2, ...)$.
Поэтому 1 лежит в идеале ${\goth m}$. Противоречие!
Значит, у ${\goth m}$ есть общий нуль.
\endproof


\хфилл

\лемма\label{_obshchij_nul_edinstvennyj_Lemma_}
Пусть $M$ --- пространство Стоуна, а $R= C(M, {\Bbb F}_2)$
-- кольцо непрерывных функций на $M$, со значениями в
${\Bbb F}_2 = \{0,1\}$. Пусть ${\goth m}\subset R$ --- максимальный идеал.
Тогда у функций $f\in {\goth m}$ есть единственный общий
нуль (точка, где они все зануляются).

\hfill

\ноиндент
{\бф Доказательство:} 
Существование общего нуля следует из
Леммы \ref{_obshchij_nul_Lemma_}.
Пусть  есть две несовпадающие точки $x_1\neq x_2 \in M$ такие,
что все $f\in {\goth m}$ зануляются в $x_1$ и $x_2$.
Поскольку непрерывные функции
на $M$ разделяют точки, гомоморфизм
\[
R \stackrel{\psi}\arrow {\Bbb F}_2 \oplus {\Bbb F}_2,\ \ \psi(f) = (f(x_1), f(x_2))
\]
сюръективен (докажите).
С другой стороны, $\psi({\goth m})=0$, поскольку
все элементы ${\goth m}$ зануляются в $x_1$, $x_2$.
Это дает сюръективный гомоморфизм 
$\psi:\; R/{\goth m}\arrow {\Bbb F}_2 \oplus {\Bbb F}_2$.
Но поскольку $R/{\goth m}={\Bbb F}_2$, это невозможно.
Противоречие! Мы доказали единственность общего нуля
максимального идеала. \endproof

\замечание
 Лемма \ref{_obshchij_nul_edinstvennyj_Lemma_}
 дает биекцию $\Spec(R) \arrow M$, где
$R=C(M, {\Bbb F}_2)$.
Докажем, что эта биекция --- гомеоморфизм.
База топологии на $M$ задается открытозамкнутыми
множествами, то есть множествами вида 
\[ 
 A_f=\{x \in M \ |\ f(x) =1\}.
\]
Те же самые $A_f$ задают базу открытых 
подмножеств в $\Spec(R)$.
\еза

Мы доказали такое утверждение

\хфилл

%%%%%%%%%%%%%%%%%%%%%%%%%%%%%%%%%%%%%%%%%%%%%%%%%%%%%%%%%%%%
\утверждение\label{_Spec(R)_gomeo_M_Utverzhdenie_}
Пусть $M$ --- пространство Стоуна, а $R= C(M, {\Bbb F}_2)$
-- кольцо непрерывных функций на $M$, со значениями в
${\Bbb F}_2 = \{0,1\}$. Тогда $M$ гомеоморфно $\Spec(R)$.
\endproof

\замечание
Аналогичный, но более простой, аргумент доказывает, что
кольцо непрерывных, ${\Bbb F}_2$-значных  функций на $\Spec(R)$ 
изоморфно $R$. Действительно, отображение
$(f, {\goth m}) \arrow f/{\goth m} \in R/{\goth m}= {\Bbb F}_2$
задает гомоморфизм колец $R \stackrel \Psi \arrow C(\Spec(R), {\Bbb F}_2)$.
Поскольку каждая непрерывная функция содержится 
в идеале, $\Psi$ это вложение (докажите).
Непрерывные функции на $\Spec(R)$ порождены
характеристическими функциями открытых
подмножеств $A_f$ (докажите), а все такие
функции получаются из $R$ по формуле
\[
\chi_{A_f}= \Psi(f).
\] 
\еза


\begin{figure}[ht]
\begin{center}
\epsfig{file=Stone.eps,width=0.4\linewidth}\\
{Marshall Harvey Stone\\
(1903 --- 1989)}
\end{center}
\end{figure}


Получается, что функторы $M \arrow C(M, {\Bbb F}_2)$
и $R \arrow \Spec(R)$ взаимно обратны. Мы доказали
следующую теорему.

\хфилл

%%%%%%%%%%%%%%%%%%%%%%%%%%%%%%%%%%%%%%%%%%%%%%%%%%%%%%%%%%%%
\теорема
(теорема Стоуна о представимости булевых алгебр)
Функтор $R \arrow \Spec(R)$ задает 
эквивалентность между категорией булевых колец и двойственной 
категорией к категории хаусдорфовых, компактных, вполне несвязных топологических
пространств. \endproof

\замечание 
Булевы алгебры и булевы кольца тоже эквивалентны
(Теорема \ref{_bulevy_algebra_kolca_ekvi_Teorema_}).
\еза

%%%%%%%%%%%%%%%%%%%%%%%%%%%%%%%%%%%%%%%%%%%%%%%%%%%%%%%%%%%%

\section{Булевы алгебры: история, замечания}

%%%%%%%%%%%%%%%%%%%%%%%%%%%%%%%%%%%%%%%%%%%%%%%%%%%%%%%%%%%%

Законы логики, которые лежат в основе 
определения булевой алгебры, сформулированы английским математиками
Джорджем Булем (George Boole) в 1847-м и Августом де Морганом 
(August de Morgan) в 1860-м. Аксиоматическое определение
булевых алгебр принадлежит Хантингтону (Edward Vermilye
Huntington, 1904), но серьезно изучать булевы алгебры
стали только после фундаментальных работ Маршалла Стоуна
(1930-е). В 1960-е годы булевы алгебры нашли широкое применение
в логике, где с их помощью доказывается невыводимость
разных теоретико-множественных гипотез, таких, как
аксиома выбора и континуум-гипотеза.


Теорема Маршалла Стоуна буквального обобщения на
более общие (не булевы) кольца не имеет. Но в основании
алгебраической геометрии лежит весьма похожий аргумент, 
принадлежащий Гротендику, который
позволяет интерпретировать алгебраические объекты (кольца)
как геометрические (пространства Зариского) и наоборот.

%%%%%%%%%%%%%%%%%%%%%%%%%%%%%%%%%%%%%%%%%%%%%%%%%%%%%%%%%%%%%%%%%%%%%%%%

\chapter{Лекция 15: Фундаментальная группа}

%%%%%%%%%%%%%%%%%%%%%%%%%%%%%%%%%%%%%%%%%%%%%%%%%%%%%%%%%%%%%%%%%%%%%%%%


%%%%%%%%%%%%%%%%%%%%%%%%%%%%%%%%%%%%%%%%%%%%%%%%%%%%%%%%%%%%

\section{Гомотопные отображения}

%%%%%%%%%%%%%%%%%%%%%%%%%%%%%%%%%%%%%%%%%%%%%%%%%%%%%%%%%%%%

\определение
Пусть $f_0, f_1:\; X \arrow Y$ --- непрерывные отображения.
{\бф Гомотопией} $f_0$ в $f_1$ называется непрерывное
отображение $f:\; X \times [0,1] \arrow Y$ такое,
что $f\restrict{X \times \{0\}}=f_0$, $f\restrict{X \times \{1\}}=f_1$.
\ео

\замечание
Гомотопные отображения --- отображения, которые можно непрерывно
продеформировать одно в другое. Иногда говорят 
"гомотопия $f_0$ в $f_1$", а иногда "гомотопия $f_0$ 
{\em к} $f_1$"; эти лексические формы эквивалентны.
\еза

\замечание
Пусть $f_0$, $f_1$ принадлежат какому-то выделенному
классу отображений, например, к классу $C$-липшицевых
отображений из метрического пространства $X$ в метрическое
пространство $Y$. Говорится, что $f:\; X \times [0,1] \arrow Y$
{\бф гомотопия в классе ${\cal A}$}, если для любого $t$,
отображение $f_t:=f\restrict{X \times \{t\}}:\; X \arrow Y$
принадлежит классу ${\cal A}$. 
\еза

\задача
Пусть $f, g:\; \R^n \arrow \R$ --- гладкие функции.
Докажите, что $f$ и $g$ гомотопны в классе гладких
отображений.
\ез

Следующее утверждение вполне очевидно.

\хфилл

%%%%%%%%%%%%%%%%%%%%%%%%%%%%%%%%%%%%%%%%%%%%%%%%
\утверждение\label{_composi_gomo_Utverzhdenie_}
Пусть $f_0, f_1:\; X \arrow Y$ --- гомотопные отображения,
а $g:\; P\arrow X$, $h:\; Y \arrow Z$ --- непрерывные
отображения. Тогда $f_0 \circ h$ гомотопно $f_1 \circ h$,
а $g \circ f_0$ гомотопно $g \circ f_1$.

\хфилл

\ноиндент
{\бф Доказательство:} 
Пусть $f:\; X \times [0,1] \arrow Y$ --- гомотопия $f_0$ в $f_1$.
Гомотопию $f_0 \circ h$ в $f_1 \circ h$
строим как композицию $f\circ h:\; X \times [0,1] \arrow Z$
(проверьте, что это гомотопия). Гомотопию 
$g \circ f_0$ в $g \circ f_1$ получим, взяв композицию
$g\times \Id_{[0,1]}:\; P \times [0,1] \arrow X \times [0,1]$
и $f:\; X \times [0,1] \arrow Y$.
\endproof

\хфилл

Следующее утверждение также тривиально, но весьма полезно.

\хфилл

%%%%%%%%%%%%%%%%%%%%%%%%%%%%%%%%%%%%%%%%%%%%%%%%
\утверждение\label{_transiti_gomo_Utverzhdenie_}
Пусть $f_0, f_1, f_2:\; X \arrow Y$ --- непрерывные
отображения, причем $f_0$ гомотопно $f_1$, а $f_1$
гомотопно $f_2$. Тогда $f_0$ гомотопно $f_2$.


\хфилл

\ноиндент
{\бф Доказательство:}
Пусть
$\tilde f:\; X \times [0,1] \arrow Y$ --- гомотопия $f_0$ в $f_1$,
а $\tilde{\tilde f}:\; X \times [1,2] \arrow Y$ --- гомотопия $f_1$ в $f_2$.
Рассмотрим отображение $f:\; X \times [0,1] \arrow Y$
\begin{equation}\label{_skle_gomoto_Equation_}
f(x, \lambda) = \begin{cases} 
\tilde f(2\lambda), & \ \ \text{если} \ \ \lambda\leq \frac 1 2\\
\tilde{\tilde f}(2\lambda), &\ \ \text{если} \ \ \lambda\geq \frac 1 2\\
\end{cases}
\end{equation}
Если $f$ непрерывно, оно, очевидно, является гомотопией
$f_0$ в $f_2$, поэтому для доказательства Утверждения
\ref{_transiti_gomo_Utverzhdenie_}, достаточно
убедиться, что $f$ непрерывно.

Легко убедиться, что отображение $\phi:\; A \arrow B$ 
непрерывно, если для каждого $Z\subset A$, образ
$\phi(\bar Z)$ замыкания $Z$ лежит в замыкании $\overline{\phi(Z)}$
(проверьте это). 

Пусть $Z \subset X \times [0,1]$ --- некоторое подмножество,
причем 
\[ Z_1= Z \cap  X \times [0,1/2]\ \ \text{и} \ \  
Z_2= Z \cap  X \times [1/2,1]. 
\]
Легко видеть, что
$\bar Z = \bar Z_1 \cup \bar Z_2$, соответственно
\[
f(\bar Z ) \subset f(\bar Z_1) \cup f (\bar Z_2) 
\subset \overline{f (Z_1)} \cup \overline{f(Z_2)}
\]
(последнее включение следует из того, что ограничения
$f\restrict {X \times [0,1/2]}$\\ и $f\restrict {X \times [1/2,1]}$
непрерывны). Поэтому $f$ непрерывно. Утверждение
\ref{_transiti_gomo_Utverzhdenie_} доказано.
\endproof


\замечание
Предыдущее утверждение означает, что гомотопии
можно "склеивать" между собой: приклеив
гомотопию из $f_0$ в $f_1$ к гомотопии $f_1$ в $f_2$,
мы получим гомотопию $f_0$ в $f_2$. Первое отображение
непрерывно деформируется во второе, второе в третье,
и взяв эти две деформации одну за другой, мы получаем,
что первое можно непрерывно продеформировать в третье.
\еза


\замечание
Из Утверждения
\ref{_transiti_gomo_Utverzhdenie_}
следует, что отношение ``$f$ гомотопно $g$" транзитивно.
Оно также рефлексивно и симметрично (проверьте).
Множество классов эквивалентности отображений
$f:\; X \arrow Y$  с точностью
до гомотопии
называется {\бф множество классов гомотопической
эквивалентности отображений}, или же {\бф множество
гомотопических классов отображений}.
\еза

\задача
Пусть $\Ob(\cac)$ --- класс всех топологических
пространств, а $\Mor(X,Y)$ --- множество гомотопических
классов непрерывных отображений из $X$ в $Y$. Докажите,
что таким образом получается категория. Эта категория
называется {\бф гомотопическая категория}, а изоморфизмы
в ней --- {\бф гомотопическими эквивалетностями}.
\ез



В дальнейшем нам понадобится следующая
тривиальная лемма.


\hfill

%%%%%%%%%%%%%%%%%%%%%%%%%%%%%%%%%%%%%%%%%%%%%%%%
\лемма \label{_homo_fu_Lemma_}
Пусть $f_0, f_1$ непрерывные отображения
из отрезка $[a, b]$ в отрезок $[c, d]$, причем
$f_i(a) = a'$, $f_i(b)=b'$. Тогда $f_0$ гомотопно
$f_1$ в классе отображений, переводящих $a$ в $a'$,
$b$ в $b'$.

\хфилл

Следующее отображение осуществляет искомую гомотопию
\begin{align*}f:\; [a, b]\times [0,1]  \arrow [c,d],\ \ \ 
f(x, t) = t f_1(x) + (1-t) f_0(x)
\end{align*}
(проверьте это).


%%%%%%%%%%%%%%%%%%%%%%%%%%%%%%%%%%%%%%%%%%%%%%%%%%%%%%%%%%%%

\section{Категория пространств с отмеченной точкой 
и пространства петель}

%%%%%%%%%%%%%%%%%%%%%%%%%%%%%%%%%%%%%%%%%%%%%%%%%%%%%%%%%%%%

\определение
Пара
\[ \bigg(\text{топологическое пространство $M$, точка
$m\in M$}\bigg)
\]
называется {\бф пространством с отмеченной точкой}
(pointed space), обозначается
$(M,m)$. Пространство с отмеченной точкой еще называют
{\бф пунктированное}, или же {\бф отмеченное} пространство.
\ео

\определение
{\бф Непрерывное отображение пунктированных \\ пространств}
$\phi:\; (X,x) \arrow (Y,y)$  --- непрерывное отображение
из $X$ в $Y$, которое переводит $x$ в $y$. 
Во избежание путаницы, непрерывные отображения
пунктированных пространств называются {\бф морфизмами
пунктированных пространств}.
\ео


Напомним, что {\бф категорией} $\cac$ называется набор
объектов $\Ob(\cac)$, таких, что для каждой пары
$X, Y \in \Ob(\cac)$ задано множество $\Mor(X,Y)$
{\бф морфизмов из $X$ в $Y$}. На морфизмах задано
отображение композиции
\[
\Mor(X,Y)\times \Mor(Y,Z) \arrow \Mor(X,Z),
\]
которое ассоциативно. В каждом множестве
$\Mor(X,X)$ выделен {\бф тождественный морфизм}
$\Id_X$, причем композиция любого морфизма $\phi$
с тождественным равна $\phi$

\замечание
Легко видеть, что {\ем пространства 
с отмеченной точкой образуют категорию:} композиция
морфизмов --- снова морфизм, композиция очевидно
ассоциативнa, а тождественное
отображение \[ \Id_{(X,x)}:\; (X,x) \arrow (X,x)\]
является морфизмом пунктированных пространств.
\еза

\определение
Пусть $f_0, f_1:\; (X,x) \arrow (Y,y)$ --
морфизмы пунктированных пространств.
{\бф Гомотопией морфизма $f_0$ к $f_1$} называется
отображение $f:\; X \times [0,1] \arrow Y$
такое, что $f(x, t) =y$ для любого $t\in [0,1]$.
\ео


\определение
Пусть $M$ --- топологическое пространство.
Напомним, что {\бф путем из $x\in M$ в $y\in M$}
называется непрерывное отображение 
$\gamma:\; [a,b] \arrow M$
из отрезка $[a,b]$ в $M$ такое, что
$\gamma(a) =x$, $\gamma(b)=y$.
\ео

\определение
Пусть $(M,m)$ --- 
топологическое пространство с отмеченной точкой.
Рассмотрим множество путей $\gamma:\; [0,1] \arrow M$
из $m$ в $m$. Такие пути называются
{\бф петлями в $M$}. Множество всех
петель обозначается $\Omega(M,m)$.

Если $M$ --- метрическое пространство,
на $\Omega(M,m)$ вводится $\sup$-\-ме\-т\-ри\-ка,
заданная формулой
\[
d(\gamma, \gamma') = \sup_{t\in [0,1]} d(\gamma(t), \gamma'(t)).
\] 
Пространство $\Omega(M,m)$ называется
{\бф пространством петель} для $M$.
Если $M$ метризуемо и локально компактно,
топология на $\Omega(M,m)$ не зависит 
от выбора метрики на $M$ (Лекция 11). В такой ситуации
мы рассматриваем пространство петель 
$\Omega(M,m)$ как топологическое пространство.
\ео

\замечание
Рассмотрим окружность $S^1$ с отмеченной точкой.
Легко видеть, что любой путь $\gamma:\; [0,1] \arrow M$ из $m$ в $m$
задает морфизм $(S^1, 0) \arrow (M,m)$, и это соответствие
взаимно однозначно. В дальнейшем, мы будем рассматривать
$\Omega(M,m)$ как множество морфизмов отмеченных пространств
$(S^1, 0) \arrow (M,m)$.
\еза

\замечание
Терминология теории категорий в применении к 
пространствам с отмеченной точкой 
может показаться избыточной. Это не так:
язык теории категорий существенно экономит время и
позволяет избежать двусмысленностей. Если вы испытываете
дискомфорт от те\-о\-ре\-ти\-ко-\-ка\-те\-гор\-но\-го
языка, всякий  раз заменяйте
"морфизм отмеченных пространств" на "непрерывное
отображение, переводящее отмеченную точку
в отмеченную точку", и постижение топологии
будет менее болезненным.
\еза

%%%%%%%%%%%%%%%%%%%%%%%%%%%%%%%%%%%%%%%%%%%%%%%%

\section{Фундаментальная группа}

%%%%%%%%%%%%%%%%%%%%%%%%%%%%%%%%%%%%%%%%%%%%%%%%

Пусть $(M,m)$ --- пунктированное топологическое пространство.
В силу Утверждения \ref{_transiti_gomo_Utverzhdenie_},
гомотопия задает отношение эквивалентности на 
множестве всех морфизмов $(S^1, 0) \arrow (M,m)$,
или, что то же самое, путей из $m$ в $m$.
Множество классов гомотопической эквивалентности 
путей из $m$ в $m$ обозначается $\pi_1(M,m)$.

Если $\gamma, \gamma':\; [0,1] \arrow M$ --
два пути из $m$ в $m$, обозначим через
$\widetilde{\gamma\gamma'}:\; [0,2]\arrow M$
путь, который получен по формуле
\[
\widetilde{\gamma\gamma'}(\lambda) = \begin{cases} 
\gamma(\lambda), & \ \ \text{если} \ \ \lambda\leq 1\\
\gamma'(\lambda-1), &\ \ \text{если} \ \ \lambda\geq 1.
\end{cases}
\]
Непрерывность этого пути доказывается тем же самым
аргументом, который использовался при доказательстве
Утверждения \ref{_transiti_gomo_Utverzhdenie_}
(отображение $f:\; X \arrow Y$ непрерывно, если замыкание 
образа любого $Z\subset X$ содержит образ замыкания $Z$).

Определим произведение $\gamma\gamma'$ формулой
$\gamma\gamma'(\lambda) = \widetilde{\gamma\gamma'}(2\lambda)$.

Следующая лемма утверждает, что это произведение
переводит гомотопные пути в гомотопные. Эта лемма
 тривиальна, и интуитивно очевидна.
В самом деле, произведение путей --- это обход $M$ вдоль
одного пути, потом вдоль другого; если мы непрерывно 
продеформируем первый путь и  второй, то произведение 
путей тоже продеформируется, а значит, будет
гомотопно исходному произведению.

\хфилл

%%%%%%%%%%%%%%%%%%%%%%%%%%%%%%%%%%%%%%%%%%%%%%%%
\лемма
Определенная выше операция на $\Omega(M,m)$
переводит гомотопные пути в гомотопные: если
$\gamma_0$ гомотопен $\gamma_1$, а 
$\gamma_0'$ гомотопен $\gamma_1'$, то
$\gamma_0\gamma_0'$ гомотопен $\gamma_1\gamma_1'$.

\хфилл

{\бф Доказательство:}
Пусть
\[ \gamma:\; [0,1]\times [0,1] \arrow M, \ \ 
\gamma':\; [0,1]\times [0,1] \arrow M\] 
-- гомотопии,
соединяющие $\gamma_0$ и $\gamma_1$, а также
$\gamma_0'$ и $\gamma_1'$. Гомотопии путей это отображения
из квадрата в $M$, а то, что концы пути остаются в $m$,
означает, что две стороны квадрата идут в $m$
(остальные две стороны идут в $\gamma_0$, $\gamma_1$
для первого квадрата, $\gamma_0'$, $\gamma_1'$ для второго).
Рассмотрим гомотопию между $\widetilde{\gamma_0\gamma_0'}$
и $\widetilde{\gamma_1\gamma_1'}$, полученную по формуле
\[
\widetilde{\gamma\gamma'}(\lambda, t) = \begin{cases} 
\gamma(\lambda,t), & \ \ \text{если} \ \ \lambda\leq 1\\
\gamma'(\lambda-1,t), &\ \ \text{если} \ \ \lambda\geq 1.
\end{cases}
\]
Вышеприведенный аргумент
(отображение $f:\; X \arrow Y$ непрерывно, если замыкание 
образа любого $Z\subset X$ содержит образ замыкания $Z$)
доказывает, что это отображение непрерывно, значит,
осуществляет гомотопию $\widetilde{\gamma_0\gamma_0'}$
и $\widetilde{\gamma_1\gamma_1'}$. Гомотопия 
$\gamma_0\gamma_0'$ и $\gamma_1\gamma_1'$
получается из этого, поскольку
$\gamma_0\gamma_0'$ и $\gamma_1\gamma_1'$
получены из $\widetilde{\gamma_0\gamma_0'}$
и $\widetilde{\gamma_1\gamma_1'}$ репараметризацией.
\endproof

\хфилл

Из этой леммы следует, что произведение
$\gamma, \gamma' \arrow \gamma\gamma'$ --
корректно определенная операция на множестве
$\pi_1(M,m)$ классов гомотопической 
эквивалентности путей из $m$ в $m$.

\хфилл

%%%%%%%%%%%%%%%%%%%%%%%%%%%%%%%%%%%%%%%%%%%%%%%%
\теорема
Определенная выше операция 
\[ \pi_1(M,m) \times \pi_1(M,m)\arrow \pi_1(M,m)\]
задает структуру группы на $\pi_1(M)$.

\хфилл

\ноиндент
{\бф Доказательство:} Нужно проверить выполнение групповых
аксиом. Роль единицы $\iota :\; [0,1]\arrow M$ 
играет отображение $[0,1]$ в точку $m\in M$. Эта петля
называется {\бф тривиальной}.
Обозначим через $\phi_0:\; [0,1]\arrow [0,1]$
отображение $t \arrow \max(0, 2t-1)$, за 
$\phi_1:\; [0,1]\arrow [0,1]$ отображение
$t \arrow \min(1, 2t)$. 



\begin{center}
\begin{minipage}{.45\linewidth}
\begin{center}\epsfig{file=phi_0.eps,width=0.9\linewidth}\\ 
{\small График функции $\phi_0(t)$} \end{center}\end{minipage}
\begin{minipage}{.45\linewidth}
\begin{center}\epsfig{file=phi_1.eps,width=0.9\linewidth}\\ 
{\small График функции $\phi_1(t)$} \end{center}\end{minipage}
\end{center}

% max(x,y)=x>y?x:y
% min(x,y)=x<y?x:y
% f(x) = max(0, 2*x-1)
% set terminal postscript eps lw 3 butt "Helvetica" 22
% set output "phi_0.eps"
% set size square
% plot [0:1] [0:1] f(x)
% f(x) = min(1, 2*x)
% set output "phi_1.eps"

Легко видеть, что $\gamma\iota= \phi_0 \circ\gamma$, а
$\iota\gamma= \phi_1 \circ\gamma$. В силу Леммы \ref{_homo_fu_Lemma_},
функции $\phi_1$ и $\phi_2$ гомотопны друг другу и тождественному
отображению $x\arrow x$ в классе функций, сохраняющих
$0$ и $1$. Kомпозиция уважает гомотопию
(Утверждениe \ref{_composi_gomo_Utverzhdenie_}), следовательно,
путь $\gamma\iota$ гомотопен $\iota\gamma$ и гомотопен
$\gamma$.

Обратный элемент $\gamma^{-1}$ к пути $\gamma$
строится так: $\gamma^{-1}(t) = \gamma(1-t)$.
Рассмотрим функцию $\Phi:\; [0,1]\arrow [0,1]$,
$\Phi(t)= \min(2t, 2-2t)$,

% f(x) = min(2*x, 2-2*x)
% set output "Phi.eps"


\begin{center}\epsfig{file=Phi.eps,width=0.4\linewidth}\\ 
{\small График функции $\Phi(t)$} \end{center}


Легко видеть, что $\gamma\gamma^{-1}=\Phi\circ\gamma$.
Поскольку $\Phi$ гомотопен отображению $t\arrow 0$
в классе функций, переводящих 0 и 1 в 0 (Лемма \ref{_homo_fu_Lemma_}), 
путь $\gamma\gamma^{-1}=\Phi\circ\gamma$ гомотопен тривиальному.

Эту гомотопию можно увидеть  из следующей картинки.

\begin{center}\epsfig{file=styagivaem.eps,width=0.95\linewidth}\\ 
{\small Стягивание петли $\gamma\gamma^{-1}$ в точку} \end{center}

Наконец, ассоциативность умножения видна из следующего
рассуждения. Пусть $\gamma_0$, $\gamma_1$, $\gamma_2$ три
пути из $m$ в $m$, а $\widetilde{\gamma_0\gamma_1\gamma_2}:\; [0,3]\arrow M$ --
петля, определенная формулой
\[
\widetilde{\gamma_0\gamma_1\gamma_2}(\lambda) = \begin{cases} 
\gamma_0(\lambda), & \ \ \text{если} \ \ 0\leq\lambda\leq 1\\
\gamma_1(\lambda-1), &\ \ \text{если} \ \ 1\leq\lambda\leq 2,\\
\gamma_2(\lambda-2), &\ \ \text{если} \ \ 2\leq\lambda\leq 3.
\end{cases}
\]
Рассмотрим функции $\psi_1, \psi_2:\; [0,1]\arrow [0,3]$
заданные формулами
\[
\psi_1(t) = \max\left(2 t, 4 \left(t-1/4\right)\right), 
\ \ \psi_2(t) = \min\left(4 t, 2\left(t+1/2 \right)\right).
\]
Функции $\psi_1$ и $\psi_2$ --- кусочно-линейные,
их графики составлены из двух прямолинейных отрезков. У $\psi_1$ 
первый отрезок соединяет (0,0) и (1/2,1), второй отрезок
соединяет (1/2,1) и (1,3). У $\psi_2$ первый отрезок 
соединяет (0,0) и (1/2,2), второй отрезок соединяет
 (1/2,2) и (1,3).

% f(x) = max(2*x, 4*(x-0.25))
% set grid
% plot [0:1] [0:3] f(x) title ""
% set size ratio 3 1,2
% set output "psi_1.eps"
% f(x) = min(4*x, 2*(x+0.5))


\begin{center}
\begin{minipage}{.45\linewidth}
\begin{center}\epsfig{file=psi_1.eps,width=0.65\linewidth}\\ 
{\small График функции $\psi_1(t)$} \end{center}\end{minipage}
\begin{minipage}{.45\linewidth}
\begin{center}\epsfig{file=psi_2.eps,width=0.65\linewidth}\\ 
{\small График функции $\phi_2(t)$} \end{center}\end{minipage}
\end{center}

Легко видеть, что 
$\gamma_0(\gamma_1\gamma_2)=
\psi_1\circ
\widetilde{\gamma_0\gamma_1\gamma_2}(\lambda)$,
и $(\gamma_0\gamma_1)\gamma_2=
\psi_2\circ
\widetilde{\gamma_0\gamma_1\gamma_2}(\lambda)$
(проверьте). Поскольку $\psi_1$ и $\psi_2$ гомотопны,
из этого следует, что гомотопны петли
$\gamma_0(\gamma_1\gamma_2)$ и
$(\gamma_0\gamma_1)\gamma_2$.
Мы доказали, что $\pi_1(M, m)$ это группа.
\endproof

\определение
Группа $\pi_1(M, m)$, определенная выше, называется
{\бф фундаментальной группой} топологического
пространства $(М, m)$.
\ео

\замечание
Пусть \[ (X,x)\stackrel f\arrow (Y,y)\] --- морфизм пунктированных
пространств. Для любого пути $\gamma\in \Omega(X,x)$,
композиция $\gamma\circ f$ задает путь в $(Y,y)$.
Пользуясь тем. что композиции $f$ с гомотопными отображениями
гомотопны (Утверждение \ref{_composi_gomo_Utverzhdenie_}),
мы получаем, что $\gamma \arrow \gamma\circ f$ определяет
отображение на классах эквивалентности петель
$\pi_1(X,x) \arrow \pi_1(Y,y)$. 
Из конструкции этого отображения сразу ясно,
что это гомоморфизм групп. Отображение
фундаментальных групп, индуцированное $f$, часто
обозначается той же самой буквой.
\еза

\замечание
Напомним, что {\бф функтор} из категории $\cac_1$ в
$\cac_2$ задается отображением \[ F:\; \Ob(\cac_1) \arrow \Ob(\cac_2)\]
и отображениями  \[ F:\; \Mor(X,Y) \arrow \Mor(F(X), F(Y)),\]
определенными для любых $X, Y\in \Ob(\cac_1)$,
переводящими $\Id_X$ в $\Id_{F(X)}$ и 
совместимыми с взятием композиций.
\еза

\замечание
Взятие фундаментальной группы определяет
функтор из категории пунктированных пространств
в категорию групп (проверьте это).
\еза


\замечание
Пусть $f_0, f_1:\; (X,x) \arrow (Y,y)$ --- гомотопные отображения.
Тогда соответствующие гомоморфизмы фундаментальных
групп совпадают. В самом деле, если $f_0$ гомотопно $f_1$,
то $\gamma\circ f_0$ гомотопно $\gamma\circ f_1$, для
любой петли $\gamma\in \Omega(X,x)$.
\еза



%%%%%%%%%%%%%%%%%%%%%%%%%%%%%%%%%%%%%%%%%%%%%%%%

\section{Стягиваемые пространства, ретракты,
гомотопическая  эквивалентность}

%%%%%%%%%%%%%%%%%%%%%%%%%%%%%%%%%%%%%%%%%%%%%%%%

\определение
Пусть $(X,x)$ --- пунктированное топологическое
пространство. Рассмотрим отображение
$\iota_x:\; X \arrow X$, переводящее все точки
$X$ в $x$. Пространство $X$ называется
{\бф стягиваемым}, если $\iota_x$ гомотопно
тождественному отображению $\Id_X$.
\ео

\пример
Пусть $(X,x)$ --- подмножество $\R^n$.
Это подмножество называется {\бф звездчатым}, если
для любой точки $x'\in X$, отрезок
$[x,x']$ целиком лежит в $X$. 
Рассмотрим отображение
\[
X \times [0,1] \stackrel \Psi \arrow X,
\]
переводящее $(x',t)$ в $x-t(x-x')$.
Легко видеть, что $\Psi$ это гомотопия
$\Id_X$ и $\iota_x$ (докажите это).
Мы получили, что любое звездчатое подмножество
стягиваемо.

\замечание
Пусть $(X,x)$ --- стягиваемое пространство.
Легко видеть, что отображение
$\iota_x:\; X \arrow X$ задает
тривиальный гомоморфизм \[ \pi_1(X,x) \stackrel{\iota_x}\arrow \pi_1(X,x)\]
(все элементы группы $\pi_1(X,x)$ переходят в 1).
Поскольку отображение $\iota_x$ гомотопно тождественному,
а гомотопные отображения топологических
пространств индуцируют одинаковые отображения
фундаментальных групп, тождественное отображение
$\pi_1(X,x)\stackrel {\Id} \arrow \pi_1(X,x)$
равно отображению, переводящему все элементы в 1.
Поэтому $\pi_1(X,x) = \{1\}$.
\еза

\определение
Если $\pi_1(X,x)=\{1\}$, пунктированное пространство
$(X,x)$ называется {\бф односвязным}. В этой ситуации
еще говорят, что у $(X,x)$ {\бф нулевая фундаментальная группа}.
\ео

\определение
Пусть $X\subset Y$ --- 
подмножество. Предположим, что существует
отображение $Y \stackrel {\Psi_1}\arrow X$, тождественное 
на $X\subset Y$, и гомотопное тождественному отображению $\Id_Y$.
В  этом случае $X$ называется {\бф деформационным
ретрактом} $Y$. Гомотопия $\Psi_1$ и $\Id_Y$
называется {\бф деформационной ретракцией $Y$ на $X$}.
\ео

\пример
Пусть $Y$ --- множество всех ненулевых векторов в $\R^n$,
а $X \subset Y$ --- единичная сфера. Рассмотрим отображение
$Y\stackrel {\Psi_1}\arrow X$, $v \arrow \frac v{|v|}$. 
Отображение $\Psi_1$ гомотопно тождественному:
\[
\Psi(v, t) = \frac v{|v|^{t} }
\]
осуществляет гомотопию $\Psi_1$ и $\Psi_0=\Id_Y$.

\хфилл

\замечание
Отмеченное пространство $(X,x)$ является стягиваемым
тогда и только тогда, когда $\{x\}$ --- деформационный ретракт $X$.
\еза

\определение
Отображение $X \stackrel f \arrow Y$ называется {\бф гомотопической
эквивалентностью}, если существуют отображения\\
$g, g':\; Y \arrow X$ такие, что $g\circ f$ гомотопно
$\Id_Y$, а $f\circ g'$ гомотопно $\Id_X$.
Аналогичным образом определяется гомотопическая
эквивалентность пунктированных пространств. Морфизм
 $(X, x) \stackrel f \arrow (Y,y)$ -- гомотопическая
эквивалентность, если заданы морфизмы 
$g, g':\; (Y,y) \arrow (X,x)$, причем $g\circ f$ гомотопно
$\Id_Y$, а $f\circ g'$ гомотопно $\Id_X$
как морфизм пунктированных пространств.
\ео

\пример
Если $X \stackrel j\hookrightarrow Y$ --- деформационный ретракт, то
пространство $X$ гомотопически эквивалентно $Y$. Действительно,
естественное вложение $X \stackrel j\hookrightarrow Y$ 
в композиции с $Y \stackrel {\Psi_1}\arrow X$ 
дает $\Id_X$, а композиция $\Psi_1 \circ j$
гомотопна $\Id_Y$, по определению деформационного
ретракта. В частности, любое стягиваемое
пространство гомотопически эквивалентно точке.

\замечание
Пусть  $(X,x) \stackrel f \arrow (Y,y)$ --- гомотопическая
эквивалентность пунктированных пространств, 
а $\pi_1(X,x) \stackrel f \arrow \pi_1(Y,y)$
соответствующее отображение фундаментальных групп.
Тогда существуют морфизмы $g, g'$, такие, что $g\circ f$ гомотопно
$\Id_Y$, а $f\circ g'$ гомотопно $\Id_X$. Значит,
у $\pi_1(X,x) \stackrel f \arrow \pi_1(Y,y)$ нет ядра
(потому что на фундаментальных группах 
$f\circ g'= \Id_{\pi_1(X,x)}$).
С другой стороны, поскольку $g\circ f=\Id_{\pi_1(Y,x)}$,
отображение
$\pi_1(X,x) \stackrel f \arrow \pi_1(Y,y)$ сюръективно.
Мы доказали, что это изоморфизм групп.
\еза

\следствие
Пусть $(X,x)$ и $(Y,y)$ --- гомотопически эквивалентные
пунктированные пространства. Тогда $\pi_1(X) \cong \pi_1(Y)$.




%%%%%%%%%%%%%%%%%%%%%%%%%%%%%%%%%%%%%%%%%%%%%%%%

\subsection{История, замечания}

%%%%%%%%%%%%%%%%%%%%%%%%%%%%%%%%%%%%%%%%%%%%%%%%

Понятие гомотопии и фундаментальной группы было 
впервые строго определено Пуанкаре, в 1895-м году. 
Гомотопные пути и фундаментальную группу 
определил в 1866-м году Камиль Жордан; впрочем, 
Жордан в работах по топологии не пользовался 
понятием группы (которое он сам же и ввел,
тоже в 1860-х, следуя Галуа, в работах 
о разрешимости алгебраических уравнений).



\begin{figure}[ht]
\begin{center}
\epsfig{file=Jordan.eps,width=0.5\linewidth}\\
{Marie Ennemond Camille Jordan \\
(1838 --- 1922)}
\end{center}
\end{figure}



Фундаментальная группа есть первый (и самый простой)
алгебраический инвариант топологических пространств.
Существует немало функторов аналогичной природы:
гомотопические группы, когомологии, К-теория, кобордизмы 
и т.д.
Эти конструкции изучаются в алгебраической
топологии, методами теории категорий.

Строгое изучение топологии затруднено
патологичностью общих топологических пространств.
Алгебраические топологи по большей части ограничиваются
{\ем CW-комплексами}, пространствами, которые склеены 
из симплексов (многомерных тетраэдров)
наподобие многогранников. 

В большинстве курсов топологии, никакие другие
пространства, кроме CW-комплексов, не рассматриваются, 
или практически не рассматриваются. Ограничив
таким образом класс изучаемых пространств, можно перевести
на язык категорий любое геометрическое утверждение.

В той (достаточно общей) геометрической ситуации, с
которой имеют дело в метрической геометрии и анализе, 
считать алгебраические инварианты
пространств чрезвычайно затруднительно. По этой причине предмет
алгебраической топологии практически никак не соотносится
с предметом общей топологии. С другой стороны,
без понимания основ общей топологии заниматься
геометрией и топологией проблематично.

Основные достижения математики XX века относились
к алгебраической топологии и алгебраической геометрии;
на протяжении столетия эти две науки параллельно 
и чрезвычайно интенсивно развивались, используя
язык категорий и гомологической алгебры, специально
разработанный для такой цели. 


Каноническим учебником алгебраической топологии
является книжка Фукса, Фоменко, "Курс гомотопической
топологии". Для других целей полезен том Свитцера
"Алгебраическая топология: гомотопии и гомологии".
Теории категорий посвящен учебник Маклейна
"Категории для работающего математика". Гомологическая алгебра
изложена в прекрасной книге "Методы гомологической алгебры"
Гельфанда-Манина.




%%%%%%%%%%%%%%%%%%%%%%%%%%%%%%%%%%%%%%%%%%%%%%%%%%%%%%%%%%%%%%%%%%%%%%%%

\chapter{Лекция 16: Накрытия Галуа}

%%%%%%%%%%%%%%%%%%%%%%%%%%%%%%%%%%%%%%%%%%%%%%%%%%%%%%%%%%%%%%%%%%%%%%%%

%%%%%%%%%%%%%%%%%%%%%%%%%%%%%%%%%%%%%%%%%%%%%%%%

\section{Факторпространства}

%%%%%%%%%%%%%%%%%%%%%%%%%%%%%%%%%%%%%%%%%%%%%%%%


\определение
Пусть $M$ --- топологическое пространство, а $\sim$ --
отношение эквивалентности. Подмножество $U \subset
M/\sim$ множества классов $M/\sim$ называется
{\бф открытым}, если его прообраз в $M$ открыт.
Проверьте, что это определяет топологию 
на $M/\sim$. Такая топология называется
{\бф фактортопологией}, а пространство
$M/\sim$ --- {\бф факторпостранством.}
\ео

\определение
Пусть задано множество $M$. Легко видеть, что множество
биекций $\operatorname{Bij}(M,M)$ из $M$ в $M$ образует группу с операцией
композиции. Напомним, что группа $G$ {\бф действует на множестве $M$},
если задан гомоморфизм из $G$ в группу биекций $\operatorname{Bij}(M,M)$.
Иначе говоря, для каждого $g\in G$, $m\in M$, задана точка
$g(m)$, при этом $g(g_1(m))= gg_1(M)$, для любых $g,
g_1\in G$.
\ео


\определение
В такой ситуации, {\бф орбитой} точки $x\in M$ называется
множество
\[
Gx := \{ y\in M \ | y = gx, g\in G\}
\]
всех точек, которые можно получить из $x$ действием $G$.
\ео


\определение
Пусть $M$ --- топологическое пространство, а $G$ --- группа.
Мы говорим, что $G$ {\бф действует на топологическом пространстве
$M$}, или $G$ {\бф непрерывно действует на $M$}, если
$G$ действует на $M$, причем для любого $g\in G$, отображение
$m \arrow gm$ непрерывно. 
\ео

\замечание
Аналогичным образом определяется действие группы $G$
на объекте категории $\cac$. {\бф Действием группы} на $X\in \Ob(\cac)$
называется отображение $G \arrow \Mor(X,X)$ такое, что
произведение элементов группы переходит в композицию
морфизмов.
\еза

\определение
Пусть $G$ --- группа, действующая на топологическом
пространстве $M$. {\бф Факторпространством} по действию
группы называется пространство классов эквивалентности
$M/\sim$, где $x\sim y$, если $x$ и $y$ лежат в одной
орбите $G$. Также факторпространство называют
{\бф пространство орбит действия $G$}.
\ео

\задача
Пусть $M$ хаусдорфово, а $G$ --- группа, непрерывно
действующая на $M$. Всегда ли факторпространство
$M/G$ хаусдорфово?
\ез

\замечание\label{_quotie_otkry_Zamechanie_}
Пусть $G$ --- группа, действующая на топологическом
пространстве $M$. Тогда естественная проекция
$M\stackrel \pi \arrow M/G$ является открытым
отображением. В самом деле, пусть $U$ --- открытое множество,
а $GU$ --- объединение всех точек вида $gu$,
$g\in G$, $u\in U$. Тогда $GU$ --- объединение 
открытых множеств вида $gU$, и оно открыто. Поскольку 
$\pi^{-1}(\pi(U))= GU$, образ $\pi(U)$ открыт в $M/G$.
\еза



%%%%%%%%%%%%%%%%%%%%%%%%%%%%%%%%%%%%%%%%%%%%%%%%

\section{Категория накрытий}

%%%%%%%%%%%%%%%%%%%%%%%%%%%%%%%%%%%%%%%%%%%%%%%%

\определение
Пусть $M$, $\tilde M$ --- 
топологические пространства, а $\pi:\; \tilde M \arrow M$
непрерывное отображение. $\pi$ называется {\бф этальным},
если у каждой точки $\tilde x\in \tilde M$ есть окрестность
$\tilde U \ni \tilde x$ такая, что 
\[ \pi\restrict {\tilde U}:\; \tilde U \arrow \pi(\tilde U) 
\]
это гомеоморфизм. 
Это отображение называется {\бф накрытием}, если
у каждой точки $x\in M$, есть окрестность
$U\ni x$ такая, что $\pi^{-1}(U)$ гомеоморфно $U \times S$,
где $S$ --- топологическое пространство с дискретной топологией,
а отображение $\pi\restrict{\pi^{-1}(U)}:\; \pi^{-1}(U)\arrow U$
при таком изоморфизме совпадает с проекцией $U \times
S\arrow U$. {\бф Базой накрытия} называется 
$M$, а его {\бф слоем} над точкой $x$ --- прообраз 
$\pi^{-1}(x)$.
\ео

\замечание
Легко видеть, что любое накрытие этально (проверьте это).
\еза

\замечание
Пусть $U\subset X$ --- открытое подмножество,
которое не является замкнутым. Отображение вложения
$j:\; U \arrow X$ этально, но не является накрытием (проверьте).
\еза

\задача
Пусть $M$ --- связно, а $\pi:\; \tilde M \arrow M$ --
накрытие. Докажите, что слой $\pi^{-1}(x)$ равномощен
$\pi^{-1}(y)$, для любых $x, y \in M$.
\ез


\пример
Отождествим окружность $S^1$ с одномерным тором $\R / {2
\pi \Z}$. Естественная проекция $\R \arrow S^1$ является
накрытием (докажите). Проекция $\R^n$ на тор $T^n = \R^n/\Z^n$
также является накрытием (докажите это). 

\определение
Пусть $G$ --- группа, действующая на топологическом
пространстве $M$. Говорится, что действие $G$ {\бф вполне разрывно},
если у каждой точки $x\in M$ есть окрестность $U$ такая,
что $U \cap gU= \emptyset$ для любого $g\in G$ такого, что
$g$ действует не тождественно в окрестности $U$. 
\ео

\утверждение
Пусть  $G$ --- группа, вполне разрывно действующая на
топологическом пространстве $M$.
Тогда  проекция
$M \stackrel \pi \arrow M/G$ является накрытием.

\хфилл

\ноиндент
{\бф Доказательство:} Пусть $x\in M$ --- любая точка,
а $U \ni x$ --- окрестность, удовлетворяющая $U \cap gU= \emptyset$ 
для любого $g\in G$ такого, что $g(x) \neq x$. Поскольку
$\pi$ открыто, $\underline U:=\pi(U)$ --- окрестность $\pi(x)$.
Естественная проекция $\pi:\; U \arrow \underline U$
по построению биективна и непрерывна. Поскольку каждое 
открытое подмножество $\underline{U_1}\subset \underline U$ получается
как образ открытого подмножества вида $GU_1$, где $U_1 \subset U$ 
(Замечание \ref{_quotie_otkry_Zamechanie_}),  $\pi:\; U \arrow \underline U$
гомеоморфизм. Поэтому проекция $M \stackrel \pi \arrow M/G$ этальна.
Поскольку $\pi^{-1}(\underline U)$ --- объединение непересекающихся
открытых множеств вида $gU$, где $g\in G$, эта проекция
является накрытием.\endproof

\hfill

\пример
Примеры вполне разрывного действия группы на 
топологическом пространстве приведены выше
($\R \arrow S^1$, $\R^n \arrow T^n$). Подобных
примеров можно изобрести немало.

\задача
Пусть $M$ --- группа верхнетреугольных матриц
(матриц, у которых на диагонали 1, ниже диагонали 0),
а $\Gamma$ --- множество  верхнетреугольных матриц
с целыми коэффициентами. Докажите, что $\Gamma$ --- группа.
Рассмотрим действие $\Gamma$ на $M$ по формуле
$\gamma(m) = \gamma\cdot m$, где $\cdot$ обозначает
умножение матриц. Докажите, что это действие
вполне разрывно. 
\ез

\определение
Пусть $\tilde M\stackrel \pi \arrow M$, 
$\tilde M'\stackrel {\pi'} \arrow M$ --- накрытия $M$.
Непрерывное отображение $\psi:\; \tilde M \arrow \tilde M'$
называется {\бф морфизмом накрытий}, если
$\psi \circ \pi'=\pi$. 
\ео

Соотношения наподобие $\psi \circ \pi'=\pi$ часто изображают
посредством диаграмм
\begin{equation}\label{_commu_morf_nakr_Equation_}
\xymatrix{
\tilde M\ar[r]^\psi\ar[rd]^{\pi} & \tilde M'\ar[d]^{\pi'}\\
& M}
\end{equation}
в вершинах которой стоят объекты категории,
а стрелочки обозначают морфизмы.
Говорится, что эта диаграмма {\бф коммутативна},
если морфизм, полученный композицией стрелочек,
идущих из одного объекта в другой, не зависит от выбора
пути по стрелочкам.

В частности, диаграмма \eqref{_commu_morf_nakr_Equation_}
коммутативна тогда и только тогда, когда 
$\psi \circ \pi'=\pi$. 


\замечание
Накрытия пространства $M$ образуют категорию. Объекты
этой категории --- накрытия $M$, а морфизмы определены
выше. Ассоциативность композиции морфизмов и существование
тождественного морфизма очевидны (проверьте это).
\еза

\замечание
Множество морфизмов из $\tilde M$ в $\tilde M'$
обозначается $\Mor(\tilde M, \tilde M')$.
Когда надо подчеркнуть зависимость от $M$,
пишут \[ \Mor_M(\tilde M, \tilde M')\]
(читается "множество морфизмов $\tilde M$ в $\tilde M'$
над $M$).
\еза

\определение
Пусть $\tilde M \stackrel \pi \arrow M$ --- накрытие.
Мы говорим, что оно {\бф расщепляется}, если 
$\tilde M = M \times V$, где $V$ --- множество
с дискретной топологией, и это разложение
совместимо с проекцией в $M$. Расщепляющееся
накрытие также называют {\бф тривиальным}.
\ео

\определение 
Пусть $\tilde M \stackrel \pi \arrow M$ --- накрытие.
Если $U \subset M$ --- открытое подмножество, то
$\pi\restrict{\pi^{-1}(U)}:\; \pi^{-1}(U)\arrow U$ --
тоже накрытие (проверьте). Оно называется
{\бф ограничением накрытия $\pi$ на $U$}.
Ограничение накрытия $\pi$ на достаточно
малое открытое множество $U \subset M$ тривиально, по 
определению накрытия. Это свойство часто выражают,
говоря, что накрытие {\бф локально тривиально},
а чтобы уточнить, что локальность понимается
как локальность по $M$ --- говорят, что
оно {\бф локально тривиально по базе},
или {\бф локально тривиально по $M$}.
\ео

\замечание
Пусть $M$ связно, а $\tilde M \arrow M$ --- тривиальное накрытие
$M$. Тогда $\tilde M = M \times S$, где $S$ --- множество
связных компонент $\tilde M$. В частности, 
все связные компоненты $\tilde M$ проектируются
на $M$ гомеоморфно.
\еза

\замечание\label{_morfizm_rasshche_Zamechanie_}
Пусть $M$ связно, а $\psi:\; \tilde M \arrow \tilde M'$ --
морфизм тривиальных накрытий $M$. Тогда 
$\tilde M = M \times S$, и $\tilde M'= M \times S'$,
где $S$, $S'$ --- множество связных компонент
$\tilde M$, $\tilde M'$.
Образ связной компоненты $\tilde M$ 
связен, следовательно, лежит целиком в связной
компоненте $\tilde M'$. Пусть $\psi_S:\; S \arrow S'$
-- отображение, индуцированное $\psi$ на связных компонентах.
Тогда
\[
\psi(m, s) = (m, \psi_S(s))
\]
(проверьте).
\еза


\определение
Пусть $M$ --- топологическое пространство. $M$ называется
{\бф локально связным}, если у $M$ есть
база связных окрестностей, и {\бф локально линейно 
связным}, если у $M$ есть база линейно
связных окрестностей.
\ео

\замечание
Отметим, что из связности не следует
локальная связность, а из линейной связности
не следует локальная линейная связность. 
С другой стороны, локально линейно связное
связное пространство линейно связно
(см. доказательство Леммы \ref{_lok_lin_sv_nakry_Lemma_} ниже).
\еза


\утверждение
Пусть $\psi:\; \tilde M \arrow \tilde M'$ --- морфизм
накрытий $M$, причем $M$ локально связно. 
Тогда $\psi$ это накрытие  $\tilde M'$. 


\хфилл

\ноиндент
{\бф Доказательство:} Это утверждение локально по $M$,
значит, можно предположить, что $M$ связно,
а накрытия $\tilde M$ и $\tilde M'$ расщепляются.
В силу Замечания \ref{_morfizm_rasshche_Zamechanie_}, 
$\tilde M = M \times S$, $\tilde M'= M \times S'$,
a $\psi(m, s) = (m, \psi_S(s))$, где $\psi_S$ --
отображение, индуцированое $\psi$ на множестве 
компонент связности $\tilde M$ и $\tilde M'$.
Но отображение $\psi(m, s) = (m, \psi_S(s))$
является тривиальным накрытием на каждой
связной компоненте $\tilde M'$. Поэтому 
$\psi$ --- локально тривиальное накрытие 
$\tilde M'$ (проверьте). 
\endproof



%%%%%%%%%%%%%%%%%%%%%%%%%%%%%%%%%%%%%%%%%%%%%%%%%%%%%%%%%%%%

\section{Односвязные пространства}

%%%%%%%%%%%%%%%%%%%%%%%%%%%%%%%%%%%%%%%%%%%%%%%%%%%%%%%%%%%%


\определение
Связное топологическое пространство $M$ называется
{\бф этально односвязным}, если любое накрытие $M$ расщепляется.
\ео


Впоследствии нам понадобится следующая
простая лемма.

\хфилл


%%%%%%%%%%%%%%%%%%%%%%%%%%%%%%%%%%%%%%%%%%%%%%%%
\лемма\label{_lok_lin_sv_nakry_Lemma_}
Пусть $\tilde M \arrow M$ --- накрытие, причем
$M$ локально линейно связно, а $\tilde M$ связно.
Тогда $\tilde M$ линейно связно.


\хфилл

\ноиндент 
{\бф Доказательство:}
Поскольку $M$ локально линейно связно,
$\tilde M$ тоже локально линейно связно.
 Объединение
всех линейно связных множеств, содержащих точку
$x\in \tilde M$, открыто в $\tilde M$, в силу
локальной линейной связности $\tilde M$.
Это задает разбиение $\tilde M$ в компоненты
линейной связности, которые открыты.
А поскольку $\tilde M$ связно, 
такая компонента всего одна.
\endproof

\hfill


%%%%%%%%%%%%%%%%%%%%%%%%%%%%%%%%%%%%%%%%%%%%%%%%
\теорема
Любое выпуклое, замкнутое подмножество в $\R^n$
этально односвязно.

\хфилл

\ноиндент 
{\бф Доказательство.} {\бф Шаг 1:}
Пусть $M$ --- выпуклое, замкнутое подмножество
$\R^n$, а $\tilde M \stackrel\pi \arrow M$ --- накрытие. 
Поскольку $M$ локально связно (докажите это), $\tilde M$
локально связно. Поэтому $\tilde M$ --- несвязное
объединение своих компонент связности. 
Для доказательства этальной односвязности $M$
достаточно убедиться, что каждая
из этих компонент связности является
тривиальным накрытием $M$. Поэтому
можно считать, что $\tilde M$ связно.

\хфилл

\ноиндент
{\бф Шаг 2:} 
Поскольку $\tilde M$ связно, а $M$
линейно связно (докажите), $\tilde M$ 
линейно связно, в силу Леммы \ref{_lok_lin_sv_nakry_Lemma_}.

\хфилл


\ноиндент
{\бф Шаг 3:}
Рассмотрим $M$ как метрическое пространство,
с евклидовой метрикой, индуцированной из $\R^n$.
Любые две точки $x, y\in \tilde M$ можно соединить
путем \[ \gamma:\; [a,b] \arrow \tilde M,\] потому что $\tilde M$ линейно связно.
Определим длину пути $l(\gamma)$
как длину его образа $\gamma_\pi:= \gamma\circ \pi$ в $M$.
Если две точки $\gamma(t), \gamma(t+\epsilon)$
содержатся в окрестности $U\subset \tilde M$, которая
гомеоморфно проектируется на выпуклое подмножество
$U\subset M$, можно заменить участок $(\gamma(t), \gamma(t+\epsilon))$
пути $\gamma$ на прообраз отрезка, соединяющего точки
$\pi(\gamma(t)), \pi(\gamma(t+\epsilon))$.
По определению накрытия, $\gamma$ покрывается
такими открытыми множествами целиком
(проверьте это). Поскольку $\gamma([a,b])$
компактен, его можно покрыть конечным набором
таких открытых множеств. Заменив каждый
сегмент $(\gamma(t), \gamma(t+\epsilon))$
на прямолинейный, как указано выше, мы 
получим путь $\gamma$, который проектируется
в ломаную $\gamma_\pi:\; [a,b] \arrow M$.
Следовательно, любые две точки можно соединить
путем конечной длины. Определим метрику
на $\tilde M$ по формуле
\[
\tilde d(x,y) = \inf_\gamma l(\gamma),
\]
где инфимум берется по всем путям, соединяющим
$x$ и $y$. Легко видеть, что эта формула
задает на $\tilde M$ метрику (докажите).


\хфилл


\ноиндент
{\бф Шаг 4:} Метрика $\tilde d$ задает на пространстве
$\tilde M$ топологию, в которой небольшой окрестностью
точки $x\in \tilde M$ будет связная компонента
прообраза $\pi^{-1}(U)$, где $U\ni \pi(x)$ --
окрестность $\pi(x)$. Следовательно, $\tilde d$
согласована с исходной топологией на $\tilde M$.

\хфилл


\ноиндент
{\бф Шаг 5:}
Докажем, что $\tilde M$ с такой метрикой полно.
 Легко видеть, что 
$\tilde d(x, y) \geq d(\pi(x), \pi(y))$ (докажите).
Поэтому для любой последовательности Коши
$\{x_i\}$ в $\tilde M$, ее образ $\{\pi(x_i)\}$ --
последовательность Коши. Поскольку $M$ полно,
$\{\pi(x_i)\}$ сходится к точке $\underline x\in M$.
Поэтому почти все члены последовательности 
$\{x_i\}$ содержатся в $\pi^{-1}(U)$, для
любой окрестности $U \ni x$. Без ограничения
общности, можно считать, что $U$ выбрано
обраниченным, следовательно, 
замыкание $\bar U$ компактно.
Выбрав $U$ достаточно малым, можно считать,
что $\pi^{-1}(\bar U)$ --- объединение
непересекающихся компактов (компонент связности
 $\pi^{-1}(\bar U)$), гомеоморфных
$\bar U$. Эти компоненты связности
отстоят друг от друга на положительное
расстояние, а значит,
почти все элементы последовательности
Коши $\{x_i\}$
содержатся в одной из связных
компонент  (докажите). Обозначим эту
связную компоненту за $U_0$.
По построению, $U_0$ изометрически
проектируется на $U$.
Следовательно, $\{x_i\}$ сходится
к $\pi^{-1}(\underline x)\cap U_0$.
Полнота $\tilde M$ доказана.

\хфилл

\ноиндент
{\бф Шаг 6:}
По построению, метрика в $\tilde M$ является
внутренней (вычисляется как длина путей),
а значит, удовлетворяет условию Хопфа-Ринова.
Также это пространство полно (шаг 5)
и локально компактно (оно локально
гомеоморфно $\R^n$). По теореме Хопфа-Ринова, 
расстояние в $\tilde M$ реализуется геодезическими,
то есть для любых $x, y \in \tilde M$, $\tilde d(x,y)=a$, есть
изометрическое вложение $[0,a]\stackrel \gamma\arrow
\tilde M$.


\хфилл

\ноиндент
{\бф Шаг 7:} Пусть $x\in \tilde M$.
Тогда $x$ содержится в $U\subset \tilde M$,
которая проектируется гомеоморфно в $\pi(U)$.
Заменив $U$ на окрестность поменьше, можно считать,
что $\pi(U)$ выпукло (докажите это). 
Для любой $y\in U$, имеем
$\tilde d (x,y)= d(\pi(x), \pi(y))$, потому
что можно соединить $\pi(x), \pi(y)$ отрезком,
а потом поднять этот отрезок в $\tilde M$,
воспользовавшись тем, что $U\arrow \pi(U)$ 
-- гомеоморфизм. Поэтому геодезические
в $\tilde M$ проектируются в отрезки прямой.
Следовательно, $\pi:\; (\tilde M, \tilde d) 
\arrow(M, d)$ --- изометрия. Мы доказали,
что $\pi$ это гомеоморфизм.
\endproof

\хфилл

\следствие
Отрезок, квадрат, шар этально односвязны. 


%%%%%%%%%%%%%%%%%%%%%%%%%%%%%%%%%%%%%%%%%%%%%%%%%%%%%%%%%%%%

\section{Поднятие накрытия}

%%%%%%%%%%%%%%%%%%%%%%%%%%%%%%%%%%%%%%%%%%%%%%%%%%%%%%%%%%%%

\определение
Пусть $\tilde M \stackrel \pi\arrow M$ --- накрытие,
а $X\stackrel \psi \arrow M$ --- непрерывное отображение.
Отображение $X \stackrel{\tilde \psi}\arrow \tilde M$ называется
{\бф поднятием} $\psi$, если следующая диаграмма коммутативна
\[
\xymatrix{
& \tilde M\ar[d]^\pi \\ 
X \ar[ru]^{\tilde \psi}\ar[r]^{\psi} & M}
\]
\ео

\определение
Пусть $X\stackrel \psi \arrow M$,
$\tilde M \stackrel \pi\arrow M$ --
непрерывные отображения. Рассмотрим подмножество
$\tilde X \subset  X \times \tilde M$,
состоящее из всех пар $(x, \tilde m)$
таких, что $\psi(x) = \pi(\tilde m)$,
с топологией, индуцированной с $X \times \tilde M$.
Это пространство называется
{\бф расслоенным произведением}
$X$ и $\tilde M$, и обозначается
$X \times_M \tilde M$.
\ео

\задача
Пусть $M$ хаусдорфово. Проверьте, что $X \times_M \tilde M$ --
замкнутое подмножество в $X \times \tilde M$.
\ез


Пусть $\tilde M \stackrel \pi\arrow M$ --- накрытие,
а $X\stackrel \psi \arrow M$ --- непрерывное отображение.
Пусть $\tilde X$ --- расслоенное произведение 
$X$ и $\tilde M$ над $M$.
Обозначим через $\pi_X$ проекцию из $\tilde X$ на $X$.
В окрестности $U \ni \psi(x)$, расслоение
$\pi$ расщепляется: $\pi^{-1}(U) = U \times S$.
Поэтому в $U_X:= \psi^{-1}(U)$, имеем
$\pi_X^{-1}(U_X) = U_X\times S$ (проверьте это).
Значит, это накрытие. Мы получили следующую простую
лемму

\хфилл

\лемма
Пусть $\tilde M \stackrel \pi\arrow M$ --- накрытие,
а $X\stackrel \psi \arrow M$ --- непрерывное отображение.
Тогда расслоенное произведение $X \times_M \tilde M$ --
накрытие $X$.

\определение
В такой ситуации, $X \times_M \tilde M\arrow X$
называется {\бф индуцированным накрытием $X$}.
\ео

Следующая теорема важная, но простая. Ее доказательство вполне
очевидно из иллюстрации, приведенной ниже.


% set terminal postscript eps lw 3 butt "Helvetica" 22
% set output "helix.eps"
% set parametric
% set size ratio -1
% set ticslevel 0
% set xrange [-1.5:1.5]
% set yrange [-1.5:1.5]
% set zrange [0:25] 
% unset border
% unset xtics
% unset ytics
% unset ztics
% set terminal png size 1200,800
% set output "helix.png"
%splot [t=0:10*pi] cos(t),sin(t),0 title "" lw 4
%replot cos(t),sin(t),t title "" lw 3

\begin{figure}[ht]
\begin{center}\ \\
\epsfig{file=helix-final.eps,width=0.8\linewidth}\\
{\small Поднятие отображения $X\stackrel \psi \arrow M$ на накрытие}
\end{center}
\end{figure}


\хфилл

%%%%%%%%%%%%%%%%%%%%%%%%%%%%%%%%%%%%%%%%%%%%%%%%
\теорема\label{_podnya_sushche_Teorema_}
Пусть $\tilde M \stackrel \pi\arrow M$ --- накрытие,
а $X\stackrel \psi \arrow M$ --- непрерывное отображение.
Предположим, что $X$ этально односвязно. Тогда
существует поднятие $X \stackrel{\tilde \psi}\arrow \tilde M$.
Более того, для любой точки $x\in X$ и любой точки 
$\tilde x \in \pi^{-1}(\psi(x))$, существует и единственно
поднятие $X \stackrel{\tilde \psi}\arrow \tilde M$
такое, что $\tilde \psi(x) = \tilde x$.

\хфилл

\ноиндент 
{\бф Доказательство:} Пусть $\pi_X:\; \tilde X\arrow X$ --- индуцированное
накрытие. Поскольку оно расщепляется, имеем $\tilde X = X \times S$.
Для каждой из компонент связности $X \subset \tilde X$, естественная
проекция $\tilde X \arrow \tilde M$ задает
поднятие $\tilde \psi:\;  X \arrow \tilde M$.
Каждое из таких поднятий единственным образом
определяется выбором компоненты связности,
но эти компоненты параметризованы точками прообраза
$\pi^{-1}(\psi(x))$: каждая из компонент содержит ровно 
одну из точек этого множества. \endproof


%%%%%%%%%%%%%%%%%%%%%%%%%%%%%%%%%%%%%%%%%%%%%%%%

\section{Накрытия и пути}

%%%%%%%%%%%%%%%%%%%%%%%%%%%%%%%%%%%%%%%%%%%%%%%%



\замечание 
Пусть $(M,m)$ --- пространство с отмеченной точкой, $(\tilde M, \tilde m)$ --
его накрытие, а $\gamma\in \Omega(M,m)$ --- петля. Поскольку отрезок
односвязен, отображение $\gamma:\; [0,1] \arrow M$ поднимается
единственным образом до пути
$\tilde \gamma:\; [0,1] \arrow \tilde M$, причем
$\tilde \gamma(0) = \tilde m$. Такой путь
называется {\бф поднятием петли $\gamma$ на 
накрытие}. Понятно, что $\tilde \gamma$ --
уже не петля: точка $\tilde \gamma(0)$ может быть
не равна $\tilde \gamma(1)$.
\еза


%%%%%%%%%%%%%%%%%%%%%%%%%%%%%%%%%%%%%%%%%%%%%%%%%%%%%%%%%%%%
\утверждение\label{_podnya_puti_gomoto_Utverzhdenie_}
Пусть  $(M,m)$ --- пунктированное пространство, \[(\tilde M,
\tilde m)\arrow (M,m)\] --
его накрытие, $\gamma, \gamma'\in \Omega(M,m)$ --
гомотопные петли, а $\tilde \gamma$, $\tilde \gamma'$
-- их поднятия. Тогда $\tilde\gamma(1) =\tilde\gamma'(1)$.

\хфилл

\ноиндент
{\бф Доказательство:} Пусть $h:\; [0,1]\times [0,1]\arrow M$ --
отображение из квадрата, которое осуществляет гомотопию
$\gamma$ и $\gamma'$. По определению,
$h$ переводит две противоположные стороны квадрата
в $\gamma$ и $\gamma'$, а две другие --- в $m$.
Поскольку квадрат односвязен, $h$ поднимается
до $\tilde h:\; [0,1]\times [0,1]\arrow \tilde M$,
таким образом, что $(0,0)$ переходит в $\tilde m$.
На двух сторонах квадрата $h$ постоянное, но поднятие
тривиального пути, очевидно, тривиально. Поэтому
на этих двух сторонах $\tilde h$ тоже постоянное.
Значит, 
\[ \tilde \gamma(1)=\tilde h(1,0)=\tilde h(1,1) 
   = \tilde \gamma'(1).
\] 
\endproof

\хфилл

\следствие
Пусть $(M,m)$ --- связное и локально линейно связное
пунктированное пространство, которое 
имеет тривиальную фундаментальную группу.
Тогда $M$ этально односвязно.

\хфилл

\ноиндент
{\бф Доказательство:} 
Рассмотрим накрытие $\tilde M \stackrel \pi \arrow M$.
Достаточно доказать, что $\tilde M$ тривиально,
если оно связно. Пусть
$x_1, x_2 \in \pi^{-1}(x)$. Поскольку $\tilde M$
локально линейно связно и связно, оно линейно
связно, а значит, точки $x_1, x_2$ можно соединить
путем $\tilde \gamma$. Рассмотрим путь
$\gamma:= \tilde\gamma\circ \pi$.
Этот путь гомотопен тривиальному, 
потому что $\pi_1(M,m)=\{1\}$. 
Поднятие тривиального пути тривиально,
а значит, оба его конца совпадают.
В силу Утверждения \ref{_podnya_puti_gomoto_Utverzhdenie_},
то же верно и для $\gamma$, значит, $x_1=x_2$.
\endproof

\хфилл

Обратное утверждение тоже верно, хотя и в более
ограничительных предположениях.

Для доказательства этого полезно
изучить, как зависит $\pi_1(M,m)$
от точки $m\in M$. Пусть $x, y\in M$,
а $\xi$ --- путь из $x$ в $y$. 
Для каждой петли $\gamma\in \Omega(M,y)$,
рассмотрим путь $\xi \gamma\xi^{-1}$, определенный
по формуле 
\[
\xi \gamma\xi^{-1}(\lambda) = \begin{cases} 
\xi(3\lambda), & \ \ \text{если} \ \ 0\leq\lambda\leq 1/3\\
\gamma(3\lambda-1), &\ \ \text{если} \ \ 1/3\leq\lambda\leq 2/3,\\
\xi(3\lambda-2), &\ \ \text{если} \ \ 2/3\leq\lambda\leq 1
\end{cases}
\]
(проходим из $x$ в $y$ по $\xi$, обходим $y$ по $\gamma$,
и возвращаемся в $x$ в обратную сторону по $\xi$).
Легко видеть, что $\xi \gamma\gamma'\xi^{-1}$ 
гомотопно $\xi \gamma\xi^{-1}\xi\gamma'\xi^{-1}$ 
(докажите). Поэтому $\gamma \arrow \xi \gamma\xi^{-1}$
задает гомоморфизм групп \[ \pi_1(M,y) \arrow \pi_1(M,x).\]
Этот гомоморфизм, очевидно, обратим (докажите), а следовательно
группа $\pi_1(M,y)$ изоморфна $\pi_1(M,x)$. 
Нетрудно убедиться, что этот изоморфизм зависит
от выбора пути $\gamma$.

Обозначим через $\pi_1(m,x)$ множество гомотопических
классов путей из $m$ в $x$. Тот же аргумент, что и выше,
позволяет построить биекцию между $\pi_1(m,x)$ и
$\pi_1(M,m)$.


Пусть $(M,m)$ --- пунктированное пространство,
связное и локально линейно связное.
Предположим, что у $M$ есть база топологии,
состоящая из множеств $U$ с $\pi_1(U)=\{1\}$.
Зафиксируем такое $U$, его точку $x$, и
пусть $\gamma$ --- путь из $m$ в $x$.
Обозначим через ${\cal U}_\gamma$ множество
всех пар $(y, \gamma' \in \pi_1(m,y))$,
где $y\in U$, и существует путь
$\nu$ из $y$ в $x$, целиком лежащий
в $U$ и такой, что $\gamma'\nu\gamma^{-1}$
гомотопно нулю (тривиальному пути).
Поскольку все пути из $x$ в $y$, лежащие
в $U$, гомотопны, естественная
проекция из ${\cal U}_\gamma$ в $U$ --- гомеоморфизм
(проверьте это).

Пусть $\tilde M$ --- множество всех пар 
$(x, \gamma \in \pi_1(m,x))$, с топологией,
база которой задана множествами вида
${\cal U}_\gamma$. Легко видеть, что
пересечение таких множеств имеет такой
же вид, и поэтому ${\cal U}_\gamma$ задает
топологию на $\tilde M$. Естественная
проекция $\tilde M \arrow M$ этальна
по построению, а поскольку
$\pi^{-1}(U)$ --- объединение
${\cal U}_\gamma$ для всех гомотопических
классов путей из $m$ в $x\in U$,
$\pi$ это накрытие.

Наконец, пусть 
$\gamma \in \pi_1(M,m)$ класс, негомотопный нулю, пусть
$\tilde \gamma$ --- его поднятие в $\tilde M$,
такое, что $\gamma(0)$ соответствует паре $(m, \gamma_0)$,
где $\gamma_m$ --- тривиальный путь. 
Тогда $\gamma(1)$ соответствует паре $(m, \gamma)$,
значит, накрытие $\tilde M \arrow M$ нетривиально.

Мы получили следующее утверждение

\хфилл

%%%%%%%%%%%%%%%%%%%%%%%%%%%%%%%%%%%%%%%%%%%%%%%%
\утверждение\label{_pi_1_odnosvyazno_Utverzhdenie_}
Пусть $(M,m)$ --- пунктированное пространство,
которое связно и локально линейно связно.
Предположим, что у $M$ есть база топологии,
состоящая из множеств $U$ с $\pi_1(U)=\{1\}$,
и $M$ этально односвязно. Тогда $\pi_1(M,m)=\{1\}$.

\endproof

\хфилл

\замечание
Исторически, односвязность $M$ означает $\pi_1(M,m)=\{1\}$.
В большинстве книг по топологии односвязность определяют именно так.
В предположениях Утверждения \ref{_pi_1_odnosvyazno_Utverzhdenie_},
$\pi_1(M,m)=\{1\}$ тогда и только тогда, когда любое накрытие
$M$ расщепляется, значит, в этой ситуации этальная
односвязность
равносильна обычной.
\еза

\замечание
Пусть $\pi_1(M, m)\neq \{1\}$. Накрытие $\tilde M$,
которое строится в доказательстве Утверждения 
\ref{_pi_1_odnosvyazno_Utverzhdenie_}, можно
построить явно, используя топологию
на пространстве путей, которую мы 
определили в Лекции 11. Пусть $\tilde \Omega(M,m)$ --- пространство
всех путей с начальной точкой $m$. Предположим,
что у каждого пути $\gamma \in \Omega(M,m)$, ведущего из $m$ в $x$,
есть окрестность в топологии на $\tilde \Omega(M,m)$
такая, что все ее точки гомотопны $\gamma$ в классе
путей из $m$ в $x$. В предположениях \ref{_pi_1_odnosvyazno_Utverzhdenie_}
это верно (докажите). Обозначим $\gamma \sim \gamma'$, если
$\gamma, \gamma'$ --- гомотопные пути из $m$ в $x$.
Обозначим через $\tilde M$ факторпространство $\tilde \Omega(M,m)$ 
по этому отношению эквивалентности. Естественная
проекция $\tilde M \arrow M$, переводящая $\gamma$ 
в $x=\gamma(1)$, является накрытием, и эквивалентна
накрытию $\tilde M$, построенному выше (проверьте).
Мы не будем пользоваться этим наблюдением.
\еза




%%%%%%%%%%%%%%%%%%%%%%%%%%%%%%%%%%%%%%%%%%%%%%%%

\section{Произведение накрытий}

%%%%%%%%%%%%%%%%%%%%%%%%%%%%%%%%%%%%%%%%%%%%%%%%

\определение
Пусть $\tilde M\stackrel {\pi} \arrow M, \tilde M'\stackrel {\pi'} \arrow M$ 
-- накрытия $M$. Рассмотрим расслоенное произведение
$\tilde M \times_M \tilde M'\subset \tilde M \times \tilde M'$, состоящее
из всех $(x,y)\in \tilde M \times \tilde M'$, таких,
что $\pi(x) = \pi'(y)$. Тогда $\tilde M \times_M \tilde M'$
называется {\бф произведением накрытий}.
\ео

\замечание 
Произведение
$\tilde M \times_M \tilde M'$ является накрытием $M$.
В самом деле, локально по $M$, $\tilde M$ изоморфно
$S\times M$, $\tilde M'$ изоморфно $S' \times M$,
следовательно, $\tilde M \times_M \tilde M'$ изоморфно
$S\times S' \times M$.
\еза 

\замечание 
Легко видеть, что для любого накрытия $M_1$ над $M$,
\[ 
\Mor(M_1, \tilde M \times_M \tilde M')= \Mor(M_1, \tilde M)
 \times \Mor(M_1, \tilde M').
\]
\еза

\определение
Пусть $\phi:\; M_1\arrow M_2$ --- морфизм накрытий $M$.
Рассмотрим подмножество $\Gamma_\phi \subset M_1\times_M M_2$,
состоящее из всех пар вида $(x, \phi(x))$.
Это подмножество называется {\бф графиком} морфизма $M$.
\ео

\замечание
Проекция графика $\Gamma_\phi$ на $M_1$ --- гомеоморфизм
(докажите это). Поэтому $\Gamma_\phi$ --- накрытие $M$.
Поскольку локально по $M$ график $\Gamma_\phi$ --- объединение
нескольких копий $M$, он открытозамкнут в $M_1\times_M M_2$.
\еза


Пусть $\tilde M \arrow M$ --- накрытие.
Чтоб подчеркнуть аналогию с теорией Галуа,
мы будем обозначать его как $[\tilde M: M]$.

\определение
Пусть $[\tilde M: M]$ --- накрытие.
Оно называется {\бф связным}, если 
$M$ и $\tilde M$ связны и локально связны.
\ео

\утверждение
Пусть $[\tilde M: M]$ --- накрытие, причем
$M$ связно и локально связно. Тогда 
$\tilde M$ является несвязным объединением
своих компонент связности $\tilde M_i$. 
Более того, каждая из компонент $\tilde M_i$
является связным накрытием $M$.

\хфилл

\ноиндент
{\бф Доказательство:} 
Поскольку $M$ локально связно, а
$\tilde M$ локально гомеоморфно $M$,
оно тоже локально связно. Поэтому
любая связная компонента $\tilde M$
открыта (докажите это). Значит, 
$\tilde M$ --- несвязное объединение
своих компонент связности.
Пусть теперь $U\subset M$ --- связное
открытое подмножество, над которым
$\tilde M$ расщепляется. Тогда 
$\pi^{-1}(U)$ --- несвязное
объединение нескольких копий $U$.
Каждая из этих копий целиком лежит
в одной из связных компонент $\tilde M_i$,
следовательно, $\tilde M_i \cap \pi^{-1}(U)$
тоже несвязное объединение копий $U$.
Поэтому $\tilde M_i$ --- накрытие.
\endproof

\хфилл


\следствие
Пусть $[\tilde M:M]$ --- связное накрытие.
Тогда график любого морфизма $\phi\in \Mor(\tilde M,
\tilde M)$ --- связная компонента в $\tilde M \times_M \tilde M$.
Связная компонента $\tilde M'\subset\tilde M \times_M \tilde M$ является
графиком морфизма  $\phi\in \Mor(\tilde M, \tilde M)$
тогда и только тогда, когда проекция $\tilde M'$ на первую
компоненту задает изоморфизм $\tilde M' \arrow \tilde M$.

\хфилл

\ноиндент
{\бф Доказательство:} 
Если $\tilde M'$ --- график, он открытозамкнут в
$\tilde M \times_M \tilde M$ (это локальное
утверждение, а локально $\tilde M'$ --- объединение
нескольких компонент $\tilde M \times_M \tilde M$).
Поэтому он является связной компонентой.

Обратное тоже верно, потому что любая
связная компонента является накрытием,
а если $\tilde M'$ гомеоморфно проектируется 
на $\tilde M$, проекция на вторую компоненту
задает морфизм накрытий из $\tilde M \cong \tilde M'$
на $\tilde M$. По построению, $\tilde M'$
является графиком этого морфизма.
\endproof

\хфилл


%%%%%%%%%%%%%%%%%%%%%%%%%%%%%%%%%%%%%%%%%%%%%%%%
\замечание\label{_Mor_is_prod_avtom_Zamechanie_}
Пусть $[\tilde M:M]$ --- связное накрытие.
Рассмотрим $\tilde M \times_M \tilde M$ как накрытие
$\tilde M$ относительно проекции на первую компоненту.
Тогда 
\[
\Mor_{\tilde M}(\tilde M, \tilde M\times_M \tilde M) =
\Mor_M(\tilde M, \tilde M).
\]
В самом деле, морфизмы $\Mor_{\tilde M}(\tilde M, \tilde M\times_M \tilde M)$
взаимно однозначно соответствуют компонентам $\tilde M\times_M \tilde M$,
которые изоморфно проектируются на $\tilde M$, но в силу
предыдущего следствия это то же самое, что морфизмы
из $\tilde M$ в себя.
\еза


%%%%%%%%%%%%%%%%%%%%%%%%%%%%%%%%%%%%%%%%%%%%%%%%

\section{Накрытия Галуа и группа Галуа}

%%%%%%%%%%%%%%%%%%%%%%%%%%%%%%%%%%%%%%%%%%%%%%%%

\определение
Пусть $[\tilde M: M ]$ --- связное накрытие.
Оно называется {\бф накрытием Галуа}, если
$\tilde M \times_M \tilde M \arrow \tilde M$
расщепляется. 
\ео

\замечание
Отметим, что 
\[
\Mor_{\tilde M}(\tilde M, \tilde M\times_M \tilde M) =
\Mor_M(\tilde M, \tilde M).
\]
(Замечание \ref{_Mor_is_prod_avtom_Zamechanie_}).
Поэтому $[\tilde M: M ]$ 
является накрытием Галуа
тогда и только тогда,
когда каждая пара $(x, y) \in \tilde M\times_M \tilde M$
принадлежит графику морфизма $\nu$. Из соображений
симметрии, график $\Gamma_\nu$ проектируется
изоморфно на оба сомножителя $\tilde M$,
значит, $\nu$ это изоморфизм.
\еза

Из этого замечания сразу вытекает следующее
полезное утверждение.

\хфилл

%%%%%%%%%%%%%%%%%%%%%%%%%%%%%%%%%%%%%%%%%%%%%%%%
\утверждение\label{_Gal_nakr_aut_Utverzhdenie_}
Пусть $\tilde M \stackrel\pi \arrow M$ --- связное накрытие.
Тогда следующие утверждения равносильны.
\begin{description}
\item[(i)] $[\tilde M: M ]$ --- накрытие Галуа
\item[(ii)] для любых точек $x, y \in \tilde M$,
таких, что $\pi(x)=\pi(y)$, существует
автоморфизм $\tilde M$, переводящий $x$ в $y$.
\end{description}
\endproof

\определение
Пусть $G$ --- группа, действующая на множестве $M$.
Действие группы называется {\бф свободным}, если
для любого неединичного $g\in G$ и любого $m \in M$,
имеем $gm\neq m$.
Действие группы называется {\бф транзитивным},
если $M$ состоит из одной орбиты.
\ео

\замечание
Пусть $[\tilde M:M]$ --- связное накрытие.
Обозначим через $\Aut [\tilde M:M]$
группу автоморфизмов $\tilde M$ над $M$.
Тогда $\Aut [\tilde M:M]$ свободно действует на
$M$. Действительно, пусть $g\in \Aut [\tilde M:M]$ --
элемент, сохраняющий $x\in M$. Тогда график 
$\Gamma_g$ пересекается с графиком тождественного
отображения $\Gamma_\Id$. Но поскольку график автоморфизма
является связной компонентой 
$\tilde M\times_M \tilde M$, они совпадают:\[ \Gamma_g=\Gamma_\Id.\]
\еза

Напомним, что {\бф слоем} накрытия
$\pi:\; \tilde M \arrow M$ называется
множество $\pi^{-1}(x)$, где $x\in M$.

Следующее утверждение немедленно следует
из Утверждения \ref{_Gal_nakr_aut_Utverzhdenie_}
(проверьте это). 

\хфилл

\утверждение
Связное накрытие $[\tilde M:M]$ является накрытием
Галуа тогда и только тогда, когда действие
$\Aut [\tilde M:M]$ на любом слое  $[\tilde M:M]$
транзитивно.

\endproof

\определение
Пусть $[\tilde M:M]$ --- накрытие Галуа.
Тогда группа автоморфизмов $\Aut [\tilde M:M]$
называется {\бф группой Галуа накрытия}.
\ео


\замечание
Пусть задано вполне несвязное действие группы $G$
на связном, локально связном топологическом пространстве
$\tilde M$. Тогда $[\tilde M: \tilde M/G]$ --- накрытие
Галуа. Обратное тоже верно: любое накрытие Галуа
имеет вид $[\tilde M: \tilde M/G]$, где $G$ --- группа
Галуа $[\tilde M: M]$ (докажите это).
\еза





%%%%%%%%%%%%%%%%%%%%%%%%%%%%%%%%%%%%%%%%%%%%%%%%

\section{Теория Галуа для накрытий}

%%%%%%%%%%%%%%%%%%%%%%%%%%%%%%%%%%%%%%%%%%%%%%%%

Следующая лемма вполне очевидна (докажите ее).

\хфилл

\лемма
Пусть $W_1\arrow W_2 \arrow W_2$ --- 
накрытия, причем $W_1\arrow W_2$ инъективно, а
$[W_2:W_3]$ расщепляется. Тогда $W_1:W_3$ тоже расщепляется.

\хфилл

%%%%%%%%%%%%%%%%%%%%%%%%%%%%%%%%%%%%%%%%%%%%%%%%
\утверждение\label{_M_1:M_2_Galua_Utverzhdenie_}
Пусть $M_1 \arrow M_2 \arrow M_3$ --- накрытия, причем
$[M_1:M_3]$ --- накрытие Галуа. Тогда
$[M_1:M_2]$ --- тоже накрытие Галуа.

\хфилл

\ноиндент
{\бф Доказательство:} Рассмотрим последовательность накрытий 
\[ M_1\times_{M_2}M_1 \arrow M_1\times_{M_3}M_1 \arrow M_1.\]
Накрытие $M_1\times_{M_3}M_1 \arrow M_1$ расщепляется,
потому что $[M_1:M_3]$ накрытие Галуа, а естественное
вложение $M_1\times_{M_2}M_1 \arrow M_1\times_{M_3}M_1$
инъективно по построению (проверьте). В силу предыдущей
леммы, из этого следует, что $M_1\times_{M_2}M_1 \arrow M_1$
расщепляется. \endproof

\хфилл


\утверждение
Пусть $M_1 \arrow M_2 \arrow M_3$ --- накрытия, причем
$[M_1:M_2]$ --- накрытие Галуа, и 
$[M_2:M_3]$ --- накрытие Галуа.
Тогда $[M_1:M_3]$ --- тоже накрытие Галуа.

\хфилл


\ноиндент
{\бф Доказательство:} Рассмотрим изоморфизм
\[
M_1\times_{M_3}M_1= M_1\times_{M_2}(M_2 \times_{M_3} М_2)\times_{M_2}М_1.
\]
В силу того, что $[M_2:M_3]$ --- накрытие Галуа,
имеем $M_2 \times_{M_3} М_2=\bigsqcup_{\alpha\in I} M_2$,
для какого-то набора индексов $I$, и поэтому
имеем
\begin{align*}
M_1\times_{M_3}M_1 =& 
M_1\times_{M_2}(M_2 \times_{M_3} М_2)\times_{M_2}М_1 \\= & 
M_1\times_{M_2} \left(\bigsqcup_{\alpha\in I} M_2\right)
\times_{M_2}М_1 = \bigsqcup_{\alpha\in I} M_1\times_{M_2}М_1.
\end{align*}
Последнее выражение есть несвязная сумма нескольких
копий $M_1$. Поэтому $[M_1\times_{M_3}M_1:M_1]$
расщепляется.
\endproof

\замечание
Пусть $[\tilde M:M]$ --- накрытие Галуа, а
$G=\Aut[\tilde M:M]$ --- его группа Галуа.
Отметим, что $G$ вполне разрывно действует
на $\tilde M$ (проверьте это), и поэтому для
любой подгруппы $G'\subset G$, определено
факторпространство $\tilde M/G'$, которое
тоже является накрытием.
\еза

\определение
Пусть $[\tilde M:M]$ --- связное накрытие.
{\бф факторнакрытием} $[\tilde M:M]$ называется
накрытие $[\tilde M':M]$ такое,
что задан сюръективный
морфизм накрытий $\tilde M\arrow \tilde M'$.
\ео


%%%%%%%%%%%%%%%%%%%%%%%%%%%%%%%%%%%%%%%%%%%%%%%%
\теорема
(Основная теорема теории Галуа) \\
Пусть $[\tilde M:M]$ --- накрытие Галуа,
$G=\Aut[\tilde M:M]$ --- его группа Галуа,
а $G'\subset G$ --- любая подгруппа.
Тогда $\tilde M/G'$ --- накрытие $M$.
Более того, любое факторнакрытие 
$[\tilde M:M]$ получается таким образом.


\хфилл

\ноиндент
{\бф Доказательство:} Пусть $[\tilde M':M]$ --- 
факторнакрытие $[\tilde M:M]$.
Тогда $ \tilde M\arrow \tilde M'\arrow M$ последовательность
накрытий, причем $[\tilde M:M]$ --- накрытие Галуа, значит,
$[\tilde M:\tilde M']$ --- тоже накрытие Галуа
(Утверждение \ref{_M_1:M_2_Galua_Utverzhdenie_}). 
Поэтому $\tilde M' = \tilde M/G'$, где
$G'$ --- группа автоморфизмов $\tilde M$ над $\tilde M'$.
Поскольку каждый такой автоморфизм является
автоморфизмом $\tilde M$ над $M$,
$G'$ --- подгруппа $\Aut[\tilde M:M]$.
Мы получили взаимно-однозначное
соответствие между факторнакрытиями
и подгруппами группы Галуа. 
\endproof


%%%%%%%%%%%%%%%%%%%%%%%%%%%%%%%%%%%%%%%%%%%%%%%%

\section{Универсальное накрытие}

%%%%%%%%%%%%%%%%%%%%%%%%%%%%%%%%%%%%%%%%%%%%%%%%


\определение
Пусть $[\tilde M:M]$ --- связное накрытие.
Оно называется {\бф этальным универсальным накрытием},
если $\tilde M$ этально односвязно.
\ео

\замечание
Универсальное накрытие является накрытием Галуа.
Действительно, раз $\tilde M$ этально односвязно, то 
$\tilde M \times_M \tilde M$ расщепляется над $\tilde M$.
\еза

\замечание
Универсальное накрытие 
единственно с точностью до изоморфизма. Действительно,
пусть $\tilde M$, $\tilde M'$ --- два
универсальных накрытия. Поскольку $\tilde M$
и $\tilde M'$ этально односвязны, любая связная компонента их
произведения $\tilde M \times_M \tilde M'$ расщепляется
над $\tilde M$ и над $\tilde M'$, значит, является
графиком изоморфизма.
\еза

\определение
Локально связное топологическое пространство $M$ называется
{\бф локально этально односвязным}, если у каждой точки
есть связная, этально односвязная окрестность.
\ео


\определение
Пусть $\{M_\alpha\stackrel{\pi_\alpha}\arrow M\}$ --- набор накрытий 
локально этально односвязного пространство $M$. Если
все эти накрытия расщепляются, 
$M_\alpha = S_\alpha \times M$, определим
$\prod_M M_\alpha$ как $M \times \prod S_\alpha$.
В общем случае, возьмем в обыкновенном произведении 
$\prod M_\alpha$ подмножество, состоящее
из точек $\prod x_\alpha$ с $\pi_\alpha(s_\alpha) =x$,
и введем на нем топологию, взяв в качестве
базы открытые подмножества в $U \times \prod S_\alpha$,
где $U\subset M$ --- открытое множество, где все
$M_\alpha$ расщепляются и имеют вид 
$\pi_\alpha^{-1} (U) = U \times S_\alpha$.
\ео

\замечание
Отметим, что произведение накрытий
не является расслоенным произведением
в смысле Тихонова. В самом деле,
произведение конечных накрытий
компактов не обязательно конечно, значит,
может не быть компактом (приведите пример,
когда произведение компактных накрытий некомпактно).
\еза

В доказательстве существования универсального
накрытия нам понадобится следующая лемма

\хфилл

%%%%%%%%%%%%%%%%%%%%%%%%%%%%%%%%%%%%%%%%%%%%%%%%
\лемма
Пусть $[\tilde M: M]$ --- связное накрытие. 
Тогда $\tilde M$ --- факторнакрытие накрытия Галуа.

\хфилл

\ноиндент
{\бф Доказательство:} 
Мы будем строить накрытие Галуа $[M_G:M]$ как компоненту
произведения $\prod_M M_\alpha$, где все $M_\alpha$ изоморфны $\tilde M$.

\хфилл

\ноиндент
{\бф Шаг 1:} Пусть $M_G \times_{\tilde M} \tilde M$ расщепляется
как накрытие $M_G$. Тогда $M_G$ --- это накрытие Галуа. В самом деле,
$M_G$ это компонента в $\prod_M M_\alpha$, но раз
$M_G \times_{\tilde M} \tilde M$ расщепляется над $M_G$,
то и $M_G \times_{\tilde M} \prod_M M_\alpha$ расщепляется
над $M_G$. Следовательно $M_G \times_{\tilde M} M_G$ расщепляется
над $M_G$. 


\хфилл

\ноиндент
{\бф Шаг 2:} Пусть $x_1 \in M_G$ выбранная точка,
а $x$ --- ее образ в $M$. Предположим, что 
для каждого $y\in \tilde M$ в слое над $x$,
существует морфизм накрытий $\phi\in \Mor(M_G, \tilde M)$,
 переводящий $x_1$ в $y$. Тогда слой проекции 
$M_G \times \tilde M \arrow M_G$ содержится
в объединении графиков всех морфизмов
$\Mor_M(M_G, \tilde M)$. Поэтому
объединение всех таких графиков
равно $M_G \times_{\tilde M} \tilde M$,
а значит, $M_G \times_{\tilde M} \tilde M$ расщепляется
над $M_G$.

\хфилл

\ноиндент
{\бф Шаг 3:} Возьмем произведение
\[ \prod_M M_\alpha,\ \  \alpha \in F_x,\] 
проиндексированное всеми
точками слоя $F_x$ над $x$, и пусть $M_G$ --- компонента,
которая содержит точку $x_1 = \prod_{y\in F_x} y$.
Обозначим через $\pi_y$ проекцию $\prod_M M_\alpha$
на компоненту, соответствующую $y\in F_x$.
Тогда $\pi_y(x_1) =y$. Следовательно, 
для любой точки $y\in \tilde M$ в слое над $x$,
существует морфизм накрытий, переводящий
$x_1$ в $y$. В силу шага 2, из этого следует,
что $M_G \times_{\tilde M} \tilde M$ расщепляется
над $M_G$, а в силу шага 1 --- что $M_G$ это
расширение Галуа. \endproof 

\хфилл


Универсальное накрытие $M$ строится как произведение
накрытий Галуа $\prod_M M_\alpha$, где $M_\alpha$ пробегает
все классы изоморфизма накрытий Галуа. 
Чтоб это произведение имело смысл, нужно сначала
убедиться, что классы изоморфизма накрытий Галуа
$M$ образуют множество. Связное накрытие $[\tilde M: M]$
задается набором множеств ${\goth S}=\{ U_\beta \subset M\}$,
над которыми $\tilde M$ расщепляется, имея 
вид $U_\beta \times S_\beta$, и изоморфизмов 
$S_\beta\cong S_\gamma$ для любых пересекающихся
$U_\beta, U_\gamma\in {\goth S}$. 
Поскольку $\tilde M$ связно, группа, порожденная
такими изоморфизмами, действует на $S_\beta$
транзитивно. Поэтому мощность 
$S_\beta$ не может быть больше $\N \times |\goth S|$ (проверьте),
а значит, мощность $S_\beta$ ограничена $\N \times 2^{|M|}$.
Поскольку мощность множества классов изоморфизма
накрытий Галуа ограничена мощностью множества
всех топологий на $S_\beta\times M$,  эта мощность
не больше $2^{2^{|М|\times \N \times 2^{|M|}}}$
(проверьте это). А коль скоро эта мощность
ограничена, классы изоморфизма накрытий Галуа
$M$ образуют множество.

\хфилл

\теорема
Пусть $M$ --- связное, локально этально односвязное
топологическое пространство, а $M_G$ --
связная компонента в $\prod_M M_\alpha$, где $M_\alpha$ пробегает
все классы изоморфизма накрытий Галуа $M$.
Тогда $M_G$ --- универсальное накрытие $M$.


\хфилл

\ноиндент
{\бф Доказательство.} {\bf Шаг 1:}
Пусть $[M_1:M]$ --- накрытие Галуа, а $M_G'$ --
произведение накрытий Галуа $M$ по всем классам
изоморфизма накрытий, кроме $M_1$.
Тогда 
\[ \prod_M M_\alpha\times_M M_1 = M_G'\times_M
M_1\times_M M_1.
\]
Поскольку $M_1\times_M M_1$ --- несвязная сумма нескольких
копий $M_1$, \[ \prod_M M_\alpha\times_M M_1\] --- несвязная
сумма нескольких копий $\prod_M M_\alpha$.
Следовательно, \[ \prod_M M_\alpha\times_M M_1\]
расщепляется над каждой связной компонентой
$\prod_M M_\alpha$, а значит, накрытие
\[ M_G \times_M M_1\arrow M_G \] тоже расщепляется.
Для любого накрытия
$[M_1:M_G]$, произведение 
$M_G \times_{M_G}M_1$ является 
поднакрытием (образом вложения накрытий) в $M_G \times_M M_1$:
\[ M_G \times_{M_G}M_1\hookrightarrow
   M_G \times_M M_1,
\] 
значит, оно тоже расщепляется. Поэтому любое накрытие
Галуа $M_1$ расщепляется над $M_G$.

\хфилл

\ноиндент {\bf Шаг 2:}
Пусть $M_2$ --- любое накрытие $M_G$
(не обязательно накрытие Галуа). В силу
предудущей леммы, $M_2$ является
факторнакрытием накрытия Галуа
 $[M_1:M_G]$, которое расщепляется.
Поэтому $M_2$ тоже расщепляется (докажите).
\endproof



%%%%%%%%%%%%%%%%%%%%%%%%%%%%%%%%%%%%%%%%%%%%%%%%%%%%%%%%%%%%%%%%%%%%%%%%

\section{Этальная фундаментальная группа}

%%%%%%%%%%%%%%%%%%%%%%%%%%%%%%%%%%%%%%%%%%%%%%%%%%%%%%%%%%%%%%%%%%%%%%%%


Пусть $M$ локально линейно связно, $\tilde M \stackrel
\pi\arrow M$ --- универсальное накрытие, а $x\in M$.
Для каждого пути $\gamma \in \Omega(M,x)$,
рассмотрим поднятие $\tilde \gamma$ в $\tilde M$.
Поскольку $\Aut[\tilde M:M]$ действует транзитивно
на $\pi^{-1}(x)$, группа $\Aut[\tilde M:M]$
действует транзитивно на множестве поднятий
$\gamma$. Рассмотрим элемент $g_\gamma \in \Aut[\tilde M:M]$,
который переводит $\tilde\gamma(0)$ в $\tilde \gamma(1)$.
Поскольку действие $\Aut[\tilde M:M]$  транзитивно
на множестве поднятий пути $\gamma$, элемент 
$g_\gamma$ не зависит от выбора поднятия.
Мы построили отображение $\pi_1(M,x) \arrow \Aut[\tilde M:M]$.

Пусть $\gamma, \gamma'\in  \in \Omega(M,x)$ --
пути, а $\tilde \gamma$, $\tilde \gamma'$ --- их
поднятия, причем $\tilde \gamma$ соединяет
$y_1$ и $y_2$, а $\tilde \gamma'$ соединяет
$y_2$ и $y_3$. Очевидно, $\gamma\gamma'$ поднимается
до пути, который соединяет $y_1$ и $y_3$.
Поэтому построенное отображение 
$\pi_1(M,x) \arrow \Aut[\tilde M:M]$ --- гомоморфизм.
Если $\pi_1(\tilde M,y_1)=\{1\}$,
класс гомотопии пути $\gamma \in \Omega(M,x)$
однозначно задается вторым концом $y_2$
поднятия $\tilde\gamma$, если $\tilde\gamma(0)=y_1$.
Поскольку $\Aut[\tilde M:M]$ действует свободно
и транзитивно на $\pi^{-1}(x)$, в такой ситуации 
$\pi_1(M,x) \arrow \Aut[\tilde M:M]$ --- биекция.

\определение
Пусть $\tilde M \stackrel
\pi\arrow M$ --- универсальное накрытие. \\
Группа Галуа $\Aut[\tilde M:M]$ называется {\бф этальной фундаментальной
группой пространства $M$}. 
\ео

\замечание
В условиях Утверждения \ref{_pi_1_odnosvyazno_Utverzhdenie_},
$\pi_1(\tilde M,y_1)=\{1\}$, и группа $\Aut[\tilde M:M]$ равна
$\pi_1(M,x)$. В этой ситуации этальная фундаментальная
группа равна обычной.
\еза

В силу основной теоремы теории Галуа,
накрытия $[M_1:M]$ однозначно соответствуют
подгруппам этальной фундаментальной группы.


\задача
Убедитесь, что накрытие $[M_1:M]$
является накрытием Галуа тогда и только тогда, когда
соответствующая ему подгруппа $\Aut[\tilde M:M]$ нормальна.
\ез



%%%%%%%%%%%%%%%%%%%%%%%%%%%%%%%%%%%%%%%%%%%%%%%%%%%%%%%%%%%%

\section{История, замечания}

%%%%%%%%%%%%%%%%%%%%%%%%%%%%%%%%%%%%%%%%%%%%%%%%%%%%%%%%%%%%

Фундаментальная группа впервые появилась в диссертации
Римана в 1851 году. Риман интересовался продолжением голоморфных
(комплексно дифференцируемых) функций в комплексной области.
Риман обнаружил, что голоморфная функция однозначно продолжается
вдоль любого пути, который не пересекается с множеством
ее полюсов, но это продолжение может зависеть от 
выбора пути. Заменив комплексную область на ее накрытие,
можно добиться того, чтобы продолжение функции
было однозначно. Таким образом в математике
появились римановы поверхности (многообразия
вещественной размерности 2), накрытия
и фундаментальная группа.

\begin{figure}[ht]
\begin{center}
\epsfig{file=Riemann.eps,width=0.50\linewidth}\\
{Georg Friedrich Bernhard Riemann\\
(1826 --- 1866)}
\end{center}
\end{figure}

Начиная с 1860-х годов, топологию римановых
поверхностей немало изучали Жордан, Мебиус и многие 
другие математики. Фундаментальная группа была определена
(как множество, и довольно неформально) Жорданом, 
а Пуанкаре в 1895-м году определил ее строго, и 
одновременно обнаружил, что это группа.

Связь фундаментальной группы с накрытиями 
прослеживалась со времен Римана, но идея
определить фундаментальную группу в терминах
накрытий принадлежит Гротендику, который 
придумал, как строить топологические
инварианты в алгебраической ситуации.

Совместно с Мишелем Рено, Гротендик в 1961-м году
опубликовал исследование "Rev\^etements \'etales et 
groupe fondamental" (SGA1), первое в серии SGA
(S\'eminaire de g\'eom\'etrie alg\'ebrique), где
исследовал фундаментальную группу алгебраических
объектов в терминах накрытий. Оказалось, что
если воспользоваться подходящим понятием накрытия, 
можно определить этальную фундаментальную группу у огромного 
числа объектов алгебры и геометрии. Интересно,
что в теории, развитой Гротендиком, частным случаем
фундаментальной группы является группа
Галуа алгебраического замыкания поля.

\begin{figure}[ht]
\begin{center}
\epsfig{file=Grothendieck-hippie.eps,width=0.85\linewidth}\\
{Alexander Grothendieck \\
(род. 28 марта 1928)}
\end{center}
\end{figure}

Простейшая ситуация, когда этот подход к 
фундаментальной группе применим на практике,
таков. Рассмотрим пространство $\C\backslash \{x_1, ... x_n\}$
($\C$ без конечного набора точек). Пусть
$B \subset A$ --- конечные подмножества $\C$.
Легко видеть, что естественное отображение 
\[ \pi_1(\C \backslash A)\arrow \pi_1(\C \backslash B)\] --
наложение (докажите это).  Рассмотрим предел
$\pi_1(\C \backslash A)$ по увеличивающимся
конечным подмножествам $A\subset \C$. Получится группа, изоморфная
группе Галуа алгебраического замыкания
$\C(t)$. Доказать это нетрудно, если
интерпретировать (вслед за Риманом)
алгебраические расширения
поля рациональных функций как накрытия
$\C\backslash \{x_1, ... x_n\}$.


В аннотации к SGA1 говорится "этот текст
излагает теорию фундаментальных групп
в алгебраической геометрии с точки зрения Кронекера,
позволяя определить фундаментальную группу
одинаковым способом для алгебраического
многообразия (в обычном смысле слова) 
и, например, для кольца целых чисел в 
числовом поле." Что именно имел в виду
Гротендик, когда упоминал "теорию 
фундаментальных групп с точки зрения Кронекера",
в SGA1 не уточняется.


Первые два тома SGA были перенабраны в \LaTeX е 
Французским Математическим Обществом 
и положены в arxiv.org: \\
{\scriptsize\tt http://arxiv.org/abs/math/0206203} (SGA1)
и {\scriptsize\tt http://arxiv.org/abs/math/0511279} (SGA2).

Более современное изложение теории этальных накрытий,
этальных когомологий и этальной фундаментальной группы
есть в книжке Милна "Этальные когомологии", но эта книжка
требует хорошего знания алгебраической геометрии.

Чрезвычайно доступно изложено все то же самое
в книге В. И. Данилова "Когомологии алгебраических многообразий. 
"Итоги" ВИНИТИ, СПМ, фунд. напр., т. 35.

%%%%%%%%%%%%%%%%%%%%%%%%%%%%%%%%%%%%%%%%%%%%%%%%%%%%%%%%%%%%%%%%%%%%%%%%

\chapter[Лекция 17: Теорема Зейферта--ван Кампена]{Лекция 17: Свободные
группы и теорема
    Зейферта--ван Кампена}

%%%%%%%%%%%%%%%%%%%%%%%%%%%%%%%%%%%%%%%%%%%%%%%%%%%%%%%%%%%%%%%%%%%%%%%%

%%%%%%%%%%%%%%%%%%%%%%%%%%%%%%%%%%%%%%%%%%%%%%%%
\section{Фундаментальная группа и универсальное накрытие}
%%%%%%%%%%%%%%%%%%%%%%%%%%%%%%%%%%%%%%%%%%%%%%%%

Начнем с повтора той части материала прошлой
лекции, который понадобится сегодня.

Накрытия пространства $M$ образуют категорию. Объекты
этой категории --- накрытия, а морфизмы --- отображения
$\psi:\; \tilde M \arrow \tilde M'$, коммутирующие с
проекцией в $M$:
\[
\xymatrix{
\tilde M\ar[r]^\psi\ar[rd]^{\pi} & \tilde M'\ar[d]^{\pi'}\\
& M}
\]
Пусть $\tilde M  \arrow M$ --- накрытие,
причем $M$ и $\tilde M$ связны, локально линейно связны,
и $\tilde M$ односвязно, то есть имеет тривиальную
фундаментальную группу. Априори, фундаментальная 
группа $\pi_1(\tilde M,m)$ зависит от выбора точки 
$m$,  но $\pi_1(\tilde M,m_1)$ изоморфно $\pi_1(\tilde M,m_2)$,
если $m_1$ и $m_2$ лежат в одной компоненте линейной связности
(это доказано на прошлой лекции). 

Такое накрытие называется {\бф универсальным}.

Поскольку $\tilde M$ односвязно, оно является накрытием Галуа.
В самом деле, пусть $\tilde M' \stackrel \pi\arrow \tilde M$ --
связное накрытие, а $x\in \tilde M$ --- любая точка.
Пространство $\tilde M'$ локально линейно связно и
линейно связно, а значит связно.
Пусть $y_1, y_2\in \pi^{-1}(x)$ --- любые точки, а
$\gamma$ --- путь, который их соединяет. Образ этого пути
$\pi(\gamma)$ --- петля в $\tilde M$, идущая из $x$ в $x$,
но такая петля всегда стягиваема, поскольку $\pi_1(\tilde M, x)=0$.
Путь $\gamma$ является поднятием петли $\pi(\gamma)$,
и коль скоро $\pi(\gamma)$ стягиваема, $\gamma$ --- тоже петля
(это было доказано в предыдущей лекции). Поэтому 
$y_1=y_2$, и $\pi^{-1}(x)$  состоит из одного элемента.
Следовательно $\tilde M'\cong \tilde M$, и все накрытия $\tilde M$
расщепляются.

% В слайдах поместить сюда картинку с поднятием!

Пусть $M_1, M_2$ --- топологические пространства, снабженные
непрерывными отображениями $\psi_1, \psi_2$ в $M$.
Напомним, что {\бф расслоенным произведением}
$M_1$ на $M_2$ над $M$ называется
подмножество $M_1\times_M M_2$, состоящее
из всех пар $(m_1, m_2)\in M_1\times M_2$, таких, 
что $\psi_1(m_1)=\psi_2(m_2)$. Мы рассматриваем 
$M_1\times_M M_2$ как подмножество в $M_1\times M_2$,
с индуцированной топологией.

Обозначим через $\Psi:\; M_1\times M_2\arrow M\times M$
отображение 
\[ 
  \Psi(m_1, m_2)  =(\psi_1(m_1),\psi_2(m_2)).
\]
Легко видеть, что $M_1\times_M M_2= \Psi^{-1}(\Delta)$,
где $\Delta\subset M\times M$ --- диагональ.
Поэтому, для $M$ хаусдорфова,
расслоенное произведение --- замкнутое подмножество
в $M_1\times M_2$.

На предыдущей лекции было доказано, что
произведение накрытий --- снова накрытие.

Вернемся к универсальному накрытию $\tilde M \arrow M$.
Любое накрытие $\tilde M$ расщепляется. 
Рассмотрим $\tilde M \times_M \tilde M$ как накрытие $\tilde M$,
с проекцией $\tilde M \times_M \tilde M\arrow \tilde M$,
переводящей $(m_1, m_2)$ в $m_1$. 

Напомним, что накрытие называется {\бф накрытием Галуа}, если
$\tilde M \times_M \tilde M\arrow\tilde M$  расщепляется. 
В силу односвязности $\tilde M$, 
$\tilde M \times_M \tilde M$
расщепляется, и $\tilde M$ --- накрытие Галуа.

Поскольку $\tilde M$ локально связно,
любое его накрытие локально связно. Локально
связное пространство состоит из несвязного объединения
своих компонент связности. Поскольку все компоненты
связности $\tilde M \times_M \tilde M$ являются
связными накрытиями $\tilde M$, они ему изоморфны.

Мы получили, что
проекция каждой из компонент связности в $\tilde M \times_M \tilde M$
в $\tilde M$ --- гомеоморфизм, что по первому аргументу,
что по второму. Следовательно, каждая из этих компонент
является графиком автоморфизма накрытия $\tilde M$ над $M$.
Действительно, пусть $X$ --- такая компонента, тогда
проекции на первый и второй аргумент $\sigma_1, \sigma_2:\; X \arrow \tilde M$
это гомеоморфизмы, а $X$ --- график гомеоморфизма
$\sigma_1^{-1} \circ \sigma_2$ (проверьте это).

Из вышесказанного следует, что каждая точка
$(x,y) \in \tilde M \times_M \tilde M$ принадлежит
графику автоморфизма $\psi\in \Aut[\tilde M: M]$,
где $\Aut[\tilde M: M]$ --- группа автоморфизмов
накрытия $\tilde M$ над $M$ (она еще называется
{\бф группой Галуа накрытия $\tilde M$}).

Обозначим проекцию $\tilde M \arrow M$ за $\pi$.
Для каждого $m\in M$, и любой пары точек
$x, y \in \pi^{-1}(m)$, найдется автоморфизм $\nu$, графику
которого $\Gamma_\nu$ принадлежит точка $(x,y)\in \tilde M \times_M\tilde M$. 
В силу того, что $\nu$ переводит $x \in \tilde M$
в $\sigma_2(\sigma_1^{-1}(x,y))$, имеем $\nu(x)=y$
(проверьте это).

Мы получили, что $\Aut[\tilde M: M]$ транзитивно действует
на множестве $\pi^{-1}(m)$. Поскольку 
$\tilde M\times_M\tilde M$ --- объединение непересекающихся
графиков морфизмов, из $\nu_1(x)=\nu_2(y)$ следует
$\nu_1=\nu_2$. Следовательно, $G:=\Aut[\tilde M: M]$
действует на $\pi^{-1}(m)$ транзитивно и свободно.
Поэтому $M = \tilde M/G$, причем действие $G$ на
$\tilde M$ свободно.

Пусть $g\in \Aut[\tilde M: M]$ --- элемент группы Галуа
универсального накрытия, переводящий $x\in \tilde M$ в $g(x)$,
где $x\in \pi^{-1}(m)$.
Рассмотрим путь $\gamma$, соединяющий $x$ и $g(x)$.
Такой путь --- единственный, с точностью до гомотопии,
поскольку $\tilde M$ односвязно (докажите это).
Образ $\gamma \circ\pi:\; [0,1] \arrow M$ --
это петля. Обозначим ее класс в $\pi_1(M,m)$
за $\gamma_g$. Поскольку путь $\gamma$ единственный
с точностью до гомотопии, $\gamma_g$
зависит только от $x$ и $g$. Но коль скоро $\Aut[\tilde M: M]$
действует на $\pi^{-1}(m)$ транзитивно, эта группа
переводит путь, соединяющий $x, g(x)$, в путь, соединяющий
$y, g(y)$, где за $y$ можно взять любую точку в $\pi^{-1}(m)$.
Следовательно, $\gamma_g\in \pi_1(M,m)$
не зависит от выбора $x$, и однозначно
определяется автоморфизмом $g\in Aut[\tilde M: M]$.
Мы получили отображение $Aut[\tilde M: M]\arrow \pi_1(M,m)$.

Если $g_1, g_2\in Aut[\tilde M: M]$,
причем $g_1$ переводит $x$ в $y$, а $g_2$ переводит $y$ в
$z$, то $g_1g_2$ переводит $x$ в $z$. Произведение
путей $\gamma_{g_1}\gamma_{g_2}$ получается как 
$\pi(\gamma\gamma')$, где $\gamma$ --- путь
из $x$ в $y$, а $\gamma'$ --- путь из $y$ в $z$.
Следовательно, $\gamma_{g_1}\gamma_{g_2}=\gamma_{g_1g_2}$,
и построенное выше отображение 
$Aut[\tilde M: M]\arrow \pi_1(M,m)$ --- гомоморфизм групп.

В прошлой лекции мы убедились, что это изоморфизм
(проверьте). Мы получили следующую теорему.

\хфилл

%%%%%%%%%%%%%%%%%%%%%%%%%%%%%%%%%%%%
\теорема
Пусть $\tilde M  \arrow M$ --- универсальное накрытие,
причем $M$ и $\tilde M$ связны, локально линейно связны,
и $\tilde M$ односвязно. Рассмотрим группу Галуа этого накрытия,
$G=Aut[\tilde M: M]$. Тогда $M = \tilde M /G$, причем
фундаментальная группа $M$ изоморфна $G$.

\endproof


\замечание
Эта теорема позволяет вычислить $\pi_1(M,m)$ для 
пространств, полученных как фактор $\tilde M/G$
односвязного $\tilde M$ по группе $G$, действующей
на $\tilde M$ вполне несвязно.
Мы немедленно получаем, что фундаментальная
группа окружности $S^1$ равна $\Z$. Действительно,
$S^1= \R /\Z$, но $\pi_1(\R)=\{1\}$, так как $\R$
стягиваемо. Аналогичный аргумент показывает, что
$\pi_1(T^n)=\Z^n$, где $T^n=\R^n /\Z^n$ --- $n$-мерный тор.
\еза

\замечание
Пусть $\tilde M_1 \arrow M_1$, $\tilde M_2\arrow M_2$ --
универсальные накрытия, $\pi_1(М_i)=G_i$. 
Поскольку $\tilde M_1\times \tilde M_2$ 
односвязно (проверьте это) и накрывает $M_1\times M_2$
(проверьте), оно является универсальным накрытием
$M_1\times M_2$. Поэтому 
$\tilde M_1\times \tilde M_2/G_1\times G_2=M_1\times M_2$,
и $\pi_1(M_1\times M_2)=G_1\times G_2$. Мы получили,
что фундаментальная группа произведения двух 
пространств --- это произведение фундаментальных
групп этих пространств.
\еза

\замечание
Этот факт можно получить, и не пользуясь накрытиями: непрерывное
отображение из любого пространства $X$ в $M_1\times M_2$ --
пара отображений из $X$ в $M_1$ и из $X$ в $M_2$.
Поэтому то же самое верно для петель и для классов гомотопий
петель.
\еза



%%%%%%%%%%%%%%%%%%%%%%%%%%%%%%%%%%%%%%%%%%%%%%%%%%%%%%%%%%%%
\section{Категория накрытий и фундаментальная группа}
%%%%%%%%%%%%%%%%%%%%%%%%%%%%%%%%%%%%%%%%%%%%%%%%%%%%%%%%%%%%

\newcommand{\Rep}{\operatorname{{\cal R}ep}}
\newcommand{\Cov}{\operatorname{{\cal C}ov}}
\newcommand{\Sets}{\operatorname{{\cal S}ets}}

Пусть $G$ --- группа, а $\Rep(G, \Sets)$ --- категория,
объекты которой --- множества с действием $G$, а морфизмы
-- отображения множеств, совместимые с действием $G$.
Эта категория называется {\бф категорией множеств с
действием $G$}. 

Предположим, что $G= Aut[\tilde M: M]$,
где $[\tilde M: M]$ --- универсальное накрытие $M$.
Для каждого множества $S$ с действием $G$, рассмотрим
произведение $\tilde M \times S$, где $G$ действует по 
формуле $g(m,s) = (g(m), g(s))$ (такое действие называется
{\бф диагональным}). Взяв дискретную топологию на $S$,
можно считать, что $\tilde M \times S$ --- топологическое
пространство. Поскольку действие $G$ на $\tilde M \times S$ 
вполне разрывно (проверьте это), фактор 
$(\tilde M \times S)/G$ --- хаусдорфово топологическое
пространство. 

\замечание
Легко видеть, что $(\tilde M \times S)/G$
является накрытием $M$. В самом деле,
локально по $M$, $\tilde M$ является
тривиальным накрытием, изоморфным $M \times G$,
где $G$ рассматривается как топологическое
пространство с дискретной топологией. В таком случае,
$(\tilde M \times S)/G$ изоморфно $M \times X$,
следовательно, стандартная проекция
$(\tilde M \times S)/G\stackrel {\pi_X} \arrow M$ --- накрытие.
Для каждой точки $m \in M$, множество $\pi_x^{-1}(m)$
снабжено биекцией на $(G\times X)/G=X$.
\еза

%%%%%%%%%%%%%%%%%%%%%%%%%%%%%%%%%%%%%%%%%%%%%%%%
\теорема\label{_mnozh_w_dejstv_nakry_Teorema_}
Пусть $\tilde M \arrow M$ --- универсальное
накрытие, а $G$ --- его группа Галуа, 
$G=Aut[\tilde M: M]$. Рассмотрим функтор 
 \[ \Rep(G, \Sets)\stackrel \Psi\arrow \Cov(M)\]
из $\Rep(G, \Sets)$ в категорию $\Cov(M)$ накрытий $M$,
построенный выше. Тогда $\Psi$ --- эквивалентность категорий.

\хфилл

\ноиндент
{\бф Доказательство:} Возьмем точку $m \in M$.
Обратный функтор \[ \Cov(M)\arrow \Rep(G, \Sets)\] строится
весьма просто. Пусть $M' \stackrel \sigma \arrow M$ --
накрытие $M$, а $S_{M'}:=\sigma^{-1}(m)$ --- его
слой над $m$. На множестве $S_{M'}$ задано действие
$\pi_1(M)$, следующим образом. Пусть $x \in S_{M'}$ --
любая точка, а $g\in G$ --- элемент фундаментальной группы.
Возьмем петлю $\gamma_g\in \Omega(M,m)$, представляющую
$g$, и пусть $\tilde \gamma_g$ --- ее поднятие в $m'$,
с начальной точкой в $x$. Такое поднятие существует
и единственно, как доказано в предыдущей лекции,
и конечная точка $\tilde \gamma_g(1)$ однозначно
определяется $x$ и $g$. Таким образом, каждое
$g\in G$ задает отображение из $S_{M'}$ в $S_{M'}$ 
Легко видеть, что произведение двух петель соответствует
композиции отображений, значит, $x \arrow \tilde \gamma_g(1)$ 
задает действие $G$ на $S_{M'}$. При морфизме накрытий,
поднятия путей переходят в поднятия путей, следовательно,
$M' \arrow S_{M'}$ задает функтор  
$\Cov(M)\stackrel \Phi\arrow \Rep(G,\Sets)$.

Композиция $\Psi \circ\Phi$ переводит множество
с действием $G$ в то же самое множество, что ясно из определения.

Чтобы убедиться, что композиция $\Phi \circ\Psi$ переводит накрытие
в эквивалентное ему, достаточно проверить это на связных
накрытиях. Но связные накрытия имеют вид $\tilde M/G'$,
для подгрупп $G'\subset G$ (это утверждение называется
"основная теорема теории Галуа для накрытий", и оно
доказано в прошлой лекции). Применение
функтора $\Phi$ к $M/G'$ дает, очевидно, множество
$\Phi(\tilde M/G')=G/G'$ с естественным действием $G$.
По построению,
\[ 
  \Psi(G/G') = (\tilde M \times G/G')/G= \tilde M/G'
\]
(проверьте это). Значит, функтор $\Phi \circ\Psi$ эквивалентен
тождественному. \endproof


\замечание
Каждое накрытие изоморфно факторнакрытию несвязной
суммы нескольких копий универсального:
\[
M' = \bigsqcup_i \tilde M/G_i;
\]
в соответствие такому накрытию можно поставить
множество
\[
S = \bigsqcup_i G/G_i
\]
с естественным действием $G$. Это задает эквивалентность
категории накрытий и категории множеств с действием $G$ чуть более
явно. Докажите самостоятельно, что это та же самая
эквивалентность категорий, что была построена выше.
\еза


%%%%%%%%%%%%%%%%%%%%%%%%%%%%%%%%%%%%%%%%%%%%%%%%%%%%%%%%%%%%
\section{Как восстановить фундаментальную группу по 
категории накрытий}
%%%%%%%%%%%%%%%%%%%%%%%%%%%%%%%%%%%%%%%%%%%%%%%%%%%%%%%%%%%%


\замечание
Пусть $G$ --- группа, а $\cac=\Rep(G, \Sets)$ --- категория
множеств с действием $G$. Рассмотрим  $G$
как объект $\cac$, то есть множество с действием
$G$, определенным формулой $(g, x)\arrow gx$. 
Тогда все морфизмы из $G$ в себя обратимы,
и группа $\Aut(G)=\Mor(G,G)$ изоморфна $G$. В самом деле,
пусть $\nu\in \Mor(G, G)$ переводит $1$ в $x$.
Тогда $\nu(g1) = g(\nu(x))= gx$. Такой морфизм,
очевидно, биективен, а композиция  $\nu_x\nu_y$ равна $\nu_{xy}$.
\еза


\лемма
Пусть $\bar G \in \Ob(\cac)$
-- объект $\Rep(G, \Sets)$, который обладает следующими
свойствами
\begin{description}
\item[(i)] Все морфизмы из $\bar G$ в себя являются изоморфизмами
\item[(ii)] Для любого $X\in \Ob(\cac)$, множество
$\Mor(\bar G, X)$ непусто.
\end{description}
Тогда $\bar G$ изоморфен $G$ как объект $\cac$.

\hfill

\noindent
{\bf Доказательство. Шаг 1:} Все объекты $\cac$ получены
объединением непересекающихся орбит действия $G$.
Каждая такая орбита имеет вид $G/G'$, где $G'$ --- стабилизатор
точки. Нетривиальные морфизмы из орбиты
$G/G_1$ в $G/G_2$ возможны, только если
$G_2\supset G_1$ (проверьте это).

\hfill

\noindent
{\bf Шаг 2:} Поскольку $\Mor(\bar G, X)$ всегда 
непусто, $\bar G$ --- объединение орбит вида $G/G_i$,
среди которых хотя бы одна изоморфна $G$.


\hfill

\noindent
{\bf Шаг 3:} Пусть $\bar G$ содержит две орбиты $X_1,
X_2$, причем $X_1$ изоморфна $G$. Рассмотрим морфизм из $\bar G$
в себя, тождественный на всех орбитах, кроме  $X_1$, и
отображающий $X_1$ в $X_2= G /G'$ (такой морфизм существует
в силу Шага 1). Такой морфизм не биективен, значит, 
он не может быть изоморфизмом.



\hfill

\noindent
{\bf Шаг 4:} Мы получили, что $\bar G$ состоит из одной орбиты
(шаг 3), причем эта орбита изоморфна $G$ (шаг 2).
Это доказывает, что $\bar G$ изоморфно $G$. \endproof

%%%%%%%%%%%%%%%%%%%%%%%%%%%%%%%%%%%%
\замечание\label{_vosst_G_iz_Rep_Zamechanie_}
Из доказанной выше леммы следует, что группа $G$ однозначно
восстанавливается по категории $\Rep(G, \Sets)$: если
категории $\Rep(G_1, \Sets)$, $\Rep(G_2, \Sets)$
эквивалентны, то группы $G_1, G_2$ изоморфны.
\еза


%%%%%%%%%%%%%%%%%%%%%%%%%%%%%%%%%%%%%%%%%%%%%%%%%%%%%%%%%%%%
\section{Свободная группа и свободное произведение групп}
%%%%%%%%%%%%%%%%%%%%%%%%%%%%%%%%%%%%%%%%%%%%%%%%%%%%%%%%%%%%

Пусть $G, H$ --- группы. Рассмотрим множество $\overline{G*H}$,
состоящее из последовательностей (слов) вида
$g_1h_1g_2h_2...g_nh_n$, где $g_i\in G$, $h_i\in H$.
Рассмотрим соотношение эквивалентности, порожденное
\begin{equation}\label{_sokra1_v_gru_Equation_}
g_1h_1g_2h_2...g_i h_i g_{i+n}h_{i+n} ... g_nh_n \sim 
g_1h_1g_2h_2... g_i  g_{i+n} g_nh_n
\end{equation}
если $h_i=1$, и 
\begin{equation}\label{_sokra2_v_gru_Equation_}
g_1h_1g_2h_2...g_i h_i g_{i+n}h_{i+n} ... g_nh_n \sim 
g_1h_1g_2h_2... g_i h_ih_{i+n} g_{i+2}...   g_nh_n
\end{equation}
если $g_{i+n}=1$. Иными словами, в каждом слове
все сочетания вида $g_1 1 g_2$ можно заменить на
$g_1g_2$, а $h_1 1 h_2$  на $h_1h_2$.
Множество классов эквивалентности обозначается $G*H$.
Слова можно умножать:
\[
g_1h_1g_2h_2...g_nh_n \cdot
g'_1h'_1g'_2h'_2...g'_{n'}h'_{n'}:=g_1h_1g_2h_2...g_nh_ng'_1h'_1g'_2h'_2...g'_{n'}h'_{n'}.
\]
Такое умножение, очевидно, ассоциативно.
Из \eqref{_sokra1_v_gru_Equation_} и \eqref{_sokra2_v_gru_Equation_}
немедленно следует, что 
\[ 
  g_1h_1g_2h_2...g_nh_nh_n^{-1}
  g_n^{-1}... h_2^{-1}g_2^{-1}h_1^{-1}g_1^{-1}=1,
\]
значит, $G*H$ это группа.

\определение
Группа $G*H$ называется {\бф свободным произведением},
или же {\бф амальгамой}, или же {\бф копроизведением} групп
$G$ и $H$.
\ео

\задача
Проверьте, что это произведение ассоциативно:
\[
(F*G)*H=F*(G*H).
\]
\ез

\замечание
Аналогичным образом определяется свободное произведение
произвольного набора групп $\{G_\alpha\}$, $\alpha\in I$.
Пусть множество $\overline{\coprod_\alpha G_\alpha}$ состоит из
 слов вида $g_1g_2g_3...g_n$, составленных из букв 
$g_i\in G_\alpha$.
Рассмотрим соотношение эквивалентности, порожденное
\[
g_1g_2...g_ig_{i+1}g_{i+2}...g_n\sim g_1g_2...g_ig_{i+2}...g_n
\]
если $g_{i+1}=1$ (можно выкинуть из слова букву $g_{i+1}$, если
$g_{i+1}$ равно 1), и 
\[
g_1g_2...g_{i-1}g_ig_{i+1}g_{i+2}...g_n\sim 
g_1g_2...g_{i-1}(g_ig_{i+1})g_{i+2}...g_n
\]
если $g_i, g_{i+1}\in G_\alpha$
(можно сгруппировать последовательно идущие 
буквы $g_i, g_{i+1}$ в $(g_ig_{i+1})$, если
они обе принадлежат одной и той же группе
$G_\alpha$. Произведение и обратный элемент
в 
\[ 
\coprod_\alpha G_\alpha := \overline{\coprod_\alpha
G_\alpha}/\sim
\]
определяется той же самой формулой, что и для 
 $G*H$; аналогичный аргумент показывает,
что это группа (докажите это). Проверьте,
что это определение для двух сомножителей
дает $G*H$. 
\еза

\определение
Определенная таким образом группа $\coprod_\alpha G_\alpha$ 
называется  {\бф свободным произведением},
или же {\бф амальгамой}, или же {\бф копроизведением}
набора $\{G_\alpha\}$.
\ео

\определение
Копроизведение
$\Z * \Z * \Z * ... *\Z$ ($n$ раз) называется 
{\бф свободной группой от $n$ образующих}. 
Копроизведение вида
$\coprod_\alpha G_\alpha,$ где все $G_\alpha\cong \Z$,
называется {\бф свободной группой}.
\ео




\замечание
Легко видеть, что естественное отображение любого 
сомножителя в свободное произведение,
$G_0 \arrow \coprod_\alpha G_\alpha$, 
$g_0 \arrow g_0$ является вложением (докажите это).
\еза




%%%%%%%%%%%%%%%%%%%%%%%%%%%%%%%%%%%%%%%%%%%%%%%%
\section{Представимые функторы}
%%%%%%%%%%%%%%%%%%%%%%%%%%%%%%%%%%%%%%%%%%%%%%%%

Пусть $\cac$ --- категория, а $A\in \Ob(\cac)$
некоторый объект. Определим {\бф хом-функтор} 
$h_A:\; \cac \arrow \Sets$ формулой
$h_A(X) = \Mor(A,X)$.

Пусть $F:\; \cac \arrow \Sets$ --
функтор из $\cac$ в множества. Напомним, что $F$
называется {\бф представимым}, если $F$ эквивалентен
хом-функтору $h_A$ для какого-то объекта
$A\in \Ob(\cac)$. В этой ситуации, $A$ называется
{\бф представляющим объектом} для функтора $F$.

\определение
Пусть $\cac, \cac'$ --- категории, а $F_1, F_2:\; \cac\arrow \cac'$
функторы из $\cac$ в $\cac'$. {\бф Естественное
преобразование функторов} 
\[ \Phi:\; F_1\arrow F_2\]
задается так. Для каждого $X\in \Ob(\cac)$, 
задан морфизм \[ \Phi_X\in \Mor(F_1(X), F_2(X),\] 
таким образом, что следующая диаграмма коммутативна для любого
$\lambda\in \Mor(X,Y)$:
\[
\begin{CD}
F_1(X) @>{F_1(\lambda)}>> F_1(Y)\\
@V{\Phi_X}VV @VV{\Phi_Y}V\\
F_2(X) @>{F_2(\lambda)}>> F_2(Y)
\end{CD}
\]
Отметим, что частным случаем
естественного преобразования функторов является эквивалентность
функторов, определенная в лекции 14.
\ео

Если $\Phi:\; h_A \arrow F$ --- естественное преобразование
функторов из $\cac$ в множества, $\Phi$ однозначно
определяется элементом $\Phi(\Id_A) \in F(A)$. Это можно
видеть из следующей коммутативной диаграммы:
\[
\begin{CD}
\Mor(A,A) @>{f \mapsto f\circ\lambda}>> \Mor(A,B)\\
@V{\Phi_A} VV @VV{\Phi_B}V\\
F(A) @>{F(\lambda)}>> F(B)
\end{CD}
\]
где $\lambda\in \Mor(A,B)$ --- любой морфизм.
Верхняя стрелка переводит $\Id_A$ в $\lambda$. 
Из коммутативности этой диаграммы получаем, что 
\[ \Phi_B(\lambda)=F(\lambda)(\Phi_A(\Id_A)),
\]
значит, отображение $\Phi_B:\; \Mor(A,B)\arrow  F(B)$
целиком определяется элементом $\Phi_A(\Id_A)\in \Mor(A,A)$.
Мы получили следующую полезную лемму

\хфилл

\лемма
(Лемма Ионеды, Yoneda lemma) 
Пусть $\cac$ --- категория, $A\in \Ob(\cac)$ ее объект,
$h_A:\; \cac \arrow \Sets$  --- хом-функтор, а $F:\; \cac \arrow \Sets$
еще функтор. Тогда множество естественных преобразований
функтора $h_A$ в $F$ биективно с $F(A)$.

\endproof

\hfill

Функторы  $F:\; \cac \arrow \Sets$ сами образуют
категорию: объекты этой категории --- функторы
$F:\; \cac \arrow \Sets$, а морфизмы --- естественные
преобразования. Из леммы Ионеды немедленно следует, что
$\Mor(h_A, h_B) = \Mor(B,A)$. Мы получили такое
утверждение.

\хфилл

\утверждение
Пусть $\cac$ --- категория, а ${\cal F}$ --- категория
представимых функторов $F:\; \cac \arrow \Sets$,
с морфизмами, которые задаются естественными
преобразованиями функторов. Тогда контравариантный 
функтор $A \arrow h_A$
задает эквивалентность категорий $\cac\arrow {\cal F}^\circ$.

\endproof

%%%%%%%%%%%%%%%%%%%%%%%%%%%%%%%%%%%%%%%%%%%%%%%%
\замечание\label{_predstavlya_edinstv_Zamechanie_}
В частности, объект категории, представляющий заданный
функтор, определяется этим функтором однозначно с
точностью до изоморфизма.
\еза

Мы излагали лемму Ионеды и эквивалентность
категорий для ковариантных функторов, но то же самое
верно и для контравариантных функторов, с заменой
ковариантного функтора $h_A(X) =\Mor(A,X)$ на 
контравариантный $h^\circ_A(X) = \Mor(X,A)$.

\begin{figure}[ht]
\begin{center}
\epsfig{file=yoneda-nobuo.eps,width=0.72\linewidth}\\
{Nobuo Yoneda\\
(1930-1996)}
\end{center}
\end{figure}


%%%%%%%%%%%%%%%%%%%%%%%%%%%%%%%%%%%%%%%%%%%%%%%%%%%%%%%%%%%%
\section{Лемма Ионеды: история, замечания}
%%%%%%%%%%%%%%%%%%%%%%%%%%%%%%%%%%%%%%%%%%%%%%%%%%%%%%%%%%%%


Нобуо Ионеда был студентом Шокичи Иянага (Shokichi Iyanaga,
1906-2006) в университете Токио. В начале 1950-х Самуэль Эйленберг
путешествовал по Японии, и Ионеда, еще студентом, 
сопровождал его  в качестве переводчика.
Закончив университет, Ионеда отправился продолжать
учебу к Эйленбергу в Принстон. В скором времени после этого,
Эйленберг уехал в Париж (Эйленберг был участником
группы Бурбаки), и в 1954-м году Ионеда поехал в Париж к 
Эйленбергу. В Париже Ионеда имел длинную беседу с 
Сондерсом  Маклейном (сначала в кафе перед Гар дю Нор, а
потом на вокзале; они закончили разговаривать
в дверях отбывающего поезда). Под впечатлением
этой беседы, Маклейн назвал лемму, которую 
он узнал от Ионеды на вокзале, леммой Ионеды.

Кроме знаменитой леммы, Ионеда придумал интерпретацию
групп $Ext^*$ в терминах расширений модулей, чем
положил начало современной гомологической алгебре.

После возвращения в Японию Ионеда не занимался 
математикой. В начале 1960-х он заинтересовался 
компьютерными науками, под влянием Шокичи Иянага, 
и был одним из авторов  языка Алгол 68, а после 
создания факультета компьютерных наук стал профессором
компьютерной науки в университете Токио.

Интересно, что лемма Ионеды сейчас входит в стандартный
курс компьютерных наук, ибо на теории категорий в значительной
степени построена теоретическая основа функционального
программирования.

Помимо Ионеды, Иянага был наставником 
Горо Азумая (изобретателя алгебр Азумаи), 
Кунихико Кодаира (открывшего сотни ключевых
теорем алгебраической и комплексной геометрии),
Кенкичи Ивасава (знаменитого автора разложения
Ивасавы и важных теорем теории чисел)
и Тсунео Тамагава, прославленного 
"числами Тамагавы".

Иянага был учеником Тейджи Такаги (1875-1960), 
который был студентом Давида Гильберта. Такаги 
был автором десятков учебников, обеспечив Японию 
университетскими учебниками по всем разделам математики.
Такаги занимался теорией чисел, и в 1898 году придумал
аксиоматическое определение поля.

%%%%%%%%%%%%%%%%%%%%%%%%%%%%%%%%%%%%%%%%%%%%%%%%%%%%%%%%%%%%
\section{Произведение и копроизведение в категории}
%%%%%%%%%%%%%%%%%%%%%%%%%%%%%%%%%%%%%%%%%%%%%%%%%%%%%%%%%%%%


Свободную группу можно охарактеризовать следующим
свойством.

\хфилл

%%%%%%%%%%%%%%%%%%%%%%%%%%%%%%%%%%%%
\утверждение\label{_universa_copro_Utverzhdenie_}
Пусть 
$G$, $H$ --- группы, $G*H$ их свободное произведение,
а $G \stackrel \iota \hookrightarrow G*H$, 
$H\stackrel \iota \hookrightarrow G*H$ --- естественные
вложения. Тогда каждая пара гомоморфизмов $G\stackrel \phi\arrow P$,
$H\stackrel \psi\arrow P$ в группу $P$ продолжается  
до гомоморфизма $G*H\stackrel {\phi*\psi}\arrow P$ таким образом, что
следующая диаграмма коммутативна:
\[
\xymatrix @R12mm @C15mm {
G\ar[rd]^\iota\ar[rdd]_\phi &&  H\ar[ld]_\iota\ar[ldd]^\psi\\
& G*H \ar[d]^{\phi*\psi}&\\
&P&}
\]
Более того, гомоморфизм $\phi*\psi$
определяется отображениями $\psi$ и $\psi$
однозначно.

\хфилл

\ноиндент
{\бф Доказательство:} Пусть $\phi*\psi$ переводит
слово вида $g_1h_1g_2h_2...g_nh_n $ в 
$\phi(g_1)\psi(h_1)\phi(g_2)\psi(h_2)...\phi(g_n)\psi(h_n)$.
Поскольку это отображение переводит эквивалентные слова
в эквивалентные, и произведение слов в произведение,
оно является гомоморфизмом. Единственность такого $\phi*\psi$
очевидна, потому что группа $G*H$ порождена образами $G$ и $H$.
\endproof

\замечание
Утверждение \ref{_universa_copro_Utverzhdenie_}
называется {\бф универсальным \\свойством копроизведения}. 
Его можно взять в качестве определения $G*H$. Действительно,
в силу универсального свойства,
$G*H$ является представляющим объектом для 
функтора $P \arrow \Mor(G,P)\times \Mor(H,P)$
из групп в множества, а такой объект единственный
(Замечание \ref{_predstavlya_edinstv_Zamechanie_}).
\еза

\замечание
Произведение
и копроизведение можно определить на языке категорий:
произведение объектов $A, B \in \Ob(\cac)$ --- объект,
представляющий функтор \[ X \arrow \Mor(X,A) \times \Mor(X,B),\]
а копроизведение объектов $A, B \in \Ob(\cac)$ --
объект, представляющий функтор \[ X\arrow \Mor(A,X) \times \Mor(B,X).\]
Аналогичным образом определяется произведение и
копроизведение любого набора объектов. Произведение
и копроизведение не всегда существует, но определено
однозначно с точностью до изоморфизма, что следует 
из леммы Ионеды (Замечание \ref{_predstavlya_edinstv_Zamechanie_}).
\еза

\замечание
В большинстве известных категорий (группы, множества,
топологические пространства, векторные пространства), 
определение произведения согласовано с категорным 
определением, приведенным выше (проверьте это).
\еза

\замечание
В категории топологических пространств,
копроизведение --- это несвязная сумма;
в категории множеств, копроизведение --
несвязное объединение, а в категории
векторных пространств, копроизведение --
это прямая сумма пространств. Проверьте это.
\еза

%%%%%%%%%%%%%%%%%%%%%%%%%%%%%%%%%%%%%%%%%%%%%%%%%%%%%%%%%%%%

\section{История свободной группы и копроизведений}

%%%%%%%%%%%%%%%%%%%%%%%%%%%%%%%%%%%%%%%%%%%%%%%%%%%%%%%%%%%%

Свободную группу открыл немецкий математик Вальтер 
фон Дайк, ученик Клейна.
Фон Дайк в 1882-м году первым дал современное определение группы
и был автором топологической классификации неориентируемых римановых
поверхностей (двумерных многообразий), которую он получил
в 1888 году. 


\begin{figure}[ht]
\begin{center}
\epsfig{file=von-Dyck.eps,width=0.47\linewidth}\\
{Walther Franz Anton von Dyck\\
(1856-1934)}
\end{center}
\end{figure}
Фон Дайка интересовали фуксовы группы (дискретные
подгруппы группы изометрий плоскости Лобачевского).
В той же самой работе 1882-го года, в которой он
определил группы, фон Дайк обратил внимание на 
то, что некоторые фуксовы группы являются 
в некотором смысле простейшими; это и были 
свободные группы. Еще фон Дайк был создателем 
Немецкого Музея Науки и Технологии
и издал полное собрание писем Кеплера.

Название "свободная группа"
принадлежит Якобу Нильсену (Jacob Nielsen), датскому математику,
ученику Макса Дэна (Max Dehn),
который в 1924-м году 
доказал, что любая подгруппа конечно-порожденной
свободной группы свободна. 

Категорное определение свободного произведения
как объекта, представляющего функтор
\[ X\arrow \Mor(A,X) \times \Mor(B,X),\]
принадлежит Сондерсу Маклейну
который в работе "Groups, categories and duality"
(1948), нашел категорные интерпретации множеству
 понятий общей  (универсальной) алгебры.

%%%%%%%%%%%%%%%%%%%%%%%%%%%%%%%%%%%%%%%%%%%%%%%%
\section{Теорема Зейферта--ван Кампена}
%%%%%%%%%%%%%%%%%%%%%%%%%%%%%%%%%%%%%%%%%%%%%%%%

\теорема
(теорема Зейферта--ван Кампена, "Seifert-van Kampen Theorem")
Пусть $M=X\cup Y$ --- объединение топологических
пространств $X$ и $Y$, причем $X,Y$ локально 
связные, локально односвязные, замкнутые в $M$ и связные,
а $X \cap Y$ связно и односвязно.
Тогда $\pi_1(M)$ изоморфно свободному
произведению $\pi_1(X) * \pi_1(Y)$.

\hfill

\noindent
{\bf Доказательство. Шаг 1:}
Категория $\Rep(G*H, \Sets)$ эквивалентна категории $\cac$
множеств, на которых задано действие $G$ и $H$ (априори,
никак не согласованное). Функтор из $\Rep(G*H, \Sets)$
в $\cac$ переводит $S$ в то же самое множество $S$, где
действие $G$ и $H$ определяется из вложений $G \hookrightarrow G*H$,
$H \hookrightarrow G*H$. Обратный функтор определяется
из того, что для каждого множества $I$ с действием $G$ и $H$
заданы гомоморфизмы $G, H \arrow \Sigma_I$, где $\Sigma_I$ --
группа биективных отображений из $I$ в $I$. В силу универсального
свойства $G*H$, такие гомоморфизмы однозначно продолжаются до гомоморфизма
$G*H \arrow \Sigma_I$. Это задает функтор из $\cac$ в $\Rep(G*H, \Sets)$,
очевидно, обратный исходному.

\hfill

\noindent
{\bf Шаг 2:} В силу Замечания \ref{_vosst_G_iz_Rep_Zamechanie_},
группа $G$ однозначно задается своей категорией $\Rep(G, \Sets)$.
Поэтому для изоморфизма $\pi_1(M)\cong \pi_1(X) * \pi_1(Y)$,
достаточно убедиться, что категория накрытий $M$
эквивалентна $\Rep(\pi_1(X) * \pi_1(Y), \Sets)$.
В силу предыдущего шага, эта категория эквивалентна
категории $\cac$ множеств с действием $\pi_1(X)$ и $\pi_1(Y)$.
Таким образом, теорема Зейферта--ван Кампена будет доказана,
если мы докажем, что категория $\Cov(M)$ накрытий $M$ эквивалентна $\cac$.



\hfill

\noindent
{\bf Шаг 3:} Функтор из $\Cov(M)$ в $\cac$ построить
весьма нетрудно. Пусть $\tilde M\stackrel\sigma\arrow M$ --- накрытие $M$.
Соответствующие накрытия 
\[ \tilde X = \sigma^{-1}(X)\arrow X, \ \ 
\tilde Y = \sigma^{-1}(Y)\arrow Y
\] называются {\бф
ограничениями $\tilde M$ на $X$ и $Y$}. Для каждой точки
$m \in  X\cap Y$, прообраз $\sigma^{-1}(m)\subset \tilde X$
наделен действием $\pi_1(X)$ 
(Теорема \ref{_mnozh_w_dejstv_nakry_Teorema_}).
Прообраз $\sigma^{-1}(m)\subset \tilde Y$
наделен действием $\pi_1(Y)$, по той же самой причине.
Мы получили, что $\tilde M \arrow \sigma^{-1}(m)$
задает функтор $\Cov(M) \stackrel \Psi\arrow \cac$.
Осталось доказать, что это эквивалентность.

\hfill

\noindent
{\bf Шаг 4:} Пусть $S \in \Ob(\cac)$ --- множество
с действием $\pi_1(X)$ и $\pi_1(Y)$,
а $\tilde X\stackrel {\sigma_X} \arrow X$,
$\tilde Y\stackrel {\sigma_Y} \arrow Y$ --- соответствующие
накрытия, существующие в силу эквивалентности
категории накрытий и категории множеств с действием
фундаментальной группы (Теорема
\ref{_mnozh_w_dejstv_nakry_Teorema_}). 
Поскольку $X\cap Y$ связно, локально связно 
и односвязно, 
\[ 
  \sigma^{-1}_X(X\cap Y)= \sigma^{-1}_Y(X\cap Y)= X\cap Y\times S,
\]
 где $S$ --- множество связных компонент $\tilde S$.
Пусть $x \in \tilde X, y\in \tilde Y$,
причем $x\in \sigma^{-1}_X(X\cap Y)$
равен $y\in \sigma^{-1}_Y(X\cap Y)$
при этом отождествлении. В таком случае
мы напишем $x\sim y$. Пусть 
$\tilde M= \tilde X \bigsqcup \tilde Y/\sim$ --- фактор
по отношению эквивалентности, определенному
таким способом. Легко видеть, что естественная
проекция $\sigma:\; \tilde M \arrow M$ --- накрытие
(проверьте это). Действительно, проекция $\sigma$ этальна,
потому что $M$ получается как фактор $X \bigsqcup Y$ по
отношению эквивалентности\footnote{Для доказательства
этого используйте замкнутость $X$ и $Y$ в $M$.} 
\[ m\sim m' \Leftrightarrow m\in X, m'\in Y, \text{\ \
  $m=m'$ в $X\cap Y$,}
\]
 а прообраз
$\sigma^{-1}(U)$ по построению гомеоморфен $U \times S$,
если $\sigma_X$, $\sigma_Y$ расщепляется 
над $U\cap X$, $U\cap Y$.

Мы построили функтор из $\cac$ в $\Cov(M)$.
Легко видеть, что этот функтор обратен $\Psi$
(проверьте это). Поэтому $\Psi$ --- это эквивалентность
категорий. 

Мы доказали, что категория $\Cov(M)$ накрытий
$M$ эквивалентна категории $\cac$ множеств с действием
$\pi_1(X)$ и $\pi_1(Y)$. На шаге 1 было доказано, что
$\cac$ эквивалентна категории $\Rep(\pi_1(X) * \pi_1(Y), \Sets)$.  В силу
Теоремы \ref{_mnozh_w_dejstv_nakry_Teorema_},
$\Cov(M)$ эквивалентна $\Rep(\pi_1(M), \Sets)$.
Используя Замечание \ref{_vosst_G_iz_Rep_Zamechanie_},
мы выводим из эквивалентности категорий
\[ \Rep(\pi_1(X) * \pi_1(Y), \Sets) \sim \Rep(\pi_1(M), \Sets)\]
 изоморфизм групп
\[ \pi_1(X) * \pi_1(Y)\cong \pi_1(M).\] \endproof

\замечание
Если $M= \bigcup M_i$, причем все частичные
пересечения $M_{i_1}\cap M_{i_2} \cap ...$
односвязны, то 
$\pi_1(M) \cong \pi_1(M_1) * \pi_1(M_2) *...$
Для конечного набора $M_i$ это можно получить
индукцией из теоремы Зейферта--ван Кампена,
пользуясь ассоциативностью копроизведения.
Для бесконечного набора, индуктивный аргумент не
годится. Впрочем, легко видеть, что доказательство,
приведенное выше, работает для любого числа
$M_i$, ценой неимоверного усложнения обозначений.
Проверьте это самостоятельно.
\еза



\begin{figure}[ht]
\begin{center}
\epsfig{file=Van_Kampen.eps,width=0.47\linewidth}\\
{ Egbert Rudolf van Kampen\\
(1908-1942)}
\end{center}
\end{figure}


%%%%%%%%%%%%%%%%%%%%%%%%%%%%%%%%%%%%%%%%%%%%%%%%%%%%%%%%%%%%
\section{История, замечания}
%%%%%%%%%%%%%%%%%%%%%%%%%%%%%%%%%%%%%%%%%%%%%%%%%%%%%%%%%%%%

Теорема Зейферта--ван Кампена была получена ван Кампеном 1933-м
году, в более общей ситуации, чем описано выше. Ван Кампен
вычислил фундаментальную группу объединения двух 
топологических пространств, не предполагая связности
и односвязности их пересечения. Аргумент, подобный
вышеприведенному, в такой ситуации не работает, 
потому что если пересечение $X$ и $Y$ несвязно,
то непонятно, куда поместить отмеченную
точку.

Современная точка зрения
на этот результат требует применения {\ем фундаментального
группоида}. {\бф Группоидом} называется категория,
все морфизмы которой --- изоморфизмы. Пусть $M$ --- топологическое
пространство. {\бф Фундаментальным группоидом} $M$
называется категория, объекты которой --- точки $M$,
а морфизмы из $x$ в $y$ --- классы гомотопии путей 
из $x$ в $y$. Композиция морфизмов соответствует
обычному умножению путей. Фундаментальный группоид
содержит всю информацию о фундаментальной группе,
но не требует неестественного выбора базовой точки.
Гротендик по этому поводу сказал так:

\хфилл

{\ем ...люди привыкли работать с фундаментальной группой,
ее образующими и соотношениями, и продолжают делать это
  даже в контексте, где такой подход абсолютно
  неадекватен, когда ясное описание группы образующими
и соотношениями можно получить только работая
  одновременно с целой кучей базовых точек, правильно
отобранных --- либо получая эквивалентное описаниюе
в алгебраическом контексте группоидов, а не групп.
Выбор путей, соединяющих базовые точки, и сведение
группоида к одной группе, безнадежно разрушает структуру
и внутренние симметрии, присущие геометрической ситуации,
приводя к безобразной куче образующих и соотношений,
которые никто и не пытается выписать, потому что
все понимают, что никакой пользы это не принесет,
и запутает картину вместо того, чтобы прояснить ее.
Я узнал об этой трудности много лет назад, в ситуациях
типа ван Кампена, когда единственная осмысленная
формулировка задачи может быть получена только
в терминах амальгамы группоидов... }

\хфилл

Теорему Зейферта--ван Кампена можно сформулировать 
следующим образом. Пусть $U$ и $V$ --- топологические
пространства. Тогда фундаментальный группоид
$U\cup V$ --- это расслоенное копроизведение фундаментальных
группоидов $U$ и $V$ (взятое в категории группоидов).
В такой общности, доказательство теоремы Зейферта--ван Кампена 
получается значительно проще, чем доказательство
ее более слабой версии, приведенное выше.

На странице Рональда Брауна (Ronald
Brown)\footnote{\tt\small http://www.bangor.ac.uk/$\tilde\ $mas010/topgpds.html}
выложены его обзоры теории групоидов и любопытная
переписка с Гротендиком, который много пропагандировал 
применение групоидов в топологии.



Теорему Зейферта--ван Кампена довольно часто называют
"теорема ван Кампена". Эгберт ван Кампен (1908-1942)
получил образование в Нидерландах, но в 1931-м году
уехал в Америку, где сотрудничал с Оскаром Зариским. 
Зариский пытался посчитать фундаментальную группу
дополнения к алгебраической кривой, и совместно
с ван Кампеном добился успеха. Теорема Зейферта--ван Кампена
получилась как побочный продукт этого сотрудничества.

Ван Кампен был вундеркинд (он защитил диссертацию 
на 21-м году жизни), и умер весьма молодым, от рака.

%%%%%%%%%%%%%%%%%%%%%%%%%%%%%%%%%%%%%%%%%%%%%%%%%%%%%%%%%%%%%%%%%%%%%%%%

\chapter[Лекция 18: Подгруппы в свободных группах]{Лекция 18: Подгруппы в \\свободных группах и теорема Нильсена-Шрайера}

%%%%%%%%%%%%%%%%%%%%%%%%%%%%%%%%%%%%%%%%%%%%%%%%%%%%%%%%%%%%%%%%%%%%%%%%

%%%%%%%%%%%%%%%%%%%%%%%%%%%%%%%%%%%%%%%%%%%%%%%%%%%%%%%%%%%%
\section[Фундаментальная группа 
букета окружностей]{Фундаментальная группа \\букета окружностей}
%%%%%%%%%%%%%%%%%%%%%%%%%%%%%%%%%%%%%%%%%%%%%%%%%%%%%%%%%%%%


\определение
{\бф Графом} называется набор вершин и набор ребер,
причем каждому ребру соответствует две вершины
(возможно, одинаковые), которые называются его {\бф
концами}, или {\бф концом и началом}, причем
каждая вершина является концом хотя бы одного из ребер.
Если двум ребрам соответствует одна и та
же вершина, эти ребра называются {\бф смежными},
а вершина --- {\бф общей вершиной} ребер.
Граф называется {\бф конечным}, если число
его ребер и вершин конечно.
\ео

\замечание
Определение графа чисто комбинаторное:
даны два множества (вершин $V$ и ребер $E$), и отображение
$E \arrow \frac{V \times V}{(x,y)\sim (y,x)}$ из множества 
ребер в множество  неупорядоченных пар вершин.  Но для
того, чтоб получить в голове какую-то геометрическую
картинку, полезно представлять себе граф как совокупность 
отрезков, соединяющих набор отмеченных точек --- вершин.
Это интуитивное представление можно формализовать,
получив понятие {\ем топологического пространства графа}.
\еза

\определение
Пусть $\Gamma$ --- граф, а $S$ --- множество его ребер.
Рассмотрим $S$ как пространство с дискретной топологией,
и пусть $X:= S \times [0,1]$ --- несвязное объединение
$S$ копий отрезка. В этом случае $s \times \{1\}$ и $s \times \{0\}$ --
точки $X$, соответствующая началу или концу отрезка.
Если у ребра $s_1$ и у ребра $s_2$ имеется общий конец, напишем
$x_1\sim x_2$, где $x_i= s_i \times \{1\}$ или $x_i= s_i \times \{0\}$
-- соответствующие точки $X$. {\бф Топологическим
пространством графа} называется факторпространство
$X/\sim$ по такому соотношению эквивалентности.
\ео

\задача
Докажите, что топологическое пространство графа всегда
хаусдорфово и локально линейно связно.
\ез

\замечание
Злоупотребляя обозначениями,
мы будем обозначать граф и его топологическое
пространство одной и той же буквой.
\еза

\определение
Граф называется {\бф связным}, если его топологическое
пространство связно.
\ео


\begin{figure}[ht]
\begin{center}\ \\
\epsfig{file=buket.eps,width=0.4\linewidth}\\
{\small Букет четырех окружностей}
\end{center}
\end{figure}

\определение
Пусть $\Gamma$ --- связный граф, у которого есть всего
одна вершина и $|I|$ ребер. Его топологическое пространство называется
{\бф букетом $|I|$ окружностей}. Оно имеет вид ромашки
сделанной из нескольких (возможно, бесконечного числа)
окружностей.
\ео

Чтобы вычислить фундаментальную группу графа, проще
всего воспользоваться теоремой Зейферта--ван Кампена.
Пусть $X$ --- букет окружностей, а $X_1, X_2, ...$ --- составляющие
его окружности. Пересечение любой пары различных $X_i$ --- единственная
вершина графа, и она, очевидно, односвязна. Получаем, что
$\pi_1(X) = \pi_1(X_1) * \pi_1(X_2) * ... = \Z *\Z * \Z* ...$.
Мы доказали такую теорему.

\хфилл

\теорема
Фундаментальная группа букета окружностей свободна.

\endproof


%%%%%%%%%%%%%%%%%%%%%%%%%%%%%%%%%%%%%%%%%%%%%%%%%%%%%%%%%%%%
\section{Деревья}
%%%%%%%%%%%%%%%%%%%%%%%%%%%%%%%%%%%%%%%%%%%%%%%%%%%%%%%%%%%%

\определение
Конечный связный граф называется {\бф конечным деревом}, если
у него $n$ вершин и $n-1$ ребро.
\ео


\begin{figure}[ht]
\begin{center}\ \\
\epsfig{file=derevo.eps,width=0.3\linewidth}\\
{\small Дерево с 22 вершинами и 21 ребром}
\end{center}
\end{figure}


Напомним, что топологическое пространство
$X$ называется {\бф деформационным ретрактом}
$Y \supset X$, если задано непрерывное отображение
$Y \stackrel j \arrow X$, тождественное на $X \subset Y$,
причем $j$ гомотопно тождественному отображению $\Id_Y$
из $Y$ в себя.

Топологические пространства $X$ и $Y$ называются
{\бф гомотопически эквивалентными},
если заданы непрерывные отображения $X \stackrel \phi \arrow Y$
и $Y \stackrel \psi\arrow X$, причем композиции
$\psi\circ \phi$ и $\phi \circ \psi$ гомотопны
тождественным.

Нетрудно доказать (см. Лекцию 15), что любое пространство
гомотопически эквивалентно своему деформационному
ретракту, а фундаментальные группы гомотопически
эквивалентных пространств изоморфны.

\определение
Пусть $\Gamma$ --- граф, а
 $\Gamma'$ --- граф, множества ребер и вершин которого
являются подмножествами в множестве
ребер и вершин $\Gamma$, причем концы
сответствующих ребер в $\Gamma$ и $\Gamma'$ те же.
  Тогда $\Gamma'$
называется {\bf  подграфом} $\Gamma$.
\ео

\замечание 
Пусть $\Gamma'$ --- подграф графа $\Gamma$.
Легко видеть, что топологическое пространство $\Gamma'$ --
замкнутое подмножество в топологическом пространстве
$\Gamma$.
\еза

\определение
{\бф Валентность} вершины графа --- количество ребер, 
от 1 до $\infty$, которые с ней соединены. Вершина валентности 1
называется {\бф висячей}, соответствующее ей ребро ---
{\бф висячее ребро}.
\ео


\begin{figure}[ht]
\begin{center}\ \\
\epsfig{file=vybrosili.eps,width=0.65\linewidth}\\ \  \\
{\small Граф, полученный выкидыванием висячего ребра}
\end{center}
\end{figure}

\лемма
Пусть $\Gamma'\subset \Gamma$ --- подграф, полученный
из $\Gamma$ выбрасыванием висячего ребра $l$ и одной вершины
(см. рисунок). Тогда $\Gamma'$ является
деформационным ретрактом $\Gamma$.

\хфилл

\ноиндент
{\бф Доказательство:}
Пусть $s$ --- второй (невыкинутый) конец ребра $l$.
Рассмотрим отображение $\psi:\; \Gamma\arrow \Gamma'$
переводящее $l$ в $s$,
и тождественное на $\Gamma'$. Очевидно, оно
непрерывно. Чтобы убедиться, что это деформационная ретракция,
построим гомотопию из $\psi$ в $\Id_\Gamma$. Предположим,
что $0$ на $l$ соответствует $s$, а 1 соответствует выкинутой вершине.
Пусть $\psi_t$ действует тождественно на $\Gamma'$,
и переводит $\lambda\in [0,1]=l$ в $t\lambda \in l$.
Легко видеть, что $\psi_t$ непрерывно,
$\psi_0=\psi$, а $\psi_1=\Id_\Gamma$.
Поэтому $\Gamma'$ --- деформационный ретракт.
\endproof

\замечание
Из этой леммы, мы получили, что любой граф 
гомотопически эквивалентен своему подграфу,
полученному выкидыванием висячего ребра.
\еза

\утверждение
Конечное дерево стягиваемо (т.е. гомотопически эквивалентно точке).

\хфилл

\ноиндент
{\бф Доказательство:}
Воспользуемся индукцией. Пусть $\Gamma$ --- дерево из
$n+1$ вершин и $n$ ребер. Поскольку валентность каждой
вершины $\geq 1$, а вершин больше, чем ребер, 
найдется вершина с валентностью 1 (проверьте это). 
Соответствующее ребро --- висячее. Обозначим через $\Gamma'$ граф,
полученный его выбрасыванием. У этого графа $n$ вершин и
$n-1$ ребро, значит, это дерево. В силу предыдущей
леммы, $\Gamma$ гомотопически эквивалентно $\Gamma'$,
а в силу предположения индукции, $\Gamma'$
стягиваемо. \endproof

\определение
Связный граф называется {\бф деревом}, если
любой его конечный, связный подграф является
конечным деревом.
\ео

\пример
Пусть $\Gamma$ --- граф, вершины которого --- конечные
последовательности натуральных чисел, а ребра соединяют
любую последовательность $A$ и $An$, где $An$ получено
из $A$ добавлением $n$. Докажите, что $\Gamma$ --- это
дерево.

\хфилл

\утверждение
Пусть $\Gamma$ --- дерево. Тогда $\Gamma$ односвязно.

\хфилл

\ноиндент
{\бф Доказательство. Шаг 0:} Отметим, что для любых 
двух точек $x, y$ в $\Gamma$, соединенных ребром $[x,y]$,
такое ребро единственно. Действительно, иначе
в $\Gamma$ был бы подграф с вершинами $x,y$ и двумя
ребрами, соединяющими эти вершины, но такой граф имеет
2 вершины, 2 ребра, и не может быть деревом.
Поэтому обозначение $[x,y]$ для соседних вершин $x, y$
однозначно задает ребро.


\хфилл

\ноиндент
{\бф Шаг 1:}
Пусть $\gamma\in \Omega(\Gamma, m)$
-- петля в $\Gamma$, а $v_1, v_2, v_3, ...$ --- различные
вершины графа, которые последовательно обходит $\gamma$. Пусть 
$0\leq t_1 < t_2 < t_2 < ... \leq 1$ --- последовательность чисел
таких, что $\gamma(v_i) = t_i$. Поскольку последовательность
$\{t_i\}$ монотонна, она сходится к пределу $t$,
но тогда \[ \gamma(\lim t_i) = \gamma(t) = \lim v_i, \]
в силу непрерывности $\gamma$. Это возможно, только
если последовательность $v_i$ конечна. 

\хфилл

\ноиндент
{\бф Шаг 2:}
Легко видеть,
что путь, идущий по дереву из вершины $x$ в вершину
$y$, не заходя ни в какую другую вершину, гомотопен
пути, идущему по ребру $[x,y]$. И в самом деле,
пусть $\Gamma_{x,y}$ --- граф, состоящий из всех
ребер $\Gamma$ с концом в $x$ и в $y$.
Поскольку $\Gamma$ это дерево, $\Gamma_{x,y}$ --- тоже
дерево, и все его ребра, кроме $[x,y]$, висячие.
Поэтому $[x,y]$ --- деформационный ретракт
$\Gamma_{x,y}$. Взяв композицию $\gamma$ с ретракцией,
получим путь, идущий по ребру $[x,y]$, и гомотопный
$\gamma$.


\хфилл

\ноиндент
{\бф Шаг 3:} Мы получили, что любая петля по $\Gamma$
гомотопически эквивалентна петле, которая обходит
вершины $v_1, v_2, v_3, ..., v_n$ по ребрам
$[v_1, v_2]$, $[v_2, v_3]$, ...
Значит, любая петля гомотопна петле, которая 
обходит конечный подграф $\Gamma$. Но конечные,
связные подграфы $\Gamma$ стягиваемы, значит,
все такие петли тоже стягиваемы.
\endproof

%%%%%%%%%%%%%%%%%%%%%%%%%%%%%%%%%%%%%%%%%%%%%%%%
\section{Унициклические графы}
%%%%%%%%%%%%%%%%%%%%%%%%%%%%%%%%%%%%%%%%%%%%%%%%


\определение
Конечный, связный граф $\Gamma$ называется
{\бф унициклическим}, если у него $n$ вершин
и $n$ ребер. 
\ео

\замечание
Пусть $\Gamma$ --- связный граф без висячих вершин,
имеющий $n$ ребер и $m$ вершин.
Поскольку каждое ребро соединяет две вершины,
и к каждой вершине присоединяются как минимум
два ребра, имеем $n \geq m$, причем равенство
имеет место, только если все вершины $\Gamma$
имеют валентность 2.
\еза

Мы получили следующую лемму


\хфилл

\лемма
Пусть $\Gamma$ --- конечный унициклический
граф без висячих вершин. Тогда все вершины $\Gamma$
двухвалентные. Более того, $\Gamma$ гомеоморфен окружности.


\хфилл

\ноиндент
{\бф Доказательство:} Воспользуемся индукцией.
Пусть $\Gamma$ --- унициклический граф без 
висячих вершин, а $s$ --- его вершина,
к которой примыкают ребра $[x,s]$ и $[s,y]$.
Легко видеть, что $\Gamma$ гомеоморфен графу,
полученному из $\Gamma$ выкидыванием вершины $s$
и заменой ребер $[x,s]$ и $[s,y]$ на $[x,y]$
(см. картинку).


\begin{figure}[ht]
\begin{center}\ \\
\epsfig{file=unicycle-simple.eps,width=0.65\linewidth}\\ 
{\small Унициклический граф  без висячих ребер 
из $n$ ребер, $n$ вершин  гомеоморфен  унициклическому графу
из  $n-1$ ребер, $n-1$ вершин}
\end{center}
\end{figure}


Воспользовавшись индукцией, получим, что
$\Gamma$ гомеоморфен унициклическому графу
с одной вершиной и одним ребром, то есть
окружности. \endproof

\хфилл

\следствие
Пусть $\Gamma$ --- конечный унициклический граф. Тогда $\Gamma$
гомотопически эквивалентентен окружности.

\хфилл

\ноиндент
{\бф Доказательство:} Пусть $\Gamma'$ получен из $\Gamma$
выкидыванием висячего ребра. Тогда $\Gamma'$ является деформационным
ретрактом $\Gamma$, значит, гомотопически эквивалентен $\Gamma$.
Будем выкидывать висячие ребра, пока они не кончатся,
получим унициклический граф, гомеоморфный окружности.
\endproof

\замечание
Аналогичный аргумент доказывает, что  связный граф,
у которого  $n$ вершин
и $n+k$ ребер, гомотопически эквивалентен букету
$k+1$ окружностей.
\еза

%%%%%%%%%%%%%%%%%%%%%%%%%%%%%%%%%%%%%%%%%%%%%%%%%%%%%%%%%%%%%%%%%%%%%%%%
\утверждение\label{_uniciklicheskij_odno_rebro_Utverzhdenie_}
Пусть $\Gamma_1$ --- дерево, полученное
из графа $\Gamma$ выкидыванием одного
ребра. Предположим, что $\Gamma$ --- не дерево.
Тогда $\pi_1(\Gamma) = \Z$.

\хфилл

\ноиндент
{\бф Доказательство:} Поскольку $\Gamma$ --- не дерево,
в $\Gamma$ содержится конечный подграф $\Gamma_1$, у которого
$n$ вершин и $n+k$ ребер, $k\geq 0$.
Поскольку выкидывание одного ребра $l$ превращает
$\Gamma_1$ в дерево, $k=0$, и граф $\Gamma_1$ --- унициклический. 
Пусть $\Gamma\backslash l$,
$\Gamma_1 \backslash l$ --- графы, полученные
из $\Gamma$, $\Gamma_1$ выкидыванием $l$.
Тогда $\Gamma$ получен объединением $\Gamma_1$
и $\Gamma\backslash l$, причем их пересечение
$\Gamma_1 \backslash l$ является деревом,
следовательно, односвязно. Применив теорему Зейферта--ван Кампена,
получим, что $\pi_1(\Gamma) = \pi_1(\Gamma_1) * \pi_1(\Gamma\backslash l)$.
Поскольку $\Gamma\backslash l$ дерево, оно односвязно, значит,
$\pi_1(\Gamma) = \pi_1(\Gamma_1)=\Z$. \endproof

\определение
Будем называть связный граф $\Gamma$ {\бф унициклическим},
если $\pi_1(\Gamma)=\Z$. 
\ео

%%%%%%%%%%%%%%%%%%%%%%%%%%%%%%%%%%%%%%%%%%%%%%%%
\section{Фундаментальная группа графа}
%%%%%%%%%%%%%%%%%%%%%%%%%%%%%%%%%%%%%%%%%%%%%%%%

\определение
Пусть $\Gamma$ --- связный граф. {\бф Остовом}
$\Gamma$ называют максимальный подграф $\Gamma'\subset \Gamma$,
который является деревом.
\ео

\замечание
Слово {\бф максимальный} в этом определении надо понимать так:
при добавлении любого ребра из $\Gamma \backslash \Gamma'$
к $\Gamma'$, он перестает быть деревом.
Применив лемму Цорна, легко убедиться, что у каждого графа есть
остов (проверьте это).
\еза




\теорема
Пусть $\Gamma$ --- связный 
граф. Тогда группа $\pi_1(\Gamma)$ свободна.

\хфилл

{\бф Доказательство. Шаг 1:}
Пусть $\Gamma'\subset \Gamma$ --- остов $\Gamma$,
полученный из $\Gamma$ выкидыванием ребер $l_\alpha$,
проиндексированными индексами 
$\alpha\in I$, а $\Gamma_{l_\alpha}$ --- объединение
$\Gamma'$ и $l_\alpha$. В силу Утверждения
\ref{_uniciklicheskij_odno_rebro_Utverzhdenie_},
$\pi_1(\Gamma_{l_\alpha})=\Z$.

\хфилл

{\бф  Шаг 2:} $\Gamma= \bigcup_{\alpha\in I} \Gamma_{l_\alpha}$, причем
пересечение $\Gamma_{l_\alpha}\cap \Gamma_{l_{\alpha'}}$
для любых $\alpha \neq\alpha'$ равно $\Gamma'$,
следовательно, односвязно. Применяя теорему
Зейферта--ван Кампена, получаем
\[
\pi_1(\Gamma) = \coprod_{\alpha\in
I}\pi_1(\Gamma_{l_\alpha}) = \coprod_{\alpha\in
I} \Z.
\]
\endproof

\замечание
Поскольку конечный граф $\Gamma$ гомотопически
эквивалентен букету сфер, свободность $\pi_1(\Gamma)$
для конечного графа немедленно следует из подсчета
фундаментальной группы букера сфер, проведенного выше.
\еза

Из приведенного выше подсчета фундаментальной группы графа
вытекает следующая важная теорема.

\хфилл

\теорема
(теорема Нильсена-Шрайера, "Nielsen-Schreier theorem")
Пусть $F$ --- свободная группа, а $G\subset F$ --- ее
подгруппа. Тогда $G$ свободна.

\хфилл

{\бф Доказательство:} Группу
$F$ можно получить как фундаментальную группу пространства
$M$, гомеоморфного букету окружностей. Пусть $\tilde M$ --
универсальное накрытие $M$, снабженное естественным
действием $F$, а $M_G= \tilde M/G$ его фактор по $G$.
Легко видеть, что $M_G$ это граф (проверьте это). 
Поскольку $\tilde M \arrow M_G$ --- универсальное накрытие,
$\pi_1(M_G) =G$. Значит $G$ --- фундаментальная группа
графа, а такая группа свободна по предыдущей теореме.
\endproof


%%%%%%%%%%%%%%%%%%%%%%%%%%%%%%%%%%%%%%%%%%%%%%%%%%%%%%%%%%%%%%%%%%%%%%%%

\part{Приложение. Вещественные числа}

\renewcommand{\PartName}{Приложение. Вещественные числа.}

\renewcommand{\chaptermark}[1]{\markboth{{\bf
  #1}}{{\sc\PartName}}}

%%%%%%%%%%%%%%%%%%%%%%%%%%%%%%%%%%%%%%%%%%%%%%%%%%%%%%%%%%%%%%%%%%%%%%%%

\setcounter{chapter}{-1}
%%%%%%%%%%%%%%%%%%%%%%%%%%%%%%%%%%%%%%%%%%%%%%%%%%%%%%%%%%%%%%%%%%%%%%%%

\chapter{Листок 0. Вещественные числа}

%%%%%%%%%%%%%%%%%%%%%%%%%%%%%%%%%%%%%%%%%%%%%%%%%%%%%%%%%%%%%%%%%%%%%%%%


\noindent
Для этого листка требуется знакомство с понятием поля.
Определения и задачи, приведенные ниже, знакомы большинству
студентов. Желающие освежить школьную программу или
вспомнить основные определения могут посмотреть этот листок
и прорешать задачи. В топологии можно обойтись без вещественных
чисел, но понятие вещественного числа --- 
ключевое в метрической геометрии; большое число примеров
топологических пространств строятся на основе вещественных чисел.

%%%%%%%%%%%%%%%%%%%%%%%%%%%%%%%%%%%%%%%%%%%%%%%%%%%%%%%%%%%%
\subs{Фундаменальные последовательности.}
%%%%%%%%%%%%%%%%%%%%%%%%%%%%%%%%%%%%%%%%%%%%%%%%%%%%%%%%%%%%

Обычно вещественные числа приближают рациональными --- например,
раскладывают число $a$ в бесконечную десятичную дробь
$a_0,a_1a_2\dots$, и рассматривают разные конечные отрезки
$a_0,a_1a_2 \dots a_n$ этой дроби как все более точные приближения
$a$. При этом некоторые дроби объявляются эквивалентными, например,
$1,00000\dots$ и $0,9999 \dots$. Оказывается, что строго определять
арифметические операции на вещественных числах и доказывать их
свойства проще, если рассматривать не конкретно десятичные дроби, а
вообще любые последовательности рациональных чисел, приближающие
данное вещественное число. При этом снова надо учитывать, что разные
последовательности могут быть эквивалентны (приближать одно и то же
число). С логической точки зрения, проще всего просто {\bf объявить}
вещественным числом множество приближающих его последовательностей
рациональных чисел. На этом основан подход Коши к строгому
построению множества вещественных чисел.

\begin{opredelenie} Будем говорить, что нечто верно для {\bf почти
всех} элементов множества, если оно верно для всех элементов, кроме
конечного их числа. Пусть $\{a_i\}= a_0, a_1, a_2, \ldots$ --
последовательность рациональных чисел. Говорят, что $\{a_i\}$ --
{\bf фундаментальная последовательность}, или {\bf
последовательность Коши}, если для любого рационального
$\epsilon >0$ существует отрезок $[x, y]$ длины 
$\epsilon$, который содержит почти все
$\{a_i\}$.
\end{opredelenie}

\begin{zadacha}
Пусть $a$ --- рациональное число. Докажите, что постоянная
последовательность $a,a,\dots$ --- последовательность Коши.
\end{zadacha}

Такую последовательность мы будем обозначать через $\{a\}$.

\begin{zadacha} Пусть $\{a_i\}$ --- последовательность Коши.
Переставим в произвольном порядке элементы $a_i$. Докажите, что
получится последовательность Коши.
\end{zadacha}

\begin{zadacha} Дана последовательность $\{a_i\}$ рациональных чисел,
принадлежащих отрезку $I = [a, b]$, $a, b \in \Q$. Докажите, что из
$\{a_i\}$ можно выбрать подпоследовательность, которая является
последовательностью Коши.
\end{zadacha}

\begin{ukazanie} Разделим отрезок $I_0=[a, b]$ пополам.
В одной из половин (обозначим ее $I_1$) содержится бесконечное число
элементов последовательности. Выкинем из $\{a_i\}$ все элементы,
кроме $a_0$, которые не лежат в $I_1$. Разделим $I_1$ пополам, и
т.д. В отрезке $I_k$, полученном на $k$-м шаге, содержатся все
элементы полученной последовательности, начиная с $k$-го, и этот
отрезок имеет длину $\frac{b-a}{2^k}$.
\end{ukazanie}

\begin{zadacha}[!] Дана монотонно возрастающая последовательность 
$a_1\leq a_2 \leq a_3 \leq\dots$. Известно, что все $a_i$ ограничены
сверху некоторой константой $C$: $a_i \leq C$. Докажите, что это
последовательность Коши.
\end{zadacha}

\begin{ukazanie} Воспользуйтесь предыдущей задачей.
\end{ukazanie}

\begin{opredelenie} Пусть $\{ a_i\}$, $\{b_i\}$ --- последовательности
Коши. Они называются {\bf эквивалентными}, если последовательность
$a_0, b_0, a_1, b_1, a_2, b_2, ... $ --- последовательность Коши.
\end{opredelenie}

\begin{zadacha}
Пусть $a$, $b$ --- два рациональных числа. Докажите, что $\{a\}$
эквивалентна $\{b\}$ тогда и только тогда, когда $a=b$.
\end{zadacha}

\begin{zadacha} Докажите, что последовательность
Коши эквивалентна любой своей подпоследовательности.
\end{zadacha}

\begin{zadacha}
Докажите, что если $\{a_i\}$ эквивалентна $\{b_i\}$, то $\{b_i\}$
эквивалентна $\{a_i\}$.
\end{zadacha}

\begin{zadacha}[!]\label{otdeleny}
Пусть $\{a_i\}$, $\{b_i\}$ --- две неэквивалентные последовательности
Коши. Докажите, что существуют два непересекающихся интервала $I_1$,
$I_2$ такие, что почти все $a_i$ лежат в $I_1$, а почти все $b_i$ --
в $I_2$.
\end{zadacha}

\begin{ukazanie} Примените определение последовательности Коши к
$\epsilon = \frac{1}{2^n}$ для всех $n$.
\end{ukazanie}

\begin{zadacha} [!] Докажите, что если последовательность $\{a_i\}$
эквивалентна последовательности $\{b_i\}$, а последовательность
$\{b_i\}$ эквивалентна последовательности $\{c_i\}$, то $\{a_i\}$
эквивалентна $\{c_i\}$. (Это свойство выражают словами
``эквивалентность последовательностей Коши транзитивна''.)
\end{zadacha}

\begin{opredelenie}
Пусть $\{a_i\}$, $\{b_i\}$ --- две неэквивалентные последовательности
Коши. Говорят, что $\{a_i\} > \{b_i\}$, если $a_i > b_i$ для почти
всех $i$.
\end{opredelenie}

\begin{zadacha}\label{order}
Пусть $\{a_i\}$, $\{b_i\}$ --- две неэквивалентные последовательности
Коши. Докажите, что или $\{a_i\} < \{b_i\}$, или $\{b_i\} <
\{a_i\}$.
\end{zadacha}

\begin{ukazanie}
Воспользуйтесь задачей~\ref{otdeleny}.
\end{ukazanie}

\begin{zadacha}\label{inte}
Пусть $\{a_i\}$, $\{b_i\}$ --- две неэквивалентные последовательности
Коши, и $\{a_i\} < \{b_i\}$. Докажите, что существуют два
рациональных числа $c$, $d$ таких, что $\{a_i\} < \{c\} <
  \{d\} < \{b_i\}$.
\end{zadacha}

\begin{ukazanie}
Воспользуйтесь предыдущим указанием.
\end{ukazanie}

\begin{zadacha}
Пусть $\{a_i\} < \{b_i\}$, а $\{b_i\}$ эквивалентно
$\{c_i\}$. Докажите, что $\{a_i\} < \{c_i\}$.
\end{zadacha}

\begin{ukazanie}
Воспользуйтесь предыдущей задачей, и определением последовательности
Коши для любого $\epsilon < |c-d|$.
\end{ukazanie}

\begin{zadacha} Пусть $\{ a_i\}$ --- последовательность
Коши, а $c\in \Q$ --- рациональное число.  Докажите, что следующие
свойства эквивалентны
\begin{enumerate}
\item $\{ a_i\}$ эквивалентна последовательности $\{c\}$.

\item В любом открытом отрезке $]x, y[$, содержащем $c$, содержится
бесконечно много элементов последовательности $\{ a_i\}$.

\item В любом открытом отрезке $]x, y[$, содержащем $c$, содержатся
почти все элементы последовательнсти $\{ a_i\}$.
\end{enumerate}
\end{zadacha}

\begin{opredelenie} Если любое из вышеуказанных свойств выполнено,
мы говорим, что $\{ a_i\}$ сходится к $c$.
\end{opredelenie}

\begin{zadacha}\label{sum} 
Пусть $\{ a_i\}$, $\{b_i\}$ --- последовательности Коши. Докажите,
что $\{ a_i + b_i \}$ и $\{ a_i-b_i \}$ --- последовательности
Коши.
\end{zadacha}

\begin{zadacha} Пусть $\{ a_i\}$, $\{b_i\}$ --- последовательности
Коши, причем $b_i$ сходится к 0. Докажите, что $\{ a_i\}$
эквивалентна $\{ a_i + b_i \}$.
\end{zadacha}

\begin{zadacha} Пусть $\{ a_i\}$, $\{b_i\}$ --- последовательности
Коши. Докажите, что $\{ a_i b_i \}$ --- последовательность Коши.
\end{zadacha}

\begin{zadacha} Докажите, что если $\{b_i\}$ сходится к 1, то $\{
a_i b_i \}$ эквивалентна $\{ a_i\}$.
\end{zadacha}

\begin{zadacha}\label{div} 
Пусть $\{ a_i\}$ --- последовательность Коши из ненулевых чисел,
которая не сходится к $0$. Докажите, что $\{ a_i^{-1}\}$ --
последовательность Коши.
\end{zadacha}

\begin{ukazanie} Докажите, что существует такой не содержащий $0$
замкнутый отрезок $[x, y]$, что почти все $\{ a_i\}$ содержатся в
$[x, y]$. Пусть почти все $\{ a_i\}$ содержатся в отрезке $I\subset
[x, y]$ длины $\epsilon$. Докажите, что все $\{ a_i^{-1}\}$, кроме
конечного числа, содержатся в отрезке $I^{-1}$ длины $\epsilon
(\min(|x|, |y|)^{-1}$.
\end{ukazanie}

\begin{opredelenie}
{\bf Классом эквивалентности} последовательности Коши $\{a_i\}$
называется множество всех последовательностей Коши, эквивалентных
$\{a_i\}$. Множество классов эквивалентностей последовательностей
Коши называется {\bf множеством действительных чисел} и обозначается
через $\R$.
\end{opredelenie}

\begin{zadacha}
Докажите, что соответствие $c \mapsto \{c\}$ задает инъективное
отображение из множества $\Q$ всех рациональных чисел в $\R$.
\end{zadacha}

\begin{zadacha}[!] Докажите, что четыре арифметических операции,
которые мы определили на $\R$ в задачах~\ref{sum}-~\ref{div}, задают
на $\R$ структуру поля.
\end{zadacha}

\subs{Дедекиндовы сечения.}

Главный недостаток определения действительных чисел через
фундаментальные последовательности --- это то, что эквивалентных
фундаментальных последовательностей очень много: определение
получается очень неявным. Трудность эта скорее психологическая. Тем
не менее, есть способ ее преодолеть --- более наглядное определение
вещественных чисел, которое предложил Дедекинд.

\begin{opredelenie}
Пусть $R \subset \Q$ --- подмножество в множестве рациональных чисел,
которое непусто и не равно всему $\Q$. Говорят, что $R$ --- {\bf
сечение Дедекинда}, если из $a \in R$ и $b < a$ следует, что $b \in
R$. Сечение Дедекинда $R$ называется {\bf замкнутым}, если существует
такое рациональное число $a$, что $b \in R$ тогда и только тогда,
когда $b \leq a$. В противном случае $R$ называется {\bf открытым}.
\end{opredelenie}

Пусть $\{a_i\}$ --- последовательность Коши. Обозначим через
$R_{\{a_i\}}$ множество таких рациональных чисел $b$, что $\{b\} <
\{a_i\}$.

\begin{zadacha}
Докажите, что $R_{\{a_i\}}$ --- сечение Дедекинда (т.е. если $b
\in R_{\{a_i\}}$, а $c < b$, то $c \in R_{\{a_i\}}$). Докажите, что
это сечение открыто.
\end{zadacha}

\begin{zadacha} Пусть $\{ a_i\}$ и $\{b_i\}$ --- эквивалентные
последовательности Коши. Докажите, что $R_{\{a_i\}} = R_{\{b_i\}}$.
\end{zadacha}

\begin{zadacha}
Пусть $\{ a_i\}$ и $\{b_i\}$ --- неэквивалентные последовательности
Коши, и $\{a_i\} < \{b_i\}$. Докажите, что $R_{\{a_i\}} \subset
R_{\{b_i\}}$, но эти множества не совпадают.
\end{zadacha}

\begin{ukazanie}
Рассмотрите точки интервала $[c,d]$ из задачи~\ref{inte}; какому из
множеств $R_{\{a_i\}}$, $R_{\{b_i\}}$ они принаделжат?
\end{ukazanie}
 
\begin{zadacha}[*] Пусть $\{ a_i\}$, $\{b_i\}$ --- две последовательности
Коши. Докажите, что они эквивалентны тогда и только тогда, когда 
$R_{\{ a_i\}} = R_{\{ b_i\}}$.
\end{zadacha}

\begin{ukazanie}
Воспользуйтесь задачей~\ref{order} (и предыдущими задачами).
\end{ukazanie}

\begin{zadacha}[*]
Пусть $R \subset \Q$ --- открытое сечение Дедекинда. Докажите, что $R
= R_{\{a_i\}}$ для какой-то фундаментальной последовательности
$\{a_i\}$.
\end{zadacha}

\begin{ukazanie}
Рассмотрите интервал $I_0=[a,b]$ такой, что $a$ лежит в $R$, а $b$ --
нет. Поделите его пополам, выберите половину $I_1$ с тем же
свойством. Повторите процесс, и возьмите в качестве $a_i$ любую
точку интервала $I_i$.
\end{ukazanie} 

Мы видим, что множество классов эквивалентности последовательностей
Коши --- это то же самое, что множество открытых сечений
Дедекинда. Поэтому действительные числа можно с тем же успехом
определять как сечения Дедекинда. В дальнейшем пользуйтесь тем из
определений, которое вам удобнее.

\begin{zadacha}[**]
Определите арифметические операции на $\R$ непосредственно через
сечения Дедeкинда, не прибегая к последовательностям Коши. Проверьте
аксиомы поля.
\end{zadacha}

\begin{ukazanie}
Чтобы определить умножение, определите сначала операции ``умножение
на положительное действительное число $a$'' и ``умножение на $-1$'',
и докажите дистибутивность для каждой из них по отдельности.
\end{ukazanie}

%%%%%%%%%%%%%%%%%%%%%%%%%%%%%%%%%%%%%%%%%%%%%%%%%%%%%%%%%%%%
\subs{Супремум и инфимум.}
%%%%%%%%%%%%%%%%%%%%%%%%%%%%%%%%%%%%%%%%%%%%%%%%%%%%%%%%%%%%

\begin{opredelenie} Пусть $A\subset \R$ --- некоторое подмножество
$\R$. Множество $A$ называется {\bf ограниченным снизу}, если все
элементы $A$ больше некоторой константы $C\in \R$.  Множество $A$
называется {\bf ограниченным сверху}, если все элементы $A$ меньше
некоторой константы $C\in \R$. Множество $A$ называется {\bf
ограниченным}, если оно ограниченно сверху и снизу.
\end{opredelenie}

\begin{opredelenie} Пусть $A\subset \R$ --- некоторое подмножество
$\R$. Инфимум $A$ (обозначается $\inf A$) есть такое число $c \in
\R$, что $c \leq a$ для всех $a\in A$, и в любом открытом отрезке
$]x, y[$, содержащем $c$, содержатся и элементы $A$. Супремум $A$
(обозначается $\sup A$) есть такое число $c \in \R$, что $c \geq a$
для всех $a\in A$, и в любом открытом отрезке $]x, y[$, содержащем
$c$, содержатся и элементы $A$.
\end{opredelenie}

\begin{zadacha} Докажите, что $\inf A$ и $\sup A$
единственны (если существуют).
\end{zadacha}

\begin{zadacha}[!] Пусть $A$ --- ограниченное сверху множество.
Докажите, что $\sup A$ существует.
\end{zadacha}

\begin{ukazanie} Рассмотрим все $a\in A$ как сечения Дедекинда,
т.е. подмножества в $\Q$. Возьмем их объединение $R$; поскольку все
$a \leq C$, это будет тоже сечение Дедекинда. Докажите, что $\inf A
= R$.
\end{ukazanie}

\begin{zadacha}[!] Пусть $A \subset \R$ --- ограниченное снизу множество.
Докажите, что $\inf A$ существует.
\end{zadacha}

\begin{zamechanie} Пусть $A\subset \R$ не ограничено снизу (сверху).
Тогда пишут $\inf A = -\infty$ ($\sup A = \infty$).
\end{zamechanie}


%%%%%%%%%%%%%%%%%%%%%%%%%%%%%%%%%%%%%%%%%%%%%%%%%%%%%%%%%%%%

\subs{Корни многочленов нечетной степени.}

\begin{zadacha}[!] Дан полином нечетной степени над $\Q$,
$P= t^{2n+1} + a_{2n} t^{2n} + a_{2n-1} t^{2n-1} + ... + a_0$.
Пусть $R_P$ --- множество всех $x\in \Q$ таких, что $P(t)<0$ на
отрезке $]-\infty, x]$. Докажите, что $R_P$ непусто.
\end{zadacha}

\begin{ukazanie} Докажите, что $R_P$ содержит 
$-\max(1, \sum |a_i|)$.
\end{ukazanie}

\begin{zadacha}[!] 
Докажите, что $R_P$ --- не все множество вещественных чисел.
\end{zadacha}

\begin{ukazanie} Докажите, что дополнение $\Q \backslash R_P$ 
содержит $\max(1, \sum |a_i|)$.
\end{ukazanie}

\begin{zadacha}[!] Докажите, что $R_P$ --- сечение Дедекинда. 
\end{zadacha}

\begin{zadacha}[!]\label{lips} 
Докажите, что $P$ удовлетворяет {\bf свойству Липшица}: для любого
отрезка $I$ существует такая константа $C > 0$, что $|P(a)-P(b)| <
C|a-b|$ для любых $a,b \in I$.
\end{zadacha}

\begin{zadacha}[!] Рассмотрим дедекиндово сечение 
$R_P$ как вещественное число. Докажите, что $P(R_P)=0$. Тем самым,
любой многочлен нечетной степени над $\R$ имеет корень.
\end{zadacha}

\begin{ukazanie}
Докажите сначала, что $P(R_P) \leq 0$. Затем докажите, что $P(R_P) <
0$ противоречит задаче~\ref{lips}.
\end{ukazanie}

%%%%%%%%%%%%%%%%%%%%%%%%%%%%%%%%%%%%%%%%%%%%%%%%
\subs{Пределы.}
%%%%%%%%%%%%%%%%%%%%%%%%%%%%%%%%%%%%%%%%%%%%%%%%

\begin{opredelenie} Пусть $A\subset \R$ --- пoдмножество в
множестве вещественных чисел, а $c$ --- вещественное число. Точка $c$
называется {\bf предельной точкой} последовательности $A$, если для
каждого открытого интервала $I = ]x, y[$, содержащего $c$, в $I$
содержится бесконечно много элементов $A$.
\end{opredelenie}

\begin{opredelenie} Пусть $\{a_i\}$ --- последовательность
вещественных чисел, а $c$ --- вещественное число. Пусть для каждого
открытого интервала $I = ]x, y[$, содержащего $c$, в $I$ содержатся
все элементы $\{a_i\}$, кроме конечного числа. Тогда говорят, что
$c$ есть {\bf предел последовательности $\{a_i\}$} (обозначается $c
= \lim_{i \to \infty} a_i $). Еще говорят, что последовательность
$a_i$ {\bf сходится к $c$}, или {\bf стремится к $c$}
\end{opredelenie}

\begin{zadacha} Пусть $c$ --- предельная точка последовательности
$\{a_i\}$. Докажите, что из нее можно выбрать подпоследовательность,
сходящуюся к $c$.
\end{zadacha}

\begin{zadacha}[*] Дана последовательность $\{a_i\}$ точек на
отрезке $[x,y]$. Докажите, что у нее есть предельные точки.
\end{zadacha}

\begin{opredelenie} Множество $A\subset \R$ называется {\bf
дискретным}, если у него нет предельных точек.
\end{opredelenie}

\begin{zadacha}[*] Пусть $\{a_i\}$ --- последовательность. Обозначим
множество всех $a_i$ за $A$.  Докажите, что $\{a_i\}$ сходится тогда
и только тогда, когда $A$ не имеет бесконечных дискретных
подмножеств, и имеет единственную предельную точку.
\end{zadacha}

\begin{zadacha} Рассмотрим последовательность $0, 1, 2, 3, 4,
\ldots$. Докажите, что у этой последовательности нет предела.
\end{zadacha}

\begin{zadacha} Рассмотрим последовательность $0, 1, 1/2, 1/3, 1/4,
\ldots$. Докажите, что эта последовательность сходится к 0.
\end{zadacha}

\begin{zadacha} Дана монотонно возрастающая последовательность 
$a_1\leq a_2 \leq a_3 \leq ...$, $a_i\in \R$. Известно, что все
$a_i$ ограничены сверху некоторой константой $C$: $a_i \leq
C$. Докажите, что эта последовательность имеет предел. Используйте
определение вещественных чисел через сечения Дедекинда.
\end{zadacha}

\begin{ukazanie} Докажите, что $\lim_{i \to \infty} a_i  = \sup\{a_i\}$,
и воспользуйтесь существованием супремума.
\end{ukazanie}

\begin{opredelenie} Пусть $\{a_i\}= a_0, a_1, a_2, \ldots$ --
последовательность вещественных чисел. Говорят, что $\{a_i\}$ --
{\bf последовательность Коши,} если для каждого $\epsilon >0$
существует отрезок $[x, y]$ длины $\epsilon$, который содержит все
$\{a_i\}$, кроме конечного числа.
\end{opredelenie}

\begin{zamechanie} То же самое определение используется для
последовательностей Коши рациональных чисел.
\end{zamechanie}

\begin{zadacha} Пусть последовательность $\{a_i\}$ сходится к
какому-нибудь вещественному числу $c$. Докажите, что это
последовательность Коши.
\end{zadacha}

\begin{zadacha} Пусть у последовательности Коши $\{a_i\}$ 
есть подпоследовательность, которая сходится к $x\in \R$. Докажите,
что $\{a_i\}$ сходится к $x$.
\end{zadacha}

\begin{zadacha} Пусть $\{a_i\}$ --- последовательность Коши.
Рассмотрим последовательность $\{b_i\}$, $b_i = \inf_{j\geq i}
a_j$. Докажите, что этот инфимум определен, и что последовательность
$b_i$ возрастает.
\end{zadacha}

\begin{zadacha} В условиях предыдущей задачи докажите, что если
последовательность $\{b_i\}$ имеет предел, то $\lim_{i \to \infty}
a_i =\lim_{i \to \infty} b_i$.
\end{zadacha}

\begin{zadacha}[!] Пусть $\{a_i\}$ --- последовательность
Коши. Докажите, что $\{a_i\}$ сходится. Используйте определение
вещественных чисел через сечения Дедекинда.
\end{zadacha}

\begin{ukazanie} Воспользуйтесь предыдущей задачей.
\end{ukazanie}

\begin{zadacha}[!] Пусть $\{a_i\}$ --- последовательность
Коши. Докажите, что $\{a_i\}$ сходится. Используйте определение
вещественных чисел через последовательности Коши.
\end{zadacha}

\begin{ukazanie} 
Пусть вещественное число $\{a_i\}$ представлено последовательностью
Коши рациональных чисел $a_i(0), a_i(1), a_i(2), ...$. Перейдя к
подпоследовательности, можно предполагать, что все $a_i$ ($i> n$)
содержатся в отрезке длины $2^{-n}$, и все $a_i(j)$
($j>m$)
содержатся в отрезке длины $2^{-m}$. Докажите, что
последовательность $\{a_i(i)\}$ --- последовательность Коши, и к
представленному ей вещественному числу сходится последовательность
$\{a_i\}$.
\end{ukazanie}

\begin{zadacha}[!] 
(теорема о двух милиционерах) Пусть $\{ a_i\}$, $\{b_i\}$,
$\{c_i\}$ --- сходящиеся последовательности вещественных чисел,
причем $a_i \leq b_i \leq c_i$ для всех $i$.  Предположим, что
$\lim_{i \to \infty} a_i = \lim_{i \to \infty} c_i =x$.  Докажите,
что $\lim_{i \to \infty} b_i =x$.
\end{zadacha}

\begin{zadacha}[*] Пусть последовательность $\{a_i\}$ сходится
к $x$. Докажите, что последовательность $b_j = \frac 1 j
\sum_{i=0}^j a_i$ сходится к $x$. Приведите пример, когда $\{b_j\}$
сходится, а $\{a_i\}$ не сходится.
\end{zadacha}

%%%%%%%%%%%%%%%%%%%%%%%%%%%%%%%%%%%%%%%%%%%%%%%%
\subs{Ряды.}
%%%%%%%%%%%%%%%%%%%%%%%%%%%%%%%%%%%%%%%%%%%%%%%%

\begin{opredelenie} Пусть $\{a_i\}$ --- последовательность
вещественных чисел. Рассмотрим последовательность частичных сумм
$\sum_{i=0}^n a_i$.  Если эта последовательность сходится, говорят,
что {\bf ряд $\sum_{i=0}^\infty a_i$ сходится}. В этом случае пишут
$\sum_{i=0}^\infty a_i =x$, где
$$ 
   x = \lim_{i \to \infty} \sum_{i=0}^n a_i.
$$
Часто пишут проще: $\sum a_i =x$.
\end{opredelenie}

\begin{opredelenie} Ряд $\sum a_i$ {\bf абсолютно сходится},
если сходится ряд $\sum |a_i|$.
\end{opredelenie}

\begin{zadacha}[!] Дан абсолютно сходящийся ряд
$\sum a_i$. Докажите, что этот ряд сходится.
\end{zadacha}

\begin{zadacha} Дан абсолютно сходящийся ряд $\sum a_i$. Пусть 
$b_i$ --- такая последовательность неотрицательных чисел, что $a_i
\geq b_i$. Докажите, что ряд $\sum b_i$ абсолютно сходится.
\end{zadacha}

\begin{zadacha}[**] Пусть $a_i$, $b_i$ --- такие последовательности
вещественных чисел, что ряды $\sum a_i^2$, $\sum b_i^2$
сходятся. Докажите, что ряд $\sum a_ib_i$ сходится.
\end{zadacha}

\begin{zadacha}[*] Пусть $a_i$ --- последовательность
положительных вещественных чисел. Предел последовательности
произведений
$$
\lim_{n\to \infty} \prod^n_{i=0} (1+a_i)
$$
обозначается $\prod^\infty_{i=0} (1+a_i)$. Если этот предел
существует, то говорят, что бесконечное произведение
$\prod^\infty_{i=0} (1+a_i)$ сходится. Пусть произведение
$\prod^\infty_{i=0} (1+a_i)$ сходится. Докажите, что ряд
$\sum^\infty_{i=0} a_i$ сходится.
\end{zadacha}

\begin{zadacha}[*] Докажите, что бесконечное произведение
$\prod^\infty_{i=0} (1+\frac 1 {3^n})$ сходится.
\end{zadacha}

\begin{zadacha}[**] Пусть ряд $\sum a_i$ сходится.
Докажите, что $\prod^\infty_{i=0} (1+a_i)$ тоже сходится.
\end{zadacha}

\begin{zadacha}[!] Пусть $a_0\geq a_1 \geq a_2 \geq \ldots$ ---
монотонно убывающая последовательность положительных вещественных
чисел, которая стремится к нулю. Рассмотрим ряд $\sum^\infty_{i=0}
(-1)^ia_i$. Докажите, что этот ряд сходится. Такой ряд называется
{\bf знакопеременным}.
\end{zadacha}

\begin{zadacha} Докажите, что ряд $\sum \frac 1{n (n+1)}$
сходится.
\end{zadacha}

\begin{ukazanie} Воспользуйтесь соотношением
$\frac 1{n (n+1)}=\frac{1}{n} - \frac 1{(n+1)}$.
\end{ukazanie}

\begin{zadacha} Докажите, что ряд 
$\sum \frac{1}{n^2}$ сходится.
\end{zadacha}

\begin{zadacha} Докажите, что ряд $\sum \frac 1{n!}$
сходится.
\end{zadacha}

\begin{zadacha}[!] Докажите, что ряд $\sum \frac 1{2^n}$
сходится. Вычислите, к чему он сходится.
\end{zadacha}

\begin{zadacha}[*] Докажите, что ряд $\sum_{n=0}^\infty \frac
{x^n}{n!}$ сходится для всех $x\in \R$.
\end{zadacha}

\begin{zadacha}[**] Рассмотрим ряд $\sum_{n=0}^\infty \frac
{x^n}{n!}$ в полном упорядоченном поле $A$. Сходится ли эта сумма
для всех $x\in A$?
\end{zadacha}

\subsection*{Благодарности}

Эти задачи существовали в виде рукописи с 2005-го года,
и за эти 5 лет я получил множество ценных замечаний и
комментариев. Особенно же я признателен Дмитрию Каледину,
который сотрудничал в составлении задач и всей программы
курса в 2005-м году. Немало исправлений было получено
от Марины Прохоровой, Александра Шеня, Виктора Прасолова
и Ивана Ремизова, которым я донельзя благодарен; и других
людей, тоже чрезвычайно достойных и замечательных.
Отдельная благодарность студентам, без которых это
сочинение не было бы даже начато.


\end{document}